\section{\class{wxStaticBoxSizer}}\label{wxstaticboxsizer}

wxStaticBoxSizer is a sizer derived from wxBoxSizer but adds a static
box around the sizer. This static box has to be created independently or the
sizer may create it itself as a convenience.

\wxheading{Derived from}

\helpref{wxBoxSizer}{wxboxsizer}\\
\helpref{wxSizer}{wxsizer}\\
\helpref{wxObject}{wxobject}

\wxheading{Include files}

<wx/sizer.h>

\wxheading{See also}

\helpref{wxSizer}{wxsizer}, \helpref{wxStaticBox}{wxstaticbox}, \helpref{wxBoxSizer}{wxboxsizer}, \helpref{Sizer overview}{sizeroverview}

\latexignore{\rtfignore{\wxheading{Members}}}


\membersection{wxStaticBoxSizer::wxStaticBoxSizer}\label{wxstaticboxsizerwxstaticboxsizer}

\func{}{wxStaticBoxSizer}{\param{wxStaticBox* }{box}, \param{int }{orient}}

\func{}{wxStaticBoxSizer}{\param{int }{orient}, \param{wxWindow }{*parent}, \param{const wxString\& }{label = wxEmptyString}}

The first constructor uses an already existing static box. It takes the
associated static box and the orientation \arg{orient}, which can be either
\texttt{wxVERTICAL} or \texttt{wxHORIZONTAL} as parameters.

The second one creates a new static box with the given label and parent window.


\membersection{wxStaticBoxSizer::GetStaticBox}\label{wxstaticboxsizergetstaticbox}

\func{wxStaticBox*}{GetStaticBox}{\void}

Returns the static box associated with the sizer.

