\section{\class{wxStopWatch}}\label{wxstopwatch}

The wxStopWatch class allow you to measure time intervals. For example, you may
use it to measure the time elapsed by some function:

\begin{verbatim}
    wxStopWatch sw;
    CallLongRunningFunction();
    wxLogMessage("The long running function took %ldms to execute",
                 sw.Time());
    sw.Pause();
    ... stopwatch is stopped now ...
    sw.Resume();
    CallLongRunningFunction();
    wxLogMessage("And calling it twice took $ldms in all", sw.Time());
\end{verbatim}

\wxheading{Include files}

<wx/timer.h>

\wxheading{See also}

\helpref{::wxStartTimer}{wxstarttimer}, \helpref{::wxGetElapsedTime}{wxgetelapsedtime}, \helpref{wxTimer}{wxtimer}

\latexignore{\rtfignore{\wxheading{Members}}}

\membersection{wxStopWatch::wxStopWatch}

\func{}{wxStopWatch}{\void}

Constructor. This starts the stop watch.

\membersection{wxStopWatch::Pause}\label{wxstopwatchpause}

\func{void}{Pause}{\void}

Pauses the stop watch. Call \helpref{wxStopWatch::Resume}{wxstopwatchresume} to resume 
time measuring again.

If this method is called several times, {\tt Resume()} must be called the same
number of times to really resume the stop watch. You may, however, call 
\helpref{Start}{wxstopwatchstart} to resume it unconditionally.

\membersection{wxStopWatch::Resume}\label{wxstopwatchresume}

\func{void}{Resume}{\void}

Resumes the stop watch which had been paused with 
\helpref{wxStopWatch::Pause}{wxstopwatchpause}.

\membersection{wxStopWatch::Start}\label{wxstopwatchstart}

\func{void}{Start}{\param{long}{ milliseconds = 0}}

(Re)starts the stop watch with a given initial value.

\membersection{wxStopWatch::Time}

\func{long}{Time}{\void}\label{wxstopwatchtime}

Returns the time in milliseconds since the start (or restart) or the last call of 
\helpref{wxStopWatch::Pause}{wxstopwatchpause}.

