\section{\class{wxEvent}}\label{wxevent}

An event is a structure holding information about an event passed to a
callback or member function. {\bf wxEvent} used to be a multipurpose
event object, and is an abstract base class for other event classes (see below).

For more information about events, see the \helpref{Event handling overview}{eventhandlingoverview}.

\perlnote{In wxPerl custome event classes should be derived from
\texttt{Wx::PlEvent} and \texttt{Wx::PlCommandEvent}.}

\wxheading{Derived from}

\helpref{wxObject}{wxobject}

\wxheading{Include files}

<wx/event.h>

\wxheading{See also}

\helpref{wxCommandEvent}{wxcommandevent},\rtfsp
\helpref{wxMouseEvent}{wxmouseevent}

\latexignore{\rtfignore{\wxheading{Members}}}

\membersection{wxEvent::wxEvent}\label{wxeventctor}

\func{}{wxEvent}{\param{int }{id = 0}, \param{wxEventType }{eventType = {\tt wxEVT\_NULL}}}

Constructor. Should not need to be used directly by an application.

\membersection{wxEvent::m\_eventObject}\label{wxeventmeventobject}

\member{wxObject*}{m\_eventObject}

The object (usually a window) that the event was generated from,
or should be sent to.

\membersection{wxEvent::m\_eventType}\label{wxeventmeventtype}

\member{WXTYPE}{m\_eventType}

The type of the event, such as wxEVENT\_TYPE\_BUTTON\_COMMAND.

\membersection{wxEvent::m\_id}\label{wxeventmid}

\member{int}{m\_id}

Identifier for the window.

\membersection{wxEvent::m\_propagationLevel}\label{wxeventmpropagationlevel}

\member{int}{m\_propagationLevel}

Indicates how many levels the event can propagate. This member is protected and
should typically only be set in the constructors of the derived classes. It
may be temporarily changed by \helpref{StopPropagation}{wxeventstoppropagation} 
and \helpref{ResumePropagation}{wxeventresumepropagation} and tested with 
\helpref{ShouldPropagate}{wxeventshouldpropagate}.

The initial value is set to either {\tt wxEVENT\_PROPAGATE\_NONE} (by
default) meaning that the event shouldn't be propagated at all or to 
{\tt wxEVENT\_PROPAGATE\_MAX} (for command events) meaning that it should be
propagated as much as necessary.

Any positive number means that the event should be propagated but no more than
the given number of times. E.g. the propagation level may be set to $1$ to
propagate the event to its parent only, but not to its grandparent.

\membersection{wxEvent::m\_skipped}\label{wxeventmskipped}

\member{bool}{m\_skipped}

Set to true by {\bf Skip} if this event should be skipped.

\membersection{wxEvent::m\_timeStamp}\label{wxeventmtimestamp}

\member{long}{m\_timeStamp}

Timestamp for this event.

\membersection{wxEvent::Clone}\label{wxeventclone}

\constfunc{virtual wxEvent*}{Clone}{\void}

Returns a copy of the event.

Any event that is posted to the wxWidgets event system for later action (via
\helpref{wxEvtHandler::AddPendingEvent}{wxevthandleraddpendingevent} or
\helpref{wxPostEvent}{wxpostevent}) must implement this method. All wxWidgets
events fully implement this method, but any derived events implemented by the
user should also implement this method just in case they (or some event
derived from them) are ever posted.

All wxWidgets events implement a copy constructor, so the easiest way of
implementing the Clone function is to implement a copy constructor for
a new event (call it MyEvent) and then define the Clone function like this:

\begin{verbatim}
    wxEvent *Clone(void) const { return new MyEvent(*this); }
\end{verbatim}

\membersection{wxEvent::GetEventObject}\label{wxeventgeteventobject}

\func{wxObject*}{GetEventObject}{\void}

Returns the object associated with the
event, if any.

\membersection{wxEvent::GetEventType}\label{wxeventgeteventtype}

\func{WXTYPE}{GetEventType}{\void}

Returns the identifier of the given event type,
such as wxEVENT\_TYPE\_BUTTON\_COMMAND.

\membersection{wxEvent::GetId}\label{wxeventgetid}

\constfunc{int}{GetId}{\void}

Returns the identifier associated with this event, such as a button command id.

\membersection{wxEvent::GetSkipped}\label{wxeventgetskipped}

\constfunc{bool}{GetSkipped}{\void}

Returns true if the event handler should be skipped, false otherwise.

\membersection{wxEvent::GetTimestamp}\label{wxeventgettimestamp}

\func{long}{GetTimestamp}{\void}

Gets the timestamp for the event.

\membersection{wxEvent::IsCommandEvent}\label{wxeventiscommandevent}

\constfunc{bool}{IsCommandEvent}{\void}

Returns true if the event is or is derived from
\helpref{wxCommandEvent}{wxcommandevent} else it returns false.
Note: Exists only for optimization purposes.


\membersection{wxEvent::ResumePropagation}\label{wxeventresumepropagation}

\func{void}{ResumePropagation}{\param{int }{propagationLevel}}

Sets the propagation level to the given value (for example returned from an
earlier call to \helpref{StopPropagation}{wxeventstoppropagation}).


\membersection{wxEvent::SetEventObject}\label{wxeventseteventobject}

\func{void}{SetEventObject}{\param{wxObject* }{object}}

Sets the originating object.

\membersection{wxEvent::SetEventType}\label{wxeventseteventtype}

\func{void}{SetEventType}{\param{WXTYPE }{typ}}

Sets the event type.

\membersection{wxEvent::SetId}\label{wxeventsetid}

\func{void}{SetId}{\param{int}{ id}}

Sets the identifier associated with this event, such as a button command id.

\membersection{wxEvent::SetTimestamp}\label{wxeventsettimestamp}

\func{void}{SetTimestamp}{\param{long }{timeStamp}}

Sets the timestamp for the event.

Sets the originating object.

\membersection{wxEvent::ShouldPropagate}\label{wxeventshouldpropagate}

\constfunc{bool}{ShouldPropagate}{\void}

Test if this event should be propagated or not, i.e. if the propagation level
is currently greater than $0$.

\membersection{wxEvent::Skip}\label{wxeventskip}

\func{void}{Skip}{\param{bool}{ skip = true}}

Called by an event handler to tell the event system that the
event handler should be skipped, and the next valid handler used
instead.

\membersection{wxEvent::StopPropagation}\label{wxeventstoppropagation}

\func{int}{StopPropagation}{\void}

Stop the event from propagating to its parent window.

Returns the old propagation level value which may be later passed to 
\helpref{ResumePropagation}{wxeventresumepropagation} to allow propagating the
event again.

