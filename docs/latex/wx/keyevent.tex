\section{\class{wxKeyEvent}}\label{wxkeyevent}

This event class contains information about keypress (character) events.

Notice that there are three different kinds of keyboard events in wxWindows:
key down and up events and char events. The difference between the first two
is clear - the first corresponds to a key press and the second to a key
release - otherwise they are identical. Just note that if the key is
maintained in a pressed state you will typically get a lot of (automatically
generated) down events but only one up so it is wrong to assume that there is
one up event corresponding to each down one.

Both key events provide untranslated key codes while the char event carries
the translated one. The untranslated code for alphanumeric keys is always
an upper case value. For the other keys it is one of {\tt WXK\_XXX} values
from the \helpref{keycodes table}{keycodes}. The translated key is, in
general, the character the user expects to appear as the result of the key
combination when typing the text into a text entry zone, for example.

A few examples to clarify this (all assume that {\sc Caps Lock} is unpressed
and the standard US keyboard): when the {\tt 'A'} key is pressed, the key down
event key code is equal to {\tt ASCII A} $== 65$. But the char event key code
is {\tt ASCII a} $== 97$. On the other hand, if you press both {\sc Shift} and
{\tt 'A'} keys simultaneously , the key code in key down event will still be
just {\tt 'A'} while the char event key code parameter will now be {\tt 'A'}
as well.

Although in this simple case it is clear that the correct key code could be
found in the key down event handler by checking the value returned by
\helpref{ShiftDown()}{wxkeyeventshiftdown}, in general you should use
{\tt EVT\_CHAR} for this as for non alphanumeric keys the translation is
keyboard-layout dependent and can only be done properly by the system itself.

Another kind of translation is done when the control key is pressed: for
example, for {\sc Ctrl-A} key press the key down event still carries the
same key code {\tt 'a'} as usual but the char event will have key code of
$1$, the ASCII value of this key combination.

You may discover how the other keys on your system behave interactively by
running the \helpref{text}{sampletext} wxWindows sample and pressing some keys
in any of the text controls shown in it.

{\bf Note:} If a key down ({\tt EVT\_KEY\_DOWN}) event is caught and
the event handler does not call {\tt event.Skip()} then the coresponding
char event ({\tt EVT\_CHAR}) will not happen.  This is by design and
enables the programs that handle both types of events to be a bit
simpler.

{\bf Note for Windows programmers:} The key and char events in wxWindows are
similar to but slightly different from Windows {\tt WM\_KEYDOWN} and
{\tt WM\_CHAR} events. In particular, Alt-x combination will generate a char
event in wxWindows (unless it is used as an accelerator).

{\bf Tip:} be sure to call {\tt event.Skip()} for events that you don't process in
key event function, otherwise menu shortcuts may cease to work under Windows.

\wxheading{Derived from}

\helpref{wxEvent}{wxevent}

\wxheading{Include files}

<wx/event.h>

\wxheading{Event table macros}

To process a key event, use these event handler macros to direct input to member
functions that take a wxKeyEvent argument.

\twocolwidtha{7cm}
\begin{twocollist}\itemsep=0pt
\twocolitem{{\bf EVT\_KEY\_DOWN(func)}}{Process a wxEVT\_KEY\_DOWN event (any key has been pressed).}
\twocolitem{{\bf EVT\_KEY\_UP(func)}}{Process a wxEVT\_KEY\_UP event (any key has been released).}
\twocolitem{{\bf EVT\_CHAR(func)}}{Process a wxEVT\_CHAR event.}
%\twocolitem{{\bf EVT\_CHAR\_HOOK(func)}}{Process a wxEVT\_CHAR\_HOOK event.}
\end{twocollist}%


\latexignore{\rtfignore{\wxheading{Members}}}

\membersection{wxKeyEvent::m\_altDown}

\member{bool}{m\_altDown}

TRUE if the Alt key is pressed down.

\membersection{wxKeyEvent::m\_controlDown}

\member{bool}{m\_controlDown}

TRUE if control is pressed down.

\membersection{wxKeyEvent::m\_keyCode}

\member{long}{m\_keyCode}

Virtual keycode. See \helpref{Keycodes}{keycodes} for a list of identifiers.

\membersection{wxKeyEvent::m\_metaDown}

\member{bool}{m\_metaDown}

TRUE if the Meta key is pressed down.

\membersection{wxKeyEvent::m\_shiftDown}

\member{bool}{m\_shiftDown}

TRUE if shift is pressed down.

\membersection{wxKeyEvent::m\_x}

\member{int}{m\_x}

X position of the event.

\membersection{wxKeyEvent::m\_y}

\member{int}{m\_y}

Y position of the event.

\membersection{wxKeyEvent::wxKeyEvent}

\func{}{wxKeyEvent}{\param{WXTYPE}{ keyEventType}}

Constructor. Currently, the only valid event types are wxEVT\_CHAR and wxEVT\_CHAR\_HOOK.

\membersection{wxKeyEvent::AltDown}

\constfunc{bool}{AltDown}{\void}

Returns TRUE if the Alt key was down at the time of the key event.

\membersection{wxKeyEvent::ControlDown}

\constfunc{bool}{ControlDown}{\void}

Returns TRUE if the control key was down at the time of the key event.

\membersection{wxKeyEvent::GetKeyCode}

\constfunc{int}{GetKeyCode}{\void}

Returns the virtual key code. ASCII events return normal ASCII values,
while non-ASCII events return values such as {\bf WXK\_LEFT} for the
left cursor key. See \helpref{Keycodes}{keycodes} for a full list of the virtual key codes.

\membersection{wxKeyEvent::GetRawKeyCode}

\constfunc{wxUint32}{GetRawKeyCode}{\void}

Returns the raw key code for this event. This is a platform-dependent scan code
which should only be used in advanced applications.

{\bf NB:} Currently the raw key codes are not supported by all ports, use
{\tt\#ifdef wxHAS\_RAW\_KEY\_CODES} to determine if this feature is available.

\membersection{wxKeyEvent::GetRawKeyFlags}

\constfunc{wxUint32}{GetRawKeyFlags}{\void}

Returns the low level key flags for this event. The flags are
platform-dependent and should only be used in advanced applications.

{\bf NB:} Currently the raw key flags are not supported by all ports, use
{\tt \#ifdef wxHAS\_RAW\_KEY\_CODES} to determine if this feature is available.

\membersection{wxKeyEvent::GetX}

\constfunc{long}{GetX}{\void}

Returns the X position of the event.

\membersection{wxKeyEvent::GetY}

\constfunc{long}{GetY}{\void}

Returns the Y position of the event.

\membersection{wxKeyEvent::MetaDown}

\constfunc{bool}{MetaDown}{\void}

Returns TRUE if the Meta key was down at the time of the key event.

\membersection{wxKeyEvent::GetPosition}

\constfunc{wxPoint}{GetPosition}{\void}

\constfunc{void}{GetPosition}{\param{long *}{x}, \param{long *}{y}}

Obtains the position at which the key was pressed.

\membersection{wxKeyEvent::HasModifiers}

\constfunc{bool}{HasModifiers}{\void}

Returns TRUE if either {\sc Ctrl} or {\sc Alt} keys was down
at the time of the key event. Note that this function does not take into
account neither {\sc Shift} nor {\sc Meta} key states (the reason for ignoring
the latter is that it is common for {\sc NumLock} key to be configured as
{\sc Meta} under X but the key presses even while {\sc NumLock} is on should
be still processed normally).

\membersection{wxKeyEvent::ShiftDown}\label{wxkeyeventshiftdown}

\constfunc{bool}{ShiftDown}{\void}

Returns TRUE if the shift key was down at the time of the key event.

