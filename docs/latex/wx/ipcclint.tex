\section{\class{wxClient}}\label{wxddeclient}

A wxClient object represents the client part of a client-server
DDE-like (Dynamic Data Exchange) conversation. The actual
DDE-based implementation using wxDDEClient is available on Windows
only, but a platform-independent, socket-based version of this
API is available using wxTCPClient, which has the same API.

To create a client which can communicate with a suitable server,
you need to derive a class from wxConnection and another from
wxClient. The custom wxConnection class will intercept
communications in a `conversation' with a server, and the custom
wxClient is required so that a user-overridden
\helpref{wxClient::OnMakeConnection}{wxddeclientonmakeconnection} 
member can return a wxConnection of the required class, when a
connection is made. Look at the IPC sample and the 
\helpref{Interprocess communications overview}{ipcoverview} for
an example of how to do this.

\wxheading{Derived from}

wxClientBase\\
\helpref{wxObject}{wxobject}

\wxheading{Include files}

<wx/ipc.h>

\wxheading{See also}

\helpref{wxServer}{wxddeserver}, 
\helpref{wxConnection}{wxddeconnection}, \helpref{Interprocess communications overview}{ipcoverview}

\latexignore{\rtfignore{\wxheading{Members}}}

\membersection{wxClient::wxClient}

\func{}{wxClient}{\void}

Constructs a client object.

\membersection{wxClient::MakeConnection}\label{wxddeclientmakeconnection}

\func{wxConnectionBase *}{MakeConnection}{\param{const wxString\& }{host}, \param{const wxString\& }{service}, \param{const wxString\& }{topic}}

Tries to make a connection with a server by host (machine name
under UNIX - use 'localhost' for same machine; ignored when using
native DDE in Windows), service name and topic string. If the
server allows a connection, a wxConnection object will be
returned. The type of wxConnection returned can be altered by
overriding the 
\helpref{wxClient::OnMakeConnection}{wxddeclientonmakeconnection} 
member to return your own derived connection object.

Under Unix, the service name may be either an integer port
identifier in which case an Internet domain socket will be used
for the communications, or a valid file name (which shouldn't
exist and will be deleted afterwards) in which case a Unix domain
socket is created.

{\bf SECURITY NOTE:} Using Internet domain sockets if extremely
insecure for IPC as there is absolutely no access control for
them, use Unix domain sockets whenever possible!

\membersection{wxClient::OnMakeConnection}\label{wxddeclientonmakeconnection}

\func{wxConnectionBase *}{OnMakeConnection}{\void}

Called by \helpref{wxClient::MakeConnection}{wxddeclientmakeconnection}, by
default this simply returns a new wxConnection object. Override
this method to return a wxConnection descendant customised for the
application.

The advantage of deriving your own connection class is that it
will enable you to intercept messages initiated by the server,
such as \helpref{wxConnection::OnAdvise}{wxddeconnectiononadvise}. You
may also want to store application-specific data in instances of
the new class.

\membersection{wxClient::ValidHost}

\func{bool}{ValidHost}{\param{const wxString\& }{host}}

Returns TRUE if this is a valid host name, FALSE otherwise. This always
returns TRUE under MS Windows.

