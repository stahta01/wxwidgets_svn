\section{\class{wxPrintData}}\label{wxprintdata}

This class holds a variety of information related to printers and
printer device contexts. This class is used to create a wxPrinterDC
and a wxPostScriptDC. It is also used as a data member of wxPrintDialogData
and wxPageSetupDialogData, as part of the mechanism for transferring data
between the print dialogs and the application.

\wxheading{Derived from}

\helpref{wxObject}{wxobject}

\wxheading{Include files}

<wx/cmndata.h>

\wxheading{See also}

\helpref{Printing framework overview}{printingoverview}, 
\helpref{wxPrintDialog}{wxprintdialog}, 
\helpref{wxPageSetupDialog}{wxpagesetupdialog}, 
\helpref{wxPrintDialogData}{wxprintdialogdata}, 
\helpref{wxPageSetupDialogData}{wxpagesetupdialogdata}, 
\helpref{wxPrintDialog Overview}{wxprintdialogoverview}, 
\helpref{wxPrinterDC}{wxprinterdc}, 
\helpref{wxPostScriptDC}{wxpostscriptdc}

\wxheading{Remarks}

The following functions are specific to PostScript printing
and have not yet been documented:

\begin{verbatim}
const wxString& GetPrinterCommand() const ;
const wxString& GetPrinterOptions() const ;
const wxString& GetPreviewCommand() const ;
const wxString& GetFilename() const ;
const wxString& GetFontMetricPath() const ;
double GetPrinterScaleX() const ;
double GetPrinterScaleY() const ;
long GetPrinterTranslateX() const ;
long GetPrinterTranslateY() const ;
// wxPRINT_MODE_PREVIEW, wxPRINT_MODE_FILE, wxPRINT_MODE_PRINTER
wxPrintMode GetPrintMode() const ;

void SetPrinterCommand(const wxString& command) ;
void SetPrinterOptions(const wxString& options) ;
void SetPreviewCommand(const wxString& command) ;
void SetFilename(const wxString& filename) ;
void SetFontMetricPath(const wxString& path) ;
void SetPrinterScaleX(double x) ;
void SetPrinterScaleY(double y) ;
void SetPrinterScaling(double x, double y) ;
void SetPrinterTranslateX(long x) ;
void SetPrinterTranslateY(long y) ;
void SetPrinterTranslation(long x, long y) ;
void SetPrintMode(wxPrintMode printMode) ;
\end{verbatim}

\latexignore{\rtfignore{\wxheading{Members}}}


\membersection{wxPrintData::wxPrintData}\label{wxprintdatactor}

\func{}{wxPrintData}{\void}

Default constructor.

\func{}{wxPrintData}{\param{const wxPrintData\&}{ data}}

Copy constructor.


\membersection{wxPrintData::\destruct{wxPrintData}}\label{wxprintdatadtor}

\func{}{\destruct{wxPrintData}}{\void}

Destructor.


\membersection{wxPrintData::GetCollate}\label{wxprintdatagetcollate}

\constfunc{bool}{GetCollate}{\void}

Returns true if collation is on.


\membersection{wxPrintData::GetBin}\label{wxprintdatagetbin}

\constfunc{wxPrintBin}{GetBin}{\void}

Returns the current bin (papersource). By default, the system is left to select
the bin (\texttt{wxPRINTBIN\_DEFAULT} is returned).

See \helpref{SetBin()}{wxprintdatasetbin} for the full list of bin values.


\membersection{wxPrintData::GetColour}\label{wxprintdatagetcolour}

\constfunc{bool}{GetColour}{\void}

Returns true if colour printing is on.


\membersection{wxPrintData::GetDuplex}\label{wxprintdatagetduplex}

\constfunc{wxDuplexMode}{GetDuplex}{\void}

Returns the duplex mode. One of wxDUPLEX\_SIMPLEX, wxDUPLEX\_HORIZONTAL, wxDUPLEX\_VERTICAL.


\membersection{wxPrintData::GetNoCopies}\label{wxprintdatagetnocopies}

\constfunc{int}{GetNoCopies}{\void}

Returns the number of copies requested by the user.


\membersection{wxPrintData::GetOrientation}\label{wxprintdatagetorientation}

\constfunc{int}{GetOrientation}{\void}

Gets the orientation. This can be wxLANDSCAPE or wxPORTRAIT.


\membersection{wxPrintData::GetPaperId}\label{wxprintdatagetpaperid}

\constfunc{wxPaperSize}{GetPaperId}{\void}

Returns the paper size id. For more information, see \helpref{wxPrintData::SetPaperId}{wxprintdatasetpaperid}.


\membersection{wxPrintData::GetPrinterName}\label{wxprintdatagetprintername}

\constfunc{const wxString\&}{GetPrinterName}{\void}

Returns the printer name. If the printer name is the empty string, it indicates that the default
printer should be used.


\membersection{wxPrintData::GetQuality}\label{wxprintdatagetquality}

\constfunc{wxPrintQuality}{GetQuality}{\void}

Returns the current print quality. This can be a positive integer, denoting the number of dots per inch, or
one of the following identifiers:

\begin{verbatim}
wxPRINT_QUALITY_HIGH
wxPRINT_QUALITY_MEDIUM
wxPRINT_QUALITY_LOW
wxPRINT_QUALITY_DRAFT
\end{verbatim}

On input you should pass one of these identifiers, but on return you may get back a positive integer
indicating the current resolution setting.


\membersection{wxPrintData::IsOk}\label{wxprintdataisok}

\constfunc{bool}{IsOk}{\void}

Returns true if the print data is valid for using in print dialogs.
This can return false on Windows if the current printer is not set, for example.
On all other platforms, it returns true.


\membersection{wxPrintData::SetBin}\label{wxprintdatasetbin}

\func{void}{SetBin}{\param{wxPrintBin }{flag}}

Sets the current bin. Possible values are:

\small{
\begin{verbatim}
enum wxPrintBin
{
    wxPRINTBIN_DEFAULT,

    wxPRINTBIN_ONLYONE,
    wxPRINTBIN_LOWER,
    wxPRINTBIN_MIDDLE,
    wxPRINTBIN_MANUAL,
    wxPRINTBIN_ENVELOPE,
    wxPRINTBIN_ENVMANUAL,
    wxPRINTBIN_AUTO,
    wxPRINTBIN_TRACTOR,
    wxPRINTBIN_SMALLFMT,
    wxPRINTBIN_LARGEFMT,
    wxPRINTBIN_LARGECAPACITY,
    wxPRINTBIN_CASSETTE,
    wxPRINTBIN_FORMSOURCE,

    wxPRINTBIN_USER,
};
\end{verbatim}
}


\membersection{wxPrintData::SetCollate}\label{wxprintdatasetcollate}

\func{void}{SetCollate}{\param{bool }{flag}}

Sets collation to on or off.


\membersection{wxPrintData::SetColour}\label{wxprintdatasetcolour}

\func{void}{SetColour}{\param{bool }{flag}}

Sets colour printing on or off.


\membersection{wxPrintData::SetDuplex}\label{wxprintdatasetduplex}

\func{void}{SetDuplex}{\param{wxDuplexMode}{ mode}}

Returns the duplex mode. One of wxDUPLEX\_SIMPLEX, wxDUPLEX\_HORIZONTAL, wxDUPLEX\_VERTICAL.


\membersection{wxPrintData::SetNoCopies}\label{wxprintdatasetnocopies}

\func{void}{SetNoCopies}{\param{int }{n}}

Sets the default number of copies to be printed out.


\membersection{wxPrintData::SetOrientation}\label{wxprintdatasetorientation}

\func{void}{SetOrientation}{\param{int }{orientation}}

Sets the orientation. This can be wxLANDSCAPE or wxPORTRAIT.


\membersection{wxPrintData::SetPaperId}\label{wxprintdatasetpaperid}

\func{void}{SetPaperId}{\param{wxPaperSize}{ paperId}}

\index{wxPaperSize}Sets the paper id. This indicates the type of paper to be used. For a mapping between
paper id, paper size and string name, see wxPrintPaperDatabase in {\tt paper.h} (not yet documented).

{\it paperId} can be one of:

{\small
\begin{verbatim}
    wxPAPER_NONE,               // Use specific dimensions
    wxPAPER_LETTER,             // Letter, 8 1/2 by 11 inches
    wxPAPER_LEGAL,              // Legal, 8 1/2 by 14 inches
    wxPAPER_A4,                 // A4 Sheet, 210 by 297 millimeters
    wxPAPER_CSHEET,             // C Sheet, 17 by 22 inches
    wxPAPER_DSHEET,             // D Sheet, 22 by 34 inches
    wxPAPER_ESHEET,             // E Sheet, 34 by 44 inches
    wxPAPER_LETTERSMALL,        // Letter Small, 8 1/2 by 11 inches
    wxPAPER_TABLOID,            // Tabloid, 11 by 17 inches
    wxPAPER_LEDGER,             // Ledger, 17 by 11 inches
    wxPAPER_STATEMENT,          // Statement, 5 1/2 by 8 1/2 inches
    wxPAPER_EXECUTIVE,          // Executive, 7 1/4 by 10 1/2 inches
    wxPAPER_A3,                 // A3 sheet, 297 by 420 millimeters
    wxPAPER_A4SMALL,            // A4 small sheet, 210 by 297 millimeters
    wxPAPER_A5,                 // A5 sheet, 148 by 210 millimeters
    wxPAPER_B4,                 // B4 sheet, 250 by 354 millimeters
    wxPAPER_B5,                 // B5 sheet, 182-by-257-millimeter paper
    wxPAPER_FOLIO,              // Folio, 8-1/2-by-13-inch paper
    wxPAPER_QUARTO,             // Quarto, 215-by-275-millimeter paper
    wxPAPER_10X14,              // 10-by-14-inch sheet
    wxPAPER_11X17,              // 11-by-17-inch sheet
    wxPAPER_NOTE,               // Note, 8 1/2 by 11 inches
    wxPAPER_ENV_9,              // #9 Envelope, 3 7/8 by 8 7/8 inches
    wxPAPER_ENV_10,             // #10 Envelope, 4 1/8 by 9 1/2 inches
    wxPAPER_ENV_11,             // #11 Envelope, 4 1/2 by 10 3/8 inches
    wxPAPER_ENV_12,             // #12 Envelope, 4 3/4 by 11 inches
    wxPAPER_ENV_14,             // #14 Envelope, 5 by 11 1/2 inches
    wxPAPER_ENV_DL,             // DL Envelope, 110 by 220 millimeters
    wxPAPER_ENV_C5,             // C5 Envelope, 162 by 229 millimeters
    wxPAPER_ENV_C3,             // C3 Envelope, 324 by 458 millimeters
    wxPAPER_ENV_C4,             // C4 Envelope, 229 by 324 millimeters
    wxPAPER_ENV_C6,             // C6 Envelope, 114 by 162 millimeters
    wxPAPER_ENV_C65,            // C65 Envelope, 114 by 229 millimeters
    wxPAPER_ENV_B4,             // B4 Envelope, 250 by 353 millimeters
    wxPAPER_ENV_B5,             // B5 Envelope, 176 by 250 millimeters
    wxPAPER_ENV_B6,             // B6 Envelope, 176 by 125 millimeters
    wxPAPER_ENV_ITALY,          // Italy Envelope, 110 by 230 millimeters
    wxPAPER_ENV_MONARCH,        // Monarch Envelope, 3 7/8 by 7 1/2 inches
    wxPAPER_ENV_PERSONAL,       // 6 3/4 Envelope, 3 5/8 by 6 1/2 inches
    wxPAPER_FANFOLD_US,         // US Std Fanfold, 14 7/8 by 11 inches
    wxPAPER_FANFOLD_STD_GERMAN, // German Std Fanfold, 8 1/2 by 12 inches
    wxPAPER_FANFOLD_LGL_GERMAN, // German Legal Fanfold, 8 1/2 by 13 inches

Windows 95 only:
    wxPAPER_ISO_B4,             // B4 (ISO) 250 x 353 mm
    wxPAPER_JAPANESE_POSTCARD,  // Japanese Postcard 100 x 148 mm
    wxPAPER_9X11,               // 9 x 11 in
    wxPAPER_10X11,              // 10 x 11 in
    wxPAPER_15X11,              // 15 x 11 in
    wxPAPER_ENV_INVITE,         // Envelope Invite 220 x 220 mm
    wxPAPER_LETTER_EXTRA,       // Letter Extra 9 \275 x 12 in
    wxPAPER_LEGAL_EXTRA,        // Legal Extra 9 \275 x 15 in
    wxPAPER_TABLOID_EXTRA,      // Tabloid Extra 11.69 x 18 in
    wxPAPER_A4_EXTRA,           // A4 Extra 9.27 x 12.69 in
    wxPAPER_LETTER_TRANSVERSE,  // Letter Transverse 8 \275 x 11 in
    wxPAPER_A4_TRANSVERSE,      // A4 Transverse 210 x 297 mm
    wxPAPER_LETTER_EXTRA_TRANSVERSE, // Letter Extra Transverse 9\275 x 12 in
    wxPAPER_A_PLUS,             // SuperA/SuperA/A4 227 x 356 mm
    wxPAPER_B_PLUS,             // SuperB/SuperB/A3 305 x 487 mm
    wxPAPER_LETTER_PLUS,        // Letter Plus 8.5 x 12.69 in
    wxPAPER_A4_PLUS,            // A4 Plus 210 x 330 mm
    wxPAPER_A5_TRANSVERSE,      // A5 Transverse 148 x 210 mm
    wxPAPER_B5_TRANSVERSE,      // B5 (JIS) Transverse 182 x 257 mm
    wxPAPER_A3_EXTRA,           // A3 Extra 322 x 445 mm
    wxPAPER_A5_EXTRA,           // A5 Extra 174 x 235 mm
    wxPAPER_B5_EXTRA,           // B5 (ISO) Extra 201 x 276 mm
    wxPAPER_A2,                 // A2 420 x 594 mm
    wxPAPER_A3_TRANSVERSE,      // A3 Transverse 297 x 420 mm
    wxPAPER_A3_EXTRA_TRANSVERSE // A3 Extra Transverse 322 x 445 mm
\end{verbatim}
}


\membersection{wxPrintData::SetPrinterName}\label{wxprintdatasetprintername}

\func{void}{SetPrinterName}{\param{const wxString\& }{printerName}}

Sets the printer name. This can be the empty string to indicate that the default
printer should be used.


\membersection{wxPrintData::SetQuality}\label{wxprintdatasetquality}

\func{void}{SetQuality}{\param{wxPrintQuality}{ quality}}

Sets the desired print quality. This can be a positive integer, denoting the number of dots per inch, or
one of the following identifiers:

\begin{verbatim}
wxPRINT_QUALITY_HIGH
wxPRINT_QUALITY_MEDIUM
wxPRINT_QUALITY_LOW
wxPRINT_QUALITY_DRAFT
\end{verbatim}

On input you should pass one of these identifiers, but on return you may get back a positive integer
indicating the current resolution setting.


\membersection{wxPrintData::operator $=$}\label{wxprintdataassign}

\func{void}{operator $=$}{\param{const wxPrintData\&}{ data}}

Assigns print data to this object.

\func{void}{operator $=$}{\param{const wxPrintSetupData\&}{ data}}

Assigns print setup data to this object. wxPrintSetupData is deprecated,
but retained for backward compatibility.

\section{\class{wxPrintDialog}}\label{wxprintdialog}

This class represents the print and print setup common dialogs.
You may obtain a \helpref{wxPrinterDC}{wxprinterdc} device context from
a successfully dismissed print dialog.

\wxheading{Derived from}

\helpref{wxDialog}{wxdialog}\\
\helpref{wxWindow}{wxwindow}\\
\helpref{wxEvtHandler}{wxevthandler}\\
\helpref{wxObject}{wxobject}

\wxheading{Include files}

<wx/printdlg.h>

\wxheading{See also}

\helpref{Printing framework overview}{printingoverview}, 
\helpref{wxPrintDialog Overview}{wxprintdialogoverview}

\latexignore{\rtfignore{\wxheading{Members}}}


\membersection{wxPrintDialog::wxPrintDialog}\label{wxprintdialogctor}

\func{}{wxPrintDialog}{\param{wxWindow* }{parent}, \param{wxPrintDialogData* }{data = NULL}}

Constructor. Pass a parent window, and optionally a pointer to a block of print
data, which will be copied to the print dialog's print data.

\wxheading{See also}

\helpref{wxPrintDialogData}{wxprintdialogdata}


\membersection{wxPrintDialog::\destruct{wxPrintDialog}}\label{wxprintdialogdtor}

\func{}{\destruct{wxPrintDialog}}{\void}

Destructor. If wxPrintDialog::GetPrintDC has {\it not} been called,
the device context obtained by the dialog (if any) will be deleted.


\membersection{wxPrintDialog::GetPrintDialogData}\label{wxprintdialoggetprintdialogdata}

\func{wxPrintDialogData\&}{GetPrintDialogData}{\void}

Returns the \helpref{print dialog data}{wxprintdialogdata} associated with the print dialog.


\membersection{wxPrintDialog::GetPrintDC}\label{wxprintdialoggetprintdc}

\func{wxDC* }{GetPrintDC}{\void}

Returns the device context created by the print dialog, if any.
When this function has been called, the ownership of the device context
is transferred to the application, so it must then be deleted
explicitly.


\membersection{wxPrintDialog::ShowModal}\label{wxprintdialogshowmodal}

\func{int}{ShowModal}{\void}

Shows the dialog, returning wxID\_OK if the user pressed OK, and wxID\_CANCEL
otherwise. After this function is called, a device context may
be retrievable using \helpref{wxPrintDialog::GetPrintDC}{wxprintdialoggetprintdc}.

\section{\class{wxPrintDialogData}}\label{wxprintdialogdata}

This class holds information related to the visual characteristics of wxPrintDialog.
It contains a wxPrintData object with underlying printing settings.

\wxheading{Derived from}

\helpref{wxObject}{wxobject}

\wxheading{Include files}

<wx/cmndata.h>

\wxheading{See also}

\helpref{Printing framework overview}{printingoverview}, 
\helpref{wxPrintDialog}{wxprintdialog}, 
\helpref{wxPrintDialog Overview}{wxprintdialogoverview}

\latexignore{\rtfignore{\wxheading{Members}}}


\membersection{wxPrintDialogData::wxPrintDialogData}\label{wxprintdialogdatactor}

\func{}{wxPrintDialogData}{\void}

Default constructor.

\func{}{wxPrintDialogData}{\param{wxPrintDialogData\&}{ dialogData}}

Copy constructor.

\func{}{wxPrintDialogData}{\param{wxPrintData\&}{ printData}}

Construct an object from a print dialog data object.


\membersection{wxPrintDialogData::\destruct{wxPrintDialogData}}\label{wxprintdialogdatadtor}

\func{}{\destruct{wxPrintDialogData}}{\void}

Destructor.


\membersection{wxPrintDialogData::EnableHelp}\label{wxprintdialogdataenablehelp}

\func{void}{EnableHelp}{\param{bool }{flag}}

Enables or disables the `Help' button.


\membersection{wxPrintDialogData::EnablePageNumbers}\label{wxprintdialogdataenablepagenumbers}

\func{void}{EnablePageNumbers}{\param{bool }{flag}}

Enables or disables the `Page numbers' controls.


\membersection{wxPrintDialogData::EnablePrintToFile}\label{wxprintdialogdataenableprinttofile}

\func{void}{EnablePrintToFile}{\param{bool }{flag}}

Enables or disables the `Print to file' checkbox.


\membersection{wxPrintDialogData::EnableSelection}\label{wxprintdialogdataenableselection}

\func{void}{EnableSelection}{\param{bool }{flag}}

Enables or disables the `Selection' radio button.


\membersection{wxPrintDialogData::GetAllPages}\label{wxprintdialogdatagetallpages}

\constfunc{bool}{GetAllPages}{\void}

Returns true if the user requested that all pages be printed.


\membersection{wxPrintDialogData::GetCollate}\label{wxprintdialogdatagetcollate}

\constfunc{bool}{GetCollate}{\void}

Returns true if the user requested that the document(s) be collated.


\membersection{wxPrintDialogData::GetFromPage}\label{wxprintdialogdatagetfrompage}

\constfunc{int}{GetFromPage}{\void}

Returns the {\it from} page number, as entered by the user.


\membersection{wxPrintDialogData::GetMaxPage}\label{wxprintdialogdatagetmaxpage}

\constfunc{int}{GetMaxPage}{\void}

Returns the {\it maximum} page number.


\membersection{wxPrintDialogData::GetMinPage}\label{wxprintdialogdatagetminpage}

\constfunc{int}{GetMinPage}{\void}

Returns the {\it minimum} page number.


\membersection{wxPrintDialogData::GetNoCopies}\label{wxprintdialogdatagetnocopies}

\constfunc{int}{GetNoCopies}{\void}

Returns the number of copies requested by the user.


\membersection{wxPrintDialogData::GetPrintData}\label{wxprintdialogdatagetprintdata}

\func{wxPrintData\&}{GetPrintData}{\void}

Returns a reference to the internal wxPrintData object.


\membersection{wxPrintDialogData::GetPrintToFile}\label{wxprintdialogdatagetprinttofile}

\constfunc{bool}{GetPrintToFile}{\void}

Returns true if the user has selected printing to a file.


\membersection{wxPrintDialogData::GetSelection}\label{wxprintdialogdatagetselection}

\constfunc{bool}{GetSelection}{\void}

Returns true if the user requested that the selection be printed (where 'selection' is
a concept specific to the application).


\membersection{wxPrintDialogData::GetToPage}\label{wxprintdialogdatagettopage}

\constfunc{int}{GetToPage}{\void}

Returns the {\it to} page number, as entered by the user.


\membersection{wxPrintDialogData::IsOk}\label{wxprintdialogdataisok}

\constfunc{bool}{IsOk}{\void}

Returns true if the print data is valid for using in print dialogs.
This can return false on Windows if the current printer is not set, for example.
On all other platforms, it returns true.


\membersection{wxPrintDialogData::SetCollate}\label{wxprintdialogdatasetcollate}

\func{void}{SetCollate}{\param{bool }{flag}}

Sets the 'Collate' checkbox to true or false.


\membersection{wxPrintDialogData::SetFromPage}\label{wxprintdialogdatasetfrompage}

\func{void}{SetFromPage}{\param{int }{page}}

Sets the {\it from} page number.


\membersection{wxPrintDialogData::SetMaxPage}\label{wxprintdialogdatasetmaxpage}

\func{void}{SetMaxPage}{\param{int }{page}}

Sets the {\it maximum} page number.


\membersection{wxPrintDialogData::SetMinPage}\label{wxprintdialogdatasetminpage}

\func{void}{SetMinPage}{\param{int }{page}}

Sets the {\it minimum} page number.


\membersection{wxPrintDialogData::SetNoCopies}\label{wxprintdialogdatasetnocopies}

\func{void}{SetNoCopies}{\param{int }{n}}

Sets the default number of copies the user has requested to be printed out.


\membersection{wxPrintDialogData::SetPrintData}\label{wxprintdialogdatasetprintdata}

\func{void}{SetPrintData}{\param{const wxPrintData\& }{printData}}

Sets the internal wxPrintData.


\membersection{wxPrintDialogData::SetPrintToFile}\label{wxprintdialogdatasetprinttofile}

\func{void}{SetPrintToFile}{\param{bool }{flag}}

Sets the 'Print to file' checkbox to true or false.


\membersection{wxPrintDialogData::SetSelection}\label{wxprintdialogdatasetselection}

\func{void}{SetSelection}{\param{bool}{ flag}}

Selects the 'Selection' radio button. The effect of printing the selection depends on how the application
implements this command, if at all.


\membersection{wxPrintDialogData::SetSetupDialog}\label{wxprintdialogdatasetsetupdialog}

\func{void}{SetSetupDialog}{\param{bool }{flag}}

Determines whether the dialog to be shown will be the Print dialog
(pass false) or Print Setup dialog (pass true).

This function has been deprecated since version 2.5.4.

\membersection{wxPrintDialogData::SetToPage}\label{wxprintdialogdatasettopage}

\func{void}{SetToPage}{\param{int }{page}}

Sets the {\it to} page number.


\membersection{wxPrintDialogData::operator $=$}\label{wxprintdialogdataassign}

\func{void}{operator $=$}{\param{const wxPrintData\&}{ data}}

Assigns print data to this object.

\func{void}{operator $=$}{\param{const wxPrintDialogData\&}{ data}}

Assigns another print dialog data object to this object.

\section{\class{wxPrinter}}\label{wxprinter}

This class represents the Windows or PostScript printer, and is the vehicle through
which printing may be launched by an application. Printing can also
be achieved through using of lower functions and classes, but
this and associated classes provide a more convenient and general
method of printing.

\wxheading{Derived from}

\helpref{wxObject}{wxobject}

\wxheading{Include files}

<wx/print.h>

\wxheading{See also}

\helpref{Printing framework overview}{printingoverview}, 
\helpref{wxPrinterDC}{wxprinterdc}, 
\helpref{wxPrintDialog}{wxprintdialog}, 
\helpref{wxPrintout}{wxprintout}, 
\helpref{wxPrintPreview}{wxprintpreview}.

\latexignore{\rtfignore{\wxheading{Members}}}


\membersection{wxPrinter::wxPrinter}\label{wxprinterctor}

\func{}{wxPrinter}{\param{wxPrintDialogData* }{data = NULL}}

Constructor. Pass an optional pointer to a block of print
dialog data, which will be copied to the printer object's local data.

\wxheading{See also}

\helpref{wxPrintDialogData}{wxprintdialogdata},
\helpref{wxPrintData}{wxprintdata}



\membersection{wxPrinter::CreateAbortWindow}\label{wxprintercreateabortwindow}

\func{void}{CreateAbortWindow}{\param{wxWindow* }{parent}, \param{wxPrintout* }{printout}}

Creates the default printing abort window, with a cancel button.


\membersection{wxPrinter::GetAbort}\label{wxprintergetabort}

\func{bool}{GetAbort}{\void}

Returns true if the user has aborted the print job.


\membersection{wxPrinter::GetLastError}\label{wxprintergetlasterror}

\func{static wxPrinterError}{GetLastError}{\void}

Return last error. Valid after calling \helpref{Print}{wxprinterprint},
\helpref{PrintDialog}{wxprinterprintdialog} or 
\helpref{wxPrintPreview::Print}{wxprintpreviewprint}. These functions 
set last error to {\bf wxPRINTER\_NO\_ERROR} if no error happened.

Returned value is one of the following:

\twocolwidtha{7cm}
\begin{twocollist}\itemsep=0pt
\twocolitem{{\bf wxPRINTER\_NO\_ERROR}}{No error happened.}
\twocolitem{{\bf wxPRINTER\_CANCELLED}}{The user cancelled printing.}
\twocolitem{{\bf wxPRINTER\_ERROR}}{There was an error during printing.}
\end{twocollist}



\membersection{wxPrinter::GetPrintDialogData}\label{wxprintergetprintdialogdata}

\func{wxPrintDialogData\&}{GetPrintDialogData}{\void}

Returns the \helpref{print data}{wxprintdata} associated with the printer object.


\membersection{wxPrinter::Print}\label{wxprinterprint}

\func{bool}{Print}{\param{wxWindow *}{parent}, \param{wxPrintout *}{printout}, \param{bool }{prompt=true}}

Starts the printing process. Provide a parent window, a user-defined wxPrintout object which controls
the printing of a document, and whether the print dialog should be invoked first.

Print could return false if there was a problem initializing the printer device context
(current printer not set, for example) or the user cancelled printing. Call
\helpref{wxPrinter::GetLastError}{wxprintergetlasterror} to get detailed
information about the kind of the error.


\membersection{wxPrinter::PrintDialog}\label{wxprinterprintdialog}

\func{wxDC*}{PrintDialog}{\param{wxWindow *}{parent}}

Invokes the print dialog. If successful (the user did not press Cancel
and no error occurred), a suitable device context will be returned
(otherwise NULL is returned -- call
\helpref{wxPrinter::GetLastError}{wxprintergetlasterror} to get detailed
information about the kind of the error).

The application must delete this device context to avoid a memory leak.


\membersection{wxPrinter::ReportError}\label{wxprinterreporterror}

\func{void}{ReportError}{\param{wxWindow *}{parent}, \param{wxPrintout *}{printout}, \param{const wxString\& }{message}}

Default error-reporting function.


\membersection{wxPrinter::Setup}\label{wxprintersetup}

\func{bool}{Setup}{\param{wxWindow *}{parent}}

Invokes the print setup dialog. Note that the setup dialog is obsolete from
Windows 95, though retained for backward compatibility.

\section{\class{wxPrinterDC}}\label{wxprinterdc}

A printer device context is specific to MSW and Mac, and allows access to any
printer with a Windows or Macintosh driver. See \helpref{wxDC}{wxdc} for further
information on device contexts, and \helpref{wxDC::GetSize}{wxdcgetsize} for
advice on achieving the correct scaling for the page.

\wxheading{Derived from}

\helpref{wxDC}{wxdc}\\
\helpref{wxObject}{wxdc}

\wxheading{Include files}

<wx/dcprint.h>

\wxheading{See also}

\helpref{Printing framework overview}{printingoverview}, 
\helpref{wxDC}{wxdc}

\latexignore{\rtfignore{\wxheading{Members}}}


\membersection{wxPrinterDC::wxPrinterDC}\label{wxprinterdcctor}

\func{}{wxPrinterDC}{\param{const wxPrintData\& }{printData}}

Constructor. Pass a \helpref{wxPrintData}{wxprintdata} object with information
necessary for setting up a suitable printer device context. This
is the recommended way to construct a wxPrinterDC.  Make sure you 
specify a reference to a \helpref{wxPrintData}{wxprintdata} object,
not a pointer - you may not even get a warning if you pass a pointer
instead.

\func{}{wxPrinterDC}{\param{const wxString\& }{driver}, \param{const wxString\& }{device}, \param{const wxString\& }{output},
 \param{const bool }{interactive = true}, \param{int }{orientation = wxPORTRAIT}}

Constructor. With empty strings for the first three arguments, the default printer dialog is
displayed. {\it device} indicates the type of printer and {\it output}
is an optional file for printing to. The {\it driver} parameter is
currently unused.  Use the {\it Ok} member to test whether the
constructor was successful in creating a usable device context.

This constructor is deprecated and retained only for backward compatibility.

\membersection{wxPrinterDC::GetPaperRect}\label{wxprinterdcgetpaperrect}

\func{wxRect}{wxPrinterDC::GetPaperRect}{}

Return the rectangle in device coordinates that corresponds to the full paper
area, including the nonprinting regions of the paper. The point (0,0) in device
coordinates is the top left corner of the page rectangle, which is the printable
area on MSW and Mac. The coordinates of the top left corner of the paper
rectangle will therefore have small negative values, while the bottom right
coordinates will be somewhat larger than the values returned by
\helpref{wxDC::GetSize}{wxdcgetsize}.


\section{\class{wxPrintout}}\label{wxprintout}

This class encapsulates the functionality of printing out an application
document. A new class must be derived and members overridden to respond to calls
such as OnPrintPage and HasPage and to render the print image onto an associated
\helpref{wxDC}{wxdc}. Instances of this class are passed to wxPrinter::Print or
to a wxPrintPreview object to initiate printing or previewing.

Your derived wxPrintout is responsible for drawing both the preview image and
the printed page. If your windows' drawing routines accept an arbitrary DC as an
argument, you can re-use those routines within your wxPrintout subclass to draw
the printout image. You may also add additional drawing elements within your
wxPrintout subclass, like headers, footers, and/or page numbers. However, the
image on the printed page will often differ from the image drawn on the screen,
as will the print preview image -- not just in the presence of headers and
footers, but typically in scale. A high-resolution printer presents a much
larger drawing surface (i.e., a higher-resolution DC); a zoomed-out preview
image presents a much smaller drawing surface (lower-resolution DC). By using
the routines FitThisSizeToXXX() and/or MapScreenSizeToXXX() within your
wxPrintout subclass to set the user scale and origin of the associated DC, you
can easily use a single drawing routine to draw on your application's windows,
to create the print preview image, and to create the printed paper image, and
achieve a common appearance to the preview image and the printed page.


\wxheading{Derived from}

\helpref{wxObject}{wxobject}

\wxheading{Include files}

<wx/print.h>

\wxheading{See also}

\helpref{Printing framework overview}{printingoverview}, 
\helpref{wxPrinterDC}{wxprinterdc}, 
\helpref{wxPrintDialog}{wxprintdialog}, 
\helpref{wxPageSetupDialog}{wxpagesetupdialog}, 
\helpref{wxPrinter}{wxprinter}, 
\helpref{wxPrintPreview}{wxprintpreview}

\latexignore{\rtfignore{\wxheading{Members}}}


\membersection{wxPrintout::wxPrintout}\label{wxprintoutctor}

\func{}{wxPrintout}{\param{const wxString\& }{title = "Printout"}}

Constructor. Pass an optional title argument - the current filename would be a good idea. This will appear in the printing list
(at least in MSW)


\membersection{wxPrintout::\destruct{wxPrintout}}\label{wxprintoutdtor}

\func{}{\destruct{wxPrintout}}{\void}

Destructor.


\membersection{wxPrintout::GetDC}\label{wxprintoutgetdc}

\func{wxDC *}{GetDC}{\void}

Returns the device context associated with the printout (given to the printout at start of
printing or previewing). This will be a wxPrinterDC if printing under Windows or Mac,
a wxPostScriptDC if printing on other platforms, and a wxMemoryDC if previewing.


\membersection{wxPrintout::GetPageInfo}\label{wxprintoutgetpageinfo}

\func{void}{GetPageInfo}{\param{int *}{minPage}, \param{int *}{maxPage}, \param{int *}{pageFrom}, \param{int *}{pageTo}}

Called by the framework to obtain information from the application about minimum
and maximum page values that the user can select, and the required page range to
be printed. By default this returns 1, 32000 for the page minimum and maximum
values, and 1, 1 for the required page range.

If {\it minPage} is zero, the page number controls in the print dialog will be disabled.

\pythonnote{When this method is implemented in a derived Python class,
it should be designed to take no parameters (other than the self
reference) and to return a tuple of four integers.
}

\perlnote{When this method is overridden in a derived class,
it must not take any parameters, and returns a 4-element list.
}


\membersection{wxPrintout::GetPageSizeMM}\label{wxprintoutgetpagesizemm}

\func{void}{GetPageSizeMM}{\param{int *}{w}, \param{int *}{h}}

Returns the size of the printer page in millimetres.

\pythonnote{This method returns the output-only parameters as a tuple.}

\perlnote{In wxPerl this method takes no arguments and returns a
2-element list {\tt ( w, h )}}


\membersection{wxPrintout::GetPageSizePixels}\label{wxprintoutgetpagesizepixels}

\func{void}{GetPageSizePixels}{\param{int *}{w}, \param{int *}{h}}

Returns the size of the printer page in pixels, called the \em{page rectangle}.
The page rectangle has a top left corner at (0,0) and a bottom right corner at
(w,h). These values may not be the same as the values returned from
\helpref{wxDC::GetSize}{wxdcgetsize}; if the printout is being used for
previewing, a memory device context is used, which uses a bitmap size reflecting
the current preview zoom. The application must take this discrepancy into
account if previewing is to be supported.

\pythonnote{This method returns the output-only parameters as a tuple.}

\perlnote{In wxPerl this method takes no arguments and returns a
2-element list {\tt ( w, h )}}


\membersection{wxPrintout::GetPaperRectPixels}\label{wxprintoutgetpaperrectpixels}

\func{wxRect}{GetPaperRectPixels}{}

Returns the rectangle that corresponds to the entire paper in pixels, called the
\em{paper rectangle}. This distinction between paper rectangle and page
rectangle reflects the fact that most printers cannot print all the way to the
edge of the paper. The page rectangle is a rectangle whose top left corner is at
(0,0) and whose width and height are given by
\helpref{wxDC::GetPageSizePixels}{wxprintoutgetpagesizepixels}. On MSW and Mac,
the page rectangle gives the printable area of the paper, while the paper
rectangle represents the entire paper, including non-printable borders. Thus,
the rectangle returned by GetPaperRectPixels will have a top left corner whose
coordinates are small negative numbers and the bottom right corner will have
values somewhat larger than the width and height given by
\helpref{wxDC::GetPageSizePixels}{wxprintoutgetpagesizepixels}. On other
platforms and for PostScript printing, the paper is treated as if its entire
area were printable, so this function will return the same rectangle as the page
rectangle.


\membersection{wxPrintout::GetPPIPrinter}\label{wxprintoutgetppiprinter}

\func{void}{GetPPIPrinter}{\param{int *}{w}, \param{int *}{h}}

Returns the number of pixels per logical inch of the printer device context.
Dividing the printer PPI by the screen PPI can give a suitable scaling factor
for drawing text onto the printer. Remember to multiply this by a scaling factor
to take the preview DC size into account. Or you can just use the
FitThisSizeToXXX() and MapScreenSizeToXXX routines below, which do most of the
scaling calculations for you.

\pythonnote{This method returns the output-only parameters as a tuple.}

\perlnote{In wxPerl this method takes no arguments and returns a
2-element list {\tt ( w, h )}}


\membersection{wxPrintout::GetPPIScreen}\label{wxprintoutgetppiscreen}

\func{void}{GetPPIScreen}{\param{int *}{w}, \param{int *}{h}}

Returns the number of pixels per logical inch of the screen device context.
Dividing the printer PPI by the screen PPI can give a suitable scaling factor
for drawing text onto the printer. If you are doing your own scaling, remember
to multiply this by a scaling factor to take the preview DC size into account.


\membersection{wxPrintout::GetTitle}\label{wxprintoutgettitle}

\func{wxString}{GetTitle}{\void}

Returns the title of the printout

\pythonnote{This method returns the output-only parameters as a tuple.}

\perlnote{In wxPerl this method takes no arguments and returns a
2-element list {\tt ( w, h )}}


\membersection{wxPrintout::HasPage}\label{wxprintouthaspage}

\func{bool}{HasPage}{\param{int}{ pageNum}}

Should be overridden to return true if the document has this page, or false
if not. Returning false signifies the end of the document. By default,
HasPage behaves as if the document has only one page.


\membersection{wxPrintout::IsPreview}\label{wxprintoutispreview}

\func{bool}{IsPreview}{\void}

Returns true if the printout is currently being used for previewing.


\membersection{wxPrintout::FitThisSizeToPaper}\label{wxprintoutfitthissizetopaper}

\func{void}{FitThisSizeToPaper}{\param{const wxSize\& }{imageSize}}

Set the user scale and device origin of the wxDC associated with this wxPrintout
so that the given image size fits entirely within the paper and the origin is at
the top left corner of the paper. Note that with most printers, the region
around the edges of the paper are not printable so that the edges of the image
could be cut off. Use this if you're managing your own page margins.

\membersection{wxPrintout::FitThisSizeToPage}\label{wxprintoutfitthissizetopage}


\func{void}{FitThisSizeToPage}{\param{const wxSize\& }{imageSize}}

Set the user scale and device origin of the wxDC associated with this wxPrintout
so that the given image size fits entirely within the page rectangle and the
origin is at the top left corner of the page rectangle. On MSW and Mac, the page
rectangle is the printable area of the page. On other platforms and PostScript
printing, the page rectangle is the entire paper. Use this if you want your
printed image as large as possible, but with the caveat that on some platforms, 
portions of the image might be cut off at the edges.


\membersection{wxPrintout::FitThisSizeToPageMargins}\label{wxprintoutfitthissizetopagemargins}

\func{void}{FitThisSizeToPageMargins}{\param{const wxSize\& }{imageSize}, \param{const wxPageSetupDialogData\& }{pageSetupData}}

Set the user scale and device origin of the wxDC associated with this wxPrintout
so that the given image size fits entirely within the page margins set in the
given wxPageSetupDialogData object. This function provides the greatest
consistency across all platforms because it does not depend on having access to
the printable area of the paper. Note that on Mac, the native wxPageSetupDialog
does not let you set the page margins; you'll have to provide your own mechanism,
or you can use the Mac-only class wxMacPageMarginsDialog.


\membersection{wxPrintout::MapScreenSizeToPaper}\label{wxprintoutmapscreensizetopaper}

\func{void}{MapScreenSizeToPaper}{}

Set the user scale and device origin of the wxDC associated with this wxPrintout
so that the printed page matches the screen size as closely as possible
and the logical origin is in the top left corner of the paper rectangle.
That is,
a 100-pixel object on screen should appear at the same size on the printed page. (It
will, of course, be larger or smaller in the preview image, depending on the zoom
factor.) Use this if you want WYSIWYG behavior, e.g., in a text editor.


\membersection{wxPrintout::MapScreenSizeToPage}\label{wxprintoutmapscreensizetopage}

\func{void}{MapScreenSizeToPage}{}

This sets the user scale of the wxDC assocated with this wxPrintout to the same
scale as \helpref{MapScreenSizeToPaper}{wxprintoutmapscreensizetopaper} but sets
the logical origin to the top left corner of the page rectangle.


\membersection{wxPrintout::MapScreenSizeToPageMargins}\label{wxprintoutmapscreensizetopagemargins}

\func{void}{MapScreenSizeToPageMargins}{\param{const wxPageSetupDialogData\& }{pageSetupData}}

This sets the user scale of the wxDC assocated with this wxPrintout to the same
scale as
\helpref{MapScreenSizeToPageMargins}{wxprintoutmapscreensizetopagemargins} but
sets the logical origin to the top left corner of the page margins specified by
the given wxPageSetupDialogData object.


\membersection{wxPrintout::MapScreenSizeToDevice}\label{wxprintoutmapscreensizetodevice}

\func{void}{MapScreenSizeToDevice}{}

Set the user scale and device origin of the wxDC associated with this wxPrintout
so that one screen pixel maps to one device pixel on the DC. That is, the user
scale is set to (1,1) and the device origin is set to (0,0). Use this if you
want to do your own scaling prior to calling wxDC drawing calls, for example, if
your underlying model is floating-point and you want to achieve maximum drawing
precision on high-resolution printers. (Note that while the underlying drawing
model of Mac OS X is floating-point, wxWidgets's drawing model scales from integer
coordinates.) You can use the GetLogicalXXXRect() routines below to obtain the
paper rectangle, page rectangle, or page margins rectangle to perform your own scaling.


\membersection{wxPrintout::GetLogicalPaperRect}\label{wxprintoutgetlogicalpaperrect}

\func{wxRect}{GetLogicalPaperRect}{}

Return the rectangle corresponding to the paper in the associated wxDC's
logical coordinates for the current user scale and device origin.


\membersection{wxPrintout::GetLogicalPageRect}\label{wxprintoutgetlogicalpagerect}

\func{wxRect}{GetLogicalPageRect}{}

Return the rectangle corresponding to the page in the associated wxDC's
logical coordinates for the current user scale and device origin. 
On MSW and Mac, this will be the printable area of the paper. On other platforms
and PostScript printing, this will be the full paper rectangle.


\membersection{wxPrintout::GetLogicalPageMarginsRect}\label{wxprintoutgetlogicalpagemarginsrect}

\func{wxRect}{GetLogicalPageMarginsRect}{\param{const wxPageSetupDialogData\& }{pageSetupData}}

Return the rectangle corresponding to the page margins specified by the given
wxPageSetupDialogData object in the associated wxDC's logical coordinates for the
current user scale and device origin. The page margins are specified
with respect to the edges of the paper on all platforms.


\membersection{wxPrintout::SetLogicalOrigin}\label{wxprintoutsetlogicalorigin}

\func{void}{SetLogicalOrigin}{\param{wxCoord }{x}, \param{wxCoord }{y}}

Set the device origin of the associated wxDC so that the current logical point
becomes the new logical origin.


\membersection{wxPrintout::OffsetLogicalOrigin}\label{wxprintoutoffsetlogicalorigin}

\func{void}{OffsetLogicalOrigin}{\param{wxCoord }{xoff}, \param{wxCoord }{yoff}}

Shift the device origin by an amount specified in logical coordinates.


\membersection{wxPrintout::OnBeginDocument}\label{wxprintoutonbegindocument}

\func{bool}{OnBeginDocument}{\param{int}{ startPage}, \param{int}{ endPage}}

Called by the framework at the start of document printing. Return false from
this function cancels the print job. OnBeginDocument is called once for every
copy printed.

The base wxPrintout::OnBeginDocument {\it must} be called (and the return value
checked) from within the overridden function, since it calls wxDC::StartDoc.

\pythonnote{If this method is overridden in a Python class then the
base class version can be called by using the method
{\tt base\_OnBeginDocument(startPage, endPage)}. }


\membersection{wxPrintout::OnEndDocument}\label{wxprintoutonenddocument}

\func{void}{OnEndDocument}{\void}

Called by the framework at the end of document printing. OnEndDocument
is called once for every copy printed.

The base wxPrintout::OnEndDocument {\it must} be called
from within the overridden function, since it calls wxDC::EndDoc.


\membersection{wxPrintout::OnBeginPrinting}\label{wxprintoutonbeginprinting}

\func{void}{OnBeginPrinting}{\void}

Called by the framework at the start of printing. OnBeginPrinting is called once for every
print job (regardless of how many copies are being printed).


\membersection{wxPrintout::OnEndPrinting}\label{wxprintoutonendprinting}

\func{void}{OnEndPrinting}{\void}

Called by the framework at the end of printing. OnEndPrinting
is called once for every print job (regardless of how many copies are being printed).


\membersection{wxPrintout::OnPreparePrinting}\label{wxprintoutonprepareprinting}

\func{void}{OnPreparePrinting}{\void}

Called once by the framework before any other demands are made of the
wxPrintout object. This gives the object an opportunity to calculate the
number of pages in the document, for example.


\membersection{wxPrintout::OnPrintPage}\label{wxprintoutonprintpage}

\func{bool}{OnPrintPage}{\param{int}{ pageNum}}

Called by the framework when a page should be printed. Returning false cancels
the print job. The application can use wxPrintout::GetDC to obtain a device
context to draw on.

\section{\class{wxPrintPreview}}\label{wxprintpreview}

Objects of this class manage the print preview process. The object is passed
a wxPrintout object, and the wxPrintPreview object itself is passed to
a wxPreviewFrame object. Previewing is started by initializing and showing
the preview frame. Unlike wxPrinter::Print, flow of control returns to the application
immediately after the frame is shown.

\wxheading{Derived from}

\helpref{wxObject}{wxobject}

\wxheading{Include files}

<wx/print.h>

\wxheading{See also}

\overview{Printing framework overview}{printingoverview}, 
\helpref{wxPrinterDC}{wxprinterdc}, 
\helpref{wxPrintDialog}{wxprintdialog}, 
\helpref{wxPrintout}{wxprintout}, 
\helpref{wxPrinter}{wxprinter}, 
\helpref{wxPreviewCanvas}{wxpreviewcanvas}, 
\helpref{wxPreviewControlBar}{wxpreviewcontrolbar}, 
\helpref{wxPreviewFrame}{wxpreviewframe}.

\latexignore{\rtfignore{\wxheading{Members}}}


\membersection{wxPrintPreview::wxPrintPreview}\label{wxprintpreviewctor}

\func{}{wxPrintPreview}{\param{wxPrintout* }{printout}, \param{wxPrintout* }{printoutForPrinting},
\param{wxPrintData* }{data=NULL}}

Constructor. Pass a printout object, an optional printout object to be
used for actual printing, and the address of an optional
block of printer data, which will be copied to the print preview object's
print data.

If {\it printoutForPrinting} is non-NULL, a {\bf Print...} button will be placed on the
preview frame so that the user can print directly from the preview interface.

Do not explicitly delete the printout objects once this destructor has been
called, since they will be deleted in the wxPrintPreview constructor.
The same does not apply to the {\it data} argument.

Test the Ok member to check whether the wxPrintPreview object was created correctly.
Ok could return false if there was a problem initializing the printer device context
(current printer not set, for example).


\membersection{wxPrintPreview::\destruct{wxPrintPreview}}\label{wxprintpreviewdtor}

\func{}{\destruct{wxPrinter}}{\void}

Destructor. Deletes both print preview objects, so do not destroy these objects
in your application.


\membersection{wxPrintPreview::GetCanvas}\label{wxprintpreviewgetcanvas}

\func{wxPreviewCanvas* }{GetCanvas}{\void}

Gets the preview window used for displaying the print preview image.


\membersection{wxPrintPreview::GetCurrentPage}\label{wxprintpreviewgetcurrentpage}

\func{int}{GetCurrentPage}{\void}

Gets the page currently being previewed.


\membersection{wxPrintPreview::GetFrame}\label{wxprintpreviewgetframe}

\func{wxFrame *}{GetFrame}{\void}

Gets the frame used for displaying the print preview canvas
and control bar.


\membersection{wxPrintPreview::GetMaxPage}\label{wxprintpreviewgetmaxpage}

\func{int}{GetMaxPage}{\void}

Returns the maximum page number.


\membersection{wxPrintPreview::GetMinPage}\label{wxprintpreviewgetminpage}

\func{int}{GetMinPage}{\void}

Returns the minimum page number.


\membersection{wxPrintPreview::GetPrintout}\label{wxprintpreviewgetprintout}

\func{wxPrintout *}{GetPrintout}{\void}

Gets the preview printout object associated with the wxPrintPreview object.


\membersection{wxPrintPreview::GetPrintoutForPrinting}\label{wxprintpreviewgetprintoutforprinting}

\func{wxPrintout *}{GetPrintoutForPrinting}{\void}

Gets the printout object to be used for printing from within the preview interface,
or NULL if none exists.


\membersection{wxPrintPreview::IsOk}\label{wxprintpreviewisok}

\func{bool}{Ok}{\void}

Returns true if the wxPrintPreview is valid, false otherwise. It could return false if there was a
problem initializing the printer device context (current printer not set, for example).


\membersection{wxPrintPreview::PaintPage}\label{wxprintpreviewpaintpage}

\func{bool}{PaintPage}{\param{wxPreviewCanvas *}{canvas}, \param{wxDC& }{dc}}

This refreshes the preview window with the preview image.
It must be called from the preview window's OnPaint member.

The implementation simply blits the preview bitmap onto
the canvas, creating a new preview bitmap if none exists.


\membersection{wxPrintPreview::Print}\label{wxprintpreviewprint}

\func{bool}{Print}{\param{bool }{prompt}}

Invokes the print process using the second wxPrintout object
supplied in the wxPrintPreview constructor.
Will normally be called by the {\bf Print...} panel item on the
preview frame's control bar.

Returns false in case of error -- call
\helpref{wxPrinter::GetLastError}{wxprintergetlasterror} to get detailed
information about the kind of the error.


\membersection{wxPrintPreview::RenderPage}\label{wxprintpreviewrenderpage}

\func{bool}{RenderPage}{\param{int }{pageNum}}

Renders a page into a wxMemoryDC. Used internally by wxPrintPreview.


\membersection{wxPrintPreview::SetCanvas}\label{wxprintpreviewsetcanvas}

\func{void}{SetCanvas}{\param{wxPreviewCanvas* }{window}}

Sets the window to be used for displaying the print preview image.


\membersection{wxPrintPreview::SetCurrentPage}\label{wxprintpreviewsetcurrentpage}

\func{void}{SetCurrentPage}{\param{int}{ pageNum}}

Sets the current page to be previewed.


\membersection{wxPrintPreview::SetFrame}\label{wxprintpreviewsetframe}

\func{void}{SetFrame}{\param{wxFrame *}{frame}}

Sets the frame to be used for displaying the print preview canvas
and control bar.


\membersection{wxPrintPreview::SetPrintout}\label{wxprintpreviewsetprintout}

\func{void}{SetPrintout}{\param{wxPrintout *}{printout}}

Associates a printout object with the wxPrintPreview object.


\membersection{wxPrintPreview::SetZoom}\label{wxprintpreviewsetzoom}

\func{void}{SetZoom}{\param{int}{ percent}}

Sets the percentage preview zoom, and refreshes the preview canvas
accordingly.

