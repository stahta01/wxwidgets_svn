\section{\class{wxExpr}}\label{wxexpr}

The {\bf wxExpr} class is the building brick of expressions similar to Prolog
clauses, or objects. It can represent an expression of type long integer, float, string, word,
or list, and lists can be nested.

\wxheading{Derived from}

None

\wxheading{Include files}

<wx/wxexpr.h>

\wxheading{See also}

\helpref{wxExpr overview}{exproverview}, \helpref{wxExprDatabase}{wxexprdatabase}

\latexignore{\rtfignore{\wxheading{Members}}}

\membersection{wxExpr::wxExpr}\label{wxexprconstr}

\func{}{wxExpr}{\param{const wxString\&}{functor}}

Construct a new clause with this form, supplying the functor name. A clause is an object
that will appear in the data file, with a list of attribute/value pairs.

\func{}{wxExpr}{\param{wxExprType}{ type}, \param{const wxString\&}{ wordOrString = ``"}}

Construct a new empty list, or a word (will be output with no quotes), or a string, depending on the
value of {\it type}.

{\it type} can be {\bf wxExprList}, {\bf wxExprWord}, or {\bf wxExprString}. If {\it type} is wxExprList,
the value of {\it wordOrString} will be ignored.

\func{}{wxExpr}{\param{long}{ value}}

Construct an integer expression.

\func{}{wxExpr}{\param{float}{ value}}

Construct a floating point expression.

\func{}{wxExpr}{\param{wxList* }{value}}

Construct a list expression. The list's nodes' data should
themselves be {\bf wxExpr}s.

wxExpr no longer uses the {\bf wxList} internally,
so this constructor turns the list into its internal
format (assuming a non-nested list) and then deletes the supplied
list.

\membersection{wxExpr::\destruct{wxExpr}}

\func{}{\destruct{wxExpr}}{\void}

Destructor.

\membersection{wxExpr::AddAttributeValue}\label{wxexpraddattributevalue}

Use these on clauses ONLY. Note that the functions for adding strings
and words must be differentiated by function name which is why
they are missing from this group (see \helpref{wxExpr::AddAttributeValueString}{wxexpraddattributevaluestring} and
\rtfsp\helpref{wxExpr::AddAttributeValueWord}{wxexpraddattributevalueword}).

\func{void}{AddAttributeValue}{\param{const wxString\&}{ attribute}, \param{float }{value}}

Adds an attribute and floating point value pair to the clause.

\func{void}{AddAttributeValue}{\param{const wxString\&}{ attribute}, \param{long }{value}}

Adds an attribute and long integer value pair to the clause.

\func{void}{AddAttributeValue}{\param{const wxString\&}{ attribute}, \param{wxList* }{value}}

Adds an attribute and list value pair to the clause, converting the list into
internal form and then deleting {\bf value}. Note that the list should not contain
nested lists (except if in internal {\bf wxExpr} form.)

\func{void}{AddAttributeValue}{\param{const wxString\&}{ attribute}, \param{wxExpr* }{value}}

Adds an attribute and wxExpr value pair to the clause. Do not delete\rtfsp
{\it value} once this function has been called.

\membersection{wxExpr::AddAttributeValueString}\label{wxexpraddattributevaluestring}

\func{void}{AddAttributeValueString}{\param{const wxString\&}{ attribute}, \param{const wxString\&}{ value}}

Adds an attribute and string value pair to the clause.

\membersection{wxExpr::AddAttributeValueStringList}\label{wxexpraddattributevaluestringlist}

\func{void}{AddAttributeValueStringList}{\param{const wxString\&}{ attribute}, \param{wxList* }{value}}

Adds an attribute and string list value pair to the clause.

Note that the list passed to this function is a list of strings, NOT a list
of {\bf wxExpr}s; it gets turned into a list of {\bf wxExpr}s
automatically. This is a convenience function, since lists of strings
are often manipulated in C++.

\membersection{wxExpr::AddAttributeValueWord}\label{wxexpraddattributevalueword}

\func{void}{AddAttributeValueWord}{\param{const wxString\&}{ attribute}, \param{const wxString\&}{ value}}

Adds an attribute and word value pair to the clause.

\membersection{wxExpr::Append}\label{wxexprappend}

\func{void}{Append}{\param{wxExpr*}{ value}}

Append the {\bf value} to the end of the list. `this' must be a list.

\membersection{wxExpr::Arg}\label{wxexprarg}

\constfunc{wxExpr* }{Arg}{\param{wxExprType}{ type}, \param{int}{ n}}

Get nth arg of the given clause (starting from 1). NULL is returned if
the expression is not a clause, or {\it n} is invalid, or the given type
does not match the actual type. See also \helpref{wxExpr::Nth}{wxexprnth}.

\membersection{wxExpr::Insert}\label{wxexprinsert}

\func{void}{Insert}{\param{wxExpr* }{value}}

Insert the {\bf value} at the start of the list. `this' must be a list.

\membersection{wxExpr::GetAttributeValue}\label{wxexprgetattributevalue}

These functions are the easiest way to retrieve attribute values, by
passing a pointer to variable. If the attribute is present, the
variable will be filled with the appropriate value.  If not, the
existing value is left alone.  This style of retrieving attributes
makes it easy to set variables to default values before calling these
functions; no code is necessary to check whether the attribute is
present or not.

\constfunc{bool}{GetAttributeValue}{\param{const wxString\&}{ attribute}, \param{wxString\&}{ value}}

Retrieve a string (or word) value.

\constfunc{bool}{GetAttributeValue}{\param{const wxString\&}{ attribute}, \param{float\& }{value}}

Retrieve a floating point value.

\constfunc{bool}{GetAttributeValue}{\param{const wxString\&}{ attribute}, \param{int\& }{value}}

Retrieve an integer value.

\constfunc{bool}{GetAttributeValue}{\param{const wxString\&}{ attribute}, \param{long\& }{value}}

Retrieve a long integer value.

\constfunc{bool}{GetAttributeValue}{\param{const wxString\&}{ attribute}, \param{wxExpr**}{ value}}

Retrieve a wxExpr pointer.

\membersection{wxExpr::GetAttributeValueStringList}\label{wxexprgetattributestringlist}

\constfunc{void}{GetAttributeValueStringList}{\param{const wxString\&}{attribute}, \param{wxList* }{value}}

Use this on clauses ONLY. See above for comments on this style of
attribute value retrieval. This function expects to receive a pointer to
a new list (created by the calling application); it will append strings
to the list if the attribute is present in the clause.

\membersection{wxExpr::AttributeValue}\label{wxexprattributevalue}

\constfunc{wxExpr*}{AttributeValue}{\param{const wxString\&}{ word}}

Use this on clauses ONLY. Searches the clause for an attribute
matching {\it word}, and returns the value associated with it.

\membersection{wxExpr::Copy}\label{wxexprcopy}

\constfunc{wxExpr*}{Copy}{\void}

Recursively copies the expression, allocating new storage space.

\membersection{wxExpr::DeleteAttributeValue}\label{wxexprdeletattributevalue}

\func{void}{DeleteAttributeValue}{\param{const wxString\&}{ attribute}}

Use this on clauses only. Deletes the attribute and its value (if any) from the
clause.

\membersection{wxExpr::Functor}\label{wxexprfunctor}

\constfunc{wxString}{Functor}{\void}

Use this on clauses only. Returns the clause's functor (object name).

\membersection{wxExpr::GetClientData}\label{wxexprgetclientdata}

\constfunc{wxObject*}{GetClientData}{\void}

Retrieve arbitrary data stored with this clause. This can be useful when
reading in data for storing a pointer to the C++ object, so when another
clause makes a reference to this clause, its C++ object can be retrieved.
See \helpref{wxExpr::SetClientData}{wxexprsetclientdata}.

\membersection{wxExpr::GetFirst}\label{wxexprgetfirst}

\constfunc{wxExpr*}{GetFirst}{\void}

If this is a list expression (or clause), gets the first element in the list.

See also \helpref{wxExpr::GetLast}{wxexprgetlast}, \helpref{wxExpr::GetNext}{wxexprgetnext}, \helpref{wxExpr::Nth}{wxexprnth}.

\membersection{wxExpr::GetLast}\label{wxexprgetlast}

\constfunc{wxExpr*}{GetLast}{\void}

If this is a list expression (or clause), gets the last element in the list.

See also \helpref{wxExpr::GetFirst}{wxexprgetfirst}, \helpref{wxExpr::GetNext}{wxexprgetnext}, \helpref{wxExpr::Nth}{wxexprnth}.

\membersection{wxExpr::GetNext}\label{wxexprgetnext}

\constfunc{wxExpr*}{GetNext}{\void}

If this is a node in a list (any wxExpr may be a node in a list), gets the
next element in the list.

See also \helpref{wxExpr::GetFirst}{wxexprgetfirst}, \helpref{wxExpr::GetLast}{wxexprgetlast}, \helpref{wxExpr::Nth}{wxexprnth}.

\membersection{wxExpr::IntegerValue}\label{wxexprintegervalue}

\constfunc{long}{IntegerValue}{\void}

Returns the integer value of the expression.

\membersection{wxExpr::Nth}\label{wxexprnth}

\constfunc{wxExpr*}{Nth}{\param{int}{ n}}

Get nth arg of the given list expression (starting from 0). NULL is returned if
the expression is not a list expression, or {\it n} is invalid. See also \helpref{wxExpr::Arg}{wxexprarg}.

Normally, you would use attribute-value pairs to add and retrieve data
from objects (clauses) in a data file. However, if the data gets complex,
you may need to store attribute values as lists, and pick them apart
yourself.

\membersection{wxExpr::RealValue}\label{wxexprrealvalue}

\constfunc{float}{RealValue}{\void}

Returns the floating point value of the expression.

\membersection{wxExpr::SetClientData}\label{wxexprsetclientdata}

\func{void}{SetClientData}{\param{wxObject *}{data}}

Associate arbitrary data with this clause. This can be useful when
reading in data for storing a pointer to the C++ object, so when another
clause makes a reference to this clause, its C++ object can be retrieved.
See \helpref{wxExpr::GetClientData}{wxexprgetclientdata}.

\membersection{wxExpr::StringValue}\label{wxexprstringvalue}

\constfunc{wxString}{StringValue}{\void}

Returns the string value of the expression.

\membersection{wxExpr::Type}\label{wxexprtype}

\constfunc{wxExprType}{Type}{\void}

Returns the type of the expression. {\bf wxExprType} is defined as follows:

\begin{verbatim}
typedef enum {
    wxExprNull,
    wxExprInteger,
    wxExprReal,
    wxExprWord,
    wxExprString,
    wxExprList
} wxExprType;
\end{verbatim}

\membersection{wxExpr::WordValue}\label{wxexprwordvalue}

\constfunc{wxString}{WordValue}{\void}

Returns the word value of the expression.

\membersection{wxExpr::WriteClause}\label{wxexprwriteprologclause}

\func{void}{WriteClause}{\param{FILE *}{ stream}}

Writes the clause to the given stream in Prolog format. Not normally needed, since
the whole {\bf wxExprDatabase} will usually be written at once. The format is:
functor, open parenthesis, list of comma-separated expressions, close parenthesis,
full stop.

\membersection{wxExpr::WriteExpr}\label{wxexprwriteexpr}

\func{void}{WriteExpr}{\param{FILE *}{ stream}}

Writes the expression (not clause) to the given stream in Prolog
format. Not normally needed, since the whole {\bf wxExprDatabase} will
usually be written at once. Lists are written in square bracketed,
comma-delimited format.

\membersection{Functions and macros}

Below are miscellaneous functions and macros associated with wxExpr objects.

\func{bool}{wxExprIsFunctor}{\param{wxExpr *}{expr}, \param{const wxString\&}{ functor}}

Checks that the functor of {\it expr} is {\it functor}.

\func{void}{wxExprCleanUp}{\void}

Cleans up the wxExpr system (YACC/LEX buffers) to avoid memory-checking warnings as the program exits.

\begin{verbatim}
#define wxMakeInteger(x) (new wxExpr((long)x))
#define wxMakeReal(x)    (new wxExpr((float)x))
#define wxMakeString(x)  (new wxExpr(PrologString, x))
#define wxMakeWord(x)    (new wxExpr(PrologWord, x))
#define wxMake(x)        (new wxExpr(x))
\end{verbatim}

Macros to help make wxExpr objects.

\section{\class{wxExprDatabase}}\label{wxexprdatabase}

The {\bf wxExprDatabase} class represents a database, or list,
of Prolog-like expressions.  Instances of this class are used for reading,
writing and creating data files.

\wxheading{Derived from}

\helpref{wxList}{wxlist}\\
\helpref{wxObject}{wxobject}

\wxheading{See also}

\helpref{wxExpr overview}{exproverview}, \helpref{wxExpr}{wxexpr}

\latexignore{\rtfignore{\wxheading{Members}}}

\membersection{wxExprDatabase::wxExprDatabase}\label{wxexprdatabaseconstr}

\func{void}{wxExprDatabase}{\param{proioErrorHandler}{ handler = 0}}

Construct a new, unhashed database, with an optional error handler. The
error handler must be a function returning a bool and taking an integer and a string
argument. When an error occurs when reading or writing a database, this function is
called. The error is given as the first argument (currently one of WXEXPR\_ERROR\_GENERAL,
WXEXPR\_ERROR\_SYNTAX) and an error message is given as the second argument. If FALSE
is returned by the error handler, processing of the wxExpr operation stops.

Another way of handling errors is simply to call \helpref{wxExprDatabase::GetErrorCount}{wxexprdatabasegeterrorcount} after
the operation, to check whether errors have occurred, instead of installing an error handler.
If the error count is more than zero, \helpref{wxExprDatabase::Write}{wxexprdatabasewrite} and
\rtfsp\helpref{wxExprDatabase::Read}{wxexprdatabaseread} will return FALSE to
the application.

For example:

\begin{verbatim}
bool myErrorHandler(int err, chat *msg)
{
  if (err == WXEXPR_ERROR_SYNTAX)
  {
    wxMessageBox(msg, "Syntax error");
  }
  return FALSE;
}

wxExprDatabase database(myErrorHandler);
\end{verbatim}

\func{}{wxExprDatabase}{\param{wxExprType}{ type}, \param{const wxString\&}{attribute},
\rtfsp\param{int}{ size = 500}, \param{proioErrorHandler}{ handler = 0}}

Construct a new database hashed on a combination of the clause functor and
a named attribute (often an integer identification).

See above for an explanation of the error handler.

\membersection{wxExprDatabase::\destruct{wxExprDatabase}}

\func{}{\destruct{wxExprDatabase}}{\void}

Delete the database and contents.

\membersection{wxExprDatabase::Append}\label{wxexprdatabaseappend}

\func{void}{Append}{\param{wxExpr* }{clause}}

Append a clause to the end of the database. If the database is hashing,
the functor and a user-specified attribute will be hashed upon, giving the
option of random access in addition to linear traversal of the database.

\membersection{wxExprDatabase::BeginFind}\label{wxexprdatabasebeginfind}

\func{void}{BeginFind}{\void}

Reset the current position to the start of the database. Subsequent
\rtfsp\helpref{wxExprDatabase::FindClause}{wxexprdatabasefindclause} calls will move the pointer.

\membersection{wxExprDatabase::ClearDatabase}\label{wxexprdatabasecleardatabase}

\func{void}{ClearDatabase}{\void}

Clears the contents of the database.

\membersection{wxExprDatabase::FindClause}\label{wxexprdatabasefindclause}

Various ways of retrieving clauses from the database. A return
value of NULL indicates no (more) clauses matching the given criteria.
Calling the functions repeatedly retrieves more matching clauses, if any.

\func{wxExpr* }{FindClause}{\param{long}{ id}}

Find a clause based on the special ``id'' attribute.

\func{wxExpr* }{FindClause}{\param{const wxString\&}{ attribute}, \param{const wxString\&}{ value}}

Find a clause which has the given attribute set to the given string or word value.

\func{wxExpr*}{FindClause}{\param{const wxString\&}{ attribute}, \param{long}{ value}}

Find a clause which has the given attribute set to the given integer value.

\func{wxExpr*}{FindClause}{\param{const wxString\&}{ attribute}, \param{float}{ value}}

Find a clause which has the given attribute set to the given floating point value.

\membersection{wxExprDatabase::FindClauseByFunctor}\label{wxexprdatabasefindclausebyfunctor}

\func{wxExpr*}{FindClauseByFunctor}{\param{const wxString\&}{ functor}}

Find the next clause with the specified functor.

\membersection{wxExprDatabase::GetErrorCount}\label{wxexprdatabasegeterrorcount}

\constfunc{int}{GetErrorCount}{\void}

Returns the number of errors encountered during the last read or write operation.

\membersection{wxExprDatabase::HashFind}\label{wxexprdatabasehashfind}

\constfunc{wxExpr*}{HashFind}{\param{const wxString\&}{ functor}, \param{long}{ value}}

Finds the clause with the given functor and with the attribute specified
in the database constructor having the given integer value.

For example,

\begin{verbatim}
// Hash on a combination of functor and integer "id" attribute when reading in
wxExprDatabase db(wxExprInteger, "id");

// Read it in
db.ReadProlog("data");

// Retrieve a clause with specified functor and id
wxExpr *clause = db.HashFind("node", 24);
\end{verbatim}

This would retrieve a clause which is written: {\tt node(id = 24, ..., )}.

\func{wxExpr*}{HashFind}{\param{const wxString\&}{ functor}, \param{const wxString\&}{ value}}

Finds the clause with the given functor and with the attribute specified
in the database constructor having the given string value.

\membersection{wxExprDatabase::Read}\label{wxexprdatabaseread}

\func{bool}{Read}{\param{const wxString\&}{ filename}}

Reads in the given file, returning TRUE if successful.

\membersection{wxExprDatabase::ReadFromString}\label{wxexprdatabasereadfromstring}

\func{bool}{ReadFromString}{\param{const wxString\&}{ buffer}}

Reads a Prolog database from the given string buffer, returning TRUE if
successful.

\membersection{wxExprDatabase::Write}\label{wxexprdatabasewrite}

\func{bool}{Write}{\param{FILE *}{stream}}

\func{bool}{Write}{\param{const wxString\&}{ filename}}

Writes the database as a Prolog-format file.


