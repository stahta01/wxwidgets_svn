\section{\class{wxIcon}}\label{wxicon}

An icon is a small rectangular bitmap usually used for denoting a
minimized application. It differs from a wxBitmap in always
having a mask associated with it for transparent drawing. On some platforms,
icons and bitmaps are implemented identically, since there is no real distinction between
a wxBitmap with a mask and an icon; and there is no specific icon format on
some platforms (X-based applications usually standardize on XPMs for small bitmaps
and icons). However, some platforms (such as Windows) make the distinction, so
a separate class is provided.

\wxheading{Derived from}

\helpref{wxBitmap}{wxbitmap}\\
\helpref{wxGDIObject}{wxgdiobject}\\
\helpref{wxObject}{wxobject}

\wxheading{Include files}

<wx/icon.h>

\wxheading{Predefined objects}

Objects:

{\bf wxNullIcon}

\wxheading{Remarks}

It is usually desirable to associate a pertinent icon with a frame. Icons
can also be used for other purposes, for example with \helpref{wxTreeCtrl}{wxtreectrl} 
and \helpref{wxListCtrl}{wxlistctrl}.

Icons have different formats on different platforms.
Therefore, separate icons will usually be created for the different
environments.  Platform-specific methods for creating a {\bf wxIcon}\rtfsp
structure are catered for, and this is an occasion where conditional
compilation will probably be required.

Note that a new icon must be created for every time the icon is to be
used for a new window. In Windows, the icon will not be
reloaded if it has already been used. An icon allocated to a frame will
be deleted when the frame is deleted.

For more information please see \helpref{Bitmap and icon overview}{wxbitmapoverview}.

\wxheading{See also}

\helpref{Bitmap and icon overview}{wxbitmapoverview}, \helpref{supported bitmap file formats}{supportedbitmapformats}, 
\helpref{wxDC::DrawIcon}{wxdcdrawicon}, \helpref{wxCursor}{wxcursor}

\latexignore{\rtfignore{\wxheading{Members}}}

\membersection{wxIcon::wxIcon}\label{wxiconconstr}

\func{}{wxIcon}{\void}

Default constructor.

\func{}{wxIcon}{\param{const wxIcon\& }{icon}}

Copy constructor.

\func{}{wxIcon}{\param{void*}{ data}, \param{int}{ type}, \param{int}{ width}, \param{int}{ height}, \param{int}{ depth = -1}}

Creates an icon from the given data, which can be of arbitrary type.

\func{}{wxIcon}{\param{const char}{ bits[]}, \param{int}{ width}, \param{int}{ height}\\
  \param{int}{ depth = 1}}

Creates an icon from an array of bits.

\func{}{wxIcon}{\param{int}{ width}, \param{int}{ height}, \param{int}{ depth = -1}}

Creates a new icon.

\func{}{wxIcon}{\param{char**}{ bits}}

\func{}{wxIcon}{\param{const char**}{ bits}}

Creates an icon from XPM data.

\func{}{wxIcon}{\param{const wxString\& }{name}, \param{long}{ type},
  \param{int}{ desiredWidth = -1}, \param{int}{ desiredHeight = -1}}

Loads an icon from a file or resource.

\wxheading{Parameters}

\docparam{bits}{Specifies an array of pixel values.}

\docparam{width}{Specifies the width of the icon.}

\docparam{height}{Specifies the height of the icon.}

\docparam{desiredWidth}{Specifies the desired width of the icon. This
parameter only has an effect in Windows (32-bit) where icon resources can contain
several icons of different sizes.}

\docparam{desiredWidth}{Specifies the desired height of the icon. This
parameter only has an effect in Windows (32-bit) where icon resources can contain
several icons of different sizes.}

\docparam{depth}{Specifies the depth of the icon. If this is omitted, the display depth of the
screen is used.}

\docparam{name}{This can refer to a resource name under MS Windows, or a filename under MS Windows and X.
Its meaning is determined by the {\it flags} parameter.}

\docparam{type}{May be one of the following:

\twocolwidtha{5cm}
\begin{twocollist}
\twocolitem{\indexit{wxBITMAP\_TYPE\_ICO}}{Load a Windows icon file.}
\twocolitem{\indexit{wxBITMAP\_TYPE\_ICO\_RESOURCE}}{Load a Windows icon from the resource database.}
\twocolitem{\indexit{wxBITMAP\_TYPE\_GIF}}{Load a GIF bitmap file.}
\twocolitem{\indexit{wxBITMAP\_TYPE\_XBM}}{Load an X bitmap file.}
\twocolitem{\indexit{wxBITMAP\_TYPE\_XPM}}{Load an XPM bitmap file.}
%\twocolitem{\indexit{wxBITMAP\_TYPE\_RESOURCE}}{Load a Windows resource name.}
\end{twocollist}

The validity of these flags depends on the platform and wxWindows configuration.
If all possible wxWindows settings are used, the Windows platform supports ICO file, ICO resource,
XPM data, and XPM file. Under wxGTK, the available formats are BMP file, XPM data, XPM file, and PNG file.
Under wxMotif, the available formats are XBM data, XBM file, XPM data, XPM file.}

\wxheading{Remarks}

The first form constructs an icon object with no data; an assignment or another member function such as Create
or LoadFile must be called subsequently.

The second and third forms provide copy constructors. Note that these do not copy the
icon data, but instead a pointer to the data, keeping a reference count. They are therefore
very efficient operations.

The fourth form constructs an icon from data whose type and value depends on
the value of the {\it type} argument.

The fifth form constructs a (usually monochrome) icon from an array of pixel values, under both
X and Windows.

The sixth form constructs a new icon.

The seventh form constructs an icon from pixmap (XPM) data, if wxWindows has been configured
to incorporate this feature.

To use this constructor, you must first include an XPM file. For
example, assuming that the file {\tt mybitmap.xpm} contains an XPM array
of character pointers called mybitmap:

\begin{verbatim}
#include "mybitmap.xpm"

...

wxIcon *icon = new wxIcon(mybitmap);
\end{verbatim}

A macro, wxICON, is available which creates an icon using an XPM
on the appropriate platform, or an icon resource on Windows.

\begin{verbatim}
wxIcon icon(wxICON(mondrian));

// Equivalent to:

#if defined(__WXGTK__) || defined(__WXMOTIF__)
wxIcon icon(mondrian_xpm);
#endif

#if defined(__WXMSW__)
wxIcon icon("mondrian");
#endif
\end{verbatim}

The eighth form constructs an icon from a file or resource. {\it name} can refer
to a resource name under MS Windows, or a filename under MS Windows and X.

Under Windows, {\it type} defaults to wxBITMAP\_TYPE\_ICO\_RESOURCE.
Under X, {\it type} defaults to wxBITMAP\_TYPE\_XPM.

\wxheading{See also}


\membersection{wxIcon::CopyFromBitmap}\label{wxiconcopyfrombitmap}

\func{void}{CopyFromBitmap}{\param{const wxBitmap\&}{ bmp}}

Copies {\it bmp} bitmap to this icon. Under MS Windows the bitmap
must have mask colour set.


\helpref{wxIcon::LoadFile}{wxiconloadfile}

\perlnote{Constructors supported by wxPerl are:\par
\begin{itemize}
\item{Wx::Icon->new( width, height, depth = -1 )}
\item{Wx::Icon->new( name, type, desiredWidth = -1, desiredHeight = -1 )}
\end{itemize}
}

\membersection{wxIcon::\destruct{wxIcon}}

\func{}{\destruct{wxIcon}}{\void}

Destroys the wxIcon object and possibly the underlying icon data.
Because reference counting is used, the icon may not actually be
destroyed at this point - only when the reference count is zero will the
data be deleted.

If the application omits to delete the icon explicitly, the icon will be
destroyed automatically by wxWindows when the application exits.

Do not delete an icon that is selected into a memory device context.

\membersection{wxIcon::GetDepth}

\constfunc{int}{GetDepth}{\void}

Gets the colour depth of the icon. A value of 1 indicates a
monochrome icon.

\membersection{wxIcon::GetHeight}\label{wxicongetheight}

\constfunc{int}{GetHeight}{\void}

Gets the height of the icon in pixels.

\membersection{wxIcon::GetWidth}\label{wxicongetwidth}

\constfunc{int}{GetWidth}{\void}

Gets the width of the icon in pixels.

\wxheading{See also}

\helpref{wxIcon::GetHeight}{wxicongetheight}

\membersection{wxIcon::LoadFile}\label{wxiconloadfile}

\func{bool}{LoadFile}{\param{const wxString\&}{ name}, \param{long}{ type}}

Loads an icon from a file or resource.

\wxheading{Parameters}

\docparam{name}{Either a filename or a Windows resource name.
The meaning of {\it name} is determined by the {\it type} parameter.}

\docparam{type}{One of the following values:

\twocolwidtha{5cm}
\begin{twocollist}
\twocolitem{{\bf wxBITMAP\_TYPE\_ICO}}{Load a Windows icon file.}
\twocolitem{{\bf wxBITMAP\_TYPE\_ICO\_RESOURCE}}{Load a Windows icon from the resource database.}
\twocolitem{{\bf wxBITMAP\_TYPE\_GIF}}{Load a GIF bitmap file.}
\twocolitem{{\bf wxBITMAP\_TYPE\_XBM}}{Load an X bitmap file.}
\twocolitem{{\bf wxBITMAP\_TYPE\_XPM}}{Load an XPM bitmap file.}
\end{twocollist}

The validity of these flags depends on the platform and wxWindows configuration.}

\wxheading{Return value}

TRUE if the operation succeeded, FALSE otherwise.

\wxheading{See also}

\helpref{wxIcon::wxIcon}{wxiconconstr}

\membersection{wxIcon::Ok}\label{wxiconok}

\constfunc{bool}{Ok}{\void}

Returns TRUE if icon data is present.

\begin{comment}
\membersection{wxIcon::SaveFile}\label{wxiconsavefile}

\func{bool}{SaveFile}{\param{const wxString\& }{name}, \param{int}{ type}, \param{wxPalette* }{palette = NULL}}

Saves an icon in the named file.

\wxheading{Parameters}

\docparam{name}{A filename. The meaning of {\it name} is determined by the {\it type} parameter.}

\docparam{type}{One of the following values:

\twocolwidtha{5cm}
\begin{twocollist}
\twocolitem{{\bf wxBITMAP\_TYPE\_ICO}}{Save a Windows icon file.}
%\twocolitem{{\bf wxBITMAP\_TYPE\_GIF}}{Save a GIF icon file.}
%\twocolitem{{\bf wxBITMAP\_TYPE\_XBM}}{Save an X bitmap file.}
\twocolitem{{\bf wxBITMAP\_TYPE\_XPM}}{Save an XPM bitmap file.}
\end{twocollist}

The validity of these flags depends on the platform and wxWindows configuration.}

\docparam{palette}{An optional palette used for saving the icon.}

\wxheading{Return value}

TRUE if the operation succeeded, FALSE otherwise.

\wxheading{Remarks}

Depending on how wxWindows has been configured, not all formats may be available.

\wxheading{See also}

\helpref{wxIcon::LoadFile}{wxiconloadfile}
\end{comment}

\membersection{wxIcon::SetDepth}\label{wxiconsetdepth}

\func{void}{SetDepth}{\param{int }{depth}}

Sets the depth member (does not affect the icon data).

\wxheading{Parameters}

\docparam{depth}{Icon depth.}

\membersection{wxIcon::SetHeight}\label{wxiconsetheight}

\func{void}{SetHeight}{\param{int }{height}}

Sets the height member (does not affect the icon data).

\wxheading{Parameters}

\docparam{height}{Icon height in pixels.}

\membersection{wxIcon::SetOk}

\func{void}{SetOk}{\param{int }{isOk}}

Sets the validity member (does not affect the icon data).

\wxheading{Parameters}

\docparam{isOk}{Validity flag.}

\membersection{wxIcon::SetWidth}

\func{void}{SetWidth}{\param{int }{width}}

Sets the width member (does not affect the icon data).

\wxheading{Parameters}

\docparam{width}{Icon width in pixels.}

\membersection{wxIcon::operator $=$}

\func{wxIcon\& }{operator $=$}{\param{const wxIcon\& }{icon}}

Assignment operator. This operator does not copy any data, but instead
passes a pointer to the data in {\it icon} and increments a reference
counter. It is a fast operation.

\wxheading{Parameters}

\docparam{icon}{Icon to assign.}

\wxheading{Return value}

Returns 'this' object.

\membersection{wxIcon::operator $==$}

\func{bool}{operator $==$}{\param{const wxIcon\& }{icon}}

Equality operator. This operator tests whether the internal data pointers are
equal (a fast test).

\wxheading{Parameters}

\docparam{icon}{Icon to compare with 'this'}

\wxheading{Return value}

Returns TRUE if the icons were effectively equal, FALSE otherwise.

\membersection{wxIcon::operator $!=$}

\func{bool}{operator $!=$}{\param{const wxIcon\& }{icon}}

Inequality operator. This operator tests whether the internal data pointers are
unequal (a fast test).

\wxheading{Parameters}

\docparam{icon}{Icon to compare with 'this'}

\wxheading{Return value}

Returns TRUE if the icons were unequal, FALSE otherwise.


