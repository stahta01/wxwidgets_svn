% manual page source format generated by PolyglotMan v3.0.9,
% available from http://polyglotman.sourceforge.net/

\section{Syntax of the builtin regular expression library}\label{wxresyn}

A {\it regular expression} describes strings of characters. It's a
pattern that matches certain strings and doesn't match others.  

\wxheading{See also}

\helpref{wxRegEx}{wxregex}

\subsection{Different Flavors of REs}\label{differentflavors}

\helpref{Syntax of the builtin regular expression library}{wxresyn}

Regular expressions (``RE''s), as defined by POSIX, come in two
flavors: {\it extended} REs (``EREs'') and {\it basic} REs (``BREs''). EREs are roughly those
of the traditional {\it egrep}, while BREs are roughly those of the traditional
 {\it ed}.  This implementation adds a third flavor, {\it advanced} REs (``AREs''), basically
EREs with some significant extensions. 

This manual page primarily describes
AREs. BREs mostly exist for backward compatibility in some old programs;
they will be discussed at the \helpref{end}{wxresynbre}. POSIX EREs are almost an exact subset
of AREs. Features of AREs that are not present in EREs will be indicated.

\subsection{Regular Expression Syntax}\label{resyntax}

\helpref{Syntax of the builtin regular expression library}{wxresyn}

These regular expressions are implemented using
the package written by Henry Spencer, based on the 1003.2 spec and some
(not quite all) of the Perl5 extensions (thanks, Henry!).  Much of the description
of regular expressions below is copied verbatim from his manual entry. 

An ARE is one or more {\it branches}, separated by `{\bf $|$}', matching anything that matches
any of the branches. 

A branch is zero or more {\it constraints} or {\it quantified
atoms}, concatenated. It matches a match for the first, followed by a match
for the second, etc; an empty branch matches the empty string. 

A quantified atom is an {\it atom} possibly followed by a single {\it quantifier}. Without a quantifier,
it matches a match for the atom. The quantifiers, and what a so-quantified
atom matches, are:

\begin{twocollist}\twocolwidtha{4cm}
\twocolitem{{\bf *}}{a sequence of 0 or more matches of the atom}
\twocolitem{{\bf +}}{a sequence of 1 or more matches of the atom}
\twocolitem{{\bf ?}}{a sequence of 0 or 1 matches of the atom}
\twocolitem{{\bf \{m\}}}{a sequence of exactly {\it m} matches of the atom}
\twocolitem{{\bf \{m,\}}}{a sequence of {\it m} or more matches of the atom}
\twocolitem{{\bf \{m,n\}}}{a sequence of {\it m} through {\it n} (inclusive)
matches of the atom; {\it m} may not exceed {\it n}}
\twocolitem{{\bf *?  +?  ??  \{m\}?  \{m,\}?  \{m,n\}?}}{{\it non-greedy} quantifiers,
which match the same possibilities, but prefer the
smallest number rather than the largest number of matches (see \helpref{Matching}{wxresynmatching})}
\end{twocollist}

The forms using {\bf \{} and {\bf \}} are known as {\it bound}s. The numbers {\it m} and {\it n} are unsigned
decimal integers with permissible values from 0 to 255 inclusive. 
An atom is one of:

\begin{twocollist}\twocolwidtha{4cm}
\twocolitem{{\bf (re)}}{(where {\it re} is any regular expression) matches a match for
 {\it re}, with the match noted for possible reporting}
\twocolitem{{\bf (?:re)}}{as previous, but
does no reporting (a ``non-capturing'' set of parentheses)}
\twocolitem{{\bf ()}}{matches an empty
string, noted for possible reporting}
\twocolitem{{\bf (?:)}}{matches an empty string, without reporting}
\twocolitem{{\bf $[chars]$}}{a {\it bracket expression}, matching any one of the {\it chars}
(see \helpref{Bracket Expressions}{wxresynbracket} for more detail)}
\twocolitem{{\bf .}}{matches any single character }
\twocolitem{{\bf $\backslash$k}}{(where {\it k} is a non-alphanumeric character)
matches that character taken as an ordinary character, e.g. $\backslash\backslash$ matches a backslash
character}
\twocolitem{{\bf $\backslash$c}}{where {\it c} is alphanumeric (possibly followed by other characters),
an {\it escape} (AREs only), see \helpref{Escapes}{wxresynescapes} below}
\twocolitem{{\bf \{}}{when followed by a character
other than a digit, matches the left-brace character `{\bf \{}'; when followed by
a digit, it is the beginning of a {\it bound} (see above)}
\twocolitem{{\bf x}}{where {\it x} is a single
character with no other significance, matches that character.}
\end{twocollist}

A {\it constraint} matches an empty string when specific conditions are met. A constraint may
not be followed by a quantifier. The simple constraints are as follows;
some more constraints are described later, under \helpref{Escapes}{wxresynescapes}.

\begin{twocollist}\twocolwidtha{4cm}
\twocolitem{{\bf \caret}}{matches at the beginning of a line}
\twocolitem{{\bf \$}}{matches at the end of a line}
\twocolitem{{\bf (?=re)}}{{\it positive lookahead}
(AREs only), matches at any point where a substring matching {\it re} begins}
\twocolitem{{\bf (?!re)}}{{\it negative lookahead} (AREs only),
matches at any point where no substring matching {\it re} begins}
\end{twocollist}

The lookahead constraints may not contain back references
(see later), and all parentheses within them are considered non-capturing.

An RE may not end with `{\bf $\backslash$}'.

\subsection{Bracket Expressions}\label{wxresynbracket}

\helpref{Syntax of the builtin regular expression library}{wxresyn}

A {\it bracket expression} is a list
of characters enclosed in `{\bf $[]$}'. It normally matches any single character from
the list (but see below). If the list begins with `{\bf \caret}', it matches any single
character (but see below) {\it not} from the rest of the list. 

If two characters
in the list are separated by `{\bf -}', this is shorthand for the full {\it range} of
characters between those two (inclusive) in the collating sequence, e.g.
 {\bf $[0-9]$} in ASCII matches any decimal digit. Two ranges may not share an endpoint,
so e.g. {\bf a-c-e} is illegal. Ranges are very collating-sequence-dependent, and portable
programs should avoid relying on them. 

To include a literal {\bf $]$} or {\bf -} in the
list, the simplest method is to enclose it in {\bf $[.$} and {\bf $.]$} to make it a collating
element (see below). Alternatively, make it the first character (following
a possible `{\bf \caret}'), or (AREs only) precede it with `{\bf $\backslash$}'.
Alternatively, for `{\bf -}', make
it the last character, or the second endpoint of a range. To use a literal
 {\bf -} as the first endpoint of a range, make it a collating element or (AREs
only) precede it with `{\bf $\backslash$}'. With the exception of these, some combinations using
 {\bf $[$} (see next paragraphs), and escapes, all other special characters lose
their special significance within a bracket expression. 

Within a bracket
expression, a collating element (a character, a multi-character sequence
that collates as if it were a single character, or a collating-sequence
name for either) enclosed in {\bf $[.$} and {\bf $.]$} stands for the
sequence of characters of that collating element.

{\it wxWidgets}: Currently no multi-character collating elements are defined.
So in {\bf $[.X.]$}, {\it X} can either be a single character literal or
the name of a character. For example, the following are both identical
 {\bf $[[.0.]-[.9.]]$} and {\bf $[[.zero.]-[.nine.]]$} and mean the same as
 {\bf $[0-9]$}.
 See \helpref{Character Names}{wxresynchars}.

%The sequence is a single element of the bracket
%expression's list. A bracket expression in a locale that has multi-character
%collating elements can thus match more than one character. So (insidiously),
%a bracket expression that starts with {\bf \caret} can match multi-character collating
%elements even if none of them appear in the bracket expression! ({\it Note:}
%Tcl currently has no multi-character collating elements. This information
%is only for illustration.) 
%
%For example, assume the collating sequence includes
%a {\bf ch} multi-character collating element. Then the RE {\bf $[[.ch.]]*c$} (zero or more
% {\bf ch}'s followed by {\bf c}) matches the first five characters of `{\bf chchcc}'. Also, the
%RE {\bf $[^c]b$} matches all of `{\bf chb}' (because {\bf $[^c]$} matches the multi-character {\bf ch}).

Within a bracket expression, a collating element enclosed in {\bf $[=$} and {\bf $=]$}
is an equivalence class, standing for the sequences of characters of all
collating elements equivalent to that one, including itself.
%(If there are
%no other equivalent collating elements, the treatment is as if the enclosing
%delimiters were `{\bf $[.$}' and `{\bf $.]$}'.) For example, if {\bf o}
%and {\bf \caret} are the members of an
%equivalence class, then `{\bf $[[$=o=$]]$}', `{\bf $[[$=\caret=$]]$}',
%and `{\bf $[o^]$}' are all synonymous.
An equivalence class may not be an endpoint of a range.

%({\it Note:}  Tcl currently
%implements only the Unicode locale. It doesn't define any equivalence classes.
%The examples above are just illustrations.) 

{\it wxWidgets}: Currently no equivalence classes are defined, so 
{\bf $[=X=]$} stands for just the single character {\it X}. 
 {\it X} can either be a single character literal or the name of a character,
see \helpref{Character Names}{wxresynchars}.

Within a bracket expression,
the name of a {\it character class} enclosed in {\bf $[:$} and {\bf $:]$} stands for the list
of all characters (not all collating elements!) belonging to that class.
Standard character classes are:

\begin{twocollist}\twocolwidtha{3cm}
\twocolitem{{\bf alpha}}{A letter.}
\twocolitem{{\bf upper}}{An upper-case letter.}
\twocolitem{{\bf lower}}{A lower-case letter.}
\twocolitem{{\bf digit}}{A decimal digit.}
\twocolitem{{\bf xdigit}}{A hexadecimal digit.}
\twocolitem{{\bf alnum}}{An alphanumeric (letter or digit).}
\twocolitem{{\bf print}}{An alphanumeric (same as alnum).}
\twocolitem{{\bf blank}}{A space or tab character.}
\twocolitem{{\bf space}}{A character producing white space in displayed text.}
\twocolitem{{\bf punct}}{A punctuation character.}
\twocolitem{{\bf graph}}{A character with a visible representation.}
\twocolitem{{\bf cntrl}}{A control character.}
\end{twocollist}

%A locale may provide others.  (Note that the  current  Tcl
%implementation  has  only one locale: the Unicode locale.)
A character class may not be used as an endpoint of a range. 

{\it wxWidgets}: In a non-Unicode build, these character classifications depend on the
current locale, and correspond to the values return by the ANSI C 'is'
functions: isalpha, isupper, etc. In Unicode mode they are based on
Unicode classifications, and are not affected by the current locale.

There are two special cases of bracket expressions:
the bracket expressions {\bf $[[:$<$:]]$} and {\bf $[[:$>$:]]$} are constraints, matching empty
strings at the beginning and end of a word respectively.  A word is defined
as a sequence of word characters that is neither preceded nor followed
by word characters. A word character is an {\it alnum} character or an underscore
({\bf \_}). These special bracket expressions are deprecated; users of AREs should
use constraint escapes instead (see \helpref{Escapes}{wxresynescapes} below). 

\subsection{Escapes}\label{wxresynescapes}

\helpref{Syntax of the builtin regular expression library}{wxresyn}

Escapes (AREs only),
which begin with a {\bf $\backslash$} followed by an alphanumeric character, come in several
varieties: character entry, class shorthands, constraint escapes, and back
references. A {\bf $\backslash$} followed by an alphanumeric character but not constituting
a valid escape is illegal in AREs. In EREs, there are no escapes: outside
a bracket expression, a {\bf $\backslash$} followed by an alphanumeric character merely stands
for that character as an ordinary character, and inside a bracket expression,
 {\bf $\backslash$} is an ordinary character. (The latter is the one actual incompatibility
between EREs and AREs.) 

Character-entry escapes (AREs only) exist to make
it easier to specify non-printing and otherwise inconvenient characters
in REs:

\begin{twocollist}\twocolwidtha{4cm}
\twocolitem{{\bf $\backslash$a}}{alert (bell) character, as in C}
\twocolitem{{\bf $\backslash$b}}{backspace, as in C}
\twocolitem{{\bf $\backslash$B}}{synonym
for {\bf $\backslash$} to help reduce backslash doubling in some applications where there
are multiple levels of backslash processing}
\twocolitem{{\bf $\backslash$c{\it X}}}{(where X is any character)
the character whose low-order 5 bits are the same as those of {\it X}, and whose
other bits are all zero}
\twocolitem{{\bf $\backslash$e}}{the character whose collating-sequence name is
`{\bf ESC}', or failing that, the character with octal value 033}
\twocolitem{{\bf $\backslash$f}}{formfeed, as in C}
\twocolitem{{\bf $\backslash$n}}{newline, as in C}
\twocolitem{{\bf $\backslash$r}}{carriage return, as in C}
\twocolitem{{\bf $\backslash$t}}{horizontal tab, as in C}
\twocolitem{{\bf $\backslash$u{\it wxyz}}}{(where {\it wxyz} is exactly four hexadecimal digits)
the Unicode
character {\bf U+{\it wxyz}} in the local byte ordering}
\twocolitem{{\bf $\backslash$U{\it stuvwxyz}}}{(where {\it stuvwxyz} is
exactly eight hexadecimal digits) reserved for a somewhat-hypothetical Unicode
extension to 32 bits}
\twocolitem{{\bf $\backslash$v}}{vertical tab, as in C are all available.}
\twocolitem{{\bf $\backslash$x{\it hhh}}}{(where
 {\it hhh} is any sequence of hexadecimal digits) the character whose hexadecimal
value is {\bf 0x{\it hhh}} (a single character no matter how many hexadecimal digits
are used).}
\twocolitem{{\bf $\backslash$0}}{the character whose value is {\bf 0}}
\twocolitem{{\bf $\backslash${\it xy}}}{(where {\it xy} is exactly two
octal digits, and is not a {\it back reference} (see below)) the character whose
octal value is {\bf 0{\it xy}}}
\twocolitem{{\bf $\backslash${\it xyz}}}{(where {\it xyz} is exactly three octal digits, and is
not a back reference (see below))
the character whose octal value is {\bf 0{\it xyz}}}
\end{twocollist}

Hexadecimal digits are `{\bf 0}'-`{\bf 9}', `{\bf a}'-`{\bf f}', and `{\bf A}'-`{\bf F}'. Octal
digits are `{\bf 0}'-`{\bf 7}'. 

The character-entry
escapes are always taken as ordinary characters. For example, {\bf $\backslash$135} is {\bf ]} in
ASCII, but {\bf $\backslash$135} does not terminate a bracket expression. Beware, however,
that some applications (e.g., C compilers) interpret  such sequences themselves
before the regular-expression package gets to see them, which may require
doubling (quadrupling, etc.) the `{\bf $\backslash$}'. 

Class-shorthand escapes (AREs only) provide
shorthands for certain commonly-used character classes:

\begin{twocollist}\twocolwidtha{4cm}
\twocolitem{{\bf $\backslash$d}}{{\bf $[[:digit:]]$}}
\twocolitem{{\bf $\backslash$s}}{{\bf $[[:space:]]$}}
\twocolitem{{\bf $\backslash$w}}{{\bf $[[:alnum:]\_]$} (note underscore)}
\twocolitem{{\bf $\backslash$D}}{{\bf $[^[:digit:]]$}}
\twocolitem{{\bf $\backslash$S}}{{\bf $[^[:space:]]$}}
\twocolitem{{\bf $\backslash$W}}{{\bf $[^[:alnum:]\_]$} (note underscore)}
\end{twocollist}

Within bracket expressions, `{\bf $\backslash$d}', `{\bf $\backslash$s}', and
`{\bf $\backslash$w}' lose their outer brackets, and `{\bf $\backslash$D}',
`{\bf $\backslash$S}', and `{\bf $\backslash$W}' are illegal. (So, for example,
 {\bf $[$a-c$\backslash$d$]$} is equivalent to {\bf $[a-c[:digit:]]$}.
Also, {\bf $[$a-c$\backslash$D$]$}, which is equivalent to
 {\bf $[a-c^[:digit:]]$}, is illegal.) 

A constraint escape (AREs only) is a constraint,
matching the empty string if specific conditions are met, written as an
escape:

\begin{twocollist}\twocolwidtha{4cm}
\twocolitem{{\bf $\backslash$A}}{matches only at the beginning of the string
(see \helpref{Matching}{wxresynmatching}, below,
for how this differs from `{\bf \caret}')}
\twocolitem{{\bf $\backslash$m}}{matches only at the beginning of a word}
\twocolitem{{\bf $\backslash$M}}{matches only at the end of a word}
\twocolitem{{\bf $\backslash$y}}{matches only at the beginning or end of a word}
\twocolitem{{\bf $\backslash$Y}}{matches only at a point that is not the beginning or end of
a word}
\twocolitem{{\bf $\backslash$Z}}{matches only at the end of the string
(see \helpref{Matching}{wxresynmatching}, below, for
how this differs from `{\bf \$}')}
\twocolitem{{\bf $\backslash${\it m}}}{(where {\it m} is a nonzero digit) a {\it back reference},
see below}
\twocolitem{{\bf $\backslash${\it mnn}}}{(where {\it m} is a nonzero digit, and {\it nn} is some more digits,
and the decimal value {\it mnn} is not greater than the number of closing capturing
parentheses seen so far) a {\it back reference}, see below}
\end{twocollist}

A word is defined
as in the specification of {\bf $[[:$<$:]]$} and {\bf $[[:$>$:]]$} above. Constraint escapes are
illegal within bracket expressions. 

A back reference (AREs only) matches
the same string matched by the parenthesized subexpression specified by
the number, so that (e.g.) {\bf ($[bc]$)$\backslash$1} matches {\bf bb} or {\bf cc} but not `{\bf bc}'.
The subexpression
must entirely precede the back reference in the RE. Subexpressions are numbered
in the order of their leading parentheses. Non-capturing parentheses do not
define subexpressions. 

There is an inherent historical ambiguity between
octal character-entry  escapes and back references, which is resolved by
heuristics, as hinted at above. A leading zero always indicates an octal
escape. A single non-zero digit, not followed by another digit, is always
taken as a back reference. A multi-digit sequence not starting with a zero
is taken as a back  reference if it comes after a suitable subexpression
(i.e. the number is in the legal range for a back reference), and otherwise
is taken as octal. 

\subsection{Metasyntax}\label{remetasyntax}

\helpref{Syntax of the builtin regular expression library}{wxresyn}

In addition to the main syntax described above,
there are some special forms and miscellaneous syntactic facilities available.

Normally the flavor of RE being used is specified by application-dependent
means. However, this can be overridden by a {\it director}. If an RE of any flavor
begins with `{\bf ***:}', the rest of the RE is an ARE. If an RE of any flavor begins
with `{\bf ***=}', the rest of the RE is taken to be a literal string, with all
characters considered ordinary characters. 

An ARE may begin with {\it embedded options}: a sequence {\bf (?xyz)}
(where {\it xyz} is one or more alphabetic characters)
specifies options affecting the rest of the RE. These supplement, and can
override, any options specified by the application. The available option
letters are:

\begin{twocollist}\twocolwidtha{4cm}
\twocolitem{{\bf b}}{rest of RE is a BRE}
\twocolitem{{\bf c}}{case-sensitive matching (usual default)}
\twocolitem{{\bf e}}{rest of RE is an ERE}
\twocolitem{{\bf i}}{case-insensitive matching (see \helpref{Matching}{wxresynmatching}, below)}
\twocolitem{{\bf m}}{historical synonym for {\bf n}}
\twocolitem{{\bf n}}{newline-sensitive matching (see \helpref{Matching}{wxresynmatching}, below)}
\twocolitem{{\bf p}}{partial newline-sensitive matching (see \helpref{Matching}{wxresynmatching}, below)}
\twocolitem{{\bf q}}{rest of RE
is a literal (``quoted'') string, all ordinary characters}
\twocolitem{{\bf s}}{non-newline-sensitive matching (usual default)}
\twocolitem{{\bf t}}{tight syntax (usual default; see below)}
\twocolitem{{\bf w}}{inverse
partial newline-sensitive (``weird'') matching (see \helpref{Matching}{wxresynmatching}, below)}
\twocolitem{{\bf x}}{expanded syntax (see below)}
\end{twocollist}

Embedded options take effect at the {\bf )} terminating the
sequence. They are available only at the start of an ARE, and may not be
used later within it. 

In addition to the usual ({\it tight}) RE syntax, in which
all characters are significant, there is an {\it expanded} syntax, available
%in all flavors of RE with the {\bf -expanded} switch, or
in AREs with the embedded
x option. In the expanded syntax, white-space characters are ignored and
all characters between a {\bf \#} and the following newline (or the end of the
RE) are ignored, permitting paragraphing and commenting a complex RE. There
are three exceptions to that basic rule:
{\itemize
\item%
a white-space character or `{\bf \#}' preceded
by `{\bf $\backslash$}' is retained 
\item%
white space or `{\bf \#}' within a bracket expression is retained
\item%
white space and comments are illegal within multi-character symbols like
the ARE `{\bf (?:}' or the BRE `{\bf $\backslash$(}' 
}
Expanded-syntax white-space characters are blank,
tab, newline, and any character that belongs to the {\it space} character class.

Finally, in an ARE, outside bracket expressions, the sequence `{\bf (?\#ttt)}' (where
 {\it ttt} is any text not containing a `{\bf )}') is a comment, completely ignored. Again,
this is not allowed between the characters of multi-character symbols like
`{\bf (?:}'. Such comments are more a historical artifact than a useful facility,
and their use is deprecated; use the expanded syntax instead. 

{\it None} of these
metasyntax extensions is available if the application (or an initial {\bf ***=}
director) has specified that the user's input be treated as a literal string
rather than as an RE. 

\subsection{Matching}\label{wxresynmatching}

\helpref{Syntax of the builtin regular expression library}{wxresyn}

In the event that an RE could match more than
one substring of a given string, the RE matches the one starting earliest
in the string. If the RE could match more than one substring starting at
that point, its choice is determined by its {\it preference}: either the longest
substring, or the shortest. 

Most atoms, and all constraints, have no preference.
A parenthesized RE has the same preference (possibly none) as the RE. A
quantified atom with quantifier {\bf \{m\}} or {\bf \{m\}?} has the same preference (possibly
none) as the atom itself. A quantified atom with other normal quantifiers
(including {\bf \{m,n\}} with {\it m} equal to {\it n}) prefers longest match. A quantified
atom with other non-greedy quantifiers (including {\bf \{m,n\}?} with {\it m} equal to
 {\it n}) prefers shortest match. A branch has the same preference as the first
quantified atom in it which has a preference. An RE consisting of two or
more branches connected by the {\bf $|$} operator prefers longest match. 

Subject to the constraints imposed by the rules for matching the whole RE, subexpressions
also match the longest or shortest possible substrings, based on their
preferences, with subexpressions starting earlier in the RE taking priority
over ones starting later. Note that outer subexpressions thus take priority
over their component subexpressions. 

Note that the quantifiers {\bf \{1,1\}} and
 {\bf \{1,1\}?} can be used to force longest and shortest preference, respectively,
on a subexpression or a whole RE. 

Match lengths are measured in characters,
not collating elements. An empty string is considered longer than no match
at all. For example, {\bf bb*} matches the three middle characters
of `{\bf abbbc}', {\bf (week$|$wee)(night$|$knights)}
matches all ten characters of `{\bf weeknights}', when {\bf (.*).*} is matched against
 {\bf abc} the parenthesized subexpression matches all three characters, and when
 {\bf (a*)*} is matched against {\bf bc} both the whole RE and the parenthesized subexpression
match an empty string. 

If case-independent matching is specified, the effect
is much as if all case distinctions had vanished from the alphabet. When
an alphabetic that exists in multiple cases appears as an ordinary character
outside a bracket expression, it is effectively transformed into a bracket
expression containing both cases, so that {\bf x} becomes `{\bf $[xX]$}'. When it appears
inside a bracket expression, all case counterparts of it are added to the
bracket expression, so that {\bf $[x]$} becomes {\bf $[xX]$} and {\bf $[^x]$} becomes `{\bf $[^xX]$}'. 

If newline-sensitive
matching is specified, {\bf .} and bracket expressions using {\bf \caret} will never match
the newline character (so that matches will never cross newlines unless
the RE explicitly arranges it) and {\bf \caret} and {\bf \$} will match the empty string after
and before a newline respectively, in addition to matching at beginning
and end of string respectively. ARE {\bf $\backslash$A} and {\bf $\backslash$Z} continue to match beginning
or end of string {\it only}. 

If partial newline-sensitive matching is specified,
this affects {\bf .} and bracket expressions as with newline-sensitive matching,
but not {\bf \caret} and `{\bf \$}'. 

If inverse partial newline-sensitive matching is specified,
this affects {\bf \caret} and {\bf \$} as with newline-sensitive matching, but not {\bf .} and bracket
expressions. This isn't very useful but is provided for symmetry. 

\subsection{Limits And Compatibility}\label{relimits}

\helpref{Syntax of the builtin regular expression library}{wxresyn}

No particular limit is imposed on the length of REs. Programs
intended to be highly portable should not employ REs longer than 256 bytes,
as a POSIX-compliant implementation can refuse to accept such REs. 

The only
feature of AREs that is actually incompatible with POSIX EREs is that {\bf $\backslash$}
does not lose its special significance inside bracket expressions. All other
ARE features use syntax which is illegal or has undefined or unspecified
effects in POSIX EREs; the {\bf ***} syntax of directors likewise is outside
the POSIX syntax for both BREs and EREs. 

Many of the ARE extensions are
borrowed from Perl, but some have been changed to clean them up, and a
few Perl extensions are not present. Incompatibilities of note include `{\bf $\backslash$b}',
`{\bf $\backslash$B}', the lack of special treatment for a trailing newline, the addition of
complemented bracket expressions to the things affected by newline-sensitive
matching, the restrictions on parentheses and back references in lookahead
constraints, and the longest/shortest-match (rather than first-match) matching
semantics. 

The matching rules for REs containing both normal and non-greedy
quantifiers have changed since early beta-test versions of this package.
(The new rules are much simpler and cleaner, but don't work as hard at guessing
the user's real intentions.) 

Henry Spencer's original 1986 {\it regexp} package, still in widespread use,
%(e.g., in pre-8.1 releases of Tcl),
implemented an early version of today's EREs. There are four incompatibilities between {\it regexp}'s
near-EREs (`RREs' for short) and AREs. In roughly increasing order of significance:
{\itemize
\item In AREs, {\bf $\backslash$} followed by an alphanumeric character is either an escape or
an error, while in RREs, it was just another way of writing the  alphanumeric.
This should not be a problem because there was no reason to write such
a sequence in RREs. 

\item {\bf \{} followed by a digit in an ARE is the beginning of
a bound, while in RREs, {\bf \{} was always an ordinary character. Such sequences
should be rare, and will often result in an error because following characters
will not look like a valid bound. 

\item In AREs, {\bf $\backslash$} remains a special character
within `{\bf $[]$}', so a literal {\bf $\backslash$} within {\bf $[]$} must be
written `{\bf $\backslash\backslash$}'. {\bf $\backslash\backslash$} also gives a literal
 {\bf $\backslash$} within {\bf $[]$} in RREs, but only truly paranoid programmers routinely doubled
the backslash. 

\item AREs report the longest/shortest match for the RE, rather
than the first found in a specified search order. This may affect some RREs
which were written in the expectation that the first match would be reported.
(The careful crafting of RREs to optimize the search order for fast matching
is obsolete (AREs examine all possible matches in parallel, and their performance
is largely insensitive to their complexity) but cases where the search
order was exploited to deliberately  find a match which was {\it not} the longest/shortest
will need rewriting.)  
}

\subsection{Basic Regular Expressions}\label{wxresynbre}

\helpref{Syntax of the builtin regular expression library}{wxresyn}

BREs differ from EREs in
several respects.  `{\bf $|$}', `{\bf +}', and {\bf ?} are ordinary characters and there is no equivalent
for their functionality. The delimiters for bounds
are {\bf $\backslash$\{} and `{\bf $\backslash$\}}', with {\bf \{} and
 {\bf \}} by themselves ordinary characters. The parentheses for nested subexpressions
are {\bf $\backslash$(} and `{\bf $\backslash$)}', with {\bf (} and {\bf )} by themselves
ordinary characters. {\bf \caret} is an ordinary
character except at the beginning of the RE or the beginning of a parenthesized
subexpression, {\bf \$} is an ordinary character except at the end of the RE or
the end of a parenthesized subexpression, and {\bf *} is an ordinary character
if it appears at the beginning of the RE or the beginning of a parenthesized
subexpression (after a possible leading `{\bf \caret}'). Finally, single-digit back references
are available, and {\bf $\backslash<$} and {\bf $\backslash>$} are synonyms
for {\bf $[[:<:]]$} and {\bf $[[:>:]]$} respectively;
no other escapes are available.  

\subsection{Regular Expression Character Names}\label{wxresynchars}

\helpref{Syntax of the builtin regular expression library}{wxresyn}

Note that the character names are case sensitive.

\begin{twocollist}
\twocolitem{NUL}{'$\backslash$0'}
\twocolitem{SOH}{'$\backslash$001'}
\twocolitem{STX}{'$\backslash$002'}
\twocolitem{ETX}{'$\backslash$003'}
\twocolitem{EOT}{'$\backslash$004'}
\twocolitem{ENQ}{'$\backslash$005'}
\twocolitem{ACK}{'$\backslash$006'}
\twocolitem{BEL}{'$\backslash$007'}
\twocolitem{alert}{'$\backslash$007'}
\twocolitem{BS}{'$\backslash$010'}
\twocolitem{backspace}{'$\backslash$b'}
\twocolitem{HT}{'$\backslash$011'}
\twocolitem{tab}{'$\backslash$t'}
\twocolitem{LF}{'$\backslash$012'}
\twocolitem{newline}{'$\backslash$n'}
\twocolitem{VT}{'$\backslash$013'}
\twocolitem{vertical-tab}{'$\backslash$v'}
\twocolitem{FF}{'$\backslash$014'}
\twocolitem{form-feed}{'$\backslash$f'}
\twocolitem{CR}{'$\backslash$015'}
\twocolitem{carriage-return}{'$\backslash$r'}
\twocolitem{SO}{'$\backslash$016'}
\twocolitem{SI}{'$\backslash$017'}
\twocolitem{DLE}{'$\backslash$020'}
\twocolitem{DC1}{'$\backslash$021'}
\twocolitem{DC2}{'$\backslash$022'}
\twocolitem{DC3}{'$\backslash$023'}
\twocolitem{DC4}{'$\backslash$024'}
\twocolitem{NAK}{'$\backslash$025'}
\twocolitem{SYN}{'$\backslash$026'}
\twocolitem{ETB}{'$\backslash$027'}
\twocolitem{CAN}{'$\backslash$030'}
\twocolitem{EM}{'$\backslash$031'}
\twocolitem{SUB}{'$\backslash$032'}
\twocolitem{ESC}{'$\backslash$033'}
\twocolitem{IS4}{'$\backslash$034'}
\twocolitem{FS}{'$\backslash$034'}
\twocolitem{IS3}{'$\backslash$035'}
\twocolitem{GS}{'$\backslash$035'}
\twocolitem{IS2}{'$\backslash$036'}
\twocolitem{RS}{'$\backslash$036'}
\twocolitem{IS1}{'$\backslash$037'}
\twocolitem{US}{'$\backslash$037'}
\twocolitem{space}{' '}
\twocolitem{exclamation-mark}{'!'}
\twocolitem{quotation-mark}{'"'}
\twocolitem{number-sign}{'\#'}
\twocolitem{dollar-sign}{'\$'}
\twocolitem{percent-sign}{'\%'}
\twocolitem{ampersand}{'\&'}
\twocolitem{apostrophe}{'$\backslash$''}
\twocolitem{left-parenthesis}{'('}
\twocolitem{right-parenthesis}{')'}
\twocolitem{asterisk}{'*'}
\twocolitem{plus-sign}{'+'}
\twocolitem{comma}{','}
\twocolitem{hyphen}{'-'}
\twocolitem{hyphen-minus}{'-'}
\twocolitem{period}{'.'}
\twocolitem{full-stop}{'.'}
\twocolitem{slash}{'/'}
\twocolitem{solidus}{'/'}
\twocolitem{zero}{'0'}
\twocolitem{one}{'1'}
\twocolitem{two}{'2'}
\twocolitem{three}{'3'}
\twocolitem{four}{'4'}
\twocolitem{five}{'5'}
\twocolitem{six}{'6'}
\twocolitem{seven}{'7'}
\twocolitem{eight}{'8'}
\twocolitem{nine}{'9'}
\twocolitem{colon}{':'}
\twocolitem{semicolon}{';'}
\twocolitem{less-than-sign}{'<'}
\twocolitem{equals-sign}{'='}
\twocolitem{greater-than-sign}{'>'}
\twocolitem{question-mark}{'?'}
\twocolitem{commercial-at}{'@'}
\twocolitem{left-square-bracket}{'$[$'}
\twocolitem{backslash}{'$\backslash$'}
\twocolitem{reverse-solidus}{'$\backslash$'}
\twocolitem{right-square-bracket}{'$]$'}
\twocolitem{circumflex}{'\caret'}
\twocolitem{circumflex-accent}{'\caret'}
\twocolitem{underscore}{'\_'}
\twocolitem{low-line}{'\_'}
\twocolitem{grave-accent}{'`'}
\twocolitem{left-brace}{'\{'}
\twocolitem{left-curly-bracket}{'\{'}
\twocolitem{vertical-line}{'$|$'}
\twocolitem{right-brace}{'\}'}
\twocolitem{right-curly-bracket}{'\}'}
\twocolitem{tilde}{'\destruct{}'}
\twocolitem{DEL}{'$\backslash$177'}
\end{twocollist}

