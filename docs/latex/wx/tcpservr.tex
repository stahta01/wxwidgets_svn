\section{\class{wxTCPServer}}\label{wxtcpserver}

A wxTCPServer object represents the server part of a client-server conversation.
It emulates a DDE-style protocol, but uses TCP/IP which is available on most platforms.

A DDE-based implementation for Windows is available using \helpref{wxDDEServer}{wxddeserver}.

\wxheading{Derived from}

wxServerBase\\
\helpref{wxObject}{wxobject}

\wxheading{Include files}

<wx/sckipc.h>

\wxheading{See also}

\helpref{wxTCPClient}{wxtcpclient}, \helpref{wxTCPConnection}{wxtcpconnection}, \helpref{IPC overview}{ipcoverview}

\latexignore{\rtfignore{\wxheading{Members}}}

\membersection{wxTCPServer::wxTCPServer}\label{wxtcpserverctor}

\func{}{wxTCPServer}{\void}

Constructs a server object.

\membersection{wxTCPServer::Create}\label{wxtcpservercreate}

\func{bool}{Create}{\param{const wxString\& }{service}}

Registers the server using the given service name. Under Unix, the
string must contain an integer id which is used as an Internet port
number. false is returned if the call failed (for example, the port
number is already in use).

\membersection{wxTCPServer::OnAcceptConnection}\label{wxtcpserveronacceptconnection}

\func{virtual wxConnectionBase *}{OnAcceptConnection}{\param{const wxString\& }{topic}}

When a client calls {\bf MakeConnection}, the server receives the
message and this member is called. The application should derive a
member to intercept this message and return a connection object of
either the standard wxTCPConnection type, or of a user-derived type. If the
topic is ``STDIO'', the application may wish to refuse the connection.
Under Unix, when a server is created the OnAcceptConnection message is
always sent for standard input and output.

