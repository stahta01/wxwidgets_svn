%%%%%%%%%%%%%%%%%%%%%%%%%%%%%%%%%%%%%%%%%%%%%%%%%%%%%%%%%%%%%%%%%%%%%%%%%%%%%%%
%% Name:        arc.tex
%% Purpose:     Overview of the archive classes
%% Author:      M.J.Wetherell
%% RCS-ID:      $Id$
%% Copyright:   2004 M.J.Wetherell
%% License:     wxWindows license
%%%%%%%%%%%%%%%%%%%%%%%%%%%%%%%%%%%%%%%%%%%%%%%%%%%%%%%%%%%%%%%%%%%%%%%%%%%%%%%

\section{Archive formats such as zip}\label{wxarc}

The archive classes handle archive formats such as zip, tar, rar and cab.
Currently \helpref{wxZip}{wxzipinputstream}
and \helpref{wxTar}{wxtarinputstream} classes are included.

For each archive type, there are the following classes (using zip here
as an example):

\begin{twocollist}\twocolwidtha{4cm}
\twocolitem{\helpref{wxZipInputStream}{wxzipinputstream}}{Input stream}
\twocolitem{\helpref{wxZipOutputStream}{wxzipoutputstream}}{Output stream}
\twocolitem{\helpref{wxZipEntry}{wxzipentry}}{Holds the meta-data for an
entry (e.g. filename, timestamp, etc.)}
\end{twocollist}

There are also abstract wxArchive classes that can be used to write code
that can handle any of the archive types,
see '\helpref{Generic archive programming}{wxarcgeneric}'.
Also see \helpref{wxFileSystem}{fs} for a higher level interface that
can handle archive files in a generic way.

The classes are designed to handle archives on both seekable streams such
as disk files, or non-seekable streams such as pipes and sockets
(see '\helpref{Archives on non-seekable streams}{wxarcnoseek}').

\wxheading{See also}

\helpref{wxFileSystem}{fs}


\subsection{Creating an archive}\label{wxarccreate}

\helpref{Archive formats such as zip}{wxarc}

Call \helpref{PutNextEntry()}{wxarchiveoutputstreamputnextentry} to
create each new entry in the archive, then write the entry's data.
Another call to PutNextEntry() closes the current entry and begins the next.

For example:

\begin{verbatim}
    wxFFileOutputStream out(_T("test.zip"));
    wxZipOutputStream zip(out);
    wxTextOutputStream txt(zip);
    wxString sep(wxFileName::GetPathSeparator());

    zip.PutNextEntry(_T("entry1.txt"));
    txt << _T("Some text for entry1.txt\n");

    zip.PutNextEntry(_T("subdir") + sep + _T("entry2.txt"));
    txt << _T("Some text for subdir/entry2.txt\n");

\end{verbatim}

The name of each entry can be a full path, which makes it possible to
store entries in subdirectories.


\subsection{Extracting an archive}\label{wxarcextract}

\helpref{Archive formats such as zip}{wxarc}

\helpref{GetNextEntry()}{wxarchiveinputstreamgetnextentry} returns a pointer
to entry object containing the meta-data for the next entry in the archive
(and gives away ownership). Reading from the input stream then returns the
entry's data. Eof() becomes true after an attempt has been made to read past
the end of the entry's data.

When there are no more entries, GetNextEntry() returns NULL and sets Eof().

\begin{verbatim}
    auto_ptr<wxZipEntry> entry;

    wxFFileInputStream in(_T("test.zip"));
    wxZipInputStream zip(in);

    while (entry.reset(zip.GetNextEntry()), entry.get() != NULL)
    {
        // access meta-data
        wxString name = entry->GetName();
        // read 'zip' to access the entry's data
    }

\end{verbatim}


\subsection{Modifying an archive}\label{wxarcmodify}

\helpref{Archive formats such as zip}{wxarc}

To modify an existing archive, write a new copy of the archive to a new file,
making any necessary changes along the way and transferring any unchanged
entries using \helpref{CopyEntry()}{wxarchiveoutputstreamcopyentry}.
For archive types which compress entry data, CopyEntry() is likely to be
much more efficient than transferring the data using Read() and Write()
since it will copy them without decompressing and recompressing them.

In general modifications are not possible without rewriting the archive,
though it may be possible in some limited cases. Even then, rewriting the
archive is usually a better choice since a failure can be handled without
losing the whole
archive. \helpref{wxTempFileOutputStream}{wxtempfileoutputstream} can
be helpful to do this.

For example to delete all entries matching the pattern "*.txt":

\begin{verbatim}
    auto_ptr<wxFFileInputStream> in(new wxFFileInputStream(_T("test.zip")));
    wxTempFileOutputStream out(_T("test.zip"));

    wxZipInputStream inzip(*in);
    wxZipOutputStream outzip(out);

    auto_ptr<wxZipEntry> entry;

    // transfer any meta-data for the archive as a whole (the zip comment
    // in the case of zip)
    outzip.CopyArchiveMetaData(inzip);

    // call CopyEntry for each entry except those matching the pattern
    while (entry.reset(inzip.GetNextEntry()), entry.get() != NULL)
        if (!entry->GetName().Matches(_T("*.txt")))
            if (!outzip.CopyEntry(entry.release(), inzip))
                break;

    // close the input stream by releasing the pointer to it, do this
    // before closing the output stream so that the file can be replaced
    in.reset();

    // you can check for success as follows
    bool success = inzip.Eof() && outzip.Close() && out.Commit();

\end{verbatim}


\subsection{Looking up an archive entry by name}\label{wxarcbyname}

\helpref{Archive formats such as zip}{wxarc}

Also see \helpref{wxFileSystem}{fs} for a higher level interface that is
more convenient for accessing archive entries by name.

To open just one entry in an archive, the most efficient way is
to simply search for it linearly by calling
 \helpref{GetNextEntry()}{wxarchiveinputstreamgetnextentry} until the
required entry is found. This works both for archives on seekable and
non-seekable streams.

The format of filenames in the archive is likely to be different
from the local filename format. For example zips and tars use
unix style names, with forward slashes as the path separator,
and absolute paths are not allowed. So if on Windows the file
"C:$\backslash$MYDIR$\backslash$MYFILE.TXT" is stored, then when reading
the entry back \helpref{GetName()}{wxarchiveentryname} will return
"MYDIR$\backslash$MYFILE.TXT". The conversion into the internal format
and back has lost some information.

So to avoid ambiguity when searching for an entry matching a local name,
it is better to convert the local name to the archive's internal format
and search for that:

\begin{verbatim}
    auto_ptr<wxZipEntry> entry;

    // convert the local name we are looking for into the internal format
    wxString name = wxZipEntry::GetInternalName(localname);

    // open the zip
    wxFFileInputStream in(_T("test.zip"));
    wxZipInputStream zip(in);

    // call GetNextEntry() until the required internal name is found
    do {
        entry.reset(zip.GetNextEntry());
    }
    while (entry.get() != NULL && entry->GetInternalName() != name);

    if (entry.get() != NULL) {
        // read the entry's data...
    }

\end{verbatim}

To access several entries randomly, it is most efficient to transfer the
entire catalogue of entries to a container such as a std::map or a
 \helpref{wxHashMap}{wxhashmap} then entries looked up by name can be
opened using the \helpref{OpenEntry()}{wxarchiveinputstreamopenentry} method.

\begin{verbatim}
    WX_DECLARE_STRING_HASH_MAP(wxZipEntry*, ZipCatalog);
    ZipCatalog::iterator it;
    wxZipEntry *entry;
    ZipCatalog cat;

    // open the zip
    wxFFileInputStream in(_T("test.zip"));
    wxZipInputStream zip(in);

    // load the zip catalog
    while ((entry = zip.GetNextEntry()) != NULL) {
        wxZipEntry*& current = cat[entry->GetInternalName()];
        // some archive formats can have multiple entries with the same name
        // (e.g. tar) though it is an error in the case of zip
        delete current;
        current = entry;
    }

    // open an entry by name
    if ((it = cat.find(wxZipEntry::GetInternalName(localname))) != cat.end()) {
        zip.OpenEntry(*it->second);
        // ... now read entry's data
    }

\end{verbatim}

To open more than one entry simultaneously you need more than one
underlying stream on the same archive:

\begin{verbatim}
    // opening another entry without closing the first requires another
    // input stream for the same file
    wxFFileInputStream in2(_T("test.zip"));
    wxZipInputStream zip2(in2);
    if ((it = cat.find(wxZipEntry::GetInternalName(local2))) != cat.end())
        zip2.OpenEntry(*it->second);

\end{verbatim}


\subsection{Generic archive programming}\label{wxarcgeneric}

\helpref{Archive formats such as zip}{wxarc}

Also see \helpref{wxFileSystem}{fs} for a higher level interface that
can handle archive files in a generic way.

The specific archive classes, such as the wxZip classes, inherit from
the following abstract classes which can be used to write code that can
handle any of the archive types:

\begin{twocollist}\twocolwidtha{5cm}
\twocolitem{\helpref{wxArchiveInputStream}{wxarchiveinputstream}}{Input stream}
\twocolitem{\helpref{wxArchiveOutputStream}{wxarchiveoutputstream}}{Output stream}
\twocolitem{\helpref{wxArchiveEntry}{wxarchiveentry}}{Holds the meta-data for an
entry (e.g. filename)}
\end{twocollist}

In order to able to write generic code it's necessary to be able to create
instances of the classes without knowing which archive type is being used.
To allow this there is a class factory for each archive type, derived from
 \helpref{wxArchiveClassFactory}{wxarchiveclassfactory}, that can create
the other classes.

For example, given {\it wxArchiveClassFactory* factory}, streams and
entries can be created like this:

\begin{verbatim}
    // create streams without knowing their type
    auto_ptr<wxArchiveInputStream> inarc(factory->NewStream(in));
    auto_ptr<wxArchiveOutputStream> outarc(factory->NewStream(out));

    // create an empty entry object
    auto_ptr<wxArchiveEntry> entry(factory->NewEntry());

\end{verbatim}

For the factory itself, the static member
 \helpref{wxArchiveClassFactory::Find()}{wxarchiveclassfactoryfind}.
can be used to find a class factory that can handle a given file
extension or mime type. For example, given {\it filename}:

\begin{verbatim}
    const wxArchiveClassFactory *factory;
    factory = wxArchiveClassFactory::Find(filename, wxSTREAM_FILEEXT);

    if (factory)
        stream = factory->NewStream(new wxFFileInputStream(filename));

\end{verbatim}

{\it Find} does not give away ownership of the returned pointer, so it
does not need to be deleted.

There are similar class factories for the filter streams that handle the
compression and decompression of a single stream, such as wxGzipInputStream.
These can be found using
 \helpref{wxFilterClassFactory::Find()}{wxfilterclassfactoryfind}.

For example, to list the contents of archive {\it filename}:

\begin{verbatim}
    auto_ptr<wxInputStream> in(new wxFFileInputStream(filename));

    if (in->IsOk())
    {
        // look for a filter handler, e.g. for '.gz'
        const wxFilterClassFactory *fcf;
        fcf = wxFilterClassFactory::Find(filename, wxSTREAM_FILEEXT);
        if (fcf) {
            in.reset(fcf->NewStream(in.release()));
            // pop the extension, so if it was '.tar.gz' it is now just '.tar'
            filename = fcf->PopExtension(filename);
        }

        // look for a archive handler, e.g. for '.zip' or '.tar'
        const wxArchiveClassFactory *acf;
        acf = wxArchiveClassFactory::Find(filename, wxSTREAM_FILEEXT);
        if (acf) {
            auto_ptr<wxArchiveInputStream> arc(acf->NewStream(in.release()));
            auto_ptr<wxArchiveEntry> entry;

            // list the contents of the archive
            while ((entry.reset(arc->GetNextEntry())), entry.get() != NULL)
                std::wcout << entry->GetName().c_str() << "\n";
        }
        else {
            wxLogError(_T("can't handle '%s'"), filename.c_str());
        }
    }

\end{verbatim}


\subsection{Archives on non-seekable streams}\label{wxarcnoseek}

\helpref{Archive formats such as zip}{wxarc}

In general, handling archives on non-seekable streams is done in the same
way as for seekable streams, with a few caveats.

The main limitation is that accessing entries randomly using
 \helpref{OpenEntry()}{wxarchiveinputstreamopenentry} 
is not possible, the entries can only be accessed sequentially in the order 
they are stored within the archive.

For each archive type, there will also be other limitations which will
depend on the order the entries' meta-data is stored within the archive.
These are not too difficult to deal with, and are outlined below.

\wxheading{PutNextEntry and the entry size}

When writing archives, some archive formats store the entry size before
the entry's data (tar has this limitation, zip doesn't). In this case
the entry's size must be passed to
 \helpref{PutNextEntry()}{wxarchiveoutputstreamputnextentry} or an error
occurs.

This is only an issue on non-seekable streams, since otherwise the archive
output stream can seek back and fix up the header once the size of the
entry is known.

For generic programming, one way to handle this is to supply the size
whenever it is known, and rely on the error message from the output
stream when the operation is not supported.

\wxheading{GetNextEntry and the weak reference mechanism}

Some archive formats do not store all an entry's meta-data before the
entry's data (zip is an example). In this case, when reading from a
non-seekable stream, \helpref{GetNextEntry()}{wxarchiveinputstreamgetnextentry} 
can only return a partially populated \helpref{wxArchiveEntry}{wxarchiveentry}
object - not all the fields are set.

The input stream then keeps a weak reference to the entry object and
updates it when more meta-data becomes available. A weak reference being
one that does not prevent you from deleting the wxArchiveEntry object - the
input stream only attempts to update it if it is still around.

The documentation for each archive entry type gives the details
of what meta-data becomes available and when. For generic programming,
when the worst case must be assumed, you can rely on all the fields
of wxArchiveEntry being fully populated when GetNextEntry() returns,
with the the following exceptions:

\begin{twocollist}\twocolwidtha{3cm}
\twocolitem{\helpref{GetSize()}{wxarchiveentrysize}}{Guaranteed to be
available after the entry has been read to \helpref{Eof()}{wxinputstreameof},
or \helpref{CloseEntry()}{wxarchiveinputstreamcloseentry} has been called}
\twocolitem{\helpref{IsReadOnly()}{wxarchiveentryisreadonly}}{Guaranteed to
be available after the end of the archive has been reached, i.e. after
GetNextEntry() returns NULL and Eof() is true}
\end{twocollist}

This mechanism allows \helpref{CopyEntry()}{wxarchiveoutputstreamcopyentry}
to always fully preserve entries' meta-data. No matter what order order
the meta-data occurs within the archive, the input stream will always
have read it before the output stream must write it.

\wxheading{wxArchiveNotifier}

Notifier objects can be used to get a notification whenever an input
stream updates a \helpref{wxArchiveEntry}{wxarchiveentry} object's data
via the weak reference mechanism.

Consider the following code which renames an entry in an archive.
This is the usual way to modify an entry's meta-data, simply set the
required field before writing it with
 \helpref{CopyEntry()}{wxarchiveoutputstreamcopyentry}:

\begin{verbatim}
    auto_ptr<wxArchiveInputStream> arc(factory->NewStream(in));
    auto_ptr<wxArchiveOutputStream> outarc(factory->NewStream(out));
    auto_ptr<wxArchiveEntry> entry;

    outarc->CopyArchiveMetaData(*arc);

    while (entry.reset(arc->GetNextEntry()), entry.get() != NULL) {
        if (entry->GetName() == from)
            entry->SetName(to);
        if (!outarc->CopyEntry(entry.release(), *arc))
            break;
    }

    bool success = arc->Eof() && outarc->Close();

\end{verbatim}

However, for non-seekable streams, this technique cannot be used for
fields such as \helpref{IsReadOnly()}{wxarchiveentryisreadonly},
which are not necessarily set when
 \helpref{GetNextEntry()}{wxarchiveinputstreamgetnextentry} returns. In
this case a \helpref{wxArchiveNotifier}{wxarchivenotifier} can be used:

\begin{verbatim}
class MyNotifier : public wxArchiveNotifier
{
public:
    void OnEntryUpdated(wxArchiveEntry& entry) { entry.SetIsReadOnly(false); }
};

\end{verbatim}

The meta-data changes are done in your notifier's
 \helpref{OnEntryUpdated()}{wxarchivenotifieronentryupdated} method,
then \helpref{SetNotifier()}{wxarchiveentrynotifier} is called before
CopyEntry():

\begin{verbatim}
    auto_ptr<wxArchiveInputStream> arc(factory->NewStream(in));
    auto_ptr<wxArchiveOutputStream> outarc(factory->NewStream(out));
    auto_ptr<wxArchiveEntry> entry;
    MyNotifier notifier;

    outarc->CopyArchiveMetaData(*arc);

    while (entry.reset(arc->GetNextEntry()), entry.get() != NULL) {
        entry->SetNotifier(notifier);
        if (!outarc->CopyEntry(entry.release(), *arc))
            break;
    }

    bool success = arc->Eof() && outarc->Close();

\end{verbatim}

SetNotifier() calls OnEntryUpdated() immediately, then the input
stream calls it again whenever it sets more fields in the entry. Since
OnEntryUpdated() will be called at least once, this technique always
works even when it is not strictly necessary to use it. For example,
changing the entry name can be done this way too and it works on seekable
streams as well as non-seekable.

