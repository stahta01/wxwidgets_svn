\section{\class{wxPaintDC}}\label{wxpaintdc}

A wxPaintDC must be constructed if an application wishes to paint on the
client area of a window from within an {\bf OnPaint} event.
This should normally be constructed as a temporary stack object; don't store
a wxPaintDC object. If you have an OnPaint handler, you {\it must} create a wxPaintDC
object within it even if you don't actually use it.

Using wxPaintDC within OnPaint is important because it automatically
sets the clipping area to the damaged area of the window. Attempts to draw
outside this area do not appear.

To draw on a window from outside {\bf OnPaint}, construct a \helpref{wxClientDC}{wxclientdc} object.

To draw on the whole window including decorations, construct a \helpref{wxWindowDC}{wxwindowdc} object
(Windows only).

\wxheading{Derived from}

\helpref{wxWindowDC}{wxwindowdc}\\
\helpref{wxDC}{wxdc}

\wxheading{Include files}

<wx/dcclient.h>

\wxheading{See also}

\helpref{wxDC}{wxdc}, \helpref{wxMemoryDC}{wxmemorydc}, \helpref{wxPaintDC}{wxpaintdc},\rtfsp
\helpref{wxWindowDC}{wxwindowdc}, \helpref{wxScreenDC}{wxscreendc}

\latexignore{\rtfignore{\wxheading{Members}}}

\membersection{wxPaintDC::wxPaintDC}

\func{}{wxPaintDC}{\param{wxWindow*}{ window}}

Constructor. Pass a pointer to the window on which you wish to paint.



