\chapter{Property sheet classes}\label{proplist}

\section{Introduction}\label{proplistintro}

The Property Sheet Classes help the programmer to specify complex dialogs and
their relationship with their associated data. By specifying data as a
wxPropertySheet containing wxProperty objects, the programmer can use
a range of available or custom wxPropertyView classes to allow the user to
edit this data. Classes derived from wxPropertyView act as mediators between the
wxPropertySheet and the actual window (and associated panel items).

For example, the wxPropertyListView is a kind of wxPropertyView which displays
data in a Visual Basic-style property list (see \helpref{the next section}{proplistappearance} for
screen shots). This is a listbox containing names and values, with
an edit control and other optional controls via which the user edits the selected
data item.

wxPropertyFormView is another kind of wxPropertyView which mediates between
the data and a panel or dialog box which has already been created. This makes it a contender for
the replacement of wxForm, since programmer-controlled layout is going to be much more
satisfactory. If automatic layout is desired, then wxPropertyListView could be used instead.

The main intention of this class library was to provide property {\it list} behaviour, but
it has been generalised as much as possible so that the concept of a property sheet and its viewers
can reduce programming effort in a range of user interface tasks.

For further details on the classes and how they are used, please see \helpref{Property classes overview}{proplistpropertyoverview}.

\subsection{The appearance and behaviour of a property list view}\label{proplistappearance}

The property list, as seen in an increasing number of development tools
such as Visual Basic and Delphi, is a convenient and compact method for
displaying and editing a number of items without the need for one
control per item, and without the need for designing a special form. The
controls are as follows:

\begin{itemize}\itemsep=0pt
\item A listbox showing the properties and their current values, which has double-click
properties dependent on the nature of the current property;
\item a text editing area at the top of the display, allowing the user to edit
the currently selected property if appropriate;
\item `confirm' and `cancel' buttons to confirm or cancel an edit (for the property, not the
whole sheet);
\item an optional list that appears when the user can make a choice from several known possible values;
\item a small Edit button to invoke `detailed editing' (perhaps showing or hiding the above value list, or
maybe invoking a common dialog);
\item optional OK/Close, Cancel and Help buttons for the whole dialog.
\end{itemize}

The concept of `detailed editing' versus quick editing gives the user a choice
of editing mode, so novice and expert behaviour can be catered for, or the user can just
use what he feels comfortable with.

Behaviour alters depending on the kind of property being edited. For example, a boolean value has
the following behaviour:

\begin{itemize}\itemsep=0pt
\item Double-clicking on the item toggles between TRUE and FALSE.
\item Showing the value list enables the user to select TRUE or FALSE.
\item The user may be able to type in the word TRUE or FALSE, or the edit control
may be read-only to disallow this since it is error-prone.
\end{itemize}

A list of strings may pop up a dialog for editing them, a simple string just allows text editing,
double-clicking a colour property may show a colour selector, double-clicking on a filename property may
show a file selector (in addition to being able to type in the name in the edit control), etc.

Note that the `type' of property, such as string or integer, does not
necessarily determine the behaviour of the property. The programmer has
to be able to specify different behaviours for the same type, depending
on the meaning of the property. For example, a colour and a filename may
both be strings, but their editing behaviour should be different. This
is why objects of type wxPropertyValidator need to be used, to define
behaviour for a given class of properties or even specific property
name.  Objects of class wxPropertyView contain a list of property
registries, which enable reuse of bunches of these validators in
different circumstances. Or a wxProperty can be explicitly set to use a
particular validator object. 

The following screen shot of the property classes test program shows the
user editing a string, which is constrained to be one of three possible
values.

\helponly{\image{}{prop1.bmp}}

The second picture shows the user having entered a integer that
was outside the range specified to the validator. Note that in this picture,
the value list is hidden because it is not used when editing an integer.

\helponly{\image{}{prop2.bmp}}

\section{Headers}\label{proplistfiles}

The property class library comprises the following files:

\begin{itemize}\itemsep=0pt
\item prop.h: base property class header
\item proplist.h: wxPropertyListView and associated classes
\item propform.h: wxPropertyListView and associated classes
\end{itemize}

\section{Topic overviews}\label{proplistoverviews}

This chapter contains a selection of topic overviews.

\subsection{Property classes overview}\label{proplistpropertyoverview}

The property classes help a programmer to express relationships between
data and physical windows, in particular:

\begin{itemize}\itemsep=0pt
\item the transfer of data to and from the physical controls;
\item the behaviour of various controls and custom windows for particular
types of data;
\item the validation of data, notifying the user when incorrect data is entered,
or even better, constraining the input so only valid data can be entered.
\end{itemize}

With a consistent framework, the programmer should be able to use existing
components and design new ones in a principled manner, to solve many data entry
requirements.

Each datum is represented in a \helpref{wxProperty}{wxproperty}, which has a name and a value.
Various C++ types are permitted in the value of a property, and the property can store a pointer
to the data instead of a copy of the data. A \helpref{wxPropertySheet}{wxpropertysheet} represents a number of these properties.

These two classes are independent from the way in which the data is visually manipulated. To
mediate between property sheets and windows, the abstract class \helpref{wxPropertyView}{wxpropertyview} is
available for programmers to derive new kinds of view. One kind of view that is available is the \helpref{wxPropertyListView}{wxpropertylistview},
which displays the data in a Visual Basic-style list, with a small number of controls for editing
the currently selected property. Another is \helpref{wxPropertyFormView}{wxpropertyformview} which
mediates between an existing dialog or panel and the property sheet.

The hard work of mediation is actually performed by validators, which are instances of classes
derived from \helpref{wxPropertyValidator}{wxpropertyvalidator}. A validator is associated with
a particular property and is responsible for
responding to user interface events, and displaying, updating and checking the property value.
Because a validator's behaviour depends largely on the kind of view being used, there has to be
a separate hierarchy of validators for each class of view. So for wxPropertyListView, there is
an abstract class \helpref{wxPropertyListValidator}{wxpropertylistvalidator} from which concrete
classes are derived, such as \helpref{wxRealListValidator}{wxreallistvalidator} and
\rtfsp\helpref{wxStringListValidator}{wxstringlistvalidator}.

A validator can be explicitly set for a property, so there is no doubt which validator
should be used to edit that property. However, it is also possible to define a registry
of validators, and have the validator chosen on the basis of the {\it role} of the property.
So a property with a ``filename" role would match the ``filename" validator, which pops
up a file selector when the user double clicks on the property.

You don't have to define your own frame or window classes: there are some predefined
that will work with the property list view. See \helpref{Window classes}{proplistwindowclasses} for
further details.

\subsubsection{Example 1: Property list view}

The following code fragment shows the essentials of creating a registry of
standard validators, a property sheet containing some properties, and
a property list view and dialog or frame. RegisterValidators will be
called on program start, and PropertySheetTest is called in response to a
menu command.

Note how some properties are created with an explicit reference to
a validator, and others are provided with a ``role'' which can be matched
against a validator in the registry.

The interface generated by this test program is shown in the section \helpref{Appearance and
behaviour of a property list view}{proplistappearance}.

\begin{verbatim}
void RegisterValidators(void)
{
  myListValidatorRegistry.RegisterValidator((wxString)"real", new wxRealListValidator);
  myListValidatorRegistry.RegisterValidator((wxString)"string", new wxStringListValidator);
  myListValidatorRegistry.RegisterValidator((wxString)"integer", new wxIntegerListValidator);
  myListValidatorRegistry.RegisterValidator((wxString)"bool", new wxBoolListValidator);
}

void PropertyListTest(Bool useDialog)
{
  wxPropertySheet *sheet = new wxPropertySheet;

  sheet->AddProperty(new wxProperty("fred", 1.0, "real"));
  sheet->AddProperty(new wxProperty("tough choice", (Bool)TRUE, "bool"));
  sheet->AddProperty(new wxProperty("ian", (long)45, "integer", new wxIntegerListValidator(-50, 50)));
  sheet->AddProperty(new wxProperty("bill", 25.0, "real", new wxRealListValidator(0.0, 100.0)));
  sheet->AddProperty(new wxProperty("julian", "one", "string"));
  sheet->AddProperty(new wxProperty("bitmap", "none", "string", new wxFilenameListValidator("Select a bitmap file", "*.bmp")));
  wxStringList *strings = new wxStringList("one", "two", "three", NULL);
  sheet->AddProperty(new wxProperty("constrained", "one", "string", new wxStringListValidator(strings)));

  wxPropertyListView *view =
    new wxPropertyListView(NULL,
     wxPROP_BUTTON_CHECK_CROSS|wxPROP_DYNAMIC_VALUE_FIELD|wxPROP_PULLDOWN);

  wxDialogBox *propDialog = NULL;
  wxPropertyListFrame *propFrame = NULL;
  if (useDialog)
  {
    propDialog = new wxPropertyListDialog(view, NULL, "Property Sheet Test", TRUE, -1, -1, 400, 500);
  }
  else
  {
    propFrame = new wxPropertyListFrame(view, NULL, "Property Sheet Test", -1, -1, 400, 500);
  }
  
  view->AddRegistry(&myListValidatorRegistry);

  if (useDialog)
  {
    view->ShowView(sheet, propDialog);
    propDialog->Centre(wxBOTH);
    propDialog->Show(TRUE);
  }
  else
  {
    propFrame->Initialize();
    view->ShowView(sheet, propFrame->GetPropertyPanel());
    propFrame->Centre(wxBOTH);
    propFrame->Show(TRUE);
  }
}
\end{verbatim}

\subsubsection{Example 2: Property form view}

This example is similar to Example 1, but uses a property form view to
edit a property sheet using a predefined dialog box.

\begin{verbatim}
void RegisterValidators(void)
{
  myFormValidatorRegistry.RegisterValidator((wxString)"real", new wxRealFormValidator);
  myFormValidatorRegistry.RegisterValidator((wxString)"string", new wxStringFormValidator);
  myFormValidatorRegistry.RegisterValidator((wxString)"integer", new wxIntegerFormValidator);
  myFormValidatorRegistry.RegisterValidator((wxString)"bool", new wxBoolFormValidator);
}

void PropertyFormTest(Bool useDialog)
{
  wxPropertySheet *sheet = new wxPropertySheet;

  sheet->AddProperty(new wxProperty("fred", 25.0, "real", new wxRealFormValidator(0.0, 100.0)));
  sheet->AddProperty(new wxProperty("tough choice", (Bool)TRUE, "bool"));
  sheet->AddProperty(new wxProperty("ian", (long)45, "integer", new wxIntegerFormValidator(-50, 50)));
  sheet->AddProperty(new wxProperty("julian", "one", "string"));
  wxStringList *strings = new wxStringList("one", "two", "three", NULL);
  sheet->AddProperty(new wxProperty("constrained", "one", "string", new wxStringFormValidator(strings)));

  wxPropertyFormView *view = new wxPropertyFormView(NULL);

  wxDialogBox *propDialog = NULL;
  wxPropertyFormFrame *propFrame = NULL;
  if (useDialog)
  {
    propDialog = new wxPropertyFormDialog(view, NULL, "Property Form Test", TRUE, -1, -1, 400, 300);
  }
  else
  {
    propFrame = new wxPropertyFormFrame(view, NULL, "Property Form Test", -1, -1, 400, 300);
    propFrame->Initialize();
  }
  
  wxPanel *panel = propDialog ? propDialog : propFrame->GetPropertyPanel();
  panel->SetLabelPosition(wxVERTICAL);
  
  // Add items to the panel
  
  (void) new wxButton(panel, (wxFunction)NULL, "OK", -1, -1, -1, -1, 0, "ok");
  (void) new wxButton(panel, (wxFunction)NULL, "Cancel", -1, -1, 80, -1, 0, "cancel");
  (void) new wxButton(panel, (wxFunction)NULL, "Update", -1, -1, 80, -1, 0, "update");
  (void) new wxButton(panel, (wxFunction)NULL, "Revert", -1, -1, -1, -1, 0, "revert");
  panel->NewLine();
  
  // The name of this text item matches the "fred" property
  (void) new wxText(panel, (wxFunction)NULL, "Fred", "", -1, -1, 90, -1, 0, "fred");
  (void) new wxCheckBox(panel, (wxFunction)NULL, "Yes or no", -1, -1, -1, -1, 0, "tough choice");
  (void) new wxSlider(panel, (wxFunction)NULL, "Sliding scale", 0, -50, 50, 100, -1, -1, wxHORIZONTAL, "ian");
  panel->NewLine();
  (void) new wxListBox(panel, (wxFunction)NULL, "Constrained", wxSINGLE, -1, -1, 100, 90, 0, NULL, 0, "constrained");

  view->AddRegistry(&myFormValidatorRegistry);

  if (useDialog)
  {
    view->ShowView(sheet, propDialog);
    view->AssociateNames();
    view->TransferToDialog();
    propDialog->Centre(wxBOTH);
    propDialog->Show(TRUE);
  }
  else
  {
    view->ShowView(sheet, propFrame->GetPropertyPanel());
    view->AssociateNames();
    view->TransferToDialog();
    propFrame->Centre(wxBOTH);
    propFrame->Show(TRUE);
  }
}
\end{verbatim}

\subsection{Validator classes overview}\label{proplistvalidatoroverview}

Classes: \helpref{Validator classes}{proplistvalidatorclasses}

The validator classes provide functionality for mediating between a wxProperty and
the actual display. There is a separate family of validator classes for each
class of view, since the differences in user interface for these views implies
that little common functionality can be shared amongst validators.



\subsubsection{wxPropertyValidator overview}\label{wxpropertyvalidatoroverview}

Class: \helpref{wxPropertyValidator}{wxpropertyvalidator}

This class is the root of all property validator classes. It contains a small
amount of common functionality, including functions to convert between
strings and C++ values.

A validator is notionally an object which sits between a property and its displayed
value, and checks that the value the user enters is correct, giving an error message
if the validation fails. In fact, the validator does more than that, and is akin to
a view class but at a finer level of detail. It is also responsible for
loading the dialog box control with the value from the property, putting it back
into the property, preparing special controls for editing the value, and
may even invoke special dialogs for editing the value in a convenient way.

In a property list dialog, there is quite a lot of scope for supplying custom dialogs,
such as file or colour selectors. For a form dialog, there is less scope because
there is no concept of `detailed editing' of a value: one control is associated with
one property, and there is no provision for invoking further dialogs. The reader
may like to work out how the form view could be extended to provide some of the
functionality of the property list!

Validator objects may be associated explicitly with a wxProperty, or they may be
indirectly associated by virtue of a property `kind' that matches validators having
that kind. In the latter case, such validators are stored in a validator registry
which is passed to the view before the dialog is shown. If the validator takes
arguments, such as minimum and maximum values in the case of a wxIntegerListValidator,
then the validator must be associated explicitly with the property. The validator
will be deleted when the property is deleted.

\subsubsection{wxPropertyListValidator overview}\label{wxpropertylistvalidatoroverview}

Class: \helpref{wxPropertyListValidator}{wxpropertylistvalidator}

This class is the abstract base class for property list view validators.
The list view acts upon a user interface containing a list of properties,
a text item for direct property value editing, confirm/cancel buttons for the value,
a pulldown list for making a choice between values, and OK/Cancel/Help buttons
for the dialog (see \helpref{property list appearance}{proplistappearance}).

By overriding virtual functions, the programmer can create custom
behaviour for different kinds of property. Custom behaviour can use just the
available controls on the property list dialog, or the validator can
invoke custom editors with quite different controls, which pop up in
`detailed editing' mode.

See the detailed class documentation for the members you should override
to give your validator appropriate behaviour.

\subsubsection{wxPropertyFormValidator overview}\label{wxpropertyformvalidatoroverview}

This class is the abstract base class for property form view validators.
The form view acts upon an existing dialog box or panel, where either the
panel item names correspond to property names, or the programmer has explicitly
associated the panel item with the property.

By overriding virtual functions, the programmer determines how
values are displayed or retrieved, and the checking that the validator does.

See the detailed class documentation for the members you should override
to give your validator appropriate behaviour.

\subsection{View classes overview}\label{proplistviewoverview}

Classes: \helpref{View classes}{proplistviewclasses}

An instance of a view class relates a property sheet with an actual window.
Currently, there are two classes of view: wxPropertyListView and wxPropertyFormView.

\subsubsection{wxPropertyView overview}\label{wxpropertyviewoverview}

Class: \helpref{wxPropertyView}{wxpropertyview}

This is the abstract base class for property views.

\subsubsection{wxPropertyListView overview}\label{wxpropertylistviewoverview}

Class: \helpref{wxPropertyListView}{wxpropertylistview}

The property list view defines the relationship between a property sheet and
a property list dialog or panel. It manages user interface events such as
clicking on a property, pressing return in the text edit field, and clicking
on Confirm or Cancel. These events cause member functions of the
class to be called, and these in turn may call member functions of
the appropriate validator to be called, to prepare controls, check the property value,
invoke detailed editing, etc.

\subsubsection{wxPropertyFormView overview}\label{wxpropertyformviewoverview}

Class: \helpref{wxPropertyFormView}{wxpropertyformview}

The property form view manages the relationship between a property sheet
and an existing dialog or panel.

You must first create a panel or dialog box for the view to work on.
The panel should contain panel items with names that correspond to
properties in your property sheet; or you can explicitly set the
panel item for each property.

Apart from any custom panel items that you wish to control independently
of the property-editing items, wxPropertyFormView takes over the
processing of item events. It can also control normal dialog behaviour such
as OK, Cancel, so you should also create some standard buttons that the property view
can recognise. Just create the buttons with standard names and the view
will do the rest. The following button names are recognised:

\begin{itemize}\itemsep=0pt
\item {\bf ok}: indicates the OK button. Calls wxPropertyFormView::OnOk. By default,
checks and updates the form values, closes and deletes the frame or dialog, then deletes the view.
\item {\bf cancel}: indicates the Cancel button. Calls wxPropertyFormView::OnCancel. By default,
closes and deletes the frame or dialog, then deletes the view.
\item {\bf help}: indicates the Help button. Calls wxPropertyFormView::OnHelp. This needs
to be overridden by the application for anything interesting to happen.
\item {\bf revert}: indicates the Revert button. Calls wxPropertyFormView::OnRevert,
which by default transfers the wxProperty values to the panel items (in effect
undoing any unsaved changes in the items).
\item {\bf update}: indicates the Revert button. Calls wxPropertyFormView::OnUpdate, which
by defaults transfers the displayed values to the wxProperty objects.
\end{itemize}

\subsection{wxPropertySheet overview}\label{wxpropertysheetoverview}

Classes: \helpref{wxPropertySheet}{wxpropertysheet}, \helpref{wxProperty}{wxproperty}, \helpref{wxPropertyValue}{wxpropertyvalue}

A property sheet defines zero or more properties. This is a bit like an explicit representation of
a C++ object. wxProperty objects can have values which are pointers to C++ values, or they
can allocate their own storage for values.

Because the property sheet representation is explicit and can be manipulated by
a program, it is a convenient form to be used for a variety of
editing purposes. wxPropertyListView and wxPropertyFormView are two classes that
specify the relationship between a property sheet and a user interface. You could imagine
other uses for wxPropertySheet, for example to generate a form-like user interface without
the need for GUI programming. Or for storing the names and values of command-line switches, with the
option to subsequently edit these values using a wxPropertyListView.

A typical use for a property sheet is to represent values of an object
which are only implicit in the current representation of it. For
example, in Visual Basic and similar programming environments, you can
`edit a button', or rather, edit the button's properties.  One of the
properties you can edit is {\it width} - but there is no explicit
representation of width in a wxWidgets button; instead, you call SetSize
and GetSize members. To translate this into a consistent,
property-oriented scheme, we could derive a new class
wxButtonWithProperties, which has two new functions: SetProperty and
GetProperty.  SetProperty accepts a property name and a value, and calls
an appropriate function for the property that is being passed.
GetProperty accepts a property name, returning a property value. So
instead of having to use the usual arbitrary set of C++ member functions
to set or access attributes of a window, programmer deals merely with
SetValue/GetValue, and property names and values.
We now have a single point at which we can modify or query an object by specifying
names and values at run-time. (The implementation of SetProperty and GetProperty
is probably quite messy and involves a large if-then-else statement to
test the property name and act accordingly.)

When the user invokes the property editor for a wxButtonWithProperties, the system
creates a wxPropertySheet with `imaginary' properties such as width, height, font size
and so on. For each property, wxButtonWithProperties::GetProperty is called, and the result is
passed to the corresponding wxProperty. The wxPropertySheet is passed to a wxPropertyListView
as described elsewhere, and the user edits away. When the user has finished editing, the system calls
wxButtonWithProperties::SetProperty to transfer the wxProperty value back into the button
by way of an appropriate call, wxWindow::SetSize in the case of width and height properties.

\section{Classes by category}\label{proplistclassesbycat}

A classification of property sheet classes by category.

\subsection{Data classes}

\begin{itemize}\itemsep=0pt
\item \helpref{wxProperty}{wxproperty}
\item \helpref{wxPropertyValue}{wxpropertyvalue}
\item \helpref{wxPropertySheet}{wxpropertysheet}
\end{itemize}

\subsection{Validator classes}\label{proplistvalidatorclasses}

Validators check that the values the user has entered for a property are
valid. They can also define specific ways of entering data, such as a
file selector for a filename, and they are responsible for transferring
values between the wxProperty and the physical display. 

Base classes:

\begin{itemize}\itemsep=0pt
\item \helpref{wxPropertyValidator}{wxproperty}
\item \helpref{wxPropertyListValidator}{wxpropertylistvalidator}
\item \helpref{wxPropertyFormValidator}{wxpropertyformvalidator}
\end{itemize}

List view validators:

\begin{itemize}\itemsep=0pt
\item \helpref{wxBoolListValidator}{wxboollistvalidator}
\item \helpref{wxFilenameListValidator}{wxfilenamelistvalidator}
\item \helpref{wxIntegerListValidator}{wxintegerlistvalidator}
\item \helpref{wxListOfStringsListValidator}{wxlistofstringslistvalidator}
\item \helpref{wxRealListValidator}{wxreallistvalidator}
\item \helpref{wxStringListValidator}{wxstringlistvalidator}
\end{itemize}

Form view validators:

\begin{itemize}\itemsep=0pt
\item \helpref{wxBoolFormValidator}{wxboolformvalidator}
\item \helpref{wxIntegerFormValidator}{wxintegerformvalidator}
\item \helpref{wxRealFormValidator}{wxrealformvalidator}
\item \helpref{wxStringFormValidator}{wxstringformvalidator}
\end{itemize}

\subsection{View classes}\label{proplistviewclasses}

View classes mediate between a property sheet and a physical window.

\begin{itemize}\itemsep=0pt
\item \helpref{wxPropertyView}{wxpropertyview}
\item \helpref{wxPropertyListView}{wxpropertylistview}
\item \helpref{wxPropertyFormView}{wxpropertyformview}
\end{itemize}

\subsection{Window classes}\label{proplistwindowclasses}

The class library defines some window classes that can be used as-is with a suitable
view class and property sheet.

\begin{itemize}\itemsep=0pt
\item \helpref{wxPropertyFormFrame}{wxpropertyformframe}
\item \helpref{wxPropertyFormDialog}{wxpropertyformdialog}
\item \helpref{wxPropertyFormPanel}{wxpropertyformpanel}
\item \helpref{wxPropertyListFrame}{wxpropertylistframe}
\item \helpref{wxPropertyListDialog}{wxpropertylistdialog}
\item \helpref{wxPropertyListPanel}{wxpropertylistpanel}
\end{itemize}

\subsection{Registry classes}

A validator registry is a list of validators that can be applied to properties in a property sheet.
There may be one or more registries per property view.

\begin{itemize}\itemsep=0pt
\item \helpref{wxPropertyValidatorRegistry}{wxpropertyvalidatorregistry}
\end{itemize}
