\section{\class{wxHelpControllerHelpProvider}}\label{wxhelpcontrollerhelpprovider}

wxHelpControllerHelpProvider is an implementation of wxHelpProvider which supports
both context identifiers and plain text help strings. If the help text is an integer,
it is passed to wxHelpController::DisplayContextPopup. Otherwise, it shows the string
in a tooltip as per wxSimpleHelpProvider. If you use this with a wxCHMHelpController instance
on windows, it will use the native style of tip window instead of \helpref{wxTipWindow}{wxtipwindow}.

You can use the convenience function {\bf wxContextId} to convert an integer context
id to a string for passing to \helpref{wxWindow::SetHelpText}{wxwindowsethelptext}.

\wxheading{Derived from}

\helpref{wxSimpleHelpProvider}{wxsimplehelpprovider}\\
\helpref{wxHelpProvider}{wxhelpprovider}

\wxheading{Include files}

<wx/cshelp.h>

\wxheading{See also}

\helpref{wxHelpProvider}{wxhelpprovider}, \helpref{wxSimpleHelpProvider}{wxsimplehelpprovider}, 
\helpref{wxContextHelp}{wxcontexthelp}, \helpref{wxWindow::SetHelpText}{wxwindowsethelptext}, 
\helpref{wxWindow::GetHelpText}{wxwindowgethelptext}

\latexignore{\rtfignore{\wxheading{Members}}}

\membersection{wxHelpControllerHelpProvider::wxHelpControllerHelpProvider}\label{wxhelpcontrollerhelpproviderwxhelpcontrollerhelpprovider}

\func{}{wxHelpControllerHelpProvider}{\param{wxHelpControllerBase* }{hc = NULL}}

Note that the instance doesn't own the help controller. The help controller
should be deleted separately.

\membersection{wxHelpControllerHelpProvider::SetHelpController}\label{wxhelpcontrollerhelpprovidersethelpcontroller}

\func{void}{SetHelpController}{\param{wxHelpControllerBase* }{hc}}

Sets the help controller associated with this help provider.

\membersection{wxHelpControllerHelpProvider::GetHelpController}\label{wxhelpcontrollerhelpprovidergethelpcontroller}

\constfunc{wxHelpControllerBase*}{GetHelpController}{\void}

Returns the help controller associated with this help provider.

