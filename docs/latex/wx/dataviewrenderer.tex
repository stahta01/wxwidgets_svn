
\section{\class{wxDataViewRenderer}}\label{wxdataviewrenderer}


This class is used by \helpref{wxDataViewCtrl}{wxdataviewctrl} to
render the individual cells. One instance of a renderer class is
owned by \helpref{wxDataViewColumn}{wxdataviewcolumn}. There is
a number of ready-to-use renderers provided:
\helpref{wxDataViewTextRenderer}{wxdataviewtextrenderer},
\helpref{wxDataViewToggleRenderer}{wxdataviewtogglerenderer},
\helpref{wxDataViewProgressRenderer}{wxdataviewprogressrenderer},
\helpref{wxDataViewBitmapRenderer}{wxdataviewbitmaprenderer},
\helpref{wxDataViewDateRenderer}{wxdataviewdaterenderer}.

Additionally, the user can write own renderers by deriving from
\helpref{wxDataViewCustomRenderer}{wxdataviewcustomrenderer}.

The {\it wxDataViewCellMode} flag controls, what actions
the cell data allows. {\it wxDATAVIEW\_CELL\_ACTIVATABLE}
indicates that the user can double click the cell and
something will happen (e.g. a window for editing a date
will pop up). {\it wxDATAVIEW\_CELL\_EDITABLE} indicates
that the user can edit the data in-place, i.e. an control
will show up after a slow click on the cell. This behaviour
is best known from changing the filename in most file 
managers etc.



{\small
\begin{verbatim}
enum wxDataViewCellMode
{
    wxDATAVIEW_CELL_INERT,
    wxDATAVIEW_CELL_ACTIVATABLE,
    wxDATAVIEW_CELL_EDITABLE
};
\end{verbatim}
}

The {\it wxDataViewCellRenderState} flag controls how the
the renderer should display its contents in a cell:

{\small
\begin{verbatim}
enum wxDataViewCellRenderState
{
    wxDATAVIEW_CELL_SELECTED    = 1,
    wxDATAVIEW_CELL_PRELIT      = 2,
    wxDATAVIEW_CELL_INSENSITIVE = 4,
    wxDATAVIEW_CELL_FOCUSED     = 8
};
\end{verbatim}
}


\wxheading{Derived from}

\helpref{wxObject}{wxobject}

\wxheading{Include files}

<wx/dataview.h>

\wxheading{Library}

\helpref{wxAdv}{librarieslist}


\membersection{wxDataViewRenderer::wxDataViewRenderer}\label{wxdataviewrendererwxdataviewrenderer}

\func{}{wxDataViewRenderer}{\param{const wxString\& }{varianttype}, \param{wxDataViewCellMode }{mode = wxDATAVIEW\_CELL\_INERT}}

Constructor.

\membersection{wxDataViewRenderer::GetMode}\label{wxdataviewrenderergetmode}

\func{virtual wxDataViewCellMode}{GetMode}{\void}

Returns the cell mode.

\membersection{wxDataViewRenderer::GetOwner}\label{wxdataviewrenderergetowner}

\func{virtual wxDataViewColumn*}{GetOwner}{\void}

Returns pointer to the owning \helpref{wxDataViewColumn}{wxdataviewcolumn}.

\membersection{wxDataViewRenderer::GetValue}\label{wxdataviewrenderergetvalue}

\func{virtual bool}{GetValue}{\param{wxVariant\& }{value}}

This methods retrieves the value from the renderer in order to
transfer the value back to the data model. Returns {\it false}
on failure.

\membersection{wxDataViewRenderer::GetVariantType}\label{wxdataviewrenderergetvarianttype}

\func{virtual wxString}{GetVariantType}{\void}

Returns a string with the type of the \helpref{wxVariant}{wxvariant}
supported by this renderer.

\membersection{wxDataViewRenderer::SetOwner}\label{wxdataviewrenderersetowner}

\func{virtual void}{SetOwner}{\param{wxDataViewColumn* }{owner}}

Sets the owning \helpref{wxDataViewColumn}{wxdataviewcolumn}. This
is usually called from within wxDataViewColumn.

\membersection{wxDataViewRenderer::SetValue}\label{wxdataviewrenderersetvalue}

\func{virtual bool}{SetValue}{\param{const wxVariant\& }{value}}

Set the value of the renderer (and thus its cell) to {\it value}.
The internal code will then render this cell with this data.


\membersection{wxDataViewRenderer::Validate}\label{wxdataviewrenderervalidate}

\func{virtual bool}{Validate}{\param{wxVariant\& }{value}}

Before data is committed to the data model, it is passed to this
method where it can be checked for validity. This can also be
used for checking a valid range or limiting the user input in
a certain aspect (e.g. max number of characters or only alphanumeric
input, ASCII only etc.). Return {\it false} if the value is
not valid.

Please note that due to implementation limitations, this validation
is done after the editing control already is destroyed and the
editing process finished.


\section{\class{wxDataViewTextRenderer}}\label{wxdataviewtextrenderer}

wxDataViewTextRenderer is used for rendering text. It supports
in-place editing if desired.


\wxheading{Derived from}

\helpref{wxDataViewRenderer}{wxdataviewrenderer}

\wxheading{Include files}

<wx/dataview.h>

\wxheading{Library}

\helpref{wxAdv}{librarieslist}


\membersection{wxDataViewTextRenderer::wxDataViewTextRenderer}\label{wxdataviewtextrendererwxdataviewtextrenderer}

\func{}{wxDataViewTextRenderer}{\param{const wxString\& }{varianttype = wxT("string")}, \param{wxDataViewCellMode }{mode = wxDATAVIEW\_CELL\_INERT}}



\section{\class{wxDataViewProgressRenderer}}\label{wxdataviewprogressrenderer}

wxDataViewProgressRenderer


\wxheading{Derived from}

\helpref{wxDataViewRenderer}{wxdataviewrenderer}

\wxheading{Include files}

<wx/dataview.h>

\wxheading{Library}

\helpref{wxAdv}{librarieslist}


\membersection{wxDataViewProgressRenderer::wxDataViewProgressRenderer}\label{wxdataviewprogressrendererwxdataviewprogressrenderer}

\func{}{wxDataViewProgressRenderer}{\param{const wxString\& }{label = wxEmptyString}, \param{const wxString\& }{varianttype = wxT("long")}, \param{wxDataViewCellMode }{mode = wxDATAVIEW\_CELL\_INERT}}



\section{\class{wxDataViewToggleRenderer}}\label{wxdataviewtogglerenderer}

wxDataViewToggleRenderer

\wxheading{Derived from}

\helpref{wxDataViewRenderer}{wxdataviewrenderer}

\wxheading{Include files}

<wx/dataview.h>

\wxheading{Library}

\helpref{wxAdv}{librarieslist}


\membersection{wxDataViewToggleRenderer::wxDataViewToggleRenderer}\label{wxdataviewtogglerendererwxdataviewtogglerenderer}

\func{}{wxDataViewToggleRenderer}{\param{const wxString\& }{varianttype = wxT("bool")}, \param{wxDataViewCellMode }{mode = wxDATAVIEW\_CELL\_INERT}}


\section{\class{wxDataViewBitmapRenderer}}\label{wxdataviewbitmaprenderer}

wxDataViewBitmapRenderer

\wxheading{Derived from}

\helpref{wxDataViewRenderer}{wxdataviewrenderer}

\wxheading{Include files}

<wx/dataview.h>

\wxheading{Library}

\helpref{wxAdv}{librarieslist}


\membersection{wxDataViewBitmapRenderer::wxDataViewBitmapRenderer}\label{wxdataviewbitmaprendererwxdataviewbitmaprenderer}

\func{}{wxDataViewBitmapRenderer}{\param{const wxString\& }{varianttype = wxT("wxBitmap")}, \param{wxDataViewCellMode }{mode = wxDATAVIEW\_CELL\_INERT}}


\section{\class{wxDataViewDateRenderer}}\label{wxdataviewdaterenderer}

wxDataViewDateRenderer


\wxheading{Derived from}

\helpref{wxDataViewRenderer}{wxdataviewrenderer}

\wxheading{Include files}

<wx/dataview.h>

\wxheading{Library}

\helpref{wxAdv}{librarieslist}

\membersection{wxDataViewDateRenderer::wxDataViewDateRenderer}\label{wxdataviewdaterendererwxdataviewdaterenderer}

\func{}{wxDataViewDateRenderer}{\param{const wxString\& }{varianttype = wxT("datetime")}, \param{wxDataViewCellMode }{mode = wxDATAVIEW\_CELL\_ACTIVATABLE}}


\section{\class{wxDataViewCustomRenderer}}\label{wxdataviewcustomrenderer}

You need to derive a new class from wxDataViewCustomRenderer in
order to write a new renderer. You need to override at least 
\helpref{SetValue}{wxdataviewrenderersetvalue},
\helpref{GetValue}{wxdataviewrenderergetvalue}, 
\helpref{GetSize}{wxdataviewcustomrenderergetsize}
and \helpref{Render}{wxdataviewcustomrendererrender}.

If you want your renderer to support in-place editing then you
also need to override 
\helpref{HasEditorCtrl}{wxdataviewcustomrendererhaseditorctrl},
\helpref{CreateEditorCtrl}{wxdataviewcustomrenderercreateeditorctrl}
and \helpref{GetValueFromEditorCtrl}{wxdataviewcustomrenderergetvaluefromeditorctrl}.
Note that a special event handler will be pushed onto that
editor control which handles <ENTER> and focus out events
in order to end the editing.

\wxheading{Derived from}

\helpref{wxDataViewRenderer}{wxdataviewrenderer}

\wxheading{Include files}

<wx/dataview.h>

\wxheading{Library}

\helpref{wxAdv}{librarieslist}

\membersection{wxDataViewCustomRenderer::wxDataViewCustomRenderer}\label{wxdataviewcustomrendererwxdataviewcustomrenderer}

\func{}{wxDataViewCustomRenderer}{\param{const wxString\& }{varianttype = wxT("string")}, \param{wxDataViewCellMode }{mode = wxDATAVIEW\_CELL\_INERT}, \param{bool }{no\_init = false}}

Constructor.

\membersection{wxDataViewCustomRenderer::\destruct{wxDataViewCustomRenderer}}\label{wxdataviewcustomrendererdtor}

\func{}{\destruct{wxDataViewCustomRenderer}}{\void}

Destructor.


\membersection{wxDataViewCustomRenderer::HasEditorCtrl}\label{wxdataviewcustomrendererhaseditorctrl}

\func{virtual bool}{HasEditorCtrl}{\void}

Override this and make it return {\it true} in order to
indicate that this renderer supports in-place editing.

\membersection{wxDataViewCustomRenderer::CreateEditorCtrl}\label{wxdataviewcustomrenderercreateeditorctrl}

\func{virtual wxControl*}{CreateEditorCtrl}{\param{wxWindow *}{parent}, \param{wxRect }{labelRect}, \param{const wxVariant \& }{value}}

Override this to create the actual editor control once editing
is about to start. {\it parent} is the parent of the editor
control, {\it labelRect} indicates the position and
size of the editor control and {\it value} is its initial value:

{\small
\begin{verbatim}
{ 
    long l = value;
    return new wxSpinCtrl( parent, wxID_ANY, wxEmptyString, 
                 labelRect.GetTopLeft(), labelRect.GetSize(), 0, 0, 100, l );
}
\end{verbatim}
}

\membersection{wxDataViewCustomRenderer::GetValueFromEditorCtrl}\label{wxdataviewcustomrenderergetvaluefromeditorctrl}

\func{virtual bool}{GetValueFromEditorCtrl}{\param{wxControl* }{editor}, \param{wxVariant \& }{value}}

Overrride this so that the renderer can get the value 
from the editor control (pointed to by {\it editor}):

{\small
\begin{verbatim}
{ 
    wxSpinCtrl *sc = (wxSpinCtrl*) editor;
    long l = sc->GetValue();
    value = l;
    return true;
}
\end{verbatim}
}

\membersection{wxDataViewCustomRenderer::Activate}\label{wxdataviewcustomrendereractivate}

\func{virtual bool}{Activate}{\param{wxRect }{cell}, \param{wxDataViewListModel* }{model}, \param{unsigned int }{col}, \param{unsigned int }{row}}

Override this to react to double clicks or <ENTER>.

\membersection{wxDataViewCustomRenderer::GetDC}\label{wxdataviewcustomrenderergetdc}

\func{virtual wxDC*}{GetDC}{\void}

Create DC on request. Internal.


\membersection{wxDataViewCustomRenderer::GetSize}\label{wxdataviewcustomrenderergetsize}

\func{virtual wxSize}{GetSize}{\void}

Return size required to show content.


\membersection{wxDataViewCustomRenderer::LeftClick}\label{wxdataviewcustomrendererleftclick}

\func{virtual bool}{LeftClick}{\param{wxPoint }{cursor}, \param{wxRect }{cell}, \param{wxDataViewListModel* }{model}, \param{unsigned int }{col}, \param{unsigned int }{row}}

Overrride this to react to a left click.

\membersection{wxDataViewCustomRenderer::Render}\label{wxdataviewcustomrendererrender}

\func{virtual bool}{Render}{\param{wxRect }{cell}, \param{wxDC* }{dc}, \param{int }{state}}

Override this to render the cell. Before this is called,
\helpref{SetValue}{wxdataviewrenderersetvalue} was called
so that this instance knows what to render.

\membersection{wxDataViewCustomRenderer::RightClick}\label{wxdataviewcustomrendererrightclick}

\func{virtual bool}{RightClick}{\param{wxPoint }{cursor}, \param{wxRect }{cell}, \param{wxDataViewListModel* }{model}, \param{unsigned int }{col}, \param{unsigned int }{row}}

Overrride this to react to a right click.

\membersection{wxDataViewCustomRenderer::StartDrag}\label{wxdataviewcustomrendererstartdrag}

\func{virtual bool}{StartDrag}{\param{wxPoint }{cursor}, \param{wxRect }{cell}, \param{wxDataViewListModel* }{model}, \param{unsigned int }{col}, \param{unsigned int }{row}}

Overrride this to start a drag operation.



