%%%%%%%%%%%%%%%%%%%%%%%%%%%%%%%%%%%%%%%%%%%%%%%%%%%%%%%%%%%%%%%%%%%%%%%%%%%%%%%
%% Name:        semaphore.tex
%% Purpose:     wxSemaphore documentation
%% Author:      Vadim Zeitlin
%% Modified by:
%% Created:     02.04.02
%% RCS-ID:      $Id$
%% Copyright:   (c) 2002 Vadim Zeitlin
%% License:     wxWindows license
%%%%%%%%%%%%%%%%%%%%%%%%%%%%%%%%%%%%%%%%%%%%%%%%%%%%%%%%%%%%%%%%%%%%%%%%%%%%%%%

\section{\class{wxSemaphore}}\label{wxsemaphore}

wxSemaphore is a counter limiting the number of threads concurrently accessing
a shared resource. This counter is always between $0$ and the maximum value
specified during the semaphore creation. When the counter is strictly greater
than $0$, a call to \helpref{Wait}{wxsemaphorewait} returns immediately and
decrements the counter. As soon as it reaches $0$, any subsequent calls to
\helpref{Wait}{wxsemaphorewait} block and only return when the semaphore
counter becomes strictly positive again as the result of calling 
\helpref{Post}{wxsemaphorepost} which increments the counter.

In general, the semaphores are useful to restrict access to a shared resource
which can only be accessed by some fixed number of clients at once. For
example, when modeling a hotel reservation system a semaphore with the counter
equal to the total number of available rooms could be created. Each time a room
is reserved, the semaphore should be acquired by calling 
\helpref{Wait}{wxsemaphorewait} and each time a room is freed it should be
released by calling \helpref{Post}{wxsemaphorepost}.

\wxheading{Derived from}

No base class

\wxheading{Include files}

<wx/thread.h>

\latexignore{\rtfignore{\wxheading{Members}}}

\membersection{wxSemaphore::wxSemaphore}\label{wxsemaphorewxsemaphore}

\func{}{wxSemaphore}{\param{int }{initialcount = 0}, \param{int }{maxcount = 0}}

Specifying a {\it maxcount} of $0$ actually makes wxSemaphore behave as if
there is no upper limit. If maxcount is $1$ the semaphore behaves exactly as a
mutex.

{\it initialcount} is the initial value of the semaphore which must be between
$0$ and {\it maxcount} (if it is not set to $0$).

\membersection{wxSemaphore::\destruct{wxSemaphore}}\label{wxsemaphoredtor}

\func{}{\destruct{wxSemaphore}}{\void}

Destructor is not virtual, don't use this class polymorphically.

\membersection{wxSemaphore::Post}\label{wxsemaphorepost}

\func{void}{Post}{\void}

Increments the semaphore count and signals one of the waiting threads in an
atomic way.

\membersection{wxSemaphore::TryWait}\label{wxsemaphoretrywait}

\func{bool}{TryWait}{\void}

Same as \helpref{Wait()}{wxsemaphorewait}, but does not block, returns
{\tt TRUE} if the semaphore was successfully acquired and {\tt FALSE} if the
count is zero and it couldn't be done.

\membersection{wxSemaphore::Wait}\label{wxsemaphorewait}

\func{void}{Wait}{\void}

Wait indefinitely until the semaphore count becomes strictly positive
and then decrement it and return.

\func{bool}{Wait}{\param{unsigned long }{timeout\_millis}}

Same as the version above, but with a timeout limit: returns {\tt TRUE} if the
semaphore was acquired and {\tt FALSE} if the timeout has elapsed

