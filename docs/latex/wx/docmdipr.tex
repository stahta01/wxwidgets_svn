\section{\class{wxDocMDIParentFrame}}\label{wxdocmdiparentframe}

The wxDocMDIParentFrame class provides a default top-level frame for
applications using the document/view framework. This class can only be used for MDI parent frames.

It cooperates with the \helpref{wxView}{wxview}, \helpref{wxDocument}{wxdocument},
\rtfsp\helpref{wxDocManager}{wxdocmanager} and \helpref{wxDocTemplates}{wxdoctemplate} classes.

See the example application in {\tt samples/docview}.

\wxheading{Derived from}

\helpref{wxMDIParentFrame}{wxmdiparentframe}\\
\helpref{wxFrame}{wxframe}\\
\helpref{wxWindow}{wxwindow}\\
\helpref{wxEvtHandler}{wxevthandler}\\
\helpref{wxObject}{wxobject}

\wxheading{See also}

\helpref{Document/view overview}{docviewoverview}, \helpref{wxMDIParentFrame}{wxmdiparentframe}

\latexignore{\rtfignore{\wxheading{Members}}}

\membersection{wxDocMDIParentFrame::wxDocMDIParentFrame}

\func{}{wxDocMDIParentFrame}{\param{wxFrame *}{parent}, \param{wxWindowID}{ id},
 \param{const wxString\& }{title}, \param{int}{ x}, \param{int}{ y}, \param{int}{ width}, \param{int}{ height},
 \param{long}{ style}, \param{const wxString\& }{name}}

Constructor.

\membersection{wxDocMDIParentFrame::\destruct{wxDocMDIParentFrame}}

\func{}{\destruct{wxDocMDIParentFrame}}{\void}

Destructor.

\membersection{wxDocMDIParentFrame::OnClose}

\func{bool}{OnClose}{\void}

Deletes all views and documents. If no user input cancelled the
operation, the function returns TRUE and the application will exit.

Since understanding how document/view clean-up takes place can be difficult,
the implementation of this function is shown below.

\begin{verbatim}
bool wxDocMDIParentFrame::OnClose(void)
{
  // Delete all views and documents
  wxNode *node = docManager->GetDocuments().First();
  while (node)
  {
    wxDocument *doc = (wxDocument *)node->Data();
    wxNode *next = node->Next();

    if (!doc->Close())
      return FALSE;

    // Implicitly deletes the document when the last
    // view is removed (deleted)
    doc->DeleteAllViews();

    // Check document is deleted
    if (docManager->GetDocuments().Member(doc))
      delete doc;

    // This assumes that documents are not connected in
    // any way, i.e. deleting one document does NOT
    // delete another.
    node = next;
  }
  return TRUE;
}
\end{verbatim}


