%%%%%%%%%%%%%%%%%%%%%%%%%%%%%%%%%%%%%%%%%%%%%%%%%%%%%%%%%%%%%%%%%%%%%%%%%%%%%%%
%% Name:        function.tex
%% Purpose:     Functions and macros
%% Author:      wxWidgets Team
%% Modified by:
%% Created:
%% RCS-ID:      $Id$
%% Copyright:   (c) wxWidgets Team
%% License:     wxWindows license
%%%%%%%%%%%%%%%%%%%%%%%%%%%%%%%%%%%%%%%%%%%%%%%%%%%%%%%%%%%%%%%%%%%%%%%%%%%%%%%

\chapter{Functions}\label{functions}
\setheader{{\it CHAPTER \thechapter}}{}{}{}{}{{\it CHAPTER \thechapter}}%
\setfooter{\thepage}{}{}{}{}{\thepage}

The functions and macros defined in wxWidgets are described here: you can
either look up a function using the alphabetical listing of them or find it in
the corresponding topic.

\section{Alphabetical functions and macros list}\label{functionsalphabetically}

\helpref{CLASSINFO}{classinfo}\\
\helpref{DECLARE\_ABSTRACT\_CLASS}{declareabstractclass}\\
\helpref{DECLARE\_APP}{declareapp}\\
\helpref{DECLARE\_CLASS}{declareclass}\\
\helpref{DECLARE\_DYNAMIC\_CLASS}{declaredynamicclass}\\
\helpref{IMPLEMENT\_ABSTRACT\_CLASS2}{implementabstractclass2}\\
\helpref{IMPLEMENT\_ABSTRACT\_CLASS}{implementabstractclass}\\
\helpref{IMPLEMENT\_APP}{implementapp}\\
\helpref{IMPLEMENT\_CLASS2}{implementclass2}\\
\helpref{IMPLEMENT\_CLASS}{implementclass}\\
\helpref{IMPLEMENT\_DYNAMIC\_CLASS2}{implementdynamicclass2}\\
\helpref{IMPLEMENT\_DYNAMIC\_CLASS}{implementdynamicclass}\\
\helpref{wxAboutBox}{wxaboutbox}\\
\helpref{wxASSERT}{wxassert}\\
\helpref{wxASSERT\_MIN\_BITSIZE}{wxassertminbitsize}\\
\helpref{wxASSERT\_MSG}{wxassertmsg}\\
\helpref{wxAtomicDec}{wxatomicdec}\\
\helpref{wxAtomicInc}{wxatomicinc}\\
\helpref{wxBase64Decode}{wxbase64decode}\\
\helpref{wxBase64Encode}{wxbase64encode}\\
\helpref{wxBeginBusyCursor}{wxbeginbusycursor}\\
\helpref{wxBell}{wxbell}\\
\helpref{wxBITMAP}{wxbitmapmacro}\\
\helpref{wxCHANGE\_UMASK}{wxchangeumask}\\
\helpref{wxCHECK}{wxcheck}\\
\helpref{wxCHECK2\_MSG}{wxcheck2msg}\\
\helpref{wxCHECK2}{wxcheck2}\\
\helpref{wxCHECK\_GCC\_VERSION}{wxcheckgccversion}\\
\helpref{wxCHECK\_MSG}{wxcheckmsg}\\
\helpref{wxCHECK\_RET}{wxcheckret}\\
\helpref{wxCHECK\_SUNCC\_VERSION}{wxchecksunccversion}\\
\helpref{wxCHECK\_VERSION}{wxcheckversion}\\
\helpref{wxCHECK\_VERSION\_FULL}{wxcheckversionfull}\\
\helpref{wxCHECK\_W32API\_VERSION}{wxcheckw32apiversion}\\
\helpref{wxClientDisplayRect}{wxclientdisplayrect}\\
\helpref{wxClipboardOpen}{functionwxclipboardopen}\\
\helpref{wxCloseClipboard}{wxcloseclipboard}\\
\helpref{wxColourDisplay}{wxcolourdisplay}\\
\helpref{wxCOMPILE\_TIME\_ASSERT}{wxcompiletimeassert}\\
\helpref{wxCOMPILE\_TIME\_ASSERT2}{wxcompiletimeassert2}\\
\helpref{wxCONCAT}{wxconcat}\\
\helpref{wxConcatFiles}{wxconcatfiles}\\
\helpref{wxConstCast}{wxconstcast}\\
\helpref{wxCopyFile}{wxcopyfile}\\
\helpref{wxCreateDynamicObject}{wxcreatedynamicobject}\\
\helpref{wxCreateFileTipProvider}{wxcreatefiletipprovider}\\
\helpref{wxCRIT\_SECT\_DECLARE}{wxcritsectdeclare}\\
\helpref{wxCRIT\_SECT\_DECLARE\_MEMBER}{wxcritsectdeclaremember}\\
\helpref{wxCRIT\_SECT\_LOCKER}{wxcritsectlocker}\\
\helpref{wxCRITICAL\_SECTION}{wxcriticalsectionmacro}\\ % wxcs already taken!
\helpref{wxDDECleanUp}{wxddecleanup}\\
\helpref{wxDDEInitialize}{wxddeinitialize}\\
\helpref{wxDROP\_ICON}{wxdropicon}\\
\helpref{wxDebugMsg}{wxdebugmsg}\\
\helpref{WXDEBUG\_NEW}{debugnew}\\
\helpref{wxDEPRECATED}{wxdeprecated}\\
\helpref{wxDEPRECATED\_BUT\_USED\_INTERNALLY}{wxdeprecatedbutusedinternally}\\
\helpref{wxDirExists}{functionwxdirexists}\\
\helpref{wxDirSelector}{wxdirselector}\\
\helpref{wxDisplayDepth}{wxdisplaydepth}\\
\helpref{wxDisplaySize}{wxdisplaysize}\\
\helpref{wxDisplaySizeMM}{wxdisplaysizemm}\\
\helpref{wxDos2UnixFilename}{wxdos2unixfilename}\\
\helpref{wxDynamicCastThis}{wxdynamiccastthis}\\
\helpref{wxDynamicCast}{wxdynamiccast}\\
\helpref{wxDYNLIB\_FUNCTION}{wxdynlibfunction}\\
\helpref{wxEmptyClipboard}{wxemptyclipboard}\\
\helpref{wxEnableTopLevelWindows}{wxenabletoplevelwindows}\\
\helpref{wxEndBusyCursor}{wxendbusycursor}\\
\helpref{wxENTER\_CRIT\_SECT}{wxentercritsect}\\
\helpref{wxEntry}{wxentry}\\
\helpref{wxEntryStart}{wxentrystart}\\
\helpref{wxEntryCleanup}{wxentrycleanup}\\
\helpref{wxEnumClipboardFormats}{wxenumclipboardformats}\\
\helpref{wxError}{wxerror}\\
\helpref{wxExecute}{wxexecute}\\
\helpref{wxExit}{wxexit}\\
\helpref{wxEXPLICIT}{wxexplicit}\\
\helpref{wxFAIL\_MSG}{wxfailmsg}\\
\helpref{wxFAIL}{wxfail}\\
\helpref{wxFatalError}{wxfatalerror}\\
\helpref{wxFileExists}{functionwxfileexists}\\
\helpref{wxFileModificationTime}{wxfilemodificationtime}\\
\helpref{wxFileNameFromPath}{wxfilenamefrompath}\\
\helpref{wxFileSelector}{wxfileselector}\\
\helpref{wxFindFirstFile}{wxfindfirstfile}\\
\helpref{wxFindMenuItemId}{wxfindmenuitemid}\\
\helpref{wxFindNextFile}{wxfindnextfile}\\
\helpref{wxFindWindowAtPointer}{wxfindwindowatpointer}\\
\helpref{wxFindWindowAtPoint}{wxfindwindowatpoint}\\
\helpref{wxFindWindowByLabel}{wxfindwindowbylabel}\\
\helpref{wxFindWindowByName}{wxfindwindowbyname}\\
\helpref{wxFinite}{wxfinite}\\
\helpref{wxGenericAboutBox}{wxgenericaboutbox}\\
\helpref{wxGetActiveWindow}{wxgetactivewindow}\\
\helpref{wxGetApp}{wxgetapp}\\
\helpref{wxGetBatteryState}{wxgetbatterystate}\\
\helpref{wxGetClipboardData}{wxgetclipboarddata}\\
\helpref{wxGetClipboardFormatName}{wxgetclipboardformatname}\\
\helpref{wxGetColourFromUser}{wxgetcolourfromuser}\\
\helpref{wxGetCwd}{wxgetcwd}\\
\helpref{wxGetDiskSpace}{wxgetdiskspace}\\
\helpref{wxGetDisplayName}{wxgetdisplayname}\\
\helpref{wxGetDisplaySize}{wxdisplaysize}\\
\helpref{wxGetDisplaySizeMM}{wxdisplaysizemm}\\
\helpref{wxGetEmailAddress}{wxgetemailaddress}\\
\helpref{wxGetEnv}{wxgetenv}\\
\helpref{wxGetFileKind}{wxgetfilekind}\\
\helpref{wxGetFontFromUser}{wxgetfontfromuser}\\
\helpref{wxGetFreeMemory}{wxgetfreememory}\\
\helpref{wxGetFullHostName}{wxgetfullhostname}\\
\helpref{wxGetHomeDir}{wxgethomedir}\\
\helpref{wxGetHostName}{wxgethostname}\\
\helpref{wxGetKeyState}{wxgetkeystate}\\
\helpref{wxGetLocalTimeMillis}{wxgetlocaltimemillis}\\
\helpref{wxGetLocalTime}{wxgetlocaltime}\\
\helpref{wxGetMousePosition}{wxgetmouseposition}\\
\helpref{wxGetMouseState}{wxgetmousestate}\\
\helpref{wxGetMultipleChoices}{wxgetmultiplechoices}\\
\helpref{wxGetNumberFromUser}{wxgetnumberfromuser}\\
\helpref{wxGetOSDirectory}{wxgetosdirectory}\\
\helpref{wxGetOsDescription}{wxgetosdescription}\\
\helpref{wxGetOsVersion}{wxgetosversion}\\
\helpref{wxGetPasswordFromUser}{wxgetpasswordfromuser}\\
\helpref{wxGetPowerType}{wxgetpowertype}\\
\helpref{wxGetPrinterCommand}{wxgetprintercommand}\\
\helpref{wxGetPrinterFile}{wxgetprinterfile}\\
\helpref{wxGetPrinterMode}{wxgetprintermode}\\
\helpref{wxGetPrinterOptions}{wxgetprinteroptions}\\
\helpref{wxGetPrinterOrientation}{wxgetprinterorientation}\\
\helpref{wxGetPrinterPreviewCommand}{wxgetprinterpreviewcommand}\\
\helpref{wxGetPrinterScaling}{wxgetprinterscaling}\\
\helpref{wxGetPrinterTranslation}{wxgetprintertranslation}\\
\helpref{wxGetProcessId}{wxgetprocessid}\\
\helpref{wxGetSingleChoiceData}{wxgetsinglechoicedata}\\
\helpref{wxGetSingleChoiceIndex}{wxgetsinglechoiceindex}\\
\helpref{wxGetSingleChoice}{wxgetsinglechoice}\\
\helpref{wxGetTempFileName}{wxgettempfilename}\\
\helpref{wxGetTextFromUser}{wxgettextfromuser}\\
\helpref{wxGetTopLevelParent}{wxgettoplevelparent}\\
\helpref{wxGetTranslation}{wxgettranslation}\\
\helpref{wxGetUTCTime}{wxgetutctime}\\
\helpref{wxGetUserHome}{wxgetuserhome}\\
\helpref{wxGetUserId}{wxgetuserid}\\
\helpref{wxGetUserName}{wxgetusername}\\
\helpref{wxGetWorkingDirectory}{wxgetworkingdirectory}\\
\helpref{wxGetenv}{wxgetenvmacro}\\
\helpref{wxHandleFatalExceptions}{wxhandlefatalexceptions}\\
\helpref{wxICON}{wxiconmacro}\\
\helpref{wxINTXX\_SWAP\_ALWAYS}{intswapalways}\\
\helpref{wxINTXX\_SWAP\_ON\_BE}{intswaponbe}\\
\helpref{wxINTXX\_SWAP\_ON\_LE}{intswaponle}\\
\helpref{wxInitAllImageHandlers}{wxinitallimagehandlers}\\
\helpref{wxInitialize}{wxinitialize}\\
\helpref{wxIsAbsolutePath}{wxisabsolutepath}\\
\helpref{wxIsBusy}{wxisbusy}\\
\helpref{wxIsClipboardFormatAvailable}{wxisclipboardformatavailable}\\
\helpref{wxIsDebuggerRunning}{wxisdebuggerrunning}\\
\helpref{wxIsEmpty}{wxisempty}\\
\helpref{wxIsMainThread}{wxismainthread}\\
\helpref{wxIsNaN}{wxisnan}\\
\helpref{wxIsPlatformLittleEndian}{wxisplatformlittleendian}\\
\helpref{wxIsPlatform64Bit}{wxisplatform64bit}\\
\helpref{wxIsWild}{wxiswild}\\
\helpref{wxJoin}{wxjoin}\\
\helpref{wxKill}{wxkill}\\
\helpref{wxLaunchDefaultBrowser}{wxlaunchdefaultbrowser}\\
\helpref{wxLEAVE\_CRIT\_SECT}{wxleavecritsect}\\
\helpref{wxLoadUserResource}{wxloaduserresource}\\
\helpref{wxLogDebug}{wxlogdebug}\\
\helpref{wxLogError}{wxlogerror}\\
\helpref{wxLogFatalError}{wxlogfatalerror}\\
\helpref{wxLogMessage}{wxlogmessage}\\
\helpref{wxLogStatus}{wxlogstatus}\\
\helpref{wxLogSysError}{wxlogsyserror}\\
\helpref{wxLogTrace}{wxlogtrace}\\
\helpref{wxLogVerbose}{wxlogverbose}\\
\helpref{wxLogWarning}{wxlogwarning}\\
\helpref{wxLL}{wxll}\\
\helpref{wxLongLongFmtSpec}{wxlonglongfmtspec}\\
\helpref{wxMakeMetafilePlaceable}{wxmakemetafileplaceable}\\
\helpref{wxMatchWild}{wxmatchwild}\\
\helpref{wxMessageBox}{wxmessagebox}\\
\helpref{wxMilliSleep}{wxmillisleep}\\
\helpref{wxMicroSleep}{wxmicrosleep}\\
\helpref{wxMkdir}{wxmkdir}\\
\helpref{wxMutexGuiEnter}{wxmutexguienter}\\
\helpref{wxMutexGuiLeave}{wxmutexguileave}\\
\helpref{wxNewId}{wxnewid}\\
\helpref{wxNow}{wxnow}\\
\helpref{wxOnAssert}{wxonassert}\\
\helpref{wxON\_BLOCK\_EXIT}{wxonblockexit}\\
\helpref{wxON\_BLOCK\_EXIT\_OBJ}{wxonblockexitobj}\\
\helpref{wxOpenClipboard}{wxopenclipboard}\\
\helpref{wxParseCommonDialogsFilter}{wxparsecommondialogsfilter}\\
\helpref{wxPathOnly}{wxpathonly}\\
\helpref{wxPLURAL}{wxplural}\\
\helpref{wxPostDelete}{wxpostdelete}\\
\helpref{wxPostEvent}{wxpostevent}\\
\helpref{wxRegisterClipboardFormat}{wxregisterclipboardformat}\\
\helpref{wxRegisterId}{wxregisterid}\\
\helpref{wxRemoveFile}{wxremovefile}\\
\helpref{wxRenameFile}{wxrenamefile}\\
\helpref{wxRmdir}{wxrmdir}\\
\helpref{wxSafeShowMessage}{wxsafeshowmessage}\\
\helpref{wxSafeYield}{wxsafeyield}\\
\helpref{wxSetClipboardData}{wxsetclipboarddata}\\
\helpref{wxSetCursor}{wxsetcursor}\\
\helpref{wxSetDisplayName}{wxsetdisplayname}\\
\helpref{wxSetEnv}{wxsetenv}\\
\helpref{wxSetPrinterCommand}{wxsetprintercommand}\\
\helpref{wxSetPrinterFile}{wxsetprinterfile}\\
\helpref{wxSetPrinterMode}{wxsetprintermode}\\
\helpref{wxSetPrinterOptions}{wxsetprinteroptions}\\
\helpref{wxSetPrinterOrientation}{wxsetprinterorientation}\\
\helpref{wxSetPrinterPreviewCommand}{wxsetprinterpreviewcommand}\\
\helpref{wxSetPrinterScaling}{wxsetprinterscaling}\\
\helpref{wxSetPrinterTranslation}{wxsetprintertranslation}\\
\helpref{wxSetWorkingDirectory}{wxsetworkingdirectory}\\
\helpref{wxShell}{wxshell}\\
\helpref{wxShowTip}{wxshowtip}\\
\helpref{wxShutdown}{wxshutdown}\\
\helpref{wxSleep}{wxsleep}\\
\helpref{wxSnprintf}{wxsnprintf}\\
\helpref{wxSplit}{wxsplit}\\
\helpref{wxSplitPath}{wxsplitfunction}\\
\helpref{wxStaticCast}{wxstaticcast}\\
\helpref{wxStrcmp}{wxstrcmp}\\
\helpref{wxStricmp}{wxstricmp}\\
\helpref{wxStringEq}{wxstringeq}\\
\helpref{wxStringMatch}{wxstringmatch}\\
\helpref{wxStringTokenize}{wxstringtokenize}\\
\helpref{wxStripMenuCodes}{wxstripmenucodes}\\
\helpref{wxStrlen}{wxstrlen}\\
\helpref{wxSTRINGIZE}{wxstringize}\\
\helpref{wxSTRINGIZE\_T}{wxstringizet}\\
\helpref{wxSUPPRESS\_GCC\_PRIVATE\_DTOR\_WARNING}{wxsuppressgccprivatedtorwarning}\\
\helpref{wxSysErrorCode}{wxsyserrorcode}\\
\helpref{wxSysErrorMsg}{wxsyserrormsg}\\
\helpref{wxT}{wxt}\\
\helpref{wxTrace}{wxtrace}\\
\helpref{WXTRACE}{trace}\\
\helpref{wxTraceLevel}{wxtracelevel}\\
\helpref{WXTRACELEVEL}{tracelevel}\\
\helpref{wxTransferFileToStream}{wxtransferfiletostream}\\
\helpref{wxTransferStreamToFile}{wxtransferstreamtofile}\\
\helpref{wxTrap}{wxtrap}\\
\helpref{wxULL}{wxull}\\
\helpref{wxUninitialize}{wxuninitialize}\\
\helpref{wxUnix2DosFilename}{wxunix2dosfilename}\\
\helpref{wxUnsetEnv}{wxunsetenv}\\
\helpref{wxUsleep}{wxusleep}\\
\helpref{wxVaCopy}{wxvacopy}\\
\helpref{wxVsnprintf}{wxvsnprintf}\\
\helpref{wxWakeUpIdle}{wxwakeupidle}\\
\helpref{wxYield}{wxyield}\\
\helpref{wx\_const\_cast}{wxconstcastraw}\\
\helpref{wx\_reinterpret\_cast}{wxreinterpretcastraw}\\
\helpref{wx\_static\_cast}{wxstaticcastraw}\\
\helpref{wx\_truncate\_cast}{wxtruncatecast}\\
\helpref{\_}{underscore}\\
\helpref{\_T}{underscoret}
\helpref{\_\_WXFUNCTION\_\_}{wxfunction}



\section{Version macros}\label{versionfunctions}

The following constants are defined in wxWidgets:

\begin{itemize}\itemsep=0pt
\item {\tt wxMAJOR\_VERSION} is the major version of wxWidgets
\item {\tt wxMINOR\_VERSION} is the minor version of wxWidgets
\item {\tt wxRELEASE\_NUMBER} is the release number
\item {\tt wxSUBRELEASE\_NUMBER} is the subrelease number which is $0$ for all
official releases
\end{itemize}

For example, the values or these constants for wxWidgets 2.1.15 are 2, 1 and
15.

Additionally, {\tt wxVERSION\_STRING} is a user-readable string containing
the full wxWidgets version and {\tt wxVERSION\_NUMBER} is a combination of the
three version numbers above: for 2.1.15, it is 2115 and it is 2200 for
wxWidgets 2.2.

The subrelease number is only used for the sources in between official releases
and so normally is not useful.

\wxheading{Include files}

<wx/version.h> or <wx/defs.h>


\membersection{wxCHECK\_GCC\_VERSION}\label{wxcheckgccversion}

\func{bool}{wxCHECK\_GCC\_VERSION}{\param{}{major, minor}}

Returns $1$ if the compiler being used to compile the code is GNU C++
compiler (g++) version major.minor or greater. Otherwise, and also if
the compiler is not GNU C++ at all, returns $0$.


\membersection{wxCHECK\_SUNCC\_VERSION}\label{wxchecksunccversion}

\func{bool}{wxCHECK\_SUNCC\_VERSION}{\param{}{major, minor}}

Returns $1$ if the compiler being used to compile the code is Sun CC Pro
compiler and its version is at least \texttt{major.minor}. Otherwise returns
$0$.


\membersection{wxCHECK\_VERSION}\label{wxcheckversion}

\func{bool}{wxCHECK\_VERSION}{\param{}{major, minor, release}}

This is a macro which evaluates to true if the current wxWidgets version is at
least major.minor.release.

For example, to test if the program is compiled with wxWidgets 2.2 or higher,
the following can be done:

\begin{verbatim}
    wxString s;
#if wxCHECK_VERSION(2, 2, 0)
    if ( s.StartsWith("foo") )
#else // replacement code for old version
    if ( strncmp(s, "foo", 3) == 0 )
#endif
    {
        ...
    }
\end{verbatim}


\membersection{wxCHECK\_VERSION\_FULL}\label{wxcheckversionfull}

\func{bool}{wxCHECK\_VERSION\_FULL}{\param{}{major, minor, release, subrel}}

Same as \helpref{wxCHECK\_VERSION}{wxcheckversion} but also checks that
\texttt{wxSUBRELEASE\_NUMBER} is at least \arg{subrel}.


\membersection{wxCHECK\_W32API\_VERSION}\label{wxcheckw32apiversion}

\func{bool}{wxCHECK\_W32API\_VERSION}{\param{}{major, minor, release}}

Returns $1$ if the version of w32api headers used is major.minor.release or
greater. Otherwise, and also if we are not compiling with mingw32/cygwin under
Win32 at all, returns $0$.



\section{Application initialization and termination}\label{appinifunctions}

The functions in this section are used on application startup/shutdown and also
to control the behaviour of the main event loop of the GUI programs.


\membersection{::wxEntry}\label{wxentry}

This initializes wxWidgets in a platform-dependent way. Use this if you are not
using the default wxWidgets entry code (e.g. main or WinMain). For example, you
can initialize wxWidgets from an Microsoft Foundation Classes application using
this function.

The following overload of wxEntry is available under all platforms:

\func{int}{wxEntry}{\param{int\&}{ argc}, \param{wxChar **}{argv}}

Under MS Windows, an additional overload suitable for calling from 
\texttt{WinMain} is available:

\func{int}{wxEntry}{\param{HINSTANCE }{hInstance}, \param{HINSTANCE }{hPrevInstance = \NULL}, \param{char *}{pCmdLine = \NULL}, \param{int }{nCmdShow = \texttt{SW\_SHOWNORMAL}}}

(notice that under Windows CE platform, and only there, the type of 
\arg{pCmdLine} is \texttt{wchar\_t *}, otherwise it is \texttt{char *}, even in
Unicode build).

\wxheading{See also}

\helpref{wxEntryStart}{wxentrystart}

\wxheading{Remarks}

To clean up wxWidgets, call wxApp::OnExit followed by the static function
wxApp::CleanUp. For example, if exiting from an MFC application that also uses wxWidgets:

\begin{verbatim}
int CTheApp::ExitInstance()
{
  // OnExit isn't called by CleanUp so must be called explicitly.
  wxTheApp->OnExit();
  wxApp::CleanUp();

  return CWinApp::ExitInstance();
}
\end{verbatim}

\wxheading{Include files}

<wx/app.h>



\membersection{::wxEntryCleanup}\label{wxentrycleanup}

\func{void}{wxEntryCleanup}{\void}

Free resources allocated by a successful call to \helpref{wxEntryStart}{wxentrystart}.

\wxheading{Include files}

<wx/init.h>


\membersection{::wxEntryStart}\label{wxentrystart}

\func{bool}{wxEntryStart}{\param{int\&}{ argc}, \param{wxChar **}{argv}}

This function can be used to perform the initialization of wxWidgets if you
can't use the default initialization code for any reason.

If the function returns \true, the initialization was successful and the global 
\helpref{wxApp}{wxapp} object \texttt{wxTheApp} has been created. Moreover, 
\helpref{wxEntryCleanup}{wxentrycleanup} must be called afterwards. If the
function returns \false, a catastrophic initialization error occured and (at
least the GUI part of) the library can't be used at all.

Notice that parameters \arg{argc} and \arg{argv} may be modified by this
function.

An additional overload of wxEntryStart() is provided under MSW only: it is
meant to be called with the parameters passed to \texttt{WinMain()}.

\func{bool}{wxEntryStart}{\param{HINSTANCE }{hInstance}, \param{HINSTANCE }{hPrevInstance = \NULL}, \param{char *}{pCmdLine = \NULL}, \param{int }{nCmdShow = \texttt{SW\_SHOWNORMAL}}}

(notice that under Windows CE platform, and only there, the type of 
\arg{pCmdLine} is \texttt{wchar\_t *}, otherwise it is \texttt{char *}, even in
Unicode build).

\wxheading{Include files}

<wx/init.h>


\membersection{::wxGetApp}\label{wxgetapp}

\func{wxAppDerivedClass\&}{wxGetApp}{\void}

This function doesn't exist in wxWidgets but it is created by using
the \helpref{IMPLEMENT\_APP}{implementapp} macro. Thus, before using it
anywhere but in the same module where this macro is used, you must make it
available using \helpref{DECLARE\_APP}{declareapp}.

The advantage of using this function compared to directly using the global
wxTheApp pointer is that the latter is of type {\tt wxApp *} and so wouldn't
allow you to access the functions specific to your application class but not
present in wxApp while wxGetApp() returns the object of the right type.


\membersection{::wxHandleFatalExceptions}\label{wxhandlefatalexceptions}

\func{bool}{wxHandleFatalExceptions}{\param{bool}{ doIt = true}}

If {\it doIt} is true, the fatal exceptions (also known as general protection
faults under Windows or segmentation violations in the Unix world) will be
caught and passed to \helpref{wxApp::OnFatalException}{wxapponfatalexception}.
By default, i.e. before this function is called, they will be handled in the
normal way which usually just means that the application will be terminated.
Calling wxHandleFatalExceptions() with {\it doIt} equal to false will restore
this default behaviour.


\membersection{::wxInitAllImageHandlers}\label{wxinitallimagehandlers}

\func{void}{wxInitAllImageHandlers}{\void}

Initializes all available image handlers. For a list of available handlers,
see \helpref{wxImage}{wximage}.

\wxheading{See also}

\helpref{wxImage}{wximage}, \helpref{wxImageHandler}{wximagehandler}

\wxheading{Include files}

<wx/image.h>


\membersection{::wxInitialize}\label{wxinitialize}

\func{bool}{wxInitialize}{\void}

This function is used in wxBase only and only if you don't create
\helpref{wxApp}{wxapp} object at all. In this case you must call it from your
{\tt main()} function before calling any other wxWidgets functions.

If the function returns \false the initialization could not be performed,
in this case the library cannot be used and
\helpref{wxUninitialize}{wxuninitialize} shouldn't be called neither.

This function may be called several times but
\helpref{wxUninitialize}{wxuninitialize} must be called for each successful
call to this function.

\wxheading{Include files}

<wx/app.h>


\membersection{::wxSafeYield}\label{wxsafeyield}

\func{bool}{wxSafeYield}{\param{wxWindow*}{ win = NULL}, \param{bool}{
    onlyIfNeeded = false}}

This function is similar to wxYield, except that it disables the user input to
all program windows before calling wxYield and re-enables it again
afterwards. If {\it win} is not NULL, this window will remain enabled,
allowing the implementation of some limited user interaction.

Returns the result of the call to \helpref{::wxYield}{wxyield}.

\wxheading{Include files}

<wx/utils.h>


\membersection{::wxUninitialize}\label{wxuninitialize}

\func{void}{wxUninitialize}{\void}

This function is for use in console (wxBase) programs only. It must be called
once for each previous successful call to \helpref{wxInitialize}{wxinitialize}.

\wxheading{Include files}

<wx/app.h>


\membersection{::wxYield}\label{wxyield}

\func{bool}{wxYield}{\void}

Calls \helpref{wxApp::Yield}{wxappyield}.

This function is kept only for backwards compatibility. Please use
the \helpref{wxApp::Yield}{wxappyield} method instead in any new code.

\wxheading{Include files}

<wx/app.h> or <wx/utils.h>


\membersection{::wxWakeUpIdle}\label{wxwakeupidle}

\func{void}{wxWakeUpIdle}{\void}

This functions wakes up the (internal and platform dependent) idle system, i.e. it
will force the system to send an idle event even if the system currently {\it is}
 idle and thus would not send any idle event until after some other event would get
sent. This is also useful for sending events between two threads and is used by
the corresponding functions \helpref{::wxPostEvent}{wxpostevent} and
\helpref{wxEvtHandler::AddPendingEvent}{wxevthandleraddpendingevent}.

\wxheading{Include files}

<wx/app.h>



\section{Process control functions}\label{processfunctions}

The functions in this section are used to launch or terminate the other
processes.


\membersection{::wxExecute}\label{wxexecute}

\func{long}{wxExecute}{\param{const wxString\& }{command}, \param{int }{sync = wxEXEC\_ASYNC}, \param{wxProcess *}{callback = NULL}}

\perlnote{In wxPerl this function is called \texttt{Wx::ExecuteCommand}}

\func{long}{wxExecute}{\param{char **}{argv}, \param{int }{flags = wxEXEC\_ASYNC}, \param{wxProcess *}{callback = NULL}}

\perlnote{In wxPerl this function is called \texttt{Wx::ExecuteArgs}}

\func{long}{wxExecute}{\param{const wxString\& }{command}, \param{wxArrayString\& }{output}, \param{int }{flags = 0}}

\perlnote{In wxPerl this function is called \texttt{Wx::ExecuteStdout} and it
only takes the {\tt command} argument,
and returns a 2-element list {\tt ( status, output )}, where {\tt output} is
an array reference.}

\func{long}{wxExecute}{\param{const wxString\& }{command}, \param{wxArrayString\& }{output}, \param{wxArrayString\& }{errors}, \param{int }{flags = 0}}

\perlnote{In wxPerl this function is called \texttt{Wx::ExecuteStdoutStderr}
and it only takes the {\tt command} argument,
and returns a 3-element list {\tt ( status, output, errors )}, where
{\tt output} and {\tt errors} are array references.}

Executes another program in Unix or Windows.

The first form takes a command string, such as {\tt "emacs file.txt"}.

The second form takes an array of values: a command, any number of
arguments, terminated by NULL.

The semantics of the third and fourth versions is different from the first two
and is described in more details below.

If {\it flags} parameter contains {\tt wxEXEC\_ASYNC} flag (the default), flow
of control immediately returns. If it contains {\tt wxEXEC\_SYNC}, the current
application waits until the other program has terminated.

In the case of synchronous execution, the return value is the exit code of
the process (which terminates by the moment the function returns) and will be
$-1$ if the process couldn't be started and typically 0 if the process
terminated successfully. Also, while waiting for the process to
terminate, wxExecute will call \helpref{wxYield}{wxyield}. Because of this, by
default this function disables all application windows to avoid unexpected
reentrancies which could result from the users interaction with the program
while the child process is running. If you are sure that it is safe to not
disable the program windows, you may pass \texttt{wxEXEC\_NODISABLE} flag to
prevent this automatic disabling from happening.

For asynchronous execution, however, the return value is the process id and
zero value indicates that the command could not be executed. As an added
complication, the return value of $-1$ in this case indicates that we didn't
launch a new process, but connected to the running one (this can only happen in
case of using DDE under Windows for command execution). In particular, in this,
and only this, case the calling code will not get the notification about
process termination.

If callback isn't NULL and if execution is asynchronous,
\helpref{wxProcess::OnTerminate}{wxprocessonterminate} will be called when
the process finishes. Specifying this parameter also allows you to redirect the
standard input and/or output of the process being launched by calling
\helpref{Redirect}{wxprocessredirect}. If the child process IO is redirected,
under Windows the process window is not shown by default (this avoids having to
flush an unnecessary console for the processes which don't create any windows
anyhow) but a {\tt wxEXEC\_NOHIDE} flag can be used to prevent this from
happening, i.e. with this flag the child process window will be shown normally.

Under Unix the flag {\tt wxEXEC\_MAKE\_GROUP\_LEADER} may be used to ensure
that the new process is a group leader (this will create a new session if
needed). Calling \helpref{wxKill}{wxkill} passing wxKILL\_CHILDREN will
kill this process as well as all of its children (except those which have
started their own session).

The {\tt wxEXEC\_NOEVENTS} flag prevents processing of any events from taking
place while the child process is running. It should be only used for very
short-lived processes as otherwise the application windows risk becoming
unresponsive from the users point of view. As this flag only makes sense with
{\tt wxEXEC\_SYNC}, {\tt wxEXEC\_BLOCK} equal to the sum of both of these flags
is provided as a convenience.

Finally, you may use the third overloaded version of this function to execute
a process (always synchronously, the contents of \arg{flags} is or'd with
\texttt{wxEXEC\_SYNC}) and capture its output in the array \arg{output}. The
fourth version adds the possibility to additionally capture the messages from
standard error output in the \arg{errors} array.

{\bf NB:} Currently wxExecute() can only be used from the main thread, calling
this function from another thread will result in an assert failure in debug
build and won't work.

\wxheading{See also}

\helpref{wxShell}{wxshell}, \helpref{wxProcess}{wxprocess}, \helpref{Exec sample}{sampleexec}.

\wxheading{Parameters}

\docparam{command}{The command to execute and any parameters to pass to it as a
single string.}

\docparam{argv}{The command to execute should be the first element of this
array, any additional ones are the command parameters and the array must be
terminated with a NULL pointer.}

\docparam{flags}{Combination of bit masks {\tt wxEXEC\_ASYNC},\rtfsp
{\tt wxEXEC\_SYNC} and {\tt wxEXEC\_NOHIDE}}

\docparam{callback}{An optional pointer to \helpref{wxProcess}{wxprocess}}

\wxheading{Include files}

<wx/utils.h>


\membersection{::wxExit}\label{wxexit}

\func{void}{wxExit}{\void}

Exits application after calling \helpref{wxApp::OnExit}{wxapponexit}.
Should only be used in an emergency: normally the top-level frame
should be deleted (after deleting all other frames) to terminate the
application. See \helpref{wxCloseEvent}{wxcloseevent} and \helpref{wxApp}{wxapp}.

\wxheading{Include files}

<wx/app.h>


\membersection{::wxJoin}\label{wxjoin}

\func{wxString}{wxJoin}{\param{const wxArrayString\&}{ arr}, \param{const wxChar}{ sep}, \param{const wxChar}{ escape = '$\backslash$'}}

Concatenate all lines of the given \helpref{wxArrayString}{wxarraystring} object using the separator \arg{sep} and returns
the result as a \helpref{wxString}{wxstring}.

If the \arg{escape} character is non-\NULL, then it's used as prefix for each occurrence of \arg{sep}
in the strings contained in \arg{arr} before joining them which is necessary
in order to be able to recover the original array contents from the string
later using \helpref{wxSplit}{wxsplit}.

\wxheading{Include files}

<wx/arrstr.h>


\membersection{::wxKill}\label{wxkill}

\func{int}{wxKill}{\param{long}{ pid}, \param{int}{ sig = wxSIGTERM}, \param{wxKillError }{*rc = NULL}, \param{int }{flags = 0}}

Equivalent to the Unix kill function: send the given signal {\it sig} to the
process with PID {\it pid}. The valid signal values are

\begin{verbatim}
enum wxSignal
{
    wxSIGNONE = 0,  // verify if the process exists under Unix
    wxSIGHUP,
    wxSIGINT,
    wxSIGQUIT,
    wxSIGILL,
    wxSIGTRAP,
    wxSIGABRT,
    wxSIGEMT,
    wxSIGFPE,
    wxSIGKILL,      // forcefully kill, dangerous!
    wxSIGBUS,
    wxSIGSEGV,
    wxSIGSYS,
    wxSIGPIPE,
    wxSIGALRM,
    wxSIGTERM       // terminate the process gently
};
\end{verbatim}

{\tt wxSIGNONE}, {\tt wxSIGKILL} and {\tt wxSIGTERM} have the same meaning
under both Unix and Windows but all the other signals are equivalent to
{\tt wxSIGTERM} under Windows.

Returns 0 on success, -1 on failure. If {\it rc} parameter is not NULL, it will
be filled with an element of {\tt wxKillError} enum:

\begin{verbatim}
enum wxKillError
{
    wxKILL_OK,              // no error
    wxKILL_BAD_SIGNAL,      // no such signal
    wxKILL_ACCESS_DENIED,   // permission denied
    wxKILL_NO_PROCESS,      // no such process
    wxKILL_ERROR            // another, unspecified error
};
\end{verbatim}

The {\it flags} parameter can be wxKILL\_NOCHILDREN (the default),
or wxKILL\_CHILDREN, in which case the child processes of this
process will be killed too. Note that under Unix, for wxKILL\_CHILDREN
to work you should have created the process by passing wxEXEC\_MAKE\_GROUP\_LEADER
to wxExecute.

\wxheading{See also}

\helpref{wxProcess::Kill}{wxprocesskill},\rtfsp
\helpref{wxProcess::Exists}{wxprocessexists},\rtfsp
\helpref{Exec sample}{sampleexec}

\wxheading{Include files}

<wx/utils.h>


\membersection{::wxGetProcessId}\label{wxgetprocessid}

\func{unsigned long}{wxGetProcessId}{\void}

Returns the number uniquely identifying the current process in the system.

If an error occurs, $0$ is returned.

\wxheading{Include files}

<wx/utils.h>


\membersection{::wxShell}\label{wxshell}

\func{bool}{wxShell}{\param{const wxString\& }{command = NULL}}

Executes a command in an interactive shell window. If no command is
specified, then just the shell is spawned.

See also \helpref{wxExecute}{wxexecute}, \helpref{Exec sample}{sampleexec}.

\wxheading{Include files}

<wx/utils.h>


\membersection{::wxShutdown}\label{wxshutdown}

\func{bool}{wxShutdown}{\param{wxShutdownFlags}{flags}}

This function shuts down or reboots the computer depending on the value of the
{\it flags}. Please notice that doing this requires the corresponding access
rights (superuser under Unix, {\tt SE\_SHUTDOWN} privilege under Windows NT)
and that this function is only implemented under Unix and Win32.

\wxheading{Parameters}

\docparam{flags}{Either {\tt wxSHUTDOWN\_POWEROFF} or {\tt wxSHUTDOWN\_REBOOT}}

\wxheading{Returns}

\true on success, \false if an error occurred.

\wxheading{Include files}

<wx/utils.h>



\section{Thread functions}\label{threadfunctions}

The functions and macros here mainly exist to make it writing the code which
may be compiled in multi thread build ({\tt wxUSE\_THREADS} $= 1$) as well as
in single thread configuration ({\tt wxUSE\_THREADS} $= 0$).

For example, a static variable must be protected against simultaneous access by
multiple threads in the former configuration but in the latter the extra
overhead of using the critical section is not needed. To solve this problem,
the \helpref{wxCRITICAL\_SECTION}{wxcriticalsectionmacro} macro may be used
to create and use the critical section only when needed.

\wxheading{Include files}

<wx/thread.h>

\wxheading{See also}

\helpref{wxThread}{wxthread}, \helpref{wxMutex}{wxmutex}, \helpref{Multithreading overview}{wxthreadoverview}



\membersection{wxCRIT\_SECT\_DECLARE}\label{wxcritsectdeclare}

\func{}{wxCRIT\_SECT\_DECLARE}{\param{}{cs}}

This macro declares a (static) critical section object named {\it cs} if
{\tt wxUSE\_THREADS} is $1$ and does nothing if it is $0$.



\membersection{wxCRIT\_SECT\_DECLARE\_MEMBER}\label{wxcritsectdeclaremember}

\func{}{wxCRIT\_SECT\_DECLARE}{\param{}{cs}}

This macro declares a critical section object named {\it cs} if
{\tt wxUSE\_THREADS} is $1$ and does nothing if it is $0$. As it doesn't
include the {\tt static} keyword (unlike
\helpref{wxCRIT\_SECT\_DECLARE}{wxcritsectdeclare}), it can be used to declare
a class or struct member which explains its name.



\membersection{wxCRIT\_SECT\_LOCKER}\label{wxcritsectlocker}

\func{}{wxCRIT\_SECT\_LOCKER}{\param{}{name}, \param{}{cs}}

This macro creates a \helpref{critical section lock}{wxcriticalsectionlocker}
object named {\it name} and associated with the critical section {\it cs} if
{\tt wxUSE\_THREADS} is $1$ and does nothing if it is $0$.



\membersection{wxCRITICAL\_SECTION}\label{wxcriticalsectionmacro}

\func{}{wxCRITICAL\_SECTION}{\param{}{name}}

This macro combines \helpref{wxCRIT\_SECT\_DECLARE}{wxcritsectdeclare} and
\helpref{wxCRIT\_SECT\_LOCKER}{wxcritsectlocker}: it creates a static critical
section object and also the lock object associated with it. Because of this, it
can be only used inside a function, not at global scope. For example:

\begin{verbatim}
int IncCount()
{
    static int s_counter = 0;

    wxCRITICAL_SECTION(counter);

    return ++s_counter;
}
\end{verbatim}

(note that we suppose that the function is called the first time from the main
thread so that the critical section object is initialized correctly by the time
other threads start calling it, if this is not the case this approach can
{\bf not} be used and the critical section must be made a global instead).



\membersection{wxENTER\_CRIT\_SECT}\label{wxentercritsect}

\func{}{wxENTER\_CRIT\_SECT}{\param{wxCriticalSection\& }{cs}}

This macro is equivalent to \helpref{cs.Enter()}{wxcriticalsectionenter} if
{\tt wxUSE\_THREADS} is $1$ and does nothing if it is $0$.



\membersection{::wxIsMainThread}\label{wxismainthread}

\func{bool}{wxIsMainThread}{\void}

Returns \true if this thread is the main one. Always returns \true if
{\tt wxUSE\_THREADS} is $0$.



\membersection{wxLEAVE\_CRIT\_SECT}\label{wxleavecritsect}

\func{}{wxLEAVE\_CRIT\_SECT}{\param{wxCriticalSection\& }{cs}}

This macro is equivalent to \helpref{cs.Leave()}{wxcriticalsectionleave} if
{\tt wxUSE\_THREADS} is $1$ and does nothing if it is $0$.



\membersection{::wxMutexGuiEnter}\label{wxmutexguienter}

\func{void}{wxMutexGuiEnter}{\void}

This function must be called when any thread other than the main GUI thread
wants to get access to the GUI library. This function will block the execution
of the calling thread until the main thread (or any other thread holding the
main GUI lock) leaves the GUI library and no other thread will enter the GUI
library until the calling thread calls \helpref{::wxMutexGuiLeave()}{wxmutexguileave}.

Typically, these functions are used like this:

\begin{verbatim}
void MyThread::Foo(void)
{
    // before doing any GUI calls we must ensure that this thread is the only
    // one doing it!

    wxMutexGuiEnter();

    // Call GUI here:
    my_window->DrawSomething();

    wxMutexGuiLeave();
}
\end{verbatim}

Note that under GTK, no creation of top-level windows is allowed in any
thread but the main one.

This function is only defined on platforms which support preemptive
threads.


\membersection{::wxMutexGuiLeave}\label{wxmutexguileave}

\func{void}{wxMutexGuiLeave}{\void}

See \helpref{::wxMutexGuiEnter()}{wxmutexguienter}.

This function is only defined on platforms which support preemptive
threads.



\section{File functions}\label{filefunctions}

\wxheading{Include files}

<wx/filefn.h>

\wxheading{See also}

\helpref{wxPathList}{wxpathlist}\\
\helpref{wxDir}{wxdir}\\
\helpref{wxFile}{wxfile}\\
\helpref{wxFileName}{wxfilename}


\membersection{::wxDos2UnixFilename}\label{wxdos2unixfilename}

\func{void}{wxDos2UnixFilename}{\param{wxChar *}{s}}

Converts a DOS to a Unix filename by replacing backslashes with forward
slashes.


\membersection{::wxFileExists}\label{functionwxfileexists}

\func{bool}{wxFileExists}{\param{const wxString\& }{filename}}

Returns true if the file exists and is a plain file.


\membersection{::wxFileModificationTime}\label{wxfilemodificationtime}

\func{time\_t}{wxFileModificationTime}{\param{const wxString\& }{filename}}

Returns time of last modification of given file.

The function returns \texttt{(time\_t)}$-1$ if an error occurred (e.g. file not
found).


\membersection{::wxFileNameFromPath}\label{wxfilenamefrompath}

\func{wxString}{wxFileNameFromPath}{\param{const wxString\& }{path}}

\func{char *}{wxFileNameFromPath}{\param{char *}{path}}

{\bf NB:} This function is obsolete, please use
\helpref{wxFileName::SplitPath}{wxfilenamesplitpath} instead.

Returns the filename for a full path. The second form returns a pointer to
temporary storage that should not be deallocated.


\membersection{::wxFindFirstFile}\label{wxfindfirstfile}

\func{wxString}{wxFindFirstFile}{\param{const wxString\& }{spec}, \param{int}{ flags = 0}}

This function does directory searching; returns the first file
that matches the path {\it spec}, or the empty string. Use \helpref{wxFindNextFile}{wxfindnextfile} to
get the next matching file. Neither will report the current directory "." or the
parent directory "..".

\wxheading{Warning}

As of wx 2.5.2, these functions are not thread-safe! (they use static variables). You probably want to use \helpref{wxDir::GetFirst}{wxdirgetfirst} or \helpref{wxDirTraverser}{wxdirtraverser} instead.

{\it spec} may contain wildcards.

{\it flags} may be wxDIR for restricting the query to directories, wxFILE for files or zero for either.

For example:

\begin{verbatim}
  wxString f = wxFindFirstFile("/home/project/*.*");
  while ( !f.empty() )
  {
    ...
    f = wxFindNextFile();
  }
\end{verbatim}


\membersection{::wxFindNextFile}\label{wxfindnextfile}

\func{wxString}{wxFindNextFile}{\void}

Returns the next file that matches the path passed to \helpref{wxFindFirstFile}{wxfindfirstfile}.

See \helpref{wxFindFirstFile}{wxfindfirstfile} for an example.


\membersection{::wxGetDiskSpace}\label{wxgetdiskspace}

\func{bool}{wxGetDiskSpace}{\param{const wxString\& }{path}, \param{wxLongLong }{*total = NULL}, \param{wxLongLong }{*free = NULL}}

This function returns the total number of bytes and number of free bytes on
the disk containing the directory {\it path} (it should exist). Both
{\it total} and {\it free} parameters may be {\tt NULL} if the corresponding
information is not needed.

\wxheading{Returns}

\true on success, \false if an error occurred (for example, the
directory doesn't exist).

\wxheading{Portability}

The generic Unix implementation depends on the system having
the \texttt{statfs()} or \texttt{statvfs()} function.

This function first appeared in wxWidgets 2.3.2.


\membersection{::wxGetFileKind}\label{wxgetfilekind}

\func{wxFileKind}{wxGetFileKind}{\param{int }{fd}}

\func{wxFileKind}{wxGetFileKind}{\param{FILE *}{fp}}

Returns the type of an open file. Possible return values are:

\begin{verbatim}
enum wxFileKind
{
  wxFILE_KIND_UNKNOWN,
  wxFILE_KIND_DISK,     // a file supporting seeking to arbitrary offsets
  wxFILE_KIND_TERMINAL, // a tty
  wxFILE_KIND_PIPE      // a pipe
};

\end{verbatim}

\wxheading{Include files}

<wx/filefn.h>


\membersection{::wxGetOSDirectory}\label{wxgetosdirectory}

\func{wxString}{wxGetOSDirectory}{\void}

Returns the Windows directory under Windows; on other platforms returns the empty string.


\membersection{::wxIsAbsolutePath}\label{wxisabsolutepath}

\func{bool}{wxIsAbsolutePath}{\param{const wxString\& }{filename}}

Returns true if the argument is an absolute filename, i.e. with a slash
or drive name at the beginning.


\membersection{::wxDirExists}\label{functionwxdirexists}

\func{bool}{wxDirExists}{\param{const wxString\& }{dirname}}

Returns true if \arg{dirname} exists and is a directory.


\membersection{::wxPathOnly}\label{wxpathonly}

\func{wxString}{wxPathOnly}{\param{const wxString\& }{path}}

Returns the directory part of the filename.


\membersection{::wxUnix2DosFilename}\label{wxunix2dosfilename}

\func{void}{wxUnix2DosFilename}{\param{wxChar *}{s}}

This function is deprecated, use \helpref{wxFileName}{wxfilename} instead.

Converts a Unix to a DOS filename by replacing forward
slashes with backslashes.


\membersection{wxCHANGE\_UMASK}\label{wxchangeumask}

\func{}{wxCHANGE\_UMASK}{\param{int }{mask}}

Under Unix this macro changes the current process umask to the given value,
unless it is equal to $-1$ in which case nothing is done, and restores it to
the original value on scope exit. It works by declaring a variable which sets
umask to \arg{mask} in its constructor and restores it in its destructor.

Under other platforms this macro expands to nothing.


\membersection{::wxConcatFiles}\label{wxconcatfiles}

\func{bool}{wxConcatFiles}{\param{const wxString\& }{file1}, \param{const wxString\& }{file2},
\param{const wxString\& }{file3}}

Concatenates {\it file1} and {\it file2} to {\it file3}, returning
true if successful.


\membersection{::wxCopyFile}\label{wxcopyfile}

\func{bool}{wxCopyFile}{\param{const wxString\& }{file1}, \param{const wxString\& }{file2}, \param{bool }{overwrite = true}}

Copies {\it file1} to {\it file2}, returning true if successful. If
{\it overwrite} parameter is \true (default), the destination file is overwritten
if it exists, but if {\it overwrite} is \false, the functions fails in this
case.

This function supports resources forks under Mac OS.


\membersection{::wxGetCwd}\label{wxgetcwd}

\func{wxString}{wxGetCwd}{\void}

Returns a string containing the current (or working) directory.


\membersection{::wxGetWorkingDirectory}\label{wxgetworkingdirectory}

\func{wxString}{wxGetWorkingDirectory}{\param{char *}{buf=NULL}, \param{int }{sz=1000}}

{\bf NB:} This function is deprecated: use \helpref{wxGetCwd}{wxgetcwd} instead.

Copies the current working directory into the buffer if supplied, or
copies the working directory into new storage (which you {\emph must} delete
yourself) if the buffer is NULL.

{\it sz} is the size of the buffer if supplied.


\membersection{::wxGetTempFileName}\label{wxgettempfilename}

\func{char *}{wxGetTempFileName}{\param{const wxString\& }{prefix}, \param{char *}{buf=NULL}}

\func{bool}{wxGetTempFileName}{\param{const wxString\& }{prefix}, \param{wxString\& }{buf}}

%% Makes a temporary filename based on {\it prefix}, opens and closes the file,
%% and places the name in {\it buf}. If {\it buf} is NULL, new store
%% is allocated for the temporary filename using {\it new}.
%%
%% Under Windows, the filename will include the drive and name of the
%% directory allocated for temporary files (usually the contents of the
%% TEMP variable). Under Unix, the {\tt /tmp} directory is used.
%%
%% It is the application's responsibility to create and delete the file.

{\bf NB:} These functions are obsolete, please use\rtfsp
\helpref{wxFileName::CreateTempFileName}{wxfilenamecreatetempfilename}\rtfsp
instead.


\membersection{::wxIsWild}\label{wxiswild}

\func{bool}{wxIsWild}{\param{const wxString\& }{pattern}}

Returns true if the pattern contains wildcards. See \helpref{wxMatchWild}{wxmatchwild}.


\membersection{::wxMatchWild}\label{wxmatchwild}

\func{bool}{wxMatchWild}{\param{const wxString\& }{pattern}, \param{const wxString\& }{text}, \param{bool}{ dot\_special}}

Returns true if the \arg{pattern}\/ matches the {\it text}\/; if {\it
dot\_special}\/ is true, filenames beginning with a dot are not matched
with wildcard characters. See \helpref{wxIsWild}{wxiswild}.


\membersection{::wxMkdir}\label{wxmkdir}

\func{bool}{wxMkdir}{\param{const wxString\& }{dir}, \param{int }{perm = 0777}}

Makes the directory \arg{dir}, returning true if successful.

{\it perm} is the access mask for the directory for the systems on which it is
supported (Unix) and doesn't have any effect on the other ones.


\membersection{::wxParseCommonDialogsFilter}\label{wxparsecommondialogsfilter}

\func{int}{wxParseCommonDialogsFilter}{\param{const wxString\& }{wildCard}, \param{wxArrayString\& }{descriptions}, \param{wxArrayString\& }{filters}}

Parses the \arg{wildCard}, returning the number of filters.
Returns 0 if none or if there's a problem.
The arrays will contain an equal number of items found before the error.
On platforms where native dialogs handle only one filter per entry,
entries in arrays are automatically adjusted.
\arg{wildCard} is in the form:
\begin{verbatim}
 "All files (*)|*|Image Files (*.jpeg *.png)|*.jpg;*.png"
\end{verbatim}

\membersection{::wxRemoveFile}\label{wxremovefile}

\func{bool}{wxRemoveFile}{\param{const wxString\& }{file}}

Removes \arg{file}, returning true if successful.


\membersection{::wxRenameFile}\label{wxrenamefile}

\func{bool}{wxRenameFile}{\param{const wxString\& }{file1}, \param{const wxString\& }{file2}, \param{bool }{overwrite = true}}

Renames \arg{file1} to \arg{file2}, returning true if successful.

If \arg{overwrite} parameter is true (default), the destination file is
overwritten if it exists, but if \arg{overwrite} is false, the functions fails
in this case.


\membersection{::wxRmdir}\label{wxrmdir}

\func{bool}{wxRmdir}{\param{const wxString\& }{dir}, \param{int}{ flags=0}}

Removes the directory {\it dir}, returning true if successful. Does not work under VMS.

The {\it flags} parameter is reserved for future use.

Please notice that there is also a wxRmDir() function which simply wraps the
standard POSIX rmdir() function and so return an integer error code instead of
a boolean value (but otherwise is currently identical to wxRmdir), don't
confuse these two functions.


\membersection{::wxSetWorkingDirectory}\label{wxsetworkingdirectory}

\func{bool}{wxSetWorkingDirectory}{\param{const wxString\& }{dir}}

Sets the current working directory, returning true if the operation succeeded.
Under MS Windows, the current drive is also changed if {\it dir} contains a drive specification.


\membersection{::wxSplit}\label{wxsplit}

\func{wxArrayString}{wxSplit}{\param{const wxString\&}{ str}, \param{const wxChar}{ sep}, \param{const wxChar}{ escape = '\\'}}

Splits the given \helpref{wxString}{wxstring} object using the separator \arg{sep} and returns the
result as a \helpref{wxArrayString}{wxarraystring}.

If the \arg{escape} character is non-\NULL, then the occurrences of \arg{sep} immediately prefixed
with \arg{escape} are not considered as separators.

Note that empty tokens will be generated if there are two or more adjacent separators.

\wxheading{See also}

\helpref{wxJoin}{wxjoin}

\wxheading{Include files}

<wx/arrstr.h>


\membersection{::wxSplitPath}\label{wxsplitfunction}

\func{void}{wxSplitPath}{\param{const wxString\&}{ fullname}, \param{wxString *}{ path}, \param{wxString *}{ name}, \param{wxString *}{ ext}}

{\bf NB:} This function is obsolete, please use
\helpref{wxFileName::SplitPath}{wxfilenamesplitpath} instead.

This function splits a full file name into components: the path (including possible disk/drive
specification under Windows), the base name and the extension. Any of the output parameters
({\it path}, {\it name} or {\it ext}) may be NULL if you are not interested in the value of
a particular component.

wxSplitPath() will correctly handle filenames with both DOS and Unix path separators under
Windows, however it will not consider backslashes as path separators under Unix (where backslash
is a valid character in a filename).

On entry, {\it fullname} should be non-NULL (it may be empty though).

On return, {\it path} contains the file path (without the trailing separator), {\it name}
contains the file name and {\it ext} contains the file extension without leading dot. All
three of them may be empty if the corresponding component is. The old contents of the
strings pointed to by these parameters will be overwritten in any case (if the pointers
are not NULL).


\membersection{::wxTransferFileToStream}\label{wxtransferfiletostream}

\func{bool}{wxTransferFileToStream}{\param{const wxString\& }{filename}, \param{ostream\& }{stream}}

Copies the given file to {\it stream}. Useful when converting an old application to
use streams (within the document/view framework, for example).

\wxheading{Include files}

<wx/docview.h>


\membersection{::wxTransferStreamToFile}\label{wxtransferstreamtofile}

\func{bool}{wxTransferStreamToFile}{\param{istream\& }{stream} \param{const wxString\& }{filename}}

Copies the given stream to the file {\it filename}. Useful when converting an old application to
use streams (within the document/view framework, for example).

\wxheading{Include files}

<wx/docview.h>



\section{Network, user and OS functions}\label{networkfunctions}

The functions in this section are used to retrieve information about the
current computer and/or user characteristics.


\membersection{::wxGetEmailAddress}\label{wxgetemailaddress}

\func{wxString}{wxGetEmailAddress}{\void}

\func{bool}{wxGetEmailAddress}{\param{char * }{buf}, \param{int }{sz}}

Copies the user's email address into the supplied buffer, by
concatenating the values returned by \helpref{wxGetFullHostName}{wxgetfullhostname}\rtfsp
and \helpref{wxGetUserId}{wxgetuserid}.

Returns true if successful, false otherwise.

\wxheading{Include files}

<wx/utils.h>


\membersection{::wxGetFreeMemory}\label{wxgetfreememory}

\func{wxMemorySize}{wxGetFreeMemory}{\void}

Returns the amount of free memory in bytes under environments which
support it, and -1 if not supported or failed to perform measurement.

\wxheading{Include files}

<wx/utils.h>


\membersection{::wxGetFullHostName}\label{wxgetfullhostname}

\func{wxString}{wxGetFullHostName}{\void}

Returns the FQDN (fully qualified domain host name) or an empty string on
error.

\wxheading{See also}

\helpref{wxGetHostName}{wxgethostname}

\wxheading{Include files}

<wx/utils.h>


\membersection{::wxGetHomeDir}\label{wxgethomedir}

\func{wxString}{wxGetHomeDir}{\void}

Return the (current) user's home directory.

\wxheading{See also}

\helpref{wxGetUserHome}{wxgetuserhome}\\
\helpref{wxStandardPaths}{wxstandardpaths}

\wxheading{Include files}

<wx/utils.h>


\membersection{::wxGetHostName}\label{wxgethostname}

\func{wxString}{wxGetHostName}{\void}

\func{bool}{wxGetHostName}{\param{char * }{buf}, \param{int }{sz}}

Copies the current host machine's name into the supplied buffer. Please note
that the returned name is {\it not} fully qualified, i.e. it does not include
the domain name.

Under Windows or NT, this function first looks in the environment
variable SYSTEM\_NAME; if this is not found, the entry {\bf HostName}\rtfsp
in the {\bf wxWidgets} section of the WIN.INI file is tried.

The first variant of this function returns the hostname if successful or an
empty string otherwise. The second (deprecated) function returns true
if successful, false otherwise.

\wxheading{See also}

\helpref{wxGetFullHostName}{wxgetfullhostname}

\wxheading{Include files}

<wx/utils.h>


\membersection{::wxGetOsDescription}\label{wxgetosdescription}

\func{wxString}{wxGetOsDescription}{\void}

Returns the string containing the description of the current platform in a
user-readable form. For example, this function may return strings like
{\tt Windows NT Version 4.0} or {\tt Linux 2.2.2 i386}.

\wxheading{See also}

\helpref{::wxGetOsVersion}{wxgetosversion}

\wxheading{Include files}

<wx/utils.h>


\membersection{::wxGetOsVersion}\label{wxgetosversion}

\func{wxOperatingSystemId}{wxGetOsVersion}{\param{int *}{major = NULL}, \param{int *}{minor = NULL}}

Gets the version and the operating system ID for currently running OS.
See \helpref{wxPlatformInfo}{wxplatforminfo} for more details about wxOperatingSystemId.

\wxheading{See also}

\helpref{::wxGetOsDescription}{wxgetosdescription},
\helpref{wxPlatformInfo}{wxplatforminfo}

\wxheading{Include files}

<wx/utils.h>


\membersection{::wxIsPlatformLittleEndian}\label{wxisplatformlittleendian}

\func{bool}{wxIsPlatformLittleEndian}{\void}

Returns \true if the current platform is little endian (instead of big endian).
The check is performed at run-time.

\wxheading{See also}

\helpref{Byte order macros}{byteordermacros}

\wxheading{Include files}

<wx/utils.h>


\membersection{::wxIsPlatform64Bit}\label{wxisplatform64bit}

\func{bool}{wxIsPlatform64Bit}{\void}

Returns \true if the operating system the program is running under is 64 bit.
The check is performed at run-time and may differ from the value available at
compile-time (at compile-time you can just check if {\tt sizeof(void*)==8})
since the program could be running in emulation mode or in a mixed 32/64 bit system
(bi-architecture operating system).

Very important: this function is not 100\% reliable on some systems given the fact
that there isn't always a standard way to do a reliable check on the OS architecture.

\wxheading{Include files}

<wx/utils.h>


\membersection{::wxGetUserHome}\label{wxgetuserhome}

\func{const wxChar *}{wxGetUserHome}{\param{const wxString\& }{user = ""}}

Returns the home directory for the given user. If the username is empty
(default value), this function behaves like
\helpref{wxGetHomeDir}{wxgethomedir}.

\wxheading{Include files}

<wx/utils.h>


\membersection{::wxGetUserId}\label{wxgetuserid}

\func{wxString}{wxGetUserId}{\void}

\func{bool}{wxGetUserId}{\param{char * }{buf}, \param{int }{sz}}

This function returns the "user id" also known as "login name" under Unix i.e.
something like "jsmith". It uniquely identifies the current user (on this system).

Under Windows or NT, this function first looks in the environment
variables USER and LOGNAME; if neither of these is found, the entry {\bf UserId}\rtfsp
in the {\bf wxWidgets} section of the WIN.INI file is tried.

The first variant of this function returns the login name if successful or an
empty string otherwise. The second (deprecated) function returns true
if successful, false otherwise.

\wxheading{See also}

\helpref{wxGetUserName}{wxgetusername}

\wxheading{Include files}

<wx/utils.h>


\membersection{::wxGetUserName}\label{wxgetusername}

\func{wxString}{wxGetUserName}{\void}

\func{bool}{wxGetUserName}{\param{char * }{buf}, \param{int }{sz}}

This function returns the full user name (something like "Mr. John Smith").

Under Windows or NT, this function looks for the entry {\bf UserName}\rtfsp
in the {\bf wxWidgets} section of the WIN.INI file. If PenWindows
is running, the entry {\bf Current} in the section {\bf User} of
the PENWIN.INI file is used.

The first variant of this function returns the user name if successful or an
empty string otherwise. The second (deprecated) function returns \true
if successful, \false otherwise.

\wxheading{See also}

\helpref{wxGetUserId}{wxgetuserid}

\wxheading{Include files}

<wx/utils.h>



\section{String functions}\label{stringfunctions}

\membersection{::wxGetTranslation}\label{wxgettranslation}

\func{const wxString\& }{wxGetTranslation}{\param{const wxString\& }{str},
  \param{const wxString\&  }{domain = wxEmptyString}}

\func{const wxString\& }{wxGetTranslation}{\param{const wxString\& }{str}, \param{const wxString\& }{strPlural}, \param{size\_t }{n},
  \param{const wxString\&  }{domain = wxEmptyString}}

This function returns the translation of string {\it str} in the current
\helpref{locale}{wxlocale}. If the string is not found in any of the loaded
message catalogs (see \helpref{internationalization overview}{internationalization}), the
original string is returned. In debug build, an error message is logged -- this
should help to find the strings which were not yet translated.  If
{\it domain} is specified then only that domain/catalog is searched
for a matching string.  As this function
is used very often, an alternative (and also common in Unix world) syntax is
provided: the \helpref{\_()}{underscore} macro is defined to do the same thing
as wxGetTranslation.

The second form is used when retrieving translation of string that has
different singular and plural form in English or different plural forms in some
other language. It takes two extra arguments: as above, \arg{str}
parameter must contain the singular form of the string to be converted and
is used as the key for the search in the catalog. The \arg{strPlural} parameter
is the plural form (in English). The parameter \arg{n} is used to determine the
plural form.  If no message catalog is found \arg{str} is returned if `n == 1',
otherwise \arg{strPlural}.

See \urlref{GNU gettext manual}{http://www.gnu.org/manual/gettext/html\_chapter/gettext\_10.html\#SEC150}
for additional information on plural forms handling. For a shorter alternative
see the \helpref{wxPLURAL()}{wxplural} macro.

Both versions call \helpref{wxLocale::GetString}{wxlocalegetstring}.

Note that this function is not suitable for literal strings in Unicode
builds, since the literal strings must be enclosed into
\helpref{\_T()}{underscoret} or \helpref{wxT}{wxt} macro which makes them
unrecognised by \texttt{xgettext}, and so they are not extracted to the message
catalog. Instead, use the \helpref{\_()}{underscore} and
\helpref{wxPLURAL}{wxplural} macro for all literal strings.


\membersection{::wxIsEmpty}\label{wxisempty}

\func{bool}{wxIsEmpty}{\param{const char *}{ p}}

Returns \true if the pointer is either {\tt NULL} or points to an empty
string, \false otherwise.


\membersection{::wxStrcmp}\label{wxstrcmp}

\func{int}{wxStrcmp}{\param{const char *}{p1}, \param{const char *}{p2}}

Returns a negative value, 0, or positive value if {\it p1} is less than, equal
to or greater than {\it p2}. The comparison is case-sensitive.

This function complements the standard C function {\it stricmp()} which performs
case-insensitive comparison.


\membersection{::wxStricmp}\label{wxstricmp}

\func{int}{wxStricmp}{\param{const char *}{p1}, \param{const char *}{p2}}

Returns a negative value, 0, or positive value if {\it p1} is less than, equal
to or greater than {\it p2}. The comparison is case-insensitive.

This function complements the standard C function {\it strcmp()} which performs
case-sensitive comparison.


\membersection{::wxStringEq}\label{wxstringeq}

\func{bool}{wxStringEq}{\param{const wxString\& }{s1}, \param{const wxString\& }{s2}}

{\bf NB:} This function is obsolete, use \helpref{wxString}{wxstring} instead.

A macro defined as:

\begin{verbatim}
#define wxStringEq(s1, s2) (s1 && s2 && (strcmp(s1, s2) == 0))
\end{verbatim}


\membersection{::wxStringMatch}\label{wxstringmatch}

\func{bool}{wxStringMatch}{\param{const wxString\& }{s1}, \param{const wxString\& }{s2},\\
  \param{bool}{ subString = true}, \param{bool}{ exact = false}}

{\bf NB:} This function is obsolete, use \helpref{wxString::Find}{wxstringfind} instead.

Returns \true if the substring {\it s1} is found within {\it s2},
ignoring case if {\it exact} is false. If {\it subString} is \false,
no substring matching is done.


\membersection{::wxStringTokenize}\label{wxstringtokenize}

\func{wxArrayString}{wxStringTokenize}{\param{const wxString\& }{str},\\
  \param{const wxString\& }{delims = wxDEFAULT\_DELIMITERS},\\
  \param{wxStringTokenizerMode }{mode = wxTOKEN\_DEFAULT}}

This is a convenience function wrapping
\helpref{wxStringTokenizer}{wxstringtokenizer} which simply returns all tokens
found in the given \arg{str} in an array.

Please see
\helpref{wxStringTokenizer::wxStringTokenizer}{wxstringtokenizerwxstringtokenizer}
for the description of the other parameters.


\membersection{::wxStrlen}\label{wxstrlen}

\func{size\_t}{wxStrlen}{\param{const char *}{ p}}

This is a safe version of standard function {\it strlen()}: it does exactly the
same thing (i.e. returns the length of the string) except that it returns 0 if
{\it p} is the {\tt NULL} pointer.


\membersection{::wxSnprintf}\label{wxsnprintf}

\func{int}{wxSnprintf}{\param{wxChar *}{buf}, \param{size\_t }{len}, \param{const wxChar *}{format}, \param{}{...}}

This function replaces the dangerous standard function {\tt sprintf()} and is
like {\tt snprintf()} available on some platforms. The only difference with
sprintf() is that an additional argument - buffer size - is taken and the
buffer is never overflowed.

Returns the number of characters copied to the buffer or -1 if there is not
enough space.

\wxheading{See also}

\helpref{wxVsnprintf}{wxvsnprintf}, \helpref{wxString::Printf}{wxstringprintf}


\membersection{wxT}\label{wxt}

\func{wxChar}{wxT}{\param{char }{ch}}

\func{const wxChar *}{wxT}{\param{const char *}{s}}

wxT() is a macro which can be used with character and string literals (in other
words, {\tt 'x'} or {\tt "foo"}) to automatically convert them to Unicode in
Unicode build configuration. Please see the
\helpref{Unicode overview}{unicode} for more information.

This macro is simply returns the value passed to it without changes in ASCII
build. In fact, its definition is:
\begin{verbatim}
#ifdef UNICODE
#define wxT(x) L ## x
#else // !Unicode
#define wxT(x) x
#endif
\end{verbatim}


\membersection{wxTRANSLATE}\label{wxtranslate}

\func{const wxChar *}{wxTRANSLATE}{\param{const char *}{s}}

This macro doesn't do anything in the program code -- it simply expands to the
value of its argument.

However it does have a purpose which is to mark the literal strings for the
extraction into the message catalog created by {\tt xgettext} program. Usually
this is achieved using \helpref{\_()}{underscore} but that macro not only marks
the string for extraction but also expands into a
\helpref{wxGetTranslation}{wxgettranslation} function call which means that it
cannot be used in some situations, notably for static array
initialization.

Here is an example which should make it more clear: suppose that you have a
static array of strings containing the weekday names and which have to be
translated (note that it is a bad example, really, as
\helpref{wxDateTime}{wxdatetime} already can be used to get the localized week
day names already). If you write

\begin{verbatim}
static const char * const weekdays[] = { _("Mon"), ..., _("Sun") };
...
// use weekdays[n] as usual
\end{verbatim}

the code wouldn't compile because the function calls are forbidden in the array
initializer. So instead you should do

\begin{verbatim}
static const char * const weekdays[] = { wxTRANSLATE("Mon"), ..., wxTRANSLATE("Sun") };
...
// use wxGetTranslation(weekdays[n])
\end{verbatim}

here.

Note that although the code {\bf would} compile if you simply omit
wxTRANSLATE() in the above, it wouldn't work as expected because there would be
no translations for the weekday names in the program message catalog and
wxGetTranslation wouldn't find them.


\membersection{::wxVsnprintf}\label{wxvsnprintf}

\func{int}{wxVsnprintf}{\param{wxChar *}{buf}, \param{size\_t }{len}, \param{const wxChar *}{format}, \param{va\_list }{argPtr}}

The same as \helpref{wxSnprintf}{wxsnprintf} but takes a {\tt va\_list }
argument instead of arbitrary number of parameters.

Note that if \texttt{wxUSE\_PRINTF\_POS\_PARAMS} is set to 1, then this function supports
positional arguments (see \helpref{wxString::Printf}{wxstringprintf} for more information).
However other functions of the same family (wxPrintf, wxSprintf, wxFprintf, wxVfprintf,
wxVfprintf, wxVprintf, wxVsprintf) currently do not to support positional parameters
even when \texttt{wxUSE\_PRINTF\_POS\_PARAMS} is 1.

\wxheading{See also}

\helpref{wxSnprintf}{wxsnprintf}, \helpref{wxString::PrintfV}{wxstringprintfv}



\membersection{\_}\label{underscore}

\func{const wxString\&}{\_}{\param{const wxString\&}{s}}

This macro expands into a call to \helpref{wxGetTranslation}{wxgettranslation}
function, so it marks the message for the extraction by {\tt xgettext} just as
\helpref{wxTRANSLATE}{wxtranslate} does, but also returns the translation of
the string for the current locale during execution.

Don't confuse this macro with \helpref{\_T()}{underscoret}!


\membersection{wxPLURAL}\label{wxplural}

\func{const wxString\&}{wxPLURAL}{\param{const wxString\&}{sing}, \param{const wxString\&}{plur}, \param{size\_t}{n}}

This macro is identical to \helpref{\_()}{underscore} but for the plural variant
of \helpref{wxGetTranslation}{wxgettranslation}.


\membersection{\_T}\label{underscoret}

\func{wxChar}{\_T}{\param{char }{ch}}

\func{const wxChar *}{\_T}{\param{const wxChar }{ch}}

This macro is exactly the same as \helpref{wxT}{wxt} and is defined in
wxWidgets simply because it may be more intuitive for Windows programmers as
the standard Win32 headers also define it (as well as yet another name for the
same macro which is {\tt \_TEXT()}).

Don't confuse this macro with \helpref{\_()}{underscore}!



\section{Dialog functions}\label{dialogfunctions}

Below are a number of convenience functions for getting input from the
user or displaying messages. Note that in these functions the last three
parameters are optional. However, it is recommended to pass a parent frame
parameter, or (in MS Windows or Motif) the wrong window frame may be brought to
the front when the dialog box is popped up.


\membersection{::wxAboutBox}\label{wxaboutbox}

\func{void}{wxAboutBox}{\param{const wxAboutDialogInfo\& }{info}}

This function shows the standard about dialog containing the information
specified in \arg{info}. If the current platform has a native about dialog
which is capable of showing all the fields in \arg{info}, the native dialog is
used, otherwise the function falls back to the generic wxWidgets version of the
dialog, i.e. does the same thing as \helpref{wxGenericAboutBox()}{wxgenericaboutbox}.

Here is an example of how this function may be used:
\begin{verbatim}
void MyFrame::ShowSimpleAboutDialog(wxCommandEvent& WXUNUSED(event))
{
    wxAboutDialogInfo info;
    info.SetName(_("My Program"));
    info.SetVersion(_("1.2.3 Beta"));
    info.SetDescription(_("This program does something great."));
    info.SetCopyright(_T("(C) 2007 Me <my@email.addre.ss>"));

    wxAboutBox(info);
}
\end{verbatim}

Please see the \helpref{dialogs sample}{sampledialogs} for more examples of
using this function and \helpref{wxAboutDialogInfo}{wxaboutdialoginfo} for the
description of the information which can be shown in the about dialog.

\wxheading{Include files}

<wx/aboutdlg.h>


\membersection{::wxBeginBusyCursor}\label{wxbeginbusycursor}

\func{void}{wxBeginBusyCursor}{\param{wxCursor *}{cursor = wxHOURGLASS\_CURSOR}}

Changes the cursor to the given cursor for all windows in the application.
Use \helpref{wxEndBusyCursor}{wxendbusycursor} to revert the cursor back
to its previous state. These two calls can be nested, and a counter
ensures that only the outer calls take effect.

See also \helpref{wxIsBusy}{wxisbusy}, \helpref{wxBusyCursor}{wxbusycursor}.

\wxheading{Include files}

<wx/utils.h>


\membersection{::wxBell}\label{wxbell}

\func{void}{wxBell}{\void}

Ring the system bell.

\wxheading{Include files}

<wx/utils.h>


\membersection{::wxCreateFileTipProvider}\label{wxcreatefiletipprovider}

\func{wxTipProvider *}{wxCreateFileTipProvider}{\param{const wxString\& }{filename},
 \param{size\_t }{currentTip}}

This function creates a \helpref{wxTipProvider}{wxtipprovider} which may be
used with \helpref{wxShowTip}{wxshowtip}.

\docparam{filename}{The name of the file containing the tips, one per line}
\docparam{currentTip}{The index of the first tip to show - normally this index
is remembered between the 2 program runs.}

\wxheading{See also}

\helpref{Tips overview}{tipsoverview}

\wxheading{Include files}

<wx/tipdlg.h>


\membersection{::wxDirSelector}\label{wxdirselector}

\func{wxString}{wxDirSelector}{\param{const wxString\& }{message = wxDirSelectorPromptStr},\\
 \param{const wxString\& }{default\_path = ""},\\
 \param{long }{style = 0}, \param{const wxPoint\& }{pos = wxDefaultPosition},\\
 \param{wxWindow *}{parent = NULL}}

Pops up a directory selector dialog. The arguments have the same meaning as
those of wxDirDialog::wxDirDialog(). The message is displayed at the top,
and the default\_path, if specified, is set as the initial selection.

The application must check for an empty return value (if the user pressed
Cancel). For example:

\begin{verbatim}
const wxString& dir = wxDirSelector("Choose a folder");
if ( !dir.empty() )
{
  ...
}
\end{verbatim}

\wxheading{Include files}

<wx/dirdlg.h>


\membersection{::wxFileSelector}\label{wxfileselector}

\func{wxString}{wxFileSelector}{\param{const wxString\& }{message}, \param{const wxString\& }{default\_path = ""},\\
 \param{const wxString\& }{default\_filename = ""}, \param{const wxString\& }{default\_extension = ""},\\
 \param{const wxString\& }{wildcard = "*.*"}, \param{int }{flags = 0}, \param{wxWindow *}{parent = NULL},\\
 \param{int}{ x = -1}, \param{int}{ y = -1}}

Pops up a file selector box. In Windows, this is the common file selector
dialog. In X, this is a file selector box with the same functionality.
The path and filename are distinct elements of a full file pathname.
If path is empty, the current directory will be used. If filename is empty,
no default filename will be supplied. The wildcard determines what files
are displayed in the file selector, and file extension supplies a type
extension for the required filename. Flags may be a combination of wxFD\_OPEN,
wxFD\_SAVE, wxFD\_OVERWRITE\_PROMPT or wxFD\_FILE\_MUST\_EXIST. Note that wxFD\_MULTIPLE
can only be used with \helpref{wxFileDialog}{wxfiledialog} and not here as this
function only returns a single file name.

Both the Unix and Windows versions implement a wildcard filter. Typing a
filename containing wildcards (*, ?) in the filename text item, and
clicking on Ok, will result in only those files matching the pattern being
displayed.

The wildcard may be a specification for multiple types of file
with a description for each, such as:

\begin{verbatim}
 "BMP files (*.bmp)|*.bmp|GIF files (*.gif)|*.gif"
\end{verbatim}

The application must check for an empty return value (the user pressed
Cancel). For example:

\begin{verbatim}
wxString filename = wxFileSelector("Choose a file to open");
if ( !filename.empty() )
{
    // work with the file
    ...
}
//else: cancelled by user
\end{verbatim}

\wxheading{Include files}

<wx/filedlg.h>


\membersection{::wxEndBusyCursor}\label{wxendbusycursor}

\func{void}{wxEndBusyCursor}{\void}

Changes the cursor back to the original cursor, for all windows in the application.
Use with \helpref{wxBeginBusyCursor}{wxbeginbusycursor}.

See also \helpref{wxIsBusy}{wxisbusy}, \helpref{wxBusyCursor}{wxbusycursor}.

\wxheading{Include files}

<wx/utils.h>


\membersection{::wxGenericAboutBox}\label{wxgenericaboutbox}

\func{void}{wxGenericAboutBox}{\param{const wxAboutDialogInfo\& }{info}}

This function does the same thing as \helpref{wxAboutBox}{wxaboutbox} except
that it always uses the generic wxWidgets version of the dialog instead of the
native one. This is mainly useful if you need to customize the dialog by e.g.
adding custom controls to it (customizing the native dialog is not currently
supported).

See the \helpref{dialogs sample}{sampledialogs} for an example of about dialog
customization.

\wxheading{See also}

\helpref{wxAboutDialogInfo}{wxaboutdialoginfo}

\wxheading{Include files}

<wx/aboutdlg.h>\\
<wx/generic/aboutdlgg.h>


\membersection{::wxGetColourFromUser}\label{wxgetcolourfromuser}

\func{wxColour}{wxGetColourFromUser}{\param{wxWindow *}{parent}, \param{const wxColour\& }{colInit}, \param{const wxString\& }{caption = wxEmptyString}}

Shows the colour selection dialog and returns the colour selected by user or
invalid colour (use \helpref{wxColour:IsOk}{wxcolourisok} to test whether a colour
is valid) if the dialog was cancelled.

\wxheading{Parameters}

\docparam{parent}{The parent window for the colour selection dialog}

\docparam{colInit}{If given, this will be the colour initially selected in the dialog.}

\docparam{caption}{If given, this will be used for the dialog caption.}

\wxheading{Include files}

<wx/colordlg.h>


\membersection{::wxGetFontFromUser}\label{wxgetfontfromuser}

\func{wxFont}{wxGetFontFromUser}{\param{wxWindow *}{parent}, \param{const wxFont\& }{fontInit}, \param{const wxString\& }{caption = wxEmptyString}}

Shows the font selection dialog and returns the font selected by user or
invalid font (use \helpref{wxFont:IsOk}{wxfontisok} to test whether a font
is valid) if the dialog was cancelled.

\wxheading{Parameters}

\docparam{parent}{The parent window for the font selection dialog}

\docparam{fontInit}{If given, this will be the font initially selected in the dialog.}

\docparam{caption}{If given, this will be used for the dialog caption.}

\wxheading{Include files}

<wx/fontdlg.h>



\membersection{::wxGetMultipleChoices}\label{wxgetmultiplechoices}

\func{size\_t}{wxGetMultipleChoices}{\\
 \param{wxArrayInt\& }{selections},\\
 \param{const wxString\& }{message},\\
 \param{const wxString\& }{caption},\\
 \param{const wxArrayString\& }{aChoices},\\
 \param{wxWindow *}{parent = NULL},\\
 \param{int}{ x = -1}, \param{int}{ y = -1},\\
 \param{bool}{ centre = true},\\
 \param{int }{width=150}, \param{int }{height=200}}

\func{size\_t}{wxGetMultipleChoices}{\\
 \param{wxArrayInt\& }{selections},\\
 \param{const wxString\& }{message},\\
 \param{const wxString\& }{caption},\\
 \param{int}{ n}, \param{const wxString\& }{choices[]},\\
 \param{wxWindow *}{parent = NULL},\\
 \param{int}{ x = -1}, \param{int}{ y = -1},\\
 \param{bool}{ centre = true},\\
 \param{int }{width=150}, \param{int }{height=200}}

Pops up a dialog box containing a message, OK/Cancel buttons and a
multiple-selection listbox. The user may choose an arbitrary (including 0)
number of items in the listbox whose indices will be returned in
{\it selection} array. The initial contents of this array will be used to
select the items when the dialog is shown.

You may pass the list of strings to choose from either using {\it choices}
which is an array of {\it n} strings for the listbox or by using a single
{\it aChoices} parameter of type \helpref{wxArrayString}{wxarraystring}.

If {\it centre} is true, the message text (which may include new line
characters) is centred; if false, the message is left-justified.

\wxheading{Include files}

<wx/choicdlg.h>

\perlnote{In wxPerl there is just an array reference in place of {\tt n}
and {\tt choices}, and no {\tt selections} parameter; the function
returns an array containing the user selections.}


\membersection{::wxGetNumberFromUser}\label{wxgetnumberfromuser}

\func{long}{wxGetNumberFromUser}{
 \param{const wxString\& }{message},
 \param{const wxString\& }{prompt},
 \param{const wxString\& }{caption},
 \param{long }{value},
 \param{long }{min = 0},
 \param{long }{max = 100},
 \param{wxWindow *}{parent = NULL},
 \param{const wxPoint\& }{pos = wxDefaultPosition}}

Shows a dialog asking the user for numeric input. The dialogs title is set to
{\it caption}, it contains a (possibly) multiline {\it message} above the
single line {\it prompt} and the zone for entering the number.

The number entered must be in the range {\it min}..{\it max} (both of which
should be positive) and {\it value} is the initial value of it. If the user
enters an invalid value or cancels the dialog, the function will return -1.

Dialog is centered on its {\it parent} unless an explicit position is given in
{\it pos}.

\wxheading{Include files}

<wx/numdlg.h>


\membersection{::wxGetPasswordFromUser}\label{wxgetpasswordfromuser}

\func{wxString}{wxGetPasswordFromUser}{\param{const wxString\& }{message}, \param{const wxString\& }{caption = ``Input text"},\\
 \param{const wxString\& }{default\_value = ``"}, \param{wxWindow *}{parent = NULL},\\
 \param{int}{ x = wxDefaultCoord}, \param{int}{ y = wxDefaultCoord}, \param{bool}{ centre = true}}

Similar to \helpref{wxGetTextFromUser}{wxgettextfromuser} but the text entered
in the dialog is not shown on screen but replaced with stars. This is intended
to be used for entering passwords as the function name implies.

\wxheading{Include files}

<wx/textdlg.h>


\membersection{::wxGetTextFromUser}\label{wxgettextfromuser}

\func{wxString}{wxGetTextFromUser}{\param{const wxString\& }{message}, \param{const wxString\& }{caption = ``Input text"},\\
 \param{const wxString\& }{default\_value = ``"}, \param{wxWindow *}{parent = NULL},\\
 \param{int}{ x = wxDefaultCoord}, \param{int}{ y = wxDefaultCoord}, \param{bool}{ centre = true}}

Pop up a dialog box with title set to {\it caption}, {\it message}, and a
\rtfsp{\it default\_value}.  The user may type in text and press OK to return this text,
or press Cancel to return the empty string.

If {\it centre} is true, the message text (which may include new line characters)
is centred; if false, the message is left-justified.

\wxheading{Include files}

<wx/textdlg.h>


\membersection{::wxGetSingleChoice}\label{wxgetsinglechoice}

\func{wxString}{wxGetSingleChoice}{\param{const wxString\& }{message},\\
 \param{const wxString\& }{caption},\\
 \param{const wxArrayString\& }{aChoices},\\
 \param{wxWindow *}{parent = NULL},\\
 \param{int}{ x = -1}, \param{int}{ y = -1},\\
 \param{bool}{ centre = true},\\
 \param{int }{width=150}, \param{int }{height=200}}

\func{wxString}{wxGetSingleChoice}{\param{const wxString\& }{message},\\
 \param{const wxString\& }{caption},\\
 \param{int}{ n}, \param{const wxString\& }{choices[]},\\
 \param{wxWindow *}{parent = NULL},\\
 \param{int}{ x = -1}, \param{int}{ y = -1},\\
 \param{bool}{ centre = true},\\
 \param{int }{width=150}, \param{int }{height=200}}

Pops up a dialog box containing a message, OK/Cancel buttons and a
single-selection listbox. The user may choose an item and press OK to return a
string or Cancel to return the empty string. Use
\helpref{wxGetSingleChoiceIndex}{wxgetsinglechoiceindex} if empty string is a
valid choice and if you want to be able to detect pressing Cancel reliably.

You may pass the list of strings to choose from either using {\it choices}
which is an array of {\it n} strings for the listbox or by using a single
{\it aChoices} parameter of type \helpref{wxArrayString}{wxarraystring}.

If {\it centre} is true, the message text (which may include new line
characters) is centred; if false, the message is left-justified.

\wxheading{Include files}

<wx/choicdlg.h>

\perlnote{In wxPerl there is just an array reference in place of {\tt n}
and {\tt choices}.}


\membersection{::wxGetSingleChoiceIndex}\label{wxgetsinglechoiceindex}

\func{int}{wxGetSingleChoiceIndex}{\param{const wxString\& }{message},\\
 \param{const wxString\& }{caption},\\
 \param{const wxArrayString\& }{aChoices},\\
 \param{wxWindow *}{parent = NULL}, \param{int}{ x = -1}, \param{int}{ y = -1},\\
 \param{bool}{ centre = true}, \param{int }{width=150}, \param{int }{height=200}}

\func{int}{wxGetSingleChoiceIndex}{\param{const wxString\& }{message},\\
 \param{const wxString\& }{caption},\\
 \param{int}{ n}, \param{const wxString\& }{choices[]},\\
 \param{wxWindow *}{parent = NULL}, \param{int}{ x = -1}, \param{int}{ y = -1},\\
 \param{bool}{ centre = true}, \param{int }{width=150}, \param{int }{height=200}}

As {\bf wxGetSingleChoice} but returns the index representing the selected
string. If the user pressed cancel, -1 is returned.

\wxheading{Include files}

<wx/choicdlg.h>

\perlnote{In wxPerl there is just an array reference in place of {\tt n}
and {\tt choices}.}


\membersection{::wxGetSingleChoiceData}\label{wxgetsinglechoicedata}

\func{wxString}{wxGetSingleChoiceData}{\param{const wxString\& }{message},\\
 \param{const wxString\& }{caption},\\
 \param{const wxArrayString\& }{aChoices},\\
 \param{const wxString\& }{client\_data[]},\\
 \param{wxWindow *}{parent = NULL},\\
 \param{int}{ x = -1}, \param{int}{ y = -1},\\
 \param{bool}{ centre = true}, \param{int }{width=150}, \param{int }{height=200}}

\func{wxString}{wxGetSingleChoiceData}{\param{const wxString\& }{message},\\
 \param{const wxString\& }{caption},\\
 \param{int}{ n}, \param{const wxString\& }{choices[]},\\
 \param{const wxString\& }{client\_data[]},\\
 \param{wxWindow *}{parent = NULL},\\
 \param{int}{ x = -1}, \param{int}{ y = -1},\\
 \param{bool}{ centre = true}, \param{int }{width=150}, \param{int }{height=200}}

As {\bf wxGetSingleChoice} but takes an array of client data pointers
corresponding to the strings, and returns one of these pointers or NULL if
Cancel was pressed. The {\it client\_data} array must have the same number of
elements as {\it choices} or {\it aChoices}!

\wxheading{Include files}

<wx/choicdlg.h>

\perlnote{In wxPerl there is just an array reference in place of {\tt n}
and {\tt choices}, and the client data array must have the
same length as the choices array.}


\membersection{::wxIsBusy}\label{wxisbusy}

\func{bool}{wxIsBusy}{\void}

Returns true if between two \helpref{wxBeginBusyCursor}{wxbeginbusycursor} and\rtfsp
\helpref{wxEndBusyCursor}{wxendbusycursor} calls.

See also \helpref{wxBusyCursor}{wxbusycursor}.

\wxheading{Include files}

<wx/utils.h>


\membersection{::wxMessageBox}\label{wxmessagebox}

\func{int}{wxMessageBox}{\param{const wxString\& }{message}, \param{const wxString\& }{caption = ``Message"}, \param{int}{ style = wxOK},\\
 \param{wxWindow *}{parent = NULL}, \param{int}{ x = -1}, \param{int}{ y = -1}}

General purpose message dialog.  {\it style} may be a bit list of the
following identifiers:

\begin{twocollist}\itemsep=0pt
\twocolitem{wxYES\_NO}{Puts Yes and No buttons on the message box. May be combined with
wxCANCEL.}
\twocolitem{wxCANCEL}{Puts a Cancel button on the message box. May only be combined with
wxYES\_NO or wxOK.}
\twocolitem{wxOK}{Puts an Ok button on the message box. May be combined with wxCANCEL.}
\twocolitem{wxICON\_EXCLAMATION}{Displays an exclamation mark symbol.}
\twocolitem{wxICON\_HAND}{Displays an error symbol.}
\twocolitem{wxICON\_ERROR}{Displays an error symbol - the same as wxICON\_HAND.}
\twocolitem{wxICON\_QUESTION}{Displays a question mark symbol.}
\twocolitem{wxICON\_INFORMATION}{Displays an information symbol.}
\end{twocollist}

The return value is one of: wxYES, wxNO, wxCANCEL, wxOK.

For example:

\begin{verbatim}
  ...
  int answer = wxMessageBox("Quit program?", "Confirm",
                            wxYES_NO | wxCANCEL, main_frame);
  if (answer == wxYES)
    main_frame->Close();
  ...
\end{verbatim}

{\it message} may contain newline characters, in which case the
message will be split into separate lines, to cater for large messages.

\wxheading{Include files}

<wx/msgdlg.h>


\membersection{::wxShowTip}\label{wxshowtip}

\func{bool}{wxShowTip}{\param{wxWindow *}{parent},
 \param{wxTipProvider *}{tipProvider},
 \param{bool }{showAtStartup = true}}

This function shows a "startup tip" to the user. The return value is the
state of the `Show tips at startup' checkbox.

\docparam{parent}{The parent window for the modal dialog}

\docparam{tipProvider}{An object which is used to get the text of the tips.
It may be created with the \helpref{wxCreateFileTipProvider}{wxcreatefiletipprovider} function.}

\docparam{showAtStartup}{Should be true if startup tips are shown, false
otherwise. This is used as the initial value for "Show tips at startup"
checkbox which is shown in the tips dialog.}

\wxheading{See also}

\helpref{Tips overview}{tipsoverview}

\wxheading{Include files}

<wx/tipdlg.h>




\section{Math functions}\label{mathfunctions}

\wxheading{Include files}

<wx/math.h>


\membersection{wxFinite}\label{wxfinite}

\func{int}{wxFinite}{\param{double }{x}}

Returns a non-zero value if {\it x} is neither infinite nor NaN (not a number),
returns 0 otherwise.


\membersection{wxIsNaN}\label{wxisnan}

\func{bool}{wxIsNaN}{\param{double }{x}}

Returns a non-zero value if {\it x} is NaN (not a number), returns 0
otherwise.




\section{GDI functions}\label{gdifunctions}

The following are relevant to the GDI (Graphics Device Interface).

\wxheading{Include files}

<wx/gdicmn.h>


\membersection{wxBITMAP}\label{wxbitmapmacro}

\func{}{wxBITMAP}{bitmapName}

This macro loads a bitmap from either application resources (on the platforms
for which they exist, i.e. Windows and OS2) or from an XPM file. It allows to
avoid using {\tt \#ifdef}s when creating bitmaps.

\wxheading{See also}

\helpref{Bitmaps and icons overview}{wxbitmapoverview},
\helpref{wxICON}{wxiconmacro}

\wxheading{Include files}

<wx/gdicmn.h>


\membersection{::wxClientDisplayRect}\label{wxclientdisplayrect}

\func{void}{wxClientDisplayRect}{\param{int *}{x}, \param{int *}{y},
\param{int *}{width}, \param{int *}{height}}

\func{wxRect}{wxGetClientDisplayRect}{\void}

Returns the dimensions of the work area on the display.  On Windows
this means the area not covered by the taskbar, etc.  Other platforms
are currently defaulting to the whole display until a way is found to
provide this info for all window managers, etc.


\membersection{::wxColourDisplay}\label{wxcolourdisplay}

\func{bool}{wxColourDisplay}{\void}

Returns true if the display is colour, false otherwise.


\membersection{::wxDisplayDepth}\label{wxdisplaydepth}

\func{int}{wxDisplayDepth}{\void}

Returns the depth of the display (a value of 1 denotes a monochrome display).


\membersection{::wxDisplaySize}\label{wxdisplaysize}

\func{void}{wxDisplaySize}{\param{int *}{width}, \param{int *}{height}}

\func{wxSize}{wxGetDisplaySize}{\void}

Returns the display size in pixels.


\membersection{::wxDisplaySizeMM}\label{wxdisplaysizemm}

\func{void}{wxDisplaySizeMM}{\param{int *}{width}, \param{int *}{height}}

\func{wxSize}{wxGetDisplaySizeMM}{\void}

Returns the display size in millimeters.


\membersection{::wxDROP\_ICON}\label{wxdropicon}

\func{wxIconOrCursor}{wxDROP\_ICON}{\param{const char *}{name}}

This macro creates either a cursor (MSW) or an icon (elsewhere) with the given
name. Under MSW, the cursor is loaded from the resource file and the icon is
loaded from XPM file under other platforms.

This macro should be used with
\helpref{wxDropSource constructor}{wxdropsourcewxdropsource}.

\wxheading{Include files}

<wx/dnd.h>


\membersection{wxICON}\label{wxiconmacro}

\func{}{wxICON}{iconName}

This macro loads an icon from either application resources (on the platforms
for which they exist, i.e. Windows and OS2) or from an XPM file. It allows to
avoid using {\tt \#ifdef}s when creating icons.

\wxheading{See also}

\helpref{Bitmaps and icons overview}{wxbitmapoverview},
\helpref{wxBITMAP}{wxbitmapmacro}

\wxheading{Include files}

<wx/gdicmn.h>


\membersection{::wxMakeMetafilePlaceable}\label{wxmakemetafileplaceable}

\func{bool}{wxMakeMetafilePlaceable}{\param{const wxString\& }{filename}, \param{int }{minX}, \param{int }{minY},
 \param{int }{maxX}, \param{int }{maxY}, \param{float }{scale=1.0}}

Given a filename for an existing, valid metafile (as constructed using \helpref{wxMetafileDC}{wxmetafiledc})
makes it into a placeable metafile by prepending a header containing the given
bounding box. The bounding box may be obtained from a device context after drawing
into it, using the functions wxDC::MinX, wxDC::MinY, wxDC::MaxX and wxDC::MaxY.

In addition to adding the placeable metafile header, this function adds
the equivalent of the following code to the start of the metafile data:

\begin{verbatim}
 SetMapMode(dc, MM_ANISOTROPIC);
 SetWindowOrg(dc, minX, minY);
 SetWindowExt(dc, maxX - minX, maxY - minY);
\end{verbatim}

This simulates the wxMM\_TEXT mapping mode, which wxWidgets assumes.

Placeable metafiles may be imported by many Windows applications, and can be
used in RTF (Rich Text Format) files.

{\it scale} allows the specification of scale for the metafile.

This function is only available under Windows.


\membersection{::wxSetCursor}\label{wxsetcursor}

\func{void}{wxSetCursor}{\param{wxCursor *}{cursor}}

Globally sets the cursor; only has an effect in Windows and GTK.
See also \helpref{wxCursor}{wxcursor}, \helpref{wxWindow::SetCursor}{wxwindowsetcursor}.



\section{Printer settings}\label{printersettings}

{\bf NB:} These routines are obsolete and should no longer be used!

The following functions are used to control PostScript printing. Under
Windows, PostScript output can only be sent to a file.

\wxheading{Include files}

<wx/dcps.h>


\membersection{::wxGetPrinterCommand}\label{wxgetprintercommand}

\func{wxString}{wxGetPrinterCommand}{\void}

Gets the printer command used to print a file. The default is {\tt lpr}.


\membersection{::wxGetPrinterFile}\label{wxgetprinterfile}

\func{wxString}{wxGetPrinterFile}{\void}

Gets the PostScript output filename.


\membersection{::wxGetPrinterMode}\label{wxgetprintermode}

\func{int}{wxGetPrinterMode}{\void}

Gets the printing mode controlling where output is sent (PS\_PREVIEW, PS\_FILE or PS\_PRINTER).
The default is PS\_PREVIEW.


\membersection{::wxGetPrinterOptions}\label{wxgetprinteroptions}

\func{wxString}{wxGetPrinterOptions}{\void}

Gets the additional options for the print command (e.g. specific printer). The default is nothing.


\membersection{::wxGetPrinterOrientation}\label{wxgetprinterorientation}

\func{int}{wxGetPrinterOrientation}{\void}

Gets the orientation (PS\_PORTRAIT or PS\_LANDSCAPE). The default is PS\_PORTRAIT.


\membersection{::wxGetPrinterPreviewCommand}\label{wxgetprinterpreviewcommand}

\func{wxString}{wxGetPrinterPreviewCommand}{\void}

Gets the command used to view a PostScript file. The default depends on the platform.


\membersection{::wxGetPrinterScaling}\label{wxgetprinterscaling}

\func{void}{wxGetPrinterScaling}{\param{float *}{x}, \param{float *}{y}}

Gets the scaling factor for PostScript output. The default is 1.0, 1.0.


\membersection{::wxGetPrinterTranslation}\label{wxgetprintertranslation}

\func{void}{wxGetPrinterTranslation}{\param{float *}{x}, \param{float *}{y}}

Gets the translation (from the top left corner) for PostScript output. The default is 0.0, 0.0.


\membersection{::wxSetPrinterCommand}\label{wxsetprintercommand}

\func{void}{wxSetPrinterCommand}{\param{const wxString\& }{command}}

Sets the printer command used to print a file. The default is {\tt lpr}.


\membersection{::wxSetPrinterFile}\label{wxsetprinterfile}

\func{void}{wxSetPrinterFile}{\param{const wxString\& }{filename}}

Sets the PostScript output filename.


\membersection{::wxSetPrinterMode}\label{wxsetprintermode}

\func{void}{wxSetPrinterMode}{\param{int }{mode}}

Sets the printing mode controlling where output is sent (PS\_PREVIEW, PS\_FILE or PS\_PRINTER).
The default is PS\_PREVIEW.


\membersection{::wxSetPrinterOptions}\label{wxsetprinteroptions}

\func{void}{wxSetPrinterOptions}{\param{const wxString\& }{options}}

Sets the additional options for the print command (e.g. specific printer). The default is nothing.


\membersection{::wxSetPrinterOrientation}\label{wxsetprinterorientation}

\func{void}{wxSetPrinterOrientation}{\param{int}{ orientation}}

Sets the orientation (PS\_PORTRAIT or PS\_LANDSCAPE). The default is PS\_PORTRAIT.


\membersection{::wxSetPrinterPreviewCommand}\label{wxsetprinterpreviewcommand}

\func{void}{wxSetPrinterPreviewCommand}{\param{const wxString\& }{command}}

Sets the command used to view a PostScript file. The default depends on the platform.


\membersection{::wxSetPrinterScaling}\label{wxsetprinterscaling}

\func{void}{wxSetPrinterScaling}{\param{float }{x}, \param{float }{y}}

Sets the scaling factor for PostScript output. The default is 1.0, 1.0.


\membersection{::wxSetPrinterTranslation}\label{wxsetprintertranslation}

\func{void}{wxSetPrinterTranslation}{\param{float }{x}, \param{float }{y}}

Sets the translation (from the top left corner) for PostScript output. The default is 0.0, 0.0.



\section{Clipboard functions}\label{clipsboard}

These clipboard functions are implemented for Windows only. The use of these functions
is deprecated and the code is no longer maintained. Use the \helpref{wxClipboard}{wxclipboard}
class instead.

\wxheading{Include files}

<wx/clipbrd.h>


\membersection{::wxClipboardOpen}\label{functionwxclipboardopen}

\func{bool}{wxClipboardOpen}{\void}

Returns true if this application has already opened the clipboard.


\membersection{::wxCloseClipboard}\label{wxcloseclipboard}

\func{bool}{wxCloseClipboard}{\void}

Closes the clipboard to allow other applications to use it.


\membersection{::wxEmptyClipboard}\label{wxemptyclipboard}

\func{bool}{wxEmptyClipboard}{\void}

Empties the clipboard.


\membersection{::wxEnumClipboardFormats}\label{wxenumclipboardformats}

\func{int}{wxEnumClipboardFormats}{\param{int}{ dataFormat}}

Enumerates the formats found in a list of available formats that belong
to the clipboard. Each call to this  function specifies a known
available format; the function returns the format that appears next in
the list.

{\it dataFormat} specifies a known format. If this parameter is zero,
the function returns the first format in the list.

The return value specifies the next known clipboard data format if the
function is successful. It is zero if the {\it dataFormat} parameter specifies
the last  format in the list of available formats, or if the clipboard
is not open.

Before it enumerates the formats function, an application must open the clipboard by using the
wxOpenClipboard function.


\membersection{::wxGetClipboardData}\label{wxgetclipboarddata}

\func{wxObject *}{wxGetClipboardData}{\param{int}{ dataFormat}}

Gets data from the clipboard.

{\it dataFormat} may be one of:

\begin{itemize}\itemsep=0pt
\item wxCF\_TEXT or wxCF\_OEMTEXT: returns a pointer to new memory containing a null-terminated text string.
\item wxCF\_BITMAP: returns a new wxBitmap.
\end{itemize}

The clipboard must have previously been opened for this call to succeed.


\membersection{::wxGetClipboardFormatName}\label{wxgetclipboardformatname}

\func{bool}{wxGetClipboardFormatName}{\param{int}{ dataFormat}, \param{const wxString\& }{formatName}, \param{int}{ maxCount}}

Gets the name of a registered clipboard format, and puts it into the buffer {\it formatName} which is of maximum
length {\it maxCount}. {\it dataFormat} must not specify a predefined clipboard format.


\membersection{::wxIsClipboardFormatAvailable}\label{wxisclipboardformatavailable}

\func{bool}{wxIsClipboardFormatAvailable}{\param{int}{ dataFormat}}

Returns true if the given data format is available on the clipboard.


\membersection{::wxOpenClipboard}\label{wxopenclipboard}

\func{bool}{wxOpenClipboard}{\void}

Opens the clipboard for passing data to it or getting data from it.


\membersection{::wxRegisterClipboardFormat}\label{wxregisterclipboardformat}

\func{int}{wxRegisterClipboardFormat}{\param{const wxString\& }{formatName}}

Registers the clipboard data format name and returns an identifier.


\membersection{::wxSetClipboardData}\label{wxsetclipboarddata}

\func{bool}{wxSetClipboardData}{\param{int}{ dataFormat}, \param{wxObject*}{ data}, \param{int}{ width}, \param{int}{ height}}

Passes data to the clipboard.

{\it dataFormat} may be one of:

\begin{itemize}\itemsep=0pt
\item wxCF\_TEXT or wxCF\_OEMTEXT: {\it data} is a null-terminated text string.
\item wxCF\_BITMAP: {\it data} is a wxBitmap.
\item wxCF\_DIB: {\it data} is a wxBitmap. The bitmap is converted to a DIB (device independent bitmap).
\item wxCF\_METAFILE: {\it data} is a wxMetafile. {\it width} and {\it height} are used to give recommended dimensions.
\end{itemize}

The clipboard must have previously been opened for this call to succeed.


\section{Miscellaneous functions}\label{miscellany}


\membersection{wxBase64Decode}\label{wxbase64decode}

\func{size\_t}{wxBase64Decode}{\param{void *}{dst}, \param{size\_t }{dstLen}, 
\param{const char * }{src}, \param{size\_t }{srcLen = wxNO\_LEN}, 
\param{wxBase64DecodeMode }{mode = wxBase64DecodeMode\_Strict}, 
\param{size\_t }{*posErr = \NULL}}

\func{wxMemoryBuffer}{wxBase64Decode}{\\
\param{const char * }{src}, \param{size\_t }{srcLen = wxNO\_LEN},\\
\param{wxBase64DecodeMode }{mode = wxBase64DecodeMode\_Strict},\\
\param{size\_t }{*posErr = \NULL}}

These function decode a Base64-encoded string. The first version is a raw
decoding function and decodes the data into the provided buffer \arg{dst} of
the given size \arg{dstLen}. An error is returned if the buffer is not large
enough -- that is not at least \helpref{wxBase64DecodedSize(srcLen)}{wxbase64decodedsize} 
bytes. The second version allocates memory internally and returns it as
\helpref{wxMemoryBuffer}{wxmemorybuffer} and is recommended for normal use.

The first version returns the number of bytes written to the buffer or the
necessary buffer size if \arg{dst} was \NULL or \texttt{wxCONV\_FAILED} on
error, e.g. if the output buffer is too small or invalid characters were
encountered in the input string. The second version returns a buffer with the
base64 decoded binary equivalent of the input string. In neither case is the
buffer NUL-terminated.

\wxheading{Parameters}

\docparam{dst}{Pointer to output buffer, may be \NULL to just compute the
necessary buffer size.}

\docparam{dstLen}{The size of the output buffer, ignored if \arg{dst} is
\NULL.}

\docparam{src}{The input string, must not be \NULL.}

\docparam{srcLen}{The length of the input string or special value
\texttt{wxNO\_LEN} if the string is \NUL-terminated and the length should be
computed by this function itself.}

\docparam{mode}{This parameter specifies the function behaviour when invalid
characters are encountered in input. By default, any such character stops the
decoding with error. If the mode is wxBase64DecodeMode\_SkipWS, then the white
space characters are silently skipped instead. And if it is
wxBase64DecodeMode\_Relaxed, then all invalid characters are skipped.}

\docparam{posErr}{If this pointer is non-\NULL and an error occurs during
decoding, it is filled with the index of the invalid character.}

\wxheading{Include files}

<wx/base64.h>


\membersection{wxBase64DecodedSize}\label{wxbase64decodedsize}

\func{size\_t}{wxBase64DecodedSize}{\param{size\_t }{srcLen}}

Returns the size of the buffer necessary to contain the data encoded in a
base64 string of length \arg{srcLen}. This can be useful for allocating a
buffer to be passed to \helpref{wxBase64Decode}{wxbase64decode}.


\membersection{wxBase64Encode}\label{wxbase64encode}

\func{size\_t}{wxBase64Encode}{\param{char *}{dst}, \param{size\_t }{dstLen}, 
\param{const void *}{src}, \param{size\_t }{srcLen}}

\func{wxString}{wxBase64Encode}{\param{const void *}{src}, \param{size\_t }{srcLen}}

\func{wxString}{wxBase64Encode}{\param{const wxMemoryBuffer\& }{buf}}

These functions encode the given data using base64. The first of them is the
raw encoding function writing the output string into provided buffer while the
other ones return the output as wxString. There is no error return for these
functions except for the first one which returns \texttt{wxCONV\_FAILED} if the
output buffer is too small. To allocate the buffer of the correct size, use 
\helpref{wxBase64EncodedSize}{wxbase64encodedsize} or call this function with 
\arg{dst} set to \NULL -- it will then return the necessary buffer size.

\wxheading{Parameters}

\docparam{dst}{The output buffer, may be \NULL to retrieve the needed buffer
size.}

\docparam{dstLen}{The output buffer size, ignored if \arg{dst} is \NULL.}

\docparam{src}{The input buffer, must not be \NULL.}

\docparam{srcLen}{The length of the input data.}

\wxheading{Include files}

<wx/base64.h>


\membersection{wxBase64EncodedSize}\label{wxbase64encodedsize}

\func{size\_t}{wxBase64EncodedSize}{\param{size\_t }{len}}

Returns the length of the string with base64 representation of a buffer of
specified size \arg{len}. This can be useful for allocating the buffer passed
to \helpref{wxBase64Encode}{wxbase64encode}.


\membersection{wxCONCAT}\label{wxconcat}

\func{}{wxCONCAT}{\param{}{x}, \param{}{y}}

This macro returns the concatenation of two tokens \arg{x} and \arg{y}.


\membersection{wxDYNLIB\_FUNCTION}\label{wxdynlibfunction}

\func{}{wxDYNLIB\_FUNCTION}{\param{}{type}, \param{}{name}, \param{}{dynlib}}

When loading a function from a DLL you always have to cast the returned
{\tt void *} pointer to the correct type and, even more annoyingly, you have to
repeat this type twice if you want to declare and define a function pointer all
in one line

This macro makes this slightly less painful by allowing you to specify the
type only once, as the first parameter, and creating a variable of this type
named after the function but with {\tt pfn} prefix and initialized with the
function \arg{name} from the \helpref{wxDynamicLibrary}{wxdynamiclibrary}
\arg{dynlib}.

\wxheading{Parameters}

\docparam{type}{the type of the function}

\docparam{name}{the name of the function to load, not a string (without quotes,
it is quoted automatically by the macro)}

\docparam{dynlib}{the library to load the function from}



\membersection{wxDEPRECATED}\label{wxdeprecated}

This macro can be used around a function declaration to generate warnings
indicating that this function is deprecated (i.e. obsolete and planned to be
removed in the future) when it is used. Only Visual C++ 7 and higher and g++
compilers currently support this functionality.

Example of use:
\begin{verbatim}
    // old function, use wxString version instead
    wxDEPRECATED( void wxGetSomething(char *buf, size_t len) );

    // ...
    wxString wxGetSomething();
\end{verbatim}


\membersection{wxDEPRECATED\_BUT\_USED\_INTERNALLY}\label{wxdeprecatedbutusedinternally}

This is a special version of \helpref{wxDEPRECATED}{wxdeprecated} macro which
only does something when the deprecated function is used from the code outside
wxWidgets itself but doesn't generate warnings when it is used from wxWidgets.
It is used with the virtual functions which are called by the library itself --
even if such function is deprecated the library still has to call it to ensure
that the existing code overriding it continues to work, but the use of this
macro ensures that a deprecation warning will be generated if this function is
used from the user code or, in case of Visual C++, even when it is simply
overridden.


\membersection{wxEXPLICIT}\label{wxexplicit}

{\tt wxEXPLICIT} is a macro which expands to the C++ {\tt explicit} keyword if
the compiler supports it or nothing otherwise. Thus, it can be used even in the
code which might have to be compiled with an old compiler without support for
this language feature but still take advantage of it when it is available.



\membersection{::wxGetKeyState}\label{wxgetkeystate}

\func{bool}{wxGetKeyState}{\param{wxKeyCode }{key}}

For normal keys, returns \true if the specified key is currently down.

For togglable keys (Caps Lock, Num Lock and Scroll Lock), returns
\true if the key is toggled such that its LED indicator is lit. There is
currently no way to test whether togglable keys are up or down.

Even though there are virtual key codes defined for mouse buttons, they
cannot be used with this function currently.

\wxheading{Include files}

<wx/utils.h>


\membersection{wxLL}\label{wxll}

\func{wxLongLong\_t}{wxLL}{\param{}{number}}

This macro is defined for the platforms with a native 64 bit integer type and
allows to define 64 bit compile time constants:

\begin{verbatim}
    #ifdef wxLongLong_t
        wxLongLong_t ll = wxLL(0x1234567890abcdef);
    #endif
\end{verbatim}

\wxheading{Include files}

<wx/longlong.h>

\wxheading{See also}

\helpref{wxULL}{wxull}, \helpref{wxLongLong}{wxlonglong}


\membersection{wxLongLongFmtSpec}\label{wxlonglongfmtspec}

This macro is defined to contain the {\tt printf()} format specifier using
which 64 bit integer numbers (i.e. those of type {\tt wxLongLong\_t}) can be
printed. Example of using it:

\begin{verbatim}
    #ifdef wxLongLong_t
        wxLongLong_t ll = wxLL(0x1234567890abcdef);
        printf("Long long = %" wxLongLongFmtSpec "x\n", ll);
    #endif
\end{verbatim}

\wxheading{See also}

\helpref{wxLL}{wxll}

\wxheading{Include files}

<wx/longlong.h>


\membersection{::wxNewId}\label{wxnewid}

\func{long}{wxNewId}{\void}

This function is deprecated as the ids generated by it can conflict with the
ids defined by the user code, use \texttt{wxID\_ANY} to assign ids which are
guaranteed to not conflict with the user-defined ids for the controls and menu
items you create instead of using this function.


Generates an integer identifier unique to this run of the program.

\wxheading{Include files}

<wx/utils.h>


\membersection{wxON\_BLOCK\_EXIT}\label{wxonblockexit}

\func{}{wxON\_BLOCK\_EXIT0}{\param{}{func}}

\func{}{wxON\_BLOCK\_EXIT1}{\param{}{func}, \param{}{p1}}

\func{}{wxON\_BLOCK\_EXIT2}{\param{}{func}, \param{}{p1}, \param{}{p2}}

This family of macros allows to ensure that the global function \arg{func}
with 0, 1, 2 or more parameters (up to some implementaton-defined limit) is
executed on scope exit, whether due to a normal function return or because an
exception has been thrown. A typical example of its usage:
\begin{verbatim}
    void *buf = malloc(size);
    wxON_BLOCK_EXIT1(free, buf);
\end{verbatim}

Please see the original article by Andrei Alexandrescu and Petru Marginean
published in December 2000 issue of \emph{C/C++ Users Journal} for more
details.

\wxheading{Include files}

<wx/scopeguard.h>

\wxheading{See also}

\helpref{wxON\_BLOCK\_EXIT\_OBJ}{wxonblockexitobj}


\membersection{wxON\_BLOCK\_EXIT\_OBJ}\label{wxonblockexitobj}

\func{}{wxON\_BLOCK\_EXIT\_OBJ0}{\param{}{obj}, \param{}{method}}

\func{}{wxON\_BLOCK\_EXIT\_OBJ1}{\param{}{obj}, \param{}{method}, \param{}{p1}}

\func{}{wxON\_BLOCK\_EXIT\_OBJ2}{\param{}{obj}, \param{}{method}, \param{}{p1}, \param{}{p2}}

This family of macros is similar to \helpref{wxON\_BLOCK\_EXIT}{wxonblockexit}
but calls a method of the given object instead of a free function.

\wxheading{Include files}

<wx/scopeguard.h>


\membersection{::wxRegisterId}\label{wxregisterid}

\func{void}{wxRegisterId}{\param{long}{ id}}

Ensures that ids subsequently generated by {\bf NewId} do not clash with
the given {\bf id}.

\wxheading{Include files}

<wx/utils.h>


\membersection{::wxDDECleanUp}\label{wxddecleanup}

\func{void}{wxDDECleanUp}{\void}

Called when wxWidgets exits, to clean up the DDE system. This no longer needs to be
called by the application.

See also \helpref{wxDDEInitialize}{wxddeinitialize}.

\wxheading{Include files}

<wx/dde.h>


\membersection{::wxDDEInitialize}\label{wxddeinitialize}

\func{void}{wxDDEInitialize}{\void}

Initializes the DDE system. May be called multiple times without harm.

This no longer needs to be called by the application: it will be called
by wxWidgets if necessary.

See also \helpref{wxDDEServer}{wxddeserver}, \helpref{wxDDEClient}{wxddeclient}, \helpref{wxDDEConnection}{wxddeconnection},\rtfsp
\helpref{wxDDECleanUp}{wxddecleanup}.

\wxheading{Include files}

<wx/dde.h>


\membersection{::wxEnableTopLevelWindows}\label{wxenabletoplevelwindows}

\func{void}{wxEnableTopLevelWindows}{\param{bool}{ enable = true}}

This function enables or disables all top level windows. It is used by
\helpref{::wxSafeYield}{wxsafeyield}.

\wxheading{Include files}

<wx/utils.h>


\membersection{::wxFindMenuItemId}\label{wxfindmenuitemid}

\func{int}{wxFindMenuItemId}{\param{wxFrame *}{frame}, \param{const wxString\& }{menuString}, \param{const wxString\& }{itemString}}

Find a menu item identifier associated with the given frame's menu bar.

\wxheading{Include files}

<wx/utils.h>


\membersection{::wxFindWindowByLabel}\label{wxfindwindowbylabel}

\func{wxWindow *}{wxFindWindowByLabel}{\param{const wxString\& }{label}, \param{wxWindow *}{parent=NULL}}

{\bf NB:} This function is obsolete, please use
\helpref{wxWindow::FindWindowByLabel}{wxwindowfindwindowbylabel} instead.

Find a window by its label. Depending on the type of window, the label may be a window title
or panel item label. If {\it parent} is NULL, the search will start from all top-level
frames and dialog boxes; if non-NULL, the search will be limited to the given window hierarchy.
The search is recursive in both cases.

\wxheading{Include files}

<wx/utils.h>


\membersection{::wxFindWindowByName}\label{wxfindwindowbyname}

\func{wxWindow *}{wxFindWindowByName}{\param{const wxString\& }{name}, \param{wxWindow *}{parent=NULL}}

{\bf NB:} This function is obsolete, please use
\helpref{wxWindow::FindWindowByName}{wxwindowfindwindowbyname} instead.

Find a window by its name (as given in a window constructor or {\bf Create} function call).
If {\it parent} is NULL, the search will start from all top-level
frames and dialog boxes; if non-NULL, the search will be limited to the given window hierarchy.
The search is recursive in both cases.

If no such named window is found, {\bf wxFindWindowByLabel} is called.

\wxheading{Include files}

<wx/utils.h>


\membersection{::wxFindWindowAtPoint}\label{wxfindwindowatpoint}

\func{wxWindow *}{wxFindWindowAtPoint}{\param{const wxPoint\& }{pt}}

Find the deepest window at the given mouse position in screen coordinates,
returning the window if found, or NULL if not.


\membersection{::wxFindWindowAtPointer}\label{wxfindwindowatpointer}

\func{wxWindow *}{wxFindWindowAtPointer}{\param{wxPoint\& }{pt}}

Find the deepest window at the mouse pointer position, returning the window
and current pointer position in screen coordinates.


\membersection{wxFromString}\label{wxfromstring}

\func{bool}{wxFromString}{\param{const wxString\& }{str},
                           \param{wxColour* }{col}}

\func{bool}{wxFromString}{\param{const wxString\& }{str},
                           \param{wxFont* }{col}}

Converts string to the type of the second argument. Returns \true on success.
See also: \helpref{wxToString}{wxtostring}.


\membersection{::wxGetActiveWindow}\label{wxgetactivewindow}

\func{wxWindow *}{wxGetActiveWindow}{\void}

Gets the currently active window (implemented for MSW and GTK only currently,
always returns \NULL in the other ports).

\wxheading{Include files}

<wx/window.h>


\membersection{::wxGetBatteryState}\label{wxgetbatterystate}

\func{wxBatteryState}{wxGetBatteryState}{\void}

Returns battery state as one of \texttt{wxBATTERY\_NORMAL\_STATE},
\texttt{wxBATTERY\_LOW\_STATE}, \texttt{wxBATTERY\_CRITICAL\_STATE},
\texttt{wxBATTERY\_SHUTDOWN\_STATE} or \texttt{wxBATTERY\_UNKNOWN\_STATE}.
\texttt{wxBATTERY\_UNKNOWN\_STATE} is also the default on platforms where
this feature is not implemented (currently everywhere but MS Windows).

\wxheading{Include files}

<wx/utils.h>


\membersection{::wxGetDisplayName}\label{wxgetdisplayname}

\func{wxString}{wxGetDisplayName}{\void}

Under X only, returns the current display name. See also \helpref{wxSetDisplayName}{wxsetdisplayname}.

\wxheading{Include files}

<wx/utils.h>


\membersection{::wxGetPowerType}\label{wxgetpowertype}

\func{wxPowerType}{wxGetPowerType}{\void}

Returns the type of power source as one of \texttt{wxPOWER\_SOCKET},
\texttt{wxPOWER\_BATTERY} or \texttt{wxPOWER\_UNKNOWN}.
\texttt{wxPOWER\_UNKNOWN} is also the default on platforms where this
feature is not implemented (currently everywhere but MS Windows).

\wxheading{Include files}

<wx/utils.h>


\membersection{::wxGetMousePosition}\label{wxgetmouseposition}

\func{wxPoint}{wxGetMousePosition}{\void}

Returns the mouse position in screen coordinates.

\wxheading{Include files}

<wx/utils.h>


\membersection{::wxGetMouseState}\label{wxgetmousestate}

\func{wxMouseState}{wxGetMouseState}{\void}

Returns the current state of the mouse.  Returns a wxMouseState
instance that contains the current position of the mouse pointer in
screen coordinants, as well as boolean values indicating the up/down
status of the mouse buttons and the modifier keys.

\wxheading{Include files}

<wx/utils.h>

wxMouseState has the following interface:

\begin{verbatim}
class wxMouseState
{
public:
    wxMouseState();

    wxCoord     GetX();
    wxCoord     GetY();

    bool        LeftDown();
    bool        MiddleDown();
    bool        RightDown();

    bool        ControlDown();
    bool        ShiftDown();
    bool        AltDown();
    bool        MetaDown();
    bool        CmdDown();

    void        SetX(wxCoord x);
    void        SetY(wxCoord y);

    void        SetLeftDown(bool down);
    void        SetMiddleDown(bool down);
    void        SetRightDown(bool down);

    void        SetControlDown(bool down);
    void        SetShiftDown(bool down);
    void        SetAltDown(bool down);
    void        SetMetaDown(bool down);
};
\end{verbatim}



\membersection{::wxGetStockLabel}\label{wxgetstocklabel}

\func{wxString}{wxGetStockLabel}{\param{wxWindowID }{id}, \param{bool }{withCodes = true}, \param{const wxString\& }{accelerator = wxEmptyString}}

Returns label that should be used for given {\it id} element.

\wxheading{Parameters}

\docparam{id}{given id of the \helpref{wxMenuItem}{wxmenuitem}, \helpref{wxButton}{wxbutton}, \helpref{wxToolBar}{wxtoolbar} tool, etc.}

\docparam{withCodes}{if false then strip accelerator code from the label;
usefull for getting labels without accelerator char code like for toolbar tooltip or
under platforms without traditional keyboard like smartphones}

\docparam{accelerator}{optional accelerator string automatically added to label; useful
for building labels for \helpref{wxMenuItem}{wxmenuitem}}

\wxheading{Include files}

<wx/stockitem.h>


\membersection{::wxGetTopLevelParent}\label{wxgettoplevelparent}

\func{wxWindow *}{wxGetTopLevelParent}{\param{wxWindow }{*win}}

Returns the first top level parent of the given window, or in other words, the
frame or dialog containing it, or {\tt NULL}.

\wxheading{Include files}

<wx/window.h>


\membersection{::wxLaunchDefaultBrowser}\label{wxlaunchdefaultbrowser}

\func{bool}{wxLaunchDefaultBrowser}{\param{const wxString\& }{url}, \param{int }{flags = $0$}}

Open the \arg{url} in user's default browser. If \arg{flags} parameter contains
\texttt{wxBROWSER\_NEW\_WINDOW} flag, a new window is opened for the URL
(currently this is only supported under Windows). The \arg{url} may also be a
local file path (with or without \texttt{file://} prefix), if it doesn't
correspond to an existing file and the URL has no scheme \texttt{http://} is
prepended to it by default.

Returns \true if the application was successfully launched.

Note that for some configurations of the running user, the application which
is launched to open the given URL may be URL-dependent (e.g. a browser may be used for
local URLs while another one may be used for remote URLs).

\wxheading{Include files}

<wx/utils.h>


\membersection{::wxLoadUserResource}\label{wxloaduserresource}

\func{wxString}{wxLoadUserResource}{\param{const wxString\& }{resourceName}, \param{const wxString\& }{resourceType=``TEXT"}}

Loads a user-defined Windows resource as a string. If the resource is found, the function creates
a new character array and copies the data into it. A pointer to this data is returned. If unsuccessful, NULL is returned.

The resource must be defined in the {\tt .rc} file using the following syntax:

\begin{verbatim}
myResource TEXT file.ext
\end{verbatim}

where {\tt file.ext} is a file that the resource compiler can find.

This function is available under Windows only.

\wxheading{Include files}

<wx/utils.h>


\membersection{::wxPostDelete}\label{wxpostdelete}

\func{void}{wxPostDelete}{\param{wxObject *}{object}}

Tells the system to delete the specified object when
all other events have been processed. In some environments, it is
necessary to use this instead of deleting a frame directly with the
delete operator, because some GUIs will still send events to a deleted window.

Now obsolete: use \helpref{wxWindow::Close}{wxwindowclose} instead.

\wxheading{Include files}

<wx/utils.h>


\membersection{::wxPostEvent}\label{wxpostevent}

\func{void}{wxPostEvent}{\param{wxEvtHandler *}{dest}, \param{wxEvent\& }{event}}

In a GUI application, this function posts {\it event} to the specified {\it dest}
object using \helpref{wxEvtHandler::AddPendingEvent}{wxevthandleraddpendingevent}.
Otherwise, it dispatches {\it event} immediately using
\helpref{wxEvtHandler::ProcessEvent}{wxevthandlerprocessevent}.
See the respective documentation for details (and caveats).

\wxheading{Include files}

<wx/app.h>


\membersection{::wxSetDisplayName}\label{wxsetdisplayname}

\func{void}{wxSetDisplayName}{\param{const wxString\& }{displayName}}

Under X only, sets the current display name. This is the X host and display name such
as ``colonsay:0.0", and the function indicates which display should be used for creating
windows from this point on. Setting the display within an application allows multiple
displays to be used.

See also \helpref{wxGetDisplayName}{wxgetdisplayname}.

\wxheading{Include files}

<wx/utils.h>


\membersection{::wxStripMenuCodes}\label{wxstripmenucodes}

\func{wxString}{wxStripMenuCodes}{\param{const wxString\& }{str}, \param{int }{flags = wxStrip\_All}}

Strips any menu codes from \arg{str} and returns the result.

By default, the functions strips both the mnemonics character (\texttt{'\&'})
which is used to indicate a keyboard shortkey, and the accelerators, which are
used only in the menu items and are separated from the main text by the
\texttt{$\backslash$t} (TAB) character. By using \arg{flags} of
\texttt{wxStrip\_Mnemonics} or \texttt{wxStrip\_Accel} to strip only the former
or the latter part, respectively.

Notice that in most cases
\helpref{wxMenuItem::GetLabelFromText}{wxmenuitemgetlabelfromtext} or
\helpref{wxControl::GetLabelText}{wxcontrolgetlabeltext} can be used instead.

\wxheading{Include files}

<wx/utils.h>


\membersection{wxSTRINGIZE}\label{wxstringize}

\func{}{wxSTRINGIZE}{\param{}{x}}

Returns the string representation of the given symbol which can be either a
literal or a macro (hence the advantage of using this macro instead of the
standard preprocessor \texttt{\#} operator which doesn't work with macros).

Notice that this macro always produces a \texttt{char} string, use 
\helpref{wxSTRINGIZE\_T}{wxstringizet} to build a wide string Unicode build.

\wxheading{See also}

\helpref{wxCONCAT}{wxconcat}


\membersection{wxSTRINGIZE\_T}\label{wxstringizet}

\func{}{wxSTRINGIZE\_T}{\param{}{x}}

Returns the string representation of the given symbol as either an ASCII or
Unicode string, depending on the current build. This is the Unicode-friendly
equivalent of \helpref{wxSTRINGIZE}{wxstringize}.


\membersection{wxSUPPRESS\_GCC\_PRIVATE\_DTOR\_WARNING}\label{wxsuppressgccprivatedtorwarning}

\func{}{wxSUPPRESS\_GCC\_PRIVATE\_DTOR\_WARNING}{\param{}{name}}

GNU C++ compiler gives a warning for any class whose destructor is private
unless it has a friend. This warning may sometimes be useful but it doesn't
make sense for reference counted class which always delete themselves (hence
destructor should be private) but don't necessarily have any friends, so this
macro is provided to disable the warning in such case. The \arg{name} parameter
should be the name of the class but is only used to construct a unique friend
class name internally. Example of using the macro:

\begin{verbatim}
    class RefCounted
    {
    public:
        RefCounted() { m_nRef = 1; }
        void IncRef() { m_nRef++ ; }
        void DecRef() { if ( !--m_nRef ) delete this; }

    private:
        ~RefCounted() { }

        wxSUPPRESS_GCC_PRIVATE_DTOR(RefCounted)
    };
\end{verbatim}

Notice that there should be no semicolon after this macro.


\membersection{wxToString}\label{wxtostring}

\func{wxString}{wxToString}{\param{const wxColour\& }{col}}

\func{wxString}{wxToString}{\param{const wxFont\& }{col}}

Converts its argument to string.
See also: \helpref{wxFromString}{wxfromstring}.


\membersection{wxULL}\label{wxull}

\func{wxLongLong\_t}{wxULL}{\param{}{number}}

This macro is defined for the platforms with a native 64 bit integer type and
allows to define unsigned 64 bit compile time constants:

\begin{verbatim}
    #ifdef wxLongLong_t
        unsigned wxLongLong_t ll = wxULL(0x1234567890abcdef);
    #endif
\end{verbatim}

\wxheading{Include files}

<wx/longlong.h>

\wxheading{See also}

\helpref{wxLL}{wxll}, \helpref{wxLongLong}{wxlonglong}


\membersection{wxVaCopy}\label{wxvacopy}

\func{void}{wxVaCopy}{\param{va\_list }{argptrDst}, \param{va\_list}{ argptrSrc}}

This macro is the same as the standard C99 \texttt{va\_copy} for the compilers
which support it or its replacement for those that don't. It must be used to
preserve the value of a \texttt{va\_list} object if you need to use it after
passing it to another function because it can be modified by the latter.

As with \texttt{va\_start}, each call to \texttt{wxVaCopy} must have a matching
\texttt{va\_end}.



\membersection{\_\_WXFUNCTION\_\_}\label{wxfunction}

\func{}{\_\_WXFUNCTION\_\_}{\void}

This macro expands to the name of the current function if the compiler supports
any of \texttt{\_\_FUNCTION\_\_}, \texttt{\_\_func\_\_} or equivalent variables
or macros or to \NULL if none of them is available.



\section{Byte order macros}\label{byteordermacros}

The endian-ness issues (that is the difference between big-endian and
little-endian architectures) are important for the portable programs working
with the external binary data (for example, data files or data coming from
network) which is usually in some fixed, platform-independent format. The
macros are helpful for transforming the data to the correct format.


\membersection{wxINTXX\_SWAP\_ALWAYS}\label{intswapalways}

\func{wxInt32}{wxINT32\_SWAP\_ALWAYS}{\param{wxInt32 }{value}}

\func{wxUint32}{wxUINT32\_SWAP\_ALWAYS}{\param{wxUint32 }{value}}

\func{wxInt16}{wxINT16\_SWAP\_ALWAYS}{\param{wxInt16 }{value}}

\func{wxUint16}{wxUINT16\_SWAP\_ALWAYS}{\param{wxUint16 }{value}}

These macros will swap the bytes of the {\it value} variable from little
endian to big endian or vice versa unconditionally, i.e. independently of the
current platform.


\membersection{wxINTXX\_SWAP\_ON\_BE}\label{intswaponbe}

\func{wxInt32}{wxINT32\_SWAP\_ON\_BE}{\param{wxInt32 }{value}}

\func{wxUint32}{wxUINT32\_SWAP\_ON\_BE}{\param{wxUint32 }{value}}

\func{wxInt16}{wxINT16\_SWAP\_ON\_BE}{\param{wxInt16 }{value}}

\func{wxUint16}{wxUINT16\_SWAP\_ON\_BE}{\param{wxUint16 }{value}}

This macro will swap the bytes of the {\it value} variable from little
endian to big endian or vice versa if the program is compiled on a
big-endian architecture (such as Sun work stations). If the program has
been compiled on a little-endian architecture, the value will be unchanged.

Use these macros to read data from and write data to a file that stores
data in little-endian (for example Intel i386) format.


\membersection{wxINTXX\_SWAP\_ON\_LE}\label{intswaponle}

\func{wxInt32}{wxINT32\_SWAP\_ON\_LE}{\param{wxInt32 }{value}}

\func{wxUint32}{wxUINT32\_SWAP\_ON\_LE}{\param{wxUint32 }{value}}

\func{wxInt16}{wxINT16\_SWAP\_ON\_LE}{\param{wxInt16 }{value}}

\func{wxUint16}{wxUINT16\_SWAP\_ON\_LE}{\param{wxUint16 }{value}}

This macro will swap the bytes of the {\it value} variable from little
endian to big endian or vice versa if the program is compiled on a
little-endian architecture (such as Intel PCs). If the program has
been compiled on a big-endian architecture, the value will be unchanged.

Use these macros to read data from and write data to a file that stores
data in big-endian format.



\section{RTTI functions}\label{rttimacros}

wxWidgets uses its own RTTI ("run-time type identification") system which
predates the current standard C++ RTTI and so is kept for backwards
compatibility reasons but also because it allows some things which the
standard RTTI doesn't directly support (such as creating a class from its
name).

The standard C++ RTTI can be used in the user code without any problems and in
general you shouldn't need to use the functions and the macros in this section
unless you are thinking of modifying or adding any wxWidgets classes.

\wxheading{See also}

\helpref{RTTI overview}{runtimeclassoverview}


\membersection{CLASSINFO}\label{classinfo}

\func{wxClassInfo *}{CLASSINFO}{className}

Returns a pointer to the wxClassInfo object associated with this class.

\wxheading{Include files}

<wx/object.h>


\membersection{DECLARE\_ABSTRACT\_CLASS}\label{declareabstractclass}

\func{}{DECLARE\_ABSTRACT\_CLASS}{className}

Used inside a class declaration to declare that the class should be
made known to the class hierarchy, but objects of this class cannot be created
dynamically. The same as DECLARE\_CLASS.

Example:

\begin{verbatim}
class wxCommand: public wxObject
{
  DECLARE_ABSTRACT_CLASS(wxCommand)

 private:
  ...
 public:
  ...
};
\end{verbatim}

\wxheading{Include files}

<wx/object.h>


\membersection{DECLARE\_APP}\label{declareapp}

\func{}{DECLARE\_APP}{className}

This is used in headers to create a forward declaration of the
\helpref{wxGetApp}{wxgetapp} function implemented by
\helpref{IMPLEMENT\_APP}{implementapp}. It creates the declaration
{\tt className\& wxGetApp(void)}.

Example:

\begin{verbatim}
  DECLARE_APP(MyApp)
\end{verbatim}

\wxheading{Include files}

<wx/app.h>


\membersection{DECLARE\_CLASS}\label{declareclass}

\func{}{DECLARE\_CLASS}{className}

Used inside a class declaration to declare that the class should be
made known to the class hierarchy, but objects of this class cannot be created
dynamically. The same as DECLARE\_ABSTRACT\_CLASS.

\wxheading{Include files}

<wx/object.h>


\membersection{DECLARE\_DYNAMIC\_CLASS}\label{declaredynamicclass}

\func{}{DECLARE\_DYNAMIC\_CLASS}{className}

Used inside a class declaration to make the class known to wxWidgets RTTI
system and also declare that the objects of this class should be dynamically
creatable from run-time type information. Notice that this implies that the
class should have a default constructor, if this is not the case consider using
\helpref{DECLARE\_CLASS}{declareclass}.

Example:

\begin{verbatim}
class wxFrame: public wxWindow
{
  DECLARE_DYNAMIC_CLASS(wxFrame)

 private:
  const wxString& frameTitle;
 public:
  ...
};
\end{verbatim}

\wxheading{Include files}

<wx/object.h>


\membersection{IMPLEMENT\_ABSTRACT\_CLASS}\label{implementabstractclass}

\func{}{IMPLEMENT\_ABSTRACT\_CLASS}{className, baseClassName}

Used in a C++ implementation file to complete the declaration of
a class that has run-time type information. The same as IMPLEMENT\_CLASS.

Example:

\begin{verbatim}
IMPLEMENT_ABSTRACT_CLASS(wxCommand, wxObject)

wxCommand::wxCommand(void)
{
...
}
\end{verbatim}

\wxheading{Include files}

<wx/object.h>


\membersection{IMPLEMENT\_ABSTRACT\_CLASS2}\label{implementabstractclass2}

\func{}{IMPLEMENT\_ABSTRACT\_CLASS2}{className, baseClassName1, baseClassName2}

Used in a C++ implementation file to complete the declaration of
a class that has run-time type information and two base classes. The same as IMPLEMENT\_CLASS2.

\wxheading{Include files}

<wx/object.h>


\membersection{IMPLEMENT\_APP}\label{implementapp}

\func{}{IMPLEMENT\_APP}{className}

This is used in the application class implementation file to make the application class known to
wxWidgets for dynamic construction. You use this instead of

Old form:

\begin{verbatim}
  MyApp myApp;
\end{verbatim}

New form:

\begin{verbatim}
  IMPLEMENT_APP(MyApp)
\end{verbatim}

See also \helpref{DECLARE\_APP}{declareapp}.

\wxheading{Include files}

<wx/app.h>


\membersection{IMPLEMENT\_CLASS}\label{implementclass}

\func{}{IMPLEMENT\_CLASS}{className, baseClassName}

Used in a C++ implementation file to complete the declaration of
a class that has run-time type information. The same as IMPLEMENT\_ABSTRACT\_CLASS.

\wxheading{Include files}

<wx/object.h>


\membersection{IMPLEMENT\_CLASS2}\label{implementclass2}

\func{}{IMPLEMENT\_CLASS2}{className, baseClassName1, baseClassName2}

Used in a C++ implementation file to complete the declaration of a
class that has run-time type information and two base classes. The
same as IMPLEMENT\_ABSTRACT\_CLASS2.

\wxheading{Include files}

<wx/object.h>


\membersection{IMPLEMENT\_DYNAMIC\_CLASS}\label{implementdynamicclass}

\func{}{IMPLEMENT\_DYNAMIC\_CLASS}{className, baseClassName}

Used in a C++ implementation file to complete the declaration of
a class that has run-time type information, and whose instances
can be created dynamically.

Example:

\begin{verbatim}
IMPLEMENT_DYNAMIC_CLASS(wxFrame, wxWindow)

wxFrame::wxFrame(void)
{
...
}
\end{verbatim}

\wxheading{Include files}

<wx/object.h>


\membersection{IMPLEMENT\_DYNAMIC\_CLASS2}\label{implementdynamicclass2}

\func{}{IMPLEMENT\_DYNAMIC\_CLASS2}{className, baseClassName1, baseClassName2}

Used in a C++ implementation file to complete the declaration of
a class that has run-time type information, and whose instances
can be created dynamically. Use this for classes derived from two
base classes.

\wxheading{Include files}

<wx/object.h>


\membersection{wxConstCast}\label{wxconstcast}

\func{classname *}{wxConstCast}{ptr, classname}

This macro expands into {\tt const\_cast<classname *>(ptr)} if the compiler
supports {\it const\_cast} or into an old, C-style cast, otherwise.

\wxheading{See also}

\helpref{wx\_const\_cast}{wxconstcastraw}\\
\helpref{wxDynamicCast}{wxdynamiccast}\\
\helpref{wxStaticCast}{wxstaticcast}


\membersection{::wxCreateDynamicObject}\label{wxcreatedynamicobject}

\func{wxObject *}{wxCreateDynamicObject}{\param{const wxString\& }{className}}

Creates and returns an object of the given class, if the class has been
registered with the dynamic class system using DECLARE... and IMPLEMENT... macros.


\membersection{WXDEBUG\_NEW}\label{debugnew}

\func{}{WXDEBUG\_NEW}{arg}

This is defined in debug mode to be call the redefined new operator
with filename and line number arguments. The definition is:

\begin{verbatim}
#define WXDEBUG_NEW new(__FILE__,__LINE__)
\end{verbatim}

In non-debug mode, this is defined as the normal new operator.

\wxheading{Include files}

<wx/object.h>


\membersection{wxDynamicCast}\label{wxdynamiccast}

\func{classname *}{wxDynamicCast}{ptr, classname}

This macro returns the pointer {\it ptr} cast to the type {\it classname *} if
the pointer is of this type (the check is done during the run-time) or
{\tt NULL} otherwise. Usage of this macro is preferred over obsoleted
wxObject::IsKindOf() function.

The {\it ptr} argument may be {\tt NULL}, in which case {\tt NULL} will be
returned.

Example:

\begin{verbatim}
    wxWindow *win = wxWindow::FindFocus();
    wxTextCtrl *text = wxDynamicCast(win, wxTextCtrl);
    if ( text )
    {
        // a text control has the focus...
    }
    else
    {
        // no window has the focus or it is not a text control
    }
\end{verbatim}

\wxheading{See also}

\helpref{RTTI overview}{runtimeclassoverview}\\
\helpref{wxDynamicCastThis}{wxdynamiccastthis}\\
\helpref{wxConstCast}{wxconstcast}\\
\helpref{wxStaticCast}{wxstaticcast}


\membersection{wxDynamicCastThis}\label{wxdynamiccastthis}

\func{classname *}{wxDynamicCastThis}{classname}

This macro is equivalent to {\tt wxDynamicCast(this, classname)} but the
latter provokes spurious compilation warnings from some compilers (because it
tests whether {\tt this} pointer is non-{\tt NULL} which is always true), so
this macro should be used to avoid them.

\wxheading{See also}

\helpref{wxDynamicCast}{wxdynamiccast}


\membersection{wxStaticCast}\label{wxstaticcast}

\func{classname *}{wxStaticCast}{ptr, classname}

This macro checks that the cast is valid in debug mode (an assert failure will
result if {\tt wxDynamicCast(ptr, classname) == NULL}) and then returns the
result of executing an equivalent of {\tt static\_cast<classname *>(ptr)}.

\wxheading{See also}

\helpref{wx\_static\_cast}{wxstaticcastraw}\\
\helpref{wxDynamicCast}{wxdynamiccast}\\
\helpref{wxConstCast}{wxconstcast}


\membersection{wx\_const\_cast}\label{wxconstcastraw}

\func{T}{wx\_const\_cast}{T, x}

Same as \texttt{const\_cast<T>(x)} if the compiler supports const cast or
\texttt{(T)x} for old compilers. Unlike \helpref{wxConstCast}{wxconstcast},
the cast it to the type \arg{T} and not to \texttt{T *} and also the order of
arguments is the same as for the standard cast.

\wxheading{See also}

\helpref{wx\_reinterpret\_cast}{wxreinterpretcastraw},\\
\helpref{wx\_static\_cast}{wxstaticcastraw}


\membersection{wx\_reinterpret\_cast}\label{wxreinterpretcastraw}

\func{T}{wx\_reinterpret\_cast}{T, x}

Same as \texttt{reinterpret\_cast<T>(x)} if the compiler supports reinterpret cast or
\texttt{(T)x} for old compilers.

\wxheading{See also}

\helpref{wx\_const\_cast}{wxconstcastraw},\\
\helpref{wx\_static\_cast}{wxstaticcastraw}


\membersection{wx\_static\_cast}\label{wxstaticcastraw}

\func{T}{wx\_static\_cast}{T, x}

Same as \texttt{static\_cast<T>(x)} if the compiler supports static cast or
\texttt{(T)x} for old compilers. Unlike \helpref{wxStaticCast}{wxstaticcast},
there are no checks being done and the meaning of the macro arguments is exactly
the same as for the standard static cast, i.e. \arg{T} is the full type name and
star is not appended to it.

\wxheading{See also}

\helpref{wx\_const\_cast}{wxconstcastraw},\\
\helpref{wx\_reinterpret\_cast}{wxreinterpretcastraw},\\
\helpref{wx\_truncate\_cast}{wxtruncatecast}


\membersection{wx\_truncate\_cast}\label{wxtruncatecast}

\func{T}{wx\_truncate\_cast}{T, x}

This case doesn't correspond to any standard cast but exists solely to make
casts which possibly result in a truncation of an integer value more readable.

\wxheading{See also}

\helpref{wx\_static\_cast}{wxstaticcastraw}


\section{Log functions}\label{logfunctions}

These functions provide a variety of logging functions: see \helpref{Log classes overview}{wxlogoverview} for
further information. The functions use (implicitly) the currently active log
target, so their descriptions here may not apply if the log target is not the
standard one (installed by wxWidgets in the beginning of the program).

\wxheading{Include files}

<wx/log.h>


\membersection{::wxDebugMsg}\label{wxdebugmsg}

\func{void}{wxDebugMsg}{\param{const wxString\& }{fmt}, \param{...}{}}

{\bf NB:} This function is now obsolete, replaced by \helpref{Log
functions}{logfunctions} and \helpref{wxLogDebug}{wxlogdebug} in particular.

Display a debugging message; under Windows, this will appear on the
debugger command window, and under Unix, it will be written to standard
error.

The syntax is identical to {\bf printf}: pass a format string and a
variable list of arguments.

{\bf Tip:} under Windows, if your application crashes before the
message appears in the debugging window, put a wxYield call after
each wxDebugMsg call. wxDebugMsg seems to be broken under WIN32s
(at least for Watcom C++): preformat your messages and use OutputDebugString
instead.

\wxheading{Include files}

<wx/utils.h>


\membersection{::wxError}\label{wxerror}

\func{void}{wxError}{\param{const wxString\& }{msg}, \param{const wxString\& }{title = "wxWidgets Internal Error"}}

{\bf NB:} This function is now obsolete, please use \helpref{wxLogError}{wxlogerror}
instead.

Displays {\it msg} and continues. This writes to standard error under
Unix, and pops up a message box under Windows. Used for internal
wxWidgets errors. See also \helpref{wxFatalError}{wxfatalerror}.

\wxheading{Include files}

<wx/utils.h>


\membersection{::wxFatalError}\label{wxfatalerror}

\func{void}{wxFatalError}{\param{const wxString\& }{msg}, \param{const wxString\& }{title = "wxWidgets Fatal Error"}}

{\bf NB:} This function is now obsolete, please use
\helpref{wxLogFatalError}{wxlogfatalerror} instead.

Displays {\it msg} and exits. This writes to standard error under Unix,
and pops up a message box under Windows. Used for fatal internal
wxWidgets errors. See also \helpref{wxError}{wxerror}.

\wxheading{Include files}

<wx/utils.h>


\membersection{::wxLogError}\label{wxlogerror}

\func{void}{wxLogError}{\param{const char *}{formatString}, \param{...}{}}

\func{void}{wxVLogError}{\param{const char *}{formatString}, \param{va\_list }{argPtr}}

The functions to use for error messages, i.e. the messages that must be shown
to the user. The default processing is to pop up a message box to inform the
user about it.


\membersection{::wxLogFatalError}\label{wxlogfatalerror}

\func{void}{wxLogFatalError}{\param{const char *}{formatString}, \param{...}{}}

\func{void}{wxVLogFatalError}{\param{const char *}{formatString}, \param{va\_list }{argPtr}}

Like \helpref{wxLogError}{wxlogerror}, but also
terminates the program with the exit code 3. Using {\it abort()} standard
function also terminates the program with this exit code.


\membersection{::wxLogWarning}\label{wxlogwarning}

\func{void}{wxLogWarning}{\param{const char *}{formatString}, \param{...}{}}

\func{void}{wxVLogWarning}{\param{const char *}{formatString}, \param{va\_list }{argPtr}}

For warnings - they are also normally shown to the user, but don't interrupt
the program work.


\membersection{::wxLogMessage}\label{wxlogmessage}

\func{void}{wxLogMessage}{\param{const char *}{formatString}, \param{...}{}}

\func{void}{wxVLogMessage}{\param{const char *}{formatString}, \param{va\_list }{argPtr}}

For all normal, informational messages. They also appear in a message box by
default (but it can be changed).

\membersection{::wxLogVerbose}\label{wxlogverbose}

\func{void}{wxLogVerbose}{\param{const char *}{formatString}, \param{...}{}}

\func{void}{wxVLogVerbose}{\param{const char *}{formatString}, \param{va\_list }{argPtr}}

For verbose output. Normally, it is suppressed, but
might be activated if the user wishes to know more details about the program
progress (another, but possibly confusing name for the same function is {\bf wxLogInfo}).


\membersection{::wxLogStatus}\label{wxlogstatus}

\func{void}{wxLogStatus}{\param{wxFrame *}{frame}, \param{const char *}{formatString}, \param{...}{}}

\func{void}{wxVLogStatus}{\param{wxFrame *}{frame}, \param{const char *}{formatString}, \param{va\_list }{argPtr}}

\func{void}{wxLogStatus}{\param{const char *}{formatString}, \param{...}{}}

\func{void}{wxVLogStatus}{\param{const char *}{formatString}, \param{va\_list }{argPtr}}

Messages logged by these functions will appear in the statusbar of the {\it
frame} or of the top level application window by default (i.e. when using
the second version of the functions).

If the target frame doesn't have a statusbar, the message will be lost.


\membersection{::wxLogSysError}\label{wxlogsyserror}

\func{void}{wxLogSysError}{\param{const char *}{formatString}, \param{...}{}}

\func{void}{wxVLogSysError}{\param{const char *}{formatString}, \param{va\_list }{argPtr}}

Mostly used by wxWidgets itself, but might be handy for logging errors after
system call (API function) failure. It logs the specified message text as well
as the last system error code ({\it errno} or {\it ::GetLastError()} depending
on the platform) and the corresponding error message. The second form
of this function takes the error code explicitly as the first argument.

\wxheading{See also}

\helpref{wxSysErrorCode}{wxsyserrorcode},
\helpref{wxSysErrorMsg}{wxsyserrormsg}


\membersection{::wxLogDebug}\label{wxlogdebug}

\func{void}{wxLogDebug}{\param{const char *}{formatString}, \param{...}{}}

\func{void}{wxVLogDebug}{\param{const char *}{formatString}, \param{va\_list }{argPtr}}

The right functions for debug output. They only do something in debug
mode (when the preprocessor symbol \_\_WXDEBUG\_\_ is defined) and expand to
nothing in release mode (otherwise).


\membersection{::wxLogTrace}\label{wxlogtrace}

\func{void}{wxLogTrace}{\param{const char *}{formatString}, \param{...}{}}

\func{void}{wxVLogTrace}{\param{const char *}{formatString}, \param{va\_list }{argPtr}}

\func{void}{wxLogTrace}{\param{const char *}{mask}, \param{const char *}{formatString}, \param{...}{}}

\func{void}{wxVLogTrace}{\param{const char *}{mask}, \param{const char *}{formatString}, \param{va\_list }{argPtr}}

\func{void}{wxLogTrace}{\param{wxTraceMask}{ mask}, \param{const char *}{formatString}, \param{...}{}}

\func{void}{wxVLogTrace}{\param{wxTraceMask}{ mask}, \param{const char *}{formatString}, \param{va\_list }{argPtr}}

As {\bf wxLogDebug}, trace functions only do something in debug build and
expand to nothing in the release one. The reason for making
it a separate function from it is that usually there are a lot of trace
messages, so it might make sense to separate them from other debug messages.

The trace messages also usually can be separated into different categories and
the second and third versions of this function only log the message if the
{\it mask} which it has is currently enabled in \helpref{wxLog}{wxlog}. This
allows to selectively trace only some operations and not others by changing
the value of the trace mask (possible during the run-time).

For the second function (taking a string mask), the message is logged only if
the mask has been previously enabled by the call to
\helpref{AddTraceMask}{wxlogaddtracemask} or by setting
\helpref{{\tt WXTRACE} environment variable}{envvars}.
The predefined string trace masks
used by wxWidgets are:

\begin{itemize}\itemsep=0pt
\item wxTRACE\_MemAlloc: trace memory allocation (new/delete)
\item wxTRACE\_Messages: trace window messages/X callbacks
\item wxTRACE\_ResAlloc: trace GDI resource allocation
\item wxTRACE\_RefCount: trace various ref counting operations
\item wxTRACE\_OleCalls: trace OLE method calls (Win32 only)
\end{itemize}

{\bf Caveats:} since both the mask and the format string are strings,
this might lead to function signature confusion in some cases:
if you intend to call the format string only version of wxLogTrace,
then add a \%s format string parameter and then supply a second string parameter for that \%s, the string mask version of wxLogTrace will erroneously get called instead, since you are supplying two string parameters to the function.
In this case you'll unfortunately have to avoid having two leading
string parameters, e.g. by adding a bogus integer (with its \%d format string).

The third version of the function only logs the message if all the bits
corresponding to the {\it mask} are set in the wxLog trace mask which can be
set by \helpref{SetTraceMask}{wxlogsettracemask}. This version is less
flexible than the previous one because it doesn't allow defining the user
trace masks easily - this is why it is deprecated in favour of using string
trace masks.

\begin{itemize}\itemsep=0pt
\item wxTraceMemAlloc: trace memory allocation (new/delete)
\item wxTraceMessages: trace window messages/X callbacks
\item wxTraceResAlloc: trace GDI resource allocation
\item wxTraceRefCount: trace various ref counting operations
\item wxTraceOleCalls: trace OLE method calls (Win32 only)
\end{itemize}


\membersection{::wxSafeShowMessage}\label{wxsafeshowmessage}

\func{void}{wxSafeShowMessage}{\param{const wxString\& }{title}, \param{const wxString\& }{text}}

This function shows a message to the user in a safe way and should be safe to
call even before the application has been initialized or if it is currently in
some other strange state (for example, about to crash). Under Windows this
function shows a message box using a native dialog instead of
\helpref{wxMessageBox}{wxmessagebox} (which might be unsafe to call), elsewhere
it simply prints the message to the standard output using the title as prefix.

\wxheading{Parameters}

\docparam{title}{The title of the message box shown to the user or the prefix
of the message string}

\docparam{text}{The text to show to the user}

\wxheading{See also}

\helpref{wxLogFatalError}{wxlogfatalerror}

\wxheading{Include files}

<wx/log.h>


\membersection{::wxSysErrorCode}\label{wxsyserrorcode}

\func{unsigned long}{wxSysErrorCode}{\void}

Returns the error code from the last system call. This function uses
{\tt errno} on Unix platforms and {\tt GetLastError} under Win32.

\wxheading{See also}

\helpref{wxSysErrorMsg}{wxsyserrormsg},
\helpref{wxLogSysError}{wxlogsyserror}


\membersection{::wxSysErrorMsg}\label{wxsyserrormsg}

\func{const wxChar *}{wxSysErrorMsg}{\param{unsigned long }{errCode = 0}}

Returns the error message corresponding to the given system error code. If
{\it errCode} is $0$ (default), the last error code (as returned by
\helpref{wxSysErrorCode}{wxsyserrorcode}) is used.

\wxheading{See also}

\helpref{wxSysErrorCode}{wxsyserrorcode},
\helpref{wxLogSysError}{wxlogsyserror}


\membersection{WXTRACE}\label{trace}

\wxheading{Include files}

<wx/object.h>

\func{}{WXTRACE}{formatString, ...}

{\bf NB:} This macro is now obsolete, replaced by \helpref{Log functions}{logfunctions}.

Calls wxTrace with printf-style variable argument syntax. Output
is directed to the current output stream (see \helpref{wxDebugContext}{wxdebugcontextoverview}).

\wxheading{Include files}

<wx/memory.h>


\membersection{WXTRACELEVEL}\label{tracelevel}

\func{}{WXTRACELEVEL}{level, formatString, ...}

{\bf NB:} This function is now obsolete, replaced by \helpref{Log functions}{logfunctions}.

Calls wxTraceLevel with printf-style variable argument syntax. Output
is directed to the current output stream (see \helpref{wxDebugContext}{wxdebugcontextoverview}).
The first argument should be the level at which this information is appropriate.
It will only be output if the level returned by wxDebugContext::GetLevel is equal to or greater than
this value.

\wxheading{Include files}

<wx/memory.h>


\membersection{::wxTrace}\label{wxtrace}

\func{void}{wxTrace}{\param{const wxString\& }{fmt}, \param{...}{}}

{\bf NB:} This function is now obsolete, replaced by \helpref{Log functions}{logfunctions}.

Takes printf-style variable argument syntax. Output
is directed to the current output stream (see \helpref{wxDebugContext}{wxdebugcontextoverview}).

\wxheading{Include files}

<wx/memory.h>


\membersection{::wxTraceLevel}\label{wxtracelevel}

\func{void}{wxTraceLevel}{\param{int}{ level}, \param{const wxString\& }{fmt}, \param{...}{}}

{\bf NB:} This function is now obsolete, replaced by \helpref{Log functions}{logfunctions}.

Takes printf-style variable argument syntax. Output
is directed to the current output stream (see \helpref{wxDebugContext}{wxdebugcontextoverview}).
The first argument should be the level at which this information is appropriate.
It will only be output if the level returned by wxDebugContext::GetLevel is equal to or greater than
this value.

\wxheading{Include files}

<wx/memory.h>



\section{Time functions}\label{timefunctions}

The functions in this section deal with getting the current time and sleeping
for the specified time interval.


\membersection{::wxGetLocalTime}\label{wxgetlocaltime}

\func{long}{wxGetLocalTime}{\void}

Returns the number of seconds since local time 00:00:00 Jan 1st 1970.

\wxheading{See also}

\helpref{wxDateTime::Now}{wxdatetimenow}

\wxheading{Include files}

<wx/stopwatch.h>


\membersection{::wxGetLocalTimeMillis}\label{wxgetlocaltimemillis}

\func{wxLongLong}{wxGetLocalTimeMillis}{\void}

Returns the number of milliseconds since local time 00:00:00 Jan 1st 1970.

\wxheading{See also}

\helpref{wxDateTime::Now}{wxdatetimenow},\\
\helpref{wxLongLong}{wxlonglong}

\wxheading{Include files}

<wx/stopwatch.h>


\membersection{::wxGetUTCTime}\label{wxgetutctime}

\func{long}{wxGetUTCTime}{\void}

Returns the number of seconds since GMT 00:00:00 Jan 1st 1970.

\wxheading{See also}

\helpref{wxDateTime::Now}{wxdatetimenow}

\wxheading{Include files}

<wx/stopwatch.h>


\membersection{::wxMicroSleep}\label{wxmicrosleep}

\func{void}{wxMicroSleep}{\param{unsigned long}{ microseconds}}

Sleeps for the specified number of microseconds. The microsecond resolution may
not, in fact, be available on all platforms (currently only Unix platforms with
nanosleep(2) may provide it) in which case this is the same as
\helpref{wxMilliSleep}{wxmillisleep}(\arg{microseconds}$/1000$).

\wxheading{Include files}

<wx/utils.h>


\membersection{::wxMilliSleep}\label{wxmillisleep}

\func{void}{wxMilliSleep}{\param{unsigned long}{ milliseconds}}

Sleeps for the specified number of milliseconds. Notice that usage of this
function is encouraged instead of calling usleep(3) directly because the
standard usleep() function is not MT safe.

\wxheading{Include files}

<wx/utils.h>


\membersection{::wxNow}\label{wxnow}

\func{wxString}{wxNow}{\void}

Returns a string representing the current date and time.

\wxheading{Include files}

<wx/utils.h>


\membersection{::wxSleep}\label{wxsleep}

\func{void}{wxSleep}{\param{int}{ secs}}

Sleeps for the specified number of seconds.

\wxheading{Include files}

<wx/utils.h>


\membersection{::wxUsleep}\label{wxusleep}

\func{void}{wxUsleep}{\param{unsigned long}{ milliseconds}}

This function is deprecated because its name is misleading: notice that the
argument is in milliseconds, not microseconds. Please use either
\helpref{wxMilliSleep}{wxmillisleep} or \helpref{wxMicroSleep}{wxmicrosleep}
depending on the resolution you need.



\section{Debugging macros and functions}\label{debugmacros}

Useful macros and functions for error checking and defensive programming.
wxWidgets defines three families of the assert-like macros:
the wxASSERT and wxFAIL macros only do anything if \_\_WXDEBUG\_\_ is defined
(in other words, in the debug build) but disappear completely in the release
build. On the other hand, the wxCHECK macros stay event in release builds but a
check failure doesn't generate any user-visible effects then. Finally, the
compile time assertions don't happen during the run-time but result in the
compilation error messages if the condition they check fail.

\wxheading{Include files}

<wx/debug.h>


\membersection{::wxOnAssert}\label{wxonassert}

\func{void}{wxOnAssert}{\param{const char *}{fileName}, \param{int}{ lineNumber}, \param{const char *}{func}, \param{const char *}{cond}, \param{const char *}{msg = NULL}}

This function is called whenever one of debugging macros fails (i.e. condition
is false in an assertion). It is only defined in the debug mode, in release
builds the \helpref{wxCHECK}{wxcheck} failures don't result in anything.

To override the default behaviour in the debug builds which is to show the user
a dialog asking whether he wants to abort the program, continue or continue
ignoring any subsequent assert failures, you may override
\helpref{wxApp::OnAssertFailure}{wxapponassertfailure} which is called by this function if
the global application object exists.


\membersection{wxASSERT}\label{wxassert}

\func{}{wxASSERT}{\param{}{condition}}

Assert macro. An error message will be generated if the condition is false in
debug mode, but nothing will be done in the release build.

Please note that the condition in wxASSERT() should have no side effects
because it will not be executed in release mode at all.

\wxheading{See also}

\helpref{wxASSERT\_MSG}{wxassertmsg},\\
\helpref{wxCOMPILE\_TIME\_ASSERT}{wxcompiletimeassert}


\membersection{wxASSERT\_MIN\_BITSIZE}\label{wxassertminbitsize}

\func{}{wxASSERT\_MIN\_BITSIZE}{\param{}{type}, \param{}{size}}

This macro results in a
\helpref{compile time assertion failure}{wxcompiletimeassert} if the size
of the given type {\it type} is less than {\it size} bits.

You may use it like this, for example:

\begin{verbatim}
    // we rely on the int being able to hold values up to 2^32
    wxASSERT_MIN_BITSIZE(int, 32);

    // can't work with the platforms using UTF-8 for wchar_t
    wxASSERT_MIN_BITSIZE(wchar_t, 16);
\end{verbatim}


\membersection{wxASSERT\_MSG}\label{wxassertmsg}

\func{}{wxASSERT\_MSG}{\param{}{condition}, \param{}{msg}}

Assert macro with message. An error message will be generated if the condition is false.

\wxheading{See also}

\helpref{wxASSERT}{wxassert},\\
\helpref{wxCOMPILE\_TIME\_ASSERT}{wxcompiletimeassert}


\membersection{wxCOMPILE\_TIME\_ASSERT}\label{wxcompiletimeassert}

\func{}{wxCOMPILE\_TIME\_ASSERT}{\param{}{condition}, \param{}{msg}}

Using {\tt wxCOMPILE\_TIME\_ASSERT} results in a compilation error if the
specified {\it condition} is false. The compiler error message should include
the {\it msg} identifier - please note that it must be a valid C++ identifier
and not a string unlike in the other cases.

This macro is mostly useful for testing the expressions involving the
{\tt sizeof} operator as they can't be tested by the preprocessor but it is
sometimes desirable to test them at the compile time.

Note that this macro internally declares a struct whose name it tries to make
unique by using the {\tt \_\_LINE\_\_} in it but it may still not work if you
use it on the same line in two different source files. In this case you may
either change the line in which either of them appears on or use the
\helpref{wxCOMPILE\_TIME\_ASSERT2}{wxcompiletimeassert2} macro.

Also note that Microsoft Visual C++ has a bug which results in compiler errors
if you use this macro with `Program Database For Edit And Continue'
(\texttt{/ZI}) option, so you shouldn't use it (`Program Database'
(\texttt{/Zi}) is ok though) for the code making use of this macro.

\wxheading{See also}

\helpref{wxASSERT\_MSG}{wxassertmsg},\\
\helpref{wxASSERT\_MIN\_BITSIZE}{wxassertminbitsize}


\membersection{wxCOMPILE\_TIME\_ASSERT2}\label{wxcompiletimeassert2}

\func{}{wxCOMPILE\_TIME\_ASSERT}{\param{}{condition}, \param{}{msg}, \param{}{name}}

This macro is identical to \helpref{wxCOMPILE\_TIME\_ASSERT2}{wxcompiletimeassert2}
except that it allows you to specify a unique {\it name} for the struct
internally defined by this macro to avoid getting the compilation errors
described \helpref{above}{wxcompiletimeassert}.


\membersection{wxFAIL}\label{wxfail}

\func{}{wxFAIL}{\void}

Will always generate an assert error if this code is reached (in debug mode).

See also: \helpref{wxFAIL\_MSG}{wxfailmsg}


\membersection{wxFAIL\_MSG}\label{wxfailmsg}

\func{}{wxFAIL\_MSG}{\param{}{msg}}

Will always generate an assert error with specified message if this code is reached (in debug mode).

This macro is useful for marking unreachable" code areas, for example
it may be used in the "default:" branch of a switch statement if all possible
cases are processed above.

\wxheading{See also}

\helpref{wxFAIL}{wxfail}


\membersection{wxCHECK}\label{wxcheck}

\func{}{wxCHECK}{\param{}{condition}, \param{}{retValue}}

Checks that the condition is true, returns with the given return value if not (FAILs in debug mode).
This check is done even in release mode.


\membersection{wxCHECK\_MSG}\label{wxcheckmsg}

\func{}{wxCHECK\_MSG}{\param{}{condition}, \param{}{retValue}, \param{}{msg}}

Checks that the condition is true, returns with the given return value if not (FAILs in debug mode).
This check is done even in release mode.

This macro may be only used in non-void functions, see also
\helpref{wxCHECK\_RET}{wxcheckret}.


\membersection{wxCHECK\_RET}\label{wxcheckret}

\func{}{wxCHECK\_RET}{\param{}{condition}, \param{}{msg}}

Checks that the condition is true, and returns if not (FAILs with given error
message in debug mode). This check is done even in release mode.

This macro should be used in void functions instead of
\helpref{wxCHECK\_MSG}{wxcheckmsg}.


\membersection{wxCHECK2}\label{wxcheck2}

\func{}{wxCHECK2}{\param{}{condition}, \param{}{operation}}

Checks that the condition is true and \helpref{wxFAIL}{wxfail} and execute
{\it operation} if it is not. This is a generalisation of
\helpref{wxCHECK}{wxcheck} and may be used when something else than just
returning from the function must be done when the {\it condition} is false.

This check is done even in release mode.


\membersection{wxCHECK2\_MSG}\label{wxcheck2msg}

\func{}{wxCHECK2}{\param{}{condition}, \param{}{operation}, \param{}{msg}}

This is the same as \helpref{wxCHECK2}{wxcheck2}, but
\helpref{wxFAIL\_MSG}{wxfailmsg} with the specified {\it msg} is called
instead of wxFAIL() if the {\it condition} is false.


\membersection{::wxTrap}\label{wxtrap}

\func{void}{wxTrap}{\void}

In debug mode (when {\tt \_\_WXDEBUG\_\_} is defined) this function generates a
debugger exception meaning that the control is passed to the debugger if one is
attached to the process. Otherwise the program just terminates abnormally.

In release mode this function does nothing.

\wxheading{Include files}

<wx/debug.h>



\membersection{::wxIsDebuggerRunning}\label{wxisdebuggerrunning}

\func{bool}{wxIsDebuggerRunning}{\void}

Returns \true if the program is running under debugger, \false otherwise.

Please note that this function is currently only implemented for Win32 and Mac
builds using CodeWarrior and always returns \false elsewhere.




\section{Environment access functions}\label{environfunctions}

The functions in this section allow to access (get) or change value of
environment variables in a portable way. They are currently implemented under
Win32 and POSIX-like systems (Unix).

% TODO add some stuff about env var inheriting but not propagating upwards (VZ)

\wxheading{Include files}

<wx/utils.h>


\membersection{wxGetenv}\label{wxgetenvmacro}

\func{wxChar *}{wxGetEnv}{\param{const wxString\&}{ var}}

This is a macro defined as {\tt getenv()} or its wide char version in Unicode
mode.

Note that under Win32 it may not return correct value for the variables set
with \helpref{wxSetEnv}{wxsetenv}, use \helpref{wxGetEnv}{wxgetenv} function
instead.


\membersection{wxGetEnv}\label{wxgetenv}

\func{bool}{wxGetEnv}{\param{const wxString\&}{ var}, \param{wxString *}{value}}

Returns the current value of the environment variable {\it var} in {\it value}.
{\it value} may be {\tt NULL} if you just want to know if the variable exists
and are not interested in its value.

Returns \true if the variable exists, \false otherwise.


\membersection{wxSetEnv}\label{wxsetenv}

\func{bool}{wxSetEnv}{\param{const wxString\&}{ var}, \param{const wxString\& }{value}}

Sets the value of the environment variable {\it var} (adding it if necessary)
to {\it value}.

Returns \true on success.

\wxheading{See also}

\helpref{wxUnsetEnv}{wxunsetenv}


\membersection{wxUnsetEnv}\label{wxunsetenv}

\func{bool}{wxUnsetEnv}{\param{const wxString\&}{ var}}

Removes the variable {\it var} from the environment.
\helpref{wxGetEnv}{wxgetenv} will return {\tt NULL} after the call to this
function.

Returns \true on success.

\wxheading{See also}

\helpref{wxSetEnv}{wxsetenv}


\section{Atomic operations}\label{atomicoperations}

When using multi-threaded applications, it is often required to access or
modify memory which is shared between threads. Atomic integer and pointer
operations are an efficient way to handle this issue (another, less efficient,
way is to use a \helpref{mutex}{wxmutex} or \helpref{critical
section}{wxcriticalsection}). A native implementation exists for Windows,
Linux, Solaris and Mac OS X, for other OS, a 
\helpref{wxCriticalSection}{wxcriticalsection} is used to protect the data.

One particular application is reference counting (used by so-called smart
pointers).

You should define your variable with the type wxAtomicInt in order to apply
atomic operations to it.

\wxheading{Include files}

<wx/atomic.h>

\membersection{::wxAtomicInc}\label{wxatomicinc}

\func{void}{wxAtomicInc}{\param{wxAtomicInt\& }{value}}

This function increments \arg{value} in an atomic manner.


\membersection{::wxAtomicDec}\label{wxatomicdec}

\func{wxInt32}{wxAtomicDec}{\param{wxAtomicInt\& }{value}}

This function decrements \arg{value} in an atomic manner.

Returns 0 if \arg{value} is 0 after decrementation or any non-zero value (not
necessarily equal to the value of the variable) otherwise.


