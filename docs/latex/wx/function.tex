\chapter{Functions}\label{functions}
\setheader{{\it CHAPTER \thechapter}}{}{}{}{}{{\it CHAPTER \thechapter}}%
\setfooter{\thepage}{}{}{}{}{\thepage}

The functions defined in wxWindows are described here.

\section{File functions}\label{filefunctions}

\wxheading{Include files}

<wx/utils.h>

\wxheading{See also}

\helpref{wxPathList}{wxpathlist}

\membersection{::wxDirExists}

\func{bool}{wxDirExists}{\param{const wxString\& }{dirname}}

Returns TRUE if the directory exists.

\membersection{::wxDos2UnixFilename}

\func{void}{Dos2UnixFilename}{\param{const wxString\& }{s}}

Converts a DOS to a Unix filename by replacing backslashes with forward
slashes.

\membersection{::wxFileExists}

\func{bool}{wxFileExists}{\param{const wxString\& }{filename}}

Returns TRUE if the file exists.

\membersection{::wxFileNameFromPath}

\func{wxString}{wxFileNameFromPath}{\param{const wxString\& }{path}}

\func{char*}{wxFileNameFromPath}{\param{char* }{path}}

Returns the filename for a full path. The second form returns a pointer to
temporary storage that should not be deallocated.

\membersection{::wxFindFirstFile}\label{wxfindfirstfile}

\func{wxString}{wxFindFirstFile}{\param{const char*}{spec}, \param{int}{ flags = 0}}

This function does directory searching; returns the first file
that matches the path {\it spec}, or the empty string. Use \helpref{wxFindNextFile}{wxfindnextfile} to
get the next matching file. Neither will report the current directory "." or the
parent directory "..".

{\it spec} may contain wildcards.

{\it flags} may be wxDIR for restricting the query to directories, wxFILE for files or zero for either.

For example:

\begin{verbatim}
  wxString f = wxFindFirstFile("/home/project/*.*");
  while ( !f.IsEmpty() )
  {
    ...
    f = wxFindNextFile();
  }
\end{verbatim}

\membersection{::wxFindNextFile}\label{wxfindnextfile}

\func{wxString}{wxFindNextFile}{\void}

Returns the next file that matches the path passed to \helpref{wxFindFirstFile}{wxfindfirstfile}.

See \helpref{wxFindFirstFile}{wxfindfirstfile} for an example.

\membersection{::wxGetOSDirectory}\label{wxgetosdirectory}

\func{wxString}{wxGetOSDirectory}{\void}

Returns the Windows directory under Windows; on other platforms returns the empty string.

\membersection{::wxIsAbsolutePath}

\func{bool}{wxIsAbsolutePath}{\param{const wxString\& }{filename}}

Returns TRUE if the argument is an absolute filename, i.e. with a slash
or drive name at the beginning.

\membersection{::wxPathOnly}

\func{wxString}{wxPathOnly}{\param{const wxString\& }{path}}

Returns the directory part of the filename.

\membersection{::wxUnix2DosFilename}

\func{void}{wxUnix2DosFilename}{\param{const wxString\& }{s}}

Converts a Unix to a DOS filename by replacing forward
slashes with backslashes.

\membersection{::wxConcatFiles}

\func{bool}{wxConcatFiles}{\param{const wxString\& }{file1}, \param{const wxString\& }{file2},
\param{const wxString\& }{file3}}

Concatenates {\it file1} and {\it file2} to {\it file3}, returning
TRUE if successful.

\membersection{::wxCopyFile}

\func{bool}{wxCopyFile}{\param{const wxString\& }{file1}, \param{const wxString\& }{file2}}

Copies {\it file1} to {\it file2}, returning TRUE if successful.

\membersection{::wxGetCwd}\label{wxgetcwd}

\func{wxString}{wxGetCwd}{\void}

Returns a string containing the current (or working) directory.

\membersection{::wxGetWorkingDirectory}

\func{wxString}{wxGetWorkingDirectory}{\param{char*}{buf=NULL}, \param{int }{sz=1000}}

This function is obsolete: use \helpref{wxGetCwd}{wxgetcwd} instead.

Copies the current working directory into the buffer if supplied, or
copies the working directory into new storage (which you must delete yourself)
if the buffer is NULL.

{\it sz} is the size of the buffer if supplied.

\membersection{::wxGetTempFileName}

\func{char*}{wxGetTempFileName}{\param{const wxString\& }{prefix}, \param{char* }{buf=NULL}}

Makes a temporary filename based on {\it prefix}, opens and closes the file,
and places the name in {\it buf}. If {\it buf} is NULL, new store
is allocated for the temporary filename using {\it new}.

Under Windows, the filename will include the drive and name of the
directory allocated for temporary files (usually the contents of the
TEMP variable). Under Unix, the {\tt /tmp} directory is used.

It is the application's responsibility to create and delete the file.

\membersection{::wxIsWild}\label{wxiswild}

\func{bool}{wxIsWild}{\param{const wxString\& }{pattern}}

Returns TRUE if the pattern contains wildcards. See \helpref{wxMatchWild}{wxmatchwild}.

\membersection{::wxMatchWild}\label{wxmatchwild}

\func{bool}{wxMatchWild}{\param{const wxString\& }{pattern}, \param{const wxString\& }{text}, \param{bool}{ dot\_special}}

Returns TRUE if the {\it pattern}\/ matches the {\it text}\/; if {\it
dot\_special}\/ is TRUE, filenames beginning with a dot are not matched
with wildcard characters. See \helpref{wxIsWild}{wxiswild}.

\membersection{::wxMkdir}

\func{bool}{wxMkdir}{\param{const wxString\& }{dir}, \param{int }{perm = 0777}}

Makes the directory {\it dir}, returning TRUE if successful.

{\it perm} is the access mask for the directory for the systems on which it is
supported (Unix) and doesn't have effect for the other ones.

\membersection{::wxRemoveFile}

\func{bool}{wxRemoveFile}{\param{const wxString\& }{file}}

Removes {\it file}, returning TRUE if successful.

\membersection{::wxRenameFile}

\func{bool}{wxRenameFile}{\param{const wxString\& }{file1}, \param{const wxString\& }{file2}}

Renames {\it file1} to {\it file2}, returning TRUE if successful.

\membersection{::wxRmdir}

\func{bool}{wxRmdir}{\param{const wxString\& }{dir}, \param{int}{ flags=0}}

Removes the directory {\it dir}, returning TRUE if successful. Does not work under VMS.

The {\it flags} parameter is reserved for future use.

\membersection{::wxSetWorkingDirectory}

\func{bool}{wxSetWorkingDirectory}{\param{const wxString\& }{dir}}

Sets the current working directory, returning TRUE if the operation succeeded.
Under MS Windows, the current drive is also changed if {\it dir} contains a drive specification.

\membersection{::wxSplitPath}\label{wxsplitfunction}

\func{void}{wxSplitPath}{\param{const char *}{ fullname}, \param{const wxString *}{ path}, \param{const wxString *}{ name}, \param{const wxString *}{ ext}}

This function splits a full file name into components: the path (including possible disk/drive
specification under Windows), the base name and the extension. Any of the output parameters
({\it path}, {\it name} or {\it ext}) may be NULL if you are not interested in the value of
a particular component.

wxSplitPath() will correctly handle filenames with both DOS and Unix path separators under
Windows, however it will not consider backslashes as path separators under Unix (where backslash
is a valid character in a filename).

On entry, {\it fullname} should be non NULL (it may be empty though).

On return, {\it path} contains the file path (without the trailing separator), {\it name}
contains the file name and {\it ext} contains the file extension without leading dot. All
three of them may be empty if the corresponding component is. The old contents of the
strings pointed to by these parameters will be overwritten in any case (if the pointers
are not NULL).

\membersection{::wxTransferFileToStream}\label{wxtransferfiletostream}

\func{bool}{wxTransferFileToStream}{\param{const wxString\& }{filename}, \param{ostream\& }{stream}}

Copies the given file to {\it stream}. Useful when converting an old application to
use streams (within the document/view framework, for example).

Use of this function requires the file wx\_doc.h to be included.

\membersection{::wxTransferStreamToFile}\label{wxtransferstreamtofile}

\func{bool}{wxTransferStreamToFile}{\param{istream\& }{stream} \param{const wxString\& }{filename}}

Copies the given stream to the file {\it filename}. Useful when converting an old application to
use streams (within the document/view framework, for example).

Use of this function requires the file wx\_doc.h to be included.

\section{Network functions}\label{networkfunctions}

\membersection{::wxGetFullHostName}\label{wxgetfullhostname}

\func{wxString}{wxGetFullHostName}{\void}

Returns the FQDN (fully qualified domain host name) or an empty string on
error.

See also: \helpref{wxGetHostName}{wxgethostname}

\wxheading{Include files}

<wx/utils.h>

\membersection{::wxGetEmailAddress}\label{wxgetemailaddress}

\func{bool}{wxGetEmailAddress}{\param{const wxString\& }{buf}, \param{int }{sz}}

Copies the user's email address into the supplied buffer, by
concatenating the values returned by \helpref{wxGetFullHostName}{wxgetfullhostname}\rtfsp
and \helpref{wxGetUserId}{wxgetuserid}.

Returns TRUE if successful, FALSE otherwise.

\wxheading{Include files}

<wx/utils.h>

\membersection{::wxGetHostName}\label{wxgethostname}

\func{wxString}{wxGetHostName}{\void}
\func{bool}{wxGetHostName}{\param{char * }{buf}, \param{int }{sz}}

Copies the current host machine's name into the supplied buffer. Please note
that the returned name is {\it not} fully qualified, i.e. it does not include
the domain name.

Under Windows or NT, this function first looks in the environment
variable SYSTEM\_NAME; if this is not found, the entry {\bf HostName}\rtfsp
in the {\bf wxWindows} section of the WIN.INI file is tried.

The first variant of this function returns the hostname if successful or an
empty string otherwise. The second (deprecated) function returns TRUE
if successful, FALSE otherwise.

See also: \helpref{wxGetFullHostName}{wxgetfullhostname}

\wxheading{Include files}

<wx/utils.h>

\section{User identification}\label{useridfunctions}

\membersection{::wxGetUserId}\label{wxgetuserid}

\func{wxString}{wxGetUserId}{\void}
\func{bool}{wxGetUserId}{\param{char * }{buf}, \param{int }{sz}}

This function returns the "user id" also known as "login name" under Unix i.e.
something like "jsmith". It uniquely identifies the current user (on this system).

Under Windows or NT, this function first looks in the environment
variables USER and LOGNAME; if neither of these is found, the entry {\bf UserId}\rtfsp
in the {\bf wxWindows} section of the WIN.INI file is tried.

The first variant of this function returns the login name if successful or an
empty string otherwise. The second (deprecated) function returns TRUE
if successful, FALSE otherwise.

See also: \helpref{wxGetUserName}{wxgetusername}

\wxheading{Include files}

<wx/utils.h>

\membersection{::wxGetUserName}\label{wxgetusername}

\func{wxString}{wxGetUserName}{\void}
\func{bool}{wxGetUserName}{\param{char * }{buf}, \param{int }{sz}}

This function returns the full user name (something like "Mr. John Smith").

Under Windows or NT, this function looks for the entry {\bf UserName}\rtfsp
in the {\bf wxWindows} section of the WIN.INI file. If PenWindows
is running, the entry {\bf Current} in the section {\bf User} of
the PENWIN.INI file is used.

The first variant of this function returns the user name if successful or an
empty string otherwise. The second (deprecated) function returns TRUE
if successful, FALSE otherwise.

See also: \helpref{wxGetUserId}{wxgetuserid}

\wxheading{Include files}

<wx/utils.h>

\section{String functions}

\membersection{::copystring}

\func{char*}{copystring}{\param{const char* }{s}}

Makes a copy of the string {\it s} using the C++ new operator, so it can be
deleted with the {\it delete} operator.

\membersection{::wxStringMatch}

\func{bool}{wxStringMatch}{\param{const wxString\& }{s1}, \param{const wxString\& }{s2},\\
  \param{bool}{ subString = TRUE}, \param{bool}{ exact = FALSE}}

Returns TRUE if the substring {\it s1} is found within {\it s2},
ignoring case if {\it exact} is FALSE. If {\it subString} is FALSE,
no substring matching is done.

\membersection{::wxStringEq}\label{wxstringeq}

\func{bool}{wxStringEq}{\param{const wxString\& }{s1}, \param{const wxString\& }{s2}}

A macro defined as:

\begin{verbatim}
#define wxStringEq(s1, s2) (s1 && s2 && (strcmp(s1, s2) == 0))
\end{verbatim}

\membersection{::IsEmpty}\label{isempty}

\func{bool}{IsEmpty}{\param{const char *}{ p}}

Returns TRUE if the string is empty, FALSE otherwise. It is safe to pass NULL
pointer to this function and it will return TRUE for it.

\membersection{::Stricmp}\label{stricmp}

\func{int}{Stricmp}{\param{const char *}{p1}, \param{const char *}{p2}}

Returns a negative value, 0, or positive value if {\it p1} is less than, equal
to or greater than {\it p2}. The comparison is case-insensitive.

This function complements the standard C function {\it strcmp()} which performs
case-sensitive comparison.

\membersection{::Strlen}\label{strlen}

\func{size\_t}{Strlen}{\param{const char *}{ p}}

This is a safe version of standard function {\it strlen()}: it does exactly the
same thing (i.e. returns the length of the string) except that it returns 0 if
{\it p} is the NULL pointer.

\membersection{::wxGetTranslation}\label{wxgettranslation}

\func{const char *}{wxGetTranslation}{\param{const char * }{str}}

This function returns the translation of string {\it str} in the current 
\helpref{locale}{wxlocale}. If the string is not found in any of the loaded
message catalogs (see \helpref{i18n overview}{internationalization}), the
original string is returned. In debug build, an error message is logged - this
should help to find the strings which were not yet translated. As this function
is used very often, an alternative syntax is provided: the \_() macro is
defined as wxGetTranslation().

\section{Dialog functions}\label{dialogfunctions}

Below are a number of convenience functions for getting input from the
user or displaying messages. Note that in these functions the last three
parameters are optional. However, it is recommended to pass a parent frame
parameter, or (in MS Windows or Motif) the wrong window frame may be brought to
the front when the dialog box is popped up.

\membersection{::wxCreateFileTipProvider}\label{wxcreatefiletipprovider}

\func{wxTipProvider *}{wxCreateFileTipProvider}{
    \param{const wxString\& }{filename},
    \param{size\_t }{currentTip}}

This function creates a \helpref{wxTipProvider}{wxtipprovider} which may be
used with \helpref{wxShowTip}{wxshowtip}.

\docparam{filename}{The name of the file containing the tips, one per line}
\docparam{currentTip}{The index of the first tip to show - normally this index
    is remembered between the 2 program runs.}

\wxheading{See also:}

\helpref{Tips overview}{tipsoverview}

\wxheading{Include files}

<wx/tipdlg.h>

\membersection{::wxFileSelector}\label{wxfileselector}

\func{wxString}{wxFileSelector}{\param{const wxString\& }{message}, \param{const wxString\& }{default\_path = ""},\\
  \param{const wxString\& }{default\_filename = ""}, \param{const wxString\& }{default\_extension = ""},\\
  \param{const wxString\& }{wildcard = ``*.*''}, \param{int }{flags = 0}, \param{wxWindow *}{parent = ""},\\
  \param{int}{ x = -1}, \param{int}{ y = -1}}

Pops up a file selector box. In Windows, this is the common file selector
dialog. In X, this is a file selector box with somewhat less functionality.
The path and filename are distinct elements of a full file pathname.
If path is empty, the current directory will be used. If filename is empty,
no default filename will be supplied. The wildcard determines what files
are displayed in the file selector, and file extension supplies a type
extension for the required filename. Flags may be a combination of wxOPEN,
wxSAVE, wxOVERWRITE\_PROMPT, wxHIDE\_READONLY, or 0.

Both the Unix and Windows versions implement a wildcard filter. Typing a
filename containing wildcards (*, ?) in the filename text item, and
clicking on Ok, will result in only those files matching the pattern being
displayed.

The wildcard may be a specification for multiple types of file 
with a description for each, such as:

\begin{verbatim}
 "BMP files (*.bmp)|*.bmp|GIF files (*.gif)|*.gif"
\end{verbatim}

The application must check for an empty return value (the user pressed
Cancel). For example:

\begin{verbatim}
const wxString& s = wxFileSelector("Choose a file to open");
if (s)
{
  ...
}
\end{verbatim}

\wxheading{Include files}

<wx/filedlg.h>

\membersection{::wxGetNumberFromUser}\label{wxgetnumberfromuser}

\func{long}{wxGetNumberFromUser}{
    \param{const wxString\& }{message},
    \param{const wxString\& }{prompt},
    \param{const wxString\& }{caption},
    \param{long }{value},
    \param{long }{min = 0},
    \param{long }{max = 100},
    \param{wxWindow *}{parent = NULL},
    \param{const wxPoint\& }{pos = wxDefaultPosition}}

Shows a dialog asking the user for numeric input. The dialogs title is set to 
{\it caption}, it contains a (possibly) multiline {\it message} above the
single line {\it prompt} and the zone for entering the number.

The number entered must be in the range {\it min}..{\it max} (both of which
should be positive) and {\it value} is the initial value of it. If the user
enters an invalid value or cancels the dialog, the function will return -1.

Dialog is centered on its {\it parent} unless an explicit position is given in 
{\it pos}.

\wxheading{Include files}

<wx/textdlg.h>

\membersection{::wxGetTextFromUser}\label{wxgettextfromuser}

\func{wxString}{wxGetTextFromUser}{\param{const wxString\& }{message}, \param{const wxString\& }{caption = ``Input text"},\\
  \param{const wxString\& }{default\_value = ``"}, \param{wxWindow *}{parent = NULL},\\
  \param{int}{ x = -1}, \param{int}{ y = -1}, \param{bool}{ centre = TRUE}}

Pop up a dialog box with title set to {\it caption}, message {\it message}, and a
\rtfsp{\it default\_value}.  The user may type in text and press OK to return this text,
or press Cancel to return the empty string.

If {\it centre} is TRUE, the message text (which may include new line characters)
is centred; if FALSE, the message is left-justified.

\wxheading{Include files}

<wx/textdlg.h>

\membersection{::wxGetMultipleChoice}\label{wxgetmultiplechoice}

\func{int}{wxGetMultipleChoice}{\param{const wxString\& }{message}, \param{const wxString\& }{caption}, \param{int}{ n}, \param{const wxString\& }{choices[]},\\
  \param{int }{nsel}, \param{int *}{selection},
  \param{wxWindow *}{parent = NULL}, \param{int}{ x = -1}, \param{int}{ y = -1},\\
  \param{bool}{ centre = TRUE}, \param{int }{width=150}, \param{int }{height=200}}

Pops up a dialog box containing a message, OK/Cancel buttons and a multiple-selection
listbox. The user may choose one or more item(s) and press OK or Cancel.

The number of initially selected choices, and array of the selected indices,
are passed in; this array will contain the user selections on exit, with
the function returning the number of selections. {\it selection} must be
as big as the number of choices, in case all are selected.

If Cancel is pressed, -1 is returned.

{\it choices} is an array of {\it n} strings for the listbox.

If {\it centre} is TRUE, the message text (which may include new line characters)
is centred; if FALSE, the message is left-justified.

\wxheading{Include files}

<wx/choicdlg.h>

\membersection{::wxGetSingleChoice}\label{wxgetsinglechoice}

\func{wxString}{wxGetSingleChoice}{\param{const wxString\& }{message}, \param{const wxString\& }{caption}, \param{int}{ n}, \param{const wxString\& }{choices[]},\\
  \param{wxWindow *}{parent = NULL}, \param{int}{ x = -1}, \param{int}{ y = -1},\\
  \param{bool}{ centre = TRUE}, \param{int }{width=150}, \param{int }{height=200}}

Pops up a dialog box containing a message, OK/Cancel buttons and a single-selection
listbox. The user may choose an item and press OK to return a string or
Cancel to return the empty string.

{\it choices} is an array of {\it n} strings for the listbox.

If {\it centre} is TRUE, the message text (which may include new line characters)
is centred; if FALSE, the message is left-justified.

\wxheading{Include files}

<wx/choicdlg.h>

\membersection{::wxGetSingleChoiceIndex}\label{wxgetsinglechoiceindex}

\func{int}{wxGetSingleChoiceIndex}{\param{const wxString\& }{message}, \param{const wxString\& }{caption}, \param{int}{ n}, \param{const wxString\& }{choices[]},\\
  \param{wxWindow *}{parent = NULL}, \param{int}{ x = -1}, \param{int}{ y = -1},\\
  \param{bool}{ centre = TRUE}, \param{int }{width=150}, \param{int }{height=200}}

As {\bf wxGetSingleChoice} but returns the index representing the selected string.
If the user pressed cancel, -1 is returned.

\wxheading{Include files}

<wx/choicdlg.h>

\membersection{::wxGetSingleChoiceData}\label{wxgetsinglechoicedata}

\func{wxString}{wxGetSingleChoiceData}{\param{const wxString\& }{message}, \param{const wxString\& }{caption}, \param{int}{ n}, \param{const wxString\& }{choices[]},\\
  \param{const wxString\& }{client\_data[]}, \param{wxWindow *}{parent = NULL}, \param{int}{ x = -1},\\
  \param{int}{ y = -1}, \param{bool}{ centre = TRUE}, \param{int }{width=150}, \param{int }{height=200}}

As {\bf wxGetSingleChoice} but takes an array of client data pointers
corresponding to the strings, and returns one of these pointers.

\wxheading{Include files}

<wx/choicdlg.h>

\membersection{::wxMessageBox}\label{wxmessagebox}

\func{int}{wxMessageBox}{\param{const wxString\& }{message}, \param{const wxString\& }{caption = ``Message"}, \param{int}{ style = wxOK \pipe wxCENTRE},\\
  \param{wxWindow *}{parent = NULL}, \param{int}{ x = -1}, \param{int}{ y = -1}}

General purpose message dialog.  {\it style} may be a bit list of the
following identifiers:

\begin{twocollist}\itemsep=0pt
\twocolitem{wxYES\_NO}{Puts Yes and No buttons on the message box. May be combined with
wxCANCEL.}
\twocolitem{wxCANCEL}{Puts a Cancel button on the message box. May be combined with
wxYES\_NO or wxOK.}
\twocolitem{wxOK}{Puts an Ok button on the message box. May be combined with wxCANCEL.}
\twocolitem{wxCENTRE}{Centres the text.}
\twocolitem{wxICON\_EXCLAMATION}{Under Windows, displays an exclamation mark symbol.}
\twocolitem{wxICON\_HAND}{Under Windows, displays a hand symbol.}
\twocolitem{wxICON\_QUESTION}{Under Windows, displays a question mark symbol.}
\twocolitem{wxICON\_INFORMATION}{Under Windows, displays an information symbol.}
\end{twocollist}

The return value is one of: wxYES, wxNO, wxCANCEL, wxOK.

For example:

\begin{verbatim}
  ...
  int answer = wxMessageBox("Quit program?", "Confirm",
                            wxYES_NO | wxCANCEL, main_frame);
  if (answer == wxYES)
    delete main_frame;
  ...
\end{verbatim}

{\it message} may contain newline characters, in which case the
message will be split into separate lines, to cater for large messages.

Under Windows, the native MessageBox function is used unless wxCENTRE
is specified in the style, in which case a generic function is used.
This is because the native MessageBox function cannot centre text.
The symbols are not shown when the generic function is used.

\wxheading{Include files}

<wx/msgdlg.h>

\membersection{::wxShowTip}\label{wxshowtip}

\func{bool}{wxShowTip}{
    \param{wxWindow *}{parent},
    \param{wxTipProvider *}{tipProvider},
    \param{bool }{showAtStartup = TRUE}}

This function shows a "startup tip" to the user.

\docparam{parent}{The parent window for the modal dialog}

\docparam{tipProvider}{An object which is used to get the text of the tips.
    It may be created with 
    \helpref{wxCreateFileTipProvider}{wxcreatefiletipprovider} function.}

\docparam{showAtStartup}{Should be TRUE if startup tips are shown, FALSE
    otherwise. This is used as the initial value for "Show tips at startup"
    checkbox which is shown in the tips dialog.}

\wxheading{See also:}

\helpref{Tips overview}{tipsoverview}

\wxheading{Include files}

<wx/tipdlg.h>

\section{GDI functions}\label{gdifunctions}

The following are relevant to the GDI (Graphics Device Interface).

\wxheading{Include files}

<wx/gdicmn.h>

\membersection{::wxColourDisplay}

\func{bool}{wxColourDisplay}{\void}

Returns TRUE if the display is colour, FALSE otherwise.

\membersection{::wxDisplayDepth}

\func{int}{wxDisplayDepth}{\void}

Returns the depth of the display (a value of 1 denotes a monochrome display).

\membersection{::wxMakeMetafilePlaceable}\label{wxmakemetafileplaceable}

\func{bool}{wxMakeMetafilePlaceable}{\param{const wxString\& }{filename}, \param{int }{minX}, \param{int }{minY},
 \param{int }{maxX}, \param{int }{maxY}, \param{float }{scale=1.0}}

Given a filename for an existing, valid metafile (as constructed using \helpref{wxMetafileDC}{wxmetafiledc})
makes it into a placeable metafile by prepending a header containing the given
bounding box. The bounding box may be obtained from a device context after drawing
into it, using the functions wxDC::MinX, wxDC::MinY, wxDC::MaxX and wxDC::MaxY.

In addition to adding the placeable metafile header, this function adds
the equivalent of the following code to the start of the metafile data:

\begin{verbatim}
 SetMapMode(dc, MM_ANISOTROPIC);
 SetWindowOrg(dc, minX, minY);
 SetWindowExt(dc, maxX - minX, maxY - minY);
\end{verbatim}

This simulates the wxMM\_TEXT mapping mode, which wxWindows assumes.

Placeable metafiles may be imported by many Windows applications, and can be
used in RTF (Rich Text Format) files.

{\it scale} allows the specification of scale for the metafile.

This function is only available under Windows.

\membersection{::wxSetCursor}\label{wxsetcursor}

\func{void}{wxSetCursor}{\param{wxCursor *}{cursor}}

Globally sets the cursor; only has an effect in Windows and GTK.
See also \helpref{wxCursor}{wxcursor}, \helpref{wxWindow::SetCursor}{wxwindowsetcursor}.

\section{Printer settings}\label{printersettings}

These routines are obsolete and should no longer be used!

The following functions are used to control PostScript printing. Under
Windows, PostScript output can only be sent to a file.

\wxheading{Include files}

<wx/dcps.h>

\membersection{::wxGetPrinterCommand}

\func{wxString}{wxGetPrinterCommand}{\void}

Gets the printer command used to print a file. The default is {\tt lpr}.

\membersection{::wxGetPrinterFile}

\func{wxString}{wxGetPrinterFile}{\void}

Gets the PostScript output filename.

\membersection{::wxGetPrinterMode}

\func{int}{wxGetPrinterMode}{\void}

Gets the printing mode controlling where output is sent (PS\_PREVIEW, PS\_FILE or PS\_PRINTER).
The default is PS\_PREVIEW.

\membersection{::wxGetPrinterOptions}

\func{wxString}{wxGetPrinterOptions}{\void}

Gets the additional options for the print command (e.g. specific printer). The default is nothing.

\membersection{::wxGetPrinterOrientation}

\func{int}{wxGetPrinterOrientation}{\void}

Gets the orientation (PS\_PORTRAIT or PS\_LANDSCAPE). The default is PS\_PORTRAIT.

\membersection{::wxGetPrinterPreviewCommand}

\func{wxString}{wxGetPrinterPreviewCommand}{\void}

Gets the command used to view a PostScript file. The default depends on the platform.

\membersection{::wxGetPrinterScaling}

\func{void}{wxGetPrinterScaling}{\param{float *}{x}, \param{float *}{y}}

Gets the scaling factor for PostScript output. The default is 1.0, 1.0.

\membersection{::wxGetPrinterTranslation}

\func{void}{wxGetPrinterTranslation}{\param{float *}{x}, \param{float *}{y}}

Gets the translation (from the top left corner) for PostScript output. The default is 0.0, 0.0.

\membersection{::wxSetPrinterCommand}

\func{void}{wxSetPrinterCommand}{\param{const wxString\& }{command}}

Sets the printer command used to print a file. The default is {\tt lpr}.

\membersection{::wxSetPrinterFile}

\func{void}{wxSetPrinterFile}{\param{const wxString\& }{filename}}

Sets the PostScript output filename.

\membersection{::wxSetPrinterMode}

\func{void}{wxSetPrinterMode}{\param{int }{mode}}

Sets the printing mode controlling where output is sent (PS\_PREVIEW, PS\_FILE or PS\_PRINTER).
The default is PS\_PREVIEW.

\membersection{::wxSetPrinterOptions}

\func{void}{wxSetPrinterOptions}{\param{const wxString\& }{options}}

Sets the additional options for the print command (e.g. specific printer). The default is nothing.

\membersection{::wxSetPrinterOrientation}

\func{void}{wxSetPrinterOrientation}{\param{int}{ orientation}}

Sets the orientation (PS\_PORTRAIT or PS\_LANDSCAPE). The default is PS\_PORTRAIT.

\membersection{::wxSetPrinterPreviewCommand}

\func{void}{wxSetPrinterPreviewCommand}{\param{const wxString\& }{command}}

Sets the command used to view a PostScript file. The default depends on the platform.

\membersection{::wxSetPrinterScaling}

\func{void}{wxSetPrinterScaling}{\param{float }{x}, \param{float }{y}}

Sets the scaling factor for PostScript output. The default is 1.0, 1.0.

\membersection{::wxSetPrinterTranslation}

\func{void}{wxSetPrinterTranslation}{\param{float }{x}, \param{float }{y}}

Sets the translation (from the top left corner) for PostScript output. The default is 0.0, 0.0.

\section{Clipboard functions}\label{clipsboard}

These clipboard functions are implemented for Windows only.

\wxheading{Include files}

<wx/clipbrd.h>

\membersection{::wxClipboardOpen}

\func{bool}{wxClipboardOpen}{\void}

Returns TRUE if this application has already opened the clipboard.

\membersection{::wxCloseClipboard}

\func{bool}{wxCloseClipboard}{\void}

Closes the clipboard to allow other applications to use it.

\membersection{::wxEmptyClipboard}

\func{bool}{wxEmptyClipboard}{\void}

Empties the clipboard.

\membersection{::wxEnumClipboardFormats}

\func{int}{wxEnumClipboardFormats}{\param{int}{dataFormat}}

Enumerates the formats found in a list of available formats that belong
to the clipboard. Each call to this  function specifies a known
available format; the function returns the format that appears next in
the list. 

{\it dataFormat} specifies a known format. If this parameter is zero,
the function returns the first format in the list. 

The return value specifies the next known clipboard data format if the
function is successful. It is zero if the {\it dataFormat} parameter specifies
the last  format in the list of available formats, or if the clipboard
is not open. 

Before it enumerates the formats function, an application must open the clipboard by using the 
wxOpenClipboard function. 

\membersection{::wxGetClipboardData}

\func{wxObject *}{wxGetClipboardData}{\param{int}{dataFormat}}

Gets data from the clipboard.

{\it dataFormat} may be one of:

\begin{itemize}\itemsep=0pt
\item wxCF\_TEXT or wxCF\_OEMTEXT: returns a pointer to new memory containing a null-terminated text string.
\item wxCF\_BITMAP: returns a new wxBitmap.
\end{itemize}

The clipboard must have previously been opened for this call to succeed.

\membersection{::wxGetClipboardFormatName}

\func{bool}{wxGetClipboardFormatName}{\param{int}{dataFormat}, \param{const wxString\& }{formatName}, \param{int}{maxCount}}

Gets the name of a registered clipboard format, and puts it into the buffer {\it formatName} which is of maximum
length {\it maxCount}. {\it dataFormat} must not specify a predefined clipboard format.

\membersection{::wxIsClipboardFormatAvailable}

\func{bool}{wxIsClipboardFormatAvailable}{\param{int}{dataFormat}}

Returns TRUE if the given data format is available on the clipboard.

\membersection{::wxOpenClipboard}

\func{bool}{wxOpenClipboard}{\void}

Opens the clipboard for passing data to it or getting data from it.

\membersection{::wxRegisterClipboardFormat}

\func{int}{wxRegisterClipboardFormat}{\param{const wxString\& }{formatName}}

Registers the clipboard data format name and returns an identifier.

\membersection{::wxSetClipboardData}

\func{bool}{wxSetClipboardData}{\param{int}{dataFormat}, \param{wxObject *}{data}, \param{int}{width}, \param{int}{height}}

Passes data to the clipboard.

{\it dataFormat} may be one of:

\begin{itemize}\itemsep=0pt
\item wxCF\_TEXT or wxCF\_OEMTEXT: {\it data} is a null-terminated text string.
\item wxCF\_BITMAP: {\it data} is a wxBitmap.
\item wxCF\_DIB: {\it data} is a wxBitmap. The bitmap is converted to a DIB (device independent bitmap).
\item wxCF\_METAFILE: {\it data} is a wxMetafile. {\it width} and {\it height} are used to give recommended dimensions.
\end{itemize}

The clipboard must have previously been opened for this call to succeed.

\section{Miscellaneous functions}\label{miscellany}

\membersection{::wxNewId}

\func{long}{wxNewId}{\void}

Generates an integer identifier unique to this run of the program.

\wxheading{Include files}

<wx/utils.h>

\membersection{::wxRegisterId}

\func{void}{wxRegisterId}{\param{long}{ id}}

Ensures that ids subsequently generated by {\bf NewId} do not clash with
the given {\bf id}.

\wxheading{Include files}

<wx/utils.h>

\membersection{::wxBeginBusyCursor}\label{wxbeginbusycursor}

\func{void}{wxBeginBusyCursor}{\param{wxCursor *}{cursor = wxHOURGLASS\_CURSOR}}

Changes the cursor to the given cursor for all windows in the application.
Use \helpref{wxEndBusyCursor}{wxendbusycursor} to revert the cursor back
to its previous state. These two calls can be nested, and a counter
ensures that only the outer calls take effect.

See also \helpref{wxIsBusy}{wxisbusy}, \helpref{wxBusyCursor}{wxbusycursor}.

\wxheading{Include files}

<wx/utils.h>

\membersection{::wxBell}

\func{void}{wxBell}{\void}

Ring the system bell.

\wxheading{Include files}

<wx/utils.h>

\membersection{::wxCreateDynamicObject}\label{wxcreatedynamicobject}

\func{wxObject *}{wxCreateDynamicObject}{\param{const wxString\& }{className}}

Creates and returns an object of the given class, if the class has been
registered with the dynamic class system using DECLARE... and IMPLEMENT... macros.

\membersection{::wxDDECleanUp}\label{wxddecleanup}

\func{void}{wxDDECleanUp}{\void}

Called when wxWindows exits, to clean up the DDE system. This no longer needs to be
called by the application.

See also helpref{wxDDEInitialize}{wxddeinitialize}.

\wxheading{Include files}

<wx/dde.h>

\membersection{::wxDDEInitialize}\label{wxddeinitialize}

\func{void}{wxDDEInitialize}{\void}

Initializes the DDE system. May be called multiple times without harm.

This no longer needs to be called by the application: it will be called
by wxWindows if necessary.

See also \helpref{wxDDEServer}{wxddeserver}, \helpref{wxDDEClient}{wxddeclient}, \helpref{wxDDEConnection}{wxddeconnection}, 
\helpref{wxDDECleanUp}{wxddecleanup}.

\wxheading{Include files}

<wx/dde.h>

\membersection{::wxDebugMsg}\label{wxdebugmsg}

\func{void}{wxDebugMsg}{\param{const wxString\& }{fmt}, \param{...}{}}

Display a debugging message; under Windows, this will appear on the
debugger command window, and under Unix, it will be written to standard
error.

The syntax is identical to {\bf printf}: pass a format string and a
variable list of arguments.

Note that under Windows, you can see the debugging messages without a
debugger if you have the DBWIN debug log application that comes with
Microsoft C++.

{\bf Tip:} under Windows, if your application crashes before the
message appears in the debugging window, put a wxYield call after
each wxDebugMsg call. wxDebugMsg seems to be broken under WIN32s
(at least for Watcom C++): preformat your messages and use OutputDebugString
instead.

This function is now obsolete, replaced by \helpref{Log functions}{logfunctions}.

\wxheading{Include files}

<wx/utils.h>

\membersection{::wxDisplaySize}

\func{void}{wxDisplaySize}{\param{int *}{width}, \param{int *}{height}}

Gets the physical size of the display in pixels.

\wxheading{Include files}

<wx/gdicmn.h>

\membersection{::wxEntry}\label{wxentry}

This initializes wxWindows in a platform-dependent way. Use this if you
are not using the default wxWindows entry code (e.g. main or WinMain). For example,
you can initialize wxWindows from an Microsoft Foundation Classes application using
this function.

\func{void}{wxEntry}{\param{HANDLE}{ hInstance}, \param{HANDLE}{ hPrevInstance},
 \param{const wxString\& }{commandLine}, \param{int}{ cmdShow}, \param{bool}{ enterLoop = TRUE}}

wxWindows initialization under Windows (non-DLL). If {\it enterLoop} is FALSE, the
function will return immediately after calling wxApp::OnInit. Otherwise, the wxWindows
message loop will be entered.

\func{void}{wxEntry}{\param{HANDLE}{ hInstance}, \param{HANDLE}{ hPrevInstance},
 \param{WORD}{ wDataSegment}, \param{WORD}{ wHeapSize}, \param{const wxString\& }{ commandLine}}

wxWindows initialization under Windows (for applications constructed as a DLL).

\func{int}{wxEntry}{\param{int}{ argc}, \param{const wxString\& *}{argv}}

wxWindows initialization under Unix.

\wxheading{Remarks}

To clean up wxWindows, call wxApp::OnExit followed by the static function
wxApp::CleanUp. For example, if exiting from an MFC application that also uses wxWindows:

\begin{verbatim}
int CTheApp::ExitInstance()
{
  // OnExit isn't called by CleanUp so must be called explicitly.
  wxTheApp->OnExit();
  wxApp::CleanUp();

  return CWinApp::ExitInstance();
}
\end{verbatim}

\wxheading{Include files}

<wx/app.h>

\membersection{::wxError}\label{wxerror}

\func{void}{wxError}{\param{const wxString\& }{msg}, \param{const wxString\& }{title = "wxWindows Internal Error"}}

Displays {\it msg} and continues. This writes to standard error under
Unix, and pops up a message box under Windows. Used for internal
wxWindows errors. See also \helpref{wxFatalError}{wxfatalerror}.

\wxheading{Include files}

<wx/utils.h>

\membersection{::wxEndBusyCursor}\label{wxendbusycursor}

\func{void}{wxEndBusyCursor}{\void}

Changes the cursor back to the original cursor, for all windows in the application.
Use with \helpref{wxBeginBusyCursor}{wxbeginbusycursor}.

See also \helpref{wxIsBusy}{wxisbusy}, \helpref{wxBusyCursor}{wxbusycursor}.

\wxheading{Include files}

<wx/utils.h>

\membersection{::wxExecute}\label{wxexecute}

\func{long}{wxExecute}{\param{const wxString\& }{command}, \param{bool }{sync = FALSE}, \param{wxProcess *}{callback = NULL}}

\func{long}{wxExecute}{\param{char **}{argv}, \param{bool }{sync = FALSE}, \param{wxProcess *}{callback = NULL}}

Executes another program in Unix or Windows.

The first form takes a command string, such as {\tt "emacs file.txt"}.

The second form takes an array of values: a command, any number of
arguments, terminated by NULL.

If {\it sync} is FALSE (the default), flow of control immediately returns.
If TRUE, the current application waits until the other program has terminated.

In the case of synchronous execution, the return value is the exit code of
the process (which terminates by the moment the function returns) and will be
$-1$ if the process couldn't be started and typically 0 if the process
terminated successfully. Also, while waiting for the process to
terminate, wxExecute will call \helpref{wxYield}{wxyield}. The caller
should ensure that this can cause no recursion, in the simples case by 
calling \helpref{wxEnableTopLevelWindows(FALSE)}{wxenabletoplevelwindows}.

For asynchronous execution, however, the return value is the process id and
zero value indicates that the command could not be executed.

If callback isn't NULL and if execution is asynchronous (note that callback
parameter can not be non NULL for synchronous execution), 
\helpref{wxProcess::OnTerminate}{wxprocessonterminate} will be called when
the process finishes.

See also \helpref{wxShell}{wxshell}, \helpref{wxProcess}{wxprocess}.

\wxheading{Include files}

<wx/utils.h>

\membersection{::wxExit}\label{wxexit}

\func{void}{wxExit}{\void}

Exits application after calling \helpref{wxApp::OnExit}{wxapponexit}.
Should only be used in an emergency: normally the top-level frame
should be deleted (after deleting all other frames) to terminate the
application. See \helpref{wxWindow::OnCloseWindow}{wxwindowonclosewindow} and \helpref{wxApp}{wxapp}.

\wxheading{Include files}

<wx/app.h>

\membersection{::wxFatalError}\label{wxfatalerror}

\func{void}{wxFatalError}{\param{const wxString\& }{msg}, \param{const wxString\& }{title = "wxWindows Fatal Error"}}

Displays {\it msg} and exits. This writes to standard error under Unix,
and pops up a message box under Windows. Used for fatal internal
wxWindows errors. See also \helpref{wxError}{wxerror}.

\wxheading{Include files}

<wx/utils.h>

\membersection{::wxFindMenuItemId}

\func{int}{wxFindMenuItemId}{\param{wxFrame *}{frame}, \param{const wxString\& }{menuString}, \param{const wxString\& }{itemString}}

Find a menu item identifier associated with the given frame's menu bar.

\wxheading{Include files}

<wx/utils.h>

\membersection{::wxFindWindowByLabel}

\func{wxWindow *}{wxFindWindowByLabel}{\param{const wxString\& }{label}, \param{wxWindow *}{parent=NULL}}

Find a window by its label. Depending on the type of window, the label may be a window title
or panel item label. If {\it parent} is NULL, the search will start from all top-level
frames and dialog boxes; if non-NULL, the search will be limited to the given window hierarchy.
The search is recursive in both cases.

\wxheading{Include files}

<wx/utils.h>

\membersection{::wxFindWindowByName}\label{wxfindwindowbyname}

\func{wxWindow *}{wxFindWindowByName}{\param{const wxString\& }{name}, \param{wxWindow *}{parent=NULL}}

Find a window by its name (as given in a window constructor or {\bf Create} function call).
If {\it parent} is NULL, the search will start from all top-level
frames and dialog boxes; if non-NULL, the search will be limited to the given window hierarchy.
The search is recursive in both cases.

If no such named window is found, {\bf wxFindWindowByLabel} is called.

\wxheading{Include files}

<wx/utils.h>

\membersection{::wxGetActiveWindow}\label{wxgetactivewindow}

\func{wxWindow *}{wxGetActiveWindow}{\void}

Gets the currently active window (Windows only).

\wxheading{Include files}

<wx/windows.h>

\membersection{::wxGetDisplayName}\label{wxgetdisplayname}

\func{wxString}{wxGetDisplayName}{\void}

Under X only, returns the current display name. See also \helpref{wxSetDisplayName}{wxsetdisplayname}.

\wxheading{Include files}

<wx/utils.h>

\membersection{::wxGetHomeDir}

\func{wxString}{wxGetHomeDir}{\param{const wxString\& }{buf}}

Fills the buffer with a string representing the user's home directory (Unix only).

\wxheading{Include files}

<wx/utils.h>

\membersection{::wxGetHostName}

\func{bool}{wxGetHostName}{\param{const wxString\& }{buf}, \param{int}{ bufSize}}

Copies the host name of the machine the program is running on into the
buffer {\it buf}, of maximum size {\it bufSize}, returning TRUE if
successful. Under Unix, this will return a machine name. Under Windows,
this returns ``windows''.

\wxheading{Include files}

<wx/utils.h>

\membersection{::wxGetElapsedTime}\label{wxgetelapsedtime}

\func{long}{wxGetElapsedTime}{\param{bool}{ resetTimer = TRUE}}

Gets the time in milliseconds since the last \helpref{::wxStartTimer}{wxstarttimer}.

If {\it resetTimer} is TRUE (the default), the timer is reset to zero
by this call.

See also \helpref{wxTimer}{wxtimer}.

\wxheading{Include files}

<wx/timer.h>

\membersection{::wxGetFreeMemory}\label{wxgetfreememory}

\func{long}{wxGetFreeMemory}{\void}

Returns the amount of free memory in Kbytes under environments which
support it, and -1 if not supported. Currently, returns a positive value
under Windows, and -1 under Unix.

\wxheading{Include files}

<wx/utils.h>

\membersection{::wxGetMousePosition}

\func{void}{wxGetMousePosition}{\param{int* }{x}, \param{int* }{y}}

Returns the mouse position in screen coordinates.

\wxheading{Include files}

<wx/utils.h>

\membersection{::wxGetOsVersion}

\func{int}{wxGetOsVersion}{\param{int *}{major = NULL}, \param{int *}{minor = NULL}}

Gets operating system version information.

\begin{twocollist}\itemsep=0pt
\twocolitemruled{Platform}{Return tyes}
\twocolitem{Macintosh}{Return value is wxMACINTOSH.}
\twocolitem{GTK}{Return value is wxGTK, {\it major} is 1, {\it minor} is 0. (for GTK 1.0.X) }
\twocolitem{Motif}{Return value is wxMOTIF\_X, {\it major} is X version, {\it minor} is X revision.}
\twocolitem{OS/2}{Return value is wxOS2\_PM.}
\twocolitem{Windows 3.1}{Return value is wxWINDOWS, {\it major} is 3, {\it minor} is 1.}
\twocolitem{Windows NT}{Return value is wxWINDOWS\_NT, {\it major} is 3, {\it minor} is 1.}
\twocolitem{Windows 95}{Return value is wxWIN95, {\it major} is 3, {\it minor} is 1.}
\twocolitem{Win32s (Windows 3.1)}{Return value is wxWIN32S, {\it major} is 3, {\it minor} is 1.}
\twocolitem{Watcom C++ 386 supervisor mode (Windows 3.1)}{Return value is wxWIN386, {\it major} is 3, {\it minor} is 1.}
\end{twocollist}

\wxheading{Include files}

<wx/utils.h>

\membersection{::wxGetResource}\label{wxgetresource}

\func{bool}{wxGetResource}{\param{const wxString\& }{section}, \param{const wxString\& }{entry},
 \param{const wxString\& *}{value}, \param{const wxString\& }{file = NULL}}

\func{bool}{wxGetResource}{\param{const wxString\& }{section}, \param{const wxString\& }{entry},
 \param{float *}{value}, \param{const wxString\& }{file = NULL}}

\func{bool}{wxGetResource}{\param{const wxString\& }{section}, \param{const wxString\& }{entry},
 \param{long *}{value}, \param{const wxString\& }{file = NULL}}

\func{bool}{wxGetResource}{\param{const wxString\& }{section}, \param{const wxString\& }{entry},
 \param{int *}{value}, \param{const wxString\& }{file = NULL}}

Gets a resource value from the resource database (for example, WIN.INI, or
.Xdefaults). If {\it file} is NULL, WIN.INI or .Xdefaults is used,
otherwise the specified file is used.

Under X, if an application class (wxApp::GetClassName) has been defined,
it is appended to the string /usr/lib/X11/app-defaults/ to try to find
an applications default file when merging all resource databases.

The reason for passing the result in an argument is that it
can be convenient to define a default value, which gets overridden
if the value exists in the resource file. It saves a separate
test for that resource's existence, and it also allows
the overloading of the function for different types.

See also \helpref{wxWriteResource}{wxwriteresource}, \helpref{wxConfigBase}{wxconfigbase}.

\wxheading{Include files}

<wx/utils.h>

\membersection{::wxGetUserId}

\func{bool}{wxGetUserId}{\param{const wxString\& }{buf}, \param{int}{ bufSize}}

Copies the user's login identity (such as ``jacs'') into the buffer {\it
buf}, of maximum size {\it bufSize}, returning TRUE if successful.
Under Windows, this returns ``user''.

\wxheading{Include files}

<wx/utils.h>

\membersection{::wxGetUserName}

\func{bool}{wxGetUserName}{\param{const wxString\& }{buf}, \param{int}{ bufSize}}

Copies the user's name (such as ``Julian Smart'') into the buffer {\it
buf}, of maximum size {\it bufSize}, returning TRUE if successful.
Under Windows, this returns ``unknown''.

\wxheading{Include files}

<wx/utils.h>

\membersection{::wxKill}\label{wxkill}

\func{int}{wxKill}{\param{long}{ pid}, \param{int}{ sig}}

Under Unix (the only supported platform), equivalent to the Unix kill function.
Returns 0 on success, -1 on failure.

Tip: sending a signal of 0 to a process returns -1 if the process does not exist.
It does not raise a signal in the receiving process.

\wxheading{Include files}

<wx/utils.h>

\membersection{::wxIsBusy}\label{wxisbusy}

\func{bool}{wxIsBusy}{\void}

Returns TRUE if between two \helpref{wxBeginBusyCursor}{wxbeginbusycursor} and\rtfsp
\helpref{wxEndBusyCursor}{wxendbusycursor} calls.

See also \helpref{wxBusyCursor}{wxbusycursor}.

\wxheading{Include files}

<wx/utils.h>

\membersection{::wxLoadUserResource}\label{wxloaduserresource}

\func{wxString}{wxLoadUserResource}{\param{const wxString\& }{resourceName}, \param{const wxString\& }{resourceType=``TEXT"}}

Loads a user-defined Windows resource as a string. If the resource is found, the function creates
a new character array and copies the data into it. A pointer to this data is returned. If unsuccessful, NULL is returned.

The resource must be defined in the {\tt .rc} file using the following syntax:

\begin{verbatim}
myResource TEXT file.ext
\end{verbatim}

where {\tt file.ext} is a file that the resource compiler can find.

One use of this is to store {\tt .wxr} files instead of including the data in the C++ file; some compilers
cannot cope with the long strings in a {\tt .wxr} file. The resource data can then be parsed
using \helpref{wxResourceParseString}{wxresourceparsestring}.

This function is available under Windows only.

\wxheading{Include files}

<wx/utils.h>

\membersection{::wxNow}\label{wxnow}

\func{wxString}{wxNow}{\void}

Returns a string representing the current date and time.

\wxheading{Include files}

<wx/utils.h>

\membersection{::wxPostDelete}\label{wxpostdelete}

\func{void}{wxPostDelete}{\param{wxObject *}{object}}

Tells the system to delete the specified object when
all other events have been processed. In some environments, it is
necessary to use this instead of deleting a frame directly with the
delete operator, because some GUIs will still send events to a deleted window.

Now obsolete: use \helpref{wxWindow::Close}{wxwindowclose} instead.

\wxheading{Include files}

<wx/utils.h>

\membersection{::wxSafeYield}\label{wxsafeyield}

\func{bool}{wxSafeYield}{\param{wxWindow*}{ win = NULL}}

This function is similar to wxYield, except that it disables the user input to
all program windows before calling wxYield and re-enables it again
afterwards. If {\it win} is not NULL, this window will remain enabled, 
allowing the implementation of some limited user interaction.

Returns the result of the call to \helpref{::wxYield}{wxyield}.

\wxheading{Include files}

<wx/utils.h>

\membersection{::wxEnableTopLevelWindows}\label{wxenabletoplevelwindows}

\func{void}{wxEnableTopLevelWindow}{\param{bool}{ enable = TRUE}}

This function enables or disables all top level windows. It is used by
\helpref{::wxSafeYield}{wxsafeyield}.

\wxheading{Include files}

<wx/utils.h>


\membersection{::wxSetDisplayName}\label{wxsetdisplayname}

\func{void}{wxSetDisplayName}{\param{const wxString\& }{displayName}}

Under X only, sets the current display name. This is the X host and display name such
as ``colonsay:0.0", and the function indicates which display should be used for creating
windows from this point on. Setting the display within an application allows multiple
displays to be used.

See also \helpref{wxGetDisplayName}{wxgetdisplayname}.

\wxheading{Include files}

<wx/utils.h>

\membersection{::wxShell}\label{wxshell}

\func{bool}{wxShell}{\param{const wxString\& }{command = NULL}}

Executes a command in an interactive shell window. If no command is
specified, then just the shell is spawned.

See also \helpref{wxExecute}{wxexecute}.

\wxheading{Include files}

<wx/utils.h>

\membersection{::wxSleep}\label{wxsleep}

\func{void}{wxSleep}{\param{int}{ secs}}

Sleeps for the specified number of seconds.

\wxheading{Include files}

<wx/utils.h>

\membersection{::wxStripMenuCodes}

\func{wxString}{wxStripMenuCodes}{\param{const wxString\& }{in}}

\func{void}{wxStripMenuCodes}{\param{char* }{in}, \param{char* }{out}}

Strips any menu codes from {\it in} and places the result
in {\it out} (or returns the new string, in the first form).

Menu codes include \& (mark the next character with an underline
as a keyboard shortkey in Windows and Motif) and $\backslash$t (tab in Windows).

\wxheading{Include files}

<wx/utils.h>

\membersection{::wxStartTimer}\label{wxstarttimer}

\func{void}{wxStartTimer}{\void}

Starts a stopwatch; use \helpref{::wxGetElapsedTime}{wxgetelapsedtime} to get the elapsed time.

See also \helpref{wxTimer}{wxtimer}.

\wxheading{Include files}

<wx/timer.h>

\membersection{::wxToLower}\label{wxtolower}

\func{char}{wxToLower}{\param{char }{ch}}

Converts the character to lower case. This is implemented as a macro for efficiency.

\wxheading{Include files}

<wx/utils.h>

\membersection{::wxToUpper}\label{wxtoupper}

\func{char}{wxToUpper}{\param{char }{ch}}

Converts the character to upper case. This is implemented as a macro for efficiency.

\wxheading{Include files}

<wx/utils.h>

\membersection{::wxTrace}\label{wxtrace}

\func{void}{wxTrace}{\param{const wxString\& }{fmt}, \param{...}{}}

Takes printf-style variable argument syntax. Output
is directed to the current output stream (see \helpref{wxDebugContext}{wxdebugcontextoverview}).

This function is now obsolete, replaced by \helpref{Log functions}{logfunctions}.

\wxheading{Include files}

<wx/memory.h>

\membersection{::wxTraceLevel}\label{wxtracelevel}

\func{void}{wxTraceLevel}{\param{int}{ level}, \param{const wxString\& }{fmt}, \param{...}{}}

Takes printf-style variable argument syntax. Output
is directed to the current output stream (see \helpref{wxDebugContext}{wxdebugcontextoverview}).
The first argument should be the level at which this information is appropriate.
It will only be output if the level returned by wxDebugContext::GetLevel is equal to or greater than
this value.

This function is now obsolete, replaced by \helpref{Log functions}{logfunctions}.

\wxheading{Include files}

<wx/memory.h>

\membersection{::wxUsleep}\label{wxusleep}

\func{void}{wxUsleep}{\param{unsigned long}{ milliseconds}}

Sleeps for the specified number of milliseconds. Notice that usage of this
function is encouraged instead of calling usleep(3) directly because the
standard usleep() function is not MT safe.

\wxheading{Include files}

<wx/utils.h>

\membersection{::wxWriteResource}\label{wxwriteresource}

\func{bool}{wxWriteResource}{\param{const wxString\& }{section}, \param{const wxString\& }{entry},
 \param{const wxString\& }{value}, \param{const wxString\& }{file = NULL}}

\func{bool}{wxWriteResource}{\param{const wxString\& }{section}, \param{const wxString\& }{entry},
 \param{float }{value}, \param{const wxString\& }{file = NULL}}

\func{bool}{wxWriteResource}{\param{const wxString\& }{section}, \param{const wxString\& }{entry},
 \param{long }{value}, \param{const wxString\& }{file = NULL}}

\func{bool}{wxWriteResource}{\param{const wxString\& }{section}, \param{const wxString\& }{entry},
 \param{int }{value}, \param{const wxString\& }{file = NULL}}

Writes a resource value into the resource database (for example, WIN.INI, or
.Xdefaults). If {\it file} is NULL, WIN.INI or .Xdefaults is used,
otherwise the specified file is used.

Under X, the resource databases are cached until the internal function
\rtfsp{\bf wxFlushResources} is called automatically on exit, when
all updated resource databases are written to their files.

Note that it is considered bad manners to write to the .Xdefaults
file under Unix, although the WIN.INI file is fair game under Windows.

See also \helpref{wxGetResource}{wxgetresource}, \helpref{wxConfigBase}{wxconfigbase}.

\wxheading{Include files}

<wx/utils.h>

\membersection{::wxYield}\label{wxyield}

\func{bool}{wxYield}{\void}

Yields control to pending messages in the windowing system. This can be useful, for example, when a
time-consuming process writes to a text window. Without an occasional
yield, the text window will not be updated properly, and (since Windows
multitasking is cooperative) other processes will not respond.

Caution should be exercised, however, since yielding may allow the
user to perform actions which are not compatible with the current task.
Disabling menu items or whole menus during processing can avoid unwanted
reentrance of code: see \helpref{::wxSafeYield}{wxsafeyield} for a better
function.

\wxheading{Include files}

<wx/utils.h>

\section{Macros}\label{macros}

These macros are defined in wxWindows.

\membersection{wxINTXX\_SWAP\_ALWAYS}\label{intswapalways}

\func{wxInt32}{wxINT32\_SWAP\_ALWAYS}{\param{wxInt32 }{value}}

\func{wxUint32}{wxUINT32\_SWAP\_ALWAYS}{\param{wxUint32 }{value}}

\func{wxInt16}{wxINT16\_SWAP\_ALWAYS}{\param{wxInt16 }{value}}

\func{wxUint16}{wxUINT16\_SWAP\_ALWAYS}{\param{wxUint16 }{value}}

This macro will swap the bytes of the {\it value} variable from little
endian to big endian or vice versa.

\membersection{wxINTXX\_SWAP\_ON\_BE}\label{intswaponbe}

\func{wxInt32}{wxINT32\_SWAP\_ON\_BE}{\param{wxInt32 }{value}}

\func{wxUint32}{wxUINT32\_SWAP\_ON\_BE}{\param{wxUint32 }{value}}

\func{wxInt16}{wxINT16\_SWAP\_ON\_BE}{\param{wxInt16 }{value}}

\func{wxUint16}{wxUINT16\_SWAP\_ON\_BE}{\param{wxUint16 }{value}}

This macro will swap the bytes of the {\it value} variable from little
endian to big endian or vice versa if the program is compiled on a
big-endian architecture (such as Sun work stations). If the program has 
been compiled on a little-endian architecture, the value will be unchanged.

Use these macros to read data from and write data to a file that stores 
data in little endian (Intel i386) format.

\membersection{wxINTXX\_SWAP\_ON\_LE}\label{intswaponle}

\func{wxInt32}{wxINT32\_SWAP\_ON\_LE}{\param{wxInt32 }{value}}

\func{wxUint32}{wxUINT32\_SWAP\_ON\_LE}{\param{wxUint32 }{value}}

\func{wxInt16}{wxINT16\_SWAP\_ON\_LE}{\param{wxInt16 }{value}}

\func{wxUint16}{wxUINT16\_SWAP\_ON\_LE}{\param{wxUint16 }{value}}

This macro will swap the bytes of the {\it value} variable from little
endian to big endian or vice versa if the program is compiled on a
little-endian architecture (such as Intel PCs). If the program has 
been compiled on a big-endian architecture, the value will be unchanged.

Use these macros to read data from and write data to a file that stores 
data in big endian format.

\membersection{CLASSINFO}\label{classinfo}

\func{wxClassInfo *}{CLASSINFO}{className}

Returns a pointer to the wxClassInfo object associated with this class.

\wxheading{Include files}

<wx/object.h>

\membersection{DECLARE\_ABSTRACT\_CLASS}

\func{}{DECLARE\_ABSTRACT\_CLASS}{className}

Used inside a class declaration to declare that the class should be
made known to the class hierarchy, but objects of this class cannot be created
dynamically. The same as DECLARE\_CLASS.

Example:

\begin{verbatim}
class wxCommand: public wxObject
{
  DECLARE_ABSTRACT_CLASS(wxCommand)

 private:
  ...
 public:
  ...
};
\end{verbatim}

\wxheading{Include files}

<wx/object.h>

\membersection{DECLARE\_APP}\label{declareapp}

\func{}{DECLARE\_APP}{className}

This is used in headers to create a forward declaration of the wxGetApp function implemented
by IMPLEMENT\_APP. It creates the declaration {\tt className\& wxGetApp(void)}.

Example:

\begin{verbatim}
  DECLARE_APP(MyApp)
\end{verbatim}

\wxheading{Include files}

<wx/app.h>

\membersection{DECLARE\_CLASS}

\func{}{DECLARE\_CLASS}{className}

Used inside a class declaration to declare that the class should be
made known to the class hierarchy, but objects of this class cannot be created
dynamically. The same as DECLARE\_ABSTRACT\_CLASS.

\wxheading{Include files}

<wx/object.h>

\membersection{DECLARE\_DYNAMIC\_CLASS}

\func{}{DECLARE\_DYNAMIC\_CLASS}{className}

Used inside a class declaration to declare that the objects of this class should be dynamically
createable from run-time type information.

Example:

\begin{verbatim}
class wxFrame: public wxWindow
{
  DECLARE_DYNAMIC_CLASS(wxFrame)

 private:
  const wxString\& frameTitle;
 public:
  ...
};
\end{verbatim}

\wxheading{Include files}

<wx/object.h>

\membersection{IMPLEMENT\_ABSTRACT\_CLASS}

\func{}{IMPLEMENT\_ABSTRACT\_CLASS}{className, baseClassName}

Used in a C++ implementation file to complete the declaration of
a class that has run-time type information. The same as IMPLEMENT\_CLASS.

Example:

\begin{verbatim}
IMPLEMENT_ABSTRACT_CLASS(wxCommand, wxObject)

wxCommand::wxCommand(void)
{
...
}
\end{verbatim}

\wxheading{Include files}

<wx/object.h>

\membersection{IMPLEMENT\_ABSTRACT\_CLASS2}

\func{}{IMPLEMENT\_ABSTRACT\_CLASS2}{className, baseClassName1, baseClassName2}

Used in a C++ implementation file to complete the declaration of
a class that has run-time type information and two base classes. The same as IMPLEMENT\_CLASS2.

\wxheading{Include files}

<wx/object.h>

\membersection{IMPLEMENT\_APP}\label{implementapp}

\func{}{IMPLEMENT\_APP}{className}

This is used in the application class implementation file to make the application class known to
wxWindows for dynamic construction. You use this instead of

Old form:

\begin{verbatim}
  MyApp myApp;
\end{verbatim}

New form:

\begin{verbatim}
  IMPLEMENT_APP(MyApp)
\end{verbatim}

See also \helpref{DECLARE\_APP}{declareapp}.

\wxheading{Include files}

<wx/app.h>

\membersection{IMPLEMENT\_CLASS}

\func{}{IMPLEMENT\_CLASS}{className, baseClassName}

Used in a C++ implementation file to complete the declaration of
a class that has run-time type information. The same as IMPLEMENT\_ABSTRACT\_CLASS.

\wxheading{Include files}

<wx/object.h>

\membersection{IMPLEMENT\_CLASS2}

\func{}{IMPLEMENT\_CLASS2}{className, baseClassName1, baseClassName2}

Used in a C++ implementation file to complete the declaration of a
class that has run-time type information and two base classes. The
same as IMPLEMENT\_ABSTRACT\_CLASS2.

\wxheading{Include files}

<wx/object.h>

\membersection{IMPLEMENT\_DYNAMIC\_CLASS}

\func{}{IMPLEMENT\_DYNAMIC\_CLASS}{className, baseClassName}

Used in a C++ implementation file to complete the declaration of
a class that has run-time type information, and whose instances
can be created dynamically.

Example:

\begin{verbatim}
IMPLEMENT_DYNAMIC_CLASS(wxFrame, wxWindow)

wxFrame::wxFrame(void)
{
...
}
\end{verbatim}

\wxheading{Include files}

<wx/object.h>

\membersection{IMPLEMENT\_DYNAMIC\_CLASS2}

\func{}{IMPLEMENT\_DYNAMIC\_CLASS2}{className, baseClassName1, baseClassName2}

Used in a C++ implementation file to complete the declaration of
a class that has run-time type information, and whose instances
can be created dynamically. Use this for classes derived from two
base classes.

\wxheading{Include files}

<wx/object.h>

\membersection{WXDEBUG\_NEW}\label{debugnew}

\func{}{WXDEBUG\_NEW}{arg}

This is defined in debug mode to be call the redefined new operator
with filename and line number arguments. The definition is:

\begin{verbatim}
#define WXDEBUG_NEW new(__FILE__,__LINE__)
\end{verbatim}

In non-debug mode, this is defined as the normal new operator.

\wxheading{Include files}

<wx/object.h>

\membersection{wxDynamicCast}\label{wxdynamiccast}

\func{}{wxDynamicCast}{ptr, classname}

This macro returns the pointer {\it ptr} cast to the type {\it classname *} if
the pointer is of this type (the check is done during the run-time) or NULL
otherwise. Usage of this macro is prefered over obsoleted wxObject::IsKindOf()
function.

The {\it ptr} argument may be NULL, in which case NULL will be returned.

Example:

\begin{verbatim}
    wxWindow *win = wxWindow::FindFocus();
    wxTextCtrl *text = wxDynamicCast(win, wxTextCtrl);
    if ( text )
    {
        // a text control has the focus...
    }
    else
    {
        // no window has the focus or it's not a text control
    }
\end{verbatim}

\wxheading{See also}

\helpref{RTTI overview}{runtimeclassoverview}

\membersection{WXTRACE}\label{trace}

\wxheading{Include files}

<wx/object.h>

\func{}{WXTRACE}{formatString, ...}

Calls wxTrace with printf-style variable argument syntax. Output
is directed to the current output stream (see \helpref{wxDebugContext}{wxdebugcontextoverview}).

This macro is now obsolete, replaced by \helpref{Log functions}{logfunctions}.

\wxheading{Include files}

<wx/memory.h>

\membersection{WXTRACELEVEL}\label{tracelevel}

\func{}{WXTRACELEVEL}{level, formatString, ...}

Calls wxTraceLevel with printf-style variable argument syntax. Output
is directed to the current output stream (see \helpref{wxDebugContext}{wxdebugcontextoverview}).
The first argument should be the level at which this information is appropriate.
It will only be output if the level returned by wxDebugContext::GetLevel is equal to or greater than
this value.

This function is now obsolete, replaced by \helpref{Log functions}{logfunctions}.

\wxheading{Include files}

<wx/memory.h>

\section{wxWindows resource functions}\label{resourcefuncs}

\overview{wxWindows resource system}{resourceformats}

This section details functions for manipulating wxWindows (.WXR) resource
files and loading user interface elements from resources.

\normalbox{Please note that this use of the word `resource' is different from that used when talking
about initialisation file resource reading and writing, using such functions
as wxWriteResource and wxGetResource. It's just an unfortunate clash of terminology.}

\helponly{For an overview of the wxWindows resource mechanism, see \helpref{the wxWindows resource system}{resourceformats}.}

See also \helpref{wxWindow::LoadFromResource}{wxwindowloadfromresource} for
loading from resource data.

{\bf Warning:} this needs updating for wxWindows 2.

\membersection{::wxResourceAddIdentifier}\label{wxresourceaddidentifier}

\func{bool}{wxResourceAddIdentifier}{\param{const wxString\& }{name}, \param{int }{value}}

Used for associating a name with an integer identifier (equivalent to dynamically\rtfsp
\verb$#$defining a name to an integer). Unlikely to be used by an application except
perhaps for implementing resource functionality for interpreted languages.

\membersection{::wxResourceClear}

\func{void}{wxResourceClear}{\void}

Clears the wxWindows resource table.

\membersection{::wxResourceCreateBitmap}

\func{wxBitmap *}{wxResourceCreateBitmap}{\param{const wxString\& }{resource}}

Creates a new bitmap from a file, static data, or Windows resource, given a valid
wxWindows bitmap resource identifier. For example, if the .WXR file contains
the following:

\begin{verbatim}
static const wxString\& aiai_resource = "bitmap(name = 'aiai_resource',\
  bitmap = ['aiai', wxBITMAP_TYPE_BMP_RESOURCE, 'WINDOWS'],\
  bitmap = ['aiai.xpm', wxBITMAP_TYPE_XPM, 'X']).";
\end{verbatim}

then this function can be called as follows:

\begin{verbatim}
  wxBitmap *bitmap  = wxResourceCreateBitmap("aiai_resource");
\end{verbatim}

\membersection{::wxResourceCreateIcon}

\func{wxIcon *}{wxResourceCreateIcon}{\param{const wxString\& }{resource}}

Creates a new icon from a file, static data, or Windows resource, given a valid
wxWindows icon resource identifier. For example, if the .WXR file contains
the following:

\begin{verbatim}
static const wxString\& aiai_resource = "icon(name = 'aiai_resource',\
  icon = ['aiai', wxBITMAP_TYPE_ICO_RESOURCE, 'WINDOWS'],\
  icon = ['aiai', wxBITMAP_TYPE_XBM_DATA, 'X']).";
\end{verbatim}

then this function can be called as follows:

\begin{verbatim}
  wxIcon *icon = wxResourceCreateIcon("aiai_resource");
\end{verbatim}

\membersection{::wxResourceCreateMenuBar}

\func{wxMenuBar *}{wxResourceCreateMenuBar}{\param{const wxString\& }{resource}}

Creates a new menu bar given a valid wxWindows menubar resource
identifier. For example, if the .WXR file contains the following:

\begin{verbatim}
static const wxString\& menuBar11 = "menu(name = 'menuBar11',\
  menu = \
  [\
    ['&File', 1, '', \
      ['&Open File', 2, 'Open a file'],\
      ['&Save File', 3, 'Save a file'],\
      [],\
      ['E&xit', 4, 'Exit program']\
    ],\
    ['&Help', 5, '', \
      ['&About', 6, 'About this program']\
    ]\
  ]).";
\end{verbatim}

then this function can be called as follows:

\begin{verbatim}
  wxMenuBar *menuBar = wxResourceCreateMenuBar("menuBar11");
\end{verbatim}


\membersection{::wxResourceGetIdentifier}

\func{int}{wxResourceGetIdentifier}{\param{const wxString\& }{name}}

Used for retrieving the integer value associated with an identifier.
A zero value indicates that the identifier was not found.

See \helpref{wxResourceAddIdentifier}{wxresourceaddidentifier}.

\membersection{::wxResourceParseData}\label{wxresourcedata}

\func{bool}{wxResourceParseData}{\param{const wxString\& }{resource}, \param{wxResourceTable *}{table = NULL}}

Parses a string containing one or more wxWindows resource objects. If
the resource objects are global static data that are included into the
C++ program, then this function must be called for each variable
containing the resource data, to make it known to wxWindows.

{\it resource} should contain data in the following form:

\begin{verbatim}
dialog(name = 'dialog1',
  style = 'wxCAPTION | wxDEFAULT_DIALOG_STYLE',
  title = 'Test dialog box',
  x = 312, y = 234, width = 400, height = 300,
  modal = 0,
  control = [wxGroupBox, 'Groupbox', '0', 'group6', 5, 4, 380, 262,
      [11, 'wxSWISS', 'wxNORMAL', 'wxNORMAL', 0]],
  control = [wxMultiText, 'Multitext', 'wxVERTICAL_LABEL', 'multitext3',
      156, 126, 200, 70, 'wxWindows is a multi-platform, GUI toolkit.',
      [11, 'wxSWISS', 'wxNORMAL', 'wxNORMAL', 0],
      [11, 'wxSWISS', 'wxNORMAL', 'wxNORMAL', 0]]).
\end{verbatim}

This function will typically be used after including a {\tt .wxr} file into
a C++ program as follows:

\begin{verbatim}
#include "dialog1.wxr"
\end{verbatim}

Each of the contained resources will declare a new C++ variable, and each
of these variables should be passed to wxResourceParseData.

\membersection{::wxResourceParseFile}

\func{bool}{wxResourceParseFile}{\param{const wxString\& }{filename}, \param{wxResourceTable *}{table = NULL}}

Parses a file containing one or more wxWindows resource objects
in C++-compatible syntax. Use this function to dynamically load
wxWindows resource data.

\membersection{::wxResourceParseString}\label{wxresourceparsestring}

\func{bool}{wxResourceParseString}{\param{const wxString\& }{resource}, \param{wxResourceTable *}{table = NULL}}

Parses a string containing one or more wxWindows resource objects. If
the resource objects are global static data that are included into the
C++ program, then this function must be called for each variable
containing the resource data, to make it known to wxWindows.

{\it resource} should contain data with the following form:

\begin{verbatim}
static const wxString\& dialog1 = "dialog(name = 'dialog1',\
  style = 'wxCAPTION | wxDEFAULT_DIALOG_STYLE',\
  title = 'Test dialog box',\
  x = 312, y = 234, width = 400, height = 300,\
  modal = 0,\
  control = [wxGroupBox, 'Groupbox', '0', 'group6', 5, 4, 380, 262,\
      [11, 'wxSWISS', 'wxNORMAL', 'wxNORMAL', 0]],\
  control = [wxMultiText, 'Multitext', 'wxVERTICAL_LABEL', 'multitext3',\
      156, 126, 200, 70, 'wxWindows is a multi-platform, GUI toolkit.',\
      [11, 'wxSWISS', 'wxNORMAL', 'wxNORMAL', 0],\
      [11, 'wxSWISS', 'wxNORMAL', 'wxNORMAL', 0]]).";
\end{verbatim}

This function will typically be used after calling \helpref{wxLoadUserResource}{wxloaduserresource} to
load an entire {\tt .wxr file} into a string.

\membersection{::wxResourceRegisterBitmapData}\label{registerbitmapdata}

\func{bool}{wxResourceRegisterBitmapData}{\param{const wxString\& }{name}, \param{const wxString\& }{xbm\_data}, \param{int }{width},
\param{int }{height}, \param{wxResourceTable *}{table = NULL}}

\func{bool}{wxResourceRegisterBitmapData}{\param{const wxString\& }{name}, \param{const wxString\& *}{xpm\_data}}

Makes \verb$#$included XBM or XPM bitmap data known to the wxWindows resource system. 
This is required if other resources will use the bitmap data, since otherwise there
is no connection between names used in resources, and the global bitmap data.

\membersection{::wxResourceRegisterIconData}

Another name for \helpref{wxResourceRegisterBitmapData}{registerbitmapdata}.

\section{Log functions}\label{logfunctions}

These functions provide a variety of logging functions: see \helpref{Log classes overview}{wxlogoverview} for
further information.

\wxheading{Include files}

<wx/log.h>

\membersection{::wxLogError}\label{wxlogerror}

\func{void}{wxLogError}{\param{const char*}{ formatString}, \param{...}{}}

The function to use for error messages, i.e. the
messages that must be shown to the user. The default processing is to pop up a
message box to inform the user about it.

\membersection{::wxLogFatalError}\label{wxlogfatalerror}

\func{void}{wxLogFatalError}{\param{const char*}{ formatString}, \param{...}{}}

Like \helpref{wxLogError}{wxlogerror}, but also
terminates the program with the exit code 3. Using {\it abort()} standard
function also terminates the program with this exit code.

\membersection{::wxLogWarning}\label{wxlogwarning}

\func{void}{wxLogWarning}{\param{const char*}{ formatString}, \param{...}{}}

For warnings - they are also normally shown to the
user, but don't interrupt the program work.

\membersection{::wxLogMessage}\label{wxlogmessage}

\func{void}{wxLogMessage}{\param{const char*}{ formatString}, \param{...}{}}

for all normal, informational messages. They also
appear in a message box by default (but it can be changed). Notice
that the standard behaviour is to not show informational messages if there are
any errors later - the logic being that the later error messages make the
informational messages preceding them meaningless.

\membersection{::wxLogVerbose}\label{wxlogverbose}

\func{void}{wxLogVerbose}{\param{const char*}{ formatString}, \param{...}{}}

For verbose output. Normally, it's suppressed, but
might be activated if the user wishes to know more details about the program
progress (another, but possibly confusing name for the same function is {\bf wxLogInfo}).

\membersection{::wxLogStatus}\label{wxlogstatus}

\func{void}{wxLogStatus}{\param{const char*}{ formatString}, \param{...}{}}

For status messages - they will go into the status
bar of the active or specified (as the first argument) \helpref{wxFrame}{wxframe} if it has one.

\membersection{::wxLogSysError}\label{wxlogsyserror}

\func{void}{wxLogSysError}{\param{const char*}{ formatString}, \param{...}{}}

Mostly used by wxWindows itself, but might be
handy for logging errors after system call (API function) failure. It logs the
specified message text as well as the last system error code ({\it errno} or {\it ::GetLastError()} depending
on the platform) and the corresponding error
message. The second form of this function takes the error code explitly as the
first argument.

\membersection{::wxLogDebug}\label{wxlogdebug}

\func{void}{wxLogDebug}{\param{const char*}{ formatString}, \param{...}{}}

The right function for debug output. It only
does anything at all in the debug mode (when the preprocessor symbol \_\_WXDEBUG\_\_ is defined)
and expands to nothing in release mode (otherwise).

\membersection{::wxLogTrace}\label{wxlogtrace}

\func{void}{wxLogTrace}{\param{const char*}{ formatString}, \param{...}{}}

\func{void}{wxLogTrace}{\param{wxTraceMask}{ mask}, \param{const char*}{ formatString}, \param{...}{}}

As {\bf wxLogDebug}, only does something in debug
build. The reason for making it a separate function from it is that usually
there are a lot of trace messages, so it might make sense to separate them
from other debug messages which would be flooded in them. Moreover, the second
version of this function takes a trace mask as the first argument which allows
to further restrict the amount of messages generated. The value of {\it mask} can be:

\begin{itemize}\itemsep=0pt
\item wxTraceMemAlloc: trace memory allocation (new/delete)
\item wxTraceMessages: trace window messages/X callbacks
\item wxTraceResAlloc: trace GDI resource allocation
\item wxTraceRefCount: trace various ref counting operations
\end{itemize}

\section{Debugging macros and functions}\label{debugmacros}

Useful macros and functins for error checking and defensive programming. ASSERTs are only
compiled if \_\_WXDEBUG\_\_ is defined, whereas CHECK macros stay in release
builds.

\wxheading{Include files}

<wx/debug.h>

\membersection{::wxOnAssert}\label{wxonassert}

\func{void}{wxOnAssert}{\param{const char*}{ fileName}, \param{int}{ lineNumber}, \param{const char*}{ msg = NULL}}

This function may be redefined to do something non trivial and is called
whenever one of debugging macros fails (i.e. condition is false in an
assertion).
% TODO: this should probably be an overridable in wxApp.

\membersection{wxASSERT}\label{wxassert}

\func{}{wxASSERT}{\param{}{condition}}

Assert macro. An error message will be generated if the condition is FALSE in
debug mode, but nothing will be done in the release build.

Please note that the condition in wxASSERT() should have no side effects
because it will not be executed in release mode at all.

See also: \helpref{wxASSERT\_MSG}{wxassertmsg}

\membersection{wxASSERT\_MSG}\label{wxassertmsg}

\func{}{wxASSERT\_MSG}{\param{}{condition}, \param{}{msg}}

Assert macro with message. An error message will be generated if the condition is FALSE.

See also: \helpref{wxASSERT}{wxassert}

\membersection{wxFAIL}\label{wxfail}

\func{}{wxFAIL}{\void}

Will always generate an assert error if this code is reached (in debug mode).

See also: \helpref{wxFAIL\_MSG}{wxfailmsg}

\membersection{wxFAIL\_MSG}\label{wxfailmsg}

\func{}{wxFAIL\_MSG}{\param{}{msg}}

Will always generate an assert error with specified message if this code is reached (in debug mode).

This macro is useful for marking unreachable" code areas, for example
it may be used in the "default:" branch of a switch statement if all possible
cases are processed above.

See also: \helpref{wxFAIL}{wxfail}

\membersection{wxCHECK}\label{wxcheck}

\func{}{wxCHECK}{\param{}{condition}, \param{}{retValue}}

Checks that the condition is true, returns with the given return value if not (FAILs in debug mode).
This check is done even in release mode.

\membersection{wxCHECK\_MSG}\label{wxcheckmsg}

\func{}{wxCHECK\_MSG}{\param{}{condition}, \param{}{retValue}, \param{}{msg}}

Checks that the condition is true, returns with the given return value if not (FAILs in debug mode).
This check is done even in release mode.

This macro may be only used in non void functions, see also 
\helpref{wxCHECK\_RET}{wxcheckret}.

\membersection{wxCHECK\_RET}\label{wxcheckret}

\func{}{wxCHECK\_RET}{\param{}{condition}, \param{}{msg}}

Checks that the condition is true, and returns if not (FAILs with given error
message in debug mode). This check is done even in release mode.

This macro should be used in void functions instead of 
\helpref{wxCHECK\_MSG}{wxcheckmsg}.

\membersection{wxCHECK2}\label{wxcheck2}

\func{}{wxCHECK2}{\param{}{condition}, \param{}{operation}}

Checks that the condition is true and \helpref{wxFAIL}{wxfail} and execute 
{\it operation} if it is not. This is a generalisation of 
\helpref{wxCHECK}{wxcheck} and may be used when something else than just
returning from the function must be done when the {\it condition} is false.

This check is done even in release mode.

\membersection{wxCHECK2\_MSG}\label{wxcheck2msg}

\func{}{wxCHECK2}{\param{}{condition}, \param{}{operation}, \param{}{msg}}

This is the same as \helpref{wxCHECK2}{wxcheck2}, but 
\helpref{wxFAIL\_MSG}{wxfailmsg} with the specified {\it msg} is called
instead of wxFAIL() if the {\it condition} is false.

