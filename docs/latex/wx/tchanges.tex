\section{Changes since 2.4.x}\label{changes}

Listed here are the depreciated and incompatible changes made to wxWidgets.

For other changes (such as additional features, bug fixes, etc.) see the changes.txt file located in the docs directory of your wxWidgets directory.



\subsection{Incompatible changes since 2.4.x}\label{incompatiblesince24}

\subsubsection{New window repainting behaviour}\label{newwindowrepainting}

Windows are no longer fully repainted when resized, use new style \windowstyle{wxFULL\_REPAINT\_ON\_RESIZE} to force this (\windowstyle{wxNO\_FULL\_REPAINT\_ON\_RESIZE} still exists but doesn't do anything any more, this behaviour is default now).



\subsubsection{Window class member changes}\label{windowclassmemberchanges}

wxWindow::m\_font and m\_backgroundColour/m\_foregroundColour are no longer always set, use \helpref{GetFont()}{wxwindowgetfont}, \helpref{GetBack}{wxwindowgetbackgroundcolour}/\helpref{ForegroundColour()}{wxwindowgetforegroundcolour} to access them, and they will be dynamically determined if necessary.



\subsubsection{Sizers Internal Overhaul}\label{sizersinternaloverhaul}

\helpref{The Sizers}{sizeroverview} have had some fundamental internal changes in the 2.5.2 and 2.5.3 releases intended to make them do more of the "Right Thing" but also be as backwards compatible as possible.  First a bit about how things used to work:

\begin{itemize}\itemsep=0pt
\item The size that a window had when \helpref{Add()}{wxsizeradd}ed to the sizer was assumed
      to be its minimal size, and that size would always be used by
      default when calculating layout size and positions, and the
      sizer itself would keep track of that minimal size.

\item If the window item was \helpref{Add()}{wxsizeradd}ed with the \windowstyle{wxADJUST\_MINSIZE} flag
      then when layout was calculated the item's \helpref{GetBestSize}{wxwindowgetbestsize} would be
      used to reset the minimal size that the sizer used.
\end{itemize}

  The main thrust of the new Sizer changes was to make behaviour like
  \windowstyle{wxADJUST\_MINSIZE} be the default, and also to push the tracking of
  the minimal size to the window itself (since it knows its own needs)
  instead of having the sizer take care of it.  Consequently these
  changes were made:

\begin{itemize}\itemsep=0pt
\item The \windowstyle{wxFIXED\_MINSIZE} flag was added to allow for the old
      behaviour.  When this flag is used the size a window has when
      \helpref{Add()}{wxsizeradd}ed to the sizer will be treated as its minimal size and it
      will not be readjusted on each layout.

\item The min size stored in wxWindow and settable with \helpref{SetSizeHints}{wxwindowsetsizehints} or
      \helpref{SetMinSize}{wxwindowsetminsize} will by default be used by the sizer (if it was set)
      as the minimal size of the sizer item.  If the minsize was not
      set (or was only partially set) then the window's best size is
      fetched and it is used instead of (or blended with) the minsize.
      \helpref{wxWindow::GetBestFittingSize}{wxwindowgetbestfittingsize} was added to facilitate getting the
      size to be used by the sizers.

\item The best size of a window is cached so it doesn't need to
      recalculated on every layout. \helpref{wxWindow::InvalidateBestSize}{wxwindowinvalidatebestsize} was
      added and should be called (usually just internally in control
      methods) whenever something is done that would make the best
      size change.

\item All \helpref{wxControls}{wxcontrol} were changed to set the minsize to what is passed
      to the constructor or their Create method, and also to set the real
      size of the control to the blending of the minsize and bestsize.
      \helpref{wxWindow::SetBestFittingSize}{wxwindowsetbestfittingsize} was added to help with this,
      although most controls don't need to call it directly because it
      is called indirectly via the \helpref{SetInitialBestSize}{wxwindowsetinitialbestsize} called in the base
      classes.
\end{itemize}

  At this time, the only situation known not to work the same as
  before is the following:

\begin{verbatim}
win = new SomeWidget(parent);
win->SetSize(SomeNonDefaultSize);
sizer->Add(win);
\end{verbatim}

  In this case the old code would have used the new size as the
  minimum, but now the sizer will use the default size as the minimum
  rather than the size set later.  It is an easy fix though, just move
  the specification of the size to the constructor (assuming that
  SomeWidget will set its minsize there like the rest of the controls
  do) or call \helpref{SetMinSize}{wxwindowsetminsize} instead of \helpref{SetSize}{wxwindowsetsize}.

  In order to fit well with this new scheme of things, all \helpref{wxControls}{wxcontrol}
  or custom controls should do the following things.  (Depending on
  how they are used you may also want to do the same thing for
  non-control custom windows.)

\begin{itemize}\itemsep=0pt
\item Either override or inherit a meaningful \helpref{DoGetBestSize}{wxwindowdogetbestsize} method
      that calculates whatever size is "best" for the control.  Once
      that size is calculated then there should normally be a call to
      \helpref{CacheBestSize}{wxwindowcachebestsize} to save it for later use, unless for some reason
      you want the best size to be recalculated on every layout.

\item Any method that changes the attributes of the control such that
      the best size will change should call \helpref{InvalidateBestSize}{wxwindowinvalidatebestsize} so it
      will be recalculated the next time it is needed.

\item The control's constructor and/or Create method should ensure
      that the minsize is set to the size passed in, and that the
      control is sized to a blending of the min size and best size.
      This can be done by calling \helpref{SetBestFittingSize}{wxwindowsetbestfittingsize}.
\end{itemize}



\subsubsection{Massive wxURL Rewrite}\label{wxurlrewrite}

\helpref{wxURL}{wxurl} has undergone some radical changes.

\begin{itemize}\itemsep=0pt
\item Many accessors of \helpref{wxURL}{wxurl} - GetHostName, GetProtocolName, and GetPath,
      have been replaced by its parent's (\helpref{wxURI}{wxuri}) counterparts - \helpref{GetServer}{wxurigetserver},
      \helpref{GetScheme}{wxurigetscheme}, and \helpref{GetPath}{wxurigetpath}, respectively.

\item ConvertToValidURI has been replaced by \helpref{wxURI}{wxuri}.  Do not use
      ConvertToValidURI for future applications.

\item ConvertFromURI has been replaced by \helpref{wxURI::Unescape}{wxuriunescape}.
\end{itemize}



\subsubsection{Less drastic incompatible changes since 2.4.x}\label{24incompatiblelessdrastic}

- no initialization/cleanup can be done in \helpref{wxApp}{wxappctor}/\helpref{~wxApp}{wxappdtor} because they are
  now called much earlier/later than before; please move any exiting code
  from there to \helpref{wxApp::OnInit()}{wxapponinit}/\helpref{OnExit()}{wxapponexit}

- also, \helpref{OnExit()}{wxapponexit} is not called if \helpref{OnInit()}{wxapponinit} fails

- finally the program exit code is \helpref{OnRun()}{wxapponrun} return value, not \helpref{OnExit()}{wxapponexit} one

- \texttt{wxTheApp} can't be assigned to any longer, use \helpref{wxApp::SetInstance()}{wxappsetinstance} instead

- wxFileType::GetIcon() returns wxIconLocation, not wxIcon

- wxColourDatabase is not a wxList any more, use AddColour to add new colours

- wxWindow::Clear() is now called ClearBackground()

- pointer returned by wxFont::GetNativeFontInfo() must not be deleted now

- wxMouseEvent::Moving() doesn't return true if mouse is being dragged any more

- (most) controls now inherit parents colours by default, override
  ShouldInheritColours() to return false if you don't want this to happen

- \helpref{wxApp::SendIdleEvents()}{wxappsendidleevents} now takes 2 arguments

- wxTabView::GetLayers() changed return type from wxList& to wxTabLayerList&
  (when WXWIN\_COMPATIBILITY\_2\_4 == 0)

- wxID\_SEPARATOR (id used for the menu separators) value changed from -1 to -2

- wxGetNumberFromUser() is now in separate wx/numdlg.h, not wx/textdlg.h

- wxChoice and wxCombobox now handle their size in the same way as in all the
  other ports under MSW, new code is actually correct but different from weird
  stuff they were doing before so the behaviour of your programs might change

- wxTaskBarIcon objects must now be destroyed before the application can exit.
  Previously, the application terminated if there were no top level windows;
  now it terminates if there are no top level windows or taskbar icons left.

- wxZlibInputStream is not by default compatible with the output of the
  2.4.x version of wxZlibOutputStream. However, there is a compatibility mode,
  switched on by passing wxZLIB\_24COMPATIBLE to the constructor.

- when WXWIN\_COMPATIBILITY\_2\_4 == 0 wxHashTable uses a new implementation
  not using wxList keyed interface (the same used when wxUSE\_STL == 1),
  the only incompatibility being that Next() returns a wxHashTable::Node*
  instead of a wxNode*.

- non-const wxDC methods GetBackground(), GetBrush(), GetFont() and GetPen()
  as well as wxWindow methods GetFont() and GetCursor() don't exist any more,
  please fix your code -- it never worked correctly anyhow if you modified the
  objects returned by these methods so you should simply switch to using const
  methods.

- wxWindow::GetFont() now returns wxFont object instead of reference

- EVT\_XXX macros are now type-safe; code that uses wrong type for event
  handler's argument will no longer compile.

- Identical functionality of wxFileDialog::ParseWildcard,
  wxGenericDirCtrl::ParseFilter, Motif and MSW parsing native dialogs
  is now accessible in ::wxParseCommonDialogsFilter

- wxNotebookSizer and wxBookCtrlSizer are now deprecated -- they are no longer
  needed, you can treat wxNotebook as any other control and put it directly
  into the sizer that was wxNotebookSizer's parent sizer in old code.

- wxFile methods now return either wxFileOffset or wxFileSize\_t which may be a
  64 bit integer type, even on 32 bit platforms, instead of off\_t and so the
  return value of wxFile::Length(), for example, shouldn't be assigned to off\_t
  variable any more (the compiler might warn you about this)

- wxListItem::m\_data is now of type wxUIntPtr, not long, for compatibility
  with 64 bit systems

- wxSizer::Add/Insert returns pointer to wxSizerItem just added so conditions
  writeen with if(Add(..)==true) will not work. Use if(Add(..)) instead.

- New wxBrush::IsHatch() checking for brush type replaces IS_HATCH macro.



\subsection{Depreciated changes since 2.4.x}\label{depreciatedsince24}

- wxURL::GetInputStream() and similar functionality has been depreciated in
  favor of other ways of connecting, such as though sockets or wxFileSystem.

- wxDocManager::GetNoHistoryFiles() renamed to GetHistoryFilesCount()

- wxSizer::Remove(wxWindow *), use Detach() instead [it is more clear]

- wxSizer::Set/GetOption(): use Set/GetProportion() instead

- wxKeyEvent::KeyCode(): use GetKeyCode instead

- wxList::Number, First, Last, Nth: use GetCount, GetFirst/Last, Item instead

- wxNode::Next, Previous, Data: use GetNext, GetPrevious, GetData instead

- wxListBase::operator wxList&(): use typesafe lists instead

- wxTheFontMapper: use wxFontMapper::Get() instead

- wxStringHashTable: use wxHashMap instead

- wxHashTableLong: use wxHashMap instead

- wxArrayString::GetStringArray: use wxCArrayString or alternative wxWidgets
                                 methods taking wxArrayString

- wxArrayString::Remove(index, count): use RemoveAt instead

- wxTreeItemId conversion to long is deprecated and shouldn't be used

- wxTreeCtrl::GetFirst/NextChild() 2nd argument now has type wxTreeItemIdValue
  and not long, please change declarations of "cookie"s in your code
  accordingly -- otherwise your code won't work on 64 bit platforms

- [MSW only] wxWindow::GetUseCtl3D(), GetTransparentBackground() and
             SetTransparent() as well as wxNO\_3D and wxUSER\_COLOURS styles

- wxList keyed interface: use wxHashMap instead

- wxColourDatabase::FindColour(): use Find() instead (NB: different ret type)

- wxHashTable::Next: use wxHashTable::Node* or
                     wxHashTable::compatibility\_iterator to store the return
                     value

- wxWave class; use wxSound instead

- The wxHIDE\_READONLY flag for wxFileDialog was not implemented
  and has now been removed

- wxTaskBarIcon::OnXXX() virtual methods: use events instead

- obsolete and not used wxUSE\_GENERIC\_DIALOGS\_IN\_MSW has been removed

- wxDbTable::wxDbTable with wxChar* deprecated, same with wxString& instead
