% -----------------------------------------------------------------------------
% wxInputStream
% -----------------------------------------------------------------------------
\section{\class{wxInputStream}}\label{wxinputstream}

\wxheading{Derived from}

\helpref{wxStreamBase}{wxstreambase}

\wxheading{Include files}

<wx/stream.h>

\wxheading{See also}

\helpref{wxStreamBuffer}{wxstreambuffer}

% -----------
% ctor & dtor
% -----------
\membersection{wxInputStream::wxInputStream}

\func{}{wxInputStream}{\void}

Creates a dummy input stream.

\func{}{wxInputStream}{\param{wxStreamBuffer *}{sbuf}}

Creates an input stream using the specified stream buffer \it{sbuf}. This
stream buffer can point to another stream.

\membersection{wxInputStream::\destruct{wxInputStream}}

\func{}{\destruct{wxInputStream}}{\void}

Destructor.

\membersection{wxInputStream::GetC}

\func{char}{GetC}{\void}

Returns the first character in the input queue and removes it.

\membersection{wxInputStream::InputStreamBuffer}

\func{wxStreamBuffer*}{InputStreamBuffer}{\void}

Returns the stream buffer associated with the input stream.

\membersection{wxInputStream::LastRead}

\constfunc{size\_t}{LastRead}{\void}

Returns the last number of bytes read.

\membersection{wxInputStream::Peek}

\func{char}{Peek}{\void}

Returns the first character in the input queue without removing it.

\membersection{wxInputStream::Read}

\func{wxInputStream\&}{Read}{\param{void *}{buffer}, \param{size\_t}{ size}}

Reads the specified amount of bytes and stores the data in \it{buffer}.

\wxheading{Warning}

The buffer absolutely needs to have at least the specified size.

\wxheading{Return value}

This function returns a reference on the current object, so the user can test
any states of the stream right away.

\func{wxInputStream\&}{Read}{\param{wxOutputStream\&}{ stream\_out}}

Reads data from the input queue and stores it in the specified output stream.
The data is read until an error is raised by one of the two streams.

\wxheading{Return value}

This function returns a reference on the current object, so the user can test
any states of the stream right away.

\membersection{wxInputStream::SeekI}

\func{off\_t}{SeekI}{\param{off\_t}{ pos}, \param{wxSeekMode}{ mode = wxFromStart}}

Changes the stream current position.

\membersection{wxInputStream::TellI}

\constfunc{off\_t}{TellI}{\void}

Returns the current stream position.

