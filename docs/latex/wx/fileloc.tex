%%%%%%%%%%%%%%%%%%%%%%%%%%%%%%%%%%%%%%%%%%%%%%%%%%%%%%%%%%%%%%%%%%%%%%%%%%%%%%%
%% Name:        fileloc.tex
%% Purpose:     wxFileLocator documentation
%% Author:      Vadim Zeitlin
%% Modified by:
%% Created:     2004-10-17
%% RCS-ID:      $Id$
%% Copyright:   (c) 2004 Vadim Zeitlin <vadim@wxwindows.org>
%% License:     wxWidgets license
%%%%%%%%%%%%%%%%%%%%%%%%%%%%%%%%%%%%%%%%%%%%%%%%%%%%%%%%%%%%%%%%%%%%%%%%%%%%%%%

\section{\class{wxFileLocator}}\label{wxfilelocator}

wxFileLocator returns the standard locations in the file system and should be
used by the programs to find their data files in a portable way.

Please note that this is not a real class because object of this type are never
created but more a namespace containing the class methods which are all static,
so to use wxFileLocator simply call its methods directly.

In the description of the methods below, the example return values are given
for the Unix, Windows and Mac OS X systems, however please note that these are
just the examples and the actual values may differ. Most importantly:
\begin{itemize}
    \item Unix: \texttt{/usr} should in general be replaced by the
        program installation prefix which is by default \texttt{/usr/local} but
        may be any other path as well.
    \item Windows: the system administrator may change the standard
        directories locations, i.e. the Windows directory may be named
        \texttt{W:$\backslash$Win2003} instead of default
        \texttt{C:$\backslash$Windows}
\end{itemize}

The strings \texttt{\textit{appname}} and \texttt{\textit{username}} should be
replaced with the value returned by \helpref{wxApp::GetAppName}{wxappgetappname} 
and the name of the currently logged in user, respectively.

The directories returned by the methods of this class may or may not exist. If
they don't exist, it's up to the caller to create them, wxFileLocator doesn't
do it.

Finally note that under Mac, these functions only work for the bundled
applications.

\wxheading{Derived from}

No base class

\wxheading{Include files}

<wx/fileloc.h>


\latexignore{\rtfignore{\wxheading{Members}}}


\membersection{wxFileLocator::GetConfigDir}\label{wxfilelocatorgetconfigdir}

\func{static wxString}{GetConfigDir}{\void}

Return the directory containing the system config files.

Example return values:
\begin{itemize}
    \item Unix: \texttt{/etc}
    \item Windows: \texttt{C:$\backslash$Windows}
    \item Mac: \texttt{/Library/Preferences}
\end{itemize}

\wxheading{See also}

\helpref{wxFileConfig}{wxfileconfig}


\membersection{wxFileLocator::GetDataDir}\label{wxfilelocatorgetdatadir}

\func{static wxString}{GetDataDir}{\void}

Return the location of the applications global, i.e. not user-specific,
data files.

Example return values:
\begin{itemize}
    \item Unix: \texttt{/usr/share/\textit{appname}}
    \item Windows: \texttt{C:$\backslash$Program Files$\backslash$\textit{appname}}
    \item Mac: \texttt{\textit{appname}.app/Contents} bundle subdirectory
\end{itemize}

\wxheading{See also}

\helpref{GetLocalDataDir}{wxfilelocatorgetlocaldatadir}


\membersection{wxFileLocator::GetLocalDataDir}\label{wxfilelocatorgetlocaldatadir}

\func{static wxString}{GetLocalDataDir}{\void}

Return the location for application data files which are host-specific and
can't, or shouldn't, be shared with the other machines.

This is the same as \helpref{GetDataDir()}{wxfilelocatorgetdatadir} except
under Unix where it returns \texttt{/etc/\textit{appname}}.


\membersection{wxFileLocator::GetPluginsDir}\label{wxfilelocatorgetpluginsdir}

\func{static wxString}{GetPluginsDir}{\void}

Return the directory where the loadable modules (plugins) live.

Example return values:
\begin{itemize}
    \item Unix: \texttt{/usr/lib/\textit{appname}}
    \item Windows: the directory of the executable file
    \item Mac: \texttt{\textit{appname}.app/Contents/Plugins} bundle subdirectory
\end{itemize}

\wxheading{See also}

\helpref{wxDynamicLibrary}{wxdynamiclibrary}


\membersection{wxFileLocator::GetUserConfigDir}\label{wxfilelocatorgetuserconfigdir}

\func{static wxString}{GetUserConfigDir}{\void}

Return the directory for the user config files:
\begin{itemize}
    \item Unix: \texttt{\verb|~|} (the home directory)
    \item Windows: \texttt{C:$\backslash$Documents and Settings$\backslash$\textit{username}}
    \item Mac: \texttt{\verb|~|/Library/Preferences}
\end{itemize}

Only use this method if you have a single configuration file to put in this
directory, otherwise \helpref{GetUserDataDir()}{wxfilelocatorgetuserdatadir} is
more appropriate.


\membersection{wxFileLocator::GetUserDataDir}\label{wxfilelocatorgetuserdatadir}

\func{static wxString}{GetUserDataDir}{\void}

Return the directory for the user-dependent application data files:
\begin{itemize}
    \item Unix: \texttt{\verb|~|/.\textit{appname}}
    \item Windows: \texttt{C:$\backslash$Documents and Settings$\backslash$\textit{username}
                             $\backslash$Application Data
                             $\backslash$\textit{appname}}
    \item Mac: \texttt{\verb|~|/Library/\textit{appname}}
\end{itemize}


\membersection{wxFileLocator::GetUserLocalDataDir}\label{wxfilelocatorgetuserlocaldatadir}

\func{static wxString}{GetUserLocalDataDir}{\void}

Return the directory for user data files which shouldn't be shared with
the other machines.

This is the same as \helpref{GetUserDataDir()}{wxfilelocatorgetuserdatadir} for
all platforms except Windows where it returns 
\texttt{C:$\backslash$Documents and Settings$\backslash$\textit{username}
                             $\backslash$Local Settings
                             $\backslash$Application Data
                             $\backslash$\textit{appname}}


