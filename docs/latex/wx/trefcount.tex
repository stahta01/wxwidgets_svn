\section{Reference counting}\label{trefcount}

\subsection{Why you shouldn't care about it}\label{refcount}

Many wxWidgets objects use a technique known as \it{reference counting}, also known
as {\it copy on write} (COW).
This means that when an object is assigned to another, no copying really takes place:
only the reference count on the shared object data is incremented and both objects
share the same data (a very fast operation).

But as soon as one of the two (or more) objects is modified, the data has to be
copied because the changes to one of the objects shouldn't be seen in the
others. As data copying only happens when the object is written to, this is
known as COW.

What is important to understand is that all this happens absolutely
transparently to the class users and that whether an object is shared or not
is not seen from the outside of the class - in any case, the result of any
operation on it is the same.

\subsection{Object comparison}\label{refcountequality}

The $==$ and $!=$ operators of \helpref{wxWidgets COW objects}{refcountlist}
always do a {\tt deep} comparison.

This means that the equality operator will return \true if two objects are
identic and not only if they share the same data.

Note that wxWidgets follows the {\it STL philosophy}: when a comparison operator cannot
be implemented efficiently (like for e.g. wxImage's $==$ operator which would need to
compare pixel-by-pixel the entire image's data), it's not implemented at all.

That's why not all reference-counted wxWidgets classes provide comparison operators.

Also note that if you only need to do a {\tt shallow} comparison between two
\helpref{wxObject}{wxobject}-derived classes, you should not use the $==$ and $!=$ operators
but rather the \helpref{wxObject::IsSameAs}{wxobjectissameas} function.


\subsection{Object destruction}\label{refcountdestruct}

When a COW object destructor is called, it may not delete the data: if it's shared,
the destructor will just decrement the shared data's reference count without destroying it.

Only when the destructor of the last object owning the data is called, the data is really
destroyed. As for all other COW-things, this happens transparently to the class users so
that you shouldn't care about it.


\subsection{List of reference-counted wxWidgets classes}\label{refcountlist}

The following classes in wxWidgets have efficient (i.e. fast) assignment operators
and copy constructors since they are reference-counted:

\helpref{wxAcceleratorTable}{wxacceleratortable}\\
\helpref{wxAnimation}{wxanimation}\\
\helpref{wxBitmap}{wxbitmap}\\
\helpref{wxBrush}{wxbrush}\\
\helpref{wxCursor}{wxcursor}\\
\helpref{wxFont}{wxfont}\\
\helpref{wxIcon}{wxicon}\\
\helpref{wxImage}{wximage}\\
\helpref{wxMetafile}{wxmetafile}\\
\helpref{wxPalette}{wxpalette}\\
\helpref{wxPen}{wxpen}\\
\helpref{wxRegion}{wxregion}\\
\helpref{wxString}{wxstring}\\
\helpref{wxVariant}{wxvariant}\\
\helpref{wxVariantData}{wxvariantdata}

Note that the list above reports the objects which are reference-counted in all ports of
wxWidgets; some ports may use this tecnique also for other classes.
\subsection{Make your own reference-counted class}\label{wxobjectoverview}

Reference counting can be implemented easily using \helpref{wxObject}{wxobject}
and \helpref{wxObjectRefData}{wxobjectrefdata} classes.

First, derive a new class from \helpref{wxObjectRefData}{wxobjectrefdata} and
put there the memory-consuming data.

Then derive a new class from \helpref{wxObject}{wxobject} and implement there
the public interface which will be seen by the user of your class.
You'll probably want to add a function to your class which does the cast from
\helpref{wxObjectRefData}{wxobjectrefdata} to your class-specific shared data; e.g.:

\begin{verbatim}
    MyClassRefData *GetData() const   { return wx_static_cast(MyClassRefData*, m_refData); }
\end{verbatim}

in fact, all times you'll need to read the data from your wxObject-derived class,
you'll need to call such function.

Very important, all times you need to actually modify the data placed inside your
wxObject-derived class, you must first call the wxObject::UnShare
function to be sure that the modifications won't affect other instances which are
eventually sharing your object's data.

