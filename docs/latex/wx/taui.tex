\section{wxAUI overview}\label{wxauioverview}

Class: \helpref{wxAuiManager}{wxauimanager}, \helpref{wxAuiPaneInfo}{wxauipaneinfo}

wxAUI stands for Advanced User Interface and the wxAUI framework
aims to give its user a cutting edge interface for use with the
wxWidgets based applications. The original wxAUI sources have
kindly been made available under the wxWindows licence
by Kirix Corp. and they have since then been integrated into
wxWidgets CVS and further improved.

wxAUI attempts to encapsulate the following aspects of the user interface:

{\bf Frame Management:}
Frame management provides the means to open, move and hide common
controls that are needed to interact with the document, and allow these
configurations to be saved into different perspectives and loaded at a
later time.

{\bf Toolbars:}
Toolbars are a specialized subset of the frame management system and
should behave similarly to other docked components. However, they also
require additional functionality, such as "spring-loaded" rebar support,
"chevron" buttons and end-user customizability.

{\bf Modeless Controls:}
Modeless controls expose a tool palette or set of options that float
above the application content while allowing it to be accessed. Usually
accessed by the toolbar, these controls disappear when an option is
selected, but may also be "torn off" the toolbar into a floating frame
of their own.

{\bf Look and Feel:}
Look and feel encompasses the way controls are drawn, both when shown
statically as well as when they are being moved. This aspect of user
interface design incorporates "special effects" such as transparent
window dragging as well as frame animation.

wxAUI adheres to the following principles:

Use native floating frames to obtain a native look and feel for all
platforms. Use existing wxWidgets code where possible, such as sizer
implementation for frame management. Use classes included in wxCore
and wxBase only. Use standard wxWidgets coding conventions. 

