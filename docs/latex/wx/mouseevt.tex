\section{\class{wxMouseEvent}}\label{wxmouseevent}

This event class contains information about mouse events.
See \helpref{wxWindow::OnMouseEvent}{wxwindowonmouseevent}.

\wxheading{Derived from}

\helpref{wxEvent}{wxevent}

\wxheading{Event table macros}

To process a mouse event, use these event handler macros to direct input to member
functions that take a wxMouseEvent argument.

\twocolwidtha{7cm}
\begin{twocollist}\itemsep=0pt
\twocolitem{{\bf EVT\_LEFT\_DOWN(func)}}{Process a wxEVT\_LEFT\_DOWN event.}
\twocolitem{{\bf EVT\_LEFT\_UP(func)}}{Process a wxEVT\_LEFT\_UP event.}
\twocolitem{{\bf EVT\_LEFT\_DCLICK(func)}}{Process a wxEVT\_LEFT\_DCLICK event.}
\twocolitem{{\bf EVT\_MIDDLE\_DOWN(func)}}{Process a wxEVT\_MIDDLE\_DOWN event.}
\twocolitem{{\bf EVT\_MIDDLE\_UP(func)}}{Process a wxEVT\_MIDDLE\_UP event.}
\twocolitem{{\bf EVT\_MIDDLE\_DCLICK(func)}}{Process a wxEVT\_MIDDLE\_DCLICK event.}
\twocolitem{{\bf EVT\_RIGHT\_DOWN(func)}}{Process a wxEVT\_RIGHT\_DOWN event.}
\twocolitem{{\bf EVT\_RIGHT\_UP(func)}}{Process a wxEVT\_RIGHT\_UP event.}
\twocolitem{{\bf EVT\_RIGHT\_DCLICK(func)}}{Process a wxEVT\_RIGHT\_DCLICK event.}
\twocolitem{{\bf EVT\_MOTION(func)}}{Process a wxEVT\_MOTION event.}
\twocolitem{{\bf EVT\_ENTER\_WINDOW(func)}}{Process a wxEVT\_ENTER\_WINDOW event.}
\twocolitem{{\bf EVT\_LEAVE\_WINDOW(func)}}{Process a wxEVT\_LEAVE\_WINDOW event.}
\end{twocollist}%

\latexignore{\rtfignore{\wxheading{Members}}}

\membersection{wxMouseEvent::m\_altDown}

\member{bool}{m\_altDown}

TRUE if the Alt key is pressed down.

\membersection{wxMouseEvent::m\_controlDown}

\member{bool}{m\_controlDown}

TRUE if control key is pressed down.

\membersection{wxMouseEvent::m\_leftDown}

\member{bool}{m\_leftDown}

TRUE if the left mouse button is currently pressed down.

\membersection{wxMouseEvent::m\_middleDown}

\member{bool}{m\_middleDown}

TRUE if the middle mouse button is currently pressed down.

\membersection{wxMouseEvent::m\_rightDown}

\member{bool}{m\_rightDown}

TRUE if the right mouse button is currently pressed down.

\membersection{wxMouseEvent::m\_leftDown}

\member{bool}{m\_leftDown}

TRUE if the left mouse button is currently pressed down.

\membersection{wxMouseEvent::m\_metaDown}

\member{bool}{m\_metaDown}

TRUE if the Meta key is pressed down.

\membersection{wxMouseEvent::m\_shiftDown}

\member{bool}{m\_shiftDown}

TRUE if shift is pressed down.

\membersection{wxMouseEvent::m\_x}

\member{float}{m\_x}

X-coordinate of the event.

\membersection{wxMouseEvent::m\_y}

\member{float}{m\_y}

Y-coordinate of the event.

\membersection{wxMouseEvent::wxMouseEvent}

\func{}{wxMouseEvent}{\param{WXTYPE}{ mouseEventType = 0}, \param{int}{ id = 0}}

Constructor. Valid event types are:

\begin{itemize}
\itemsep=0pt
\item {\bf wxEVT\_ENTER\_WINDOW}
\item {\bf wxEVT\_LEAVE\_WINDOW}
\item {\bf wxEVT\_LEFT\_DOWN}
\item {\bf wxEVT\_LEFT\_UP}
\item {\bf wxEVT\_LEFT\_DCLICK}
\item {\bf wxEVT\_MIDDLE\_DOWN}
\item {\bf wxEVT\_MIDDLE\_UP}
\item {\bf wxEVT\_MIDDLE\_DCLICK}
\item {\bf wxEVT\_RIGHT\_DOWN}
\item {\bf wxEVT\_RIGHT\_UP}
\item {\bf wxEVT\_RIGHT\_DCLICK}
\item {\bf wxEVT\_MOTION}
\end{itemize}

\membersection{wxMouseEvent::AltDown}

\func{bool}{AltDown}{\void}

Returns TRUE if the Alt key was down at the time of the event.

\membersection{wxMouseEvent::Button}

\func{bool}{Button}{\param{int}{ button}}

Returns TRUE if the identified mouse button is changing state. Valid
values of {\it button} are 1, 2 or 3 for left, middle and right
buttons respectively.

Not all mice have middle buttons so a portable application should avoid
this one.

\membersection{wxMouseEvent::ButtonDClick}\label{buttondclick}

\func{bool}{ButtonDClick}{\param{int}{ but = -1}}

If the argument is omitted, this returns TRUE if the event was a mouse
double click event. Otherwise the argument specifies which double click event
was generated (1, 2 or 3 for left, middle and right buttons respectively).

\membersection{wxMouseEvent::ButtonDown}

\func{bool}{ButtonDown}{\param{int}{ but = -1}}

If the argument is omitted, this returns TRUE if the event was a mouse
button down event. Otherwise the argument specifies which button-down event
was generated (1, 2 or 3 for left, middle and right buttons respectively).

\membersection{wxMouseEvent::ButtonUp}

\func{bool}{ButtonUp}{\param{int}{ but = -1}}

If the argument is omitted, this returns TRUE if the event was a mouse
button up event. Otherwise the argument specifies which button-up event
was generated (1, 2 or 3 for left, middle and right buttons respectively).

\membersection{wxMouseEvent::ControlDown}

\func{bool}{ControlDown}{\void}

Returns TRUE if the control key was down at the time of the event.

\membersection{wxMouseEvent::Dragging}

\func{bool}{Dragging}{\void}

Returns TRUE if this was a dragging event (motion while a button is depressed).

\membersection{wxMouseEvent::Entering}\label{wxmouseevententering}

\func{bool}{Entering}{\void}

Returns TRUE if the mouse was entering the window (MS Windows and Motif).

See also \helpref{wxMouseEvent::Leaving}{wxmouseeventleaving}.

\membersection{wxMouseEvent::GetX}\label{wxmouseeventgetx}

\func{float}{GetX}{\void}

Returns X coordinate of the mouse event position.

\membersection{wxMouseEvent::GetY}\label{wxmouseeventgety}

\func{float}{GetY}{\void}

Returns Y coordinate of the mouse event position.

\membersection{wxMouseEvent::IsButton}

\func{bool}{IsButton}{\void}

Returns TRUE if the event was a mouse button event (not necessarily a button down event -
that may be tested using {\it ButtonDown}).

\membersection{wxMouseEvent::Leaving}\label{wxmouseeventleaving}

\func{bool}{Leaving}{\void}

Returns TRUE if the mouse was leaving the window (MS Windows and Motif).

See also \helpref{wxMouseEvent::Entering}{wxmouseevententering}.

\membersection{wxMouseEvent::LeftDClick}

\func{bool}{LeftDClick}{\void}

Returns TRUE if the event was a left double click.

\membersection{wxMouseEvent::LeftDown}

\func{bool}{LeftDown}{\void}

Returns TRUE if the left mouse button changed to down.

\membersection{wxMouseEvent::LeftIsDown}

\func{bool}{LeftIsDown}{\void}

Returns TRUE if the left mouse button is currently down, independent
of the current event type.

\membersection{wxMouseEvent::LeftUp}

\func{bool}{LeftUp}{\void}

Returns TRUE if the left mouse button changed to up.

\membersection{wxMouseEvent::MetaDown}

\func{bool}{MetaDown}{\void}

Returns TRUE if the Meta key was down at the time of the event.

\membersection{wxMouseEvent::MiddleDClick}

\func{bool}{MiddleDClick}{\void}

Returns TRUE if the event was a middle double click.

\membersection{wxMouseEvent::MiddleDown}

\func{bool}{MiddleDown}{\void}

Returns TRUE if the middle mouse button changed to down.

\membersection{wxMouseEvent::MiddleIsDown}

\func{bool}{MiddleIsDown}{\void}

Returns TRUE if the middle mouse button is currently down, independent
of the current event type.

\membersection{wxMouseEvent::MiddleUp}

\func{bool}{MiddleUp}{\void}

Returns TRUE if the middle mouse button changed to up.

\membersection{wxMouseEvent::Moving}

\func{bool}{Moving}{\void}

Returns TRUE if this was a motion event (no buttons depressed).

\membersection{wxMouseEvent::Position}

\func{void}{Position}{\param{float *}{x}, \param{float *}{y}}

Sets *x and *y to the position at which the event occurred. If the
window is a window, the position is converted to logical units
(according to the current mapping mode) with scrolling taken into
account. To get back to device units (for example to calculate where on the
screen to place a dialog box associated with a window mouse event), use
\rtfsp{\bf wxDC::LogicalToDeviceX} and {\bf wxDC::LogicalToDeviceY}.

For example, the following code calculates screen pixel coordinates
from the frame position, window view start (assuming the window is the only
subwindow on the frame and therefore at the top left of it), and the
logical event position. A menu is popped up at the position where the
mouse click occurred. (Note that the application should also check that
the dialog box will be visible on the screen, since the click could have
occurred near the screen edge!)

\begin{verbatim}
  float event_x, event_y;
  event.Position(&event_x, &event_y);
  frame->GetPosition(&x, &y);
  window->ViewStart(&x1, &y1);
  int mouse_x = (int)(window->GetDC()->LogicalToDeviceX(event_x + x - x1);
  int mouse_y = (int)(window->GetDC()->LogicalToDeviceY(event_y + y - y1);

  char *choice = wxGetSingleChoice("Menu", "Pick a node action",
                                 no_choices, choices, frame, mouse_x, mouse_y);
\end{verbatim}

\membersection{wxMouseEvent::RightDClick}

\func{bool}{RightDClick}{\void}

Returns TRUE if the event was a right double click.

\membersection{wxMouseEvent::RightDown}

\func{bool}{RightDown}{\void}

Returns TRUE if the right mouse button changed to down.

\membersection{wxMouseEvent::RightIsDown}

\func{bool}{RightIsDown}{\void}

Returns TRUE if the right mouse button is currently down, independent
of the current event type.

\membersection{wxMouseEvent::RightUp}

\func{bool}{RightUp}{\void}

Returns TRUE if the right mouse button changed to up.

\membersection{wxMouseEvent::ShiftDown}

\func{bool}{ShiftDown}{\void}

Returns TRUE if the shift key was down at the time of the event.

