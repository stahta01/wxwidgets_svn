\section{\class{wxWindow}}\label{wxwindow}

wxWindow is the base class for all windows. Any children of the window will be deleted
automatically by the destructor before the window itself is deleted.

Please note that we documented a number of handler functions (OnChar(), OnMouse() etc.) in this
help text. These must not be called by a user program and are documented only for illustration.
On several platforms, only a few of these handlers are actually written (they are not always
needed) and if you are uncertain on how to add a certain behaviour to a window class, intercept
the respective event as usual and call \helpref{wxEvent::Skip}{wxeventskip} so that the native
platform can implement its native behaviour or just ignore the event if nothing needs to be
done.

\wxheading{Derived from}

\helpref{wxEvtHandler}{wxevthandler}\\
\helpref{wxObject}{wxobject}

\wxheading{Include files}

<wx/window.h>

\wxheading{Window styles}

The following styles can apply to all windows, although they will not always make sense for a particular
window class or on all platforms.

\twocolwidtha{5cm}%
\begin{twocollist}\itemsep=0pt
\twocolitem{\windowstyle{wxSIMPLE\_BORDER}}{Displays a thin border around the window. wxBORDER is the old name
for this style. }
\twocolitem{\windowstyle{wxDOUBLE\_BORDER}}{Displays a double border. Windows only.}
\twocolitem{\windowstyle{wxSUNKEN\_BORDER}}{Displays a sunken border.}
\twocolitem{\windowstyle{wxRAISED\_BORDER}}{Displays a raised border. GTK only. }
\twocolitem{\windowstyle{wxSTATIC\_BORDER}}{Displays a border suitable for a static control. Windows only. }
\twocolitem{\windowstyle{wxTRANSPARENT\_WINDOW}}{The window is transparent, that is, it will not receive paint
events. Windows only.}
\twocolitem{\windowstyle{wxNO\_3D}}{Prevents the children of this window taking on 3D styles, even though
the application-wide policy is for 3D controls. Windows only.}
\twocolitem{\windowstyle{wxTAB\_TRAVERSAL}}{Use this to enable tab traversal for non-dialog windows.}
\twocolitem{\windowstyle{wxWANTS\_CHARS}}{Use this to indicate that the window
wants to get all char events - even for keys like TAB or ENTER which are
usually used for dialog navigation and which wouldn't be generated without
this style}
\twocolitem{\windowstyle{wxNO\_FULL\_REPAINT\_ON\_RESIZE}}{Disables repainting
the window completely when its size is changed - you will have to repaint the
new window area manually if you use this style. Currently only has an effect for
Windows.}
\twocolitem{\windowstyle{wxVSCROLL}}{Use this style to enable a vertical scrollbar. (Still used?) }
\twocolitem{\windowstyle{wxHSCROLL}}{Use this style to enable a horizontal scrollbar. (Still used?) }
\twocolitem{\windowstyle{wxCLIP\_CHILDREN}}{Use this style to eliminate flicker caused by the background being
repainted, then children being painted over them. Windows only.}
\end{twocollist}

See also \helpref{window styles overview}{windowstyles}.

\wxheading{See also}

\helpref{Event handling overview}{eventhandlingoverview}

\latexignore{\rtfignore{\wxheading{Members}}}

\membersection{wxWindow::wxWindow}\label{wxwindowctor}

\func{}{wxWindow}{\void}

Default constructor.

\func{}{wxWindow}{\param{wxWindow*}{ parent}, \param{wxWindowID }{id},
 \param{const wxPoint\& }{pos = wxDefaultPosition},
 \param{const wxSize\& }{size = wxDefaultSize},
 \param{long }{style = 0},
 \param{const wxString\& }{name = wxPanelNameStr}}

Constructs a window, which can be a child of a frame, dialog or any other non-control window.

\wxheading{Parameters}

\docparam{parent}{Pointer to a parent window.}

\docparam{id}{Window identifier. If -1, will automatically create an identifier.}

\docparam{pos}{Window position. wxDefaultPosition is (-1, -1) which indicates that wxWindows
should generate a default position for the window. If using the wxWindow class directly, supply
an actual position.}

\docparam{size}{Window size. wxDefaultSize is (-1, -1) which indicates that wxWindows
should generate a default size for the window. If no suitable size can  be found, the
window will be sized to 20x20 pixels so that the window is visible but obviously not
correctly sized. }

\docparam{style}{Window style. For generic window styles, please see \helpref{wxWindow}{wxwindow}.}

\docparam{name}{Window name.}

\membersection{wxWindow::\destruct{wxWindow}}

\func{}{\destruct{wxWindow}}{\void}

Destructor. Deletes all subwindows, then deletes itself. Instead of using
the {\bf delete} operator explicitly, you should normally
use \helpref{wxWindow::Destroy}{wxwindowdestroy} so that wxWindows
can delete a window only when it is safe to do so, in idle time.

\wxheading{See also}

\helpref{Window deletion overview}{windowdeletionoverview},\rtfsp
\helpref{wxWindow::OnCloseWindow}{wxwindowonclosewindow},\rtfsp
\helpref{wxWindow::Destroy}{wxwindowdestroy},\rtfsp
\helpref{wxCloseEvent}{wxcloseevent}

\membersection{wxWindow::AddChild}

\func{virtual void}{AddChild}{\param{wxWindow* }{child}}

Adds a child window.  This is called automatically by window creation
functions so should not be required by the application programmer.

\wxheading{Parameters}

\docparam{child}{Child window to add.}

\membersection{wxWindow::CaptureMouse}\label{wxwindowcapturemouse}

\func{virtual void}{CaptureMouse}{\void}

Directs all mouse input to this window. Call \helpref{wxWindow::ReleaseMouse}{wxwindowreleasemouse} to
release the capture.

\wxheading{See also}

\helpref{wxWindow::ReleaseMouse}{wxwindowreleasemouse}

\membersection{wxWindow::Center}\label{wxwindowcenter}

\func{void}{Center}{\param{int}{ direction}}

A synonym for \helpref{Centre}{wxwindowcentre}.

\membersection{wxWindow::CenterOnParent}\label{wxwindowcenteronparent}

\func{void}{CenterOnParent}{\param{int}{ direction}}

A synonym for \helpref{CentreOnParent}{wxwindowcentreonparent}.

\membersection{wxWindow::CenterOnScreen}\label{wxwindowcenteronscreen}

\func{void}{CenterOnScreen}{\param{int}{ direction}}

A synonym for \helpref{CentreOnScreen}{wxwindowcentreonscreen}.

\membersection{wxWindow::Centre}\label{wxwindowcentre}

\func{void}{Centre}{\param{int}{ direction = wxBOTH}}

Centres the window.

\wxheading{Parameters}

\docparam{direction}{Specifies the direction for the centering. May be {\tt wxHORIZONTAL}, {\tt wxVERTICAL}\rtfsp
or {\tt wxBOTH}. It may also include {\tt wxCENTRE\_ON\_SCREEN} flag
if you want to center the window on the entire screen and not on its
parent window.}

The flag {\tt wxCENTRE\_FRAME} is obsolete and should not be used any longer
(it has no effect).

\wxheading{Remarks}

If the window is a top level one (i.e. doesn't have a parent), it will be
centered relative to the screen anyhow.

\wxheading{See also}

\helpref{wxWindow::Center}{wxwindowcenter}

\membersection{wxWindow::CentreOnParent}\label{wxwindowcentreonparent}

\func{void}{CentreOnParent}{\param{int}{ direction = wxBOTH}}

Centres the window on its parent. This is a more readable synonym for
\helpref{Centre}{wxwindowcentre}.

\wxheading{Parameters}

\docparam{direction}{Specifies the direction for the centering. May be {\tt wxHORIZONTAL}, {\tt wxVERTICAL}\rtfsp
or {\tt wxBOTH}.}

\wxheading{Remarks}

This methods provides for a way to center top level windows over their
parents instead of the entire screen.  If there is no parent or if the
window is not a top level window, then behaviour is the same as
\helpref{wxWindow::Centre}{wxwindowcentre}.

\wxheading{See also}

\helpref{wxWindow::CentreOnScreen}{wxwindowcenteronscreen}

\membersection{wxWindow::CentreOnScreen}\label{wxwindowcentreonscreen}

\func{void}{CentreOnScreen}{\param{int}{ direction = wxBOTH}}

Centres the window on screen. This only works for top level windows -
otherwise, the window will still be centered on its parent.

\wxheading{Parameters}

\docparam{direction}{Specifies the direction for the centering. May be {\tt wxHORIZONTAL}, {\tt wxVERTICAL}\rtfsp
or {\tt wxBOTH}.}

\wxheading{See also}

\helpref{wxWindow::CentreOnParent}{wxwindowcenteronparent}

\membersection{wxWindow::Clear}\label{wxwindowclear}

\func{void}{Clear}{\void}

Clears the window by filling it with the current background colour. Does not
cause an erase background event to be generated.

\membersection{wxWindow::ClientToScreen}

\constfunc{virtual void}{ClientToScreen}{\param{int* }{x}, \param{int* }{y}}

\perlnote{In wxPerl this method returns a 2-element list intead of
modifying its parameters.}

\constfunc{virtual wxPoint}{ClientToScreen}{\param{const wxPoint\&}{ pt}}

Converts to screen coordinates from coordinates relative to this window.

\docparam{x}{A pointer to a integer value for the x coordinate. Pass the client coordinate in, and
a screen coordinate will be passed out.}

\docparam{y}{A pointer to a integer value for the y coordinate. Pass the client coordinate in, and
a screen coordinate will be passed out.}

\docparam{pt}{The client position for the second form of the function.}

\pythonnote{In place of a single overloaded method name, wxPython
implements the following methods:\par
\indented{2cm}{\begin{twocollist}
\twocolitem{{\bf ClientToScreen(point)}}{Accepts and returns a wxPoint}
\twocolitem{{\bf ClientToScreenXY(x, y)}}{Returns a 2-tuple, (x, y)}
\end{twocollist}}
}

\membersection{wxWindow::Close}\label{wxwindowclose}

\func{virtual bool}{Close}{\param{bool}{ force = FALSE}}

The purpose of this call is to provide a safer way of destroying a window than using
the {\it delete} operator.

\wxheading{Parameters}

\docparam{force}{FALSE if the window's close handler should be able to veto the destruction
of this window, TRUE if it cannot.}

\wxheading{Remarks}

Close calls the \helpref{close handler}{wxcloseevent} for the window, providing an opportunity for the window to
choose whether to destroy the window.

The close handler should check whether the window is being deleted forcibly,
using \helpref{wxCloseEvent::GetForce}{wxcloseeventgetforce}, in which case it should
destroy the window using \helpref{wxWindow::Destroy}{wxwindowdestroy}.

Applies to managed windows (wxFrame and wxDialog classes) only.

{\it Note} that calling Close does not guarantee that the window will be destroyed; but it
provides a way to simulate a manual close of a window, which may or may not be implemented by
destroying the window. The default implementation of wxDialog::OnCloseWindow does not
necessarily delete the dialog, since it will simply simulate an wxID\_CANCEL event which
itself only hides the dialog.

To guarantee that the window will be destroyed, call \helpref{wxWindow::Destroy}{wxwindowdestroy} instead.

\wxheading{See also}

\helpref{Window deletion overview}{windowdeletionoverview},\rtfsp
\helpref{wxWindow::OnCloseWindow}{wxwindowonclosewindow},\rtfsp
\helpref{wxWindow::Destroy}{wxwindowdestroy},\rtfsp
\helpref{wxCloseEvent}{wxcloseevent}

\membersection{wxWindow::ConvertDialogToPixels}\label{wxwindowconvertdialogtopixels}

\func{wxPoint}{ConvertDialogToPixels}{\param{const wxPoint\&}{ pt}}

\func{wxSize}{ConvertDialogToPixels}{\param{const wxSize\&}{ sz}}

Converts a point or size from dialog units to pixels.

For the x dimension, the dialog units are multiplied by the average character width
and then divided by 4.

For the y dimension, the dialog units are multiplied by the average character height
and then divided by 8.

\wxheading{Remarks}

Dialog units are used for maintaining a dialog's proportions even if the font changes.
Dialogs created using Dialog Editor optionally use dialog units.

You can also use these functions programmatically. A convenience macro is defined:

{\small
\begin{verbatim}
#define wxDLG_UNIT(parent, pt) parent->ConvertDialogToPixels(pt)
\end{verbatim}
}

\wxheading{See also}

\helpref{wxWindow::ConvertPixelsToDialog}{wxwindowconvertpixelstodialog}

\pythonnote{In place of a single overloaded method name, wxPython
implements the following methods:\par
\indented{2cm}{\begin{twocollist}
\twocolitem{{\bf ConvertDialogPointToPixels(point)}}{Accepts and returns a wxPoint}
\twocolitem{{\bf ConvertDialogSizeToPixels(size)}}{Accepts and returns a wxSize}
\end{twocollist}}

Additionally, the following helper functions are defined:\par
\indented{2cm}{\begin{twocollist}
\twocolitem{{\bf wxDLG\_PNT(win, point)}}{Converts a wxPoint from dialog
units to pixels}
\twocolitem{{\bf wxDLG\_SZE(win, size)}}{Converts a wxSize from dialog
units to pixels}
\end{twocollist}}
}


\membersection{wxWindow::ConvertPixelsToDialog}\label{wxwindowconvertpixelstodialog}

\func{wxPoint}{ConvertPixelsToDialog}{\param{const wxPoint\&}{ pt}}

\func{wxSize}{ConvertPixelsToDialog}{\param{const wxSize\&}{ sz}}

Converts a point or size from pixels to dialog units.

For the x dimension, the pixels are multiplied by 4 and then divided by the average
character width.

For the y dimension, the pixels are multipled by 8 and then divided by the average
character height.

\wxheading{Remarks}

Dialog units are used for maintaining a dialog's proportions even if the font changes.
Dialogs created using Dialog Editor optionally use dialog units.

\wxheading{See also}

\helpref{wxWindow::ConvertDialogToPixels}{wxwindowconvertdialogtopixels}


\pythonnote{In place of a single overloaded method name, wxPython
implements the following methods:\par
\indented{2cm}{\begin{twocollist}
\twocolitem{{\bf ConvertDialogPointToPixels(point)}}{Accepts and returns a wxPoint}
\twocolitem{{\bf ConvertDialogSizeToPixels(size)}}{Accepts and returns a wxSize}
\end{twocollist}}
}

\membersection{wxWindow::Destroy}\label{wxwindowdestroy}

\func{virtual bool}{Destroy}{\void}

Destroys the window safely. Use this function instead of the delete operator, since
different window classes can be destroyed differently. Frames and dialogs
are not destroyed immediately when this function is called - they are added
to a list of windows to be deleted on idle time, when all the window's events
have been processed. This prevents problems with events being sent to non-existant
windows.

\wxheading{Return value}

TRUE if the window has either been successfully deleted, or it has been added
to the list of windows pending real deletion.

\membersection{wxWindow::DestroyChildren}

\func{virtual void}{DestroyChildren}{\void}

Destroys all children of a window.  Called automatically by the destructor.

\membersection{wxWindow::DragAcceptFiles}\label{wxwindowdragacceptfiles}

\func{virtual void}{DragAcceptFiles}{\param{bool}{ accept}}

Enables or disables elibility for drop file events (OnDropFiles).

\wxheading{Parameters}

\docparam{accept}{If TRUE, the window is eligible for drop file events. If FALSE, the window
will not accept drop file events.}

\wxheading{Remarks}

Windows only.

\wxheading{See also}

\helpref{wxWindow::OnDropFiles}{wxwindowondropfiles}

\membersection{wxWindow::Enable}\label{wxwindowenable}

\func{virtual void}{Enable}{\param{bool}{ enable}}

Enable or disable the window for user input.

\wxheading{Parameters}

\docparam{enable}{If TRUE, enables the window for input. If FALSE, disables the window.}

\wxheading{See also}

\helpref{wxWindow::IsEnabled}{wxwindowisenabled}

\membersection{wxWindow::FindFocus}\label{wxwindowfindfocus}

\func{static wxWindow*}{FindFocus}{\void}

Finds the window or control which currently has the keyboard focus.

\wxheading{Remarks}

Note that this is a static function, so it can be called without needing a wxWindow pointer.

\wxheading{See also}

\helpref{wxWindow::SetFocus}{wxwindowsetfocus}

\membersection{wxWindow::FindWindow}\label{wxwindowfindwindow}

\func{wxWindow*}{FindWindow}{\param{long}{ id}}

Find a child of this window, by identifier.

\func{wxWindow*}{FindWindow}{\param{const wxString\&}{ name}}

Find a child of this window, by name.

\pythonnote{In place of a single overloaded method name, wxPython
implements the following methods:\par
\indented{2cm}{\begin{twocollist}
\twocolitem{{\bf FindWindowById(id)}}{Accepts an integer}
\twocolitem{{\bf FindWindowByName(name)}}{Accepts a string}
\end{twocollist}}
}

\membersection{wxWindow::Fit}\label{wxwindowfit}

\func{virtual void}{Fit}{\void}

Sizes the window so that it fits around its subwindows. This function won't do
anything if there are no subwindows.

\membersection{wxWindow::GetBackgroundColour}\label{wxwindowgetbackgroundcolour}

\constfunc{virtual wxColour}{GetBackgroundColour}{\void}

Returns the background colour of the window.

\wxheading{See also}

\helpref{wxWindow::SetBackgroundColour}{wxwindowsetbackgroundcolour},\rtfsp
\helpref{wxWindow::SetForegroundColour}{wxwindowsetforegroundcolour},\rtfsp
\helpref{wxWindow::GetForegroundColour}{wxwindowgetforegroundcolour},\rtfsp
\helpref{wxWindow::OnEraseBackground}{wxwindowonerasebackground}

\membersection{wxWindow::GetBestSize}\label{wxwindowgetbestsize}

\constfunc{virtual wxSize}{GetBestSize}{\void}

This functions returns the best acceptable minimal size for the window. For
example, for a static control, it will be the minimal size such that the
control label is not truncated. For windows containing subwindows (typically
\helpref{wxPanel}{wxpanel}), the size returned by this function will be the
same as the size the window would have had after calling
\helpref{Fit}{wxwindowfit}.

\membersection{wxWindow::GetCaret}\label{wxwindowgetcaret}

\constfunc{wxCaret *}{GetCaret}{\void}

Returns the \helpref{caret}{wxcaret} associated with the window.

\membersection{wxWindow::GetCharHeight}

\constfunc{virtual int}{GetCharHeight}{\void}

Returns the character height for this window.

\membersection{wxWindow::GetCharWidth}

\constfunc{virtual int}{GetCharWidth}{\void}

Returns the average character width for this window.

\membersection{wxWindow::GetChildren}

\func{wxList\&}{GetChildren}{\void}

Returns a reference to the list of the window's children.

\membersection{wxWindow::GetClientSize}\label{wxwindowgetclientsize}

\constfunc{virtual void}{GetClientSize}{\param{int* }{width}, \param{int* }{height}}

\perlnote{In wxPerl this method takes no parameter and returns
a 2-element list {\tt ( width, height )}.}

\constfunc{virtual wxSize}{GetClientSize}{\void}

This gets the size of the window `client area' in pixels.  The client area is the
area which may be drawn on by the programmer, excluding title bar, border etc.

\wxheading{Parameters}

\docparam{width}{Receives the client width in pixels.}

\docparam{height}{Receives the client height in pixels.}

\pythonnote{In place of a single overloaded method name, wxPython
implements the following methods:\par
\indented{2cm}{\begin{twocollist}
\twocolitem{{\bf GetClientSizeTuple()}}{Returns a 2-tuple of (width, height)}
\twocolitem{{\bf GetClientSize()}}{Returns a wxSize object}
\end{twocollist}}
}

\membersection{wxWindow::GetConstraints}\label{wxwindowgetconstraints}

\constfunc{wxLayoutConstraints*}{GetConstraints}{\void}

Returns a pointer to the window's layout constraints, or NULL if there are none.

\membersection{wxWindow::GetDropTarget}\label{wxwindowgetdroptarget}

\constfunc{wxDropTarget*}{GetDropTarget}{\void}

Returns the associated drop target, which may be NULL.

\wxheading{See also}

\helpref{wxWindow::SetDropTarget}{wxwindowsetdroptarget},
\helpref{Drag and drop overview}{wxdndoverview}

\membersection{wxWindow::GetEventHandler}\label{wxwindowgeteventhandler}

\constfunc{wxEvtHandler*}{GetEventHandler}{\void}

Returns the event handler for this window. By default, the window is its
own event handler.

\wxheading{See also}

\helpref{wxWindow::SetEventHandler}{wxwindowseteventhandler},\rtfsp
\helpref{wxWindow::PushEventHandler}{wxwindowpusheventhandler},\rtfsp
\helpref{wxWindow::PopEventHandler}{wxwindowpusheventhandler},\rtfsp
\helpref{wxEvtHandler::ProcessEvent}{wxevthandlerprocessevent},\rtfsp
\helpref{wxEvtHandler}{wxevthandler}\rtfsp

\membersection{wxWindow::GetExtraStyle}\label{wxwindowgetextrastyle}

\constfunc{long}{GetExtraStyle}{\void}

Returns the extra style bits for the window.

\membersection{wxWindow::GetFont}\label{wxwindowgetfont}

\constfunc{wxFont\&}{GetFont}{\void}

Returns a reference to the font for this window.

\wxheading{See also}

\helpref{wxWindow::SetFont}{wxwindowsetfont}

\membersection{wxWindow::GetForegroundColour}\label{wxwindowgetforegroundcolour}

\func{virtual wxColour}{GetForegroundColour}{\void}

Returns the foreground colour of the window.

\wxheading{Remarks}

The interpretation of foreground colour is open to interpretation according
to the window class; it may be the text colour or other colour, or it may not
be used at all.

\wxheading{See also}

\helpref{wxWindow::SetForegroundColour}{wxwindowsetforegroundcolour},\rtfsp
\helpref{wxWindow::SetBackgroundColour}{wxwindowsetbackgroundcolour},\rtfsp
\helpref{wxWindow::GetBackgroundColour}{wxwindowgetbackgroundcolour}

\membersection{wxWindow::GetGrandParent}

\constfunc{wxWindow*}{GetGrandParent}{\void}

Returns the grandparent of a window, or NULL if there isn't one.

\membersection{wxWindow::GetHandle}\label{wxwindowgethandle}

\constfunc{void*}{GetHandle}{\void}

Returns the platform-specific handle of the physical window. Cast it to an appropriate
handle, such as {\bf HWND} for Windows, {\bf Widget} for Motif or {\bf GtkWidget} for GTK.

\pythonnote{This method will return an integer in wxPython.}

\membersection{wxWindow::GetHelpText}\label{wxwindowgethelptext}

\constfunc{virtual wxString}{GetHelpText}{\void}

Gets the help text to be used as context-sensitive help for this window.

Note that the text is actually stored by the current \helpref{wxHelpProvider}{wxhelpprovider} implementation,
and not in the window object itself.

\wxheading{See also}

\helpref{SetHelpText}{wxwindowsethelptext}, \helpref{wxHelpProvider}{wxhelpprovider}

\membersection{wxWindow::GetId}\label{wxwindowgetid}

\constfunc{int}{GetId}{\void}

Returns the identifier of the window.

\wxheading{Remarks}

Each window has an integer identifier. If the application has not provided one
(or the default Id -1) an unique identifier with a negative value will be generated.

\wxheading{See also}

\helpref{wxWindow::SetId}{wxwindowsetid},\rtfsp
\helpref{Window identifiers}{windowids}

\membersection{wxWindow::GetLabel}

\constfunc{virtual wxString }{GetLabel}{\void}

Generic way of getting a label from any window, for
identification purposes.

\wxheading{Remarks}

The interpretation of this function differs from class to class.
For frames and dialogs, the value returned is the title. For buttons or static text controls, it is
the button text. This function can be useful for meta-programs (such as testing
tools or special-needs access programs) which need to identify windows
by name.

\membersection{wxWindow::GetName}\label{wxwindowgetname}

\constfunc{virtual wxString }{GetName}{\void}

Returns the window's name.

\wxheading{Remarks}

This name is not guaranteed to be unique; it is up to the programmer to supply an appropriate
name in the window constructor or via \helpref{wxWindow::SetName}{wxwindowsetname}.

\wxheading{See also}

\helpref{wxWindow::SetName}{wxwindowsetname}

\membersection{wxWindow::GetParent}

\constfunc{virtual wxWindow*}{GetParent}{\void}

Returns the parent of the window, or NULL if there is no parent.

\membersection{wxWindow::GetPosition}\label{wxwindowgetposition}

\constfunc{virtual void}{GetPosition}{\param{int* }{x}, \param{int* }{y}}

\constfunc{wxPoint}{GetPosition}{\void}

This gets the position of the window in pixels, relative to the parent window or
if no parent, relative to the whole display.

\wxheading{Parameters}

\docparam{x}{Receives the x position of the window.}

\docparam{y}{Receives the y position of the window.}

\pythonnote{In place of a single overloaded method name, wxPython
implements the following methods:\par
\indented{2cm}{\begin{twocollist}
\twocolitem{{\bf GetPosition()}}{Returns a wxPoint}
\twocolitem{{\bf GetPositionTuple()}}{Returns a tuple (x, y)}
\end{twocollist}}
}

\perlnote{In wxPerl there are two methods instead of a single overloaded
method:\par
\indented{2cm}{\begin{twocollist}
\twocolitem{{\bf GetPosition()}}{Returns a Wx::Point}
\twocolitem{{\bf GetPositionXY()}}{Returns a 2-element list
 {\tt ( x, y )}}
\end{twocollist}
}}

\membersection{wxWindow::GetRect}\label{wxwindowgetrect}

\constfunc{virtual wxRect}{GetRect}{\void}

Returns the size and position of the window as a \helpref{wxRect}{wxrect} object.

\membersection{wxWindow::GetScrollThumb}\label{wxwindowgetscrollthumb}

\func{virtual int}{GetScrollThumb}{\param{int }{orientation}}

Returns the built-in scrollbar thumb size.

\wxheading{See also}

\helpref{wxWindow::SetScrollbar}{wxwindowsetscrollbar}

\membersection{wxWindow::GetScrollPos}\label{wxwindowgetscrollpos}

\func{virtual int}{GetScrollPos}{\param{int }{orientation}}

Returns the built-in scrollbar position.

\wxheading{See also}

See \helpref{wxWindow::SetScrollbar}{wxwindowsetscrollbar}

\membersection{wxWindow::GetScrollRange}\label{wxwindowgetscrollrange}

\func{virtual int}{GetScrollRange}{\param{int }{orientation}}

Returns the built-in scrollbar range.

\wxheading{See also}

\helpref{wxWindow::SetScrollbar}{wxwindowsetscrollbar}

\membersection{wxWindow::GetSize}\label{wxwindowgetsize}

\constfunc{virtual void}{GetSize}{\param{int* }{width}, \param{int* }{height}}

\constfunc{virtual wxSize}{GetSize}{\void}

This gets the size of the entire window in pixels.

\wxheading{Parameters}

\docparam{width}{Receives the window width.}

\docparam{height}{Receives the window height.}

\pythonnote{In place of a single overloaded method name, wxPython
implements the following methods:\par
\indented{2cm}{\begin{twocollist}
\twocolitem{{\bf GetSize()}}{Returns a wxSize}
\twocolitem{{\bf GetSizeTuple()}}{Returns a 2-tuple (width, height)}
\end{twocollist}}
}

\perlnote{In wxPerl there are two methods instead of a single overloaded
method:\par
\indented{2cm}{\begin{twocollist}
\twocolitem{{\bf GetSize()}}{Returns a Wx::Size}
\twocolitem{{\bf GetSizeWH()}}{Returns a 2-element list
 {\tt ( width, height )}}
\end{twocollist}
}}

\membersection{wxWindow::GetSizer}\label{wxwindowgetsizer}

\constfunc{const wxSizer *}{GetSizer}{\void}

Return the sizer associated with the window by a previous call to 
\helpref{SetSizer()}{wxwindowsetsizer} or {\tt NULL}.

\constfunc{virtual void}{GetTextExtent}{\param{const wxString\& }{string}, \param{int* }{x}, \param{int* }{y},
 \param{int* }{descent = NULL}, \param{int* }{externalLeading = NULL},
 \param{const wxFont* }{font = NULL}, \param{bool}{ use16 = FALSE}}

Gets the dimensions of the string as it would be drawn on the
window with the currently selected font.

\wxheading{Parameters}

\docparam{string}{String whose extent is to be measured.}

\docparam{x}{Return value for width.}

\docparam{y}{Return value for height.}

\docparam{descent}{Return value for descent (optional).}

\docparam{externalLeading}{Return value for external leading (optional).}

\docparam{font}{Font to use instead of the current window font (optional).}

\docparam{use16}{If TRUE, {\it string} contains 16-bit characters. The default is FALSE.}


\pythonnote{In place of a single overloaded method name, wxPython
implements the following methods:\par
\indented{2cm}{\begin{twocollist}
\twocolitem{{\bf GetTextExtent(string)}}{Returns a 2-tuple,  (width, height)}
\twocolitem{{\bf GetFullTextExtent(string, font=NULL)}}{Returns a
4-tuple, (width, height, descent, externalLeading) }
\end{twocollist}}
}

\perlnote{In wxPerl this method takes only the {\tt string} and optionally
 {\tt font} parameters, and returns a 4-element list
 {\tt ( x, y, descent, externalLeading )}.}

\membersection{wxWindow::GetTitle}\label{wxwindowgettitle}

\func{virtual wxString}{GetTitle}{\void}

Gets the window's title. Applicable only to frames and dialogs.

\wxheading{See also}

\helpref{wxWindow::SetTitle}{wxwindowsettitle}

\membersection{wxWindow::GetUpdateRegion}\label{wxwindowgetupdateregion}

\constfunc{virtual wxRegion}{GetUpdateRegion}{\void}

Returns the region specifying which parts of the window have been damaged. Should
only be called within an \helpref{OnPaint}{wxwindowonpaint} event handler.

\wxheading{See also}

\helpref{wxRegion}{wxregion}, \helpref{wxRegionIterator}{wxregioniterator}, \helpref{wxWindow::OnPaint}{wxwindowonpaint}

\membersection{wxWindow::GetValidator}\label{wxwindowgetvalidator}

\constfunc{wxValidator*}{GetValidator}{\void}

Returns a pointer to the current validator for the window, or NULL if there is none.

\membersection{wxWindow::GetWindowStyleFlag}\label{wxwindowgetwindowstyleflag}

\constfunc{long}{GetWindowStyleFlag}{\void}

Gets the window style that was passed to the constructor or {\bf Create}
method. {\bf GetWindowStyle()} is another name for the same function.

\membersection{wxWindow::InitDialog}\label{wxwindowinitdialog}

\func{void}{InitDialog}{\void}

Sends an \helpref{wxWindow::OnInitDialog}{wxwindowoninitdialog} event, which
in turn transfers data to the dialog via validators.

\wxheading{See also}

\helpref{wxWindow::OnInitDialog}{wxwindowoninitdialog}

\membersection{wxWindow::IsEnabled}\label{wxwindowisenabled}

\constfunc{virtual bool}{IsEnabled}{\void}

Returns TRUE if the window is enabled for input, FALSE otherwise.

\wxheading{See also}

\helpref{wxWindow::Enable}{wxwindowenable}

\membersection{wxWindow:IsExposed}\label{wxwindowisexposed}

\constfunc{bool}{IsExposed}{\param{int }{x}, \param{int }{y}}

\constfunc{bool}{IsExposed}{\param{wxPoint }{\&pt}}

\constfunc{bool}{IsExposed}{\param{int }{x}, \param{int }{y}, \param{int }{w}, \param{int }{h}}

\constfunc{bool}{IsExposed}{\param{wxRect }{\&rect}}

Returns TRUE if the given point or rectange area has been exposed since the
last repaint. Call this in an paint event handler to optimize redrawing by
only redrawing those areas, which have been exposed.

\pythonnote{In place of a single overloaded method name, wxPython
implements the following methods:\par
\indented{2cm}{\begin{twocollist}
\twocolitem{{\bf IsExposed(x,y, w=0,h=0}}{}
\twocolitem{{\bf IsExposedPoint(pt)}}{}
\twocolitem{{\bf IsExposedRect(rect)}}{}
\end{twocollist}}}

\membersection{wxWindow::IsRetained}\label{wxwindowisretained}

\constfunc{virtual bool}{IsRetained}{\void}

Returns TRUE if the window is retained, FALSE otherwise.

\wxheading{Remarks}

Retained windows are only available on X platforms.

\membersection{wxWindow::IsShown}\label{wxwindowisshown}

\constfunc{virtual bool}{IsShown}{\void}

Returns TRUE if the window is shown, FALSE if it has been hidden.

\membersection{wxWindow::IsTopLevel}\label{wxwindowistoplevel}

\constfunc{bool}{IsTopLevel}{\void}

Returns TRUE if the given window is a top-level one. Currently all frames and
dialogs are considered to be top-level windows (even if they have a parent
window).

\membersection{wxWindow::Layout}\label{wxwindowlayout}

\func{void}{Layout}{\void}

Invokes the constraint-based layout algorithm or the sizer-based algorithm
for this window.

See \helpref{wxWindow::SetAutoLayout}{wxwindowsetautolayout} on when
this function gets called automatically using auto layout.

\membersection{wxWindow::LoadFromResource}\label{wxwindowloadfromresource}

\func{virtual bool}{LoadFromResource}{\param{wxWindow* }{parent},\rtfsp
\param{const wxString\& }{resourceName}, \param{const wxResourceTable* }{resourceTable = NULL}}

Loads a panel or dialog from a resource file.

\wxheading{Parameters}

\docparam{parent}{Parent window.}

\docparam{resourceName}{The name of the resource to load.}

\docparam{resourceTable}{The resource table to load it from. If this is NULL, the
default resource table will be used.}

\wxheading{Return value}

TRUE if the operation succeeded, otherwise FALSE.

\membersection{wxWindow::Lower}\label{wxwindowlower}

\func{void}{Lower}{\void}

Lowers the window to the bottom of the window hierarchy if it is a managed window (dialog
or frame).

\membersection{wxWindow::MakeModal}\label{wxwindowmakemodal}

\func{virtual void}{MakeModal}{\param{bool }{flag}}

Disables all other windows in the application so that
the user can only interact with this window. (This function
is not implemented anywhere).

\wxheading{Parameters}

\docparam{flag}{If TRUE, this call disables all other windows in the application so that
the user can only interact with this window. If FALSE, the effect is reversed.}

\membersection{wxWindow::Move}\label{wxwindowmove}

\func{void}{Move}{\param{int}{ x}, \param{int}{ y}}

\func{void}{Move}{\param{const wxPoint\&}{ pt}}

Moves the window to the given position.

\wxheading{Parameters}

\docparam{x}{Required x position.}

\docparam{y}{Required y position.}

\docparam{pt}{\helpref{wxPoint}{wxpoint} object representing the position.}

\wxheading{Remarks}

Implementations of SetSize can also implicitly implement the
wxWindow::Move function, which is defined in the base wxWindow class
as the call:

\begin{verbatim}
  SetSize(x, y, -1, -1, wxSIZE_USE_EXISTING);
\end{verbatim}

\wxheading{See also}

\helpref{wxWindow::SetSize}{wxwindowsetsize}

\pythonnote{In place of a single overloaded method name, wxPython
implements the following methods:\par
\indented{2cm}{\begin{twocollist}
\twocolitem{{\bf Move(point)}}{Accepts a wxPoint}
\twocolitem{{\bf MoveXY(x, y)}}{Accepts a pair of integers}
\end{twocollist}}
}

\membersection{wxWindow::OnActivate}\label{wxwindowonactivate}

\func{void}{OnActivate}{\param{wxActivateEvent\&}{ event}}

Called when a window is activated or deactivated.

\wxheading{Parameters}

\docparam{event}{Object containing activation information.}

\wxheading{Remarks}

If the window is being activated, \helpref{wxActivateEvent::GetActive}{wxactivateeventgetactive} returns TRUE,
otherwise it returns FALSE (it is being deactivated).

\wxheading{See also}

\helpref{wxActivateEvent}{wxactivateevent},\rtfsp
\helpref{Event handling overview}{eventhandlingoverview}

\membersection{wxWindow::OnChar}\label{wxwindowonchar}

\func{void}{OnChar}{\param{wxKeyEvent\&}{ event}}

Called when the user has pressed a key that is not a modifier (SHIFT, CONTROL or ALT).

\wxheading{Parameters}

\docparam{event}{Object containing keypress information. See \helpref{wxKeyEvent}{wxkeyevent} for
details about this class.}

\wxheading{Remarks}

This member function is called in response to a keypress. To intercept this event,
use the EVT\_CHAR macro in an event table definition. Your {\bf OnChar} handler may call this
default function to achieve default keypress functionality.

Note that the ASCII values do not have explicit key codes: they are passed as ASCII
values.

Note that not all keypresses can be intercepted this way. If you wish to intercept modifier
keypresses, then you will need to use \helpref{wxWindow::OnKeyDown}{wxwindowonkeydown} or
\helpref{wxWindow::OnKeyUp}{wxwindowonkeyup}.

Most, but not all, windows allow keypresses to be intercepted.

{\bf Tip:} be sure to call {\tt event.Skip()} for events that you don't process in this function,
otherwise menu shortcuts may cease to work under Windows.

\wxheading{See also}

\helpref{wxWindow::OnKeyDown}{wxwindowonkeydown}, \helpref{wxWindow::OnKeyUp}{wxwindowonkeyup},\rtfsp
\helpref{wxKeyEvent}{wxkeyevent}, \helpref{wxWindow::OnCharHook}{wxwindowoncharhook},\rtfsp
\helpref{Event handling overview}{eventhandlingoverview}

\membersection{wxWindow::OnCharHook}\label{wxwindowoncharhook}

\func{void}{OnCharHook}{\param{wxKeyEvent\&}{ event}}

This member is called to allow the window to intercept keyboard events
before they are processed by child windows.

\wxheading{Parameters}

\docparam{event}{Object containing keypress information. See \helpref{wxKeyEvent}{wxkeyevent} for
details about this class.}

\wxheading{Remarks}

This member function is called in response to a keypress, if the window is active. To intercept this event,
use the EVT\_CHAR\_HOOK macro in an event table definition. If you do not process a particular
keypress, call \helpref{wxEvent::Skip}{wxeventskip} to allow default processing.

An example of using this function is in the implementation of escape-character processing for wxDialog,
where pressing ESC dismisses the dialog by {\bf OnCharHook} 'forging' a cancel button press event.

Note that the ASCII values do not have explicit key codes: they are passed as ASCII
values.

This function is only relevant to top-level windows (frames and dialogs), and under
Windows only. Under GTK the normal EVT\_CHAR\_ event has the functionality, i.e.
you can intercepts it and if you don't call \helpref{wxEvent::Skip}{wxeventskip}
the window won't get the event.

\wxheading{See also}

\helpref{wxKeyEvent}{wxkeyevent}, \helpref{wxWindow::OnCharHook}{wxwindowoncharhook},\rtfsp
\helpref{wxApp::OnCharHook}{wxapponcharhook},\rtfsp
\helpref{Event handling overview}{eventhandlingoverview}

\membersection{wxWindow::OnCommand}\label{wxwindowoncommand}

\func{virtual void}{OnCommand}{\param{wxEvtHandler\& }{object}, \param{wxCommandEvent\& }{event}}

This virtual member function is called if the control does not handle the command event.

\wxheading{Parameters}

\docparam{object}{Object receiving the command event.}

\docparam{event}{Command event}

\wxheading{Remarks}

This virtual function is provided mainly for backward compatibility. You can also intercept commands
from child controls by using an event table, with identifiers or identifier ranges to identify
the control(s) in question.

\wxheading{See also}

\helpref{wxCommandEvent}{wxcommandevent},\rtfsp
\helpref{Event handling overview}{eventhandlingoverview}

\membersection{wxWindow::OnClose}\label{wxwindowonclose}

\func{virtual bool}{OnClose}{\void}

Called when the user has tried to close a a frame
or dialog box using the window manager (X) or system menu (Windows).

{\bf Note:} This is an obsolete function.
It is superceded by the \helpref{wxWindow::OnCloseWindow}{wxwindowonclosewindow} event
handler.

\wxheading{Return value}

If TRUE is returned by OnClose, the window will be deleted by the system, otherwise the
attempt will be ignored. Do not delete the window from within this handler, although
you may delete other windows.

\wxheading{See also}

\helpref{Window deletion overview}{windowdeletionoverview},\rtfsp
\helpref{wxWindow::Close}{wxwindowclose},\rtfsp
\helpref{wxWindow::OnCloseWindow}{wxwindowonclosewindow},\rtfsp
\helpref{wxCloseEvent}{wxcloseevent}

\membersection{wxWindow::OnCloseWindow}\label{wxwindowonclosewindow}

\func{void}{OnCloseWindow}{\param{wxCloseEvent\& }{event}}

This is an event handler function called when the user has tried to close a a frame
or dialog box using the window manager (X) or system menu (Windows). It is
called via the \helpref{wxWindow::Close}{wxwindowclose} function, so
that the application can also invoke the handler programmatically.

Use the EVT\_CLOSE event table macro to handle close events.

You should check whether the application is forcing the deletion of the window
using \helpref{wxCloseEvent::GetForce}{wxcloseeventgetforce}. If this is TRUE,
destroy the window using \helpref{wxWindow::Destroy}{wxwindowdestroy}.
If not, it is up to you whether you respond by destroying the window.

(Note: GetForce is now superceded by CanVeto. So to test whether forced destruction of
the window is required, test for the negative of CanVeto. If CanVeto returns FALSE,
it is not possible to skip window deletion.)

If you don't destroy the window, you should call \helpref{wxCloseEvent::Veto}{wxcloseeventveto} to
let the calling code know that you did not destroy the window. This allows the \helpref{wxWindow::Close}{wxwindowclose} function
to return TRUE or FALSE depending on whether the close instruction was honoured or not.

\wxheading{Remarks}

The \helpref{wxWindow::OnClose}{wxwindowonclose} virtual function remains
for backward compatibility with earlier versions of wxWindows. The
default {\bf OnCloseWindow} handler for wxFrame and wxDialog will call {\bf OnClose},
destroying the window if it returns TRUE or if the close is being forced.

\wxheading{See also}

\helpref{Window deletion overview}{windowdeletionoverview},\rtfsp
\helpref{wxWindow::Close}{wxwindowclose},\rtfsp
\helpref{wxWindow::OnClose}{wxwindowonclose},\rtfsp
\helpref{wxWindow::Destroy}{wxwindowdestroy},\rtfsp
\helpref{wxCloseEvent}{wxcloseevent},\rtfsp
\helpref{wxApp::OnQueryEndSession}{wxapponqueryendsession},\rtfsp
\helpref{wxApp::OnEndSession}{wxapponendsession}

\membersection{wxWindow::OnDropFiles}\label{wxwindowondropfiles}

\func{void}{OnDropFiles}{\param{wxDropFilesEvent\&}{ event}}

Called when files have been dragged from the file manager to the window.

\wxheading{Parameters}

\docparam{event}{Drop files event. For more information, see \helpref{wxDropFilesEvent}{wxdropfilesevent}.}

\wxheading{Remarks}

The window must have previously been enabled for dropping by calling
\rtfsp\helpref{wxWindow::DragAcceptFiles}{wxwindowdragacceptfiles}.

This event is only generated under Windows.

To intercept this event, use the EVT\_DROP\_FILES macro in an event table definition.

\wxheading{See also}

\helpref{wxDropFilesEvent}{wxdropfilesevent}, \helpref{wxWindow::DragAcceptFiles}{wxwindowdragacceptfiles},\rtfsp
\helpref{Event handling overview}{eventhandlingoverview}

\membersection{wxWindow::OnEraseBackground}\label{wxwindowonerasebackground}

\func{void}{OnEraseBackground}{\param{wxEraseEvent\&}{ event}}

Called when the background of the window needs to be erased.

\wxheading{Parameters}

\docparam{event}{Erase background event. For more information, see \helpref{wxEraseEvent}{wxeraseevent}.}

\wxheading{Remarks}

Under non-Windows platforms, this event is simulated (simply generated just before the
paint event) and may cause flicker. It is therefore recommended that
you set the text background colour explicitly in order to prevent flicker.
The default background colour under GTK is grey.

To intercept this event, use the EVT\_ERASE\_BACKGROUND macro in an event table definition.

\wxheading{See also}

\helpref{wxEraseEvent}{wxeraseevent}, \helpref{Event handling overview}{eventhandlingoverview}

\membersection{wxWindow::OnKeyDown}\label{wxwindowonkeydown}

\func{void}{OnKeyDown}{\param{wxKeyEvent\&}{ event}}

Called when the user has pressed a key, before it is translated into an ASCII value using other
modifier keys that might be pressed at the same time.

\wxheading{Parameters}

\docparam{event}{Object containing keypress information. See \helpref{wxKeyEvent}{wxkeyevent} for
details about this class.}

\wxheading{Remarks}

This member function is called in response to a key down event. To intercept this event,
use the EVT\_KEY\_DOWN macro in an event table definition. Your {\bf OnKeyDown} handler may call this
default function to achieve default keypress functionality.

Note that not all keypresses can be intercepted this way. If you wish to intercept special
keys, such as shift, control, and function keys, then you will need to use \helpref{wxWindow::OnKeyDown}{wxwindowonkeydown} or
\helpref{wxWindow::OnKeyUp}{wxwindowonkeyup}.

Most, but not all, windows allow keypresses to be intercepted.

{\bf Tip:} be sure to call {\tt event.Skip()} for events that you don't process in this function,
otherwise menu shortcuts may cease to work under Windows.

\wxheading{See also}

\helpref{wxWindow::OnChar}{wxwindowonchar}, \helpref{wxWindow::OnKeyUp}{wxwindowonkeyup},\rtfsp
\helpref{wxKeyEvent}{wxkeyevent}, \helpref{wxWindow::OnCharHook}{wxwindowoncharhook},\rtfsp
\helpref{Event handling overview}{eventhandlingoverview}

\membersection{wxWindow::OnKeyUp}\label{wxwindowonkeyup}

\func{void}{OnKeyUp}{\param{wxKeyEvent\&}{ event}}

Called when the user has released a key.

\wxheading{Parameters}

\docparam{event}{Object containing keypress information. See \helpref{wxKeyEvent}{wxkeyevent} for
details about this class.}

\wxheading{Remarks}

This member function is called in response to a key up event. To intercept this event,
use the EVT\_KEY\_UP macro in an event table definition. Your {\bf OnKeyUp} handler may call this
default function to achieve default keypress functionality.

Note that not all keypresses can be intercepted this way. If you wish to intercept special
keys, such as shift, control, and function keys, then you will need to use \helpref{wxWindow::OnKeyDown}{wxwindowonkeydown} or
\helpref{wxWindow::OnKeyUp}{wxwindowonkeyup}.

Most, but not all, windows allow key up events to be intercepted.

\wxheading{See also}

\helpref{wxWindow::OnChar}{wxwindowonchar}, \helpref{wxWindow::OnKeyDown}{wxwindowonkeydown},\rtfsp
\helpref{wxKeyEvent}{wxkeyevent}, \helpref{wxWindow::OnCharHook}{wxwindowoncharhook},\rtfsp
\helpref{Event handling overview}{eventhandlingoverview}

\membersection{wxWindow::OnKillFocus}\label{wxwindowonkillfocus}

\func{void}{OnKillFocus}{\param{wxFocusEvent\& }{event}}

Called when a window's focus is being killed.

\wxheading{Parameters}

\docparam{event}{The focus event. For more information, see \helpref{wxFocusEvent}{wxfocusevent}.}

\wxheading{Remarks}

To intercept this event, use the macro EVT\_KILL\_FOCUS in an event table definition.

Most, but not all, windows respond to this event.

\wxheading{See also}

\helpref{wxFocusEvent}{wxfocusevent}, \helpref{wxWindow::OnSetFocus}{wxwindowonsetfocus},\rtfsp
\helpref{Event handling overview}{eventhandlingoverview}

\membersection{wxWindow::OnIdle}\label{wxwindowonidle}

\func{void}{OnIdle}{\param{wxIdleEvent\& }{event}}

Provide this member function for any processing which needs to be done
when the application is idle.

\wxheading{See also}

\helpref{wxApp::OnIdle}{wxapponidle}, \helpref{wxIdleEvent}{wxidleevent}

\membersection{wxWindow::OnInitDialog}\label{wxwindowoninitdialog}

\func{void}{OnInitDialog}{\param{wxInitDialogEvent\&}{ event}}

Default handler for the wxEVT\_INIT\_DIALOG event. Calls \helpref{wxWindow::TransferDataToWindow}{wxwindowtransferdatatowindow}.

\wxheading{Parameters}

\docparam{event}{Dialog initialisation event.}

\wxheading{Remarks}

Gives the window the default behaviour of transferring data to child controls via
the validator that each control has.

\wxheading{See also}

\helpref{wxValidator}{wxvalidator}, \helpref{wxWindow::TransferDataToWindow}{wxwindowtransferdatatowindow}

\membersection{wxWindow::OnMenuCommand}\label{wxwindowonmenucommand}

\func{void}{OnMenuCommand}{\param{wxCommandEvent\& }{event}}

Called when a menu command is received from a menu bar.

\wxheading{Parameters}

\docparam{event}{The menu command event. For more information, see \helpref{wxCommandEvent}{wxcommandevent}.}

\wxheading{Remarks}

A function with this name doesn't actually exist; you can choose any member function to receive
menu command events, using the EVT\_COMMAND macro for individual commands or EVT\_COMMAND\_RANGE for
a range of commands.

\wxheading{See also}

\helpref{wxCommandEvent}{wxcommandevent},\rtfsp
\helpref{wxWindow::OnMenuHighlight}{wxwindowonmenuhighlight},\rtfsp
\helpref{Event handling overview}{eventhandlingoverview}

\membersection{wxWindow::OnMenuHighlight}\label{wxwindowonmenuhighlight}

\func{void}{OnMenuHighlight}{\param{wxMenuEvent\& }{event}}

Called when a menu select is received from a menu bar: that is, the
mouse cursor is over a menu item, but the left mouse button has not been
pressed.

\wxheading{Parameters}

\docparam{event}{The menu highlight event. For more information, see \helpref{wxMenuEvent}{wxmenuevent}.}

\wxheading{Remarks}

You can choose any member function to receive
menu select events, using the EVT\_MENU\_HIGHLIGHT macro for individual menu items or EVT\_MENU\_HIGHLIGHT\_ALL macro
for all menu items.

The default implementation for \helpref{wxFrame::OnMenuHighlight}{wxframeonmenuhighlight} displays help
text in the first field of the status bar.

This function was known as {\bf OnMenuSelect} in earlier versions of wxWindows, but this was confusing
since a selection is normally a left-click action.

\wxheading{See also}

\helpref{wxMenuEvent}{wxmenuevent},\rtfsp
\helpref{wxWindow::OnMenuCommand}{wxwindowonmenucommand},\rtfsp
\helpref{Event handling overview}{eventhandlingoverview}


\membersection{wxWindow::OnMouseEvent}\label{wxwindowonmouseevent}

\func{void}{OnMouseEvent}{\param{wxMouseEvent\&}{ event}}

Called when the user has initiated an event with the
mouse.

\wxheading{Parameters}

\docparam{event}{The mouse event. See \helpref{wxMouseEvent}{wxmouseevent} for
more details.}

\wxheading{Remarks}

Most, but not all, windows respond to this event.

To intercept this event, use the EVT\_MOUSE\_EVENTS macro in an event table definition, or individual
mouse event macros such as EVT\_LEFT\_DOWN.

\wxheading{See also}

\helpref{wxMouseEvent}{wxmouseevent},\rtfsp
\helpref{Event handling overview}{eventhandlingoverview}

\membersection{wxWindow::OnMove}\label{wxwindowonmove}

\func{void}{OnMove}{\param{wxMoveEvent\& }{event}}

Called when a window is moved.

\wxheading{Parameters}

\docparam{event}{The move event. For more information, see \helpref{wxMoveEvent}{wxmoveevent}.}

\wxheading{Remarks}

Use the EVT\_MOVE macro to intercept move events.

\wxheading{Remarks}

Not currently implemented.

\wxheading{See also}

\helpref{wxMoveEvent}{wxmoveevent},\rtfsp
\helpref{wxFrame::OnSize}{wxframeonsize},\rtfsp
\helpref{Event handling overview}{eventhandlingoverview}

\membersection{wxWindow::OnPaint}\label{wxwindowonpaint}

\func{void}{OnPaint}{\param{wxPaintEvent\& }{event}}

Sent to the event handler when the window must be refreshed.

\wxheading{Parameters}

\docparam{event}{Paint event. For more information, see \helpref{wxPaintEvent}{wxpaintevent}.}

\wxheading{Remarks}

Use the EVT\_PAINT macro in an event table definition to intercept paint events.

Note that In a paint event handler, the application must {\it always} create a \helpref{wxPaintDC}{wxpaintdc} object,
even if you do not use it. Otherwise, under MS Windows, refreshing for this and other windows will go wrong.

For example:

\small{%
\begin{verbatim}
  void MyWindow::OnPaint(wxPaintEvent\& event)
  {
      wxPaintDC dc(this);

      DrawMyDocument(dc);
  }
\end{verbatim}
}%

You can optimize painting by retrieving the rectangles
that have been damaged and only repainting these. The rectangles are in
terms of the client area, and are unscrolled, so you will need to do
some calculations using the current view position to obtain logical,
scrolled units.

Here is an example of using the \helpref{wxRegionIterator}{wxregioniterator} class:

{\small%
\begin{verbatim}
// Called when window needs to be repainted.
void MyWindow::OnPaint(wxPaintEvent\& event)
{
  wxPaintDC dc(this);

  // Find Out where the window is scrolled to
  int vbX,vbY;                     // Top left corner of client
  GetViewStart(&vbX,&vbY);

  int vX,vY,vW,vH;                 // Dimensions of client area in pixels
  wxRegionIterator upd(GetUpdateRegion()); // get the update rect list

  while (upd)
  {
    vX = upd.GetX();
    vY = upd.GetY();
    vW = upd.GetW();
    vH = upd.GetH();

    // Alternatively we can do this:
    // wxRect rect;
    // upd.GetRect(&rect);

    // Repaint this rectangle
    ...some code...

    upd ++ ;
  }
}
\end{verbatim}
}%

\wxheading{See also}

\helpref{wxPaintEvent}{wxpaintevent},\rtfsp
\helpref{wxPaintDC}{wxpaintdc},\rtfsp
\helpref{Event handling overview}{eventhandlingoverview}

\membersection{wxWindow::OnScroll}\label{wxwindowonscroll}

\func{void}{OnScroll}{\param{wxScrollWinEvent\& }{event}}

Called when a scroll window event is received from one of the window's built-in scrollbars.

\wxheading{Parameters}

\docparam{event}{Command event. Retrieve the new scroll position by
calling \helpref{wxScrollEvent::GetPosition}{wxscrolleventgetposition}, and the
scrollbar orientation by calling \helpref{wxScrollEvent::GetOrientation}{wxscrolleventgetorientation}.}

\wxheading{Remarks}

Note that it is not possible to distinguish between horizontal and vertical scrollbars
until the function is executing (you can't have one function for vertical, another
for horizontal events).

\wxheading{See also}

\helpref{wxScrollWinEvent}{wxscrollwinevent},\rtfsp
\helpref{Event handling overview}{eventhandlingoverview}

\membersection{wxWindow::OnSetFocus}\label{wxwindowonsetfocus}

\func{void}{OnSetFocus}{\param{wxFocusEvent\& }{event}}

Called when a window's focus is being set.

\wxheading{Parameters}

\docparam{event}{The focus event. For more information, see \helpref{wxFocusEvent}{wxfocusevent}.}

\wxheading{Remarks}

To intercept this event, use the macro EVT\_SET\_FOCUS in an event table definition.

Most, but not all, windows respond to this event.

\wxheading{See also}

\helpref{wxFocusEvent}{wxfocusevent}, \helpref{wxWindow::OnKillFocus}{wxwindowonkillfocus},\rtfsp
\helpref{Event handling overview}{eventhandlingoverview}

\membersection{wxWindow::OnSize}\label{wxwindowonsize}

\func{void}{OnSize}{\param{wxSizeEvent\& }{event}}

Called when the window has been resized.

\wxheading{Parameters}

\docparam{event}{Size event. For more information, see \helpref{wxSizeEvent}{wxsizeevent}.}

\wxheading{Remarks}

You may wish to use this for frames to resize their child windows as appropriate.

Note that the size passed is of
the whole window: call \helpref{wxWindow::GetClientSize}{wxwindowgetclientsize} for the area which may be
used by the application.

When a window is resized, usually only a small part of the window is damaged and you
may only need to repaint that area. However, if your drawing depends on the size of the window,
you may need to clear the DC explicitly and repaint the whole window. In which case, you
may need to call \helpref{wxWindow::Refresh}{wxwindowrefresh} to invalidate the entire window.

\wxheading{See also}

\helpref{wxSizeEvent}{wxsizeevent},\rtfsp
\helpref{Event handling overview}{eventhandlingoverview}

\membersection{wxWindow::OnSysColourChanged}\label{wxwindowonsyscolourchanged}

\func{void}{OnSysColourChanged}{\param{wxOnSysColourChangedEvent\& }{event}}

Called when the user has changed the system colours. Windows only.

\wxheading{Parameters}

\docparam{event}{System colour change event. For more information, see \helpref{wxSysColourChangedEvent}{wxsyscolourchangedevent}.}

\wxheading{See also}

\helpref{wxSysColourChangedEvent}{wxsyscolourchangedevent},\rtfsp
\helpref{Event handling overview}{eventhandlingoverview}

\membersection{wxWindow::PopEventHandler}\label{wxwindowpopeventhandler}

\constfunc{wxEvtHandler*}{PopEventHandler}{\param{bool }{deleteHandler = FALSE}}

Removes and returns the top-most event handler on the event handler stack.

\wxheading{Parameters}

\docparam{deleteHandler}{If this is TRUE, the handler will be deleted after it is removed. The
default value is FALSE.}

\wxheading{See also}

\helpref{wxWindow::SetEventHandler}{wxwindowseteventhandler},\rtfsp
\helpref{wxWindow::GetEventHandler}{wxwindowgeteventhandler},\rtfsp
\helpref{wxWindow::PushEventHandler}{wxwindowpusheventhandler},\rtfsp
\helpref{wxEvtHandler::ProcessEvent}{wxevthandlerprocessevent},\rtfsp
\helpref{wxEvtHandler}{wxevthandler}\rtfsp

\membersection{wxWindow::PopupMenu}\label{wxwindowpopupmenu}

\func{bool}{PopupMenu}{\param{wxMenu* }{menu}, \param{const wxPoint\& }{pos}}

\func{bool}{PopupMenu}{\param{wxMenu* }{menu}, \param{int }{x}, \param{int }{y}}

Pops up the given menu at the specified coordinates, relative to this
window, and returns control when the user has dismissed the menu. If a
menu item is selected, the corresponding menu event is generated and will be
processed as usually.

\wxheading{Parameters}

\docparam{menu}{Menu to pop up.}

\docparam{pos}{The position where the menu will appear.}

\docparam{x}{Required x position for the menu to appear.}

\docparam{y}{Required y position for the menu to appear.}

\wxheading{See also}

\helpref{wxMenu}{wxmenu}

\wxheading{Remarks}

Just before the menu is popped up, \helpref{wxMenu::UpdateUI}{wxmenuupdateui} is called
to ensure that the menu items are in the correct state. The menu does not get deleted
by the window.

\pythonnote{In place of a single overloaded method name, wxPython
implements the following methods:\par
\indented{2cm}{\begin{twocollist}
\twocolitem{{\bf PopupMenu(menu, point)}}{Specifies position with a wxPoint}
\twocolitem{{\bf PopupMenuXY(menu, x, y)}}{Specifies position with two integers (x, y)}
\end{twocollist}}
}

\membersection{wxWindow::PushEventHandler}\label{wxwindowpusheventhandler}

\func{void}{PushEventHandler}{\param{wxEvtHandler* }{handler}}

Pushes this event handler onto the event stack for the window.

\wxheading{Parameters}

\docparam{handler}{Specifies the handler to be pushed.}

\wxheading{Remarks}

An event handler is an object that is capable of processing the events
sent to a window. By default, the window is its own event handler, but
an application may wish to substitute another, for example to allow
central implementation of event-handling for a variety of different
window classes.

\helpref{wxWindow::PushEventHandler}{wxwindowpusheventhandler} allows
an application to set up a chain of event handlers, where an event not handled by one event handler is
handed to the next one in the chain. Use \helpref{wxWindow::PopEventHandler}{wxwindowpopeventhandler} to
remove the event handler.

\wxheading{See also}

\helpref{wxWindow::SetEventHandler}{wxwindowseteventhandler},\rtfsp
\helpref{wxWindow::GetEventHandler}{wxwindowgeteventhandler},\rtfsp
\helpref{wxWindow::PopEventHandler}{wxwindowpusheventhandler},\rtfsp
\helpref{wxEvtHandler::ProcessEvent}{wxevthandlerprocessevent},\rtfsp
\helpref{wxEvtHandler}{wxevthandler}

\membersection{wxWindow::Raise}\label{wxwindowraise}

\func{void}{Raise}{\void}

Raises the window to the top of the window hierarchy if it is a managed window (dialog
or frame).

\membersection{wxWindow::Refresh}\label{wxwindowrefresh}

\func{virtual void}{Refresh}{\param{bool}{ eraseBackground = TRUE}, \param{const wxRect* }{rect
= NULL}}

Causes a message or event to be generated to repaint the
window.

\wxheading{Parameters}

\docparam{eraseBackground}{If TRUE, the background will be
erased.}

\docparam{rect}{If non-NULL, only the given rectangle will
be treated as damaged.}

\membersection{wxWindow::ReleaseMouse}\label{wxwindowreleasemouse}

\func{virtual void}{ReleaseMouse}{\void}

Releases mouse input captured with \helpref{wxWindow::CaptureMouse}{wxwindowcapturemouse}.

\wxheading{See also}

\helpref{wxWindow::CaptureMouse}{wxwindowcapturemouse}

\membersection{wxWindow::RemoveChild}\label{wxwindowremovechild}

\func{virtual void}{RemoveChild}{\param{wxWindow* }{child}}

Removes a child window.  This is called automatically by window deletion
functions so should not be required by the application programmer.

\wxheading{Parameters}

\docparam{child}{Child window to remove.}

\membersection{wxWindow::Reparent}\label{wxwindowreparent}

\func{virtual bool}{Reparent}{\param{wxWindow* }{newParent}}

Reparents the window, i.e the window will be removed from its
current parent window (e.g. a non-standard toolbar in a wxFrame)
and then re-inserted into another. Available on Windows and GTK.

\wxheading{Parameters}

\docparam{newParent}{New parent.}

\membersection{wxWindow::ScreenToClient}\label{wxwindowscreentoclient}

\constfunc{virtual void}{ScreenToClient}{\param{int* }{x}, \param{int* }{y}}

\constfunc{virtual wxPoint}{ScreenToClient}{\param{const wxPoint\& }{pt}}

Converts from screen to client window coordinates.

\wxheading{Parameters}

\docparam{x}{Stores the screen x coordinate and receives the client x coordinate.}

\docparam{y}{Stores the screen x coordinate and receives the client x coordinate.}

\docparam{pt}{The screen position for the second form of the function.}

\pythonnote{In place of a single overloaded method name, wxPython
implements the following methods:\par
\indented{2cm}{\begin{twocollist}
\twocolitem{{\bf ScreenToClient(point)}}{Accepts and returns a wxPoint}
\twocolitem{{\bf ScreenToClientXY(x, y)}}{Returns a 2-tuple, (x, y)}
\end{twocollist}}
}


\membersection{wxWindow::ScrollWindow}\label{wxwindowscrollwindow}

\func{virtual void}{ScrollWindow}{\param{int }{dx}, \param{int }{dy}, \param{const wxRect*}{ rect = NULL}}

Physically scrolls the pixels in the window and move child windows accordingly.

\wxheading{Parameters}

\docparam{dx}{Amount to scroll horizontally.}

\docparam{dy}{Amount to scroll vertically.}

\docparam{rect}{Rectangle to invalidate. If this is NULL, the whole window is invalidated. If you
pass a rectangle corresponding to the area of the window exposed by the scroll, your painting handler
can optimize painting by checking for the invalidated region. This parameter is ignored under GTK.}

\wxheading{Remarks}

Use this function to optimise your scrolling implementations, to minimise the area that must be
redrawn. Note that it is rarely required to call this function from a user program.

\membersection{wxWindow::SetAcceleratorTable}\label{wxwindowsetacceleratortable}

\func{virtual void}{SetAcceleratorTable}{\param{const wxAcceleratorTable\&}{ accel}}

Sets the accelerator table for this window. See \helpref{wxAcceleratorTable}{wxacceleratortable}.

\membersection{wxWindow::SetAutoLayout}\label{wxwindowsetautolayout}

\func{void}{SetAutoLayout}{\param{bool}{ autoLayout}}

Determines whether the \helpref{wxWindow::Layout}{wxwindowlayout} function will
be called automatically when the window is resized. Use in connection with 
\helpref{wxWindow::SetSizer}{wxwindowsetsizer} and 
\helpref{wxWindow::SetConstraints}{wxwindowsetconstraints} for laying out
subwindows.

\wxheading{Parameters}

\docparam{autoLayout}{Set this to TRUE if you wish the Layout function to be called
from within wxWindow::OnSize functions.}

\wxheading{See also}

\helpref{wxWindow::SetConstraints}{wxwindowsetconstraints}

\membersection{wxWindow::SetBackgroundColour}\label{wxwindowsetbackgroundcolour}

\func{virtual void}{SetBackgroundColour}{\param{const wxColour\& }{colour}}

Sets the background colour of the window.

\wxheading{Parameters}

\docparam{colour}{The colour to be used as the background colour.}

\wxheading{Remarks}

The background colour is usually painted by the default\rtfsp
\helpref{wxWindow::OnEraseBackground}{wxwindowonerasebackground} event handler function
under Windows and automatically under GTK.

Note that setting the background colour does not cause an immediate refresh, so you
may wish to call \helpref{wxWindow::Clear}{wxwindowclear} or \helpref{wxWindow::Refresh}{wxwindowrefresh} after
calling this function.

Use this function with care under GTK as the new appearance of the window might
not look equally well when used with "Themes", i.e GTK's ability to change its
look as the user wishes with run-time loadable modules.

\wxheading{See also}

\helpref{wxWindow::GetBackgroundColour}{wxwindowgetbackgroundcolour},\rtfsp
\helpref{wxWindow::SetForegroundColour}{wxwindowsetforegroundcolour},\rtfsp
\helpref{wxWindow::GetForegroundColour}{wxwindowgetforegroundcolour},\rtfsp
\helpref{wxWindow::Clear}{wxwindowclear},\rtfsp
\helpref{wxWindow::Refresh}{wxwindowrefresh},\rtfsp
\helpref{wxWindow::OnEraseBackground}{wxwindowonerasebackground}

\membersection{wxWindow::SetCaret}\label{wxwindowsetcaret}

\constfunc{void}{SetCaret}{\param{wxCaret *}{caret}}

Sets the \helpref{caret}{wxcaret} associated with the window.

\membersection{wxWindow::SetClientSize}\label{wxwindowsetclientsize}

\func{virtual void}{SetClientSize}{\param{int}{ width}, \param{int}{ height}}

\func{virtual void}{SetClientSize}{\param{const wxSize\&}{ size}}

This sets the size of the window client area in pixels. Using this function to size a window
tends to be more device-independent than \helpref{wxWindow::SetSize}{wxwindowsetsize}, since the application need not
worry about what dimensions the border or title bar have when trying to fit the window
around panel items, for example.

\wxheading{Parameters}

\docparam{width}{The required client area width.}

\docparam{height}{The required client area height.}

\docparam{size}{The required client size.}

\pythonnote{In place of a single overloaded method name, wxPython
implements the following methods:\par
\indented{2cm}{\begin{twocollist}
\twocolitem{{\bf SetClientSize(size)}}{Accepts a wxSize}
\twocolitem{{\bf SetClientSizeWH(width, height)}}{}
\end{twocollist}}
}

\membersection{wxWindow::SetCursor}\label{wxwindowsetcursor}

\func{virtual void}{SetCursor}{\param{const wxCursor\&}{cursor}}

% VZ: the docs are correct, if the code doesn't behave like this, it must be
%     changed
Sets the window's cursor. Notice that the window cursor also sets it for the
children of the window implicitly.

The {\it cursor} may be {\tt wxNullCursor} in which case the window cursor will
be reset back to default.

\wxheading{Parameters}

\docparam{cursor}{Specifies the cursor that the window should normally display.}

\wxheading{See also}

\helpref{::wxSetCursor}{wxsetcursor}, \helpref{wxCursor}{wxcursor}

\membersection{wxWindow::SetConstraints}\label{wxwindowsetconstraints}

\func{void}{SetConstraints}{\param{wxLayoutConstraints* }{constraints}}

Sets the window to have the given layout constraints. The window
will then own the object, and will take care of its deletion.
If an existing layout constraints object is already owned by the
window, it will be deleted.

\wxheading{Parameters}

\docparam{constraints}{The constraints to set. Pass NULL to disassociate and delete the window's
constraints.}

\wxheading{Remarks}

You must call \helpref{wxWindow::SetAutoLayout}{wxwindowsetautolayout} to tell a window to use
the constraints automatically in OnSize; otherwise, you must override OnSize and call Layout()
explicitly. When setting both a wxLayoutConstraints and a \helpref{wxSizer}{wxsizer}, only the
sizer will have effect.

\membersection{wxWindow::SetDropTarget}\label{wxwindowsetdroptarget}

\func{void}{SetDropTarget}{\param{wxDropTarget*}{ target}}

Associates a drop target with this window.

If the window already has a drop target, it is deleted.

\wxheading{See also}

\helpref{wxWindow::GetDropTarget}{wxwindowgetdroptarget},
\helpref{Drag and drop overview}{wxdndoverview}

\membersection{wxWindow::SetEventHandler}\label{wxwindowseteventhandler}

\func{void}{SetEventHandler}{\param{wxEvtHandler* }{handler}}

Sets the event handler for this window.

\wxheading{Parameters}

\docparam{handler}{Specifies the handler to be set.}

\wxheading{Remarks}

An event handler is an object that is capable of processing the events
sent to a window. By default, the window is its own event handler, but
an application may wish to substitute another, for example to allow
central implementation of event-handling for a variety of different
window classes.

It is usually better to use \helpref{wxWindow::PushEventHandler}{wxwindowpusheventhandler} since
this sets up a chain of event handlers, where an event not handled by one event handler is
handed to the next one in the chain.

\wxheading{See also}

\helpref{wxWindow::GetEventHandler}{wxwindowgeteventhandler},\rtfsp
\helpref{wxWindow::PushEventHandler}{wxwindowpusheventhandler},\rtfsp
\helpref{wxWindow::PopEventHandler}{wxwindowpusheventhandler},\rtfsp
\helpref{wxEvtHandler::ProcessEvent}{wxevthandlerprocessevent},\rtfsp
\helpref{wxEvtHandler}{wxevthandler}

\membersection{wxWindow::SetExtraStyle}\label{wxwindowsetextrastyle}

\func{void}{SetExtraStyle}{\param{long }{exStyle}}

Sets the extra style bits for the window. The currently defined extra style
bits are:

\twocolwidtha{5cm}%
\begin{twocollist}\itemsep=0pt
\twocolitem{\windowstyle{wxWS\_EX\_VALIDATE\_RECURSIVELY}}{TransferDataTo/FromWindow()
and Validate() methods will recursively descend into all children of the
window if it has this style flag set.}
\end{twocollist}

\membersection{wxWindow::SetFocus}\label{wxwindowsetfocus}

\func{virtual void}{SetFocus}{\void}

This sets the window to receive keyboard input.

\membersection{wxWindow::SetFont}\label{wxwindowsetfont}

\func{void}{SetFont}{\param{const wxFont\& }{font}}

Sets the font for this window.

\wxheading{Parameters}

\docparam{font}{Font to associate with this window.}

\wxheading{See also}

\helpref{wxWindow::GetFont}{wxwindowgetfont}

\membersection{wxWindow::SetForegroundColour}\label{wxwindowsetforegroundcolour}

\func{virtual void}{SetForegroundColour}{\param{const wxColour\& }{colour}}

Sets the foreground colour of the window.

\wxheading{Parameters}

\docparam{colour}{The colour to be used as the foreground colour.}

\wxheading{Remarks}

The interpretation of foreground colour is open to interpretation according
to the window class; it may be the text colour or other colour, or it may not
be used at all.

Note that when using this functions under GTK, you will disable the so called "themes",
i.e. the user chosen apperance of windows and controls, including the themes of
their parent windows.

\wxheading{See also}

\helpref{wxWindow::GetForegroundColour}{wxwindowgetforegroundcolour},\rtfsp
\helpref{wxWindow::SetBackgroundColour}{wxwindowsetbackgroundcolour},\rtfsp
\helpref{wxWindow::GetBackgroundColour}{wxwindowgetbackgroundcolour}

\membersection{wxWindow::SetHelpText}\label{wxwindowsethelptext}

\func{virtual void}{SetHelpText}{\param{const wxString\& }{helpText}}

Sets the help text to be used as context-sensitive help for this window.

Note that the text is actually stored by the current \helpref{wxHelpProvider}{wxhelpprovider} implementation,
and not in the window object itself.

\wxheading{See also}

\helpref{GetHelpText}{wxwindowgethelptext}, \helpref{wxHelpProvider}{wxhelpprovider}

\membersection{wxWindow::SetId}\label{wxwindowsetid}

\func{void}{SetId}{\param{int}{ id}}

Sets the identifier of the window.

\wxheading{Remarks}

Each window has an integer identifier. If the application has not provided one,
an identifier will be generated. Normally, the identifier should be provided
on creation and should not be modified subsequently.

\wxheading{See also}

\helpref{wxWindow::GetId}{wxwindowgetid},\rtfsp
\helpref{Window identifiers}{windowids}

\membersection{wxWindow::SetName}\label{wxwindowsetname}

\func{virtual void}{SetName}{\param{const wxString\& }{name}}

Sets the window's name.

\wxheading{Parameters}

\docparam{name}{A name to set for the window.}

\wxheading{See also}

\helpref{wxWindow::GetName}{wxwindowgetname}

\membersection{wxWindow::SetPalette}\label{wxwindowsetpalette}

\func{virtual void}{SetPalette}{\param{wxPalette* }{palette}}

Obsolete - use \helpref{wxDC::SetPalette}{wxdcsetpalette} instead.

\membersection{wxWindow::SetScrollbar}\label{wxwindowsetscrollbar}

\func{virtual void}{SetScrollbar}{\param{int }{orientation}, \param{int }{position},\rtfsp
\param{int }{thumbSize}, \param{int }{range},\rtfsp
\param{bool }{refresh = TRUE}}

Sets the scrollbar properties of a built-in scrollbar.

\wxheading{Parameters}

\docparam{orientation}{Determines the scrollbar whose page size is to be set. May be wxHORIZONTAL or wxVERTICAL.}

\docparam{position}{The position of the scrollbar in scroll units.}

\docparam{thumbSize}{The size of the thumb, or visible portion of the scrollbar, in scroll units.}

\docparam{range}{The maximum position of the scrollbar.}

\docparam{refresh}{TRUE to redraw the scrollbar, FALSE otherwise.}

\wxheading{Remarks}

Let's say you wish to display 50 lines of text, using the same font.
The window is sized so that you can only see 16 lines at a time.

You would use:

{\small%
\begin{verbatim}
  SetScrollbar(wxVERTICAL, 0, 16, 50);
\end{verbatim}
}

Note that with the window at this size, the thumb position can never go
above 50 minus 16, or 34.

You can determine how many lines are currently visible by dividing the current view
size by the character height in pixels.

When defining your own scrollbar behaviour, you will always need to recalculate
the scrollbar settings when the window size changes. You could therefore put your
scrollbar calculations and SetScrollbar
call into a function named AdjustScrollbars, which can be called initially and also
from your \helpref{wxWindow::OnSize}{wxwindowonsize} event handler function.

\wxheading{See also}

\helpref{Scrolling overview}{scrollingoverview},\rtfsp
\helpref{wxScrollBar}{wxscrollbar}, \helpref{wxScrolledWindow}{wxscrolledwindow}

\begin{comment}
\membersection{wxWindow::SetScrollPage}\label{wxwindowsetscrollpage}

\func{virtual void}{SetScrollPage}{\param{int }{orientation}, \param{int }{pageSize}, \param{bool }{refresh = TRUE}}

Sets the page size of one of the built-in scrollbars.

\wxheading{Parameters}

\docparam{orientation}{Determines the scrollbar whose page size is to be set. May be wxHORIZONTAL or wxVERTICAL.}

\docparam{pageSize}{Page size in scroll units.}

\docparam{refresh}{TRUE to redraw the scrollbar, FALSE otherwise.}

\wxheading{Remarks}

The page size of a scrollbar is the number of scroll units that the scroll thumb travels when you
click on the area above/left of or below/right of the thumb. Normally you will want a whole visible
page to be scrolled, i.e. the size of the current view (perhaps the window client size). This
value has to be adjusted when the window is resized, since the page size will have changed.

In addition to specifying how far the scroll thumb travels when paging, in Motif and some versions of Windows
the thumb changes size to reflect the page size relative to the length of the document. When the
document size is only slightly bigger than the current view (window) size, almost all of the scrollbar
will be taken up by the thumb. When the two values become the same, the scrollbar will (on some systems)
disappear.

Currently, this function should be called before SetPageRange, because of a quirk in the Windows
handling of pages and ranges.

\wxheading{See also}

\helpref{wxWindow::SetScrollPos}{wxwindowsetscrollpos},\rtfsp
\helpref{wxWindow::GetScrollPos}{wxwindowsetscrollpos},\rtfsp
\helpref{wxWindow::GetScrollPage}{wxwindowsetscrollpage},\rtfsp
\helpref{wxScrollBar}{wxscrollbar}, \helpref{wxScrolledWindow}{wxscrolledwindow}
\end{comment}

\membersection{wxWindow::SetScrollPos}\label{wxwindowsetscrollpos}

\func{virtual void}{SetScrollPos}{\param{int }{orientation}, \param{int }{pos}, \param{bool }{refresh = TRUE}}

Sets the position of one of the built-in scrollbars.

\wxheading{Parameters}

\docparam{orientation}{Determines the scrollbar whose position is to be set. May be wxHORIZONTAL or wxVERTICAL.}

\docparam{pos}{Position in scroll units.}

\docparam{refresh}{TRUE to redraw the scrollbar, FALSE otherwise.}

\wxheading{Remarks}

This function does not directly affect the contents of the window: it is up to the
application to take note of scrollbar attributes and redraw contents accordingly.

\wxheading{See also}

\helpref{wxWindow::SetScrollbar}{wxwindowsetscrollbar},\rtfsp
\helpref{wxWindow::GetScrollPos}{wxwindowsetscrollpos},\rtfsp
\helpref{wxWindow::GetScrollThumb}{wxwindowgetscrollthumb},\rtfsp
\helpref{wxScrollBar}{wxscrollbar}, \helpref{wxScrolledWindow}{wxscrolledwindow}

\begin{comment}
\membersection{wxWindow::SetScrollRange}\label{wxwindowsetscrollrange}

\func{virtual void}{SetScrollRange}{\param{int }{orientation}, \param{int }{range}, \param{bool }{refresh = TRUE}}

Sets the range of one of the built-in scrollbars.

\wxheading{Parameters}

\docparam{orientation}{Determines the scrollbar whose range is to be set. May be wxHORIZONTAL or wxVERTICAL.}

\docparam{range}{Scroll range.}

\docparam{refresh}{TRUE to redraw the scrollbar, FALSE otherwise.}

\wxheading{Remarks}

The range of a scrollbar is the number of steps that the thumb may travel, rather than the total
object length of the scrollbar. If you are implementing a scrolling window, for example, you
would adjust the scroll range when the window is resized, by subtracting the window view size from the
total virtual window size. When the two sizes are the same (all the window is visible), the range goes to zero
and usually the scrollbar will be automatically hidden.

\wxheading{See also}

\helpref{wxWindow::SetScrollPos}{wxwindowsetscrollpos},\rtfsp
\helpref{wxWindow::SetScrollPage}{wxwindowsetscrollpage},\rtfsp
\helpref{wxWindow::GetScrollPos}{wxwindowsetscrollpos},\rtfsp
\helpref{wxWindow::GetScrollPage}{wxwindowsetscrollpage},\rtfsp
\helpref{wxScrollBar}{wxscrollbar}, \helpref{wxScrolledWindow}{wxscrolledwindow}
\end{comment}

\membersection{wxWindow::SetSize}\label{wxwindowsetsize}

\func{virtual void}{SetSize}{\param{int}{ x}, \param{int}{ y}, \param{int}{ width}, \param{int}{ height},
 \param{int}{ sizeFlags = wxSIZE\_AUTO}}

\func{virtual void}{SetSize}{\param{const wxRect\&}{ rect}}

Sets the size and position of the window in pixels.

\func{virtual void}{SetSize}{\param{int}{ width}, \param{int}{ height}}

\func{virtual void}{SetSize}{\param{const wxSize\&}{ size}}

Sets the size of the window in pixels.

\wxheading{Parameters}

\docparam{x}{Required x position in pixels, or -1 to indicate that the existing
value should be used.}

\docparam{y}{Required y position in pixels, or -1 to indicate that the existing
value should be used.}

\docparam{width}{Required width in pixels, or -1 to indicate that the existing
value should be used.}

\docparam{height}{Required height position in pixels, or -1 to indicate that the existing
value should be used.}

\docparam{size}{\helpref{wxSize}{wxsize} object for setting the size.}

\docparam{rect}{\helpref{wxRect}{wxrect} object for setting the position and size.}

\docparam{sizeFlags}{Indicates the interpretation of other parameters. It is a bit list of the following:

{\bf wxSIZE\_AUTO\_WIDTH}: a -1 width value is taken to indicate
a wxWindows-supplied default width.\\
{\bf wxSIZE\_AUTO\_HEIGHT}: a -1 height value is taken to indicate
a wxWindows-supplied default width.\\
{\bf wxSIZE\_AUTO}: -1 size values are taken to indicate
a wxWindows-supplied default size.\\
{\bf wxSIZE\_USE\_EXISTING}: existing dimensions should be used
if -1 values are supplied.\\
{\bf wxSIZE\_ALLOW\_MINUS\_ONE}: allow dimensions of -1 and less to be interpreted
as real dimensions, not default values.
}

\wxheading{Remarks}

The second form is a convenience for calling the first form with default
x and y parameters, and must be used with non-default width and height values.

The first form sets the position and optionally size, of the window.
Parameters may be -1 to indicate either that a default should be supplied
by wxWindows, or that the current value of the dimension should be used.

\wxheading{See also}

\helpref{wxWindow::Move}{wxwindowmove}

\pythonnote{In place of a single overloaded method name, wxPython
implements the following methods:\par
\indented{2cm}{\begin{twocollist}
\twocolitem{{\bf SetDimensions(x, y, width, height, sizeFlags=wxSIZE\_AUTO)}}{}
\twocolitem{{\bf SetSize(size)}}{}
\twocolitem{{\bf SetPosition(point)}}{}
\end{twocollist}}
}

\membersection{wxWindow::SetSizeHints}\label{wxwindowsetsizehints}

\func{virtual void}{SetSizeHints}{\param{int}{ minW=-1}, \param{int}{ minH=-1}, \param{int}{ maxW=-1}, \param{int}{ maxH=-1},
 \param{int}{ incW=-1}, \param{int}{ incH=-1}}

Allows specification of minimum and maximum window sizes, and window size increments.
If a pair of values is not set (or set to -1), the default values will be used.

\wxheading{Parameters}

\docparam{minW}{Specifies the minimum width allowable.}

\docparam{minH}{Specifies the minimum height allowable.}

\docparam{maxW}{Specifies the maximum width allowable.}

\docparam{maxH}{Specifies the maximum height allowable.}

\docparam{incW}{Specifies the increment for sizing the width (Motif/Xt only).}

\docparam{incH}{Specifies the increment for sizing the height (Motif/Xt only).}

\wxheading{Remarks}

If this function is called, the user will not be able to size the window outside the
given bounds.

The resizing increments are only significant under Motif or Xt.

\membersection{wxWindow::SetSizer}\label{wxwindowsetsizer}

\func{void}{SetSizer}{\param{wxSizer* }{sizer}}

Sets the window to have the given layout sizer. The window
will then own the object, and will take care of its deletion.
If an existing layout constraints object is already owned by the
window, it will be deleted.

\wxheading{Parameters}

\docparam{sizer}{The sizer to set. Pass NULL to disassociate and delete the window's
sizer.}

\wxheading{Remarks}

You must call \helpref{wxWindow::SetAutoLayout}{wxwindowsetautolayout} to tell a window to use
the sizer automatically in OnSize; otherwise, you must override OnSize and call Layout()
explicitly. When setting both a wxSizer and a \helpref{wxLayoutConstraints}{wxlayoutconstraints},
only the sizer will have effect.

\membersection{wxWindow::SetTitle}\label{wxwindowsettitle}

\func{virtual void}{SetTitle}{\param{const wxString\& }{title}}

Sets the window's title. Applicable only to frames and dialogs.

\wxheading{Parameters}

\docparam{title}{The window's title.}

\wxheading{See also}

\helpref{wxWindow::GetTitle}{wxwindowgettitle}

\membersection{wxWindow::SetValidator}\label{wxwindowsetvalidator}

\func{virtual void}{SetValidator}{\param{const wxValidator\&}{ validator}}

Deletes the current validator (if any) and sets the window validator, having called wxValidator::Clone to
create a new validator of this type.

\membersection{wxWindow::SetToolTip}\label{wxwindowsettooltip}

\func{void}{SetToolTip}{\param{const wxString\& }{tip}}

\func{void}{SetToolTip}{\param{wxToolTip* }{tip}}

Attach a tooltip to the window.

See also: \helpref{GetToolTip}{wxwindowgettooltip},
          \helpref{wxToolTip}{wxtooltip}


\membersection{wxWindow::GetToolTip}\label{wxwindowgettooltip}

\constfunc{wxToolTip*}{GetToolTip}{\void}

Get the associated tooltip or NULL if none.



\membersection{wxWindow::SetWindowStyle}\label{wxwindowsetwindowstyle}

\func{void}{SetWindowStyle}{\param{long}{ style}}

Identical to \helpref{SetWindowStyleFlag}{wxwindowsetwindowstyleflag}.

\membersection{wxWindow::SetWindowStyleFlag}\label{wxwindowsetwindowstyleflag}

\func{virtual void}{SetWindowStyleFlag}{\param{long}{ style}}

Sets the style of the window. Please note that some styles cannot be changed
after the window creation and that \helpref{Refresh()}{wxwindowrefresh} might
be called after changing the others for the change to take place immediately.

See \helpref{Window styles}{windowstyles} for more information about flags.

\wxheading{See also}

\helpref{GetWindowStyleFlag}{wxwindowgetwindowstyleflag}

\membersection{wxWindow::Show}\label{wxwindowshow}

\func{virtual bool}{Show}{\param{bool}{ show}}

Shows or hides the window.

\wxheading{Parameters}

\docparam{show}{If TRUE, displays the window and brings it to the front. Otherwise,
hides the window.}

\wxheading{See also}

\helpref{wxWindow::IsShown}{wxwindowisshown}

\membersection{wxWindow::TransferDataFromWindow}\label{wxwindowtransferdatafromwindow}

\func{virtual bool}{TransferDataFromWindow}{\void}

Transfers values from child controls to data areas specified by their validators. Returns
FALSE if a transfer failed.

If the window has {\tt wxWS\_EX\_VALIDATE\_RECURSIVELY} extra style flag set,
the method will also call TransferDataFromWindow() of all child windows.

\wxheading{See also}

\helpref{wxWindow::TransferDataToWindow}{wxwindowtransferdatatowindow},\rtfsp
\helpref{wxValidator}{wxvalidator}, \helpref{wxWindow::Validate}{wxwindowvalidate}

\membersection{wxWindow::TransferDataToWindow}\label{wxwindowtransferdatatowindow}

\func{virtual bool}{TransferDataToWindow}{\void}

Transfers values to child controls from data areas specified by their validators.

If the window has {\tt wxWS\_EX\_VALIDATE\_RECURSIVELY} extra style flag set,
the method will also call TransferDataToWindow() of all child windows.

\wxheading{Return value}

Returns FALSE if a transfer failed.

\wxheading{See also}

\helpref{wxWindow::TransferDataFromWindow}{wxwindowtransferdatafromwindow},\rtfsp
\helpref{wxValidator}{wxvalidator}, \helpref{wxWindow::Validate}{wxwindowvalidate}

\membersection{wxWindow::Validate}\label{wxwindowvalidate}

\func{virtual bool}{Validate}{\void}

Validates the current values of the child controls using their validators.

If the window has {\tt wxWS\_EX\_VALIDATE\_RECURSIVELY} extra style flag set,
the method will also call Validate() of all child windows.

\wxheading{Return value}

Returns FALSE if any of the validations failed.

\wxheading{See also}

\helpref{wxWindow::TransferDataFromWindow}{wxwindowtransferdatafromwindow},\rtfsp
\helpref{wxWindow::TransferDataFromWindow}{wxwindowtransferdatafromwindow},\rtfsp
\helpref{wxValidator}{wxvalidator}

\membersection{wxWindow::WarpPointer}\label{wxwindowwarppointer}

\func{void}{WarpPointer}{\param{int}{ x}, \param{int}{ y}}

Moves the pointer to the given position on the window.

\wxheading{Parameters}

\docparam{x}{The new x position for the cursor.}

\docparam{y}{The new y position for the cursor.}

