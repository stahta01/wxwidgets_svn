%%%%%%%%%%%%%%%%%%%%%%%%%%%%%%%%%%%%%%%%%%%%%%%%%%%%%%%%%%%%%%%%%%%%%%%%%%%%%%%
%% Name:        cmdlpars.tex
%% Purpose:     wxCmdLineParser documentation
%% Author:      Vadim Zeitlin
%% Modified by:
%% Created:     27.03.00
%% RCS-ID:      $Id$
%% Copyright:   (c) Vadim Zeitlin
%% License:     wxWindows license
%%%%%%%%%%%%%%%%%%%%%%%%%%%%%%%%%%%%%%%%%%%%%%%%%%%%%%%%%%%%%%%%%%%%%%%%%%%%%%%

\section{\class{wxCmdLineParser}}\label{wxcmdlineparser}

wxCmdLineParser is a class for parsing the command line.

It has the following features:

\begin{enumerate}\itemsep=0pt
\item distinguishes options, switches and parameters; allows option grouping
\item allows both short and long options
\item automatically generates the usage message from the command line description
\item does type checks on the options values (number, date, $\ldots$).
\end{enumerate}

To use it you should follow these steps:

\begin{enumerate}\itemsep=0pt
\item \helpref{construct}{wxcmdlineparserconstruction} an object of this class
giving it the command line to parse and optionally its description or use 
{\tt AddXXX()} functions later
\item call {\tt Parse()}
\item use {\tt Found()} to retrieve the results
\end{enumerate}

In the documentation below the following terminology is used:

\begin{twocollist}\itemsep=0pt
\twocolitem{switch}{This is a boolean option which can be given or not, but
which doesn't have any value. We use the word switch to distinguish such boolean
options from more generic options like those described below. For example, 
{\tt -v} might be a switch meaning "enable verbose mode".}
\twocolitem{option}{Option for us here is something which comes with a value 0
unlike a switch. For example, {\tt -o:filename} might be an option which allows
to specify the name of the output file.}
\twocolitem{parameter}{This is a required program argument.}
\end{twocollist}

\wxheading{Derived from}

No base class

\wxheading{Include files}

<wx/cmdline.h>

\wxheading{Constants}

The structure wxCmdLineEntryDesc is used to describe the one command
line switch, option or parameter. An array of such structures should be passed
to \helpref{SetDesc()}{wxcmdlineparsersetdesc}. Also, the meanings of parameters
of the {\tt AddXXX()} functions are the same as of the corresponding fields in
this structure:

\begin{verbatim}
struct wxCmdLineEntryDesc
{
    wxCmdLineEntryType kind;
    const wxChar *shortName;
    const wxChar *longName;
    const wxChar *description;
    wxCmdLineParamType type;
    int flags;
};
\end{verbatim}

The type of a command line entity is in the {\tt kind} field and may be one of
the following constants:

{\small%
\begin{verbatim}
enum wxCmdLineEntryType
{
    wxCMD_LINE_SWITCH,
    wxCMD_LINE_OPTION,
    wxCMD_LINE_PARAM,
    wxCMD_LINE_NONE         // use this to terminate the list
}
\end{verbatim}
}

The field {\tt shortName} is the usual, short, name of the switch or the option.
{\tt longName} is the corresponding long name or NULL if the option has no long
name. Both of these fields are unused for the parameters. Both the short and
long option names can contain only letters, digits and the underscores.

{\tt description} is used by the \helpref{Usage()}{wxcmdlineparserusage} method
to construct a help message explaining the syntax of the program.

The possible values of {\tt type} which specifies the type of the value accepted
by an option or parameter are:

{\small%
\begin{verbatim}
enum wxCmdLineParamType
{
    wxCMD_LINE_VAL_STRING,  // default
    wxCMD_LINE_VAL_NUMBER,
    wxCMD_LINE_VAL_DATE,
    wxCMD_LINE_VAL_NONE
}
\end{verbatim}
}

Finally, the {\tt flags} field is a combination of the following bit masks:

{\small%
\begin{verbatim}
enum
{
    wxCMD_LINE_OPTION_MANDATORY = 0x01, // this option must be given
    wxCMD_LINE_PARAM_OPTIONAL   = 0x02, // the parameter may be omitted
    wxCMD_LINE_PARAM_MULTIPLE   = 0x04, // the parameter may be repeated
    wxCMD_LINE_OPTION_HELP      = 0x08, // this option is a help request
    wxCMD_LINE_NEEDS_SEPARATOR  = 0x10, // must have sep before the value
}
\end{verbatim}
}

Notice that by default (i.e. if flags are just $0$), options are optional (sic)
and each call to \helpref{AddParam()}{wxcmdlineparseraddparam} allows one more
parameter - this may be changed by giving non-default flags to it, i.e. use 
{\tt wxCMD\_LINE\_OPTION\_MANDATORY} to require that the option is given and 
{\tt wxCMD\_LINE\_PARAM\_OPTIONAL} to make a parameter optional. Also, 
{\tt wxCMD\_LINE\_PARAM\_MULTIPLE} may be specified if the programs accepts a
variable number of parameters - but it only can be given for the last parameter
in the command line description. If you use this flag, you will probably need to
use \helpref{GetParamCount}{wxcmdlineparsergetparamcount} to retrieve the number
of parameters effectively specified after calling 
\helpref{Parse}{wxcmdlineparserparse}.

The last flag {\tt wxCMD\_LINE\_NEEDS\_SEPARATOR} can be specified to require a
separator (either a colon, an equal sign or white space) between the option
name and its value. By default, no separator is required.

\wxheading{See also}

\helpref{wxApp::argc}{wxappargc} and \helpref{wxApp::argv}{wxappargv}\\
console sample

%%%%%%%%%%%%% Methods by group %%%%%%%%%%%%%
\latexignore{\rtfignore{\wxheading{Function groups}}}


\membersection{Construction}\label{wxcmdlineparserconstruction}

Before \helpref{Parse}{wxcmdlineparserparse} can be called, the command line
parser object must have the command line to parse and also the rules saying
which switches, options and parameters are valid - this is called command line
description in what follows.

You have complete freedom of choice as to when specify the required information,
the only restriction is that it must be done before calling 
\helpref{Parse}{wxcmdlineparserparse}.

To specify the command line to parse you may use either one of constructors
accepting it (\tt{wxCmdLineParser(argc, argv)} or \tt{wxCmdLineParser(const wxString&)} usually) 
or, if you use the default constructor, you can do it later by calling 
\helpref{SetCmdLine}{wxcmdlineparsersetcmdline}.

The same holds for command line description: it can be specified either in
the constructor (\helpref{without command line}{wxcmdlineparserwxcmdlineparserdesc} or 
\helpref{together with it}{wxcmdlineparserwxcmdlineparserdescargc}) or
constructed later using either \helpref{SetDesc}{wxcmdlineparsersetdesc} or
combination of \helpref{AddSwitch}{wxcmdlineparseraddswitch}, 
\helpref{AddOption}{wxcmdlineparseraddoption} and 
\helpref{AddParam}{wxcmdlineparseraddparam} methods.

Using constructors or \helpref{SetDesc}{wxcmdlineparsersetdesc} uses a (usually 
{\tt const static}) table containing the command line description. If you want
to decide which options to accept during the run-time, using one of the 
{\tt AddXXX()} functions above might be preferable.


\membersection{Customization}\label{wxcmdlineparsercustomization}

wxCmdLineParser has several global options which may be changed by the
application. All of the functions described in this section should be called
before \helpref{Parse}{wxcmdlineparserparse}.

First global option is the support for long (also known as GNU-style) options.
The long options are the ones which start with two dashes ({\tt "--"}) and look
like this: {\tt --verbose}, i.e. they generally are complete words and not some
abbreviations of them. As long options are used by more and more applications,
they are enabled by default, but may be disabled with 
\helpref{DisableLongOptions}{wxcmdlineparserdisablelongoptions}.

Another global option is the set of characters which may be used to start an
option (otherwise, the word on the command line is assumed to be a parameter).
Under Unix, {\tt '-'} is always used, but Windows has at least two common
choices for this: {\tt '-'} and {\tt '/'}. Some programs also use {\tt '+'}.
The default is to use what suits most the current platform, but may be changed
with \helpref{SetSwitchChars}{wxcmdlineparsersetswitchchars} method.

Finally, \helpref{SetLogo}{wxcmdlineparsersetlogo} can be used to show some
application-specific text before the explanation given by 
\helpref{Usage}{wxcmdlineparserusage} function.


\membersection{Parsing command line}\label{wxcmdlineparserparsing}

After the command line description was constructed and the desired options were
set, you can finally call \helpref{Parse}{wxcmdlineparserparse} method.
It returns $0$ if the command line was correct and was parsed, $-1$ if the help
option was specified (this is a separate case as, normally, the program will
terminate after this) or a positive number if there was an error during the
command line parsing.

In the latter case, the appropriate error message and usage information are
logged by wxCmdLineParser itself using the standard wxWidgets logging functions.


\membersection{Getting results}\label{wxcmdlineparsergettingresults}

After calling \helpref{Parse}{wxcmdlineparserparse} (and if it returned $0$),
you may access the results of parsing using one of overloaded {\tt Found()}
methods.

For a simple switch, you will simply call 
\helpref{Found}{wxcmdlineparserfound} to determine if the switch was given
or not, for an option or a parameter, you will call a version of {\tt Found()} 
which also returns the associated value in the provided variable. All 
{\tt Found()} functions return true if the switch or option were found in the
command line or false if they were not specified.

%%%%%%%%%%%%% Methods in alphabetic order %%%%%%%%%%%%%
\helponly{\insertatlevel{2}{

\wxheading{Members}

}}


\membersection{wxCmdLineParser::wxCmdLineParser}\label{wxcmdlineparserwxcmdlineparser}

\func{}{wxCmdLineParser}{\void}

Default constructor. You must use 
\helpref{SetCmdLine}{wxcmdlineparsersetcmdline} later.

\func{}{wxCmdLineParser}{\param{int }{argc}, \param{char** }{argv}}

\func{}{wxCmdLineParser}{\param{int }{argc}, \param{wchar\_t** }{argv}}

Constructors which specify the command line to parse. This is the traditional
(Unix) command line format. The parameters {\it argc} and {\it argv} have the
same meaning as for {\tt main()} function.

The second overloaded constructor is only available in Unicode build. The
first one is available in both ANSI and Unicode modes because under some
platforms the command line arguments are passed as ASCII strings even to
Unicode programs.

\func{}{wxCmdLineParser}{\param{const wxString\& }{cmdline}}

Constructor which specify the command line to parse in Windows format. The parameter 
{\it cmdline} has the same meaning as the corresponding parameter of 
{\tt WinMain()}.

\func{}{wxCmdLineParser}{\param{const wxCmdLineEntryDesc* }{desc}}

Specifies the \helpref{command line description}{wxcmdlineparsersetdesc} but not the
command line. You must use \helpref{SetCmdLine}{wxcmdlineparsersetcmdline} later.

\func{}{wxCmdLineParser}{\param{const wxCmdLineEntryDesc* }{desc}, \param{int }{argc}, \param{char** }{argv}}

Specifies both the command line (in Unix format) and the 
\helpref{command line description}{wxcmdlineparsersetdesc}.

\func{}{wxCmdLineParser}{\param{const wxCmdLineEntryDesc* }{desc}, \param{const wxString\& }{cmdline}}

Specifies both the command line (in Windows format) and the 
\helpref{command line description}{wxcmdlineparsersetdesc}.

\membersection{wxCmdLineParser::ConvertStringToArgs}\label{wxcmdlineparserconvertstringtoargs}

\func{static wxArrayString}{ConvertStringToArgs}{\param{const wxChar }{*cmdline}}

Breaks down the string containing the full command line in words. The words are
separated by whitespace. The quotes can be used in the input string to quote
the white space and the back slashes can be used to quote the quotes.


\membersection{wxCmdLineParser::SetCmdLine}\label{wxcmdlineparsersetcmdline}

\func{void}{SetCmdLine}{\param{int }{argc}, \param{char** }{argv}}

\func{void}{SetCmdLine}{\param{int }{argc}, \param{wchar\_t** }{argv}}

Set command line to parse after using one of the constructors which don't do it.
The second overload of this function is only available in Unicode build.

\func{void}{SetCmdLine}{\param{const wxString\& }{cmdline}}

Set command line to parse after using one of the constructors which don't do it.



\membersection{wxCmdLineParser::\destruct{wxCmdLineParser}}\label{wxcmdlineparserdtor}

\func{}{\destruct{wxCmdLineParser}}{\void}

Frees resources allocated by the object.

{\bf NB:} destructor is not virtual, don't use this class polymorphically.


\membersection{wxCmdLineParser::SetSwitchChars}\label{wxcmdlineparsersetswitchchars}

\func{void}{SetSwitchChars}{\param{const wxString\& }{switchChars}}

{\it switchChars} contains all characters with which an option or switch may
start. Default is {\tt "-"} for Unix, {\tt "-/"} for Windows.


\membersection{wxCmdLineParser::EnableLongOptions}\label{wxcmdlineparserenablelongoptions}

\func{void}{EnableLongOptions}{\param{bool }{enable = true}}

Enable or disable support for the long options.

As long options are not (yet) POSIX-compliant, this option allows to disable
them.

\wxheading{See also}

\helpref{Customization}{wxcmdlineparsercustomization} and \helpref{AreLongOptionsEnabled}{wxcmdlineparserarelongoptionsenabled}


\membersection{wxCmdLineParser::DisableLongOptions}\label{wxcmdlineparserdisablelongoptions}

\func{void}{DisableLongOptions}{\void}

Identical to \helpref{EnableLongOptions(false)}{wxcmdlineparserenablelongoptions}.


\membersection{wxCmdLineParser::AreLongOptionsEnabled}\label{wxcmdlineparserarelongoptionsenabled}

\func{bool}{AreLongOptionsEnabled}{\void}

Returns true if long options are enabled, otherwise false.

\wxheading{See also}

\helpref{EnableLongOptions}{wxcmdlineparserenablelongoptions}


\membersection{wxCmdLineParser::SetLogo}\label{wxcmdlineparsersetlogo}

\func{void}{SetLogo}{\param{const wxString\& }{logo}}

{\it logo} is some extra text which will be shown by 
\helpref{Usage}{wxcmdlineparserusage} method.


\membersection{wxCmdLineParser::SetDesc}\label{wxcmdlineparsersetdesc}

\func{void}{SetDesc}{\param{const wxCmdLineEntryDesc* }{desc}}

Construct the command line description

Take the command line description from the wxCMD\_LINE\_NONE terminated table.

Example of usage:

\begin{verbatim}
static const wxCmdLineEntryDesc cmdLineDesc[] =
{
    { wxCMD_LINE_SWITCH, "v", "verbose", "be verbose" },
    { wxCMD_LINE_SWITCH, "q", "quiet",   "be quiet" },

    { wxCMD_LINE_OPTION, "o", "output",  "output file" },
    { wxCMD_LINE_OPTION, "i", "input",   "input dir" },
    { wxCMD_LINE_OPTION, "s", "size",    "output block size", wxCMD_LINE_VAL_NUMBER },
    { wxCMD_LINE_OPTION, "d", "date",    "output file date", wxCMD_LINE_VAL_DATE },

    { wxCMD_LINE_PARAM,  NULL, NULL, "input file", wxCMD_LINE_VAL_STRING, wxCMD_LINE_PARAM_MULTIPLE },

    { wxCMD_LINE_NONE }
};

wxCmdLineParser parser;

parser.SetDesc(cmdLineDesc);
\end{verbatim}


\membersection{wxCmdLineParser::AddSwitch}\label{wxcmdlineparseraddswitch}

\func{void}{AddSwitch}{\param{const wxString\& }{name}, \param{const wxString\& }{lng = wxEmptyString}, \param{const wxString\& }{desc = wxEmptyString}, \param{int }{flags = 0}}

Add a switch {\it name} with an optional long name {\it lng} (no long name if it
is empty, which is default), description {\it desc} and flags {\it flags} to the
command line description.


\membersection{wxCmdLineParser::AddOption}\label{wxcmdlineparseraddoption}

\func{void}{AddOption}{\param{const wxString\& }{name}, \param{const wxString\& }{lng = wxEmptyString}, \param{const wxString\& }{desc = wxEmptyString}, \param{wxCmdLineParamType }{type = wxCMD\_LINE\_VAL\_STRING}, \param{int }{flags = 0}}

Add an option {\it name} with an optional long name {\it lng} (no long name if
it is empty, which is default) taking a value of the given type (string by
default) to the command line description.


\membersection{wxCmdLineParser::AddParam}\label{wxcmdlineparseraddparam}

\func{void}{AddParam}{\param{const wxString\& }{desc = wxEmptyString}, \param{wxCmdLineParamType }{type = wxCMD\_LINE\_VAL\_STRING}, \param{int }{flags = 0}}

Add a parameter of the given {\it type} to the command line description.


\membersection{wxCmdLineParser::Parse}\label{wxcmdlineparserparse}

\func{int}{Parse}{\param{bool }{giveUsage = {\tt true}}}

Parse the command line, return $0$ if ok, $-1$ if {\tt "-h"} or {\tt "--help"} 
option was encountered and the help message was given or a positive value if a
syntax error occurred.

\wxheading{Parameters}

\docparam{giveUsage}{If {\tt true} (default), the usage message is given if a
syntax error was encountered while parsing the command line or if help was
requested. If {\tt false}, only error messages about possible syntax errors
are given, use \helpref{Usage}{wxcmdlineparserusage} to show the usage message
from the caller if needed.}


\membersection{wxCmdLineParser::Usage}\label{wxcmdlineparserusage}

\func{void}{Usage}{\void}

Give the standard usage message describing all program options. It will use the
options and parameters descriptions specified earlier, so the resulting message
will not be helpful to the user unless the descriptions were indeed specified.

\wxheading{See also}

\helpref{SetLogo}{wxcmdlineparsersetlogo}


\membersection{wxCmdLineParser::Found}\label{wxcmdlineparserfound}

\constfunc{bool}{Found}{\param{const wxString\& }{name}}

Returns \true if the given switch was found, false otherwise.

\constfunc{bool}{Found}{\param{const wxString\& }{name}, \param{wxString* }{value}}

Returns \true if an option taking a string value was found and stores the
value in the provided pointer (which should not be NULL).

\constfunc{bool}{Found}{\param{const wxString\& }{name}, \param{long* }{value}}

Returns \true if an option taking an integer value was found and stores
the value in the provided pointer (which should not be NULL).

\constfunc{bool}{Found}{\param{const wxString\& }{name}, \param{wxDateTime* }{value}}

Returns \true if an option taking a date value was found and stores the
value in the provided pointer (which should not be NULL).


\membersection{wxCmdLineParser::GetParamCount}\label{wxcmdlineparsergetparamcount}

\constfunc{size\_t}{GetParamCount}{\void}

Returns the number of parameters found. This function makes sense mostly if you
had used {\tt wxCMD\_LINE\_PARAM\_MULTIPLE} flag.


\membersection{wxCmdLineParser::GetParam}\label{wxcmdlineparsergetparam}

\constfunc{wxString}{GetParam}{\param{size\_t }{n = 0u}}

Returns the value of Nth parameter (as string only).

\wxheading{See also}

\helpref{GetParamCount}{wxcmdlineparsergetparamcount}

