\section{\class{wxString}}\label{wxstring}

wxString is a class representing a character string. Please see the 
\helpref{wxString overview}{wxstringoverview} for more information about it. As explained
there, wxString implements about 90\% of methods of the std::string class (iterators
are not supported, nor all methods which use them).
These standard functions are not documented in this manual so please see the STL documentation.
The behaviour of all these functions is identical to the behaviour described
there (except that wxString is sensitive to null character).

You may notice that wxString sometimes has many functions which do the same
thing like, for example, \helpref{Length()}{wxstringlength}, 
\helpref{Len()}{wxstringlen} and {\tt length()} which all return the string
length. In all cases of such duplication the {\tt std::string}-compatible
method ({\tt length()} in this case, always the lowercase version) should be
used as it will ensure smoother transition to {\tt std::string} when wxWidgets
starts using it instead of wxString.

Also please note that in this manual \texttt{char} is sometimes used instead of 
\texttt{wxChar} because it hasn't been fully updated yet. Please substitute as
necessary and refer to the sources in case of a doubt.


\wxheading{Derived from}

None

\wxheading{Include files}

<wx/string.h>

\wxheading{Predefined objects}

Objects:

{\bf wxEmptyString}

\wxheading{See also}

\overview{Overview}{wxstringoverview}

\latexignore{\rtfignore{\wxheading{Function groups}}}

\membersection{Constructors and assignment operators}\label{constructorsinwxstring}

A string may be constructed either from a C string, (some number of copies of)
a single character or a wide (UNICODE) string. For all constructors (except the
default which creates an empty string) there is also a corresponding assignment
operator.

\helpref{wxString}{wxstringconstruct}\\
\helpref{operator $=$}{wxstringoperatorassign}\\
\helpref{\destruct{wxString}}{wxstringdestruct}

\membersection{String length}\label{lengthfunctionsinwxstring}

These functions return the string length and check whether the string is empty
or empty it.

\helpref{Len}{wxstringlen}\\
\helpref{IsEmpty}{wxstringisempty}\\
\helpref{operator!}{wxstringoperatornot}\\
\helpref{Empty}{wxstringempty}\\
\helpref{Clear}{wxstringclear}

\membersection{Character access}\label{characteraccessinwxstring}

Many functions in this section take a character index in the string. As with C
strings and/or arrays, the indices start from $0$, so the first character of a
string is string[$0$]. Attempt to access a character beyond the end of the
string (which may be even $0$ if the string is empty) will provoke an assert
failure in \helpref{debug build}{debuggingoverview}, but no checks are done in
release builds.

This section also contains both implicit and explicit conversions to C style
strings. Although implicit conversion is quite convenient, it is advised to use
explicit \helpref{c\_str()}{wxstringcstr} method for the sake of clarity. Also
see \helpref{overview}{wxstringadvices} for the cases where it is necessary to
use it.

\helpref{GetChar}{wxstringgetchar}\\
\helpref{GetWritableChar}{wxstringgetwritablechar}\\
\helpref{SetChar}{wxstringsetchar}\\
\helpref{Last}{wxstringlast}\\
\helpref{operator []}{wxstringoperatorbracket}\\
\helpref{c\_str}{wxstringcstr}\\
\helpref{mb\_str}{wxstringmbstr}\\
\helpref{wc\_str}{wxstringwcstr}\\
\helpref{fn\_str}{wxstringfnstr}\\
\helpref{operator const char*}{wxstringoperatorconstcharpt}

\membersection{Concatenation}\label{concatenationinwxstring}

Anything may be concatenated (appended to) with a string. However, you can't
append something to a C string (including literal constants), so to do this it
should be converted to a wxString first.

\helpref{operator \cinsert}{wxstringoperatorout}\\
\helpref{operator $+=$}{wxstringplusequal}\\
\helpref{operator $+$}{wxstringoperatorplus}\\
\helpref{Append}{wxstringappend}\\
\helpref{Prepend}{wxstringprepend}

\membersection{Comparison}\label{comparisoninwxstring}

The default comparison function \helpref{Cmp}{wxstringcmp} is case-sensitive and
so is the default version of \helpref{IsSameAs}{wxstringissameas}. For case
insensitive comparisons you should use \helpref{CmpNoCase}{wxstringcmpnocase} or
give a second parameter to IsSameAs. This last function is may be more
convenient if only equality of the strings matters because it returns a boolean
true value if the strings are the same and not 0 (which is usually false in C)
as {\tt Cmp()} does.

\helpref{Matches}{wxstringmatches} is a poor man's regular expression matcher:
it only understands '*' and '?' metacharacters in the sense of DOS command line
interpreter.

\helpref{StartsWith}{wxstringstartswith} is helpful when parsing a line of
text which should start with some predefined prefix and is more efficient than
doing direct string comparison as you would also have to precalculate the
length of the prefix then.

\helpref{Cmp}{wxstringcmp}\\
\helpref{CmpNoCase}{wxstringcmpnocase}\\
\helpref{IsSameAs}{wxstringissameas}\\
\helpref{Matches}{wxstringmatches}\\
\helpref{StartsWith}{wxstringstartswith}

\membersection{Substring extraction}\label{substringextractioninwxstring}

These functions allow to extract substring from this string. All of them don't
modify the original string and return a new string containing the extracted
substring.

\helpref{Mid}{wxstringmid}\\
\helpref{operator()}{wxstringoperatorparenth}\\
\helpref{Left}{wxstringleft}\\
\helpref{Right}{wxstringright}\\
\helpref{BeforeFirst}{wxstringbeforefirst}\\
\helpref{BeforeLast}{wxstringbeforelast}\\
\helpref{AfterFirst}{wxstringafterfirst}\\
\helpref{AfterLast}{wxstringafterlast}\\
\helpref{StartsWith}{wxstringstartswith}

\membersection{Case conversion}\label{caseconversioninwxstring}

The MakeXXX() variants modify the string in place, while the other functions
return a new string which contains the original text converted to the upper or
lower case and leave the original string unchanged.

\helpref{MakeUpper}{wxstringmakeupper}\\
\helpref{Upper}{wxstringupper}\\
\helpref{MakeLower}{wxstringmakelower}\\
\helpref{Lower}{wxstringlower}

\membersection{Searching and replacing}\label{searchingandreplacinginwxstring}

These functions replace the standard {\it strchr()} and {\it strstr()} 
functions.

\helpref{Find}{wxstringfind}\\
\helpref{Replace}{wxstringreplace}

\membersection{Conversion to numbers}\label{conversiontonumbersinwxstring}

The string provides functions for conversion to signed and unsigned integer and
floating point numbers. All three functions take a pointer to the variable to
put the numeric value in and return true if the {\bf entire} string could be
converted to a number.

\helpref{ToLong}{wxstringtolong}\\
\helpref{ToULong}{wxstringtoulong}\\
\helpref{ToDouble}{wxstringtodouble}

\membersection{Writing values into the string}\label{writingintostringinwxstring}

Both formatted versions (\helpref{Printf}{wxstringprintf}) and stream-like
insertion operators exist (for basic types only). Additionally, the 
\helpref{Format}{wxstringformat} function allows to use simply append
formatted value to a string:

\begin{verbatim}
    // the following 2 snippets are equivalent

    wxString s = "...";
    s += wxString::Format("%d", n);

    wxString s;
    s.Printf("...%d", n);
\end{verbatim}

\helpref{Format}{wxstringformat}\\
\helpref{FormatV}{wxstringformatv}\\
\helpref{Printf}{wxstringprintf}\\
\helpref{PrintfV}{wxstringprintfv}\\
\helpref{operator \cinsert}{wxstringoperatorout}

\membersection{Memory management}\label{memoryinwxstring}

These are "advanced" functions and they will be needed quite rarely. 
\helpref{Alloc}{wxstringalloc} and \helpref{Shrink}{wxstringshrink} are only
interesting for optimization purposes. 
\helpref{GetWriteBuf}{wxstringgetwritebuf} may be very useful when working with
some external API which requires the caller to provide a writable buffer, but
extreme care should be taken when using it: before performing any other
operation on the string \helpref{UngetWriteBuf}{wxstringungetwritebuf} {\bf
must} be called!

\helpref{Alloc}{wxstringalloc}\\
\helpref{Shrink}{wxstringshrink}\\
\helpref{GetWriteBuf}{wxstringgetwritebuf}\\
\helpref{UngetWriteBuf}{wxstringungetwritebuf}

\membersection{Miscellaneous}\label{miscellaneousinwxstring}

Other string functions.

\helpref{Trim}{wxstringtrim}\\
\helpref{Pad}{wxstringpad}\\
\helpref{Truncate}{wxstringtruncate}

\membersection{wxWidgets 1.xx compatibility functions}\label{backwardcompatibilityinwxstring}

These functions are deprecated, please consider using new wxWidgets 2.0
functions instead of them (or, even better, std::string compatible variants).

\helpref{SubString}{wxstringsubstring}\\
\helpref{sprintf}{wxstringsprintf}\\
\helpref{CompareTo}{wxstringcompareto}\\
\helpref{Length}{wxstringlength}\\
\helpref{Freq}{wxstringfreq}\\
\helpref{LowerCase}{wxstringlowercase}\\
\helpref{UpperCase}{wxstringuppercase}\\
\helpref{Strip}{wxstringstrip}\\
\helpref{Index}{wxstringindex}\\
\helpref{Remove}{wxstringremove}\\
\helpref{First}{wxstringfirst}\\
\helpref{Last}{wxstringlast}\\
\helpref{Contains}{wxstringcontains}\\
\helpref{IsNull}{wxstringisnull}\\
\helpref{IsAscii}{wxstringisascii}\\
\helpref{IsNumber}{wxstringisnumber}\\
\helpref{IsWord}{wxstringisword}

\membersection{std::string compatibility functions}\label{wxstringat}

The supported functions are only listed here, please see any STL reference for
their documentation.

\begin{verbatim}
    // take nLen chars starting at nPos
  wxString(const wxString& str, size_t nPos, size_t nLen);
    // take all characters from pStart to pEnd (poor man's iterators)
  wxString(const void *pStart, const void *pEnd);

  // lib.string.capacity
    // return the length of the string
  size_t size() const;
    // return the length of the string
  size_t length() const;
    // return the maximum size of the string
  size_t max_size() const;
    // resize the string, filling the space with c if c != 0
  void resize(size_t nSize, char ch = '\0');
    // delete the contents of the string
  void clear();
    // returns true if the string is empty
  bool empty() const;

  // lib.string.access
    // return the character at position n
  char at(size_t n) const;
    // returns the writable character at position n
  char& at(size_t n);

  // lib.string.modifiers
    // append a string
  wxString& append(const wxString& str);
    // append elements str[pos], ..., str[pos+n]
  wxString& append(const wxString& str, size_t pos, size_t n);
    // append first n (or all if n == npos) characters of sz
  wxString& append(const char *sz, size_t n = npos);

    // append n copies of ch
  wxString& append(size_t n, char ch);

    // same as `this_string = str'
  wxString& assign(const wxString& str);
    // same as ` = str[pos..pos + n]
  wxString& assign(const wxString& str, size_t pos, size_t n);
    // same as `= first n (or all if n == npos) characters of sz'
  wxString& assign(const char *sz, size_t n = npos);
    // same as `= n copies of ch'
  wxString& assign(size_t n, char ch);

    // insert another string
  wxString& insert(size_t nPos, const wxString& str);
    // insert n chars of str starting at nStart (in str)
  wxString& insert(size_t nPos, const wxString& str, size_t nStart, size_t n);

    // insert first n (or all if n == npos) characters of sz
  wxString& insert(size_t nPos, const char *sz, size_t n = npos);
    // insert n copies of ch
  wxString& insert(size_t nPos, size_t n, char ch);

    // delete characters from nStart to nStart + nLen
  wxString& erase(size_t nStart = 0, size_t nLen = npos);

    // replaces the substring of length nLen starting at nStart
  wxString& replace(size_t nStart, size_t nLen, const char* sz);
    // replaces the substring with nCount copies of ch
  wxString& replace(size_t nStart, size_t nLen, size_t nCount, char ch);
    // replaces a substring with another substring
  wxString& replace(size_t nStart, size_t nLen,
                    const wxString& str, size_t nStart2, size_t nLen2);
    // replaces the substring with first nCount chars of sz
    wxString& replace(size_t nStart, size_t nLen,
                      const char* sz, size_t nCount);

    // swap two strings
  void swap(wxString& str);

    // All find() functions take the nStart argument which specifies the
    // position to start the search on, the default value is 0. All functions
    // return npos if there were no match.

    // find a substring
  size_t find(const wxString& str, size_t nStart = 0) const;

    // find first n characters of sz
  size_t find(const char* sz, size_t nStart = 0, size_t n = npos) const;

    // find the first occurrence of character ch after nStart
  size_t find(char ch, size_t nStart = 0) const;

    // rfind() family is exactly like find() but works right to left

    // as find, but from the end
  size_t rfind(const wxString& str, size_t nStart = npos) const;

    // as find, but from the end
  size_t rfind(const char* sz, size_t nStart = npos,
          size_t n = npos) const;
    // as find, but from the end
  size_t rfind(char ch, size_t nStart = npos) const;

    // find first/last occurrence of any character in the set

    //
  size_t find_first_of(const wxString& str, size_t nStart = 0) const;
    //
  size_t find_first_of(const char* sz, size_t nStart = 0) const;
    // same as find(char, size_t)
  size_t find_first_of(char c, size_t nStart = 0) const;
    //
  size_t find_last_of (const wxString& str, size_t nStart = npos) const;
    //
  size_t find_last_of (const char* s, size_t nStart = npos) const;
    // same as rfind(char, size_t)
  size_t find_last_of (char c, size_t nStart = npos) const;

    // find first/last occurrence of any character not in the set

    //
  size_t find_first_not_of(const wxString& str, size_t nStart = 0) const;
    //
  size_t find_first_not_of(const char* s, size_t nStart = 0) const;
    //
  size_t find_first_not_of(char ch, size_t nStart = 0) const;
    //
  size_t find_last_not_of(const wxString& str, size_t nStart=npos) const;
    //
  size_t find_last_not_of(const char* s, size_t nStart = npos) const;
    //
  size_t find_last_not_of(char ch, size_t nStart = npos) const;

    // All compare functions return a negative, zero or positive value
    // if the [sub]string is less, equal or greater than the compare() argument.

    // just like strcmp()
  int compare(const wxString& str) const;
    // comparison with a substring
  int compare(size_t nStart, size_t nLen, const wxString& str) const;
    // comparison of 2 substrings
  int compare(size_t nStart, size_t nLen,
              const wxString& str, size_t nStart2, size_t nLen2) const;
    // just like strcmp()
  int compare(const char* sz) const;
    // substring comparison with first nCount characters of sz
  int compare(size_t nStart, size_t nLen,
              const char* sz, size_t nCount = npos) const;

  // substring extraction
  wxString substr(size_t nStart = 0, size_t nLen = npos) const;
\end{verbatim}

%%%%% MEMBERS HERE %%%%%
\helponly{\insertatlevel{2}{

\wxheading{Members}

}}

\membersection{wxString::wxString}\label{wxstringconstruct}

\func{}{wxString}{\void}

Default constructor. Initializes the string to {\tt ""} (empty string).

\func{}{wxString}{\param{const wxString\&}{ x}}

Copy constructor.

\func{}{wxString}{\param{char}{ ch}, \param{size\_t}{ n = 1}}

Constructs a string of {\it n} copies of character {\it ch}.

\func{}{wxString}{\param{const char*}{ psz}, \param{size\_t}{ nLength = wxSTRING\_MAXLEN}}

Takes first {\it nLength} characters from the C string {\it psz}.
The default value of {\tt wxSTRING\_MAXLEN} means to take all the string.

Note that this constructor may be used even if {\it psz} points to a buffer
with binary data (i.e. containing {\tt NUL} characters) as long as you provide
the correct value for {\it nLength}. However, the default form of it works
only with strings without intermediate {\tt NUL}s because it uses 
{\tt strlen()} to calculate the effective length and it would not give correct
results otherwise.

\func{}{wxString}{\param{const unsigned char*}{ psz}, \param{size\_t}{ nLength = wxSTRING\_MAXLEN}}

For compilers using unsigned char: takes first {\it nLength} characters from the C string {\it psz}.
The default value of {\tt wxSTRING\_MAXLEN} means take all the string.

{\bf Note:} In Unicode build, all of the above constructors take
{\tt wchar\_t} arguments instead of {\tt char}.

\wxheading{Constructors with conversion}

The following constructors allow you to construct wxString from wide string
in ANSI build or from C string in Unicode build.

\func{}{wxString}{\param{const wchar\_t*}{ psz}, \param{wxMBConv\&}{ conv}, \param{size\_t}{ nLength = wxSTRING\_MAXLEN}}

Initializes the string from first \arg{nLength} characters of wide string. 
The default value of {\tt wxSTRING\_MAXLEN} means take all the string.
In ANSI build, \arg{conv}'s 
\helpref{WC2MB}{wxmbconvwc2mb} method is called to
convert \arg{psz} to wide string. It is ignored in Unicode build.

\func{}{wxString}{\param{const char*}{ psz}, \param{wxMBConv\&}{ conv}, \param{size\_t}{ nLength = wxSTRING\_MAXLEN}}

Initializes the string from first \arg{nLength} characters of C string.
The default value of {\tt wxSTRING\_MAXLEN} means take all the string.
In Unicode build, \arg{conv}'s 
\helpref{MB2WC}{wxmbconvmb2wc} method is called to
convert \arg{psz} to wide string. It is ignored in ANSI build.

\wxheading{See also}

\helpref{wxMBConv classes}{mbconvclasses}, \helpref{mb\_str}{wxstringmbstr},
\helpref{wc\_str}{wxstringwcstr}

\membersection{wxString::\destruct{wxString}}\label{wxstringdestruct}

\func{}{\destruct{wxString}}{\void}

String destructor. Note that this is not virtual, so wxString must not be inherited from.

\membersection{wxString::Alloc}\label{wxstringalloc}

\func{void}{Alloc}{\param{size\_t}{ nLen}}

Preallocate enough space for wxString to store {\it nLen} characters. This function
may be used to increase speed when the string is constructed by repeated
concatenation as in

\begin{verbatim}

// delete all vowels from the string
wxString DeleteAllVowels(const wxString& original)
{
    wxString result;

    size_t len = original.length();

    result.Alloc(len);

    for ( size_t n = 0; n < len; n++ )
    {
        if ( strchr("aeuio", tolower(original[n])) == NULL )
            result += original[n];
    }

    return result;
}

\end{verbatim}

because it will avoid the need to reallocate string memory many times (in case
of long strings). Note that it does not set the maximal length of a string - it
will still expand if more than {\it nLen} characters are stored in it. Also, it
does not truncate the existing string (use 
\helpref{Truncate()}{wxstringtruncate} for this) even if its current length is
greater than {\it nLen}

\membersection{wxString::Append}\label{wxstringappend}

\func{wxString\&}{Append}{\param{const char*}{ psz}}

Concatenates {\it psz} to this string, returning a reference to it.

\func{wxString\&}{Append}{\param{char}{ ch}, \param{int}{ count = 1}}

Concatenates character {\it ch} to this string, {\it count} times, returning a reference
to it.

\membersection{wxString::AfterFirst}\label{wxstringafterfirst}

\constfunc{wxString}{AfterFirst}{\param{char}{ ch}}

Gets all the characters after the first occurrence of {\it ch}.
Returns the empty string if {\it ch} is not found.

\membersection{wxString::AfterLast}\label{wxstringafterlast}

\constfunc{wxString}{AfterLast}{\param{char}{ ch}}

Gets all the characters after the last occurrence of {\it ch}.
Returns the whole string if {\it ch} is not found.

\membersection{wxString::BeforeFirst}\label{wxstringbeforefirst}

\constfunc{wxString}{BeforeFirst}{\param{char}{ ch}}

Gets all characters before the first occurrence of {\it ch}.
Returns the whole string if {\it ch} is not found.

\membersection{wxString::BeforeLast}\label{wxstringbeforelast}

\constfunc{wxString}{BeforeLast}{\param{char}{ ch}}

Gets all characters before the last occurrence of {\it ch}.
Returns the empty string if {\it ch} is not found.

\membersection{wxString::c\_str}\label{wxstringcstr}

\constfunc{const wxChar *}{c\_str}{\void}

Returns a pointer to the string data ({\tt const char*} in ANSI build,
{\tt const wchar\_t*} in Unicode build).

\wxheading{See also}

\helpref{mb\_str}{wxstringmbstr}, \helpref{wc\_str}{wxstringwcstr},
\helpref{fn\_str}{wxstringfnstr}

\membersection{wxString::Clear}\label{wxstringclear}

\func{void}{Clear}{\void}

Empties the string and frees memory occupied by it.

See also: \helpref{Empty}{wxstringempty}

\membersection{wxString::Cmp}\label{wxstringcmp}

\constfunc{int}{Cmp}{\param{const wxString\&}{ s}}

\constfunc{int}{Cmp}{\param{const char*}{ psz}}

Case-sensitive comparison.

Returns a positive value if the string is greater than the argument, zero if
it is equal to it or a negative value if it is less than the argument (same semantics
as the standard {\it strcmp()} function).

See also \helpref{CmpNoCase}{wxstringcmpnocase}, \helpref{IsSameAs}{wxstringissameas}.

\membersection{wxString::CmpNoCase}\label{wxstringcmpnocase}

\constfunc{int}{CmpNoCase}{\param{const wxString\&}{ s}}

\constfunc{int}{CmpNoCase}{\param{const char*}{ psz}}

Case-insensitive comparison.

Returns a positive value if the string is greater than the argument, zero if
it is equal to it or a negative value if it is less than the argument (same semantics
as the standard {\it strcmp()} function).

See also \helpref{Cmp}{wxstringcmp}, \helpref{IsSameAs}{wxstringissameas}.

\membersection{wxString::CompareTo}\label{wxstringcompareto}

\begin{verbatim}
#define NO_POS ((int)(-1)) // undefined position
enum caseCompare {exact, ignoreCase};
\end{verbatim}

\constfunc{int}{CompareTo}{\param{const char*}{ psz}, \param{caseCompare}{ cmp = exact}}

Case-sensitive comparison. Returns 0 if equal, 1 if greater or -1 if less.

\membersection{wxString::Contains}\label{wxstringcontains}

\constfunc{bool}{Contains}{\param{const wxString\&}{ str}}

Returns 1 if target appears anywhere in wxString; else 0.

\membersection{wxString::Empty}\label{wxstringempty}

\func{void}{Empty}{\void}

Makes the string empty, but doesn't free memory occupied by the string.

See also: \helpref{Clear()}{wxstringclear}.

\membersection{wxString::Find}\label{wxstringfind}

\constfunc{int}{Find}{\param{char}{ ch}, \param{bool}{ fromEnd = false}}

Searches for the given character. Returns the starting index, or -1 if not found.

\constfunc{int}{Find}{\param{const char*}{ sz}}

Searches for the given string. Returns the starting index, or -1 if not found.

\membersection{wxString::First}\label{wxstringfirst}

\func{int}{First}{\param{char}{ c}}

\constfunc{int}{First}{\param{const char*}{ psz}}

\constfunc{int}{First}{\param{const wxString\&}{ str}}

Same as \helpref{Find}{wxstringfind}.

\membersection{wxString::fn\_str}\label{wxstringfnstr}

\constfunc{const wchar\_t*}{fn\_str}{\void}

\constfunc{const char*}{fn\_str}{\void}

\constfunc{const wxCharBuffer}{fn\_str}{\void}

Returns string representation suitable for passing to OS' functions for
file handling. In ANSI build, this is same as \helpref{c\_str}{wxstringcstr}.
In Unicode build, returned value can be either wide character string
or C string in charset matching the {\tt wxConvFileName} object, depending on
the OS.

\wxheading{See also}

\helpref{wxMBConv}{wxmbconv},
\helpref{wc\_str}{wxstringwcstr}, \helpref{mb\_str}{wxstringwcstr}

\membersection{wxString::Format}\label{wxstringformat}

\func{static wxString}{Format}{\param{const wxChar }{*format}, \param{}{...}}

This static function returns the string containing the result of calling 
\helpref{Printf}{wxstringprintf} with the passed parameters on it.

\wxheading{See also}

\helpref{FormatV}{wxstringformatv}, \helpref{Printf}{wxstringprintf}

\membersection{wxString::FormatV}\label{wxstringformatv}

\func{static wxString}{FormatV}{\param{const wxChar }{*format}, \param{va\_list }{argptr}}

This static function returns the string containing the result of calling 
\helpref{PrintfV}{wxstringprintfv} with the passed parameters on it.

\wxheading{See also}

\helpref{Format}{wxstringformat}, \helpref{PrintfV}{wxstringprintfv}

\membersection{wxString::Freq}\label{wxstringfreq}

\constfunc{int}{Freq}{\param{char }{ch}}

Returns the number of occurrences of {\it ch} in the string.

\membersection{wxString::FromAscii}\label{wxstringfromascii}

\func{static wxString }{FromAscii}{\param{const char*}{ s}}

\func{static wxString }{FromAscii}{\param{const char}{ c}}

Converts the string or character from an ASCII, 7-bit form
to the native wxString representation. Most useful when using
a Unicode build of wxWidgets.
Use \helpref{wxString constructors}{wxstringconstruct} if you
need to convert from another charset.

\membersection{wxString::GetChar}\label{wxstringgetchar}

\constfunc{char}{GetChar}{\param{size\_t}{ n}}

Returns the character at position {\it n} (read-only).

\membersection{wxString::GetData}\label{wxstringgetdata}

\constfunc{const wxChar*}{GetData}{\void}

wxWidgets compatibility conversion. Returns a constant pointer to the data in the string.

\membersection{wxString::GetWritableChar}\label{wxstringgetwritablechar}

\func{char\&}{GetWritableChar}{\param{size\_t}{ n}}

Returns a reference to the character at position {\it n}.

\membersection{wxString::GetWriteBuf}\label{wxstringgetwritebuf}

\func{wxChar*}{GetWriteBuf}{\param{size\_t}{ len}}

Returns a writable buffer of at least {\it len} bytes.
It returns a pointer to a new memory block, and the
existing data will not be copied.

Call \helpref{wxString::UngetWriteBuf}{wxstringungetwritebuf} as soon as possible
to put the string back into a reasonable state.

\membersection{wxString::Index}\label{wxstringindex}

\constfunc{size\_t}{Index}{\param{char}{ ch}}

\constfunc{size\_t}{Index}{\param{const char*}{ sz}}

Same as \helpref{wxString::Find}{wxstringfind}.

% TODO
%\membersection{wxString::insert}\label{wxstringinsert}
% Wrong!
%\func{void}{insert}{\param{const wxString\&}{ str}, \param{size\_t}{ index}}
%
%Add new element at the given position.
%
\membersection{wxString::IsAscii}\label{wxstringisascii}

\constfunc{bool}{IsAscii}{\void}

Returns true if the string contains only ASCII characters.

\membersection{wxString::IsEmpty}\label{wxstringisempty}

\constfunc{bool}{IsEmpty}{\void}

Returns true if the string is empty.

\membersection{wxString::IsNull}\label{wxstringisnull}

\constfunc{bool}{IsNull}{\void}

Returns true if the string is empty (same as \helpref{IsEmpty}{wxstringisempty}).

\membersection{wxString::IsNumber}\label{wxstringisnumber}

\constfunc{bool}{IsNumber}{\void}

Returns true if the string is an integer (with possible sign).

\membersection{wxString::IsSameAs}\label{wxstringissameas}

\constfunc{bool}{IsSameAs}{\param{const char*}{ psz}, \param{bool}{ caseSensitive = true}}

Test for string equality, case-sensitive (default) or not.

caseSensitive is true by default (case matters).

Returns true if strings are equal, false otherwise.

See also \helpref{Cmp}{wxstringcmp}, \helpref{CmpNoCase}{wxstringcmpnocase}

\constfunc{bool}{IsSameAs}{\param{char}{ c}, \param{bool}{ caseSensitive = true}}

Test whether the string is equal to the single character {\it c}. The test is
case-sensitive if {\it caseSensitive} is true (default) or not if it is false.

Returns true if the string is equal to the character, false otherwise.

See also \helpref{Cmp}{wxstringcmp}, \helpref{CmpNoCase}{wxstringcmpnocase}

\membersection{wxString::IsWord}\label{wxstringisword}

\constfunc{bool}{IsWord}{\void}

Returns true if the string is a word. TODO: what's the definition of a word?

\membersection{wxString::Last}\label{wxstringlast}

\constfunc{char}{Last}{\void}

Returns the last character.

\func{char\&}{Last}{\void}

Returns a reference to the last character (writable).

\membersection{wxString::Left}\label{wxstringleft}

\constfunc{wxString}{Left}{\param{size\_t}{ count}}

Returns the first {\it count} characters of the string.

\membersection{wxString::Len}\label{wxstringlen}

\constfunc{size\_t}{Len}{\void}

Returns the length of the string.

\membersection{wxString::Length}\label{wxstringlength}

\constfunc{size\_t}{Length}{\void}

Returns the length of the string (same as Len).

\membersection{wxString::Lower}\label{wxstringlower}

\constfunc{wxString}{Lower}{\void}

Returns this string converted to the lower case.

\membersection{wxString::LowerCase}\label{wxstringlowercase}

\func{void}{LowerCase}{\void}

Same as MakeLower.

\membersection{wxString::MakeLower}\label{wxstringmakelower}

\func{wxString\&}{MakeLower}{\void}

Converts all characters to lower case and returns the result.

\membersection{wxString::MakeUpper}\label{wxstringmakeupper}

\func{wxString\&}{MakeUpper}{\void}

Converts all characters to upper case and returns the result.

\membersection{wxString::Matches}\label{wxstringmatches}

\constfunc{bool}{Matches}{\param{const char*}{ szMask}}

Returns true if the string contents matches a mask containing '*' and '?'.

\membersection{wxString::mb\_str}\label{wxstringmbstr}

\constfunc{const char*}{mb\_str}{\param{wxMBConv\&}{ conv}}

\constfunc{const wxCharBuffer}{mb\_str}{\param{wxMBConv\&}{ conv}}

Returns multibyte (C string) representation of the string.
In Unicode build, converts using \arg{conv}'s \helpref{cWC2MB}{wxmbconvcwc2mb}
method and returns wxCharBuffer. In ANSI build, this function is same
as \helpref{c\_str}{wxstringcstr}.
The macro wxWX2MBbuf is defined as the correct return type (without const).

\wxheading{See also}

\helpref{wxMBConv}{wxmbconv},
\helpref{c\_str}{wxstringcstr}, \helpref{wc\_str}{wxstringwcstr},
\helpref{fn\_str}{wxstringfnstr}

\membersection{wxString::Mid}\label{wxstringmid}

\constfunc{wxString}{Mid}{\param{size\_t}{ first}, \param{size\_t}{ count = wxSTRING\_MAXLEN}}

Returns a substring starting at {\it first}, with length {\it count}, or the rest of
the string if {\it count} is the default value.

\membersection{wxString::Pad}\label{wxstringpad}

\func{wxString\&}{Pad}{\param{size\_t}{ count}, \param{char}{ pad = ' '}, \param{bool}{ fromRight = true}}

Adds {\it count} copies of {\it pad} to the beginning, or to the end of the string (the default).

Removes spaces from the left or from the right (default).

\membersection{wxString::Prepend}\label{wxstringprepend}

\func{wxString\&}{Prepend}{\param{const wxString\&}{ str}}

Prepends {\it str} to this string, returning a reference to this string.

\membersection{wxString::Printf}\label{wxstringprintf}

\func{int}{Printf}{\param{const char* }{pszFormat}, \param{}{...}}

Similar to the standard function {\it sprintf()}. Returns the number of
characters written, or an integer less than zero on error.

{\bf NB:} This function will use a safe version of {\it vsprintf()} (usually called 
{\it vsnprintf()}) whenever available to always allocate the buffer of correct
size. Unfortunately, this function is not available on all platforms and the
dangerous {\it vsprintf()} will be used then which may lead to buffer overflows.

\membersection{wxString::PrintfV}\label{wxstringprintfv}

\func{int}{PrintfV}{\param{const char* }{pszFormat}, \param{va\_list}{ argPtr}}

Similar to vprintf. Returns the number of characters written, or an integer less than zero
on error.

\membersection{wxString::Remove}\label{wxstringremove}

\func{wxString\&}{Remove}{\param{size\_t}{ pos}}

Same as Truncate. Removes the portion from {\it pos} to the end of the string.

\func{wxString\&}{Remove}{\param{size\_t}{ pos}, \param{size\_t}{ len}}

Removes {\it len} characters from the string, starting at {\it pos}.

\membersection{wxString::RemoveLast}\label{wxstringremovelast}

\func{wxString\&}{RemoveLast}{\void}

Removes the last character.

\membersection{wxString::Replace}\label{wxstringreplace}

\func{size\_t}{Replace}{\param{const char*}{ szOld}, \param{const char*}{ szNew}, \param{bool}{ replaceAll = true}}

Replace first (or all) occurrences of substring with another one.

{\it replaceAll}: global replace (default), or only the first occurrence.

Returns the number of replacements made.

\membersection{wxString::Right}\label{wxstringright}

\constfunc{wxString}{Right}{\param{size\_t}{ count}}

Returns the last {\it count} characters.

\membersection{wxString::SetChar}\label{wxstringsetchar}

\func{void}{SetChar}{\param{size\_t}{ n}, \param{char}{ch}}

Sets the character at position {\it n}.

\membersection{wxString::Shrink}\label{wxstringshrink}

\func{void}{Shrink}{\void}

Minimizes the string's memory. This can be useful after a call to 
\helpref{Alloc()}{wxstringalloc} if too much memory were preallocated.

\membersection{wxString::sprintf}\label{wxstringsprintf}

\func{void}{sprintf}{\param{const char* }{ fmt}}

The same as Printf.

\membersection{wxString::StartsWith}\label{wxstringstartswith}

\constfunc{bool}{StartsWith}{\param{const wxChar }{*prefix}, \param{wxString }{*rest = NULL}}

This function can be used to test if the string starts with the specified 
{\it prefix}. If it does, the function will return {\tt true} and put the rest
of the string (i.e. after the prefix) into {\it rest} string if it is not 
{\tt NULL}. Otherwise, the function returns {\tt false} and doesn't modify the 
{\it rest}.

\membersection{wxString::Strip}\label{wxstringstrip}

\begin{verbatim}
enum stripType {leading = 0x1, trailing = 0x2, both = 0x3};
\end{verbatim}

\constfunc{wxString}{Strip}{\param{stripType}{ s = trailing}}

Strip characters at the front and/or end. The same as Trim except that it
doesn't change this string.

\membersection{wxString::SubString}\label{wxstringsubstring}

\constfunc{wxString}{SubString}{\param{size\_t}{ from}, \param{size\_t}{ to}}

Deprecated, use \helpref{Mid}{wxstringmid} instead (but note that parameters
have different meaning).

Returns the part of the string between the indices {\it from} and {\it to}
inclusive.

\membersection{wxString::ToAscii}\label{wxstringtoascii}

\constfunc{const char*}{ToAscii}{\void}

Converts the string to an ASCII, 7-bit string (ANSI builds only).

\constfunc{const wxCharBuffer}{ToAscii}{\void}

Converts the string to an ASCII, 7-bit string in the form of
a wxCharBuffer (Unicode builds only).

Note that this conversion only works if the string contains only ASCII
characters. The \helpref{mb\_str}{wxstringmbstr} method provides more
powerful means of converting wxString to C string.

\membersection{wxString::ToDouble}\label{wxstringtodouble}

\constfunc{bool}{ToDouble}{\param{double}{ *val}}

Attempts to convert the string to a floating point number. Returns true on
success (the number is stored in the location pointed to by {\it val}) or false
if the string does not represent such number.

\wxheading{See also}

\helpref{wxString::ToLong}{wxstringtolong},\\
\helpref{wxString::ToULong}{wxstringtoulong}

\membersection{wxString::ToLong}\label{wxstringtolong}

\constfunc{bool}{ToLong}{\param{long}{ *val}, \param{int }{base = $10$}}

Attempts to convert the string to a signed integer in base {\it base}. Returns
{\tt true} on success in which case the number is stored in the location
pointed to by {\it val} or {\tt false} if the string does not represent a
valid number in the given base.

The value of {\it base} must be comprised between $2$ and $36$, inclusive, or
be a special value $0$ which means that the usual rules of {\tt C} numbers are
applied: if the number starts with {\tt 0x} it is considered to be in base
$16$, if it starts with {\tt 0} - in base $8$ and in base $10$ otherwise. Note
that you may not want to specify the base $0$ if you are parsing the numbers
which may have leading zeroes as they can yield unexpected (to the user not
familiar with C) results.

\wxheading{See also}

\helpref{wxString::ToDouble}{wxstringtodouble},\\
\helpref{wxString::ToULong}{wxstringtoulong}

\membersection{wxString::ToULong}\label{wxstringtoulong}

\constfunc{bool}{ToULong}{\param{unsigned long}{ *val}, \param{int }{base = $10$}}

Attempts to convert the string to an unsigned integer in base {\it base}.
Returns {\tt true} on success in which case the number is stored in the
location pointed to by {\it val} or {\tt false} if the string does not
represent a valid number in the given base. Please notice that this function
behaves in the same way as the standard \texttt{strtoul()} and so it simply
converts negative numbers to unsigned representation instead of rejecting them
(e.g. $-1$ is returned as \texttt{ULONG\_MAX}).

See \helpref{wxString::ToLong}{wxstringtolong} for the more detailed
description of the {\it base} parameter.

\wxheading{See also}

\helpref{wxString::ToDouble}{wxstringtodouble},\\
\helpref{wxString::ToLong}{wxstringtolong}

\membersection{wxString::Trim}\label{wxstringtrim}

\func{wxString\&}{Trim}{\param{bool}{ fromRight = true}}

Removes spaces from the left or from the right (default).

\membersection{wxString::Truncate}\label{wxstringtruncate}

\func{wxString\&}{Truncate}{\param{size\_t}{ len}}

Truncate the string to the given length.

\membersection{wxString::UngetWriteBuf}\label{wxstringungetwritebuf}

\func{void}{UngetWriteBuf}{\void}

\func{void}{UngetWriteBuf}{\param{size\_t }{len}}

Puts the string back into a reasonable state (in which it can be used
normally), after
\rtfsp\helpref{wxString::GetWriteBuf}{wxstringgetwritebuf} was called.

The version of the function without the {\it len} parameter will calculate the
new string length itself assuming that the string is terminated by the first
{\tt NUL} character in it while the second one will use the specified length
and thus is the only version which should be used with the strings with
embedded {\tt NUL}s (it is also slightly more efficient as {\tt strlen()} 
doesn't have to be called).

\membersection{wxString::Upper}\label{wxstringupper}

\constfunc{wxString}{Upper}{\void}

Returns this string converted to upper case.

\membersection{wxString::UpperCase}\label{wxstringuppercase}

\func{void}{UpperCase}{\void}

The same as MakeUpper.

\membersection{wxString::wc\_str}\label{wxstringwcstr}

\constfunc{const wchar\_t*}{wc\_str}{\param{wxMBConv\&}{ conv}}

\constfunc{const wxWCharBuffer}{wc\_str}{\param{wxMBConv\&}{ conv}}

Returns wide character representation of the string.
In ANSI build, converts using \arg{conv}'s \helpref{cMB2WC}{wxmbconvcmb2wc}
method and returns wxWCharBuffer. In Unicode build, this function is same
as \helpref{c\_str}{wxstringcstr}.
The macro wxWX2WCbuf is defined as the correct return type (without const).

\wxheading{See also}

\helpref{wxMBConv}{wxmbconv},
\helpref{c\_str}{wxstringcstr}, \helpref{mb\_str}{wxstringwcstr},
\helpref{fn\_str}{wxstringfnstr}

\membersection{wxString::operator!}\label{wxstringoperatornot}

\constfunc{bool}{operator!}{\void}

Empty string is false, so !string will only return true if the string is empty.
This allows the tests for NULLness of a {\it const char *} pointer and emptiness
of the string to look the same in the code and makes it easier to port old code
to wxString.

See also \helpref{IsEmpty()}{wxstringisempty}.

\membersection{wxString::operator $=$}\label{wxstringoperatorassign}

\func{wxString\&}{operator $=$}{\param{const wxString\&}{ str}}

\func{wxString\&}{operator $=$}{\param{const char*}{ psz}}

\func{wxString\&}{operator $=$}{\param{char}{ c}}

\func{wxString\&}{operator $=$}{\param{const unsigned char*}{ psz}}

\func{wxString\&}{operator $=$}{\param{const wchar\_t*}{ pwz}}

Assignment: the effect of each operation is the same as for the corresponding
constructor (see \helpref{wxString constructors}{wxstringconstruct}).

\membersection{wxString::operator $+$}\label{wxstringoperatorplus}

Concatenation: all these operators return a new string equal to the
concatenation of the operands.

\func{wxString}{operator $+$}{\param{const wxString\&}{ x}, \param{const wxString\&}{ y}}

\func{wxString}{operator $+$}{\param{const wxString\&}{ x}, \param{const char*}{ y}}

\func{wxString}{operator $+$}{\param{const wxString\&}{ x}, \param{char}{ y}}

\func{wxString}{operator $+$}{\param{const char*}{ x}, \param{const wxString\&}{ y}}

\membersection{wxString::operator $+=$}\label{wxstringplusequal}

\func{void}{operator $+=$}{\param{const wxString\&}{ str}}

\func{void}{operator $+=$}{\param{const char*}{ psz}}

\func{void}{operator $+=$}{\param{char}{ c}}

Concatenation in place: the argument is appended to the string.

\membersection{wxString::operator []}\label{wxstringoperatorbracket}

\func{wxChar\&}{operator []}{\param{size\_t}{ i}}

\constfunc{wxChar}{operator []}{\param{size\_t}{ i}}

\func{wxChar\&}{operator []}{\param{int}{ i}}

\constfunc{wxChar}{operator []}{\param{int}{ i}}

Element extraction.

\membersection{wxString::operator ()}\label{wxstringoperatorparenth}

\func{wxString}{operator ()}{\param{size\_t}{ start}, \param{size\_t}{ len}}

Same as Mid (substring extraction).

\membersection{wxString::operator \cinsert}\label{wxstringoperatorout}

\func{wxString\&}{operator \cinsert}{\param{const wxString\&}{ str}}

\func{wxString\&}{operator \cinsert}{\param{const char*}{ psz}}

\func{wxString\&}{operator \cinsert}{\param{char }{ch}}

Same as $+=$.

\func{wxString\&}{operator \cinsert}{\param{int}{ i}}

\func{wxString\&}{operator \cinsert}{\param{float}{ f}}

\func{wxString\&}{operator \cinsert}{\param{double}{ d}}

These functions work as C++ stream insertion operators: they insert the given
value into the string. Precision or format cannot be set using them, you can use 
\helpref{Printf}{wxstringprintf} for this.

\membersection{wxString::operator \cextract}\label{wxstringoperatorin}

\func{friend istream\&}{operator \cextract}{\param{istream\&}{ is}, \param{wxString\&}{ str}}

Extraction from a stream.

\membersection{wxString::operator const char*}\label{wxstringoperatorconstcharpt}

\constfunc{}{operator const char*}{\void}

Implicit conversion to a C string.

\membersection{Comparison operators}\label{wxstringcomparison}

\func{bool}{operator $==$}{\param{const wxString\&}{ x}, \param{const wxString\&}{ y}}

\func{bool}{operator $==$}{\param{const wxString\&}{ x}, \param{const char*}{ t}}

\func{bool}{operator $!=$}{\param{const wxString\&}{ x}, \param{const wxString\&}{ y}}

\func{bool}{operator $!=$}{\param{const wxString\&}{ x}, \param{const char*}{ t}}

\func{bool}{operator $>$}{\param{const wxString\&}{ x}, \param{const wxString\&}{ y}}

\func{bool}{operator $>$}{\param{const wxString\&}{ x}, \param{const char*}{ t}}

\func{bool}{operator $>=$}{\param{const wxString\&}{ x}, \param{const wxString\&}{ y}}

\func{bool}{operator $>=$}{\param{const wxString\&}{ x}, \param{const char*}{ t}}

\func{bool}{operator $<$}{\param{const wxString\&}{ x}, \param{const wxString\&}{ y}}

\func{bool}{operator $<$}{\param{const wxString\&}{ x}, \param{const char*}{ t}}

\func{bool}{operator $<=$}{\param{const wxString\&}{ x}, \param{const wxString\&}{ y}}

\func{bool}{operator $<=$}{\param{const wxString\&}{ x}, \param{const char*}{ t}}

\wxheading{Remarks}

These comparisons are case-sensitive.


\section{\class{wxStringBuffer}}\label{wxstringbuffer}

This tiny class allows to conveniently access the \helpref{wxString}{wxstring} 
internal buffer as a writable pointer without any risk of forgetting to restore
the string to the usable state later.

For example, assuming you have a low-level OS function called 
{\tt GetMeaningOfLifeAsString(char *)} returning the value in the provided
buffer (which must be writable, of course) you might call it like this:

\begin{verbatim}
    wxString theAnswer;
    GetMeaningOfLifeAsString(wxStringBuffer(theAnswer, 1024));
    if ( theAnswer != "42" )
    {
        wxLogError("Something is very wrong!");
    }
\end{verbatim}

Note that the exact usage of this depends on whether on not wxUSE\_STL is enabled.  If
wxUSE\_STL is enabled, wxStringBuffer creates a separate empty character buffer, and
if wxUSE\_STL is disabled, it uses GetWriteBuf() from wxString, keeping the same buffer
wxString uses intact.  In other words, relying on wxStringBuffer containing the old 
wxString data is probably not a good idea if you want to build your program in both
with and without wxUSE\_STL.

\wxheading{Derived from}

None

\wxheading{Include files}

<wx/string.h>

\latexignore{\rtfignore{\wxheading{Members}}}

\membersection{wxStringBuffer::wxStringBuffer}\label{wxstringbufferctor}

\func{}{wxStringBuffer}{\param{const wxString\& }{str}, \param{size\_t }{len}}

Constructs a writable string buffer object associated with the given string
and containing enough space for at least {\it len} characters. Basically, this
is equivalent to calling \helpref{GetWriteBuf}{wxstringgetwritebuf} and
saving the result.

\membersection{wxStringBuffer::\destruct{wxStringBuffer}}\label{wxstringbufferdtor}

\func{}{\destruct{wxStringBuffer}}{\void}

Restores the string passed to the constructor to the usable state by calling 
\helpref{UngetWriteBuf}{wxstringungetwritebuf} on it.

\membersection{wxStringBuffer::operator wxChar *}\label{wxstringbufferwxchar}

\func{wxChar *}{operator wxChar *}{\void}

Returns the writable pointer to a buffer of the size at least equal to the
length specified in the constructor.



\section{\class{wxStringBufferLength}}\label{wxstringbufferlength}

This tiny class allows to conveniently access the \helpref{wxString}{wxstring} 
internal buffer as a writable pointer without any risk of forgetting to restore
the string to the usable state later, and allows the user to set the internal
length of the string.

For example, assuming you have a low-level OS function called 
{\tt int GetMeaningOfLifeAsString(char *)} copying the value in the provided
buffer (which must be writable, of course), and returning the actual length
of the string, you might call it like this:

\begin{verbatim}
    wxString theAnswer;
    wxStringBuffer theAnswerBuffer(theAnswer, 1024);
    int nLength = GetMeaningOfLifeAsString(theAnswerBuffer);
    theAnswerBuffer.SetLength(nLength);
    if ( theAnswer != "42" )
    {
        wxLogError("Something is very wrong!");
    }
\end{verbatim}

Note that the exact usage of this depends on whether on not wxUSE\_STL is enabled.  If
wxUSE\_STL is enabled, wxStringBuffer creates a separate empty character buffer, and
if wxUSE\_STL is disabled, it uses GetWriteBuf() from wxString, keeping the same buffer
wxString uses intact.  In other words, relying on wxStringBuffer containing the old 
wxString data is probably not a good idea if you want to build your program in both
with and without wxUSE\_STL.

Note that SetLength {\tt must} be called before wxStringBufferLength destructs.

\wxheading{Derived from}

None

\wxheading{Include files}

<wx/string.h>

\latexignore{\rtfignore{\wxheading{Members}}}

\membersection{wxStringBufferLength::wxStringBufferLength}\label{wxstringbufferlengthctor}

\func{}{wxStringBufferLength}{\param{const wxString\& }{str}, \param{size\_t }{len}}

Constructs a writable string buffer object associated with the given string
and containing enough space for at least {\it len} characters. Basically, this
is equivalent to calling \helpref{GetWriteBuf}{wxstringgetwritebuf} and
saving the result.

\membersection{wxStringBufferLength::\destruct{wxStringBufferLength}}\label{wxstringbufferlengthdtor}

\func{}{\destruct{wxStringBufferLength}}{\void}

Restores the string passed to the constructor to the usable state by calling 
\helpref{UngetWriteBuf}{wxstringungetwritebuf} on it.

\membersection{wxStringBufferLength::SetLength}\label{wxstringbufferlengthsetlength}

\func{void}{SetLength}{\param{size\_t }{nLength}}

Sets the internal length of the string referred to by wxStringBufferLength to 
{\it nLength} characters.

Must be called before wxStringBufferLength destructs.

\membersection{wxStringBufferLength::operator wxChar *}\label{wxstringbufferlengthwxchar}

\func{wxChar *}{operator wxChar *}{\void}

Returns the writable pointer to a buffer of the size at least equal to the
length specified in the constructor.


