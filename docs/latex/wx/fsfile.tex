%
% automatically generated by HelpGen from
% fsfile.tex at 21/Mar/99 23:00:52
%

\section{\class{wxFSFile}}\label{wxfsfile}

This class represents a single file opened by \helpref{wxFileSystem}{wxfilesystem}.
It provides more information than wxWindow's input stream 
(stream, filename, mime type, anchor).

{\bf Note:} Any pointer returned by a method of wxFSFile is valid
only as long as the wxFSFile object exists. For example a call to GetStream()
doesn't {\it create} the stream but only returns the pointer to it. In
other words after 10 calls to GetStream() you will obtain ten identical
pointers.

\wxheading{Derived from}

\helpref{wxObject}{wxobject}

\wxheading{Include files}

<wx/filesys.h>

\wxheading{See Also}

\helpref{wxFileSystemHandler}{wxfilesystemhandler}, 
\helpref{wxFileSystem}{wxfilesystem}, 
\helpref{Overview}{fs}

\latexignore{\rtfignore{\wxheading{Members}}}

\membersection{wxFSFile::wxFSFile}\label{wxfsfilewxfsfile}

\func{}{wxFSFile}{\param{wxInputStream }{*stream}, \param{const wxString\& }{loc}, \param{const wxString\& }{mimetype}, \param{const wxString\& }{anchor}, \param{wxDateTime }{modif}}

Constructor. You probably won't use it. See Notes for details.

\wxheading{Parameters}

\docparam{stream}{The input stream that will be used to access data}

\docparam{location}{The full location (aka filename) of the file}

\docparam{mimetype}{MIME type of this file. Mime type is either extension-based or HTTP Content-Type}

\docparam{anchor}{Anchor. See \helpref{GetAnchor()}{wxfsfilegetanchor} for details.}

If you are not sure of the meaning of these params, see the description of the GetXXXX()
functions.

\wxheading{Notes}

It is seldom used by the application programmer but you will need it if
you are writing your own virtual FS. For example you may need something
similar to wxMemoryInputStream, but because wxMemoryInputStream
doesn't free the memory when destroyed and thus passing a memory stream
pointer into wxFSFile constructor would lead to memory leaks, you
can write your own class derived from wxFSFile:

\begin{verbatim}
class wxMyFSFile : public wxFSFile
{
    private:
        void *m_Mem;
    public:
        wxMyFSFile(.....)
	~wxMyFSFile() {free(m_Mem);}
	    // of course dtor is virtual ;-)
};
\end{verbatim}

\membersection{wxFSFile::GetAnchor}\label{wxfsfilegetanchor}

\constfunc{const wxString\&}{GetAnchor}{\void}

Returns anchor (if present). The term of {\bf anchor} can be easily
explained using few examples:

\begin{verbatim}
index.htm#anchor                      /* 'anchor' is anchor */
index/wx001.htm                       /* NO anchor here!    */
archive/main.zip#zip:index.htm#global /* 'global'           */
archive/main.zip#zip:index.htm        /* NO anchor here!    */
\end{verbatim}

Usually an anchor is presented only if the MIME type is 'text/html'.
But it may have some meaning with other files;
for example myanim.avi\#200 may refer to position in animation
or reality.wrl\#MyView may refer to a predefined view in VRML.

\membersection{wxFSFile::GetLocation}\label{wxfsfilegetlocation}

\constfunc{const wxString\&}{GetLocation}{\void}

Returns full location of the file, including path and protocol. 
Examples : 

\begin{verbatim}
http://www.wxwidgets.org
http://www.ms.mff.cuni.cz/~vsla8348/wxhtml/archive.zip#zip:info.txt
file:/home/vasek/index.htm
relative-file.htm
\end{verbatim}

\membersection{wxFSFile::GetMimeType}\label{wxfsfilegetmimetype}

\constfunc{const wxString\&}{GetMimeType}{\void}

Returns the MIME type of the content of this file. It is either
extension-based (see wxMimeTypesManager) or extracted from
HTTP protocol Content-Type header.

\membersection{wxFSFile::GetModificationTime}\label{wxfsfilegetmodificationtime}

\constfunc{wxDateTime}{GetModificationTime}{\void}

Returns time when this file was modified.

\membersection{wxFSFile::GetStream}\label{wxfsfilegetstream}

\constfunc{wxInputStream*}{GetStream}{\void}

Returns pointer to the stream. You can use the returned
stream to directly access data. You may suppose
that the stream provide Seek and GetSize functionality
(even in the case of the HTTP protocol which doesn't provide
this by default. wxHtml uses local cache to work around
this and to speed up the connection).

