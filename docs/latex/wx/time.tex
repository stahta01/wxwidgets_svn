\section{\class{wxTime}}\label{wxtime}

Representation of time and date.

NOTE: this class should be
used with caution, since it is not fully tested. It will be replaced
with a new wxDateTime class in the near future.

\wxheading{Derived from}

\helpref{wxObject}{wxobject}

\wxheading{Include files}

<wx/time.h>

\wxheading{Data structures}

{\small \begin{verbatim}
typedef unsigned short hourTy;
\end{verbatim}}

{\small \begin{verbatim}
typedef unsigned short minuteTy;
\end{verbatim}}

{\small \begin{verbatim}
typedef unsigned short secondTy;
\end{verbatim}}

{\small \begin{verbatim}
typedef unsigned long clockTy;
\end{verbatim}}

{\small \begin{verbatim}
enum tFormat { wx12h, wx24h };
\end{verbatim}}

{\small \begin{verbatim}
enum tPrecision { wxStdMinSec, wxStdMin };
\end{verbatim}}

\wxheading{See also}

\helpref{wxDate}{wxDate}

\latexignore{\rtfignore{\wxheading{Members}}}

\membersection{wxTime::wxTime}\label{wxtimewxtime}

\func{}{wxTime}{\void}

Initialize the object using the current time.

\func{}{wxTime}{\param{clockTy }{s}}

Initialize the object using the number of seconds that have elapsed since ???.

\func{}{wxTime}{\param{const wxTime\&}{ time}}

Copy constructor.

\func{}{wxTime}{\param{hourTy }{h}, \param{minuteTy }{m}, \param{secondTy }{s = 0}, \param{bool }{dst = FALSE}}

Initialize using hours, minutes, seconds, and whether DST time.

\func{}{wxTime}{\param{const wxDate\&}{ date}, \param{hourTy }{h = 0}, \param{minuteTy }{m = 0}, \param{secondTy }{s = 0}, \param{bool }{dst = FALSE}}

Initialize using a \helpref{wxDate}{wxdate} object, hours, minutes, seconds, and whether DST time.

\membersection{wxTime::GetDay}\label{wxtimegetday}

\constfunc{int}{GetDay}{\void}

Returns the day of the month.

\membersection{wxTime::GetDayOfWeek}\label{wxtimegetdatofweek}

\constfunc{int}{GetDayOfWeek}{\void}

Returns the day of the week, a number from 0 to 6 where 0 is Sunday and 6 is Saturday.

\membersection{wxTime::GetHour}\label{wxtimegethour}

\constfunc{hourTy}{GetHour}{\void}

Returns the hour in local time.

\membersection{wxTime::GetHourGMT}\label{wxtimegethourgmt}

\constfunc{hourTy}{GetHourGMT}{\void}

Returns the hour in GMT.

\membersection{wxTime::GetMinute}\label{wxtimegetminute}

\constfunc{minuteTy}{GetMinute}{\void}

Returns the minute in local time.

\membersection{wxTime::GetMinuteGMT}\label{wxtimegetminutegmt}

\constfunc{minuteTy}{GetMinuteGMT}{\void}

Returns the minute in GMT.

\membersection{wxTime::GetMonth}\label{wxtimegetmonth}

\constfunc{int}{GetMonth}{\void}

Returns the month.

\membersection{wxTime::GetSecond}\label{wxtimegetsecond}

\constfunc{secondTy}{GetSecond}{\void}

Returns the second in local time or GMT.

\membersection{wxTime::GetSecondGMT}\label{wxtimegetsecondgmt}

\constfunc{secondTy}{GetSecondGMT}{\void}

Returns the second in GMT.

\membersection{wxTime::GetSeconds}\label{wxtimegetseconds}

\constfunc{clockTy}{GetSeconds}{\void}

Returns the number of seconds since ???.

\membersection{wxTime::GetYear}\label{wxtimegetyear}

\constfunc{int}{GetYear}{\void}

Returns the year.

\membersection{wxTime::FormatTime}\label{wxtimeformattime}

\constfunc{char*}{FormatTime}{\void}

Formats the time according to the current formatting options: see \helpref{wxTime::SetFormat}{wxtimesetformat}.

\membersection{wxTime::IsBetween}\label{wxtimeisbetween}

\constfunc{bool}{IsBetween}{\param{const wxTime\& }{a}, \param{const wxTime\& }{b}}

Returns TRUE if this time is between the two given times.

\membersection{wxTime::Max}\label{wxtimemax}

\constfunc{wxTime}{Max}{\param{const wxTime\& }{time}}

Returns the maximum of the two times.

\membersection{wxTime::Min}\label{wxtimemin}

\constfunc{wxTime}{Min}{\param{const wxTime\& }{time}}

Returns the minimum of the two times.

\membersection{wxTime::SetFormat}\label{wxtimesetformat}

\func{static void}{SetFormat}{\param{const tFormat}{ format = wx12h},
  \param{const tPrecision}{ precision = wxStdMinSec}}

Sets the format and precision.

\membersection{wxTime::operator char*}\label{wxtimestring}

\func{operator}{char*}{\void}

Returns a pointer to a static char* containing the formatted time.

\membersection{wxTime::operator wxDate}\label{wxtimewxdate}

\constfunc{operator}{wxDate}{\void}

Converts the wxTime into a wxDate.

\membersection{wxTime::operator $=$}\label{wxtimeoperator}

\func{void}{operator $=$}{\param{const wxTime\& }{t}}

Assignment operator.

\membersection{wxTime::operator $<$}\label{wxtimeoperatorle}

\constfunc{bool}{operator $<$}{\param{const wxTime\& }{t}}

Less than operator.

\membersection{wxTime::operator $<=$}\label{wxtimeoperatorleq}

\constfunc{bool}{operator $<=$}{\param{const wxTime\& }{t}}

Less than or equal to operator.

\membersection{wxTime::operator $>$}\label{wxtimeoperatorge}

\constfunc{bool}{operator $>$}{\param{const wxTime\& }{t}}

Greater than operator.

\membersection{wxTime::operator $>=$}\label{wxtimeoperatorgeq}

\constfunc{bool}{operator $>=$}{\param{const wxTime\& }{t}}

Greater than or equal to operator.

\membersection{wxTime::operator $==$}\label{wxtimeoperatoreq}

\constfunc{bool}{operator $==$}{\param{const wxTime\& }{t}}

Equality operator.

\membersection{wxTime::operator $!=$}\label{wxtimeoperatorneq}

\constfunc{bool}{operator $!=$}{\param{const wxTime\& }{t}}

Inequality operator.

\membersection{wxTime::operator $+$}\label{wxtimeoperatorplus}

\constfunc{bool}{operator $+$}{\param{long }{sec}}

Addition operator.

\membersection{wxTime::operator $-$}\label{wxtimeoperatorminus}

\constfunc{bool}{operator $-$}{\param{long }{sec}}

Subtraction operator.

\membersection{wxTime::operator $+=$}\label{wxtimeoperatorpluseq}

\constfunc{bool}{operator $+=$}{\param{long }{sec}}

Increment operator.

\membersection{wxTime::operator $-=$}\label{wxtimeoperatorminuseq}

\constfunc{bool}{operator $-=$}{\param{long }{sec}}

Decrement operator.

