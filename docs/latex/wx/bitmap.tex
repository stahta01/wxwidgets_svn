\section{\class{wxBitmap}}\label{wxbitmap}

%\overview{Overview}{wxbitmapoverview}
%
This class encapsulates the concept of a platform-dependent bitmap,
either monochrome or colour.

\wxheading{Derived from}

\helpref{wxGDIObject}{wxgdiobject}\\
\helpref{wxObject}{wxobject}

\wxheading{Include files}

<wx/bitmap.h>

\wxheading{Predefined objects}

Objects:

{\bf wxNullBitmap}

\wxheading{See also}

\helpref{wxBitmap overview}{wxbitmapoverview},
\helpref{supported bitmap file formats}{supportedbitmapformats},
\helpref{wxDC::Blit}{wxdcblit},
\helpref{wxIcon}{wxicon}, \helpref{wxCursor}{wxcursor}, \helpref{wxBitmap}{wxbitmap},
\helpref{wxMemoryDC}{wxmemorydc}

\latexignore{\rtfignore{\wxheading{Members}}}

\membersection{wxBitmap::wxBitmap}\label{wxbitmapconstr}

\func{}{wxBitmap}{\void}

Default constructor.

\func{}{wxBitmap}{\param{const wxBitmap\& }{bitmap}}

Copy constructor.

\func{}{wxBitmap}{\param{void*}{ data}, \param{int}{ type}, \param{int}{ width}, \param{int}{ height}, \param{int}{ depth = -1}}

Creates a bitmap from the given data, which can be of arbitrary type.
Windows only, I think.

\func{}{wxBitmap}{\param{const char}{ bits[]}, \param{int}{ width}, \param{int}{ height}\\
  \param{int}{ depth = 1}}

Creates a bitmap from an array of bits.

Note that the bit depth is ignored on GTK+ and Motif. If you want to create a bitmap
from something else than a 1-bit data array, use the \helpref{wxImage}{wximage} class.

\func{}{wxBitmap}{\param{int}{ width}, \param{int}{ height}, \param{int}{ depth = -1}}

Creates a new bitmap. A depth of -1 indicates the depth of the current screen or
visual. Some platforms only support 1 for monochrome and -1 for the current colour
setting.

\func{}{wxBitmap}{\param{const char**}{ bits}}

Creates a bitmap from XPM data.

\func{}{wxBitmap}{\param{const wxString\& }{name}, \param{long}{ type}}

Loads a bitmap from a file or resource.

\wxheading{Parameters}

\docparam{bits}{Specifies an array of pixel values.}

\docparam{width}{Specifies the width of the bitmap.}

\docparam{height}{Specifies the height of the bitmap.}

\docparam{depth}{Specifies the depth of the bitmap. If this is omitted, the display depth of the
screen is used.}

\docparam{name}{This can refer to a resource name under MS Windows, or a filename under MS Windows and X.
Its meaning is determined by the {\it type} parameter.}

\docparam{type}{May be one of the following:

\twocolwidtha{5cm}
\begin{twocollist}
\twocolitem{{\bf \indexit{wxBITMAP\_TYPE\_BMP}}}{Load a Windows bitmap file.}
\twocolitem{{\bf \indexit{wxBITMAP\_TYPE\_BMP\_RESOURCE}}}{Load a Windows bitmap from the resource database.}
\twocolitem{{\bf \indexit{wxBITMAP\_TYPE\_GIF}}}{Load a GIF bitmap file.}
\twocolitem{{\bf \indexit{wxBITMAP\_TYPE\_XBM}}}{Load an X bitmap file.}
\twocolitem{{\bf \indexit{wxBITMAP\_TYPE\_XPM}}}{Load an XPM bitmap file.}
\twocolitem{{\bf \indexit{wxBITMAP\_TYPE\_RESOURCE}}}{Load a Windows resource name.}
\end{twocollist}

The validity of these flags depends on the platform and wxWindows configuration.
If all possible wxWindows settings are used, the Windows platform supports BMP file, BMP resource,
XPM data, and XPM. Under wxGTK, the available formats are BMP file, XPM data, XPM file, and PNG file.
Under wxMotif, the available formats are XBM data, XBM file, XPM data, XPM file.

In addition, wxBitmap can read all formats that \helpref{wxImage}{wximage} can, which currently include 
wxBITMAP\_TYPE\_JPEG, wxBITMAP\_TYPE\_TIF, wxBITMAP\_TYPE\_PNG, wxBITMAP\_TYPE\_GIF, wxBITMAP\_TYPE\_PCX, 
and wxBITMAP\_TYPE\_PNM. Of course, you must have wxImage handlers loaded. }

\wxheading{Remarks}

The first form constructs a bitmap object with no data; an assignment or another member function such as Create
or LoadFile must be called subsequently.

The second and third forms provide copy constructors. Note that these do not copy the
bitmap data, but instead a pointer to the data, keeping a reference count. They are therefore
very efficient operations.

The fourth form constructs a bitmap from data whose type and value depends on
the value of the {\it type} argument.

The fifth form constructs a (usually monochrome) bitmap from an array of pixel values, under both
X and Windows.

The sixth form constructs a new bitmap.

The seventh form constructs a bitmap from pixmap (XPM) data, if wxWindows has been configured
to incorporate this feature.

To use this constructor, you must first include an XPM file. For
example, assuming that the file {\tt mybitmap.xpm} contains an XPM array
of character pointers called mybitmap:

\begin{verbatim}
#include "mybitmap.xpm"

...

wxBitmap *bitmap = new wxBitmap(mybitmap);
\end{verbatim}

The eighth form constructs a bitmap from a file or resource. {\it name} can refer
to a resource name under MS Windows, or a filename under MS Windows and X.

Under Windows, {\it type} defaults to wxBITMAP\_TYPE\_BMP\_RESOURCE.
Under X, {\it type} defaults to wxBITMAP\_TYPE\_XPM.

\wxheading{See also}

\helpref{wxBitmap::LoadFile}{wxbitmaploadfile}

\pythonnote{Constructors supported by wxPython are:\par
\indented{2cm}{\begin{twocollist}
\twocolitem{\bf{wxBitmap(name, flag)}}{Loads a bitmap from a file}
\twocolitem{\bf{wxBitmapFromData(data, type, width, height, depth=1)}}{Creates
a bitmap from the given data, which can be of arbitrary type.}
\twocolitem{\bf{wxNoRefBitmap(name, flag)}}{This one won't own the
reference, so Python won't call the destructor, this is good for toolbars
and such where the parent will manage the bitmap.}
\twocolitem{\bf{wxEmptyBitmap(width, height, depth = -1)}}{Creates an
empty bitmap with the given specifications}
\end{twocollist}}
}

\membersection{wxBitmap::\destruct{wxBitmap}}

\func{}{\destruct{wxBitmap}}{\void}

Destroys the wxBitmap object and possibly the underlying bitmap data.
Because reference counting is used, the bitmap may not actually be
destroyed at this point - only when the reference count is zero will the
data be deleted.

If the application omits to delete the bitmap explicitly, the bitmap will be
destroyed automatically by wxWindows when the application exits.

Do not delete a bitmap that is selected into a memory device context.

\membersection{wxBitmap::AddHandler}\label{wxbitmapaddhandler}

\func{static void}{AddHandler}{\param{wxBitmapHandler*}{ handler}}

Adds a handler to the end of the static list of format handlers.

\docparam{handler}{A new bitmap format handler object. There is usually only one instance
of a given handler class in an application session.}

\wxheading{See also}

\helpref{wxBitmapHandler}{wxbitmaphandler}

\membersection{wxBitmap::CleanUpHandlers}

\func{static void}{CleanUpHandlers}{\void}

Deletes all bitmap handlers.

This function is called by wxWindows on exit.

\membersection{wxBitmap::Create}

\func{virtual bool}{Create}{\param{int}{ width}, \param{int}{ height}, \param{int}{ depth = -1}}

Creates a fresh bitmap. If the final argument is omitted, the display depth of
the screen is used.

\func{virtual bool}{Create}{\param{void*}{ data}, \param{int}{ type}, \param{int}{ width}, \param{int}{ height}, \param{int}{ depth = -1}}

Creates a bitmap from the given data, which can be of arbitrary type.

\wxheading{Parameters}

\docparam{width}{The width of the bitmap in pixels.}

\docparam{height}{The height of the bitmap in pixels.}

\docparam{depth}{The depth of the bitmap in pixels. If this is -1, the screen depth is used.}

\docparam{data}{Data whose type depends on the value of {\it type}.}

\docparam{type}{A bitmap type identifier - see \helpref{wxBitmap::wxBitmap}{wxbitmapconstr} for a list
of possible values.}

\wxheading{Return value}

TRUE if the call succeeded, FALSE otherwise.

\wxheading{Remarks}

The first form works on all platforms. The portability of the second form depends on the
type of data.

\wxheading{See also}

\helpref{wxBitmap::wxBitmap}{wxbitmapconstr}

\membersection{wxBitmap::FindHandler}

\func{static wxBitmapHandler*}{FindHandler}{\param{const wxString\& }{name}}

Finds the handler with the given name.

\func{static wxBitmapHandler*}{FindHandler}{\param{const wxString\& }{extension}, \param{long}{ bitmapType}}

Finds the handler associated with the given extension and type.

\func{static wxBitmapHandler*}{FindHandler}{\param{long }{bitmapType}}

Finds the handler associated with the given bitmap type.

\docparam{name}{The handler name.}

\docparam{extension}{The file extension, such as ``bmp".}

\docparam{bitmapType}{The bitmap type, such as wxBITMAP\_TYPE\_BMP.}

\wxheading{Return value}

A pointer to the handler if found, NULL otherwise.

\wxheading{See also}

\helpref{wxBitmapHandler}{wxbitmaphandler}

\membersection{wxBitmap::GetDepth}

\constfunc{int}{GetDepth}{\void}

Gets the colour depth of the bitmap. A value of 1 indicates a
monochrome bitmap.

\membersection{wxBitmap::GetHandlers}

\func{static wxList\&}{GetHandlers}{\void}

Returns the static list of bitmap format handlers.

\wxheading{See also}

\helpref{wxBitmapHandler}{wxbitmaphandler}

\membersection{wxBitmap::GetHeight}\label{wxbitmapgetheight}

\constfunc{int}{GetHeight}{\void}

Gets the height of the bitmap in pixels.

\membersection{wxBitmap::GetPalette}\label{wxbitmapgetpalette}

\constfunc{wxPalette*}{GetPalette}{\void}

Gets the associated palette (if any) which may have been loaded from a file
or set for the bitmap.

\wxheading{See also}

\helpref{wxPalette}{wxpalette}

\membersection{wxBitmap::GetMask}\label{wxbitmapgetmask}

\constfunc{wxMask*}{GetMask}{\void}

Gets the associated mask (if any) which may have been loaded from a file
or set for the bitmap.

\wxheading{See also}

\helpref{wxBitmap::SetMask}{wxbitmapsetmask}, \helpref{wxMask}{wxmask}

\membersection{wxBitmap::GetWidth}\label{wxbitmapgetwidth}

\constfunc{int}{GetWidth}{\void}

Gets the width of the bitmap in pixels.

\wxheading{See also}

\helpref{wxBitmap::GetHeight}{wxbitmapgetheight}

\membersection{wxBitmap::GetSubBitmap}\label{wxbitmapgetsubbitmap}

\constfunc{wxBitmap}{GetSubBitmap}{\param{const wxRect&}{rect}}

Returns a sub bitmap of the current one as long as the rect belongs entirely to 
the bitmap. This function preserves bit depth and mask information.

\membersection{wxBitmap::InitStandardHandlers}

\func{static void}{InitStandardHandlers}{\void}

Adds the standard bitmap format handlers, which, depending on wxWindows
configuration, can be handlers for Windows bitmap, Windows bitmap resource, and XPM.

This function is called by wxWindows on startup.

\wxheading{See also}

\helpref{wxBitmapHandler}{wxbitmaphandler}

\membersection{wxBitmap::InsertHandler}

\func{static void}{InsertHandler}{\param{wxBitmapHandler*}{ handler}}

Adds a handler at the start of the static list of format handlers.

\docparam{handler}{A new bitmap format handler object. There is usually only one instance
of a given handler class in an application session.}

\wxheading{See also}

\helpref{wxBitmapHandler}{wxbitmaphandler}

\membersection{wxBitmap::LoadFile}\label{wxbitmaploadfile}

\func{bool}{LoadFile}{\param{const wxString\&}{ name}, \param{long}{ type}}

Loads a bitmap from a file or resource.

\wxheading{Parameters}

\docparam{name}{Either a filename or a Windows resource name.
The meaning of {\it name} is determined by the {\it type} parameter.}

\docparam{type}{One of the following values:

\twocolwidtha{5cm}
\begin{twocollist}
\twocolitem{{\bf wxBITMAP\_TYPE\_BMP}}{Load a Windows bitmap file.}
\twocolitem{{\bf wxBITMAP\_TYPE\_BMP\_RESOURCE}}{Load a Windows bitmap from the resource database.}
\twocolitem{{\bf wxBITMAP\_TYPE\_GIF}}{Load a GIF bitmap file.}
\twocolitem{{\bf wxBITMAP\_TYPE\_XBM}}{Load an X bitmap file.}
\twocolitem{{\bf wxBITMAP\_TYPE\_XPM}}{Load an XPM bitmap file.}
\end{twocollist}

The validity of these flags depends on the platform and wxWindows configuration.

In addition, wxBitmap can read all formats that \helpref{wxImage}{wximage} can 
(wxBITMAP\_TYPE\_JPEG, wxBITMAP\_TYPE\_PNG, wxBITMAP\_TYPE\_GIF, wxBITMAP\_TYPE\_PCX, wxBITMAP\_TYPE\_PNM).
(Of course you must have wxImage handlers loaded.) }

\wxheading{Return value}

TRUE if the operation succeeded, FALSE otherwise.

\wxheading{Remarks}

A palette may be associated with the bitmap if one exists (especially for
colour Windows bitmaps), and if the code supports it. You can check
if one has been created by using the \helpref{GetPalette}{wxbitmapgetpalette} member.

\wxheading{See also}

\helpref{wxBitmap::SaveFile}{wxbitmapsavefile}

\membersection{wxBitmap::Ok}\label{wxbitmapok}

\constfunc{bool}{Ok}{\void}

Returns TRUE if bitmap data is present.

\membersection{wxBitmap::RemoveHandler}

\func{static bool}{RemoveHandler}{\param{const wxString\& }{name}}

Finds the handler with the given name, and removes it. The handler
is not deleted.

\docparam{name}{The handler name.}

\wxheading{Return value}

TRUE if the handler was found and removed, FALSE otherwise.

\wxheading{See also}

\helpref{wxBitmapHandler}{wxbitmaphandler}

\membersection{wxBitmap::SaveFile}\label{wxbitmapsavefile}

\func{bool}{SaveFile}{\param{const wxString\& }{name}, \param{int}{ type}, \param{wxPalette* }{palette = NULL}}

Saves a bitmap in the named file.

\wxheading{Parameters}

\docparam{name}{A filename. The meaning of {\it name} is determined by the {\it type} parameter.}

\docparam{type}{One of the following values:

\twocolwidtha{5cm}
\begin{twocollist}
\twocolitem{{\bf wxBITMAP\_TYPE\_BMP}}{Save a Windows bitmap file.}
\twocolitem{{\bf wxBITMAP\_TYPE\_GIF}}{Save a GIF bitmap file.}
\twocolitem{{\bf wxBITMAP\_TYPE\_XBM}}{Save an X bitmap file.}
\twocolitem{{\bf wxBITMAP\_TYPE\_XPM}}{Save an XPM bitmap file.}
\end{twocollist}

The validity of these flags depends on the platform and wxWindows configuration.

In addition, wxBitmap can save all formats that \helpref{wxImage}{wximage} can 
(wxBITMAP\_TYPE\_JPEG, wxBITMAP\_TYPE\_PNG).
(Of course you must have wxImage handlers loaded.) }

\docparam{palette}{An optional palette used for saving the bitmap.}
% TODO: this parameter should
%probably be eliminated; instead the app should set the palette before saving.

\wxheading{Return value}

TRUE if the operation succeeded, FALSE otherwise.

\wxheading{Remarks}

Depending on how wxWindows has been configured, not all formats may be available.

\wxheading{See also}

\helpref{wxBitmap::LoadFile}{wxbitmaploadfile}

\membersection{wxBitmap::SetDepth}\label{wxbitmapsetdepth}

\func{void}{SetDepth}{\param{int }{depth}}

Sets the depth member (does not affect the bitmap data).

\wxheading{Parameters}

\docparam{depth}{Bitmap depth.}

\membersection{wxBitmap::SetHeight}\label{wxbitmapsetheight}

\func{void}{SetHeight}{\param{int }{height}}

Sets the height member (does not affect the bitmap data).

\wxheading{Parameters}

\docparam{height}{Bitmap height in pixels.}

\membersection{wxBitmap::SetMask}\label{wxbitmapsetmask}

\func{void}{SetMask}{\param{wxMask* }{mask}}

Sets the mask for this bitmap.

\wxheading{Remarks}

The bitmap object owns the mask once this has been called.

\wxheading{See also}

\helpref{wxBitmap::GetMask}{wxbitmapgetmask}, \helpref{wxMask}{wxmask}

\membersection{wxBitmap::SetOk}

\func{void}{SetOk}{\param{int }{isOk}}

Sets the validity member (does not affect the bitmap data).

\wxheading{Parameters}

\docparam{isOk}{Validity flag.}

\membersection{wxBitmap::SetPalette}\label{wxbitmapsetpalette}

\func{void}{SetPalette}{\param{wxPalette* }{palette}}

Sets the associated palette: it will be deleted in the wxBitmap
destructor, so if you do not wish it to be deleted automatically,
reset the palette to NULL before the bitmap is deleted.

\wxheading{Parameters}

\docparam{palette}{The palette to set.}

\wxheading{Remarks}

The bitmap object owns the palette once this has been called.

\wxheading{See also}

\helpref{wxPalette}{wxpalette}

\membersection{wxBitmap::SetWidth}

\func{void}{SetWidth}{\param{int }{width}}

Sets the width member (does not affect the bitmap data).

\wxheading{Parameters}

\docparam{width}{Bitmap width in pixels.}

\membersection{wxBitmap::operator $=$}

\func{wxBitmap\& }{operator $=$}{\param{const wxBitmap\& }{bitmap}}

Assignment operator. This operator does not copy any data, but instead
passes a pointer to the data in {\it bitmap} and increments a reference
counter. It is a fast operation.

\wxheading{Parameters}

\docparam{bitmap}{Bitmap to assign.}

\wxheading{Return value}

Returns 'this' object.

\membersection{wxBitmap::operator $==$}

\func{bool}{operator $==$}{\param{const wxBitmap\& }{bitmap}}

Equality operator. This operator tests whether the internal data pointers are
equal (a fast test).

\wxheading{Parameters}

\docparam{bitmap}{Bitmap to compare with 'this'}

\wxheading{Return value}

Returns TRUE if the bitmaps were effectively equal, FALSE otherwise.

\membersection{wxBitmap::operator $!=$}

\func{bool}{operator $!=$}{\param{const wxBitmap\& }{bitmap}}

Inequality operator. This operator tests whether the internal data pointers are
unequal (a fast test).

\wxheading{Parameters}

\docparam{bitmap}{Bitmap to compare with 'this'}

\wxheading{Return value}

Returns TRUE if the bitmaps were unequal, FALSE otherwise.

\section{\class{wxBitmapHandler}}\label{wxbitmaphandler}

\overview{Overview}{wxbitmapoverview}

This is the base class for implementing bitmap file loading/saving, and bitmap creation from data.
It is used within wxBitmap and is not normally seen by the application.

If you wish to extend the capabilities of wxBitmap, derive a class from wxBitmapHandler
and add the handler using \helpref{wxBitmap::AddHandler}{wxbitmapaddhandler} in your
application initialisation.

\wxheading{Derived from}

\helpref{wxObject}{wxobject}

\wxheading{Include files}

<wx/bitmap.h>

\wxheading{See also}

\helpref{wxBitmap}{wxbitmap}, \helpref{wxIcon}{wxicon}, \helpref{wxCursor}{wxcursor}

\latexignore{\rtfignore{\wxheading{Members}}}

\membersection{wxBitmapHandler::wxBitmapHandler}\label{wxbitmaphandlerconstr}

\func{}{wxBitmapHandler}{\void}

Default constructor. In your own default constructor, initialise the members
m\_name, m\_extension and m\_type.

\membersection{wxBitmapHandler::\destruct{wxBitmapHandler}}

\func{}{\destruct{wxBitmapHandler}}{\void}

Destroys the wxBitmapHandler object.

\membersection{wxBitmapHandler::Create}

\func{virtual bool}{Create}{\param{wxBitmap* }{bitmap}, \param{void*}{ data}, \param{int}{ type}, \param{int}{ width}, \param{int}{ height}, \param{int}{ depth = -1}}

Creates a bitmap from the given data, which can be of arbitrary type. The wxBitmap object {\it bitmap} is
manipulated by this function.

\wxheading{Parameters}

\docparam{bitmap}{The wxBitmap object.}

\docparam{width}{The width of the bitmap in pixels.}

\docparam{height}{The height of the bitmap in pixels.}

\docparam{depth}{The depth of the bitmap in pixels. If this is -1, the screen depth is used.}

\docparam{data}{Data whose type depends on the value of {\it type}.}

\docparam{type}{A bitmap type identifier - see \helpref{wxBitmapHandler::wxBitmapHandler}{wxbitmapconstr} for a list
of possible values.}

\wxheading{Return value}

TRUE if the call succeeded, FALSE otherwise (the default).

\membersection{wxBitmapHandler::GetName}

\constfunc{wxString}{GetName}{\void}

Gets the name of this handler.

\membersection{wxBitmapHandler::GetExtension}

\constfunc{wxString}{GetExtension}{\void}

Gets the file extension associated with this handler.

\membersection{wxBitmapHandler::GetType}

\constfunc{long}{GetType}{\void}

Gets the bitmap type associated with this handler.

\membersection{wxBitmapHandler::LoadFile}\label{wxbitmaphandlerloadfile}

\func{bool}{LoadFile}{\param{wxBitmap* }{bitmap}, \param{const wxString\&}{ name}, \param{long}{ type}}

Loads a bitmap from a file or resource, putting the resulting data into {\it bitmap}.

\wxheading{Parameters}

\docparam{bitmap}{The bitmap object which is to be affected by this operation.}

\docparam{name}{Either a filename or a Windows resource name.
The meaning of {\it name} is determined by the {\it type} parameter.}

\docparam{type}{See \helpref{wxBitmap::wxBitmap}{wxbitmapconstr} for values this can take.}

\wxheading{Return value}

TRUE if the operation succeeded, FALSE otherwise.

\wxheading{See also}

\helpref{wxBitmap::LoadFile}{wxbitmaploadfile}\\
\helpref{wxBitmap::SaveFile}{wxbitmapsavefile}\\
\helpref{wxBitmapHandler::SaveFile}{wxbitmaphandlersavefile}

\membersection{wxBitmapHandler::SaveFile}\label{wxbitmaphandlersavefile}

\func{bool}{SaveFile}{\param{wxBitmap* }{bitmap}, \param{const wxString\& }{name}, \param{int}{ type}, \param{wxPalette* }{palette = NULL}}

Saves a bitmap in the named file.

\wxheading{Parameters}

\docparam{bitmap}{The bitmap object which is to be affected by this operation.}

\docparam{name}{A filename. The meaning of {\it name} is determined by the {\it type} parameter.}

\docparam{type}{See \helpref{wxBitmap::wxBitmap}{wxbitmapconstr} for values this can take.}

\docparam{palette}{An optional palette used for saving the bitmap.}

\wxheading{Return value}

TRUE if the operation succeeded, FALSE otherwise.

\wxheading{See also}

\helpref{wxBitmap::LoadFile}{wxbitmaploadfile}\\
\helpref{wxBitmap::SaveFile}{wxbitmapsavefile}\\
\helpref{wxBitmapHandler::LoadFile}{wxbitmaphandlerloadfile}

\membersection{wxBitmapHandler::SetName}

\func{void}{SetName}{\param{const wxString\& }{name}}

Sets the handler name.

\wxheading{Parameters}

\docparam{name}{Handler name.}

\membersection{wxBitmapHandler::SetExtension}

\func{void}{SetExtension}{\param{const wxString\& }{extension}}

Sets the handler extension.

\wxheading{Parameters}

\docparam{extension}{Handler extension.}

\membersection{wxBitmapHandler::SetType}

\func{void}{SetType}{\param{long }{type}}

Sets the handler type.

\wxheading{Parameters}

\docparam{name}{Handler type.}


