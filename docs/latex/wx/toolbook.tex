%%%%%%%%%%%%%%%%%%%%%%%%%%%%%%%%%%%%%%%%%%%%%%%%%%%%%%%%%%%%%%%%%%%%%%%%%%%%%%%
%% Name:        toolbook.tex
%% Purpose:     wxToolbook documentation
%% Author:      Julian Smart
%% Modified by:
%% Created:     2006-01-30
%% RCS-ID:      $Id$
%% Copyright:   (c) 2006 Julian Smart
%% License:     wxWindows license
%%%%%%%%%%%%%%%%%%%%%%%%%%%%%%%%%%%%%%%%%%%%%%%%%%%%%%%%%%%%%%%%%%%%%%%%%%%%%%%

\section{\class{wxToolbook}}\label{wxtoolbook}

wxToolbook is a class similar to \helpref{wxNotebook}{wxnotebook} but which
uses a \helpref{wxToolBar}{wxtoolbar} to show the labels instead of the
tabs.

There is no documentation for this class yet but its usage is
identical to wxNotebook (except for the features clearly related to tabs
only), so please refer to that class documentation for now. You can also
use the \helpref{notebook sample}{samplenotebook} to see wxToolbook in action.

\wxheading{Derived from}

wxBookCtrlBase\\
\helpref{wxControl}{wxcontrol}\\
\helpref{wxControl}{wxcontrol}\\
\helpref{wxWindow}{wxwindow}\\
\helpref{wxEvtHandler}{wxevthandler}\\
\helpref{wxObject}{wxobject}

\wxheading{Include files}

<wx/toolbook.h>

\wxheading{Window styles}

\twocolwidtha{5cm}
\begin{twocollist}\itemsep=0pt

\twocolitem{\windowstyle{wxBK\_DEFAULT}}{Choose the default location for the
labels depending on the current platform (currently always the top).}
\end{twocollist}

\wxheading{See also}

\helpref{wxBookCtrl}{wxbookctrloverview}, \helpref{wxNotebook}{wxnotebook}, \helpref{notebook sample}{samplenotebook}

