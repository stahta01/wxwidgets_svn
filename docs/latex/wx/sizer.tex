\section{\class{wxSizer}}\label{wxsizer}

wxSizer is the abstract base class used for laying out subwindows in a window. You
cannot use wxSizer directly; instead, you will have to use \helpref{wxBoxSizer}{wxboxsizer}, 
\helpref{wxStaticBoxSizer}{wxstaticboxsizer} or \helpref{wxNotebookSizer}{wxnotebooksizer}.

The layout algorithm used by sizers in wxWindows is closely related to layout
in other GUI toolkits, such as Java's AWT, the GTK toolkit or the Qt toolkit. It is
based upon the idea of the individual subwindows reporting their minimal required
size and their ability to get stretched if the size of the parent window has changed.
This will most often mean, that the programmer does not set the original size of
a dialog in the beginning, rather the dialog will assigned a sizer and this sizer
will be queried about the recommended size. The sizer in turn will query its
children, which can be normal windows, empty space or other sizers, so that
a hierarchy of sizers can be constructed. Note that wxSizer does not derive from wxWindow
and thus do not interfere with tab ordering and requires very little resources compared
to a real window on screen.

What makes sizers so well fitted for use in wxWindows is the fact that every control
reports its own minimal size and the algorithm can handle differences in font sizes
or different window (dialog item) sizes on different platforms without problems. If e.g.
the standard font as well as the overall design of Motif widgets requires more space than
on Windows, the initial dialog size will automatically be bigger on Motif than on Windows.

\pythonnote{If you wish to create a sizer class in wxPython you should
derive the class from {\tt wxPySizer} in order to get Python-aware
capabilities for the various virtual methods.}

\wxheading{Derived from}

\helpref{wxObject}{wxobject}

\latexignore{\rtfignore{\wxheading{Members}}}

\membersection{wxSizer::wxSizer}\label{wxsizerwxsizer}

\func{}{wxSizer}{\void}

The constructor. Note that wxSizer is an abstract base class and may not
be instantiated.

\membersection{wxSizer::\destruct{wxSizer}}\label{wxsizerdtor}

\func{}{\destruct{wxSizer}}{\void}

The destructor.

\membersection{wxSizer::Add}\label{wxsizeradd}

\func{void}{Add}{\param{wxWindow* }{window}, \param{int }{option = 0},\param{int }{flag = 0}, \param{int }{border = 0}, \param{wxObject* }{userData = NULL}}

\func{void}{Add}{\param{wxSizer* }{sizer}, \param{int }{option = 0}, \param{int }{flag = 0}, \param{int }{border = 0}, \param{wxObject* }{userData = NULL}}

\func{void}{Add}{\param{int }{width}, \param{int }{height}, \param{int }{option = 0}, \param{int }{flag = 0}, \param{int }{border = 0}, \param{wxObject* }{userData = NULL}}

Adds the {\it window} to the sizer. As wxSizer itself is an abstract class, the parameters
have no meaning in the wxSizer class itself, but as there currently is only one class
deriving directly from wxSizer and this class does not override these methods, the meaning
of the parameters is described here:

\docparam{window}{The window to be added to the sizer. Its initial size (either set explicitly by the
user or calculated internally when using wxDefaultSize) is interpreted as the minimal and in many
cases also the initial size. This is particularly useful in connection with \helpref{SetSizeHints}{wxsizersetsizehints}.}

\docparam{sizer}{The (child-)sizer to be added to the sizer. This allows placing a child sizer in a
sizer and thus to create hierarchies of sizers (typically a vertical box as the top sizer and several
horizontal boxes on the level beneath).}

\docparam{width and height}{The dimension of a spacer to be added to the sizer. Adding spacers to sizers
gives more flexilibilty in the design of dialogs; imagine for example a horizontal box with two buttons at the
bottom of a dialog: you might want to insert a space between the two buttons and make that space stretchable
using the {\it option} flag and the result will be that the left button will be aligned with the left
side of the dialog and the right button with the right side - the space in between will shrink and grow with
the dialog.}

\docparam{option}{Although the meaning of this parameter is undefined in wxSizer, it is used in wxBoxSizer
to indicate if a child of a sizer can change its size in the main orientation of the wxBoxSizer - where
0 stands for not changable and a value of more than zero is interpreted relative to the value of other
children of the same wxBoxSizer. For example, you might have a horizontal wxBoxSizer with three children, two
of which are supposed to change their size with the sizer. Then the two stretchable windows would get a
value of 1 each to make them grow and shrink equally with the sizer's horizontal dimension.}

\docparam{flag}{This parameter can be used to set a number of flags which can be combined using
the binary OR operator |. Two main behaviours are defined using these flags. One is the border
around a window: the {\it border} parameter determines the border width whereas the flags given here
determine where the border may be (wxTOP, wxBOTTOM, wxLEFT, wxRIGHT or wxALL). The other flags
determine the child window's behaviour if the size of the sizer changes. However this is not - in contrast to
the {\it option} flag - in the main orientation, but in the respectively other orientation. So
if you created a wxBoxSizer with the wxVERTICAL option, these flags will be relevant if the
sizer changes its horizontal size. A child may get resized to completely fill out the new size (using
either wxGROW or wxEXPAND), it may get proportionally resized (wxSHAPED), it may get centered (wxALIGN\_CENTER
or wxALIGN\_CENTRE) or it may get aligned to either side (wxALIGN\_LEFT and wxALIGN\_TOP are set to 0
and thus represent the default, wxALIGN\_RIGHT and wxALIGN\_BOTTOM have their obvious meaning).
With proportional resize, a child may also be centered in the main orientation using
wxALIGN\_CENTER\_VERTICAL (same as wxALIGN\_CENTRE\_VERTICAL) and wxALIGN\_CENTER\_HORIZONTAL
(same as wxALIGN\_CENTRE\_HORIZONTAL) flags.}

\docparam{border}{Determines the border width, if the {\it flag} parameter is set to any border.}

\docparam{userData}{Allows an extra object to be attached to the sizer
item, for use in derived classes when sizing information is more
complex than the {\it option} and {\it flag} will allow for.}

\membersection{wxSizer::CalcMin}\label{wxsizercalcmin}

\func{wxSize}{CalcMin}{\void}

This method is abstract and has to be overwritten by any derived class.
Here, the sizer will do the actual calculation of its children minimal sizes.

\membersection{wxSizer::Fit}\label{wxsizerfit}

\func{void}{Fit}{\param{wxWindow* }{window}}

Tell the sizer to resize the {\it window} to match the sizer's minimal size. This
is commonly done in the constructor of the window itself, see sample in the description
of \helpref{wxBoxSizer}{wxboxsizer}.

\membersection{wxSizer::GetSize}\label{wxsizergetsize}

\func{wxSize}{GetSize}{\void}

Returns the current size of the sizer.

\membersection{wxSizer::GetPosition}\label{wxsizergetposition}

\func{wxPoint}{GetPosition}{\void}

Returns the current position of the sizer.

\membersection{wxSizer::GetMinSize}\label{wxsizergetminsize}

\func{wxSize}{GetMinSize}{\void}

Returns the minimal size of the sizer. This is either the combined minimal
size of all the children and their borders or the minimal size set by 
\helpref{SetMinSize}{wxsizersetminsize}, depending on which is bigger.

\membersection{wxSizer::Layout}\label{wxsizerlayout}

\func{void}{Layout}{\void}

Call this to force layout of the children anew, e.g. after having added a child
to or removed a child (window, other sizer or space) from the sizer while keeping
the current dimension.

\membersection{wxSizer::Prepend}\label{wxsizerprepend}

\func{void}{Prepend}{\param{wxWindow* }{window}, \param{int }{option = 0}, \param{int }{flag = 0}, \param{int }{border = 0}, \param{wxObject* }{userData = NULL}}

\func{void}{Prepend}{\param{wxSizer* }{sizer}, \param{int }{option = 0}, \param{int }{flag = 0}, \param{int }{border = 0}, \param{wxObject* }{userData = NULL}}

\func{void}{Prepend}{\param{int }{width}, \param{int }{height}, \param{int }{option = 0}, \param{int }{flag = 0}, \param{int }{border= 0}, \param{wxObject* }{userData = NULL}}

Same as \helpref{wxSizer::Add}{wxsizeradd}, but prepends the items to the beginning of the
list of items (windows, subsizers or spaces) owned by this sizer.

\membersection{wxSizer::RecalcSizes}\label{wxsizerrecalcsizes}

\func{void}{RecalcSizes}{\void}

This method is abstract and has to be overwritten by any derived class.
Here, the sizer will do the actual calculation of its children's positions
and sizes.

\membersection{wxSizer::Remove}\label{wxsizerremove}

\func{bool}{Remove}{\param{wxWindow* }{window}}

\func{bool}{Remove}{\param{wxSizer* }{sizer}}

\func{bool}{Remove}{\param{int }{nth}}

Removes a child from the sizer. {\it window} is the window to be removed, {\it sizer} is the
equivalent sizer and {\it nth} is the position of the child in the sizer, typically 0 for
the first item. This method does not cause any layout or resizing to take place and does
not delete the window itself. Call \helpref{wxSizer::Layout}{wxsizerlayout} to update
the layout "on screen" after removing a child fom the sizer.

Returns TRUE if the child item was found and removed, FALSE otherwise.

\membersection{wxSizer::SetDimension}\label{wxsizersetdimension}

\func{void}{SetDimension}{\param{int }{x}, \param{int }{y}, \param{int }{width}, \param{int }{height}}

Call this to force the sizer to take the given dimension and thus force the items owned
by the sizer to resize themselves according to the rules defined by the paramater in the 
\helpref{Add}{wxsizeradd} and \helpref{Prepend}{wxsizerprepend} methods.

\membersection{wxSizer::SetMinSize}\label{wxsizersetminsize}

\func{void}{SetMinSize}{\param{int }{width}, \param{int }{height}}

\func{void}{SetMinSize}{\param{wxSize }{size}}

Call this to give the sizer a minimal size. Normally, the sizer will calculate its
minimal size based purely on how much space its children need. After calling this
method \helpref{GetMinSize}{wxsizergetminsize} will return either the minimal size
as requested by its children or the minimal size set here, depending on which is
bigger.

\membersection{wxSizer::SetItemMinSize}\label{wxsizersetitemminsize}

\func{void}{SetItemMinSize}{\param{wxWindow* }{window}, \param{int}{ width}, \param{int}{ height}}

\func{void}{SetItemMinSize}{\param{wxSizer* }{sizer}, \param{int}{ width}, \param{int}{ height}}

\func{void}{SetItemMinSize}{\param{int}{ pos}, \param{int}{ width}, \param{int}{ height}}

Set an item's minimum size by window, sizer, or position. The item will be found recursively
in the sizer's descendants. This function enables an application to set the size of an item
after initial creation.

\membersection{wxSizer::SetSizeHints}\label{wxsizersetsizehints}

\func{void}{SetSizeHints}{\param{wxWindow* }{window}}

Tell the sizer to set the minimal size of the {\it window} to match the sizer's minimal size.
This is commonly done in the constructor of the window itself, see sample in the description
of \helpref{wxBoxSizer}{wxboxsizer} if the window is resizable (as are many dialogs under Unix and
frames on probably all platforms).

