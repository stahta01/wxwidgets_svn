\chapter{Introduction}\label{introduction}
\pagenumbering{arabic}%
\setheader{{\it CHAPTER \thechapter}}{}{}{}{}{{\it CHAPTER \thechapter}}%
\setfooter{\thepage}{}{}{}{}{\thepage}%

\section{What is wxWindows?}

wxWindows is a C++ framework providing GUI (Graphical User
Interface) and other facilities on more than one platform.  It currently
supports subsets of Motif, Xt and MS Windows (16-bit, Windows 95 and Windows NT).

wxWindows was originally developed at the Artificial Intelligence
Applications Institute, University of Edinburgh, for internal use.
wxWindows has been released into the public domain in the hope
that others will also find it useful. Version 2.0 is written and
maintained by Julian Smart and Markus Holzem, with support from users.

This manual discusses wxWindows in the context of multi-platform
development.\helpignore{For more detail on the wxWindows version 2.0 API
(Application Programming Interface) please refer to the separate
wxWindows reference manual.}

Please note that in the following, ``MS Windows" often refers to all
platforms related to Microsoft Windows, including 16-bit and 32-bit
variants, unless otherwise stated. All trademarks are acknowledged.

\section{Why another cross-platform development tool?}

wxWindows was developed to provide a cheap and flexible way to maximize
investment in GUI application development.  While a number of commercial
class libraries already exist for cross-platform development,
none met all of the following criteria:

\begin{enumerate}\itemsep=0pt
\item low price;
\item source availability;
\item simplicity of programming;
\item support for a wide range of compilers.
\end{enumerate}

As public domain software and a project open to everyone, wxWindows has
benefited from comments, ideas, bug fixes, enhancements and the sheer
enthusiasm of users, especially via the Internet. This gives wxWindows a
certain advantage over its commercial brothers, and a robustness against
the transience of one individual or company. This openness and
availability of source code is especially important when the future of
thousands of lines of application code may depend upon the longevity of
the underlying class library.

In writing wxWindows, completeness has sometimes been traded for
portability and simplicity of programming. Version 2.0 goes much
further than previous versions in terms of generality and features,
allowing applications to be produced
that are often indistinguishable from those produced using single-platform
toolkits
such as Motif and MFC.

wxWindows 2.0 currently maps to two native APIs: Motif and
MS Windows. An Xt port is also in preparation.

The importance of using a platform-independent class library cannot be
overstated, since GUI application development is very time-consuming,
and sustained popularity of particular GUIs cannot be guaranteed.
Code can very quickly become obsolete if it addresses the wrong
platform or audience.  wxWindows helps to insulate the programmer from
these winds of change. Although wxWindows may not be suitable for
every application, it provides access to most of the functionality a
GUI program normally requires, plus some extras such as form
construction, interprocess communication and PostScript output, and
can of course be extended as needs dictate.  As a bonus, it provides
a cleaner programming interface than the native
APIs. Programmers may find it worthwhile to use wxWindows even if they
are developing on only one platform.

It is impossible to sum up the functionality of wxWindows in a few paragraphs, but
here are some of the benefits:

\begin{itemize}\itemsep=0pt
\item Low cost (free, in fact!)
\item You get the source.
\item Several example programs.
\item Over 200 pages of printable and on-line documentation.
\item Simple-to-use, object-oriented API.
\item Graphics calls include splines, polylines, rounded rectangles, etc.
\item Constraint-based layout option.
\item Print/preview and document/view architectures.
\item Status line facility, toolbar
\item Easy, object-oriented interprocess comms (DDE subset) under UNIX and
MS Windows.
\item Encapsulated PostScript generation under UNIX, normal MS Windows printing on the
PC.
\item MDI support under Windows.
\item Can be used to create DLLs under Windows, dynamic libraries on the Sun.
\item Common dialogs for file browsing, printing, colour selection, etc.
\item Under MS Windows, support for creating metafiles and copying
them to the clipboard.
\item Hypertext help facility, with an API for invocation from applications.
\item Dialog Editor for building dialogs.
\end{itemize}

\section{Changes from version 1.xx}\label{versionchanges}

These are a few of the major differences between versions 1.xx and 2.0.

Removals:

\begin{itemize}\itemsep=0pt
\item XView is no longer supported;
\item Mac is not yet supported;
\item all controls (panel items) no longer have labels attached to them;
\item wxForm removed;
\item wxCanvasDC, wxPanelDC removed (replaced by wxClientDC, wxWindowDC, wxPaintDC which
can be used for any window);
\item wxMultiText, wxTextWindow, wxText removed and replaced by wxTextCtrl;
\item classes no longer divided into generic and platform-specific parts, for efficiency.
\end{itemize}

Additions and changes:

\begin{itemize}\itemsep=0pt
\item class hierarchy changed, and restrictions about subwindow nesting lifted;
\item header files reorganised to conform to normal C++ standards;
\item classes less dependent on each another, to reduce executable size;
\item wxString used instead of char* wherever possible;
\item the number of separate but mandatory utilities reduced;
\item the event system has been overhauled, with
virtual functions and callbacks being replaced with MFC-like event tables;
\item new controls, such as wxTreeCtrl, wxListCtrl, wxSpinButton;
\item less inconsistency about what events can be handled, so for example
mouse clicks or key presses on controls can now be intercepted;
\item the status bar is now a separate class, wxStatusBar, and is
implemented in generic wxWindows code;
\item some renaming of controls for greater consistency;
\item wxBitmap has the notion of bitmap handlers to allow for extension to new formats
without ifdefing;
\item new dialogs: wxPageSetupDialog, wxFileDialog, wxDirDialog,
wxMessageDialog, wxSingleChoiceDialog, wxTextEntryDialog;
\item GDI objects are reference-counted and are now passed to most functions
by reference, making memory management far easier;
\item wxSystemSettings class allows querying for various system-wide properties
such as dialog font, colours, user interface element sizes, and so on;
\item better platform look and feel conformance;
\item toolbar functionality now separated out into a family of classes with the
same API;
\item device contexts are no longer accessed using wxWindow::GetDC - they are created
temporarily with the window as an argument;
\item events from sliders and scrollbars can be handled more flexibly;
\item the handling of window close events has been changed in line with the new
event system, but backward {\bf OnClose} compatibility has been retained;
\item the concept of {\it validator} has been added to allow much easier coding of
the relationship between controls and application data;
\item the documentation has been revised, with more cross-referencing.
\end{itemize}

Platform-specific changes:

\begin{itemize}\itemsep=0pt
\item The Windows header file (windows.h) is no longer included by wxWindows headers;
\item wx.dll supported under Visual C++;
\item the full range of Windows 95 window decorations are supported, such as modal frame
borders;
\item MDI classes brought out of wxFrame into separate classes, and made more flexible.
\end{itemize}

\section{wxWindows requirements}\label{requirements}

To make use of wxWindows, you currently need one or both of the
following setups.

(a) PC:

\begin{enumerate}\itemsep=0pt
\item A 486 or higher PC running MS Windows.
\item One of Microsoft Visual C++, Borland C++, Watcom C++, MetroWerks C++,
Symantec C++, GNU-WIN32.
\item At least 30 MB of disk space.
\end{enumerate}

(b) UNIX:

\begin{enumerate}\itemsep=0pt
\item Almost any C++ compiler, including GNU C++.
\item Almost any UNIX workstation (VMS is supported too) and Motif 1.2 or higher (not necessary
for the Xt version)
\item At least 30 MB of disk space.
\end{enumerate}

\section{Availability and location of wxWindows}

wxWindows is currently available from the Artificial Intelligence
Applications Institute by anonymous FTP and World Wide Web:

\begin{verbatim}
  ftp://ftp.aiai.ed.ac.uk/pub/packages/wxwin
  http://web.ukonline.co.uk/julian.smart/wxwin
\end{verbatim}

\section{Acknowledgments}

Thanks are due to the AIAI for being willing to release wxWindows into
the public domain, and to our patient wives Harriet and Tanja.

The Internet has been an essential prop when coming up against tricky
problems. Thanks to those who answered our
queries or submitted bug fixes and enhancements; wxWindows is very
much a team effort.

Hermann Dunkel contributed XPM support; Arthur Seaton wrote the memory
checking code; Olaf Klein and Patrick Halke wrote the ODBC classes;
Harri Pasanen and Robin Dunn wrote wxPython and contributed to the
wxExtend library.

Markus Holzem write the Xt port. Jonathan Tonberg, Bill Hale,
Cecil Coupe, Thomaso Paoletti, Thomas Fettig, and others slaved away
writing the Mac port.  Keith Gary Boyce ported wxWindows to the free
GNU-WIN32 compiler, refusing to give up when shortcuts were suggested.

Many thanks also to: Timothy Peters, Jamshid Afshar, Patrick Albert, C. Buckley,
Robin Corbet, Harco de Hilster, Josep Fortiana, Torsten Liermann, Tatu
M\"{a}nnist\"{o}, Ian Perrigo, Giordano Pezzoli, Petr Smilauer, Neil Smith,
Kari Syst\"{a}, Jyrki Tuomi, Edward Zimmermann, Ian Brown, and many
others.

`Graphplace', the basis for the wxGraphLayout library, is copyright Dr. Jos
T.J. van Eijndhoven of Eindhoven University of Technology. The code has
been used in wxGraphLayout with his permission.

We also acknowledge the author of XFIG, the excellent UNIX drawing tool,
from the source of which we have borrowed some spline drawing code.
His copyright is included below.

{\it XFig2.1 is copyright (c) 1985 by Supoj Sutanthavibul. Permission to
use, copy, modify, distribute, and sell this software and its
documentation for any purpose is hereby granted without fee, provided
that the above copyright notice appear in all copies and that both that
copyright notice and this permission notice appear in supporting
documentation, and that the name of M.I.T. not be used in advertising or
publicity pertaining to distribution of the software without specific,
written prior permission.  M.I.T. makes no representations about the
suitability of this software for any purpose.  It is provided ``as is''
without express or implied warranty.}

\chapter{Multi-platform development with wxWindows}\label{multiplat}
\setheader{{\it CHAPTER \thechapter}}{}{}{}{}{{\it CHAPTER \thechapter}}%
\setfooter{\thepage}{}{}{}{}{\thepage}%

This chapter describes the practical details of using wxWindows. Please
see the file install.txt for up-to-date installation instructions, and
changes.txt for differences between versions.

\section{Include files}

The main include file is {\tt "wx.h"}; this includes the most commonly
used modules of wxWindows.

To save on compilation time, include only those header files relevant to the
source file. If you are using precompiled headers, you should include
the following section before any other includes:

\begin{verbatim}
// For compilers that support precompilation, includes "wx.h".
#include "wx_prec.h"

#ifdef __BORLANDC__
#pragma hdrstop
#endif

#ifndef WX_PRECOMP
... include minimum set of files necessary here ...
#endif

... now your other include files ...
\end{verbatim}

The file {\tt "wx\_prec.h"} includes {\tt "wx.h"}. Although this incantation
may seem quirky, it is in fact the end result of a lot of experimentation,
and several Windows compilers to use precompilation (those tested are Microsoft Visual C++, Borland C++
and Watcom C++).

Borland precompilation is largely automatic. Visual C++ requires specification of {\tt "wx\_prec.h"} as
the file to use for precompilation. Watcom C++ is automatic apart from the specification of
the .pch file. Watcom C++ is strange in requiring the precompiled header to be used only for
object files compiled in the same directory as that in which the precompiled header was created.
Therefore, the wxWindows Watcom C++ makefiles go through hoops deleting and recreating
a single precompiled header file for each module, thus preventing an accumulation of many
multi-megabyte .pch files.

\section{Libraries}

Under UNIX, use the library libwx\_motif.a
(Motif).  Under Windows, use the library wx.lib for stand-alone Windows
applications, or wxdll.lib for creating DLLs.

\section{Configuration}

The following lists the options configurable in the file
\rtfsp{\tt include/base/wx\_setup.h.} Some settings are a matter
of taste, some help with platform-specific problems, and
others can be set to minimize the size of the library.

\subsection{General features}

\begin{twocollist}\itemsep=0pt
\twocolitem{USE\_CLIPBOARD}{If 1, clipboard code is compiled (Windows only).}
\twocolitem{USE\_CONSTRAINTS}{If 1, the constaint-based window layout system is compiled.}
\twocolitem{USE\_DOC\_VIEW\_ARCHITECTURE}{If 1, wxDocument, wxView and related classes are compiled.}
\twocolitem{USE\_DYNAMIC\_CLASSES}{If 1, the run-time class macros and classes are compiled. Recommended,
and necessary for the document/view framework.}
\twocolitem{USE\_EXTENDED\_STATICS}{If 1, wxStaticItem code is compiled for enhanced panel decorative items.
Not rigorously tested, and not documented.}
\twocolitem{USE\_HELP}{If 1, interface to help system is compiled.}
\twocolitem{USE\_GAUGE}{If 1, the wxGauge class compiled.}
\twocolitem{USE\_GLOBAL\_MEMORY\_OPERATORS}{If 1, redefines global new and delete operators to be compatible
with the extended arguments of the debugging wxObject new and delete operators. If this causes problems
for your compiler, set to 0.}
\twocolitem{USE\_GNU\_WXSTRING}{If 1, the enhanced GNU wxString and regular expression class are compiled
in place of the normal wxString class. See contrib/wxstring for details.}
\twocolitem{USE\_IMAGE\_LOADING\_IN\_MSW}{Use code to allow dynamic .BMP loading
under MS Windows.}
\twocolitem{USE\_IMAGE\_LOADING\_IN\_X}{Use code in utils/image to allow dynamic .BMP/.GIF loading
under X.}
\twocolitem{USE\_RESOURCE\_LOADING\_IN\_MSW}{Use code to allow dynamic .ICO/.CUR loading
under MS Windows.}
\twocolitem{USE\_IPC}{If 1, interprocess communication code is compiled.}
\twocolitem{USE\_MEMORY\_TRACING}{If 1, enables debugging versions of wxObject::new and wxObject::delete
if the value of DEBUG is defined to more than 0.}
\twocolitem{USE\_METAFILE}{If 1, Windows Metafile code is compiled.}
\twocolitem{USE\_PANEL\_IN\_PANEL}{If 1, experimental panel-in-panel code is used
for common dialog boxes. Not recommended, since tab traversal can suffer.}
\twocolitem{USE\_POSTSCRIPT}{If 1, PostScript code is compiled.}
\twocolitem{USE\_POSTSCRIPT\_ARCHITECTURE\_IN\_MSW}{Set to 1 to enable the printing architecture
to make use of either native Windows printing facilities, or the wxPostScriptDC class depending
on the wxApp::SetPrintMode setting.}
\twocolitem{USE\_PRINTING\_ARCHITECTURE}{If 1, wxPrinter, wxPrintout and related classes are compiled
for the print/preview framework.}
\twocolitem{USE\_RESOURCES}{If 1, win.ini or .Xdefaults-style resource read/write code is compiled.}
\twocolitem{USE\_SCROLLBAR}{If 1, wxScrollBar class is compiled. Not rigorously tested, and not documented.}
\twocolitem{USE\_SPLINES}{If 1, spline code is compiled.}
\twocolitem{USE\_TOOLBAR}{If 1, the wxToolBar class is compiled.}
\twocolitem{USE\_TYPEDEFS}{If 1, a typedef will be used for wxPoint instead of
a class declaration, to reduce overhead and avoid a Microsoft C++ memory bug.}
\twocolitem{USE\_VLBOX}{If 1, wxVirtListBox code is compiled for a virtual listbox item.
Not rigorously tested, and not documented.}
\twocolitem{USE\_WX\_RESOURCES}{If 1, wxWindows resource file (.WXR) code is compiled.}
\twocolitem{USE\_XFIG\_SPLINE\_CODE}{If 1, XFig-derived code is used for spline
drawing. If 0, AIAI code is used, which is slower.}
\twocolitem{USE\_XPM\_IN\_X}{If 1, XPM (colour pixmap) facilities will be compiled and used
in wxBitmap under X.}
\twocolitem{USE\_XPM\_IN\_MSW}{If 1, XPM (colour pixmap) facilities will be compiled and used
in wxBitmap under MS Windows.}
\end{twocollist}

\subsection{X features}

\begin{twocollist}
\twocolitem{DEFAULT\_FILE\_SELECTOR\_SIZE}{Let Motif choose the size of
XmFileSelectionBox. Otherwise, size is 500x600.}
\twocolitem{PIXEL0\_DISABLE}{Define to disallow allocation of pixel 0 (wxXOR problem).}
\twocolitem{USE\_GADGETS}{Use gadgets where possible rather than Widgets for items.
Default is to use Gadgets.}
\twocolitem{USE\_BUTTON\_GADGET}{Use gadgets for buttons. This can intefere with
default button selection, so the default is zero.}
\end{twocollist}

\subsection{Windows and NT features}

\begin{twocollist}
\twocolitem{CTL3D}{CTL3D should only be used for 16-bit Windows programs.
On Windows 95 and NT, native 3D effects are used. If you want to
use it and don't already have CTL3D installed, copy the files in
contrib/ctl3d to appropriate places (ctl3dv2.lib/ctl3d32.lib into your compiler lib
directory, ctl3d.h into an include directory, and ctl3dv2.dll into
windows/system). You may need to find a compiler-specific version of ctl3dv2.lib
or ctl3d32.lib. Define CTL3D to be 1 in wx\_setup.h and link your executables with ctl3dv2.lib
or ctl3d32.lib.}
\twocolitem{USE\_ITSY\_BITSY}{If 1, compiles in code to support tiny window titlebars.}
\twocolitem{USE\_ODBC}{If 1, compiles wxDatabase and wxRecordSet classes for ODBC
access. Requires sql.h, sqlext.h files if set to 1 (see topic on database support).}
\end{twocollist}

\section{Makefiles}

At the moment there is no attempt to make UNIX makefiles and
PC makefiles compatible, i.e. one makefile is required for
each environment.

Sample makefiles for UNIX (suffix .UNX), MS C++ (suffix .DOS and .NT), Borland
C++ (.BCC) and Symantec C++ (.SC) are included for the library, demos
and utilities. The NT, Borland and Symantec makefiles cannot be
guaranteed to be up-to-date since the author does not have
these compilers.

The controlling makefile for wxWindows is in the platform-specific
directory, such as {\tt src/msw} or {\tt src/x}. This makefile will
recursively execute the makefile in {\tt src/base}.

\subsection{Windows makefiles}

For Microsoft C++, normally it is only necessary to type {\tt nmake -f
makefile.dos} (or an alias or batch file which does this). By default,
binaries are made with debugging information, and no optimization. Use
FINAL=1 on the command line to remove debugging information (this only
really necessary at the link stage), and DLL=1 to make a DLL version of
the library, if building a library.

\subsection{UNIX makefiles}

TODO.

Debugging information is included by default; you may add DEBUG= as an
argument to make to compile without it, or use the UNIX {\bf strip}
command to remove debugging information from an executable.

\normalbox{{\it Important note:} Most compiler flags are kept centrally in
src/make.env, which is included by all other makefiles. This is the
file to edit to tailor wxWindows compilation to your environment.}

\section{Windows-specific files}

wxWindows application compilation under MS Windows requires at least two
extra files, resource and module definition files.

\subsection{Resource file}\label{resources}

The least that must be defined in the Windows resource file (extension RC)
is the following statement:

\begin{verbatim}
rcinclude wx.rc
\end{verbatim}

which includes essential internal wxWindows definitions.  The resource script
may also contain references to icons, cursors, etc., for example:

\begin{verbatim}
wxicon icon wx.ico
\end{verbatim}

The icon can then be referenced by name when creating a frame icon. See
the MS Windows SDK documentation.

\normalbox{Note: include wx.rc {\it after} any ICON statements
so programs that search your executable for icons (such
as the Program Manager) find your application icon first.}

\subsection{Module definition file}

A module definition file (extension DEF) looks like the following:

\begin{verbatim}
NAME         Hello
DESCRIPTION  'Hello'
EXETYPE      WINDOWS
STUB         'WINSTUB.EXE'
CODE         PRELOAD MOVEABLE DISCARDABLE
DATA         PRELOAD MOVEABLE MULTIPLE
HEAPSIZE     1024
STACKSIZE    8192
\end{verbatim}

The only lines which will usually have to be changed per application are
NAME and DESCRIPTION.

\section{Memory models and memory allocation}\label{memorymodels}

Under UNIX, memory allocation isn't a problem. Under Windows, the only
really viable way to go is to use the large model, which uses the global
heap instead of the local heap for memory allocation. Unless more than
one read-write data segment is used,% (see \helpref{large data}{largedata}
large model programs may still have multiple instances under MS
C/C++ 7. Microsoft give the following guidelines for producing
multiple-instance large model programs:

\begin{itemize}\itemsep=0pt
\item Do not use {\tt /ND} to name extra data segments unless the segment is READONLY.
\item Use the .DEF file to mark extra data segments READONLY.
\item Do not use \_\_far or FAR to mark data items.
\item Use {\tt /PACKDATA} to combine data segments.
\item Use {\tt /Gt65500 /Gx} to force all data into the default data segment.
\end{itemize}

Even with the single-instance limitation, the productivity benefit is
worth it in the majority of cases. Note that some other multi-platform
class libraries also have this restriction. (If more than one instance
really is required, create several copies of the program with different
names.)

Having chosen the large model, just use C++ `new', `delete' (and if
necessary `malloc' and `free') in the normal way. The only restrictions
now encountered are a maximum of 64 KB for a single program segment and
for a single data item, unless huge model is selected.

For Borland users, use the data threshold switch, and the following is
also recommended:

\begin{itemize}\itemsep=0pt
\item Check ``Automatic Far Data Segments"
\item Check ``Put Constant Strings into Code Segment"
\end{itemize}

See also the Frequently Asked Questions document for further details
on using Borland with wxWindows.

\subsection{Allocating and deleting wxWindows objects}

In general, classes derived from wxWindow must dynamically allocated
with {\it new} and deleted with {\it delete}. If you delete a window,
all of its children and descendants will be automatically deleted,
so you don't need to delete these descendants explicitly.

Don't statically create a window unless you know that the window
cannot be deleted dynamically. Modal dialogs, such as those used
in the {\tt dialogs} sample, can usually be created statically,
if you know that the OK or Cancel button does not destroy the dialog.

Most drawing objects, such as wxPen, wxBrush, wxFont, and wxBitmap, should be
created dynamically. They are cleaned up automatically on program exit.
wxColourMap is an exception to this rule (currently). In particular,
do not attempt to create these objects globally before OnInit() has a chance
to be called, because wxWindows might not have done essential internal initialisation
(including creation of lists containing all instances of wxPen, wxBrush etc.)

If you decide to allocate a C++ array of objects (such as wxBitmap) that may
be cleaned up by wxWindows, make sure you delete the array explicitly
before wxWindows has a chance to do so on exit, since calling {\it delete} on
array members will cause memory problems.

wxColour can be created statically: it is not automatically cleaned
up and is unlikely to be shared between other objects; it is lightweight
enough for copies to be made.

Beware of deleting objects such as a wxPen or wxBitmap if they are still in use.
Windows is particularly sensitive to this: so make sure you
make calls like wxDC::SetPen(NULL) or wxDC::SelectObject(NULL) before deleting
a drawing object that may be in use. Code that doesn't do this will probably work
fine on some platforms, and then fail under Windows.

\section{Dynamic Link Libraries}

wxWindows may be used to produce DLLs which run under MS Windows. Note that
this is not the same thing as having wxWindows as a DLL, which is not
currently possible. For Microsoft C++, use the makefile with the argument DLL=1 to produce
a version of the wxWindows library which may be used in a DLL application.
There is a bug in Microsoft C++ which makes the compiler complain about returned floats,
which goes away when the {\tt /Os} option is used, which is why that flag is
set in the makefile.

For making wxWindows as a Sun dynamic library, there are comments in the
UNIX makefile for the appropriate flags for AT\&T C++. Sorry, I haven't
investigated the flags needed for other compilers.

\section{Conditional compilation}

One of the purposes of wxWindows is to reduce the need for conditional
compilation in source code, which can be messy and confusing to follow.
However, sometimes it is necessary to incorporate platform-specific
features (such as metafile use under MS Windows). The following identifiers
may be used for this purpose, along with any user-supplied ones:

\begin{itemize}
\item {\tt wx\_x} - for code which should work under any X toolkit
\item {\tt wx\_motif} - for code which should work under Motif only
\item {\tt wx\_msw} - for code which should work under Microsoft Windows only
\item {\tt wx\_xt} - for code which should work under Xt only
\end{itemize}

For example:

\begin{verbatim}
  ...
#ifdef wx_x
  (void)wxMessageBox("Sorry, metafiles not available under X.");
#endif
#ifdef wx_msw
  wxMetaFileDC dc;
  DrawIt(dc);
  wxMetaFile *mf = dc.Close();
  mf->SetClipboard();
  delete mf;
#endif
  ...
\end{verbatim}

\section{Building on-line help}

wxWindows has its own help system from version 1.30: wxHelp. It can be
used to view the wxWindows class library reference, and also to provide
on-line help for your wxWindows applications. The API, made accessible
by including {\tt wx\_help.h}, allows you to load files and display
specific sections, using DDE to communicate between the application and
wxHelp.

wxHelp files can be marked up by hand from ASCII files within wxHelp,
or may be generated from other files, as is the case with the wxWindows
documentation.

It is possible to use the platform-specific help
system (e.g. WinHelp) instead of wxHelp.

See {\tt install.txt}, the wxHelp documentation (in {\tt
utils/wxhelp/docs}) and \helpref{wxHelp}{wxhelp} for further details.

\section{C++ issues}

There are cases where a C++ program will compile and run fine under one
environment, and then fail to compile using a different compiler. Some
caveats are given below, from experience with the GNU C++ compiler (GCC)
and MS C/C++ compiler version 7.

\subsection{Templates}

wxWindows does not use templates for two main reasons: one, it is a
notoriously unportable feature, and two, the author is irrationally
suspicious of them and prefers to use casts. More compilers are
now implementing templates, and so it will probably be safe to use
them soon without fear of portability problems.

\subsection{Precompiled headers}

Some compilers, such as Borland C++ and Microsoft C++, support
precompiled headers. This can save a great deal of compiling time. The
recommended approach is to precompile {\tt ``wx.h''}, using this
precompiled header for compiling both wxWindows itself and any
wxWindows applications. For Windows compilers, two dummy source files
are provided (one for normal applications and one for creating DLLs)
to allow initial creation of the precompiled header.

However, there are several downsides to using precompiled headers. One
is that to take advantage of the facility, you often need to include
more header files than would normally be the case. This means that
changing a header file will cause more recompilations (in the case of
wxWindows, everything needs to be recompiled since everything includes
{\tt ``wx.h''}!)

A related problem is that for compilers that don't have precompiled
headers, including a lot of header files slows down compilation
considerably. For this reason, you will find (in the common
X and Windows parts of the library) conditional
compilation that under UNIX, includes a minimal set of headers;
and when using Visual C++, includes {\tt wx.h}. This should help provide
the optimal compilation for each compiler, although it is
biassed towards the precompiled headers facility available
in Microsoft C++.

\section{File handling}

When building an application which may be used under different
environments, one difficulty is coping with documents which may be
moved to different directories on other machines. Saving a file which
has pointers to full pathnames is going to be inherently unportable. One
approach is to store filenames on their own, with no directory
information.  The application searches through a number of locally
defined directories to find the file. To support this, the class {\bf
wxPathList} makes adding directories and searching for files easy, and
the global function {\bf FileNameFromPath} allows the application to
strip off the filename from the path if the filename must be stored.
This has undesirable ramifications for people who have documents of the
same name in different directories.

As regards the limitations of DOS 8+3 single-case filenames versus
unrestricted UNIX filenames, the best solution is to use DOS filenames
for your application, and also for document filenames {\it if} the user
is likely to be switching platforms regularly. Obviously this latter
choice is up to the application user to decide.  Some programs (such as
YACC and LEX) generate filenames incompatible with DOS; the best
solution here is to have your UNIX makefile rename the generated files
to something more compatible before transferring the source to DOS.
Transferring DOS files to UNIX is no problem, of course, apart from EOL
conversion for which there should be a utility available (such as
dos2unix).

See also the File Functions section of the reference manual for
descriptions of miscellaneous file handling functions.

\chapter{Utilities supplied with wxWindows}\label{utilities}
\setheader{{\it CHAPTER \thechapter}}{}{}{}{}{{\it CHAPTER \thechapter}}%
\setfooter{\thepage}{}{}{}{}{\thepage}%

A number of `extras' are supplied with wxWindows, to complement
the GUI functionality in the main class library. These are found
below the utils directory and usually have their own source, library
and documentation directories. For larger user-contributed packages,
see the directory /pub/packages/wxwin/contrib.

\section{wxHelp}\label{wxhelp}

wxHelp is a stand-alone program, written using wxWindows,
for displaying hypertext help. It is necessary since not all target
systems (notably X) supply an adequate
standard for on-line help. wxHelp is modelled on the MS Windows help
system, with contents, search and browse buttons, but does not reformat
text to suit the size of window, as WinHelp does, and its input files
are uncompressed ASCII with some embedded font commands and an .xlp
extension. Most wxWindows documentation (user manuals and class
references) is supplied in wxHelp format, and also in Windows Help
format.

Note that an application can be programmed to use Windows Help under
MS Windows, and wxHelp under X. An alternative help viewer under X is
Mosaic, a World Wide Web viewer that uses HTML as its native hypertext
format. However, this is not currently integrated with wxWindows
applications.

wxHelp works in two modes---edit and end-user. In edit mode, an ASCII
file may be marked up with different fonts and colours, and divided into
sections. In end-user mode, no editing is possible, and the user browses
principally by clicking on highlighted blocks.

When an application invokes wxHelp, subsequent sections, blocks or
files may be viewed using the same instance of wxHelp since the two
programs are linked using wxWindows interprocess communication
facilities. When the application exits, that application's instance of
wxHelp may be made to exit also.  See the {\bf wxHelpControllerBase} entry in the
reference section for how an application controls wxHelp.

\section{Tex2RTF}\label{textortf}

Supplied with wxWindows is a utility called Tex2RTF for converting\rtfsp
\LaTeX\ manuals to the following formats:

\begin{description}
\item[wxHelp]
wxWindows help system format (XLP).
\item[Linear RTF]
Rich Text Format suitable for importing into a word processor.
\item[Windows Help RTF]
Rich Text Format suitable for compiling into a WinHelp HLP file with the
help compiler.
\item[HTML]
HTML is the native format for Mosaic, the main hypertext viewer for
the World Wide Web. Since it is freely available it is a good candidate
for being the wxWindows help system under X, as an alternative to wxHelp.
\end{description}

Tex2RTF is used for the wxWindows manuals and can be used independently
by authors wishing to create on-line and printed manuals from the same\rtfsp
\LaTeX\ source.  Please see the separate documentation for Tex2RTF.

\section{wxTreeLayout}

This is a simple class library for drawing trees in a reasonably pretty
fashion. It provides only minimal default drawing capabilities, since
the algorithm is meant to be used for implementing custom tree-based
tools.

Directed graphs may also be drawn using this library, if cycles are
removed before the nodes and arcs are passed to the algorithm.

Tree displays are used in many applications: directory browsers,
hypertext systems, class browsers, and decision trees are a few
possibilities.

See the separate manual and the directory utils/wxtree.

\section{wxGraphLayout}

The wxGraphLayout class is based on a tool called `graphplace' by Dr.
Jos T.J. van Eijndhoven of Eindhoven University of Technology. Given a
(possibly cyclic) directed graph, it does its best to lay out the nodes
in a sensible manner. There are many applications (such as diagramming)
where it is required to display a graph with no human intervention. Even
if manual repositioning is later required, this algorithm can make a good
first attempt.

See the separate manual and the directory utils/wxgraph. 

\section{wxImage}\label{wximage}

This is a collection of GIF/BMP/XBM bitmap loading and displaying
routines for X.

\section{MFUTILS}\label{mfutils}

A very modest step towards reading Windows metafiles on the
any platform. Julian Smart's ClockWorks program demonstrates
how extremely simple metafiles may be read and displayed (in this
case, to be used as clock hands).

\section{Colours}\label{coloursampler}

A colour sampler for viewing colours and their names on each
platform.

%
\chapter{Tutorial}\label{tutorial}
\setheader{{\it CHAPTER \thechapter}}{}{}{}{}{{\it CHAPTER \thechapter}}%
\setfooter{\thepage}{}{}{}{}{\thepage}%

To be written.

\chapter{Programming strategies}\label{strategies}
\setheader{{\it CHAPTER \thechapter}}{}{}{}{}{{\it CHAPTER \thechapter}}%
\setfooter{\thepage}{}{}{}{}{\thepage}%

This chapter is intended to list strategies that may be useful when
writing and debugging wxWindows programs. If you have any good tips,
please submit them for inclusion here.

\section{Strategies for reducing programming errors}

\subsection{Use ASSERT}

Although I haven't done this myself within wxWindows, it is good
practice to use ASSERT statements liberally, that check for conditions that
should or should not hold, and print out appropriate error messages.
These can be compiled out of a non-debugging version of wxWindows
and your application. Using ASSERT is an example of `defensive programming':
it can alert you to problems later on.

\subsection{Use wxString in preference to character arrays}

Using wxString can be much safer and more convenient than using char *.
Again, I haven't practised what I'm preaching, but I'm now trying to use
wxString wherever possible. You can reduce the possibility of memory
leaks substantially, and it's much more convenient to use the overloaded
operators than functions such as strcmp. wxString won't add a significant
overhead to your program; the overhead is compensated for by easier
manipulation (which means less code).

The same goes for other data types: use classes wherever possible.

\section{Strategies for portability}

\subsection{Use relative positioning or constraints}

Don't use absolute panel item positioning if you can avoid it. Different GUIs have
very differently sized panel items. Consider using the constraint system, although this
can be complex to program. If you needs are simple, the default relative positioning
behaviour may be adequate (using default position values and wxPanel::NewLine).

Alternatively, you could use alternative .wrc (wxWindows resource files) on different
platforms, with slightly different dimensions in each. Or space your panel items out
to avoid problems.

\subsection{Use wxWindows resource files}

Use .wrc (wxWindows resource files) where possible, because they can be easily changed
independently of source code. Bitmap resources can be set up to load different
kinds of bitmap depending on platform (see the section on resource files).

\section{Strategies for debugging}

\subsection{Positive thinking}

It's common to blow up the problem in one's imagination, so that it seems to threaten
weeks, months or even years of work. The problem you face may seem insurmountable:
but almost never is. Once you have been programming for some time, you will be able
to remember similar incidents that threw you into the depths of despair. But
remember, you always solved the problem, somehow!

Perseverance is often the key, even though a seemingly trivial problem
can take an apparently inordinate amount of time to solve. In the end,
you will probably wonder why you worried so much. That's not to say it
isn't painful at the time. Try not to worry -- there are many more important
things in life.

\subsection{Simplify the problem}

Reduce the code exhibiting the problem to the smallest program possible
that exhibits the problem. If it is not possible to reduce a large and
complex program to a very small program, then try to ensure your code
doesn't hide the problem (you may have attempted to minimize the problem
in some way: but now you want to expose it).

With luck, you can add a small amount of code that causes the program
to go from functioning to non-functioning state. This should give a clue
to the problem. In some cases though, such as memory leaks or wrong
deallocation, this can still give totally spurious results!

\subsection{Genetic mutation}

If we had sophisticated genetic algorithm tools that could be applied
to programming, we could use them. Until then, a common -- if rather irrational --
technique is to just make arbitrary changes to the code until something
different happens. You may have an intuition why a change will make a difference;
otherwise, just try altering the order of code, comment lines out, anything
to get over an impasse. Obviously, this is usually a last resort.

\subsection{Use a debugger}

This sounds like facetious advice, but it's surprising how often people
don't use a debugger. Often it's an overhead to install or learn how to
use a debugger, but it really is essential for anything but the most
trivial programs. Some platforms don't allow for debugging, such
as WIN32s under Windows 3.x. In this case, you might be advised to
debug under 16-bit Windows and when you're confident, compile for
WIN32s. In fact WIN32s can be very strict about bad memory handling,
so testing out under WIN32s is a good thing to do even if you're
not going to distribute this version. (Unless you've got a good memory checking,
utility, of course!) Tracking bugs under WIN32s can involve a lot of debug message
insertion and relinking, so make sure your compiler has a fast linker
(e.g. Watcom, Symantec).

\subsection{Use tracing code}

You can use wxDebugMsg statements (or the wxDebugStreamBuf class) to
output to a debugging window such as DBWIN under Windows, or standard
error under X. If compiling in DEBUG mode, you can use TRACE statements
that will be compiled out of the final build of your application.

Using tracing statements may be more convenient than using the debugger
in some circumstances (such as when your debugger doesn't support a lot
of debugging code, or you wish to print a bunch of variables).

\subsection{Use wxObject::Dump and the wxDebugContext class}

It's good practice to implement the Dump member function for all
classes derived from wxObject. You can then make use of wxDebugContext
to dump out information on all objects in the program, if DEBUG is
defined to be more than zero. You can use wxDebugContext to check for
memory leaks and corrupt memory. See the debugging topic in the
reference manual for more information.

\subsection{Check Windows debug messages}

Under Windows, it's worth running your program with DBWIN running or
some other program that shows Windows-generated debug messages. It's
possible it'll show invalid handles being used. You may have fun seeing
what commercial programs cause these normally hidden errors! Microsoft
recommend using the debugging version of Windows, which shows up even
more problems. However, I doubt it's worth the hassle for most
applications. wxWindows is designed to minimize the possibility of such
errors, but they can still happen occasionally, slipping through unnoticed
because they are not severe enough to cause a crash.
