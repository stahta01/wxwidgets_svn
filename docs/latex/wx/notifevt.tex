\section{\class{wxNotifyEvent}}\label{wxnotifyevent}

This class is not used by the event handlers by itself, but is a base class
for other event classes (such as \helpref{wxNotebookEvent}{wxnotebookevent}).

It (or an object of a derived class) is sent when the controls state is being
changed and allows the program to \helpref{Veto()}{wxnotifyeventveto} this
change if it wants to prevent it from happening.

\wxheading{Derived from}

\helpref{wxCommandEvent}{wxcommandevent}\\
\helpref{wxEvent}{wxevent}\\
\helpref{wxEvtHandler}{wxevthandler}\\
\helpref{wxObject}{wxobject}

\wxheading{Include files}

<wx/event.h>

\wxheading{Event table macros}

None

\wxheading{See also}

\helpref{wxNotebookEvent}{wxnotebookevent}

\latexignore{\rtfignore{\wxheading{Members}}}

\membersection{wxNotifyEvent::wxNotifyEvent}\label{wxnotifyeventconstr}

\func{}{wxNotifyEvent}{\param{wxEventType}{ eventType = wxEVT\_NULL}, \param{int}{ id = 0}}

Constructor (used internally by wxWindows only).

\membersection{wxNotifyEvent::Allow}\label{wxnotifyeventallow}

\func{void}{Allow}{\void}

This is the opposite of \helpref{Veto()}{wxnotifyeventveto}: it explicitly
allows the event to be processed. For most events it is not necessary to call
this method as the events are allowed anyhow but some are forbidden by default
(this will be mentioned in the corresponding event description).

\membersection{wxNotifyEvent::IsAllowed}\label{wxnotifyeventisallowed}

\constfunc{bool}{IsAllowed}{\void}

Returns TRUE if the change is allowed (\helpref{Veto()}{wxnotifyeventveto} 
hasn't been called) or FALSE otherwise (if it was).

\membersection{wxNotifyEvent::Veto}\label{wxnotifyeventveto}

\func{void}{Veto}{\void}

Prevents the change announced by this event from happening.

It is in general a good idea to notify the user about the reasons for vetoing
the change because otherwise the applications behaviour (which just refuses to
do what the user wants) might be quite surprising.

