\chapter{Porting from wxWindows 1.xx}\label{porting}

This addendum gives guidelines and tips for porting applications from
version 1.xx of wxWindows to version 2.0.

The first section offers tips for writing 1.xx applications in a way to
minimize porting time. The following sections detail the changes and
how you can modify your application to be 2.0-compliant.

You may be worrying that porting to 2.0 will be a lot of work,
particularly if you have only recently started using 1.xx. In fact,
the wxWindows 2.0 API has far more in common with 1.xx than it has differences.
The main challenges are using the new event system, doing without the default
panel item layout, and the lack of automatic labels in some controls.

Please don't be freaked out by the jump to 2.0! For one thing, 1.xx is still available
and will be supported by the user community for some time. And when you have
changed to 2.0, we hope that you will appreciate the benefits in terms
of greater flexibility, better user interface aesthetics, improved C++ conformance,
improved compilation speed, and many other enhancements. The revised architecture
of 2.0 will ensure that wxWindows can continue to evolve for the forseeable
future.

{\it Please note that this document is a work in progress.}

\section{Preparing for version 2.0}\label{portingpreparing}

Even before compiling with version 2.0, there's also a lot you can do right now to make porting
relatively simple. Here are a few tips.

\begin{itemize}
\item {\bf Use constraints or .wxr resources} for layout, rather than the default layout scheme.
Constraints should be the same in 2.0, and resources will be translated.
\item {\bf Use separate wxMessage items} instead of labels for wxText, wxMultiText,
wxChoice, wxComboBox. These labels will disappear in 2.0. Use separate
wxMessages whether you're creating controls programmatically or using
the dialog editor. The future dialog editor will be able to translate
from old to new more accurately if labels are separated out.
\item {\bf Parameterise functions that use wxDC} or derivatives, i.e. make the wxDC
an argument to all functions that do drawing. Minimise the use of
wxWindow::GetDC and definitely don't store wxDCs long-term
because in 2.0, you can't use GetDC() and wxDCs are not persistent.
You will use wxClientDC, wxPaintDC stack objects instead. Minimising
the use of GetDC() will ensure that there are very few places you
have to change drawing code for 2.0.
\item {\bf Don't set GDI objects} (wxPen, wxBrush etc.) in windows or wxCanvasDCs before they're
needed (e.g. in constructors) - do so within your drawing routine instead. In
2.0, these settings will only take effect between the construction and destruction
of temporary wxClient/PaintDC objects.
\item {\bf Don't rely} on arguments to wxDC functions being floating point - they will
be 32-bit integers in 2.0.
\item {\bf Don't use the wxCanvas member functions} that duplicate wxDC functions, such as SetPen and DrawLine, since
they are going.
\item {\bf Using member callbacks} called from global callback functions will make the transition
easier - see the FAQ
for some notes on using member functions for callbacks. wxWindows 2.0 will banish global
callback functions (and OnMenuCommand), and nearly all event handling will be done by functions taking a single event argument.
So in future you will have code like:

{\small\begin{verbatim}
void MyFrame::OnOK(wxCommandEvent& event)
{
        ...
}
\end{verbatim}
}%

You may find that writing the extra code to call a member function isn't worth it at this stage,
but the option is there.
\item {\bf Use wxString wherever possible.} 2.0 replaces char * with wxString
in most cases, and if you use wxString to receive strings returned from
wxWindows functions (except when you need to save the pointer if deallocation is required), there should
be no conversion problems later on.
\item Be aware that under Windows, {\bf font sizes will change} to match standard Windows
font sizes (for example, a 12-point font will appear bigger than before). Write your application
to be flexible where fonts are concerned.
Don't rely on fonts being similarly-sized across platforms, as they were (by chance) between
Windows and X under wxWindows 1.66. Yes, this is not easy... but I think it's better to conform to the
standards of each platform, and currently the size difference makes it difficult to
conform to Windows UI standards. You may eventually wish to build in a global 'fudge-factor' to compensate
for size differences. The old font sizing will still be available via wx\_setup.h, so do not panic...
\item {\bf Consider dropping wxForm usage}:
wxPropertyFormView can be used in a wxForm-like way, except that you specify a pre-constructed panel
or dialog; or you can use a wxPropertyListView to show attributes in a scrolling list - you don't even need
to lay panel items out.

Because wxForm uses a number of features to be dropped in wxWindows 2.0, it cannot be
supported in the future, at least in its present state.
\item {\bf When creating a wxListBox}, put the wxLB\_SINGLE, wxLB\_MULTIPLE, wxLB\_EXTENDED styles in the window style parameter, and put
zero in the {\it multiple} parameter. The {\it multiple} parameter will be removed in 2.0.
\item {\bf For MDI applications}, don't reply on MDI being run-time-switchable in the way that the
MDI sample is. In wxWindows 2.0, MDI functionality is separated into distinct classes.
\end{itemize}

\section{The new event system}\label{portingeventsystem}

The way that events are handled has been radically changed in wxWindows 2.0. Please
read the topic `Event handling overview' in the wxWindows 2.0 manual for background
on this.

\subsection{Callbacks}

Instead of callbacks for panel items, menu command events, control commands and other events are directed to
the originating window, or an ancestor, or an event handler that has been plugged into the window
or its ancestor. Event handlers always have one argument, a derivative of wxEvent.

For menubar commands, the {\bf OnMenuCommand} member function will be replaced by a series of separate member functions,
each of which responds to a particular command. You need to add these (non-virtual) functions to your
frame class, add a DECLARE\_EVENT\_TABLE entry to the class, and then add an event table to
your implementation file, as a BEGIN\_EVENT\_TABLE and END\_EVENT\_TABLE block. The
individual event mapping macros will be of the form:

\begin{verbatim}
BEGIN_EVENT_TABLE(MyFrame, wxFrame)
    EVT_MENU(MYAPP_NEW, MyFrame::OnNew)
    EVT_MENU(wxID_EXIT, MyFrame::OnExit)
END_EVENT_TABLE()
\end{verbatim}

Control commands, such as button commands, can be routed to a derived button class,
the parent window, or even the frame. Here, you use a function of the form EVT\_BUTTON(id, func).
Similar macros exist for other control commands.

\subsection{Other events}

To intercept other events, you used to override virtual functions, such as OnSize. Now, while you can use
the OnSize name for such event handlers (or any other name of your choice), it has only a single argument
(wxSizeEvent) and must again be `mapped' using the EVT\_SIZE macro. The same goes for all other events,
including OnClose (although in fact you can still use the old, virtual form of OnClose for the time being).

\section{Class hierarchy}\label{portingclasshierarchy}

The class hierarchy has changed somewhat. wxToolBar and wxButtonBar
classes have been split into several classes, and are derived from wxControl (which was
called wxItem). wxPanel derives from wxWindow instead of from wxCanvas, which has
disappeared in favour of wxScrolledWindow (since all windows are now effectively canvases
which can be drawn into). The status bar has become a class in its own right, wxStatusBar.

There are new MDI classes so that wxFrame does not have to be overloaded with this
functionality.

There are new device context classes, with wxPanelDC and wxCanvasDC disappearing.
See \helpref{Device contexts and painting}{portingdc}.

\section{GDI objects}\label{portinggdiobjects}

These objects - instances of classes such as wxPen, wxBrush, wxBitmap (but not wxColour) -
are now implemented with reference-counting. This makes assignment a very cheap operation,
and also means that management of the resource is largely automatic. You now pass {\it references} to
objects to functions such as wxDC::SetPen, not pointers, so you will need to derefence your pointers.
The device context does not store a copy of the pen
itself, but takes a copy of it (via reference counting), and the object's data gets freed up
when the reference count goes to zero. The application does not have to worry so much about
who the object belongs to: it can pass the reference, then destroy the object without
leaving a dangling pointer inside the device context.

For the purposes of code migration, you can use the old style of object management - maintaining
pointers to GDI objects, and using the FindOrCreate... functions. However, it is preferable to
keep this explicit management to a minimum, instead creating objects on the fly as needed, on the stack,
unless this causes too much of an overhead in your application.

At a minimum, you will have to make sure that calls to SetPen, SetBrush etc. work. Also, where you pass NULL to these
functions, you will need to use an identifier such as wxNullPen or wxNullBrush.

\section{Dialogs and controls}\label{portingdialogscontrols}

\wxheading{Labels}

Most controls no longer have labels and values as they used to in 1.xx. Instead, labels
should be created separately using wxStaticText (the new name for wxMessage). This will
need some reworking of dialogs, unfortunately; programmatic dialog creation that doesn't
use constraints will be especially hard-hit. Perhaps take this opportunity to make more
use of dialog resources or constraints. Or consider using the wxPropertyListView class
which can do away with dialog layout issues altogether by presenting a list of editable
properties.

\wxheading{Constructors}

All window constructors have two main changes, apart from the label issue mentioned above.
Windows now have integer identifiers; and position and size are now passed as wxPoint and
wxSize objects. In addition, some windows have a wxValidator argument.

\wxheading{Show versus ShowModal}

If you have used or overridden the {\bf wxDialog::Show} function in the past, you may find
that modal dialogs no longer work as expected. This is because the function for modal showing
is now {\bf wxDialog:ShowModal}. This is part of a more fundamental change in which a
control may tell the dialog that it caused the dismissal of a dialog, by
calling {\bf wxDialog::EndModal} or {\bf wxWindow::SetReturnCode}. Using this
information, {\bf ShowModal} now returns the id of the control that caused dismissal,
giving greater feedback to the application than just TRUE or FALSE.

If you overrode or called {\bf wxDialog::Show}, use {\bf ShowModal} and test for a returned identifier,
commonly wxID\_OK or wxID\_CANCEL.

\wxheading{wxItem}

This is renamed wxControl.

\wxheading{wxText, wxMultiText and wxTextWindow}

These classes no longer exist and are replaced by the single class wxTextCtrl.
Multi-line text items are created using the wxTE\_MULTILINE style.

\wxheading{wxButton}

Bitmap buttons are now a separate class, instead of being part of wxBitmap.

\wxheading{wxMessage}

Bitmap messages are now a separate class, wxStaticBitmap, and wxMessage
is renamed wxStaticText.

\wxheading{wxGroupBox}

wxGroupBox is renamed wxStaticBox.

\wxheading{wxForm}

Note that wxForm is no longer supported in wxWindows 2.0. Consider using the wxPropertyFormView class
instead, which takes standard dialogs and panels and associates controls with property objects.
You may also find that the new validation method, combined with dialog resources, is easier
and more flexible than using wxForm.

\section{Device contexts and painting}\label{portingdc}

In wxWindows 2.0, device contexts are used for drawing into, as per 1.xx, but the way
they are accessed and constructed is a bit different.

You no longer use {\bf GetDC} to access device contexts for panels, dialogs and canvases.
Instead, you create a temporary device context, which means that any window or control can be drawn
into. The sort of device context you create depends on where your code is called from. If
painting within an {\bf OnPaint} handler, you create a wxPaintDC. If not within an {\bf OnPaint} handler,
you use a wxClientDC or wxWindowDC. You can still parameterise your drawing code so that it
doesn't have to worry about what sort of device context to create - it uses the DC it is passed
from other parts of the program.

You {\bf must } create a wxPaintDC if you define an OnPaint handler, even if you do not
actually use this device context, or painting will not work correctly under Windows.

If you used device context functions with wxPoint or wxIntPoint before, please note
that wxPoint now contains integer members, and there is a new class wxRealPoint. wxIntPoint
no longer exists.

wxMetaFile and wxMetaFileDC have been renamed to wxMetafile and wxMetafileDC.

\section{Miscellaneous}

\subsection{Strings}

wxString has replaced char* in the majority of cases. For passing strings into functions,
this should not normally require you to change your code if the syntax is otherwise the
same. This is because C++ will automatically convert a char* or const char* to a wxString by virtue
of appropriate wxString constructors.

However, when a wxString is returned from a function in wxWindows 2.0 where a char* was
returned in wxWindows 1.xx, your application will need to be changed. Usually you can
simplify your application's allocation and deallocation of memory for the returned string,
and simply assign the result to a wxString object. For example, replace this:

{\small\begin{verbatim}
  char* s = wxFunctionThatReturnsString();
  s = copystring(s); // Take a copy in case it's temporary
  .... // Do something with it
  delete[] s;
\end{verbatim}
}

with this:

{\small\begin{verbatim}
  wxString s = wxFunctionThatReturnsString();
  .... // Do something with it
\end{verbatim}
}

To indicate an empty return value or a problem, a function may return either the
empty string (``") or a null string. You can check for a null string with wxString::IsNull().

\subsection{Use of const}

The {\bf const} keyword is now used to denote constant functions that do not affect the
object, and for function arguments to denote that the object passed cannot be changed.

This should not affect your application except for where you are overriding virtual functions
which now have a different signature. If functions are not being called which were previously,
check whether there is a parameter mismatch (or function type mismatch) involving consts.

Try to use the {\bf const} keyword in your own code where possible.

\section{Backward compatibility}\label{portingcompat}

Some wxWindows 1.xx functionality has been left to ease the transition to 2.0. This functionality
(usually) only works if you compile with WXWIN\_COMPATIBILITY set to 1 in setup.h.

Mostly this defines old names to be the new names (e.g. wxRectangle is defined to be wxRect).

\section{Quick reference}\label{portingquickreference}

This section allows you to quickly find features that
need to be converted.

\subsection{Include files}

Use the form:

\begin{verbatim}
#include <wx/wx.h>
#include <wx/button.h>
\end{verbatim}

For precompiled header support, use this form:

\begin{verbatim}
// For compilers that support precompilation, includes "wx.h".
#include <wx/wxprec.h>

#ifdef __BORLANDC__
    #pragma hdrstop
#endif

// Any files you want to include if not precompiling by including
// the whole of <wx/wx.h>
#ifndef WX_PRECOMP
    #include <stdio.h>
    #include <wx/setup.h>
    #include <wx/bitmap.h>
    #include <wx/brush.h>
#endif

// Any files you want to include regardless of precompiled headers
#include <wx/toolbar.h>
\end{verbatim}

\subsection{IPC classes}

These are now separated out into wxDDEServer/Client/Connection (Windows only) and wxTCPServer/Client/Connection
(Windows and Unix). Take care to use wxString for your overridden function arguments, instead of char*, as per
the documentation.

\subsection{MDI style frames}

MDI is now implemented as a family of separate classes, so you can't switch to MDI just by
using a different frame style. Please see the documentation for the MDI frame classes, and the MDI
sample may be helpful too.

\subsection{OnActivate}

Replace the arguments with one wxActivateEvent\& argument, make sure the function isn't virtual,
and add an EVT\_ACTIVATE event table entry.

\subsection{OnChar}

This is now a non-virtual function, with the same wxKeyEvent\& argument as before.
Add an EVT\_CHAR macro to the event table
for your window, and the implementation of your function will need very few changes.

\subsection{OnClose}

The old virtual function OnClose is now obsolete.
Add an OnCloseWindow event handler using an EVT\_CLOSE event table entry. For details
about window destruction, see the Windows Deletion Overview in the manual. This is a subtle
topic so please read it very carefully. Basically, OnCloseWindow is now responsible for
destroying a window with Destroy(), but the default implementation (for example for wxDialog) may not
destroy the window, so to be sure, always provide this event handler so it's obvious what's going on.

\subsection{OnEvent}

This is now a non-virtual function, with the same wxMouseEvent\& argument as before. However
you may wish to rename it OnMouseEvent. Add an EVT\_MOUSE\_EVENTS macro to the event table
for your window, and the implementation of your function will need very few changes.
However, if you wish to intercept different events using different functions, you can
specify specific events in your event table, such as EVT\_LEFT\_DOWN.

Your OnEvent function is likely to have references to GetDC(), so make sure you create
a wxClientDC instead. See \helpref{Device contexts}{portingdc}.

If you are using a wxScrolledWindow (formerly wxCanvas), you should call
PrepareDC(dc) to set the correct translation for the current scroll position.

\subsection{OnMenuCommand}

You need to replace this virtual function with a series of non-virtual functions, one for
each case of your old switch statement. Each function takes a wxCommandEvent\& argument.
Create an event table for your frame
containing EVT\_MENU macros, and insert DECLARE\_EVENT\_TABLE() in your frame class, as
per the samples.

\subsection{OnPaint}

This is now a non-virtual function, with a wxPaintEvent\& argument.
Add an EVT\_PAINT macro to the event table
for your window.

Your function {\it must} create a wxPaintDC object, instead of using GetDC to
obtain the device context.

If you are using a wxScrolledWindow (formerly wxCanvas), you should call
PrepareDC(dc) to set the correct translation for the current scroll position.

\subsection{OnSize}

Replace the arguments with one wxSizeEvent\& argument, make it non-virtual, and add to your
event table using EVT\_SIZE.

\subsection{wxApp definition}

The definition of OnInit has changed. Return a bool value, not a wxFrame.

Also, do {\it not} declare a global application object. Instead, use the macros
DECLARE\_APP and IMPLEMENT\_APP as per the samples. Remove any occurrences of IMPLEMENT\_WXWIN\_MAIN:
this is subsumed in IMPLEMENT\_APP.

\subsection{wxButton}

For bitmap buttons, use wxBitmapButton.

\subsection{wxCanvas}

Change the name to wxScrolledWindow.

\subsection{wxDialogBox}

Change the name to wxDialog, and for modal dialogs, use ShowModal instead of Show.

\subsection{wxDialog::Show}

If you used {\bf Show} to show a modal dialog or to override the standard
modal dialog {\bf Show}, use {\bf ShowModal} instead.

\wxheading{See also}

\helpref{Dialogs and controls}{portingdialogscontrols}

\subsection{wxForm}

Sorry, this class is no longer available. Try using the wxPropertyListView or wxPropertyFormView class
instead, or use .wxr files and validators.

\subsection{wxPoint}

The old wxPoint is called wxRealPoint, and wxPoint now uses integers.

\subsection{wxRectangle}

This is now called wxRect.

\subsection{wxScrollBar}

The function names have changed for this class: please refer to the documentation for wxScrollBar. Instead
of setting properties individually, you will call SetScrollbar with several parameters.

\subsection{wxText, wxMultiText, wxTextWindow}

Change all these to wxTextCtrl. Add the window style wxTE\_MULTILINE if you
wish to have a multi-line text control.

\subsection{wxToolBar}

This name is an alias for the most popular form of toolbar for your platform. There is now a family
of toolbar classes, with for example wxToolBar95, wxToolBarMSW and wxToolBarSimple classes existing
under Windows 95.

Toolbar management is supported by frames, so calling wxFrame::CreateToolBar and adding tools is usually
enough, and the SDI or MDI frame will manage the positioning for you. The client area of the frame is the space
left over when the menu bar, toolbar and status bar have been taken into account.

