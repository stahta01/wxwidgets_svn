%%%%%%%%%%%%%%%%%%%%%%%%%%%%%%%%%%%%%%%%%%%%%%%%%%%%%%%%%%%%%%%%%%%%%%%%%%%%%%%
%% Name:        convauto.tex
%% Purpose:     wxConvAuto documentation
%% Author:      Vadim Zeitlin
%% Created:     2007-08-26
%% RCS-ID:      $Id:$
%% Copyright:   (c) 2007 Vadim Zeitlin <vadim@wxwidgets.org>
%% License:     wxWindows license
%%%%%%%%%%%%%%%%%%%%%%%%%%%%%%%%%%%%%%%%%%%%%%%%%%%%%%%%%%%%%%%%%%%%%%%%%%%%%%%

\section{\class{wxConvAuto}}\label{wxconvauto}

This class implements a Unicode to/from multibyte converter capable of
automatically recognizing the encoding of the multibyte text on input. The
logic used is very simple: the class uses the BOM (byte order mark) if it's
present and tries to interpret the input as UTF-8 otherwise. If this fails, the
input is interpreted as being in the default multibyte encoding which can be
specified in the constructor of a wxConvAuto instance and, in turn, defaults to
the value of \helpref{GetFallbackEncoding}{wxconvautogetdefaultmbencoding} if
not explicitly given.

For the conversion from Unicode to multibyte, the same encoding as was
previously used for multibyte to Unicode conversion is reused. If there had
been no previous multibyte to Unicode conversion, UTF-8 is used by default.
Notice that once the multibyte encoding is automatically detected, it doesn't
change any more, i.e. it is entirely determined by the first use of wxConvAuto
object in the multibyte-to-Unicode direction. However creating a copy of
wxConvAuto object, either via the usual copy constructor or assignment
operator, or using \helpref{Clone}{wxmbconvclone} method, resets the
automatically detected encoding so that the new copy will try to detect the
encoding of the input on first use.

This class is used by default in wxWidgets classes and functions reading text
from files such as \helpref{wxFile}{wxfile}, \helpref{wxFFile}{wxffile}, 
\helpref{wxTextFile}{wxtextfile}, \helpref{wxFileConfig}{wxfileconfig} and
various stream classes so the encoding set with its 
\helpref{SetFallbackEncoding}{wxconvautosetdefaultmbencoding} method will
affect how these classes treat input files. In particular, use this method
to change the fall-back multibyte encoding used to interpret the contents of
the files whose contents isn't valid UTF-8 or to disallow it completely.

\wxheading{Derived from}

\helpref{wxMBConv}{wxmbconv}

\wxheading{Include files}

<wx/convauto.h>

\wxheading{Library}

\helpref{wxBase}{librarieslist}

\wxheading{See also}

\helpref{wxMBConv classes overview}{mbconvclasses}


\latexignore{\rtfignore{\wxheading{Members}}}

\membersection{wxConvAuto::wxConvAuto}\label{wxconvautowxconvauto}

\func{}{wxConvAuto}{\param{wxFontEncoding }{enc = wxFONTENCODING\_DEFAULT}}

Constructs a new wxConvAuto instance. The object will try to detect the input
of the multibyte text given to its \helpref{ToWChar}{wxmbconvtowchar} method
automatically but if the automatic detection of Unicode encodings fails, the
fall-back encoding \arg{enc} will be used to interpret it as multibyte text.
The default value of this parameter, \texttt{wxFONTENCODING\_DEFAULT} means
that the global default value which can be set using
\helpref{SetFallbackEncoding}{wxconvautosetdefaultmbencoding} method should be
used. As with that method, passing \texttt{wxFONTENCODING\_MAX} inhibits using
this encoding completely so the input multibyte text will always be interpreted
as UTF-8 in the absence of BOM and the conversion will fail if the input
doesn't form valid UTF-8 sequence. Another special value is
\texttt{wxFONTENCODING\_SYSTEM} which means to use the encoding currently used
on the user system, i.e. the encoding returned by
\helpref{wxLocale::GetSystemEncoding}{wxlocalegetsystemencoding}. Any other
encoding will be used as is, e.g. passing \texttt{wxFONTENCODING\_ISO8859\_1} 
ensures that non-UTF-8 input will be treated as latin1.


\membersection{wxConvAuto::DisableFallbackEncoding}\label{wxconvautodisablefallbackencoding}

\func{static void}{DisableFallbackEncoding}{\void}

Disable the use of the fall back encoding: if the input doesn't have a BOM and
is not valid UTF-8, the conversion will fail.


\membersection{wxConvAuto::GetFallbackEncoding}\label{wxconvautogetdefaultmbencoding}

\func{static wxFontEncoding}{GetFallbackEncoding}{\void}

Returns the encoding used by default by wxConvAuto if no other encoding is
explicitly specified in constructor. By default, returns 
\texttt{wxFONTENCODING\_ISO8859\_1} but can be changed using 
\helpref{SetFallbackEncoding}{wxconvautosetdefaultmbencoding} method.


\membersection{wxConvAuto::SetFallbackEncoding}\label{wxconvautosetdefaultmbencoding}

\func{static void}{SetFallbackEncoding}{\param{wxFontEncoding }{enc}}

Changes the encoding used by default by wxConvAuto if no other encoding is
explicitly specified in constructor. The default value, which can be retrieved
using \helpref{GetFallbackEncoding}{wxconvautogetdefaultmbencoding}, is 
\texttt{wxFONTENCODING\_ISO8859\_1}.

Special values of \texttt{wxFONTENCODING\_SYSTEM} or 
\texttt{wxFONTENCODING\_MAX} can be used for \arg{enc} parameter to use the
encoding of the current user locale as fall back or not use any encoding for
fall back at all, respectively (just as with the similar constructor
parameter). However \texttt{wxFONTENCODING\_DEFAULT} value cannot be used here.

