\section{\class{wxGLCanvas}}\label{wxglcanvas}

wxGLCanvas is a class for displaying OpenGL graphics. It is always used in
conjunction with \helpref{wxGLContext}{wxglcontext} as the context can only be
be made current (i.e. active for the OpenGL commands) when it is associated to
a wxGLCanvas.

More precisely, you first need to create a wxGLCanvas window and then create an
instance of a \helpref{wxGLContext}{wxglcontext} that is initialized with this
wxGLCanvas and then later use either \helpref{wxGLCanvas::SetCurrent}{wxglcanvassetcurrent} 
with the instance of the \helpref{wxGLContext}{wxglcontext} or 
\helpref{wxGLContext::SetCurrent}{wxglcontextsetcurrent} with the instance of
the \helpref{wxGLCanvas}{wxglcanvas} (which might be not the same as was used
for the creation of the context) to bind the OpenGL state that is represented
by the rendering context to the canvas, and then finally call 
\helpref{wxGLCanvas::SwapBuffers}{wxglcanvasswapbuffers} to swap the buffers of
the OpenGL canvas and thus show your current output.

Notice that previous versions of wxWidgets used to implicitly create a
wxGLContext inside wxGLCanvas itself. This is still supported in the current
version but is deprecated now and will be removed in the future, please update
your code to create the rendering contexts explicitly.

To set up the attributes for the canvas (number of bits for the depth buffer,
number of bits for the stencil buffer and so on) you should set up the correct values of
the {\it attribList} parameter. The values that should be set up and their meanings will be described below.

Notice that OpenGL is not enabled by default. To switch it on, you need to edit
setup.h under Windows and set {\tt wxUSE\_GLCANVAS} to $1$ (you may also need
to have to add {\tt opengl32.lib} and {\tt glu32.lib} to the list of libraries
your program is linked with). On Unix, pass {\tt --with-opengl} to configure.

\wxheading{Derived from}

\helpref{wxWindow}{wxwindow}\\
\helpref{wxEvtHandler}{wxevthandler}\\
\helpref{wxObject}{wxobject}

\wxheading{Include files}

<wx/glcanvas.h>

\wxheading{Library}

\helpref{wxGl}{librarieslist}

\wxheading{Window styles}

There are no specific window styles for this class.

See also \helpref{window styles overview}{windowstyles}.

\wxheading{Constants}

The generic GL implementation doesn't support many of these options, such as stereo, auxiliary buffers,
alpha channel, and accum buffer. Other implementations may support them.

\twocolwidtha{5cm}
\begin{twocollist}\itemsep=0pt
\twocolitem{\windowstyle{WX\_GL\_RGBA}}{Use true colour}
\twocolitem{\windowstyle{WX\_GL\_BUFFER\_SIZE}}{Bits for buffer if not WX\_GL\_RGBA}
\twocolitem{\windowstyle{WX\_GL\_LEVEL}}{0 for main buffer, >0 for overlay, <0 for underlay}
\twocolitem{\windowstyle{WX\_GL\_DOUBLEBUFFER}}{Use doublebuffer}
\twocolitem{\windowstyle{WX\_GL\_STEREO}}{Use stereoscopic display}
\twocolitem{\windowstyle{WX\_GL\_AUX\_BUFFERS}}{Number of auxiliary buffers (not all implementations support this option)}
\twocolitem{\windowstyle{WX\_GL\_MIN\_RED}}{Use red buffer with most bits (> MIN\_RED bits)}
\twocolitem{\windowstyle{WX\_GL\_MIN\_GREEN}}{Use green buffer with most bits (> MIN\_GREEN bits) }
\twocolitem{\windowstyle{WX\_GL\_MIN\_BLUE}}{Use blue buffer with most bits (> MIN\_BLUE bits) }
\twocolitem{\windowstyle{WX\_GL\_MIN\_ALPHA}}{Use alpha buffer with most bits (> MIN\_ALPHA bits)}
\twocolitem{\windowstyle{WX\_GL\_DEPTH\_SIZE}}{Bits for Z-buffer (0,16,32)}
\twocolitem{\windowstyle{WX\_GL\_STENCIL\_SIZE}}{Bits for stencil buffer}
\twocolitem{\windowstyle{WX\_GL\_MIN\_ACCUM\_RED}}{Use red accum buffer with most bits (> MIN\_ACCUM\_RED bits)}
\twocolitem{\windowstyle{WX\_GL\_MIN\_ACCUM\_GREEN}}{Use green buffer with most bits (> MIN\_ACCUM\_GREEN bits)}
\twocolitem{\windowstyle{WX\_GL\_MIN\_ACCUM\_BLUE}}{Use blue buffer with most bits (> MIN\_ACCUM\_BLUE bits)}
\twocolitem{\windowstyle{WX\_GL\_MIN\_ACCUM\_ALPHA}}{Use blue buffer with most bits (> MIN\_ACCUM\_ALPHA bits)}
\end{twocollist}

\wxheading{See also}

\helpref{wxGLContext}{wxglcontext}

\latexignore{\rtfignore{\wxheading{Members}}}


\membersection{wxGLCanvas::wxGLCanvas}\label{wxglcanvasconstr}

\func{void}{wxGLCanvas}{\param{wxWindow* }{parent}, \param{wxWindowID}{ id = wxID\_ANY},
 \param{const int*}{ attribList = NULL},
 \param{const wxPoint\&}{ pos = wxDefaultPosition}, \param{const wxSize\&}{ size = wxDefaultSize},
 \param{long}{ style=0}, \param{const wxString\& }{name="GLCanvas"},
 \param{const wxPalette\&}{ palette = wxNullPalette}}

Creates a window with the given parameters. Notice that you need to create and
use a \helpref{wxGLContext}{wxglcontext} to output to this window.

If \arg{attribList} is not specified, double buffered RGBA mode is used.

\docparam{parent}{Pointer to a parent window.}

\docparam{id}{Window identifier. If -1, will automatically create an identifier.}

\docparam{pos}{Window position. wxDefaultPosition is (-1, -1) which indicates that wxWidgets
should generate a default position for the window.}

\docparam{size}{Window size. wxDefaultSize is (-1, -1) which indicates that wxWidgets should
generate a default size for the window. If no suitable size can be found, the window will be sized to 20x20 pixels so that the window is visible but obviously not correctly sized.}

\docparam{style}{Window style.}

\docparam{name}{Window name.}

\docparam{attribList}{Array of integers. With this parameter you can set the device context attributes associated to this window.
This array is zero-terminated: it should be set up with constants described in the table above.
If a constant should be followed by a value, put it in the next array position.
For example, the WX\_GL\_DEPTH\_SIZE should be followed by the value that indicates the number of
bits for the depth buffer, so:}

\begin{verbatim}
attribList[index] = WX_GL_DEPTH_SIZE;
attribList[index+1] = 32;
and so on.
\end{verbatim}

\docparam{palette}{Palette for indexed colour (i.e. non WX\_GL\_RGBA) mode.
Ignored under most platforms.}


\membersection{wxGLCanvas::SetCurrent}\label{wxglcanvassetcurrent}

\func{bool}{SetCurrent}{ \param{const wxGLContext&}{ context} }

Makes the OpenGL state that is represented by the OpenGL rendering context
\arg{context} current, i.e. it will be used by all subsequent OpenGL calls.

This is equivalent to \helpref{wxGLContext::SetCurrent}{wxglcontextsetcurrent} 
called with this window as parameter.

Note that this function may only be called when the window is shown on screen,
in particular it can't usually be called from the constructor as the window
isn't yet shown at this moment.

Returns \false if an error occurred.


\membersection{wxGLCanvas::SetColour}\label{wxglcanvassetcolour}

\func{void}{SetColour}{\param{const wxString\&}{ colour}}

Sets the current colour for this window (using \texttt{glcolor3f()}), using the
wxWidgets colour database to find a named colour.


\membersection{wxGLCanvas::SwapBuffers}\label{wxglcanvasswapbuffers}

\func{bool}{SwapBuffers}{\void}

Swaps the double-buffer of this window, making the back-buffer the front-buffer and vice versa,
so that the output of the previous OpenGL commands is displayed on the window.

Returns \false if an error occurred.

