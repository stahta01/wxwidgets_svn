%%%%%%%%%%%%%%%%%%%%%%%%%%%%%%%%%%%%%%%%%%%%%%%%%%%%%%%%%%%%%%%%%%%%%%%%%%%%%%%
%% Name:        dataform.tex
%% Purpose:     wxDataFormat documentation
%% Author:      Vadim Zeitlin
%% Modified by:
%% Created:     03.11.99
%% RCS-ID:      $Id$
%% Copyright:   (c) Vadim Zeitlin
%% License:     wxWidgets license
%%%%%%%%%%%%%%%%%%%%%%%%%%%%%%%%%%%%%%%%%%%%%%%%%%%%%%%%%%%%%%%%%%%%%%%%%%%%%%%

\section{\class{wxDataFormat}}\label{wxdataformat}

A wxDataFormat is an encapsulation of a platform-specific format handle which
is used by the system for the clipboard and drag and drop operations. The
applications are usually only interested in, for example, pasting data from the
clipboard only if the data is in a format the program understands and a data
format is something which uniquely identifies this format.

On the system level, a data format is usually just a number ({\tt CLIPFORMAT}
under Windows or {\tt Atom} under X11, for example) and the standard formats
are, indeed, just numbers which can be implicitly converted to wxDataFormat.
The standard formats are:

\begin{twocollist}\itemsep=1cm
\twocolitem{wxDF\_INVALID}{An invalid format - used as default argument for
functions taking a wxDataFormat argument sometimes}
\twocolitem{wxDF\_TEXT}{Text format (wxString)}
\twocolitem{wxDF\_BITMAP}{A bitmap (wxBitmap)}
\twocolitem{wxDF\_METAFILE}{A metafile (wxMetafile, Windows only)}
\twocolitem{wxDF\_FILENAME}{A list of filenames}
\twocolitem{wxDF\_HTML}{An HTML string. This is only valid when passed to wxSetClipboardData
when compiled with Visual C++ in non-Unicode mode}
\end{twocollist}

As mentioned above, these standard formats may be passed to any function taking
wxDataFormat argument because wxDataFormat has an implicit conversion from
them (or, to be precise from the type {\tt wxDataFormat::NativeFormat} which is
the type used by the underlying platform for data formats).

Aside the standard formats, the application may also use custom formats which
are identified by their names (strings) and not numeric identifiers. Although
internally custom format must be created (or {\it registered}) first, you
shouldn't care about it because it is done automatically the first time the
wxDataFormat object corresponding to a given format name is created. The only
implication of this is that you should avoid having global wxDataFormat objects
with non-default constructor because their constructors are executed before the
program has time to perform all necessary initialisations and so an attempt to
do clipboard format registration at this time will usually lead to a crash!

\wxheading{Virtual functions to override}

None

\wxheading{Derived from}

None

\wxheading{See also}

\helpref{Clipboard and drag and drop overview}{wxdndoverview}, 
\helpref{DnD sample}{samplednd}, 
\helpref{wxDataObject}{wxdataobject}

\latexignore{\rtfignore{\wxheading{Members}}}

\membersection{wxDataFormat::wxDataFormat}\label{wxdataformatwxdataformatdef}

\func{}{wxDataFormat}{\param{NativeFormat}{ format = wxDF\_INVALID}}

Constructs a data format object for one of the standard data formats or an
empty data object (use \helpref{SetType}{wxdataformatsettype} or 
\helpref{SetId}{wxdataformatsetid} later in this case)

\perlnote{In wxPerl this function is named {\tt newNative}.}

\membersection{wxDataFormat::wxDataFormat}\label{wxdataformatwxdataformat}

\func{}{wxDataFormat}{\param{const wxChar }{*format}}

Constructs a data format object for a custom format identified by its name 
{\it format}.

\perlnote{In wxPerl this function is named {\tt newUser}.}

\membersection{wxDataFormat::operator $==$}\label{wxdataformatoperatoreq}

\constfunc{bool}{operator $==$}{\param{const wxDataFormat\&}{ format}}

Returns TRUE if the formats are equal.

\membersection{wxDataFormat::operator $!=$}\label{wxdataformatoperatorneq}

\constfunc{bool}{operator $!=$}{\param{const wxDataFormat\&}{ format}}

Returns TRUE if the formats are different.

\membersection{wxDataFormat::GetId}\label{wxdataformatgetid}

\constfunc{wxString}{GetId}{\void}

Returns the name of a custom format (this function will fail for a standard
format).

\membersection{wxDataFormat::GetType}\label{wxdataformatgettype}

\constfunc{NativeFormat}{GetType}{\void}

Returns the platform-specific number identifying the format.

\membersection{wxDataFormat::SetId}\label{wxdataformatsetid}

\func{void}{SetId}{\param{const wxChar }{*format}}

Sets the format to be the custom format identified by the given name.

\membersection{wxDataFormat::SetType}\label{wxdataformatsettype}

\func{void}{SetType}{\param{NativeFormat}{ format}}

Sets the format to the given value, which should be one of wxDF\_XXX constants.

