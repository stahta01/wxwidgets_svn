%%%%%%%%%%%%%%%%%%%%%%%%%%%%%%%%%%%%%%%%%%%%%%%%%%%%%%%%%%%%%%%%%%%%%%%%%%%%%%%
%% Name:        tsamples.tex
%% Purpose:     Samples description
%% Author:      Vadim Zeitlin
%% Modified by:
%% Created:     02.11.99
%% RCS-ID:      $Id$
%% Copyright:   (c) wxWidgets team
%% License:     wxWindows license
%%%%%%%%%%%%%%%%%%%%%%%%%%%%%%%%%%%%%%%%%%%%%%%%%%%%%%%%%%%%%%%%%%%%%%%%%%%%%%%
% NB: please keep the subsections in alphabetic order!

\section{wxWidgets samples}\label{samples}

Probably the best way to learn wxWidgets is by reading the source of some 50+
samples provided with it. Many aspects of wxWidgets programming can be learnt
from them, but sometimes it is not simple to just choose the right sample to
look at. This overview aims at describing what each sample does/demonstrates to
make it easier to find the relevant one if a simple grep through all sources
didn't help. They also provide some notes about using the samples and what
features of wxWidgets are they supposed to test.

There are currently more than 50 different samples as part of wxWidgets and
this list is not complete. You should start your tour of wxWidgets with the
\helpref{minimal sample}{sampleminimal} which is the wxWidgets version of
"Hello, world!". It shows the basic structure of wxWidgets program and is the
most commented sample of all - looking at its source code is recommended.

The next most useful sample is probably the \helpref{controls}{samplecontrols}
one which shows many of wxWidgets standard controls, such as buttons,
listboxes, checkboxes, comboboxes etc.

Other, more complicated controls, have their own samples. In this category you
may find the following samples showing the corresponding controls:

\begin{twocollist}\itemsep=0pt
\twocolitem{\helpref{wxCalendarCtrl}{samplecalendar}}{Calendar a.k.a. date picker control}
\twocolitem{\helpref{wxListCtrl}{samplelistctrl}}{List view control}
\twocolitem{\helpref{wxTreeCtrl}{sampletreectrl}}{Tree view control}
\twocolitem{\helpref{wxGrid}{samplegrid}}{Grid control}
\end{twocollist}

Finally, it might be helpful to do a search in the entire sample directory if
you can't find the sample showing the control you are interested in by
name. Most classes contained in wxWidgets occur in at least one of the samples.


\subsection{Minimal sample}\label{sampleminimal}

The minimal sample is what most people will know under the term Hello World,
i.e. a minimal program that doesn't demonstrate anything apart from what is
needed to write a program that will display a "hello" dialog. This is usually
a good starting point for learning how to use wxWidgets.


\subsection{Art provider sample}\label{sampleartprovider}

The {\tt artprov} sample shows how you can customize the look of standard
wxWidgets dialogs by replacing default bitmaps/icons with your own versions.
It also shows how you can use wxArtProvider to
get stock bitmaps for use in your application.


\subsection{Calendar sample}\label{samplecalendar}

This font shows the \helpref{calendar control}{wxcalendarctrl} in action. It
shows how to configure the control (see the different options in the calendar
menu) and also how to process the notifications from it.


\subsection{Checklist sample}\label{samplechecklist}

This sample demonstrates use of the \helpref{wxCheckListBox}{wxchecklistbox}
class intercepting check, select and double click events. It also tests
use of various methods modifying the control, such as by deleting items
from it or inserting new ones (these functions are actually implemented in
the parent class \helpref{wxListBox}{wxlistbox} so the sample tests that class
as well). The layout of the dialog is created using a \helpref{wxBoxSizer}{wxboxsizer}
demonstrating a simple dynamic layout.


\subsection{Config sample}\label{sampleconfig}

This sample demonstrates the \helpref{wxConfig}{wxconfigbase} classes in a platform
independent way, i.e. it uses text based files to store a given configuration under
Unix and uses the Registry under Windows.

See \helpref{wxConfig overview}{wxconfigoverview} for the descriptions of all
features of this class.


\subsection{Controls sample}\label{samplecontrols}

The controls sample is the main test program for most simple controls used in
wxWidgets. The sample tests their basic functionality, events, placement,
modification in terms of colour and font as well as the possibility to change
the controls programmatically, such as adding an item to a list box etc. Apart
from that, the sample uses a \helpref{wxNotebook}{wxnotebook} and tests most
features of this special control (using bitmap in the tabs, using
\helpref{wxSizers}{wxsizer} and \helpref{constraints}{wxlayoutconstraints} within
notebook pages, advancing pages programmatically and vetoing a page change
by intercepting the \helpref{wxNotebookEvent}{wxnotebookevent}.

The various controls tested are listed here:

\begin{twocollist}\itemsep=0pt
\twocolitem{\helpref{wxButton}{wxbutton}}{Push button control, displaying text}
\twocolitem{\helpref{wxBitmapButton}{wxbitmapbutton}}{Push button control, displaying a bitmap}
\twocolitem{\helpref{wxCheckBox}{wxcheckbox}}{Checkbox control}
\twocolitem{\helpref{wxChoice}{wxchoice}}{Choice control (a combobox without the editable area)}
\twocolitem{\helpref{wxComboBox}{wxcombobox}}{A choice with an editable area}
\twocolitem{\helpref{wxGauge}{wxgauge}}{A control to represent a varying quantity, such as time remaining}
\twocolitem{\helpref{wxStaticBox}{wxstaticbox}}{A static, or group box for visually grouping related controls}
\twocolitem{\helpref{wxListBox}{wxlistbox}}{A list of strings for single or multiple selection}
\twocolitem{wxSpinCtrl}{A spin ctrl with a text field and a `up-down' control}
\twocolitem{\helpref{wxSpinButton}{wxspinbutton}}{A spin or `up-down' control}
\twocolitem{\helpref{wxStaticText}{wxstatictext}}{One or more lines of non-editable text}
\twocolitem{\helpref{wxStaticBitmap}{wxstaticbitmap}}{A control to display a bitmap}
\twocolitem{\helpref{wxRadioBox}{wxradiobox}}{A group of radio buttons}
\twocolitem{\helpref{wxRadioButton}{wxradiobutton}}{A round button to be used with others in a mutually exclusive way}
\twocolitem{\helpref{wxSlider}{wxslider}}{A slider that can be dragged by the user}
\end{twocollist}


\subsection{Database sample}\label{sampledb}

The database sample is a small test program showing how to use the ODBC
classes written by Remstar Intl.  Obviously, this sample requires a
database with ODBC support to be correctly installed on your system.


\subsection{DebugRpt sample}\label{sampledebugrpt}

This sample shows how to use \helpref{wxDebugReport}{wxdebugreport} class to
generate a debug report in case of a program crash or otherwise. On start up,
it proposes to either crash itself (by dereferencing a NULL pointer) or
generate debug report without doing it. Next it initializes the debug report
with standard information adding a custom file to it (just a timestamp) and
allows to view the information gathered using 
\helpref{wxDebugReportPreview}{wxdebugreportpreview}.

For the report processing part of the sample to work you should make available
a Web server accepting form uploads, otherwise 
\helpref{wxDebugReportUpload}{wxdebugreportupload} will report an error.


\subsection{Dialogs sample}\label{sampledialogs}

This sample shows how to use the common dialogs available from wxWidgets. These
dialogs are described in detail in the \helpref{Common dialogs overview}{commondialogsoverview}.


\subsection{Dialup sample}\label{sampledialup}

This sample shows the \helpref{wxDialUpManager}{wxdialupmanager}
class. In the status bar, it displays the information gathered through its
interface: in particular, the current connection status (online or offline) and
whether the connection is permanent (in which case a string `LAN' appears in
the third status bar field - but note that you may be on a LAN not
connected to the Internet, in which case you will not see this) or not.

Using the menu entries, you may also dial or hang up the line if you have a
modem attached and (this only makes sense for Windows) list the available
connections.


\subsection{DnD sample}\label{samplednd}

This sample shows both clipboard and drag and drop in action. It is quite non
trivial and may be safely used as a basis for implementing the clipboard and
drag and drop operations in a real-life program.

When you run the sample, its screen is split in several parts. On the top,
there are two listboxes which show the standard derivations of
\helpref{wxDropTarget}{wxdroptarget}:
\helpref{wxTextDropTarget}{wxtextdroptarget} and
\helpref{wxFileDropTarget}{wxfiledroptarget}.

The middle of the sample window is taken by the log window which shows what is
going on (of course, this only works in debug builds) and may be helpful to see
the sequence of steps of data transfer.

Finally, the last part is used for dragging text from it to either one of the
listboxes (only one will accept it) or another application. The last
functionality available from the main frame is to paste a bitmap from the
clipboard (or, in the case of the Windows version, also a metafile) - it will be
shown in a new frame.

So far, everything we mentioned was implemented with minimal amount of code
using standard wxWidgets classes. The more advanced features are demonstrated
if you create a shape frame from the main frame menu. A shape is a geometric
object which has a position, size and color. It models some
application-specific data in this sample. A shape object supports its own
private \helpref{wxDataFormat}{wxdataformat} which means that you may cut and
paste it or drag and drop (between one and the same or different shapes) from
one sample instance to another (or the same). However, chances are that no
other program supports this format and so shapes can also be rendered as
bitmaps which allows them to be pasted/dropped in many other applications
(and, under Windows, also as metafiles which are supported by most of Windows
programs as well - try Write/Wordpad, for example).

Take a look at DnDShapeDataObject class to see how you may use
\helpref{wxDataObject}{wxdataobject} to achieve this.


\subsection{Dynamic sample}\label{sampledynamic}

This sample is a very small sample that demonstrates use of the
\helpref{wxEvtHandler::Connect}{wxevthandlerconnect} method. This method
should be used whenever it is not known at compile time which control
will receive which event or which controls are actually going to be in
a dialog or frame. This is most typically the case for any scripting
language that would work as a wrapper for wxWidgets or programs where
forms or similar datagrams can be created by the users.

See also the \helpref{event sample}{sampleevent}


\subsection{Event sample}\label{sampleevent}

The event sample demonstrates various features of the wxWidgets events. It
shows using dynamic events and connecting/disconnecting the event handlers
during run time and also using
\helpref{PushEventHandler()}{wxwindowpusheventhandler} and
\helpref{PopEventHandler()}{wxwindowpopeventhandler}.

It replaces the old dynamic sample.


\subsection{Except(ions) sample}\label{sampleexcept}

This very simple sample shows how to use C++ exceptions in wxWidgets programs,
i.e. where to catch the exception which may be thrown by the program code. It
doesn't do anything very exciting by itself, you need to study its code to
understand what goes on.

You need to build the library with \texttt{wxUSE\_EXCEPTIONS} being set to $1$
and compile your code with C++ exceptions support to be able to build this
sample.


\subsection{Exec sample}\label{sampleexec}

The exec sample demonstrates the \helpref{wxExecute}{wxexecute} and
\helpref{wxShell}{wxshell} functions. Both of them are used to execute the
external programs and the sample shows how to do this synchronously (waiting
until the program terminates) or asynchronously (notification will come later).

It also shows how to capture the output of the child process in both
synchronous and asynchronous cases and how to kill the processes with
\helpref{wxProcess::Kill}{wxprocesskill} and test for their existence with
\helpref{wxProcess::Exists}{wxprocessexists}.


\subsection{Font sample}\label{samplefont}

The font sample demonstrates \helpref{wxFont}{wxfont},
\helpref{wxFontEnumerator}{wxfontenumerator} and
\helpref{wxFontMapper}{wxfontmapper} classes. It allows you to see the fonts
available (to wxWidgets) on the computer and shows all characters of the
chosen font as well.


\subsection{Grid sample}\label{samplegrid}

TODO.


\subsection{HTML samples}\label{samplehtml}

Eight HTML samples (you can find them in directory {\tt samples/html})
cover all features of the HTML sub-library.

{\bf Test} demonstrates how to create \helpref{wxHtmlWindow}{wxhtmlwindow}
and also shows most supported HTML tags.

{\bf Widget} shows how you can embed ordinary controls or windows within an
HTML page. It also nicely explains how to write new tag handlers and extend
the library to work with unsupported tags.

{\bf About} may give you an idea how to write good-looking About boxes.

{\bf Zip} demonstrates use of virtual file systems in wxHTML. The zip archives
handler (ships with wxWidgets) allows you to access HTML pages stored
in a compressed archive as if they were ordinary files.

{\bf Virtual} is yet another virtual file systems demo. This one generates pages at run-time.
You may find it useful if you need to display some reports in your application.

{\bf Printing} explains use of \helpref{wxHtmlEasyPrinting}{wxhtmleasyprinting}
class which serves as as-simple-as-possible interface for printing HTML
documents without much work. In fact, only few function calls are sufficient.

{\bf Help} and {\bf Helpview} are variations on displaying HTML help
(compatible with MS HTML Help Workshop). {\it Help} shows how to embed
\helpref{wxHtmlHelpController}{wxhtmlhelpcontroller} in your application
while {\it Helpview} is a simple tool that only pops up the help window and
displays help books given at command line.


\subsection{Image sample}\label{sampleimage}

The image sample demonstrates use of the \helpref{wxImage}{wximage} class
and shows how to download images in a variety of formats, currently PNG, GIF,
TIFF, JPEG, BMP, PNM and PCX. The top of the sample shows two rectangles, one
of which is drawn directly in the window, the other one is drawn into a
\helpref{wxBitmap}{wxbitmap}, converted to a wxImage, saved as a PNG image
and then reloaded from the PNG file again so that conversions between wxImage
and wxBitmap as well as loading and saving PNG files are tested.

At the bottom of the main frame there is a test for using a monochrome bitmap by
drawing into a \helpref{wxMemoryDC}{wxmemorydc}. The bitmap is then drawn
specifying the foreground and background colours with
\helpref{wxDC::SetTextForeground}{wxdcsettextforeground} and
\helpref{wxDC::SetTextBackground}{wxdcsettextbackground} (on the left). The
bitmap is then converted to a wxImage and the foreground colour (black) is
replaced with red using \helpref{wxImage::Replace}{wximagereplace}.


\subsection{Internat(ionalization) sample}\label{sampleinternat}

The not very clearly named internat sample demonstrates the wxWidgets
internationalization (i18n for short from now on) features. To be more
precise, it only shows localization support, i.e. support for translating the
program messages into another language while true i18n would also involve
changing the other aspects of the programs behaviour.

More information about this sample can be found in the {\tt readme.txt} file in
its directory. Please see also \helpref{i18n overview}{internationalization}.


\subsection{Layout sample}\label{samplelayout}

The layout sample demonstrates the two different layout systems offered
by wxWidgets. When starting the program, you will see a frame with some
controls and some graphics. The controls will change their size whenever
you resize the entire frame and the exact behaviour of the size changes
is determined using the \helpref{wxLayoutConstraints}{wxlayoutconstraints}
class. See also the \helpref{overview}{constraintsoverview} and the
\helpref{wxIndividualLayoutConstraint}{wxindividuallayoutconstraint}
class for further information.

The menu in this sample offers two more tests, one showing how to use
a \helpref{wxBoxSizer}{wxboxsizer} in a simple dialog and the other one
showing how to use sizers in connection with a \helpref{wxNotebook}{wxnotebook}
class. See also \helpref{wxSizer}{wxsizer}.


\subsection{Listctrl sample}\label{samplelistctrl}

This sample shows the \helpref{wxListCtrl}{wxlistctrl} control. Different modes
supported by the control (list, icons, small icons, report) may be chosen from
the menu.

The sample also provides some timings for adding/deleting/sorting a lot of
(several thousands) items into the control.


\subsection{Mediaplayer sample}\label{samplemediaplayer}

This sample demonstrates how to use all the features of
\helpref{wxMediaCtrl}{wxmediactrl} and play various types of sound, video,
and other files.


It replaces the old dynamic sample.

\subsection{Notebook sample}\label{samplenotebook}

This samples shows \helpref{wxBookCtrl}{wxbookctrloverview} family of controls.
Although initially it was written to demonstrate \helpref{wxNotebook}{wxnotebook}
only, it can now be also used to see \helpref{wxListbook}{wxlistbook} and
\helpref{wxChoicebook}{wxchoicebook} in action. Test each of the controls, their
orientation, images and pages using commands through menu.



\subsection{Render sample}\label{samplerender}

This sample shows how to replace the default wxWidgets
\helpref{renderer}{wxrenderernative} and also how to write a shared library
(DLL) implementing a renderer and load and unload it during the run-time.



\subsection{Rotate sample}\label{samplerotate}

This is a simple example which demonstrates how to rotate an image with
the \helpref{wxImage::Rotate}{wximagerotate} method. The rotation can
be done without interpolation (left mouse button) which will be faster,
or with interpolation (right mouse button) which is slower but gives
better results.


\subsection{Scroll subwindow sample}\label{samplescrollsub}

This sample demonstrates use of the \helpref{wxScrolledWindow}{wxscrolledwindow}
class including placing subwindows into it and drawing simple graphics. It uses the
\helpref{SetTargetWindow}{wxscrolledwindowsettargetwindow} method and thus the effect
of scrolling does not show in the scrolled window itself, but in one of its subwindows.

Additionally, this samples demonstrates how to optimize drawing operations in wxWidgets,
in particular using the \helpref{wxWindow::IsExposed}{wxwindowisexposed} method with
the aim to prevent unnecessary drawing in the window and thus reducing or removing
flicker on screen.


\subsection{Sockets sample}\label{samplesockets}

The sockets sample demonstrates how to use the communication facilities
provided by \helpref{wxSocket}{wxsocketbase}. There are two different
applications in this sample: a server, which is implemented using a
\helpref{wxSocketServer}{wxsocketserver} object, and a client, which
is implemented as a \helpref{wxSocketClient}{wxsocketclient}.

The server binds to the local address, using TCP port number 3000,
sets up an event handler to be notified of incoming connection requests
({\bf wxSOCKET\_CONNECTION} events), and sits there, waiting for clients
({\it listening}, in socket parlance). For each accepted connection,
a new \helpref{wxSocketBase}{wxsocketbase} object is created. These
socket objects are independent from the server that created them, so
they set up their own event handler, and then request to be notified
of {\bf wxSOCKET\_INPUT} (incoming data) or {\bf wxSOCKET\_LOST}
(connection closed at the remote end) events. In the sample, the event
handler is the same for all connections; to find out which socket the
event is addressed to, the \helpref{GetSocket}{wxsocketeventgetsocket} function
is used.

Although it might take some time to get used to the event-oriented
system upon which wxSocket is built, the benefits are many. See, for
example, that the server application, while being single-threaded
(and of course without using fork() or ugly select() loops) can handle
an arbitrary number of connections.

The client starts up unconnected, so you can use the Connect... option
to specify the address of the server you are going to connect to (the
TCP port number is hard-coded as 3000). Once connected, a number of
tests are possible. Currently, three tests are implemented. They show
how to use the basic IO calls in \helpref{wxSocketBase}{wxsocketbase},
such as \helpref{Read}{wxsocketbaseread}, \helpref{Write}{wxsocketbasewrite},
\helpref{ReadMsg}{wxsocketbasereadmsg} and \helpref{WriteMsg}{wxsocketbasewritemsg},
and how to set up the correct IO flags depending on what you are going to
do. See the comments in the code for more information. Note that because
both clients and connection objects in the server set up an event handler
to catch {\bf wxSOCKET\_LOST} events, each one is immediately notified
if the other end closes the connection.

There is also a URL test which shows how to use
the \helpref{wxURL}{wxurl} class to fetch data from a given URL.

The sockets sample is work in progress. Some things to do:

\begin{itemize}\itemsep=0pt
\item More tests for basic socket functionality.
\item More tests for protocol classes (wxProtocol and its descendants).
\item Tests for the recently added (and still in alpha stage) datagram sockets.
\item New samples which actually do something useful (suggestions accepted).
\end{itemize}


\subsection{Sound sample}\label{samplesound}

The {\tt sound} sample shows how to use \helpref{wxSound}{wxsound} for simple
audio output (e.g. notifications).


\subsection{Statbar sample}\label{samplestatbar}

This sample shows how to create and use wxStatusBar. Although most of the
samples have a statusbar, they usually only create a default one and only
do it once.

Here you can see how to recreate the statusbar (with possibly different number
of fields) and how to use it to show icons/bitmaps and/or put arbitrary
controls into it.


\subsection{Text sample}\label{sampletext}

This sample demonstrates four features: firstly the use and many variants of
the \helpref{wxTextCtrl}{wxtextctrl} class (single line, multi line, read only,
password, ignoring TAB, ignoring ENTER).

Secondly it shows how to intercept a \helpref{wxKeyEvent}{wxkeyevent} in both
the raw form using the {\tt EVT\_KEY\_UP} and {\tt EVT\_KEY\_DOWN} macros and the
higher level from using the {\tt EVT\_CHAR} macro. All characters will be logged
in a log window at the bottom of the main window. By pressing some of the function
keys, you can test some actions in the text ctrl as well as get statistics on the
text ctrls, which is useful for testing if these statistics actually are correct.

Thirdly, on platforms which support it, the sample will offer to copy text to the
\helpref{wxClipboard}{wxclipboard} and to paste text from it. The GTK version will
use the so called PRIMARY SELECTION, which is the pseudo clipboard under X and
best known from pasting text to the XTerm program.

Last not least: some of the text controls have tooltips and the sample also shows
how tooltips can be centrally disabled and their latency controlled.


\subsection{Thread sample}\label{samplethread}

This sample demonstrates use of threads in connection with GUI programs.
There are two fundamentally different ways to use threads in GUI programs and
either way has to take care of the fact that the GUI library itself usually
is not multi-threading safe, i.e. that it might crash if two threads try to
access the GUI class simultaneously. One way to prevent that is have a normal
GUI program in the main thread and some worker threads which work in the
background. In order to make communication between the main thread and the
worker threads possible, wxWidgets offers the \helpref{wxPostEvent}{wxpostevent}
function and this sample makes use of this function.

The other way to use a so called Mutex (such as those offered in the \helpref{wxMutex}{wxmutex}
class) that prevent threads from accessing the GUI classes as long as any other
thread accesses them. For this, wxWidgets has the \helpref{wxMutexGuiEnter}{wxmutexguienter}
and \helpref{wxMutexGuiLeave}{wxmutexguileave} functions, both of which are
used and tested in the sample as well.

See also \helpref{Multithreading overview}{wxthreadoverview} and \helpref{wxThread}{wxthread}.


\subsection{Toolbar sample}\label{sampletoolbar}

The toolbar sample shows the \helpref{wxToolBar}{wxtoolbar} class in action.

The following things are demonstrated:

\begin{itemize}\itemsep=0pt
\item Creating the toolbar using \helpref{wxToolBar::AddTool}{wxtoolbaraddtool}
and \helpref{wxToolBar::AddControl}{wxtoolbaraddcontrol}: see
MyApp::InitToolbar in the sample.
\item Using {\tt EVT\_UPDATE\_UI} handler for automatically enabling/disabling
toolbar buttons without having to explicitly call EnableTool. This is done
in MyFrame::OnUpdateCopyAndCut.
\item Using \helpref{wxToolBar::DeleteTool}{wxtoolbardeletetool} and
\helpref{wxToolBar::InsertTool}{wxtoolbarinserttool} to dynamically update the
toolbar.
\end{itemize}

Some buttons in the main toolbar are check buttons, i.e. they stay checked when
pressed. On the platforms which support it, the sample also adds a combobox
to the toolbar showing how you can use arbitrary controls and not only buttons
in it.

If you toggle another toolbar in the sample (using {\tt Ctrl-A}) you will also
see the radio toolbar buttons in action: the first three buttons form a radio
group, i.e. checking any of them automatically unchecks the previously
checked one.


\subsection{Treectrl sample}\label{sampletreectrl}

This sample demonstrates using the \helpref{wxTreeCtrl}{wxtreectrl} class. Here
you may see how to process various notification messages sent by this control
and also when they occur (by looking at the messages in the text control in
the bottom part of the frame).

Adding, inserting and deleting items and branches from the tree as well as
sorting (in default alphabetical order as well as in custom one) is
demonstrated here as well - try the corresponding menu entries.


\subsection{Wizard sample}\label{samplewizard}

This sample shows the so-called wizard dialog (implemented using
\helpref{wxWizard}{wxwizard} and related classes). It shows almost all
features supported:

\begin{itemize}\itemsep=0pt
\item Using bitmaps with the wizard and changing them depending on the page
shown (notice that wxValidationPage in the sample has a different image from
the other ones)
\item Using \helpref{TransferDataFromWindow}{wxwindowtransferdatafromwindow}
to verify that the data entered is correct before passing to the next page
(done in wxValidationPage which forces the user to check a checkbox before
continuing).
\item Using more elaborated techniques to allow returning to the previous
page, but not continuing to the next one or vice versa (in wxRadioboxPage)
\item This (wxRadioboxPage) page also shows how the page may process the {\tt
Cancel} button itself instead of relying on the wizard parent to do it.
\item Normally, the order of the pages in the wizard is known at compile-time,
but sometimes it depends on the user choices: wxCheckboxPage shows how to
dynamically decide which page to display next (see also
\helpref{wxWizardPage}{wxwizardpage})
\end{itemize}

