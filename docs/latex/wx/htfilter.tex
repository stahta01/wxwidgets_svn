%
% automatically generated by HelpGen from
% htmlfilter.tex at 29/Mar/99 18:35:09
%

\section{\class{wxHtmlFilter}}\label{wxhtmlfilter}

This class is the parent class of input filters for \helpref{wxHtmlWindow}{wxhtmlwindow}.
It allows you to read and display files of different file formats.

\wxheading{Derived from}

\helpref{wxObject}{wxobject}

\wxheading{Include files}

<wx/html/htmlfilt.h>


\wxheading{See Also}

\helpref{Overview}{filters}

\latexignore{\rtfignore{\wxheading{Members}}}

\membersection{wxHtmlFilter::wxHtmlFilter}\label{wxhtmlfilterwxhtmlfilter}

\func{}{wxHtmlFilter}{\void}

Constructor.

\membersection{wxHtmlFilter::CanRead}\label{wxhtmlfiltercanread}

\func{bool}{CanRead}{\param{const wxFSFile\& }{file}}

Returns true if this filter is capable of reading file {\it file}.

Example:

\begin{verbatim}
bool MyFilter::CanRead(const wxFSFile& file)
{
    return (file.GetMimeType() == "application/x-ugh");
}
\end{verbatim}

\membersection{wxHtmlFilter::ReadFile}\label{wxhtmlfilterreadfile}

\func{wxString}{ReadFile}{\param{const wxFSFile\& }{file}}

Reads the file and returns string with HTML document.

Example:

\begin{verbatim}
wxString MyImgFilter::ReadFile(const wxFSFile& file)
{
    return "<html><body><img src=\"" +
           file.GetLocation() +
	   "\"></body></html>";
}
\end{verbatim}

