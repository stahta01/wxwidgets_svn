\section{\class{wxCondition}}\label{wxcondition}

wxCondition variables correspond to pthread conditions or to Win32 event
objects. They may be used in a multithreaded application to wait until the
given condition becomes true which happens when the condition becomes signaled.

For example, if a worker thread is doing some long task and another thread has
to wait until it is finished, the latter thread will wait on the condition
object and the worker thread will signal it on exit (this example is not
perfect because in this particular case it would be much better to just 
\helpref{Wait()}{wxthreadwait} for the worker thread, but if there are several
worker threads it already makes much more sense).

Note that a call to \helpref{Signal()}{wxconditionsignal} may happen before the
other thread calls \helpref{Wait()}{wxconditionwait} and, just as with the
pthread conditions, the signal is then lost and so if you want to be sure to
get it you must use a mutex together with the condition variable.

\wxheading{Example}

This example shows how a main thread may launch a worker thread which starts
running and then waits until the main thread signals it to continue:

\begin{verbatim}
class MyWaitingThread : public wxThread
{
public:
    MyWaitingThread(wxMutex *mutex, wxCondition *condition)
    {
        m_mutex = mutex;
        m_condition = condition;

        Create();
    }

    virtual ExitCode Entry()
    {
        // wait for the signal from the main thread: it is absolutely necessary
        // to look the mutex before doing it!
        m_mutex->Lock();
        m_condition->Signal();
        m_mutex->Unlock();

        ... do our job ...

        return 0;
    }

private:
    wxCondition *m_condition;
};

int main()
{
    wxMutex mutex;
    wxCondition condition(mutex);

    for ( int i = 0; i < 10; i++ )
    {
        MyWaitingThread *thread = new MyWaitingThread(&mutex, &condition);

        thread->Run();
    }

    // wake up one of the threads
    condition.Signal();

    // wake up all the other ones
    condition.Broadcast();

    ... wait until they terminate or do something else ...

    return 0;
}
\end{verbatim}

\wxheading{Derived from}

None.

\wxheading{Include files}

<wx/thread.h>

\wxheading{See also}

\helpref{wxThread}{wxthread}, \helpref{wxMutex}{wxmutex}

\latexignore{\rtfignore{\wxheading{Members}}}

\membersection{wxCondition::wxCondition}\label{wxconditionconstr}

\func{}{wxCondition}{\param{wxMutex\& }{mutex}}

Default and only constructor. The {\it mutex} must be locked by the caller
before calling \helpref{Wait}{wxconditionwait} function.

\membersection{wxCondition::\destruct{wxCondition}}

\func{}{\destruct{wxCondition}}{\void}

Destroys the wxCondition object. The destructor is not virtual so this class
should not be used polymorphically.

\membersection{wxCondition::Broadcast}\label{wxconditionbroadcast}

\func{void}{Broadcast}{\void}

Broadcasts to all waiting threads, waking all of them up. Note that this method
may be called whether the mutex associated with this condition is locked or
not.

\wxheading{See also}

\helpref{wxCondition::Signal}{wxconditionsignal}

\membersection{wxCondition::Signal}\label{wxconditionsignal}

\func{void}{Signal}{\void}

Signals the object waking up at most one thread. If several threads are waiting
on the same condition, the exact thread which is woken up is undefined. If no
threads are waiting, the signal is lost and the condition would have to be
signalled again to wake up any thread which may start waiting on it later.

Note that this method may be called whether the mutex associated with this
condition is locked or not.

\wxheading{See also}

\helpref{wxCondition::Broadcast}{wxconditionbroadcast}

\membersection{wxCondition::Wait}\label{wxconditionwait}

\func{void}{Wait}{\void}

Waits until the condition is signalled.

\func{bool}{Wait}{\param{unsigned long}{ sec}, \param{unsigned long}{ nsec}}

Waits until the condition is signalled or the timeout has elapsed.

Note that the mutex associated with this condition {\bf must} be acquired by
the thread before calling this method.

\wxheading{Parameters}

\docparam{sec}{Timeout in seconds}

\docparam{nsec}{Timeout nanoseconds component (added to {\it sec}).}

\wxheading{Return value}

The second form returns {\tt TRUE} if the condition has been signalled, or
{\tt FALSE} if it returned because the timeout has elapsed.


