%%%%%%%%%%%%%%%%%%%%%%%%%%%%%%%%%%%%%%%%%%%%%%%%%%%%%%%%%%%%%%%%%%%%%%%%%%%%%%%
%% Name:        maxzevt.tex
%% Purpose:     wxMaximizeEvent documentation
%% Author:      Vadim Zeitlin
%% Modified by:
%% Created:
%% RCS-ID:      $Id$
%% Copyright:   (c) 2001 Vadim Zeitlin
%% License:     wxWidgets license
%%%%%%%%%%%%%%%%%%%%%%%%%%%%%%%%%%%%%%%%%%%%%%%%%%%%%%%%%%%%%%%%%%%%%%%%%%%%%%%

\section{\class{wxMaximizeEvent}}\label{wxmaximizeevent}

An event being sent when the frame is maximized (minimized) or restored.

\wxheading{Derived from}

\helpref{wxEvent}{wxevent}\\
\helpref{wxObject}{wxobject}

\wxheading{Include files}

<wx/event.h>

\wxheading{Event table macros}

To process a maximize event, use this event handler macro to direct input to a
member function that takes a wxMaximizeEvent argument.

\twocolwidtha{7cm}
\begin{twocollist}\itemsep=0pt
\twocolitem{{\bf EVT\_MAXIMIZE(func)}}{Process a wxEVT\_MAXIMIZE event.}
\end{twocollist}%

\wxheading{See also}

\helpref{Event handling overview}{eventhandlingoverview},\rtfsp
\helpref{wxFrame::Maximize}{wxframemaximize},\rtfsp
\helpref{wxFrame::IsMaximized}{wxframeismaximized}

\latexignore{\rtfignore{\wxheading{Members}}}

\membersection{wxMaximizeEvent::wxMaximizeEvent}\label{wxmaximizeeventctor}

\func{}{wxMaximizeEvent}{\param{int }{id = 0}}

Constructor.

