\section{\class{wxHelpController}}\label{wxhelpcontroller}

This is a family of classes by which
applications may invoke a help viewer to provide on-line help.

A help controller allows an application to display help, at the contents
or at a particular topic, and shut the help program down on termination.
This avoids proliferation of many instances of the help viewer whenever the
user requests a different topic via the application's menus or buttons.

Typically, an application will create a help controller instance
when it starts, and immediately call {\bf Initialize}\rtfsp
to associate a filename with it. The help viewer will only get run, however,
just before the first call to display something.

Most help controller classes actually derive from wxHelpControllerBase and have
names of the form wxXXXHelpController or wxHelpControllerXXX. An
appropriate class is aliased to the name wxHelpController for each platform, as follows:

\begin{itemize}\itemsep=0pt
\item On Windows, wxWinHelpController is used.
\item On all other platforms, wxHelpControllerHtml is used if wxHTML is
compiled into wxWindows; otherwise wxExtHelpController is used (for invoking an external
browser).
\end{itemize}

The remaining help controller classess need to be named
explicitly by an application that wishes to make use of them.

There are currently the following help controller classes defined:

\begin{itemize}\itemsep=0pt
\item wxWinHelpController, for controlling Windows Help.
\item wxExtHelpController, for controlling external browsers under Unix.
The default browser is Netscape Navigator. The 'help' sample shows its use.
\item wxHelpControllerHtml, using \helpref{wxHTML}{wxhtml} to display help. The API for this class
is reasonably close to the wxWindows help controller API; see {\tt wx/helpwxht.h} for
details of use.
\item \helpref{wxHtmlHelpController}{wxhtmlhelpcontroller}, a more sophisticated help controller using \helpref{wxHTML}{wxhtml}, in
a similar style to the Windows HTML Help viewer and using some of the same files.
Although it has an API compatible with other help controllers, it has more advanced features, so it is
recommended that you use the specific API for this class instead.
\end{itemize}

\wxheading{Derived from}

wxHelpControllerBase\\
\helpref{wxObject}{wxobject}

\wxheading{Include files}

<wx/help.h> (wxWindows chooses the appropriate help controller class)\\
<wx/helpbase.h> (wxHelpControllerBase class)\\
<wx/helpwin.h> (Windows Help controller)\\
<wx/generic/helpext.h> (external HTML browser controller)\\
<wx/generic/helpwxht.h> (simple wxHTML-based help controller)\\
<wx/html/helpctrl.h> (advanced wxHTML based help controller: wxHtmlHelpController)

\wxheading{See also}

\helpref{wxHtmlHelpController}{wxhtmlhelpcontroller}, \helpref{wxHTML}{wxhtml}

\latexignore{\rtfignore{\wxheading{Members}}}

\membersection{wxHelpController::wxHelpController}

\func{}{wxHelpController}{\void}

Constructs a help instance object, but does not invoke the help viewer.

\membersection{wxHelpController::\destruct{wxHelpController}}

\func{}{\destruct{wxHelpController}}{\void}

Destroys the help instance, closing down the viewer if it is running.

\membersection{wxHelpController::Initialize}\label{wxhelpcontrollerinitialize}

\func{virtual void}{Initialize}{\param{const wxString\& }{file}}

\func{virtual void}{Initialize}{\param{const wxString\& }{file}, \param{int}{ server}}

Initializes the help instance with a help filename, and optionally a server (socket)
number if using wxHelp. Does not invoke the help viewer.
This must be called directly after the help instance object is created and before
any attempts to communicate with the viewer.

You may omit the file extension and a suitable one will be chosen. For
wxHtmlHelpController, the extensions zip, htb and hhp will be appended while searching for
a suitable file. For WinHelp, the hlp extension is appended. For wxHelpControllerHtml (the
standard HTML help controller), the filename is assumed to be a directory containing
the HTML files.

\membersection{wxHelpController::DisplayBlock}\label{wxhelpcontrollerdisplayblock}

\func{virtual bool}{DisplayBlock}{\param{long}{ blockNo}}

If the help viewer is not running, runs it and displays the file at the given block number.

{\it WinHelp:} Refers to the context number.

{\it External HTML help:} the same as for \helpref{wxHelpController::DisplaySection}{wxhelpcontrollerdisplaysection}.

{\it wxHtmlHelpController:} {\it sectionNo} is an identifier as specified in the {\tt .hhc} file. See \helpref{Help files format}{helpformat}.

\membersection{wxHelpController::DisplayContents}\label{wxhelpcontrollerdisplaycontents}

\func{virtual bool}{DisplayContents}{\void}

If the help viewer is not running, runs it and displays the
contents.

\membersection{wxHelpController::DisplaySection}\label{wxhelpcontrollerdisplaysection}

\func{virtual bool}{DisplaySection}{\param{int}{ sectionNo}}

If the help viewer is not running, runs it and displays the given section.

{\it WinHelp:} {\it sectionNo} is a context id.

{\it External HTML help/wxHTML based help:} wxExtHelpController and wxHelpControllerHtml implement {\it sectionNo} as an id in a map file, which is of the form:

\begin{verbatim}
0  wx.html             ; Index
1  wx34.html#classref  ; Class reference
2  wx204.html          ; Function reference
\end{verbatim}

{\it wxHtmlHelpController:} {\it sectionNo} is an identifier as specified in the {\tt .hhc} file. See \helpref{Help files format}{helpformat}.

\membersection{wxHelpController::GetFrameParameters}\label{wxhelpcontrollergetframeparameters}

\func{virtual wxFrame *}{GetFrameParameters}{\param{const wxSize * }{size = NULL}, \param{const wxPoint * }{pos = NULL},
 \param{bool *}{newFrameEachTime = NULL}}

This reads the current settings for the help frame in the case of the
wxHelpControllerHtml, setting the frame size, position and
the newFrameEachTime parameters to the last values used. It also
returns the pointer to the last opened help frame. This can be used
for example, to automatically close the help frame on program
shutdown.

wxHtmlHelpController returns the frame,
size and position.

For all other help controllers, this function does nothing
and just returns NULL.

\wxheading{Parameters}

\docparam{viewer}{This defaults to "netscape" for wxExtHelpController.}

\docparam{flags}{This defaults to wxHELP\_NETSCAPE for wxExtHelpController, indicating
that the viewer is a variant of Netscape Navigator.}

\membersection{wxHelpController::KeywordSearch}\label{wxhelpcontrollerkeywordsearch}

\func{virtual bool}{KeywordSearch}{\param{const wxString\& }{keyWord}}

If the help viewer is not running, runs it, and searches for sections matching the given keyword. If one
match is found, the file is displayed at this section.

{\it WinHelp:} If more than one match is found, 
the first topic is displayed.

{\it External HTML help:} If more than one match is found, 
a choice of topics is displayed.

{\it wxHtmlHelpController:} see \helpref{wxHtmlHelpController::KeywordSearch}{wxhtmlhelpcontrollerkeywordsearch}.

\membersection{wxHelpController::LoadFile}\label{wxhelpcontrollerloadfile}

\func{virtual bool}{LoadFile}{\param{const wxString\& }{file = ""}}

If the help viewer is not running, runs it and loads the given file.
If the filename is not supplied or is
empty, the file specified in {\bf Initialize} is used. If the viewer is
already displaying the specified file, it will not be reloaded. This
member function may be used before each display call in case the user
has opened another file.

wxHtmlHelpController ignores this call.

\membersection{wxHelpController::OnQuit}\label{wxhelpcontrolleronquit}

\func{virtual bool}{OnQuit}{\void}

Overridable member called when this application's viewer is quit by the user.

This does not work for all help controllers.

\membersection{wxHelpController::SetFrameParameters}\label{wxhelpcontrollersetframeparameters}

\func{virtual void}{SetFrameParameters}{\param{const wxString \& }{title},
 \param{const wxSize \& }{size}, \param{const wxPoint \& }{pos = wxDefaultPosition},
 \param{bool }{newFrameEachTime = FALSE}}

For wxHelpControllerHtml, this allows the application to set the
default frame title, size and position for the frame. If the title
contains \%s, this will be replaced with the page title. If the
parameter newFrameEachTime is set, the controller will open a new
help frame each time it is called.

For wxHtmlHelpController, the title is set (again with \%s indicating the
page title) and also the size and position of the frame if the frame is already
open. {\it newFrameEachTime} is ignored.

For all other help controllers this function has no effect.

\membersection{wxHelpController::SetViewer}\label{wxhelpcontrollersetviewer}

\func{virtual void}{SetViewer}{\param{const wxString\& }{viewer}, \param{long}{ flags}}

Sets detailed viewer information. So far this is only relevant to wxExtHelpController.

Some examples of usage:

\begin{verbatim}
  m_help.SetViewer("kdehelp");
  m_help.SetViewer("gnome-help-browser");
  m_help.SetViewer("netscape", wxHELP_NETSCAPE);
\end{verbatim}

\membersection{wxHelpController::Quit}\label{wxhelpcontrollerquit}

\func{virtual bool}{Quit}{\void}

If the viewer is running, quits it by disconnecting.

For Windows Help, the viewer will only close if no other application is using it.

