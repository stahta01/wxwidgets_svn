\section{\class{wxImage}}\label{wximage}

This class encapsulates a platform-independent image. An image can be created
from data, or using the constructor taking a wxBitmap object. An image
can be loaded from a file in a variety of formats, and is extensible to new formats
via image format handlers. Functions are available to set and get image bits, so
it can be used for basic image manipulation.

A wxImage cannot (currently) be drawn directly to a \helpref{wxDC}{wxdc}. Instead, 
a platform-specific \helpref{wxBitmap}{wxbitmap} object must be created from it using
the \helpref{ConvertToBitmap}{wximageconverttobitmap} function. This bitmap can then
be drawn in a device context, using \helpref{wxDC::DrawBitmap}{wxdcdrawbitmap}.

One colour value of the image may be used as a mask colour which will lead to the automatic
creation of a \helpref{wxMask}{wxmask} object associated to the bitmap object.

\wxheading{Available image handlers}

The following image handlers are available. {\bf wxBMPHandler} is always
installed by default. To use other image formats, install the appropiate
handler with \helpref{wxImage::AddHandler}{wximageaddhandler} or 
\helpref{wxInitAllImageHandlers}{wxinitallimagehandlers}.

\twocolwidtha{5cm}%
\begin{twocollist}
\twocolitem{{\bf \indexit{wxBMPHandler}}}{Only for loading, always installed.}
\twocolitem{{\bf \indexit{wxPNGHandler}}}{For loading and saving.}
\twocolitem{{\bf \indexit{wxJPEGHandler}}}{For loading and saving.}
\twocolitem{{\bf \indexit{wxGIFHandler}}}{Only for loading, due to legal issues.}
\twocolitem{{\bf \indexit{wxPCXHandler}}}{For loading and saving (see below).}
\twocolitem{{\bf \indexit{wxPNMHandler}}}{For loading and saving (see below).}
\twocolitem{{\bf \indexit{wxTIFFHandler}}}{For loading.}
\end{twocollist}

When saving in PCX format, {\bf wxPCXHandler} will count the number of
different colours in the image; if there are 256 or less colours, it will
save as 8 bit, else it will save as 24 bit.

Loading PNMs only works for ASCII or raw RGB images. When saving in
PNM format, {\bf wxPNMHandler} will always save as raw RGB.

\wxheading{Derived from}

\helpref{wxObject}{wxobject}

\wxheading{Include files}

<wx/image.h>

\wxheading{See also}

\helpref{wxBitmap}{wxbitmap}, 
\helpref{wxInitAllImageHandlers}{wxinitallimagehandlers}

\latexignore{\rtfignore{\wxheading{Members}}}

\membersection{wxImage::wxImage}\label{wximageconstr}

\func{}{wxImage}{\void}

Default constructor.

\func{}{wxImage}{\param{const wxImage\& }{image}}

Copy constructor.

\func{}{wxImage}{\param{const wxBitmap\&}{ bitmap}}

Constructs an image from a platform-dependent bitmap. This preserves
mask information so that bitmaps and images can be converted back
and forth without loss in that respect.

\func{}{wxImage}{\param{int}{ width}, \param{int}{ height}}

Creates an image with the given width and height.

\func{}{wxImage}{\param{int}{ width}, \param{int}{ height}, \param{unsigned char*}{ data}, \param{bool}{ static_data=FALSE}}

Creates an image from given data with the given width and height. If 
{\it static_data} is TRUE, then wxImage will not delete the actual
image data in its destructor, otherwise it will free it by calling
{\it free()}.

\func{}{wxImage}{\param{const wxString\& }{name}, \param{long}{ type = wxBITMAP\_TYPE\_ANY}}

\func{}{wxImage}{\param{const wxString\& }{name}, \param{const wxString\&}{ mimetype}}

Loads an image from a file.

\func{}{wxImage}{\param{wxInputStream\& }{stream}, \param{long}{ type = wxBITMAP\_TYPE\_ANY}}

\func{}{wxImage}{\param{wxInputStream\& }{stream}, \param{const wxString\&}{ mimetype}}

Loads an image from an input stream.

\wxheading{Parameters}

\docparam{width}{Specifies the width of the image.}

\docparam{height}{Specifies the height of the image.}

\docparam{name}{Name of the file from which to load the image.}

\docparam{stream}{Opened input stream from which to load the image. Currently, the stream must support seeking.}

\docparam{type}{May be one of the following:

\twocolwidtha{5cm}%
\begin{twocollist}
\twocolitem{{\bf \indexit{wxBITMAP\_TYPE\_BMP}}}{Load a Windows bitmap file.}
\twocolitem{{\bf \indexit{wxBITMAP\_TYPE\_GIF}}}{Load a GIF bitmap file.}
\twocolitem{{\bf \indexit{wxBITMAP\_TYPE\_JPEG}}}{Load a JPEG bitmap file.}
\twocolitem{{\bf \indexit{wxBITMAP\_TYPE\_PNG}}}{Load a PNG bitmap file.}
\twocolitem{{\bf \indexit{wxBITMAP\_TYPE\_PCX}}}{Load a PCX bitmap file.}
\twocolitem{{\bf \indexit{wxBITMAP\_TYPE\_PNM}}}{Load a PNM bitmap file.}
\twocolitem{{\bf \indexit{wxBITMAP\_TYPE\_TIF}}}{Load a TIFF bitmap file.}
\twocolitem{{\bf \indexit{wxBITMAP\_TYPE\_ANY}}}{Will try to autodetect the format.}
\end{twocollist}}

\docparam{mimetype}{MIME type string (for example 'image/jpeg')}

\wxheading{Remarks}

Depending on how wxWindows has been configured, not all formats may be available.

Note: any handler other than BMP must be previously
initialized with \helpref{wxImage::AddHandler}{wximageaddhandler} or 
\helpref{wxInitAllImageHandlers}{wxinitallimagehandlers}.

\wxheading{See also}

\helpref{wxImage::LoadFile}{wximageloadfile}

\pythonnote{Constructors supported by wxPython are:\par
\indented{2cm}{\begin{twocollist}
\twocolitem{{\bf wxImage(name, flag)}}{Loads an image from a file}
\twocolitem{{\bf wxNullImage()}}{Create a null image (has no size or
image data)}
\twocolitem{{\bf wxEmptyImage(width, height)}}{Creates an empty image
of the given size}
\twocolitem{{\bf wxImageFromMime(name, mimetype}}{Creates an image from
the given file of the given mimetype}
\twocolitem{{\bf wxImageFromBitmap(bitmap)}}{Creates an image from a
platform-dependent bitmap}
\end{twocollist}}
}

\membersection{wxImage::\destruct{wxImage}}

\func{}{\destruct{wxImage}}{\void}

Destructor.

\membersection{wxImage::AddHandler}\label{wximageaddhandler}

\func{static void}{AddHandler}{\param{wxImageHandler*}{ handler}}

Adds a handler to the end of the static list of format handlers.

\docparam{handler}{A new image format handler object. There is usually only one instance
of a given handler class in an application session.}

\wxheading{See also}

\helpref{wxImageHandler}{wximagehandler}

\pythonnote{In wxPython this static method is named {\tt wxImage_AddHandler}.}
\membersection{wxImage::CleanUpHandlers}

\func{static void}{CleanUpHandlers}{\void}

Deletes all image handlers.

This function is called by wxWindows on exit.

\membersection{wxImage::ConvertToBitmap}\label{wximageconverttobitmap}

\constfunc{wxBitmap}{ConvertToBitmap}{\void}

Converts the image to a platform-specific bitmap object. This has to be done
to actually display an image as you cannot draw an image directly on a window.
The resulting bitmap will use the colour depth of the current system which entails
that a colour reduction has to take place. 

When in 8-bit mode (PseudoColour mode), the GTK port will use a color cube created 
on program start-up to look up colors. This ensures a very fast conversion, but
the image quality won't be perfect (and could be better for photo images using more
sophisticated dithering algorithms).

On Windows, if there is a palette present (set with SetPalette), it will be used when
creating the wxBitmap (most useful in 8-bit display mode). On other platforms,
the palette is currently ignored.

\membersection{wxImage::Copy}\label{wximagecopy}

\constfunc{wxImage}{Copy}{\void}

Returns an identical copy of the image.

\membersection{wxImage::Create}\label{wximagecreate}

\func{bool}{Create}{\param{int}{ width}, \param{int}{ height}}

Creates a fresh image.

\wxheading{Parameters}

\docparam{width}{The width of the image in pixels.}

\docparam{height}{The height of the image in pixels.}

\wxheading{Return value}

TRUE if the call succeeded, FALSE otherwise.

\membersection{wxImage::Destroy}\label{wximagedestroy}

\func{bool}{Destroy}{\void}

Destroys the image data.

\membersection{wxImage::FindHandler}

\func{static wxImageHandler*}{FindHandler}{\param{const wxString\& }{name}}

Finds the handler with the given name.

\func{static wxImageHandler*}{FindHandler}{\param{const wxString\& }{extension}, \param{long}{ imageType}}

Finds the handler associated with the given extension and type.

\func{static wxImageHandler*}{FindHandler}{\param{long }{imageType}}

Finds the handler associated with the given image type.

\func{static wxImageHandler*}{FindHandlerMime}{\param{const wxString\& }{mimetype}}

Finds the handler associated with the given MIME type.

\docparam{name}{The handler name.}

\docparam{extension}{The file extension, such as ``bmp".}

\docparam{imageType}{The image type, such as wxBITMAP\_TYPE\_BMP.}

\docparam{mimetype}{MIME type.}

\wxheading{Return value}

A pointer to the handler if found, NULL otherwise.

\wxheading{See also}

\helpref{wxImageHandler}{wximagehandler}

\membersection{wxImage::GetBlue}\label{wximagegetblue}

\constfunc{unsigned char}{GetBlue}{\param{int}{ x}, \param{int}{ y}}

Returns the blue intensity at the given coordinate.

\membersection{wxImage::GetData}\label{wximagegetdata}

\constfunc{unsigned char*}{GetData}{\void}

Returns the image data as an array. This is most often used when doing
direct image manipulation. The return value points to an array of
chararcters in RGBGBRGB... format.

\membersection{wxImage::GetGreen}\label{wximagegetgreen}

\constfunc{unsigned char}{GetGreen}{\param{int}{ x}, \param{int}{ y}}

Returns the green intensity at the given coordinate.

\membersection{wxImage::GetRed}\label{wximagegetred}

\constfunc{unsigned char}{GetRed}{\param{int}{ x}, \param{int}{ y}}

Returns the red intensity at the given coordinate.

\membersection{wxImage::GetHandlers}

\func{static wxList\&}{GetHandlers}{\void}

Returns the static list of image format handlers.

\wxheading{See also}

\helpref{wxImageHandler}{wximagehandler}

\membersection{wxImage::GetHeight}\label{wximagegetheight}

\constfunc{int}{GetHeight}{\void}

Gets the height of the image in pixels.

\membersection{wxImage::GetMaskBlue}\label{wximagegetmaskblue}

\constfunc{unsigned char}{GetMaskBlue}{\void}

Gets the blue value of the mask colour.

\membersection{wxImage::GetMaskGreen}\label{wximagegetmaskgreen}

\constfunc{unsigned char}{GetMaskGreen}{\void}

Gets the green value of the mask colour.

\membersection{wxImage::GetMaskRed}\label{wximagegetmaskred}

\constfunc{unsigned char}{GetMaskRed}{\void}

Gets the red value of the mask colour.

\membersection{wxImage::GetPalette}\label{wximagegetpalette}

\constfunc{const wxPalette\&}{GetPalette}{\void}

Returns the palette associated with the image. Currently the palette is only
used in ConvertToBitmap under Windows.

Eventually wxImage handlers will set the palette if one exists in the image file.

\membersection{wxImage::GetSubImage}\label{wximagegetsubimage}

\constfunc{wxImage}{GetSubImage}{\param{const wxRect\&}{ rect}}

Returns a sub image of the current one as long as the rect belongs entirely to 
the image.

\membersection{wxImage::GetWidth}\label{wximagegetwidth}

\constfunc{int}{GetWidth}{\void}

Gets the width of the image in pixels.

\wxheading{See also}

\helpref{wxImage::GetHeight}{wximagegetheight}

\membersection{wxImage::HasMask}\label{wximagehasmask}

\constfunc{bool}{HasMask}{\void}

Returns TRUE if there is a mask active, FALSE otherwise.

\membersection{wxImage::InitStandardHandlers}

\func{static void}{InitStandardHandlers}{\void}

Internal use only. Adds standard image format handlers. It only install BMP
for the time being, which is used by wxBitmap.

This function is called by wxWindows on startup, and shouldn't be called by
the user.

\wxheading{See also}

\helpref{wxImageHandler}{wximagehandler}, 
\helpref{wxInitAllImageHandlers}{wxinitallimagehandlers}

\membersection{wxImage::InsertHandler}

\func{static void}{InsertHandler}{\param{wxImageHandler*}{ handler}}

Adds a handler at the start of the static list of format handlers.

\docparam{handler}{A new image format handler object. There is usually only one instance
of a given handler class in an application session.}

\wxheading{See also}

\helpref{wxImageHandler}{wximagehandler}

\membersection{wxImage::LoadFile}\label{wximageloadfile}

\func{bool}{LoadFile}{\param{const wxString\&}{ name}, \param{long}{ type = wxBITMAP\_TYPE\_ANY}}

\func{bool}{LoadFile}{\param{const wxString\&}{ name}, \param{const wxString\&}{ mimetype}}

Loads an image from a file. If no handler type is provided, the library will
try to autodetect the format.

\func{bool}{LoadFile}{\param{wxInputStream\&}{ stream}, \param{long}{ type}}

\func{bool}{LoadFile}{\param{wxInputStream\&}{ stream}, \param{const wxString\&}{ mimetype}}

Loads an image from an input stream.

\wxheading{Parameters}

\docparam{name}{Name of the file from which to load the image.}

\docparam{stream}{Opened input stream from which to load the image. Currently, the stream must support seeking.}

\docparam{type}{One of the following values:

\twocolwidtha{5cm}%
\begin{twocollist}
\twocolitem{{\bf wxBITMAP\_TYPE\_BMP}}{Load a Windows image file.}
\twocolitem{{\bf wxBITMAP\_TYPE\_GIF}}{Load a GIF image file.}
\twocolitem{{\bf wxBITMAP\_TYPE\_JPEG}}{Load a JPEG image file.}
\twocolitem{{\bf wxBITMAP\_TYPE\_PCX}}{Load a PCX image file.}
\twocolitem{{\bf wxBITMAP\_TYPE\_PNG}}{Load a PNG image file.}
\twocolitem{{\bf wxBITMAP\_TYPE\_PNM}}{Load a PNM image file.}
\twocolitem{{\bf wxBITMAP\_TYPE\_TIF}}{Load a TIFF image file.}
\twocolitem{{\bf wxBITMAP\_TYPE\_ANY}}{Will try to autodetect the format.}
\end{twocollist}}

\docparam{mimetype}{MIME type string (for example 'image/jpeg')}

\wxheading{Remarks}

Depending on how wxWindows has been configured, not all formats may be available.

\wxheading{Return value}

TRUE if the operation succeeded, FALSE otherwise.

\wxheading{See also}

\helpref{wxImage::SaveFile}{wximagesavefile}

\pythonnote{In place of a single overloaded method name, wxPython
implements the following methods:\par
\indented{2cm}{\begin{twocollist}
\twocolitem{{\bf LoadFile(filename, type)}}{Loads an image of the given
type from a file}
\twocolitem{{\bf LoadMimeFile(filename, mimetype)}}{Loads an image of the given
mimetype from a file}
\end{twocollist}}
}


\membersection{wxImage::Ok}\label{wximageok}

\constfunc{bool}{Ok}{\void}

Returns TRUE if image data is present.

\membersection{wxImage::RemoveHandler}

\func{static bool}{RemoveHandler}{\param{const wxString\& }{name}}

Finds the handler with the given name, and removes it. The handler
is not deleted.

\docparam{name}{The handler name.}

\wxheading{Return value}

TRUE if the handler was found and removed, FALSE otherwise.

\wxheading{See also}

\helpref{wxImageHandler}{wximagehandler}

\membersection{wxImage::SaveFile}\label{wximagesavefile}

\func{bool}{SaveFile}{\param{const wxString\& }{name}, \param{int}{ type}}

\func{bool}{SaveFile}{\param{const wxString\& }{name}, \param{const wxString\&}{ mimetype}}

Saves a image in the named file.

\func{bool}{SaveFile}{\param{wxOutputStream\& }{stream}, \param{int}{ type}}

\func{bool}{SaveFile}{\param{wxOutputStream\& }{stream}, \param{const wxString\&}{ mimetype}}

Saves a image in the given stream.

\wxheading{Parameters}

\docparam{name}{Name of the file to save the image to.}

\docparam{stream}{Opened output stream to save the image to.}

\docparam{type}{Currently three types can be used:

\twocolwidtha{5cm}%
\begin{twocollist}
\twocolitem{{\bf wxBITMAP\_TYPE\_JPEG}}{Save a JPEG image file.}
\twocolitem{{\bf wxBITMAP\_TYPE\_PNG}}{Save a PNG image file.}
\twocolitem{{\bf wxBITMAP\_TYPE\_PCX}}{Save a PCX image file (tries to save as 8-bit if possible, falls back to 24-bit otherwise).}
\twocolitem{{\bf wxBITMAP\_TYPE\_PNM}}{Save a PNM image file (as raw RGB always).}
\end{twocollist}}

\docparam{mimetype}{MIME type.}

\wxheading{Return value}

TRUE if the operation succeeded, FALSE otherwise.

\wxheading{Remarks}

Depending on how wxWindows has been configured, not all formats may be available.

\wxheading{See also}

\helpref{wxImage::LoadFile}{wximageloadfile}

\pythonnote{In place of a single overloaded method name, wxPython
implements the following methods:\par
\indented{2cm}{\begin{twocollist}
\twocolitem{{\bf SaveFile(filename, type)}}{Saves the image using the given
type to the named file}
\twocolitem{{\bf SaveMimeFile(filename, mimetype)}}{Saves the image using the given
mimetype to the named file}
\end{twocollist}}
}

\membersection{wxImage::Mirror}\label{wximagemirror}

\constfunc{wxImage}{Mirror}{\param{bool}{ horizontally = TRUE}}

Returns a mirrored copy of the image. The parameter {\it horizontally}
indicates the orientation.

\membersection{wxImage::Replace}\label{wximagereplace}

\func{void}{Replace}{\param{unsigned char}{ r1}, \param{unsigned char}{ g1}, \param{unsigned char}{ b1},
\param{unsigned char}{ r2}, \param{unsigned char}{ g2}, \param{unsigned char}{ b2}}

Replaces the colour specified by {\it r1,g1,b1} by the colour {\it r2,g2,b2}.

\membersection{wxImage::Rescale}\label{wximagerescale}

\func{wxImage \&}{Rescale}{\param{int}{ width}, \param{int}{ height}}

Changes the size of the image in-place: after a call to this function, the
image will have the given width and height.

Returns the (modified) image itself.

\wxheading{See also}

\helpref{Scale}{wximagescale}

\membersection{wxImage::Rotate}\label{wximagerotate}

\func{wxImage}{Rotate}{\param{double}{ angle}, \param{const wxPoint\& }{rotationCentre},
 \param{bool}{ interpolating = TRUE}, \param{wxPoint*}{ offsetAfterRotation = NULL}}

Rotates the image about the given point, by {\it angle} radians. Passing TRUE
to {\it interpolating} results in better image quality, but is slower. If the
image has a mask, then the mask colour is used for the uncovered pixels in the
rotated image background. Else, black (rgb 0, 0, 0) will be used.

Returns the rotated image, leaving this image intact.

\membersection{wxImage::Rotate90}\label{wximagerotate90}

\constfunc{wxImage}{Rotate90}{\param{bool}{ clockwise = TRUE}}

Returns a copy of the image rotated 90 degrees in the direction
indicated by {\it clockwise}.

\membersection{wxImage::Scale}\label{wximagescale}

\constfunc{wxImage}{Scale}{\param{int}{ width}, \param{int}{ height}}

Returns a scaled version of the image. This is also useful for
scaling bitmaps in general as the only other way to scale bitmaps
is to blit a wxMemoryDC into another wxMemoryDC.

It may be mentioned that the GTK port uses this function internally
to scale bitmaps when using mapping modes in wxDC. 

Example:

\begin{verbatim}
    // get the bitmap from somewhere
    wxBitmap bmp = ...;

    // rescale it to have size of 32*32
    if ( bmp.GetWidth() != 32 || bmp.GetHeight() != 32 )
    {
        wxImage image(bmp);
        bmp = image.Scale(32, 32).ConvertToBitmap();

        // another possibility:
        image.Rescale(32, 32);
        bmp = image;
    }

\end{verbatim}

\wxheading{See also}

\helpref{Rescale}{wximagerescale}

\membersection{wxImage::SetData}\label{wximagesetdata}

\func{void}{SetData}{\param{unsigned char*}{data}}

Sets the image data without performing checks. The data given must have
the size (width*height*3) or results will be unexpected. Don't use this
method if you aren't sure you know what you are doing.

\membersection{wxImage::SetMask}\label{wximagesetmask}

\func{void}{SetMask}{\param{bool}{ hasMask = TRUE}}

Specifies whether there is a mask or not. The area of the mask is determined by the current mask colour.

\membersection{wxImage::SetMaskColour}\label{wximagesetmaskcolour}

\func{void}{SetMaskColour}{\param{unsigned char }{red}, \param{unsigned char }{blue}, \param{unsigned char }{green}}

Sets the mask colour for this image (and tells the image to use the mask).

\membersection{wxImage::SetPalette}\label{wximagesetpalette}

\func{void}{SetPalette}{\param{const wxPalette\&}{ palette}}

Associates a palette with the image. Currently, the palette is not used.

\membersection{wxImage::SetRGB}\label{wximagesetrgb}

\func{void}{SetRGB}{\param{int }{x}, \param{int }{y}, \param{unsigned char }{red}, \param{unsigned char }{green}, \param{unsigned char }{blue}}

Sets the pixel at the given coordinate. This routine performs bounds-checks
for the coordinate so it can be considered a safe way to manipulate the
data, but in some cases this might be too slow so that the data will have to
be set directly. In that case you will have to get access to the image data 
using the \helpref{GetData}{wximagegetdata} method.

\membersection{wxImage::operator $=$}

\func{wxImage\& }{operator $=$}{\param{const wxImage\& }{image}}

Assignment operator. This operator does not copy any data, but instead
passes a pointer to the data in {\it image} and increments a reference
counter. It is a fast operation.

\wxheading{Parameters}

\docparam{image}{Image to assign.}

\wxheading{Return value}

Returns 'this' object.

\membersection{wxImage::operator $==$}

\func{bool}{operator $==$}{\param{const wxImage\& }{image}}

Equality operator. This operator tests whether the internal data pointers are
equal (a fast test).

\wxheading{Parameters}

\docparam{image}{Image to compare with 'this'}

\wxheading{Return value}

Returns TRUE if the images were effectively equal, FALSE otherwise.

\membersection{wxImage::operator $!=$}

\func{bool}{operator $!=$}{\param{const wxImage\& }{image}}

Inequality operator. This operator tests whether the internal data pointers are
unequal (a fast test).

\wxheading{Parameters}

\docparam{image}{Image to compare with 'this'}

\wxheading{Return value}

Returns TRUE if the images were unequal, FALSE otherwise.

\section{\class{wxImageHandler}}\label{wximagehandler}

This is the base class for implementing image file loading/saving, and image creation from data.
It is used within wxImage and is not normally seen by the application.

If you wish to extend the capabilities of wxImage, derive a class from wxImageHandler
and add the handler using \helpref{wxImage::AddHandler}{wximageaddhandler} in your
application initialisation.

\wxheading{Note (Legal Issue)}

This software is based in part on the work of the Independent JPEG Group.

(Applies when wxWindows is linked with JPEG support. wxJPEGHandler uses libjpeg
created by IJG.)

\wxheading{Derived from}

\helpref{wxObject}{wxobject}

\wxheading{Include files}

<wx/image.h>

\wxheading{See also}

\helpref{wxImage}{wximage}, 
\helpref{wxInitAllImageHandlers}{wxinitallimagehandlers}

\latexignore{\rtfignore{\wxheading{Members}}}

\membersection{wxImageHandler::wxImageHandler}\label{wximagehandlerconstr}

\func{}{wxImageHandler}{\void}

Default constructor. In your own default constructor, initialise the members
m\_name, m\_extension and m\_type.

\membersection{wxImageHandler::\destruct{wxImageHandler}}

\func{}{\destruct{wxImageHandler}}{\void}

Destroys the wxImageHandler object.

\membersection{wxImageHandler::GetName}

\constfunc{wxString}{GetName}{\void}

Gets the name of this handler.

\membersection{wxImageHandler::GetExtension}

\constfunc{wxString}{GetExtension}{\void}

Gets the file extension associated with this handler.

\membersection{wxImageHandler::GetImageCount}\label{wximagehandlergetimagecount}

\func{int}{GetImageCount}{\param{wxInputStream\&}{ stream}}

If the image file contains more than one image and the image handler is capable 
of retrieving these individually, this function will return the number of
available images.

\docparam{stream}{Opened input stream for reading image data. Currently, the stream must support seeking.}

\wxheading{Return value}

Number of available images. For most image handles, this defaults to 1.

\membersection{wxImageHandler::GetType}

\constfunc{long}{GetType}{\void}

Gets the image type associated with this handler.

\membersection{wxImageHandler::GetMimeType}

\constfunc{wxString}{GetMimeType}{\void}

Gets the MIME type associated with this handler.

\membersection{wxImageHandler::LoadFile}\label{wximagehandlerloadfile}

\func{bool}{LoadFile}{\param{wxImage* }{image}, \param{wxInputStream\&}{ stream}, \param{bool}{ verbose=TRUE}, \param{int}{ index=0}}

Loads a image from a stream, putting the resulting data into {\it image}. If the image file contains
more than one image and the image handler is capable of retrieving these individually, {\it index}
indicates which image to read from the stream.

\wxheading{Parameters}

\docparam{image}{The image object which is to be affected by this operation.}

\docparam{stream}{Opened input stream for reading image data.}

\docparam{verbose}{If set to TRUE, errors reported by the image handler will produce wxLogMessages.}

\docparam{index}{The index of the image in the file (starting from zero).}

\wxheading{Return value}

TRUE if the operation succeeded, FALSE otherwise.

\wxheading{See also}

\helpref{wxImage::LoadFile}{wximageloadfile}, 
\helpref{wxImage::SaveFile}{wximagesavefile}, 
\helpref{wxImageHandler::SaveFile}{wximagehandlersavefile}

\membersection{wxImageHandler::SaveFile}\label{wximagehandlersavefile}

\func{bool}{SaveFile}{\param{wxImage* }{image}, \param{wxOutputStream\& }{stream}}

Saves a image in the output stream.

\wxheading{Parameters}

\docparam{image}{The image object which is to be affected by this operation.}

\docparam{stream}{Opened output stream for writing the data.}

\wxheading{Return value}

TRUE if the operation succeeded, FALSE otherwise.

\wxheading{See also}

\helpref{wxImage::LoadFile}{wximageloadfile}, 
\helpref{wxImage::SaveFile}{wximagesavefile}, 
\helpref{wxImageHandler::LoadFile}{wximagehandlerloadfile}

\membersection{wxImageHandler::SetName}

\func{void}{SetName}{\param{const wxString\& }{name}}

Sets the handler name.

\wxheading{Parameters}

\docparam{name}{Handler name.}

\membersection{wxImageHandler::SetExtension}

\func{void}{SetExtension}{\param{const wxString\& }{extension}}

Sets the handler extension.

\wxheading{Parameters}

\docparam{extension}{Handler extension.}

\membersection{wxImageHandler::SetMimeType}\label{wximagehandlersetmimetype}

\func{void}{SetMimeType}{\param{const wxString\& }{mimetype}}

Sets the handler MIME type.

\wxheading{Parameters}

\docparam{mimename}{Handler MIME type.}

\membersection{wxImageHandler::SetType}

\func{void}{SetType}{\param{long }{type}}

Sets the handler type.

\wxheading{Parameters}

\docparam{name}{Handler type.}

