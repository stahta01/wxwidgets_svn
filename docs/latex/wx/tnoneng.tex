\section{Writing non-English applications}\label{nonenglishoverview}

This article describes how to write applications that communicate with
user in language other than English. Unfortunately many languages use
different charsets under Unix and Windows (and other platforms, to make
situation even more complicated). These charsets usually differ in so
many characters it is impossible to use same texts under all platforms.
wxWindows provide mechanism that helps you avoid distributing many
identical, only differently encoded, packages with your application 
(e.g. help files and menu items in iso8859-13 and windows-1257). Thanks
to this mechanism you can distribute only let's say iso8859-13 data 
and it will be handled transparently under all systems.

Please read \helpref{Internationalization}\label{internationalization} which
describes locales concept.

Whereever in the following text {\it iso8859-2} and {\it windows-1250} are
used, any encodings are meant and any encodings may be substituted there.


\wxheading{Locales}

TODO

\wxheading{Converting data}

before storing / after loading

TODO

\wxheading{Font mapping}

TODO

\wxheading{Help files}

If you're using \helpref{wxHtmlHelpController}{wxhtmlhelpcontroller} there is
no problem at all. You must only make sure that all HTML files contain
META tag, e.g.

\begin{verbatim}
<meta http-equiv="Content-Type" content="iso8859-2">
\end{verbatim}

and that hhp project file contains one additional line in {\tt OPTIONS}
section:

\begin{verbatim}
Charset=iso8859-2
\end{verbatim}

This additional entry tells HTML help controller what encoding is used
in contents and index tables.
