\section{\class{wxDataStream}}\label{wxdatastream}

This class provides functions that read and write integers or double in a
portable way. So, a file written by an Intel processor can be read by a
Sparc or anything else.

\latexignore{\rtfignore{\wxheading{Members}}}

\membersection{wxDataStream::wxDataStream}\label{wxwaveconstr}

\func{}{wxDataStream}{\param{istream\&}{ stream}}

Constructs a datastream object from a C++ input stream. Only read methods will
be available.

\wxheading{Parameters}

\docparam{stream}{The C++ input stream.}

\func{}{wxDataStream}{\param{istream\&}{ stream}}

Constructs a datastream object from a C++ input stream. Only read methods will
be available.

\wxheading{Parameters}

\docparam{stream}{The C++ input stream.}

\membersection{wxDataStream::\destruct{wxDataStream}}

\func{}{\destruct{wxDataStream}}{\void}

Destroys the wxDataStream object.

\membersection{wxDataStream::Read8}

\func{unsigned char}{Read8}{\void}

Reads a single byte from the stream.

\membersection{wxDataStream::Read16}

\func{unsigned short}{Read16}{\void}

Reads a 16 bit integer from the stream.

\membersection{wxDataStream::Read32}

\func{unsigned long}{Read32}{\void}

Reads a 32 bit integer from the stream.

\membersection{wxDataStream::ReadDouble}

\func{double}{ReadDouble}{\void}

Reads a double (IEEE encoded) from the stream.

\membersection{wxDataStream::ReadString}

\func{wxString}{wxDataStream::ReadString}{\void}

Reads a string from a stream. Actually, this function first reads a byte
specifying the length of the string (without the last null character) and then
reads the string.

\membersection{wxDataStream::ReadLine}

\func{wxString}{wxDataStream::ReadLine}{\void}

Reads a line from the stream. A line is a string which ends with \\n or \\r\\n.

\membersection{wxDataStream::Write8}

\func{void}{wxDataStream::Write8}{{\param unsigned char }{i8}}

Writes the single byte {\it i8} to the stream.

\membersection{wxDataStream::Write16}

\func{void}{wxDataStream::Write16}{{\param unsigned short }{i16}}

Writes the 16 bit integer {\it i16} to the stream.

\membersection{wxDataStream::Write32}

\func{void}{wxDataStream::Write32}{{\param unsigned long }{i32}}

Writes the 32 bit integer {\it i32} to the stream.

\membersection{wxDataStream::WriteDouble}

\func{void}{wxDataStream::WriteDouble}{{\param double }{f}}

Writes the double {\it f} to the stream using the IEEE format.

\membersection{wxDataStream::WriteString}

\func{void}{wxDataStream::WriteString}{{\param const wxString\& }{string}}

Writes {\it string} to the stream. Actually, this method writes the size of
the string before writing {\it string} itself.

\membersection{wxDataStream::WriteLine}

\func{void}{wxDataStream::WriteLine}{{\param const wxString\& }{string}}

Writes {\it string} as a line. Depending on the operating system, it adds
\\n or \\r\\n.

%%% Local Variables: 
%%% mode: latex
%%% TeX-master: "referenc"
%%% End: 
