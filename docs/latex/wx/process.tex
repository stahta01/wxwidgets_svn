\section{\class{wxProcess}}\label{wxprocess}

The objects of this class are used in conjunction with the
\helpref{wxExecute}{wxexecute} function. When a wxProcess object is passed to
wxExecute(), its \helpref{OnTerminate()}{wxprocessonterminate} virtual method
is called when the process terminates. This allows the program to be
(asynchronously) notified about the process termination and also retrieve its
exit status which is unavailable from wxExecute() in the case of
asynchronous execution.

Please note that if the process termination notification is processed by the
parent, it is responsible for deleting the wxProcess object which sent it.
However, if it is not processed, the object will delete itself and so the
library users should only delete those objects whose notifications have been
processed (and call \helpref{Detach()}{wxprocessdetach} for others).

wxProcess also supports IO redirection of the child process. For this, you have
to call its \helpref{Redirect}{wxprocessredirect} method before passing it to
\helpref{wxExecute}{wxexecute}. If the child process was launched successfully,
\helpref{GetInputStream}{wxprocessgetinputstream},
\helpref{GetOutputStream}{wxprocessgetoutputstream} and
\helpref{GetErrorStream}{wxprocessgeterrorstream} can then be used to retrieve
the streams corresponding to the child process standard output, input and
error output respectively.

\perlnote{In wxPerl this class has an additional {\tt Destroy} method,
for explicit destruction.}

\wxheading{Derived from}

\helpref{wxEvtHandler}{wxevthandler}

\wxheading{Include files}

<wx/process.h>

\wxheading{Library}

\helpref{wxBase}{librarieslist}

\wxheading{See also}

\helpref{wxExecute}{wxexecute}\\
\helpref{exec sample}{sampleexec}

\latexignore{\rtfignore{\wxheading{Members}}}

\membersection{wxProcess::wxProcess}\label{wxprocessctor}

\func{}{wxProcess}{\param{wxEvtHandler *}{ parent = NULL}, \param{int}{ id = -1}}

\func{}{wxProcess}{\param{int }{flags}}

Constructs a process object. {\it id} is only used in the case you want to
use wxWidgets events. It identifies this object, or another window that will
receive the event.

If the {\it parent} parameter is different from NULL, it will receive
a wxEVT\_END\_PROCESS notification event (you should insert EVT\_END\_PROCESS
macro in the event table of the parent to handle it) with the given {\it id}.

The second constructor creates an object without any associated parent (and
hence no id neither) but allows to specify the {\it flags} which can have the
value of {\tt wxPROCESS\_DEFAULT} or {\tt wxPROCESS\_REDIRECT}. Specifying the
former value has no particular effect while using the latter one is equivalent
to calling \helpref{Redirect}{wxprocessredirect}.

\wxheading{Parameters}

\docparam{parent}{The event handler parent.}

\docparam{id}{id of an event.}

\docparam{flags}{either {\tt wxPROCESS\_DEFAULT} or {\tt wxPROCESS\_REDIRECT}}

\membersection{wxProcess::\destruct{wxProcess}}\label{wxprocessdtor}

\func{}{\destruct{wxProcess}}{\void}

Destroys the wxProcess object.

\membersection{wxProcess::CloseOutput}\label{wxprocesscloseoutput}

\func{void}{CloseOutput}{\void}

Closes the output stream (the one connected to the stdin of the child
process). This function can be used to indicate to the child process that
there is no more data to be read - usually, a filter program will only
terminate when the input stream is closed.

\membersection{wxProcess::Detach}\label{wxprocessdetach}

\func{void}{Detach}{\void}

Normally, a wxProcess object is deleted by its parent when it receives the
notification about the process termination. However, it might happen that the
parent object is destroyed before the external process is terminated (e.g. a
window from which this external process was launched is closed by the user)
and in this case it {\bf should not delete} the wxProcess object, but
{\bf should call Detach()} instead. After the wxProcess object is detached
from its parent, no notification events will be sent to the parent and the
object will delete itself upon reception of the process termination
notification.

\membersection{wxProcess::GetErrorStream}\label{wxprocessgeterrorstream}

\constfunc{wxInputStream* }{GetErrorStream}{\void}

Returns an input stream which corresponds to the standard error output (stderr)
of the child process.

\membersection{wxProcess::GetInputStream}\label{wxprocessgetinputstream}

\constfunc{wxInputStream* }{GetInputStream}{\void}

It returns an input stream corresponding to the standard output stream of the
subprocess. If it is NULL, you have not turned on the redirection.
See \helpref{wxProcess::Redirect}{wxprocessredirect}.

\membersection{wxProcess::GetOutputStream}\label{wxprocessgetoutputstream}

\constfunc{wxOutputStream* }{GetOutputStream}{\void}

It returns an output stream correspoding to the input stream of the subprocess.
If it is NULL, you have not turned on the redirection.
See \helpref{wxProcess::Redirect}{wxprocessredirect}.

\membersection{wxProcess::IsErrorAvailable}\label{wxprocessiserroravailable}

\constfunc{bool}{IsErrorAvailable}{\void}

Returns {\tt true} if there is data to be read on the child process standard
error stream.

\wxheading{See also}

\helpref{IsInputAvailable}{wxprocessisinputavailable}

\membersection{wxProcess::IsInputAvailable}\label{wxprocessisinputavailable}

\constfunc{bool}{IsInputAvailable}{\void}

Returns {\tt true} if there is data to be read on the child process standard
output stream. This allows to write simple (and extremely inefficient)
polling-based code waiting for a better mechanism in future wxWidgets versions.

See the \helpref{exec sample}{sampleexec} for an example of using this
function.

\wxheading{See also}

\helpref{IsInputOpened}{wxprocessisinputopened}

\membersection{wxProcess::IsInputOpened}\label{wxprocessisinputopened}

\constfunc{bool}{IsInputOpened}{\void}

Returns {\tt true} if the child process standard output stream is opened.

\membersection{wxProcess::Kill}\label{wxprocesskill}

\func{static wxKillError}{Kill}{\param{int}{ pid}, \param{wxSignal}{ signal = wxSIGNONE}, \param{int }{flags = wxKILL\_NOCHILDREN}}

Send the specified signal to the given process. Possible signal values are:

\begin{verbatim}
enum wxSignal
{
    wxSIGNONE = 0,  // verify if the process exists under Unix
    wxSIGHUP,
    wxSIGINT,
    wxSIGQUIT,
    wxSIGILL,
    wxSIGTRAP,
    wxSIGABRT,
    wxSIGEMT,
    wxSIGFPE,
    wxSIGKILL,      // forcefully kill, dangerous!
    wxSIGBUS,
    wxSIGSEGV,
    wxSIGSYS,
    wxSIGPIPE,
    wxSIGALRM,
    wxSIGTERM       // terminate the process gently
};
\end{verbatim}

{\tt wxSIGNONE}, {\tt wxSIGKILL} and {\tt wxSIGTERM} have the same meaning
under both Unix and Windows but all the other signals are equivalent to
{\tt wxSIGTERM} under Windows.

The {\it flags} parameter can be wxKILL\_NOCHILDREN (the default),
or wxKILL\_CHILDREN, in which case the child processes of this
process will be killed too. Note that under Unix, for wxKILL\_CHILDREN
to work you should have created the process passing wxEXEC\_MAKE\_GROUP\_LEADER.

Returns the element of {\tt wxKillError} enum:

\begin{verbatim}
enum wxKillError
{
    wxKILL_OK,              // no error
    wxKILL_BAD_SIGNAL,      // no such signal
    wxKILL_ACCESS_DENIED,   // permission denied
    wxKILL_NO_PROCESS,      // no such process
    wxKILL_ERROR            // another, unspecified error
};
\end{verbatim}

\wxheading{See also}

\helpref{wxProcess::Exists}{wxprocessexists},\rtfsp
\helpref{wxKill}{wxkill},\rtfsp
\helpref{Exec sample}{sampleexec}

\membersection{wxProcess::Exists}\label{wxprocessexists}

\func{static bool}{Exists}{\param{int}{ pid}}

Returns {\tt true} if the given process exists in the system.

\wxheading{See also}

\helpref{wxProcess::Kill}{wxprocesskill},\rtfsp
\helpref{Exec sample}{sampleexec}

\membersection{wxProcess::OnTerminate}\label{wxprocessonterminate}

\func{void}{OnTerminate}{\param{int}{ pid}, \param{int}{ status}}

It is called when the process with the pid {\it pid} finishes.
It raises a wxWidgets event when it isn't overridden.

\docparam{pid}{The pid of the process which has just terminated.}

\docparam{status}{The exit code of the process.}

\membersection{wxProcess::Open}\label{wxprocessopen}

\func{static wxProcess *}{Open}{\param{const wxString\& }{cmd}, \param{int }{flags = wxEXEC\_ASYNC}}

This static method replaces the standard {\tt popen()} function: it launches
the process specified by the {\it cmd} parameter and returns the wxProcess
object which can be used to retrieve the streams connected to the standard
input, output and error output of the child process.

If the process couldn't be launched, {\tt NULL} is returned. Note that in any
case the returned pointer should {\bf not} be deleted, rather the process
object will be destroyed automatically when the child process terminates. This
does mean that the child process should be told to quit before the main program
exits to avoid memory leaks.

\wxheading{Parameters}

\docparam{cmd}{The command to execute, including optional arguments.}
\docparam{flags}{The flags to pass to \helpref{wxExecute}{wxexecute}.
  NOTE: wxEXEC\_SYNC should not be used.}

\wxheading{Return value}

A pointer to new wxProcess object or {\tt NULL} on error.

\wxheading{See also}

\helpref{wxExecute}{wxexecute}

\membersection{wxProcess::GetPid}\label{wxprocessgetpid}

\constfunc{long}{GetPid}{\void}

Returns the process ID of the process launched by \helpref{Open}{wxprocessopen}.

\membersection{wxProcess::Redirect}\label{wxprocessredirect}

\func{void}{Redirect}{\void}

Turns on redirection. wxExecute will try to open a couple of pipes
to catch the subprocess stdio. The caught input stream is returned by
GetOutputStream() as a non-seekable stream. The caught output stream is returned
by GetInputStream() as a non-seekable stream.

