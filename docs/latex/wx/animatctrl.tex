%%%%%%%%%%%%%%%%%%%%%%%%%%%%%%%%%%%%%%%%%%%%%%%%%%%%%%%%%%%%%%%%%%%%%%%%%%%%%%%
%% Name:        animatctrl.tex
%% Purpose:     wxAnimationCtrl documentation
%% Author:      Francesco Montorsi
%% Created:     24-9-2006
%% RCS-ID:      $Id$
%% Copyright:   (c) 2006 Francesco Montorsi
%% License:     wxWindows license
%%%%%%%%%%%%%%%%%%%%%%%%%%%%%%%%%%%%%%%%%%%%%%%%%%%%%%%%%%%%%%%%%%%%%%%%%%%%%%%

\section{\class{wxAnimationCtrl}}\label{wxanimationctrl}

This is a static control which displays an animation.
wxAnimationCtrl API is simple as possible and won't give you full control on the
animation; if you need it then use \helpref{wxMediaCtrl}{wxmediactrl}.

This control is useful to display a (small) animation while doing a long task
(e.g. a "throbber").

It is only available if \texttt{wxUSE\_ANIMATIONCTRL} is set to $1$ (the default).

\wxheading{Derived from}

\helpref{wxControl}{wxcontrol}\\
\helpref{wxWindow}{wxwindow}\\
\helpref{wxEvtHandler}{wxevthandler}\\
\helpref{wxObject}{wxobject}

\wxheading{Include files}

<wx/animate.h>

\wxheading{Window styles}

\twocolwidtha{5cm}%
\begin{twocollist}\itemsep=0pt
\twocolitem{\windowstyle{wxAC\_DEFAULT\_STYLE}}{The default style: wxNO_BORDER.}
\twocolitem{\windowstyle{wxAC\_NO_AUTORESIZE}}{By default, the control will adjust
its size to exactly fit to the size of the animation when \helpref{SetAnimation}{wxanimationctrlsetanimation}
is called. If this style flag is given, the control will not change its size}
\end{twocollist}

%\wxheading{Event handling}
%
%\twocolwidtha{7cm}%
%\begin{twocollist}\itemsep=0pt
%\twocolitem{{\bf EVT\_ANIMATION\_END(id, func)}}{}
%\end{twocollist}

\wxheading{See also}

\helpref{wxAnimation}{wxanimation}


\latexignore{\rtfignore{\wxheading{Members}}}

\membersection{wxAnimationCtrl::wxAnimationCtrl}\label{wxanimationctrl}

\func{}{wxAnimationCtrl}{\param{wxWindow *}{parent},\rtfsp
\param{wxWindowID}{ id},\rtfsp
\param{const wxAnimation\& }{anim},\rtfsp
\param{const wxPoint\& }{pos = wxDefaultPosition},\rtfsp
\param{const wxSize\& }{size = wxDefaultSize},\rtfsp
\param{long}{ style = wxAC\_DEFAULT\_STYLE},\rtfsp
\param{const wxString\& }{name = ``animationctrl"}}

Initializes the object and calls \helpref{Create}{wxanimationctrlcreate} with
all the parameters.


\membersection{wxAnimationCtrl::Create}\label{wxanimationctrlcreate}

\func{bool}{Create}{\param{wxWindow *}{parent},\rtfsp
\param{wxWindowID}{ id},\rtfsp
\param{const wxAnimation\& }{anim},\rtfsp
\param{const wxPoint\& }{pos = wxDefaultPosition},\rtfsp
\param{const wxSize\& }{size = wxDefaultSize},\rtfsp
\param{long}{ style = wxAC\_DEFAULT\_STYLE},\rtfsp
\param{const wxString\& }{name = ``animationctrl"}}

\wxheading{Parameters}

\docparam{parent}{Parent window, must not be non-\texttt{NULL}.}

\docparam{id}{The identifier for the control.}

\docparam{anim}{The initial animation shown in the control.}

\docparam{pos}{Initial position.}

\docparam{size}{Initial size.}

\docparam{style}{The window style, see {\tt wxAC\_*} flags.}

\docparam{name}{Control name.}

After control creation you must explicitely call \helpref{Play}{wxanimationctrlplay}
to start to play the animation. Until that function won't be called, the first frame
of the animation is displayed.

\wxheading{Return value}

\true if the control was successfully created or \false if creation failed.


\membersection{wxAnimationCtrl::GetAnimation}\label{wxanimationctrlgetanimation}

\constfunc{wxAnimation}{GetAnimation}{\void}

Returns the animation associated with this control.


\membersection{wxAnimationCtrl::IsPlaying}\label{wxanimationctrlisplaying}

\constfunc{bool}{IsPlaying}{\void}

Returns \true if the animation is being played.


\membersection{wxAnimationCtrl::LoadFile}\label{wxanimationctrlloadfile}

\func{bool}{LoadFile}{\param{const wxString \&}{ file}, \param{wxAnimationType }{ animType = wxANIMATION\_TYPE\_ANY}}

Loads the animation from the given file and calls \helpref{SetAnimation}{wxanimationctrlsetanimation}.
See \helpref{wxAnimation::LoadFile}{wxanimationloadfile} for more info.


\membersection{wxAnimationCtrl::Play}\label{wxanimationctrlplay}

\func{bool}{Play}{\void}

Starts playing the animation.
The animation is always played in loop mode (unless the last frame of the animation
has an infinite delay time) and always start from the first frame
(even if you \helpref{atopped}{wxanimationctrlstop} it while some other frame was
displayed).


\membersection{wxAnimationCtrl::SetAnimation}\label{wxanimationctrlsetanimation}

\func{void}{SetAnimation}{\param{const wxAnimation \&}{ anim}}

Sets the animation to play in this control.
If the previous animation is being played, it's \helpref{Stopped}{wxanimationctrlstop}.

Until \helpref{Play}{wxanimationctrlplay} isn't called, the first frame
of the animation is displayed.

If {\tt wxNullAnimation} is given as animation, the control will be cleared to display
the background colour (see \helpref{wxWindow::GetBackgroundColour}{wxwindowgetbackgroundcolour}).


\membersection{wxAnimationCtrl::Stop}\label{wxanimationctrlstop}

\func{void}{Stop}{\void}

Stops playing the animation.
The control will show the last frame rendered of the current animation until
\helpref{Play}{wxanimationctrlplay} is called.
