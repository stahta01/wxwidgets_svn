%%%%%%%%%%%%%%%%%%%%%%%%%%%%%%%%%%%%%%%%%%%%%%%%%%%%%%%%%%%%%%%%%%%%%%%%%%%%%%%
%% Name:        filename.tex
%% Purpose:     wxFileName documentation
%% Author:      Vadim Zeitlin
%% Modified by:
%% Created:     30.11.01
%% RCS-ID:      $Id$
%% Copyright:   (c) 2001 Vadim Zeitlin
%% License:     wxWindows license
%%%%%%%%%%%%%%%%%%%%%%%%%%%%%%%%%%%%%%%%%%%%%%%%%%%%%%%%%%%%%%%%%%%%%%%%%%%%%%%

\section{\class{wxFileName}}\label{wxfilename}

wxFileName encapsulates a file name. This class serves two purposes: first, it
provides the functions to split the file names into components and to recombine
these components in the full file name which can then be passed to the OS file
functions (and \helpref{wxWidgets functions}{filefunctions} wrapping them).
Second, it includes the functions for working with the files itself. Note that
to change the file data you should use \helpref{wxFile}{wxfile} class instead.
wxFileName provides functions for working with the file attributes.

\wxheading{Derived from}

No base class

\wxheading{Include files}

<wx/filename.h>

\wxheading{Data structures}

Many wxFileName methods accept the path format argument which is by\rtfsp
{\tt wxPATH\_NATIVE} by default meaning to use the path format native for the
current platform.

The path format affects the operation of wxFileName functions in several ways:
first and foremost, it defines the path separator character to use, but it also
affects other things such as whether the path has the drive part or not.

\begin{verbatim}
enum wxPathFormat
{
    wxPATH_NATIVE = 0,      // the path format for the current platform
    wxPATH_UNIX,
    wxPATH_BEOS = wxPATH_UNIX,
    wxPATH_MAC,
    wxPATH_DOS,
    wxPATH_WIN = wxPATH_DOS,
    wxPATH_OS2 = wxPATH_DOS,
    wxPATH_VMS,

    wxPATH_MAX // Not a valid value for specifying path format
}
\end{verbatim}

\latexignore{\rtfignore{\wxheading{Function groups}}}


\membersection{File name format}\label{filenameformat}

wxFileName currently supports the file names in the Unix, DOS/Windows, Mac OS
and VMS formats. Although these formats are quite different, wxFileName tries
to treat them all in the same generic way. It supposes that all file names
consist of the following parts: the volume (also known as drive under Windows
or device under VMS), the path which is a sequence of directory names separated
by the \helpref{path separators}{wxfilenamegetpathseparators} and the full
filename itself which, in turn, is composed from the base file name and the
extension. All of the individual components of the file name may be empty and,
for example, the volume name is always empty under Unix, but if they are all
empty simultaneously, the filename object is considered to be in an invalid
state and \helpref{IsOk}{wxfilenameisok} returns {\tt false} for it.

File names can be case-sensitive or not, the function\rtfsp
\helpref{IsCaseSensitive}{wxfilenameiscasesensitive} allows to determine this.

The rules for determining if the file name is absolute or relative also depends
on the file name format and the only portable way to answer to this question is
to use \helpref{IsAbsolute}{wxfilenameisabsolute} method. To ensure that the
filename is absolute you may use \helpref{MakeAbsolute}{wxfilenamemakeabsolute}.
There is also an inverse function 
\helpref{MakeRelativeTo}{wxfilenamemakerelativeto} which undoes what
\helpref{Normalize(wxPATH\_NORM\_DOTS)}{wxfilenamenormalize} does.

Other functions returning information about the file format provided by this
class are \helpref{GetVolumeSeparator}{wxfilenamegetvolumeseparator},\rtfsp
\helpref{IsPathSeparator}{wxfilenameispathseparator}.

\helpref{IsRelative}{wxfilenameisrelative}


\membersection{File name construction}\label{filenameconstruction}

TODO.


\membersection{File tests}\label{filetests}

Before doing the other tests you should use \helpref{IsOk}{wxfilenameisok} to
verify that the filename is well defined. If it is, 
\helpref{FileExists}{wxfilenamefileexists} can be used to test if a file with
such name exists and \helpref{DirExists}{wxfilenamedirexists} - if a directory
with this name exists.

File names should be compared using \helpref{SameAs}{wxfilenamesameas} method
or \helpref{$==$}{wxfilenameoperatorequal}.


\membersection{File name components}\label{filenamecomponents}

These functions allow to examine and modify the individual directories of the
path:

\helpref{AppendDir}{wxfilenameappenddir}\\
\helpref{InsertDir}{wxfilenameinsertdir}\\
\helpref{GetDirCount}{wxfilenamegetdircount}
\helpref{PrependDir}{wxfilenameprependdir}\\
\helpref{RemoveDir}{wxfilenameremovedir}\\
\helpref{RemoveLastDir}{wxfilenameremovelastdir}

To change the components of the file name individually you can use the
following functions:

\helpref{GetExt}{wxfilenamegetext}\\
\helpref{GetName}{wxfilenamegetname}\\
\helpref{GetVolume}{wxfilenamegetvolume}\\
\helpref{HasExt}{wxfilenamehasext}\\
\helpref{HasName}{wxfilenamehasname}\\
\helpref{HasVolume}{wxfilenamehasvolume}\\
\helpref{SetExt}{wxfilenamesetext}\\
\helpref{ClearExt}{wxfilenameclearext}\\
\helpref{SetEmptyExt}{wxfilenamesetemptyext}\\
\helpref{SetName}{wxfilenamesetname}\\
\helpref{SetVolume}{wxfilenamesetvolume}\\


\membersection{Operations}\label{filenameoperations}

These methods allow to work with the file creation, access and modification
times. Note that not all filesystems under all platforms implement these times
in the same way. For example, the access time under Windows has a resolution of
one day (so it is really the access date and not time). The access time may be
updated when the file is executed or not depending on the platform.

\helpref{GetModificationTime}{wxfilenamegetmodificationtime}\\
\helpref{GetTimes}{wxfilenamegettimes}\\
\helpref{SetTimes}{wxfilenamesettimes}\\
\helpref{Touch}{wxfilenametouch}

Other file system operations functions are:

\helpref{Mkdir}{wxfilenamemkdir}\\
\helpref{Rmdir}{wxfilenamermdir}

\latexignore{\rtfignore{\wxheading{Members}}}


\membersection{wxFileName::wxFileName}\label{wxfilenamewxfilename}

\func{}{wxFileName}{\void}

Default constructor.

\func{}{wxFileName}{\param{const wxFileName\& }{filename}}

Copy constructor.

\func{}{wxFileName}{\param{const wxString\& }{fullpath}, \param{wxPathFormat }{format = wxPATH\_NATIVE}}

Constructor taking a full filename. If it terminates with a '/', a directory path
is constructed (the name will be empty), otherwise a file name and
extension are extracted from it.

\func{}{wxFileName}{\param{const wxString\& }{path}, \param{const wxString\& }{name}, \param{wxPathFormat }{format = wxPATH\_NATIVE}}

Constructor from a directory name and a file name.

\func{}{wxFileName}{\param{const wxString\& }{path}, \param{const wxString\& }{name}, \param{const wxString\& }{ext}, \param{wxPathFormat }{format = wxPATH\_NATIVE}}

Constructor from a directory name, base file name and extension.

\func{}{wxFileName}{\param{const wxString\& }{volume}, \param{const wxString\& }{path}, \param{const wxString\& }{name}, \param{const wxString\& }{ext}, \param{wxPathFormat }{format = wxPATH\_NATIVE}}

Constructor from a volume name, a directory name, base file name and extension.


\membersection{wxFileName::AppendDir}\label{wxfilenameappenddir}

\func{void}{AppendDir}{\param{const wxString\& }{dir}}

Appends a directory component to the path. This component should contain a
single directory name level, i.e. not contain any path or volume separators nor
should it be empty, otherwise the function does nothing (and generates an
assert failure in debug build).


\membersection{wxFileName::Assign}\label{wxfilenameassign}

\func{void}{Assign}{\param{const wxFileName\& }{filepath}}

\func{void}{Assign}{\param{const wxString\& }{fullpath}, \param{wxPathFormat }{format = wxPATH\_NATIVE}}

\func{void}{Assign}{\param{const wxString\& }{volume}, \param{const wxString\& }{path}, \param{const wxString\& }{name}, \param{const wxString\& }{ext}, \param{bool }{hasExt}, \param{wxPathFormat }{format = wxPATH\_NATIVE}}

\func{void}{Assign}{\param{const wxString\& }{volume}, \param{const wxString\& }{path}, \param{const wxString\& }{name}, \param{const wxString\& }{ext}, \param{wxPathFormat }{format = wxPATH\_NATIVE}}

\func{void}{Assign}{\param{const wxString\& }{path}, \param{const wxString\& }{name}, \param{wxPathFormat }{format = wxPATH\_NATIVE}}

\func{void}{Assign}{\param{const wxString\& }{path}, \param{const wxString\& }{name}, \param{const wxString\& }{ext}, \param{wxPathFormat }{format = wxPATH\_NATIVE}}

Creates the file name from various combinations of data.


\membersection{wxFileName::AssignCwd}\label{wxfilenameassigncwd}

\func{static void}{AssignCwd}{\param{const wxString\& }{volume = wxEmptyString}}

Makes this object refer to the current working directory on the specified
volume (or current volume if {\it volume} is empty).

\wxheading{See also}

\helpref{GetCwd}{wxfilenamegetcwd}


\membersection{wxFileName::AssignDir}\label{wxfilenameassigndir}

\func{void}{AssignDir}{\param{const wxString\& }{dir}, \param{wxPathFormat }{format = wxPATH\_NATIVE}}

Sets this file name object to the given directory name. The name and extension
will be empty.


\membersection{wxFileName::AssignHomeDir}\label{wxfilenameassignhomedir}

\func{void}{AssignHomeDir}{\void}

Sets this file name object to the home directory.


\membersection{wxFileName::AssignTempFileName}\label{wxfilenameassigntempfilename}

\func{void}{AssignTempFileName}{\param{const wxString\& }{prefix}, \param{wxFile *}{fileTemp = {\tt NULL}}}

The function calls \helpref{CreateTempFileName}{wxfilenamecreatetempfilename} to
create a temporary file and sets this object to the name of the file. If a
temporary file couldn't be created, the object is put into the\rtfsp
\helpref{invalid}{wxfilenameisok} state.


\membersection{wxFileName::Clear}\label{wxfilenameclear}

\func{void}{Clear}{\void}

Reset all components to default, uninitialized state.


\membersection{wxFileName::ClearExt}\label{wxfilenameclearext}

\func{void}{SetClearExt}{\void}

Removes the extension from the file name resulting in a 
file name with no trailing dot.

\wxheading{See also}

\helpref{SetExt}{wxfilenamesetext}
\helpref{SetEmptyExt}{wxfilenamesetemptyext}

\membersection{wxFileName::CreateTempFileName}\label{wxfilenamecreatetempfilename}

\func{static wxString}{CreateTempFileName}{\param{const wxString\& }{prefix}, \param{wxFile *}{fileTemp = {\tt NULL}}}

Returns a temporary file name starting with the given {\it prefix}. If
the {\it prefix} is an absolute path, the temporary file is created in this
directory, otherwise it is created in the default system directory for the
temporary files or in the current directory.

If the function succeeds, the temporary file is actually created. If\rtfsp
{\it fileTemp} is not {\tt NULL}, this file will be opened using the name of
the temporary file. When possible, this is done in an atomic way ensuring that
no race condition occurs between the temporary file name generation and opening
it which could often lead to security compromise on the multiuser systems.
If {\it fileTemp} is {\tt NULL}, the file is only created, but not opened.

Under Unix, the temporary file will have read and write permissions for the
owner only to minimize the security problems.

\wxheading{Parameters}

\docparam{prefix}{Prefix to use for the temporary file name construction}

\docparam{fileTemp}{The file to open or {\tt NULL} to just get the name}

\wxheading{Return value}

The full temporary file name or an empty string on error.


\membersection{wxFileName::DirExists}\label{wxfilenamedirexists}

\constfunc{bool}{DirExists}{\void}

\func{static bool}{DirExists}{\param{const wxString\& }{dir}}

Returns {\tt true} if the directory with this name exists.


\membersection{wxFileName::DirName}\label{wxfilenamedirname}

\func{static wxFileName}{DirName}{\param{const wxString\& }{dir}, \param{wxPathFormat }{format = wxPATH\_NATIVE}}

Returns the object corresponding to the directory with the given name.
The {\it dir} parameter may have trailing path separator or not.



\membersection{wxFileName::FileExists}\label{wxfilenamefileexists}

\constfunc{bool}{FileExists}{\void}

\func{static bool}{FileExists}{\param{const wxString\& }{file}}

Returns {\tt true} if the file with this name exists.

\wxheading{See also}

\helpref{DirExists}{wxfilenamedirexists}



\membersection{wxFileName::FileName}\label{wxfilenamefilename}

\func{static wxFileName}{FileName}{\param{const wxString\& }{file}, \param{wxPathFormat }{format = wxPATH\_NATIVE}}

Returns the file name object corresponding to the given {\it file}. This
function exists mainly for symmetry with \helpref{DirName}{wxfilenamedirname}.



\membersection{wxFileName::GetCwd}\label{wxfilenamegetcwd}

\func{static wxString}{GetCwd}{\param{const wxString\& }{volume = ""}}

Retrieves the value of the current working directory on the specified volume. If
the volume is empty, the program's current working directory is returned for the
current volume.

\wxheading{Return value}

The string containing the current working directory or an empty string on
error.

\wxheading{See also}

\helpref{AssignCwd}{wxfilenameassigncwd}


\membersection{wxFileName::GetDirCount}\label{wxfilenamegetdircount}

\constfunc{size\_t}{GetDirCount}{\void}

Returns the number of directories in the file name.


\membersection{wxFileName::GetDirs}\label{wxfilenamegetdirs}

\constfunc{const wxArrayString\&}{GetDirs}{\void}

Returns the directories in string array form.


\membersection{wxFileName::GetExt}\label{wxfilenamegetext}

\constfunc{wxString}{GetExt}{\void}

Returns the file name extension.


\membersection{wxFileName::GetForbiddenChars}\label{wxfilenamegetforbiddenchars}

\func{static wxString}{GetForbiddenChars}{\param{wxPathFormat }{format = wxPATH\_NATIVE}}

Returns the characters that can't be used in filenames and directory names for the specified format.


\membersection{wxFileName::GetFormat}\label{wxfilenamegetformat}

\func{static wxPathFormat}{GetFormat}{\param{wxPathFormat }{format = wxPATH\_NATIVE}}

Returns the canonical path format for this platform.


\membersection{wxFileName::GetFullName}\label{wxfilenamegetfullname}

\constfunc{wxString}{GetFullName}{\void}

Returns the full name (including extension but excluding directories).


\membersection{wxFileName::GetFullPath}\label{wxfilenamegetfullpath}

\constfunc{wxString}{GetFullPath}{\param{wxPathFormat }{format = wxPATH\_NATIVE}}

Returns the full path with name and extension.


\membersection{wxFileName::GetHomeDir}\label{wxfilenamegethomedir}

\func{static wxString}{GetHomeDir}{\void}

Returns the home directory.


\membersection{wxFileName::GetLongPath}\label{wxfilenamegetlongpath}

\constfunc{wxString}{GetLongPath}{\void}

Return the long form of the path (returns identity on non-Windows platforms)


\membersection{wxFileName::GetModificationTime}\label{wxfilenamegetmodificationtime}

\constfunc{wxDateTime}{GetModificationTime}{\void}

Returns the last time the file was last modified.


\membersection{wxFileName::GetName}\label{wxfilenamegetname}

\constfunc{wxString}{GetName}{\void}

Returns the name part of the filename.


\membersection{wxFileName::GetPath}\label{wxfilenamegetpath}

\constfunc{wxString}{GetPath}{\param{int }{flags = {\tt wxPATH\_GET\_VOLUME}}, \param{wxPathFormat }{format = wxPATH\_NATIVE}}

Returns the path part of the filename (without the name or extension). The
possible flags values are:

\twocolwidtha{5cm}
\begin{twocollist}\itemsep=0pt
\twocolitem{{\bf wxPATH\_GET\_VOLUME}}{Return the path with the volume (does
nothing for the filename formats without volumes), otherwise the path without
volume part is returned.}
\twocolitem{{\bf wxPATH\_GET\_SEPARATOR}}{Return the path with the trailing
separator, if this flag is not given there will be no separator at the end of
the path.}
\end{twocollist}


\membersection{wxFileName::GetPathSeparator}\label{wxfilenamegetpathseparator}

\func{static wxChar}{GetPathSeparator}{\param{wxPathFormat }{format = wxPATH\_NATIVE}}

Returns the usually used path separator for this format. For all formats but 
{\tt wxPATH\_DOS} there is only one path separator anyhow, but for DOS there
are two of them and the native one, i.e. the backslash is returned by this
method.

\wxheading{See also}

\helpref{GetPathSeparators}{wxfilenamegetpathseparators}


\membersection{wxFileName::GetPathSeparators}\label{wxfilenamegetpathseparators}

\func{static wxString}{GetPathSeparators}{\param{wxPathFormat }{format = wxPATH\_NATIVE}}

Returns the string containing all the path separators for this format. For all
formats but {\tt wxPATH\_DOS} this string contains only one character but for
DOS and Windows both {\tt '/'} and {\tt '\textbackslash'} may be used as
separators.

\wxheading{See also}

\helpref{GetPathSeparator}{wxfilenamegetpathseparator}


\membersection{wxFileName::GetPathTerminators}\label{wxfilenamegetpathterminators}

\func{static wxString}{GetPathTerminators}{\param{wxPathFormat }{format = wxPATH\_NATIVE}}

Returns the string of characters which may terminate the path part. This is the
same as \helpref{GetPathSeparators}{wxfilenamegetpathseparators} except for VMS
path format where $]$ is used at the end of the path part.


\membersection{wxFileName::GetShortPath}\label{wxfilenamegetshortpath}

\constfunc{wxString}{GetShortPath}{\void}

Return the short form of the path (returns identity on non-Windows platforms).


\membersection{wxFileName::GetTimes}\label{wxfilenamegettimes}

\constfunc{bool}{GetTimes}{\param{wxDateTime* }{dtAccess}, \param{wxDateTime* }{dtMod}, \param{wxDateTime* }{dtCreate}}

Returns the last access, last modification and creation times. The last access
time is updated whenever the file is read or written (or executed in the case
of Windows), last modification time is only changed when the file is written
to. Finally, the creation time is indeed the time when the file was created
under Windows and the inode change time under Unix (as it is impossible to
retrieve the real file creation time there anyhow) which can also be changed
by many operations after the file creation.

Any of the pointers may be {\tt NULL} if the corresponding time is not
needed.

\wxheading{Return value}

{\tt true} on success, {\tt false} if we failed to retrieve the times.


\membersection{wxFileName::GetVolume}\label{wxfilenamegetvolume}

\constfunc{wxString}{GetVolume}{\void}

Returns the string containing the volume for this file name, empty if it
doesn't have one or if the file system doesn't support volumes at all (for
example, Unix).


\membersection{wxFileName::GetVolumeSeparator}\label{wxfilenamegetvolumeseparator}

\func{static wxString}{GetVolumeSeparator}{\param{wxPathFormat }{format = wxPATH\_NATIVE}}

Returns the string separating the volume from the path for this format.


\membersection{wxFileName::HasExt}\label{wxfilenamehasext}

\constfunc{bool}{HasExt}{\void}

Returns {\tt true} if an extension is present.


\membersection{wxFileName::HasName}\label{wxfilenamehasname}

\constfunc{bool}{HasName}{\void}

Returns {\tt true} if a name is present.


\membersection{wxFileName::HasVolume}\label{wxfilenamehasvolume}

\constfunc{bool}{HasVolume}{\void}

Returns {\tt true} if a volume specifier is present.


\membersection{wxFileName::InsertDir}\label{wxfilenameinsertdir}

\func{void}{InsertDir}{\param{size\_t }{before}, \param{const wxString\& }{dir}}

Inserts a directory component before the zero-based position in the directory
list. Please see \helpref{AppendDir}{wxfilenameappenddir} for important notes.


\membersection{wxFileName::IsAbsolute}\label{wxfilenameisabsolute}

\func{bool}{IsAbsolute}{\param{wxPathFormat }{format = wxPATH\_NATIVE}}

Returns {\tt true} if this filename is absolute.


\membersection{wxFileName::IsCaseSensitive}\label{wxfilenameiscasesensitive}

\func{static bool}{IsCaseSensitive}{\param{wxPathFormat }{format = wxPATH\_NATIVE}}

Returns {\tt true} if the file names of this type are case-sensitive.


\membersection{wxFileName::IsOk}\label{wxfilenameisok}

\constfunc{bool}{IsOk}{\void}

Returns {\tt true} if the filename is valid, {\tt false} if it is not
initialized yet. The assignment functions and
\helpref{Clear}{wxfilenameclear} may reset the object to the uninitialized,
invalid state (the former only do it on failure).


\membersection{wxFileName::IsPathSeparator}\label{wxfilenameispathseparator}

\func{static bool}{IsPathSeparator}{\param{wxChar }{ch}, \param{wxPathFormat }{format = wxPATH\_NATIVE}}

Returns {\tt true} if the char is a path separator for this format.


\membersection{wxFileName::IsRelative}\label{wxfilenameisrelative}

\func{bool}{IsRelative}{\param{wxPathFormat }{format = wxPATH\_NATIVE}}

Returns {\tt true} if this filename is not absolute.


\membersection{wxFileName::IsDir}\label{wxfilenameisdir}

\constfunc{bool}{IsDir}{\void}

Returns {\tt true} if this object represents a directory, {\tt false} otherwise
(i.e. if it is a file). Note that this method doesn't test whether the
directory or file really exists, you should use 
\helpref{DirExists}{wxfilenamedirexists} or 
\helpref{FileExists}{wxfilenamefileexists} for this.

\membersection{wxFileName::MacFindDefaultTypeAndCreator}\label{wxfilenamemacfinddefaulttypeandcreator}

\func{static bool}{MacFindDefaultTypeAndCreator}{\param{const wxString\& }{ext}, \param{wxUint32* }{type}, \param{wxUint32* }{creator}}

On Mac OS, gets the common type and creator for the given extension.

\membersection{wxFileName::MacRegisterDefaultTypeAndCreator}\label{wxfilenamemacregisterdefaulttypeandcreator}

\func{static void}{MacRegisterDefaultTypeAndCreator}{\param{const wxString\& }{ext}, \param{wxUint32 }{type}, \param{wxUint32 }{creator}}

On Mac OS, registers application defined extensions and their default type and creator.

\membersection{wxFileName::MacSetDefaultTypeAndCreator}\label{wxfilenamemacsetdefaulttypeandcreator}

\func{bool}{MacSetDefaultTypeAndCreator}{\void}

On Mac OS, looks up the appropriate type and creator from the registration and then sets it.

\membersection{wxFileName::MakeAbsolute}\label{wxfilenamemakeabsolute}

\func{bool}{MakeAbsolute}{\param{const wxString\& }{cwd = wxEmptyString}, \param{wxPathFormat }{format = wxPATH\_NATIVE}}

Make the file name absolute. This is a shortcut for
{\tt \helpref{Normalize}{wxfilenamenormalize}(wxPATH\_NORM\_DOTS | wxPATH\_NORM\_ABSOLUTE | wxPATH\_NORM\_TILDE, cwd, format)}.

\wxheading{See also}

\helpref{MakeRelativeTo}{wxfilenamemakerelativeto},
\helpref{Normalize}{wxfilenamenormalize},
\helpref{IsAbsolute}{wxfilenameisabsolute}


\membersection{wxFileName::MakeRelativeTo}\label{wxfilenamemakerelativeto}

\func{bool}{MakeRelativeTo}{\param{const wxString\& }{pathBase = wxEmptyString}, \param{wxPathFormat }{format = wxPATH\_NATIVE}}

This function tries to put this file name in a form relative to {\it pathBase}.
In other words, it returns the file name which should be used to access this
file if the current directory were {\it pathBase}.

\docparam{pathBase}{the directory to use as root, current directory is used by
default}

\docparam{format}{the file name format, native by default}

\wxheading{Return value}

{\tt true} if the file name has been changed, {\tt false} if we failed to do
anything with it (currently this only happens if the file name is on a volume
different from the volume specified by {\it pathBase}).

\wxheading{See also}

\helpref{Normalize}{wxfilenamenormalize}


\membersection{wxFileName::Mkdir}\label{wxfilenamemkdir}

\func{bool}{Mkdir}{\param{int }{perm = 0777}, \param{int }{flags = $0$}}

\func{static bool}{Mkdir}{\param{const wxString\& }{dir}, \param{int }{perm = 0777}, \param{int }{flags = $0$}}

\docparam{dir}{the directory to create}

\docparam{parm}{the permissions for the newly created directory}

\docparam{flags}{if the flags contain {\tt wxPATH\_MKDIR\_FULL} flag,
try to create each directory in the path and also don't return an error
if the target directory already exists.}

\wxheading{Return value}

Returns {\tt true} if the directory was successfully created, {\tt false}
otherwise.


\membersection{wxFileName::Normalize}\label{wxfilenamenormalize}

\func{bool}{Normalize}{\param{int }{flags = wxPATH\_NORM\_ALL}, \param{const wxString\& }{cwd = wxEmptyString}, \param{wxPathFormat }{format = wxPATH\_NATIVE}}

Normalize the path. With the default flags value, the path will be
made absolute, without any ".." and "." and all environment
variables will be expanded in it.

\docparam{flags}{The kind of normalization to do with the file name. It can be
any or-combination of the following constants:

\begin{twocollist}
\twocolitem{{\bf wxPATH\_NORM\_ENV\_VARS}}{replace env vars with their values}
\twocolitem{{\bf wxPATH\_NORM\_DOTS}}{squeeze all .. and . and prepend cwd}
\twocolitem{{\bf wxPATH\_NORM\_TILDE}}{Unix only: replace ~ and ~user}
\twocolitem{{\bf wxPATH\_NORM\_CASE}}{if filesystem is case insensitive, transform to lower case}
\twocolitem{{\bf wxPATH\_NORM\_ABSOLUTE}}{make the path absolute}
\twocolitem{{\bf wxPATH\_NORM\_LONG}}{make the path the long form}
\twocolitem{{\bf wxPATH\_NORM\_SHORTCUT}}{resolve if it is a shortcut (Windows only)}
\twocolitem{{\bf wxPATH\_NORM\_ALL}}{all of previous flags except \texttt{wxPATH\_NORM\_CASE}}
\end{twocollist}
}%

\docparam{cwd}{If not empty, this directory will be used instead of current
working directory in normalization.}

\docparam{format}{The file name format, native by default.}


\membersection{wxFileName::PrependDir}\label{wxfilenameprependdir}

\func{void}{PrependDir}{\param{const wxString\& }{dir}}

Prepends a directory to the file path. Please see 
\helpref{AppendDir}{wxfilenameappenddir} for important notes.



\membersection{wxFileName::RemoveDir}\label{wxfilenameremovedir}

\func{void}{RemoveDir}{\param{size\_t }{pos}}

Removes the specified directory component from the path.

\wxheading{See also}

\helpref{GetDirCount}{wxfilenamegetdircount}


\membersection{wxFileName::RemoveLastDir}\label{wxfilenameremovelastdir}

\func{void}{RemoveLastDir}{\void}

Removes last directory component from the path.


\membersection{wxFileName::Rmdir}\label{wxfilenamermdir}

\func{bool}{Rmdir}{\void}

\func{static bool}{Rmdir}{\param{const wxString\& }{dir}}

Deletes the specified directory from the file system.


\membersection{wxFileName::SameAs}\label{wxfilenamesameas}

\constfunc{bool}{SameAs}{\param{const wxFileName\& }{filepath}, \param{wxPathFormat }{format = wxPATH\_NATIVE}}

Compares the filename using the rules of this platform.


\membersection{wxFileName::SetCwd}\label{wxfilenamesetcwd}

\func{bool}{SetCwd}{\void}

\func{static bool}{SetCwd}{\param{const wxString\& }{cwd}}

Changes the current working directory.


\membersection{wxFileName::SetExt}\label{wxfilenamesetext}

\func{void}{SetExt}{\param{const wxString\& }{ext}}

Sets the extension of the file name. Setting an empty string
as the extension will remove the extension resulting in a file 
name without a trailing dot, unlike a call to 
\helpref{SetEmptyExt}{wxfilenamesetemptyext}.

\wxheading{See also}

\helpref{SetEmptyExt}{wxfilenamesetemptyext}
\helpref{ClearExt}{wxfilenameclearext}

\membersection{wxFileName::SetEmptyExt}\label{wxfilenamesetemptyext}

\func{void}{SetEmptyExt}{\void}

Sets the extension of the file name to be an empty extension. 
This is different from having no extension at all as the file 
name will have a trailing dot after a call to this method.

\wxheading{See also}

\helpref{SetExt}{wxfilenamesetext}
\helpref{ClearExt}{wxfilenameclearext}

\membersection{wxFileName::SetFullName}\label{wxfilenamesetfullname}

\func{void}{SetFullName}{\param{const wxString\& }{fullname}}

The full name is the file name and extension (but without the path).


\membersection{wxFileName::SetName}\label{wxfilenamesetname}

\func{void}{SetName}{\param{const wxString\& }{name}}

Sets the name.


\membersection{wxFileName::SetTimes}\label{wxfilenamesettimes}

\func{bool}{SetTimes}{\param{const wxDateTime* }{dtAccess}, \param{const wxDateTime* }{dtMod}, \param{const wxDateTime* }{dtCreate}}

Sets the file creation and last access/modification times (any of the pointers may be NULL).


\membersection{wxFileName::SetVolume}\label{wxfilenamesetvolume}

\func{void}{SetVolume}{\param{const wxString\& }{volume}}

Sets the volume specifier.


\membersection{wxFileName::SplitPath}\label{wxfilenamesplitpath}

\func{static void}{SplitPath}{\param{const wxString\& }{fullpath}, \param{wxString* }{volume}, \param{wxString* }{path}, \param{wxString* }{name}, \param{wxString* }{ext}, \param{bool }{*hasExt = \texttt{NULL}}, \param{wxPathFormat }{format = wxPATH\_NATIVE}}

\func{static void}{SplitPath}{\param{const wxString\& }{fullpath}, \param{wxString* }{volume}, \param{wxString* }{path}, \param{wxString* }{name}, \param{wxString* }{ext}, \param{wxPathFormat }{format = wxPATH\_NATIVE}}

\func{static void}{SplitPath}{\param{const wxString\& }{fullpath}, \param{wxString* }{path}, \param{wxString* }{name}, \param{wxString* }{ext}, \param{wxPathFormat }{format = wxPATH\_NATIVE}}

This function splits a full file name into components: the volume (with the
first version) path (including the volume in the second version), the base name
and the extension. Any of the output parameters ({\it volume}, {\it path}, 
{\it name} or {\it ext}) may be {\tt NULL} if you are not interested in the
value of a particular component. Also, {\it fullpath} may be empty on entry.

On return, {\it path} contains the file path (without the trailing separator), 
{\it name} contains the file name and {\it ext} contains the file extension
without leading dot. All three of them may be empty if the corresponding
component is. The old contents of the strings pointed to by these parameters
will be overwritten in any case (if the pointers are not {\tt NULL}).

Note that for a filename ``foo.'' the extension is present, as indicated by the
trailing dot, but empty. If you need to cope with such cases, you should use 
\arg{hasExt} instead of relying on testing whether \arg{ext} is empty or not.


\membersection{wxFileName::SplitVolume}\label{wxfilenamesplitvolume}

\func{static void}{SplitVolume}{\param{const wxString\& }{fullpath}, \param{wxString* }{volume}, \param{wxString* }{path}, \param{wxPathFormat }{format = wxPATH\_NATIVE}}

Splits the given \arg{fullpath} into the volume part (which may be empty) and
the pure path part, not containing any volume.

\wxheading{See also}

\helpref{SplitPath}{wxfilenamesplitpath}


\membersection{wxFileName::Touch}\label{wxfilenametouch}

\func{bool}{Touch}{\void}

Sets the access and modification times to the current moment.


\membersection{wxFileName::operator=}\label{wxfilenameoperatorassign}

\func{wxFileName\& operator}{operator=}{\param{const wxFileName\& }{filename}}

\func{wxFileName\& operator}{operator=}{\param{const wxString\& }{filename}}

Assigns the new value to this filename object.


\membersection{wxFileName::operator==}\label{wxfilenameoperatorequal}

\constfunc{bool operator}{operator==}{\param{const wxFileName\& }{filename}}

\constfunc{bool operator}{operator==}{\param{const wxString\& }{filename}}

Returns {\tt true} if the filenames are equal. The string {\it filenames} is
interpreted as a path in the native filename format.


\membersection{wxFileName::operator!=}\label{wxfilenameoperatornotequal}

\constfunc{bool operator}{operator!=}{\param{const wxFileName\& }{filename}}

\constfunc{bool operator}{operator!=}{\param{const wxString\& }{filename}}

Returns {\tt true} if the filenames are different. The string {\it filenames}
is interpreted as a path in the native filename format.

