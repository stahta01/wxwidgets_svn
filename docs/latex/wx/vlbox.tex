%%%%%%%%%%%%%%%%%%%%%%%%%%%%%%%%%%%%%%%%%%%%%%%%%%%%%%%%%%%%%%%%%%%%%%%%%%%%%%%
%% Name:        vlbox.tex
%% Purpose:     wxVListBox documentation
%% Author:      Vadim Zeitlin
%% Modified by:
%% Created:     01.06.03
%% RCS-ID:      $Id$
%% Copyright:   (c) 2003 Vadim Zeitlin <vadim@wxwindows.org>
%% License:     wxWindows license
%%%%%%%%%%%%%%%%%%%%%%%%%%%%%%%%%%%%%%%%%%%%%%%%%%%%%%%%%%%%%%%%%%%%%%%%%%%%%%%

\section{\class{wxVListBox}}\label{wxvlistbox}

wxVListBox is a listbox-like control with the following two main differences
from a regular listbox: it can have an arbitrarily huge number of items because
it doesn't store them itself but uses \helpref{OnDrawItem()}{wxvlistboxondrawitem}
callback to draw them (so it is a {\Large V}irtual listbox) and its items can
have variable height as determined by
\helpref{OnMeasureItem()}{wxvlistboxonmeasureitem} (so it is also a listbox
with the lines of {\Large V}ariable height).

Also, as a consequence of its virtual nature, it doesn't have any methods to
append or insert items in it as it isn't necessary to do it: you just have to
call \helpref{SetItemCount()}{wxvlistboxsetitemcount} to tell the control how
many items it should display. Of course, this also means that you will never
use this class directly because it has pure virtual functions, but will need to
derive your own class, such as \helpref{wxHtmlListBox}{wxhtmllistbox}, from it.

However it emits the same events as \helpref{wxListBox}{wxlistbox} and the same
event macros may be used with it.

\wxheading{Derived from}

\helpref{wxVScrolledWindow}{wxvscrolledwindow}\\
\helpref{wxPanel}{wxpanel}\\
\helpref{wxWindow}{wxwindow}\\
\helpref{wxEvtHandler}{wxevthandler}\\
\helpref{wxObject}{wxobject}

\wxheading{Include files}

<wx/vlbox.h>

\wxheading{See also}

\helpref{wxSimpleHtmlListBox}{wxsimplehtmllistbox}, \helpref{wxHtmlListBox}{wxhtmllistbox}

\latexignore{\rtfignore{\wxheading{Members}}}


\membersection{wxVListBox::wxVListBox}\label{wxvlistboxctor}

\func{}{wxVListBox}{\param{wxWindow* }{parent}, \param{wxWindowID }{id = wxID\_ANY}, \param{const wxPoint\& }{pos = wxDefaultPosition}, \param{const wxSize\& }{size = wxDefaultSize}, \param{size\_t }{countItems = 0}, \param{long }{style = 0}, \param{const wxString\& }{name = wxVListBoxNameStr}}

Normal constructor which calls \helpref{Create()}{wxvlistboxcreate} internally.

\func{}{wxVListBox}{\void}

Default constructor, you must call \helpref{Create()}{wxvlistboxcreate} later.


\membersection{wxVListBox::Clear}\label{wxvlistboxclear}

\func{void}{Clear}{\void}

Deletes all items from the control.


\membersection{wxVListBox::Create}\label{wxvlistboxcreate}

\func{bool}{Create}{\param{wxWindow* }{parent}, \param{wxWindowID }{id = wxID\_ANY}, \param{const wxPoint\& }{pos = wxDefaultPosition}, \param{const wxSize\& }{size = wxDefaultSize}, \param{long }{style = 0}, \param{const wxString\& }{name = wxVListBoxNameStr}}

Creates the control. To finish creating it you also should call
\helpref{SetItemCount()}{wxvlistboxsetitemcount} to let it know about the
number of items it contains.

The only special style which may be used with wxVListBox is {\tt wxLB\_MULTIPLE}
which indicates that the listbox should support multiple selection.

Returns {\tt true} on success or {\tt false} if the control couldn't be created


\membersection{wxVListBox::DeselectAll}\label{wxvlistboxdeselectall}

\func{bool}{DeselectAll}{\void}

Deselects all the items in the listbox.

Returns {\tt true} if any items were changed, i.e. if there had been any
selected items before, or {\tt false} if all the items were already deselected.

This method is only valid for multi selection listboxes.

\wxheading{See also}

\helpref{SelectAll}{wxvlistboxselectall}, \helpref{Select}{wxvlistboxselect}


\membersection{wxVListBox::GetFirstSelected}\label{wxvlistboxgetfirstselected}

\constfunc{int}{GetFirstSelected}{\param{unsigned long\& }{cookie}}

Returns the index of the first selected item in the listbox or
{\tt wxNOT\_FOUND} if no items are currently selected.

\arg{cookie} is an opaque parameter which should be passed to the subsequent
calls to \helpref{GetNextSelected}{wxvlistboxgetnextselected}. It is needed in
order to allow parallel iterations over the selected items.

Here is a typical example of using these functions:
\begin{verbatim}
unsigned long cookie;
int item = hlbox->GetFirstSelected(cookie);
while ( item != wxNOT_FOUND )
{
    ... process item ...
    item = hlbox->GetNextSelected(cookie);
}
\end{verbatim}

This method is only valid for multi selection listboxes.


\membersection{wxVListBox::GetItemCount}\label{wxvlistboxgetitemcount}

\constfunc{size\_t}{GetItemCount}{\void}

Get the number of items in the control.

\wxheading{See also}

\helpref{SetItemCount()}{wxvlistboxsetitemcount}


\membersection{wxVListBox::GetMargins}\label{wxvlistboxgetmargins}

\constfunc{wxPoint}{GetMargins}{\void}

Returns the margins used by the control. The {\tt x} field of the returned
point is the horizontal margin and the {\tt y} field is the vertical one.

\wxheading{See also}

\helpref{SetMargins}{wxvlistboxsetmargins}


\membersection{wxVListBox::GetNextSelected}\label{wxvlistboxgetnextselected}

\constfunc{int}{GetNextSelected}{\param{unsigned long\& }{cookie}}

Returns the index of the next selected item or {\tt wxNOT\_FOUND} if there are
no more.

This method is only valid for multi selection listboxes.

\wxheading{See also}

\helpref{GetFirstSelected}{wxvlistboxgetfirstselected}


\membersection{wxVListBox::GetSelectedCount}\label{wxvlistboxgetselectedcount}

\constfunc{size\_t}{GetSelectedCount}{\void}

Returns the number of the items currently selected.

It is valid for both single and multi selection controls. In the former case it
may only return $0$ or $1$ however.

\wxheading{See also}

\helpref{IsSelected}{wxvlistboxisselected},\\
\helpref{GetFirstSelected}{wxvlistboxgetfirstselected},\\
\helpref{GetNextSelected}{wxvlistboxgetnextselected}


\membersection{wxVListBox::GetSelection}\label{wxvlistboxgetselection}

\constfunc{int}{GetSelection}{\void}

Get the currently selected item or {\tt wxNOT\_FOUND} if there is no selection.


\membersection{wxVListBox::GetSelectionBackground}\label{wxvlistboxgetselectionbackground}

\constfunc{const wxColour\&}{GetSelectionBackground}{\void}

Returns the background colour used for the selected cells. By default the
standard system colour is used.

\wxheading{See also}

\helpref{wxSystemSettings::GetColour}{wxsystemsettingsgetcolour},\\
\helpref{SetSelectionBackground}{wxvlistboxsetselectionbackground}


\membersection{wxVListBox::HasMultipleSelection}\label{wxvlistboxishasmultipleselection}

\constfunc{bool}{HasMultipleSelection}{\void}

Returns {\tt true} if the listbox was created with {\tt wxLB\_MULTIPLE} style
and so supports multiple selection or {\tt false} if it is a single selection
listbox.


\membersection{wxVListBox::IsCurrent}\label{wxvlistboxiscurrent}

\constfunc{bool}{IsCurrent}{\param{size\_t }{item}}

Returns {\tt true} if this item is the current one, {\tt false} otherwise.

Current item is always the same as selected one for the single selection
listbox and in this case this method is equivalent to
\helpref{IsSelected}{wxvlistboxisselected} but they are different for multi
selection listboxes where many items may be selected but only one (at most) is
current.


\membersection{wxVListBox::IsSelected}\label{wxvlistboxisselected}

\constfunc{bool}{IsSelected}{\param{size\_t }{item}}

Returns {\tt true} if this item is selected, {\tt false} otherwise.


\membersection{wxVListBox::OnDrawBackground}\label{wxvlistboxondrawbackground}

\constfunc{void}{OnDrawBackground}{\param{wxDC\& }{dc}, \param{const wxRect\& }{rect}, \param{size\_t }{n}}

This method is used to draw the items background and, maybe, a border
around it.

The base class version implements a reasonable default behaviour which
consists in drawing the selected item with the standard background
colour and drawing a border around the item if it is either selected or
current.


\membersection{wxVListBox::OnDrawItem}\label{wxvlistboxondrawitem}

\constfunc{void}{OnDrawItem}{\param{wxDC\& }{dc}, \param{const wxRect\& }{rect}, \param{size\_t }{n}}

The derived class must implement this function to actually draw the item
with the given index on the provided DC.

\wxheading{Parameters}

\docparam{dc}{The device context to use for drawing}

\docparam{rect}{The bounding rectangle for the item being drawn (DC clipping
region is set to this rectangle before calling this function)}

\docparam{n}{The index of the item to be drawn}


\membersection{wxVListBox::OnDrawSeparator}\label{wxvlistboxondrawseparator}

\constfunc{void}{OnDrawSeparator}{\param{wxDC\& }{dc}, \param{wxRect\& }{rect}, \param{size\_t }{n}}

This method may be used to draw separators between the lines. The rectangle
passed to it may be modified, typically to deflate it a bit before passing to
\helpref{OnDrawItem()}{wxvlistboxondrawitem}.

The base class version of this method doesn't do anything.

\wxheading{Parameters}

\docparam{dc}{The device context to use for drawing}

\docparam{rect}{The bounding rectangle for the item}

\docparam{n}{The index of the item}


\membersection{wxVListBox::OnMeasureItem}\label{wxvlistboxonmeasureitem}

\constfunc{wxCoord}{OnMeasureItem}{\param{size\_t }{n}}

The derived class must implement this method to return the height of the
specified item (in pixels).


\membersection{wxVListBox::Select}\label{wxvlistboxselect}

\func{bool}{Select}{\param{size\_t }{item}, \param{bool }{select = true}}

Selects or deselects the specified item which must be valid (i.e. not
equal to {\tt wxNOT\_FOUND}).

Return {\tt true} if the items selection status has changed or {\tt false}
otherwise.

This function is only valid for the multiple selection listboxes, use
\helpref{SetSelection}{wxvlistboxsetselection} for the single selection ones.


\membersection{wxVListBox::SelectAll}\label{wxvlistboxselectall}

\func{bool}{SelectAll}{\void}

Selects all the items in the listbox.

Returns {\tt true} if any items were changed, i.e. if there had been any
unselected items before, or {\tt false} if all the items were already selected.

This method is only valid for multi selection listboxes.

\wxheading{See also}

\helpref{DeselectAll}{wxvlistboxdeselectall}, \helpref{Select}{wxvlistboxselect}


\membersection{wxVListBox::SelectRange}\label{wxvlistboxselectrange}

\func{bool}{SelectRange}{\param{size\_t }{from}, \param{size\_t }{to}}

Selects all items in the specified range which may be given in any order.

Return {\tt true} if the items selection status has changed or {\tt false}
otherwise.

This method is only valid for multi selection listboxes.

\wxheading{See also}

\helpref{SelectAll}{wxvlistboxselectall}, \helpref{Select}{wxvlistboxselect}

\membersection{wxVListBox::SetItemCount}\label{wxvlistboxsetitemcount}

\func{void}{SetItemCount}{\param{size\_t }{count}}

Set the number of items to be shown in the control.

This is just a synonym for
\helpref{wxVScrolledWindow::SetLineCount()}{wxvscrolledwindowsetlinecount}.


\membersection{wxVListBox::SetMargins}\label{wxvlistboxsetmargins}

\func{void}{SetMargins}{\param{const wxPoint\& }{pt}}

\func{void}{SetMargins}{\param{wxCoord }{x}, \param{wxCoord }{y}}

Set the margins: horizontal margin is the distance between the window
border and the item contents while vertical margin is half of the
distance between items.

By default both margins are $0$.


\membersection{wxVListBox::SetSelection}\label{wxvlistboxsetselection}

\func{void}{SetSelection}{\param{int }{selection}}

Set the selection to the specified item, if it is $-1$ the selection is
unset. The selected item will be automatically scrolled into view if it isn't
currently visible.

This method may be used both with single and multiple selection listboxes.


\membersection{wxVListBox::SetSelectionBackground}\label{wxvlistboxsetselectionbackground}

\func{void}{SetSelectionBackground}{\param{const wxColour\& }{col}}

Sets the colour to be used for the selected cells background. The background of
the standard cells may be changed by simply calling
\helpref{SetBackgroundColour}{wxwindowsetbackgroundcolour}.

\wxheading{See also}

\helpref{GetSelectionBackground}{wxvlistboxgetselectionbackground}


\membersection{wxVListBox::Toggle}\label{wxvlistboxtoggle}

\func{void}{Toggle}{\param{size\_t }{item}}

Toggles the state of the specified \arg{item}, i.e. selects it if it was
unselected and deselects it if it was selected.

This method is only valid for multi selection listboxes.

\wxheading{See also}

\helpref{Select}{wxvlistboxselect}

