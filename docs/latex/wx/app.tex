\section{\class{wxApp}}\label{wxapp}

The {\bf wxApp} class represents the application itself. It is used
to:

\begin{itemize}\itemsep=0pt
\item set and get application-wide properties;
\item implement the windowing system message or event loop;
\item initiate application processing via \helpref{wxApp::OnInit}{wxapponinit};
\item allow default processing of events not handled by other
objects in the application.
\end{itemize}

You should use the macro IMPLEMENT\_APP(appClass) in your application implementation
file to tell wxWindows how to create an instance of your application class.

Use DECLARE\_APP(appClass) in a header file if you want the wxGetApp function (which returns
a reference to your application object) to be visible to other files.

\wxheading{Derived from}

\helpref{wxEvtHandler}{wxevthandler}\\
\helpref{wxObject}{wxobject}

\wxheading{Include files}

<wx/app.h>

\wxheading{See also}

\helpref{wxApp overview}{wxappoverview}

\latexignore{\rtfignore{\wxheading{Members}}}

\membersection{wxApp::wxApp}

\func{void}{wxApp}{\void}

Constructor. Called implicitly with a definition of a wxApp object.

% VZ: there is no such feature so far...
% The argument is a language identifier; this is an experimental
% feature and will be expanded and documented in future versions.

\membersection{wxApp::\destruct{wxApp}}

\func{void}{\destruct{wxApp}}{\void}

Destructor. Will be called implicitly on program exit if the wxApp
object is created on the stack.

\membersection{wxApp::argc}\label{wxappargc}

\member{int}{argc}

Number of command line arguments (after environment-specific processing).

\membersection{wxApp::argv}\label{wxappargv}

\member{char **}{argv}

Command line arguments (after environment-specific processing).

\membersection{wxApp::CreateLogTarget}\label{wxappcreatelogtarget}

\func{virtual wxLog*}{CreateLogTarget}{\void}

Creates a wxLog class for the application to use for logging errors. The default
implementation returns a new wxLogGui class.

\wxheading{See also}

\helpref{wxLog}{wxlog}

\membersection{wxApp::Dispatch}\label{wxappdispatch}

\func{void}{Dispatch}{\void}

Dispatches the next event in the windowing system event queue.

This can be used for programming event loops, e.g.

\begin{verbatim}
  while (app.Pending())
    Dispatch();
\end{verbatim}

\wxheading{See also}

\helpref{wxApp::Pending}{wxapppending}

\membersection{wxApp::GetAppName}\label{wxappgetappname}

\constfunc{wxString}{GetAppName}{\void}

Returns the application name.

\wxheading{Remarks}

wxWindows sets this to a reasonable default before
calling \helpref{wxApp::OnInit}{wxapponinit}, but the application can reset it at will.

\membersection{wxApp::GetAuto3D}\label{wxappgetauto3d}

\constfunc{bool}{GetAuto3D}{\void}

Returns TRUE if 3D control mode is on, FALSE otherwise.

\wxheading{See also}

\helpref{wxApp::SetAuto3D}{wxappsetauto3d}

\membersection{wxApp::GetClassName}\label{wxappgetclassname}

\constfunc{wxString}{GetClassName}{\void}

Gets the class name of the application. The class name may be used in a platform specific
manner to refer to the application.

\wxheading{See also}

\helpref{wxApp::SetClassName}{wxappsetclassname}

\membersection{wxApp::GetExitOnFrameDelete}\label{wxappgetexitonframedelete}

\constfunc{bool}{GetExitFrameOnDelete}{\void}

Returns TRUE if the application will exit when the top-level window is deleted, FALSE
otherwise.

\wxheading{See also}

\helpref{wxApp::SetExitOnFrameDelete}{wxappsetexitonframedelete}

\membersection{wxApp::GetTopWindow}\label{wxappgettopwindow}

\constfunc{wxWindow *}{GetTopWindow}{\void}

Returns a pointer to the top window.

\wxheading{Remarks}

If the top window hasn't been set using \helpref{wxApp::SetTopWindow}{wxappsettopwindow}, this
function will find the first top-level window (frame or dialog) and return that.

\wxheading{See also}

\helpref{SetTopWindow}{wxappsettopwindow}

\membersection{wxApp::GetUseBestVisual}\label{wxappgetusebestvisual}

\constfunc{bool}{GetUseBestVisual}{\void}

Returns TRUE if the application will use the best visual on systems that support
different visuals, FALSE otherwise.

\wxheading{See also}

\helpref{SetUseBestVisual}{wxappsetusebestvisual}

\membersection{wxApp::GetVendorName}\label{wxappgetvendorname}

\constfunc{wxString}{GetVendorName}{\void}

Returns the application's vendor name.

\membersection{wxApp::ExitMainLoop}\label{wxappexitmainloop}

\func{void}{ExitMainLoop}{\void}

Call this to explicitly exit the main message (event) loop.
You should normally exit the main loop (and the application) by deleting
the top window.

\membersection{wxApp::Initialized}\label{wxappinitialized}

\func{bool}{Initialized}{\void}

Returns TRUE if the application has been initialized (i.e. if\rtfsp
\helpref{wxApp::OnInit}{wxapponinit} has returned successfully).  This can be useful for error
message routines to determine which method of output is best for the
current state of the program (some windowing systems may not like
dialogs to pop up before the main loop has been entered).

\membersection{wxApp::MainLoop}\label{wxappmainloop}

\func{int}{MainLoop}{\void}

Called by wxWindows on creation of the application. Override this if you wish
to provide your own (environment-dependent) main loop.

\wxheading{Return value}

Returns 0 under X, and the wParam of the WM\_QUIT message under Windows.

\membersection{wxApp::OnActivate}\label{wxapponactivate}

\func{void}{OnActivate}{\param{wxActivateEvent\& }{event}}

Provide this member function to know whether the application is being
activated or deactivated (Windows only).

\wxheading{See also}

\helpref{wxWindow::OnActivate}{wxwindowonactivate}, \helpref{wxActivateEvent}{wxactivateevent}

\membersection{wxApp::OnExit}\label{wxapponexit}

\func{int}{OnExit}{\void}

Provide this member function for any processing which needs to be done as
the application is about to exit.

\membersection{wxApp::OnCharHook}\label{wxapponcharhook}

\func{void}{OnCharHook}{\param{wxKeyEvent\&}{ event}}

This event handler function is called (under Windows only) to allow the window to intercept keyboard events
before they are processed by child windows.

\wxheading{Parameters}

\docparam{event}{The keypress event.}

\wxheading{Remarks}

Use the wxEVT\_CHAR\_HOOK macro in your event table.

If you use this member, you can selectively consume keypress events by calling\rtfsp
\helpref{wxEvent::Skip}{wxeventskip} for characters the application is not interested in.

\wxheading{See also}

\helpref{wxKeyEvent}{wxkeyevent}, \helpref{wxWindow::OnChar}{wxwindowonchar},\rtfsp
\helpref{wxWindow::OnCharHook}{wxwindowoncharhook}, \helpref{wxDialog::OnCharHook}{wxdialogoncharhook}

\membersection{wxApp::OnFatalException}\label{wxapponfatalexception}

\func{void}{OnFatalException}{\void}

This function may be called if something fatal happens: an unhandled
exception under Win32 or a a fatal signal under Unix, for example. However,
this will not happen by default: you have to explicitly call 
\helpref{wxHandleFatalExceptions}{wxhandlefatalexceptions} to enable this.

Generally speaking, this function should only show a message to the user and
return. You may attempt to save unsaved data but this is not guaranteed to
work and, in fact, probably won't.

\wxheading{See also}

\helpref{wxHandleFatalExcetions}{wxhandlefatalexceptions}

\membersection{wxApp::OnIdle}\label{wxapponidle}

\func{void}{OnIdle}{\param{wxIdleEvent\& }{event}}

Override this member function for any processing which needs to be done
when the application is idle. You should call wxApp::OnIdle from your own function,
since this forwards OnIdle events to windows and also performs garbage collection for
windows whose destruction has been delayed.

wxWindows' strategy for OnIdle processing is as follows. After pending user interface events for an
application have all been processed, wxWindows sends an OnIdle event to the application object. wxApp::OnIdle itself
sends an OnIdle event to each application window, allowing windows to do idle processing such as updating
their appearance. If either wxApp::OnIdle or a window OnIdle function requested more time, by
caling \helpref{wxIdleEvent::RequestMore}{wxidleeventrequestmore}, wxWindows will send another OnIdle
event to the application object. This will occur in a loop until either a user event is found to be
pending, or OnIdle requests no more time. Then all pending user events are processed until the system
goes idle again, when OnIdle is called, and so on.

\wxheading{See also}

\helpref{wxWindow::OnIdle}{wxwindowonidle}, \helpref{wxIdleEvent}{wxidleevent},\rtfsp
\helpref{wxWindow::SendIdleEvents}{wxappsendidleevents}

\membersection{wxApp::OnEndSession}\label{wxapponendsession}

\func{void}{OnEndSession}{\param{wxCloseEvent\& }{event}}

This is an event handler function called when the operating system or GUI session is
about to close down. The application has a chance to silently save information,
and can optionally close itself.

Use the EVT\_END\_SESSION event table macro to handle query end session events.

The default handler calls \helpref{wxWindow::Close}{wxwindowclose} with a TRUE argument
(forcing the application to close itself silently).

\wxheading{Remarks}

Under X, OnEndSession is called in response to the `die' event.

Under Windows, OnEndSession is called in response to the WM\_ENDSESSION message.

\wxheading{See also}

\helpref{wxWindow::Close}{wxwindowclose},\rtfsp
\helpref{wxWindow::OnCloseWindow}{wxwindowonclosewindow},\rtfsp
\helpref{wxCloseEvent}{wxcloseevent},\rtfsp
\helpref{wxApp::OnQueryEndSession}{wxapponqueryendsession}

\membersection{wxApp::OnInit}\label{wxapponinit}

\func{bool}{OnInit}{\void}

This must be provided by the application, and will usually create the
application's main window, optionally calling \helpref{wxApp::SetTopWindow}{wxappsettopwindow}.

Return TRUE to continue processing, FALSE to exit the application.

\membersection{wxApp::OnQueryEndSession}\label{wxapponqueryendsession}

\func{void}{OnQueryEndSession}{\param{wxCloseEvent\& }{event}}

This is an event handler function called when the operating system or GUI session is
about to close down. Typically, an application will try to save unsaved documents
at this point.

If \helpref{wxCloseEvent::CanVeto}{wxcloseeventcanveto} returns TRUE, the application
is allowed to veto the shutdown by calling \helpref{wxCloseEvent::Veto}{wxcloseeventveto}.
The application might veto the shutdown after prompting for documents to be saved, and the
user has cancelled the save.

Use the EVT\_QUERY\_END\_SESSION event table macro to handle query end session events.

You should check whether the application is forcing the deletion of the window
using \helpref{wxCloseEvent::GetForce}{wxcloseeventgetforce}. If this is TRUE,
destroy the window using \helpref{wxWindow::Destroy}{wxwindowdestroy}.
If not, it is up to you whether you respond by destroying the window.

The default handler calls \helpref{wxWindow::Close}{wxwindowclose} on the top-level window,
and vetoes the shutdown if Close returns FALSE. This will be sufficient for many applications.

\wxheading{Remarks}

Under X, OnQueryEndSession is called in response to the `save session' event.

Under Windows, OnQueryEndSession is called in response to the WM\_QUERYENDSESSION message.

\wxheading{See also}

\helpref{wxWindow::Close}{wxwindowclose},\rtfsp
\helpref{wxWindow::OnCloseWindow}{wxwindowonclosewindow},\rtfsp
\helpref{wxCloseEvent}{wxcloseevent},\rtfsp
\helpref{wxApp::OnEndSession}{wxapponendsession}

\membersection{wxApp::ProcessMessage}\label{wxappprocessmessage}

\func{bool}{ProcessMessage}{\param{MSG *}{msg}}

Windows-only function for processing a message. This function
is called from the main message loop, checking for windows that
may wish to process it. The function returns TRUE if the message
was processed, FALSE otherwise. If you use wxWindows with another class
library with its own message loop, you should make sure that this
function is called to allow wxWindows to receive messages. For example,
to allow co-existance with the Microsoft Foundation Classes, override
the PreTranslateMessage function:

\begin{verbatim}
// Provide wxWindows message loop compatibility
BOOL CTheApp::PreTranslateMessage(MSG *msg)
{
  if (wxTheApp && wxTheApp->ProcessMessage(msg))
    return TRUE;
  else
    return CWinApp::PreTranslateMessage(msg);
}
\end{verbatim}

\membersection{wxApp::Pending}\label{wxapppending}

\func{bool}{Pending}{\void}

Returns TRUE if unprocessed events are in the window system event queue.

\wxheading{See also}

\helpref{wxApp::Dispatch}{wxappdispatch}

\membersection{wxApp::SendIdleEvents}\label{wxappsendidleevents}

\func{bool}{SendIdleEvents}{\void}

Sends idle events to all top-level windows.

\func{bool}{SendIdleEvents}{\param{wxWindow*}{ win}}

Sends idle events to a window and its children.

\wxheading{Remarks}

These functions poll the top-level windows, and their children, for idle event processing.
If TRUE is returned, more OnIdle processing is requested by one or more window.

\wxheading{See also}

\helpref{wxApp::OnIdle}{wxapponidle}, \helpref{wxWindow::OnIdle}{wxwindowonidle}, \helpref{wxIdleEvent}{wxidleevent}

\membersection{wxApp::SetAppName}\label{wxappsetappname}

\func{void}{SetAppName}{\param{const wxString\& }{name}}

Sets the name of the application. The name may be used in dialogs
(for example by the document/view framework). A default name is set by
wxWindows.

\wxheading{See also}

\helpref{wxApp::GetAppName}{wxappgetappname}

\membersection{wxApp::SetAuto3D}\label{wxappsetauto3d}

\func{void}{SetAuto3D}{\param{const bool}{ auto3D}}

Switches automatic 3D controls on or off.

\wxheading{Parameters}

\docparam{auto3D}{If TRUE, all controls will be created with 3D appearances unless
overridden for a control or dialog. The default is TRUE}

\wxheading{Remarks}

This has an effect on Windows only.

\wxheading{See also}

\helpref{wxApp::GetAuto3D}{wxappgetauto3d}

\membersection{wxApp::SetClassName}\label{wxappsetclassname}

\func{void}{SetClassName}{\param{const wxString\& }{name}}

Sets the class name of the application. This may be used in a platform specific
manner to refer to the application.

\wxheading{See also}

\helpref{wxApp::GetClassName}{wxappgetclassname}

\membersection{wxApp::SetExitOnFrameDelete}\label{wxappsetexitonframedelete}

\func{void}{SetExitOnFrameDelete}{\param{bool}{ flag}}

Allows the programmer to specify whether the application will exit when the
top-level frame is deleted.

\wxheading{Parameters}

\docparam{flag}{If TRUE (the default), the application will exit when the top-level frame is
deleted. If FALSE, the application will continue to run.}

\membersection{wxApp::SetTopWindow}\label{wxappsettopwindow}

\func{void}{SetTopWindow}{\param{wxWindow* }{window}}

Sets the `top' window. You can call this from within \helpref{wxApp::OnInit}{wxapponinit} to
let wxWindows know which is the main window. You don't have to set the top window;
it is only a convenience so that (for example) certain dialogs without parents can use a
specific window as the top window. If no top window is specified by the application,
wxWindows just uses the first frame or dialog in its top-level window list, when it
needs to use the top window.

\wxheading{Parameters}

\docparam{window}{The new top window.}

\wxheading{See also}

\helpref{wxApp::GetTopWindow}{wxappgettopwindow}, \helpref{wxApp::OnInit}{wxapponinit}


\membersection{wxApp::SetVendorName}\label{wxappsetvendorname}

\func{void}{SetVendorName}{\param{const wxString\& }{name}}

Sets the name of application's vendor. The name will be used
in registry access. A default name is set by
wxWindows.

\wxheading{See also}

\helpref{wxApp::GetVendorName}{wxappgetvendorname}

\membersection{wxApp::GetStdIcon}\label{wxappgetstdicon}

\func{virtual wxIcon}{GetStdIcon}{\param{int }{which}} const

Returns the icons used by wxWindows internally, e.g. the ones used for 
message boxes.  This function is used internally and 
can be overridden by the user to change the default icons.

\wxheading{Parameters}

\docparam{which}{One of the wxICON\_XXX specifies which icon to return.}

See \helpref{wxMessageBox}{wxmessagebox} for a list of icon identifiers.

\membersection{wxApp::SetUseBestVisual}\label{wxappsetusebestvisual}

\func{void}{SetUseBestVisual}{\param{bool}{ flag}}

Allows the programmer to specify whether the application will use the best visual
on systems that support several visual on the same display. This is typically the
case under Solaris and IRIX, where the default visual is only 8-bit whereas certain
appications are supposed to run in TrueColour mode.

Note that this function has to be called in the constructor of the {\tt wxApp} 
instance and won't have any effect when called later on.

This function currently only has effect under GTK.

\wxheading{Parameters}

\docparam{flag}{If TRUE, the app will use the best visual.}

