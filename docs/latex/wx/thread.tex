\section{\class{wxThread}}\label{wxthread}

A thread is basically a path of execution through a program. Threads are
sometimes called {\it light-weight processes}, but the fundamental difference
between threads and processes is that memory spaces of different processes are
separated while all threads share the same address space. 

While it makes it much easier to share common data between several threads, it also 
makes it much easier to shoot oneself in the foot, so careful use of synchronization 
objects such as \helpref{mutexes}{wxmutex} or \helpref{critical sections}{wxcriticalsection} is recommended. In addition, don't create global thread 
objects because they allocate memory in their constructor, which will cause 
problems for the memory checking system.

\wxheading{Derived from}

None.

\wxheading{Include files}

<wx/thread.h>

\wxheading{See also}

\helpref{wxMutex}{wxmutex}, \helpref{wxCondition}{wxcondition}, \helpref{wxCriticalSection}{wxcriticalsection}

\latexignore{\rtfignore{\wxheading{Members}}}

\membersection{Types of wxThreads}\label{typeswxthread}

There are two types of threads in wxWidgets: {\it detached} and {\it joinable},
modeled after the the POSIX thread API. This is different from the Win32 API
where all threads are joinable. 

By default wxThreads in wxWidgets use the detached behavior. Detached threads
delete themselves once they have completed, either by themselves when they complete 
processing or through a call to \helpref{wxThread::Delete}{wxthreaddelete}, and thus 
must be created on the heap (through the new operator, for example). Conversely, 
joinable threads do not delete themselves when they are done processing and as such
are safe to create on the stack. Joinable threads also provide the ability
for one to get value it returned from \helpref{wxThread::Entry}{wxthreadentry}
through \helpref{wxThread::Wait}{wxthreadwait}.

You shouldn't hurry to create all the threads joinable, however, because this
has a disadvantage as well: you {\bf must} Wait() for a joinable thread or the
system resources used by it will never be freed, and you also must delete the
corresponding wxThread object yourself if you did not create it on the stack. In 
contrast, detached threads are of the "fire-and-forget" kind: you only have to start 
a detached thread and it will terminate and destroy itself.

\membersection{wxThread deletion}\label{deletionwxthread}

Regardless of whether it has terminated or not, you should call 
\helpref{wxThread::Wait}{wxthreadwait} on a joinable thread to release its
memory, as outlined in \helpref{Types of wxThreads}{typeswxthread}. If you created
a joinable thread on the heap, remember to delete it manually with the delete 
operator or similar means as only detached threads handle this type of memory 
management.

Since detached threads delete themselves when they are finished processing,
you should take care when calling a routine on one. If you are certain the 
thread is still running and would like to end it, you may call 
\helpref{wxThread::Delete}{wxthreaddelete} to gracefully end it (which implies
that the thread will be deleted after that call to Delete()). It should be
implied that you should never attempt to delete a detached thread with the 
delete operator or similar means. 

As mentioned, \helpref{wxThread::Wait}{wxthreadwait} or 
\helpref{wxThread::Delete}{wxthreaddelete} attempts to gracefully terminate
a joinable and detached thread, respectively. It does this by waiting until
the thread in question calls \helpref{wxThread::TestDestroy}{wxthreadtestdestroy}
or ends processing (returns from \helpref{wxThread::Entry}{wxthreadentry}).

Obviously, if the thread does call TestDestroy() and does not end the calling
thread will come to halt. This is why it is important to call TestDestroy() in
the Entry() routine of your threads as often as possible.

As a last resort you can end the thread immediately through 
\helpref{wxThread::Kill}{wxthreadkill}. It is strongly recommended that you
do not do this, however, as it does not free the resources associated with
the object (although the wxThread object of detached threads will still be
deleted) and could leave the C runtime library in an undefined state.

\membersection{wxWidgets calls in secondary threads}\label{secondarywxthread}

All threads other then the "main application thread" (the one
\helpref{wxApp::OnInit}{wxapponinit} or your main function runs in, for 
example) are considered "secondary threads". These include all threads created 
by \helpref{wxThread::Create}{wxthreadcreate} or the corresponding constructors.

GUI calls, such as those to a \helpref{wxWindow}{wxwindow} or 
\helpref{wxBitmap}{wxbitmap} are explicitly not safe at all in secondary threads 
and could end your application prematurely. This is due to several reasons,
including the underlying native API and the fact that wxThread does not run a 
GUI event loop similar to other APIs as MFC. 

A workaround that works on some wxWidgets ports is calling \helpref{wxMutexGUIEnter}{wxmutexguienter} 
before any GUI calls and then calling \helpref{wxMutexGUILeave}{wxmutexguileave} afterwords. However,
the recommended way is to simply process the GUI calls in the main thread 
through an event that is posted by either \helpref{wxPostEvent}{wxpostevent} or
\helpref{wxEvtHandler::AddPendingEvent}{wxevthandleraddpendingevent}. This does 
not imply that calls to these classes are thread-safe, however, as most 
wxWidgets classes are not thread-safe, including wxString.

\membersection{Don't poll a wxThread}\label{dontpollwxthread}

A common problem users experience with wxThread is that in their main thread
they will check the thread every now and then to see if it has ended through
\helpref{wxThread::IsRunning}{wxthreadisrunning}, only to find that their 
application has run into problems because the thread is using the default
behavior and has already deleted itself. Naturally, they instead attempt to
use joinable threads in place of the previous behavior.

However, polling a wxThread for when it has ended is in general a bad idea -
in fact calling a routine on any running wxThread should be avoided if 
possible. Instead, find a way to notify yourself when the thread has ended.
Usually you only need to notify the main thread, in which case you can post
an event to it via \helpref{wxPostEvent}{wxpostevent} or
\helpref{wxEvtHandler::AddPendingEvent}{wxevthandleraddpendingevent}. In 
the case of secondary threads you can call a routine of another class
when the thread is about to complete processing and/or set the value
of a variable, possibly using \helpref{mutexes}{wxmutex} and/or other 
synchronization means if necessary.

\membersection{wxThread::wxThread}\label{wxthreadctor}

\func{}{wxThread}{\param{wxThreadKind }{kind = wxTHREAD\_DETACHED}}

This constructor creates a new detached (default) or joinable C++ thread object. It
does not create or start execution of the real thread -- for this you should
use the \helpref{Create}{wxthreadcreate} and \helpref{Run}{wxthreadrun} methods.

The possible values for {\it kind} parameters are:

\twocolwidtha{7cm}
\begin{twocollist}\itemsep=0pt
\twocolitem{{\bf wxTHREAD\_DETACHED}}{Creates a detached thread.}
\twocolitem{{\bf wxTHREAD\_JOINABLE}}{Creates a joinable thread.}
\end{twocollist}


\membersection{wxThread::\destruct{wxThread}}\label{wxthreaddtor}

\func{}{\destruct{wxThread}}{\void}

The destructor frees the resources associated with the thread. Notice that you
should never delete a detached thread -- you may only call
\helpref{Delete}{wxthreaddelete} on it or wait until it terminates (and auto
destructs) itself. Because the detached threads delete themselves, they can
only be allocated on the heap.

Joinable threads should be deleted explicitly. The \helpref{Delete}{wxthreaddelete} and \helpref{Kill}{wxthreadkill} functions
will not delete the C++ thread object. It is also safe to allocate them on
stack.


\membersection{wxThread::Create}\label{wxthreadcreate}

\func{wxThreadError}{Create}{\param{unsigned int }{stackSize = 0}}

Creates a new thread. The thread object is created in the suspended state, and you
should call \helpref{Run}{wxthreadrun} to start running it.  You may optionally
specify the stack size to be allocated to it (Ignored on platforms that don't
support setting it explicitly, eg. Unix system without
\texttt{pthread\_attr\_setstacksize}). If you do not specify the stack size,
the system's default value is used.

{\bf Warning:} It is a good idea to explicitly specify a value as systems'
default values vary from just a couple of KB on some systems (BSD and
OS/2 systems) to one or several MB (Windows, Solaris, Linux). So, if you
have a thread that requires more than just a few KB of memory, you will
have mysterious problems on some platforms but not on the common ones. On the
other hand, just indicating a large stack size by default will give you
performance issues on those systems with small default stack since those
typically use fully committed memory for the stack. On the contrary, if
use a lot of threads (say several hundred), virtual adress space can get tight
unless you explicitly specify a smaller amount of thread stack space for each
thread.


\wxheading{Return value}

One of:

\twocolwidtha{7cm}
\begin{twocollist}\itemsep=0pt
\twocolitem{{\bf wxTHREAD\_NO\_ERROR}}{There was no error.}
\twocolitem{{\bf wxTHREAD\_NO\_RESOURCE}}{There were insufficient resources to create a new thread.}
\twocolitem{{\bf wxTHREAD\_RUNNING}}{The thread is already running.}
\end{twocollist}


\membersection{wxThread::Delete}\label{wxthreaddelete}

\func{wxThreadError}{Delete}{\void}

Calling \helpref{Delete}{wxthreaddelete} gracefully terminates a 
detached thread, either when the thread calls \helpref{TestDestroy}{wxthreadtestdestroy} or finished processing.

(Note that while this could work on a joinable thread you simply should not
call this routine on one as afterwards you may not be able to call 
\helpref{wxThread::Wait}{wxthreadwait} to free the memory of that thread).

See \helpref{wxThread deletion}{deletionwxthread} for a broader explanation of this routine.

%%FIXME: What does this return and why?

\membersection{wxThread::Entry}\label{wxthreadentry}

\func{virtual ExitCode}{Entry}{\void}

This is the entry point of the thread. This function is pure virtual and must
be implemented by any derived class. The thread execution will start here.

The returned value is the thread exit code which is only useful for
joinable threads and is the value returned by \helpref{Wait}{wxthreadwait}.

This function is called by wxWidgets itself and should never be called
directly.


\membersection{wxThread::Exit}\label{wxthreadexit}

\func{void}{Exit}{\param{ExitCode }{exitcode = 0}}

This is a protected function of the wxThread class and thus can only be called
from a derived class. It also can only be called in the context of this
thread, i.e. a thread can only exit from itself, not from another thread.

This function will terminate the OS thread (i.e. stop the associated path of
execution) and also delete the associated C++ object for detached threads.
\helpref{wxThread::OnExit}{wxthreadonexit} will be called just before exiting.


\membersection{wxThread::GetCPUCount}\label{wxthreadgetcpucount}

\func{static int}{GetCPUCount}{\void}

Returns the number of system CPUs or -1 if the value is unknown.

\wxheading{See also}

\helpref{SetConcurrency}{wxthreadsetconcurrency}


\membersection{wxThread::GetCurrentId}\label{wxthreadgetcurrentid}

\func{static unsigned long}{GetCurrentId}{\void}

Returns the platform specific thread ID of the current thread as a
long.  This can be used to uniquely identify threads, even if they are
not wxThreads.


\membersection{wxThread::GetId}\label{wxthreadgetid}

\constfunc{unsigned long}{GetId}{\void}

Gets the thread identifier: this is a platform dependent number that uniquely identifies the
thread throughout the system during its existence (i.e. the thread identifiers may be reused).


\membersection{wxThread::GetPriority}\label{wxthreadgetpriority}

\constfunc{int}{GetPriority}{\void}

Gets the priority of the thread, between zero and 100.

The following priorities are defined:

\twocolwidtha{7cm}
\begin{twocollist}\itemsep=0pt
\twocolitem{{\bf WXTHREAD\_MIN\_PRIORITY}}{0}
\twocolitem{{\bf WXTHREAD\_DEFAULT\_PRIORITY}}{50}
\twocolitem{{\bf WXTHREAD\_MAX\_PRIORITY}}{100}
\end{twocollist}


\membersection{wxThread::IsAlive}\label{wxthreadisalive}

\constfunc{bool}{IsAlive}{\void}

Returns \true if the thread is alive (i.e. started and not terminating).

Note that this function can only safely be used with joinable threads, not
detached ones as the latter delete themselves and so when the real thread is
no longer alive, it is not possible to call this function because
the wxThread object no longer exists.

\membersection{wxThread::IsDetached}\label{wxthreadisdetached}

\constfunc{bool}{IsDetached}{\void}

Returns \true if the thread is of the detached kind, \false if it is a joinable
one.


\membersection{wxThread::IsMain}\label{wxthreadismain}

\func{static bool}{IsMain}{\void}

Returns \true if the calling thread is the main application thread.


\membersection{wxThread::IsPaused}\label{wxthreadispaused}

\constfunc{bool}{IsPaused}{\void}

Returns \true if the thread is paused.


\membersection{wxThread::IsRunning}\label{wxthreadisrunning}

\constfunc{bool}{IsRunning}{\void}

Returns \true if the thread is running.

This method may only be safely used for joinable threads, see the remark in 
\helpref{IsAlive}{wxthreadisalive}.


\membersection{wxThread::Kill}\label{wxthreadkill}

\func{wxThreadError}{Kill}{\void}

Immediately terminates the target thread. {\bf This function is dangerous and should
be used with extreme care (and not used at all whenever possible)!} The resources
allocated to the thread will not be freed and the state of the C runtime library
may become inconsistent. Use \helpref{Delete()}{wxthreaddelete} for detached 
threads or \helpref{Wait()}{wxthreadwait} for joinable threads instead.

For detached threads Kill() will also delete the associated C++ object.
However this will not happen for joinable threads and this means that you will
still have to delete the wxThread object yourself to avoid memory leaks.
In neither case \helpref{OnExit}{wxthreadonexit} of the dying thread will be
called, so no thread-specific cleanup will be performed.

This function can only be called from another thread context, i.e. a thread
cannot kill itself.

It is also an error to call this function for a thread which is not running or
paused (in the latter case, the thread will be resumed first) -- if you do it,
a {\tt wxTHREAD\_NOT\_RUNNING} error will be returned.


\membersection{wxThread::OnExit}\label{wxthreadonexit}

\func{void}{OnExit}{\void}

Called when the thread exits. This function is called in the context of the
thread associated with the wxThread object, not in the context of the main
thread. This function will not be called if the thread was
\helpref{killed}{wxthreadkill}.

This function should never be called directly.


\membersection{wxThread::Pause}\label{wxthreadpause}

\func{wxThreadError}{Pause}{\void}

Suspends the thread. Under some implementations (Win32), the thread is
suspended immediately, under others it will only be suspended when it calls
\helpref{TestDestroy}{wxthreadtestdestroy} for the next time (hence, if the
thread doesn't call it at all, it won't be suspended).

This function can only be called from another thread context.


\membersection{wxThread::Run}\label{wxthreadrun}

\func{wxThreadError}{Run}{\void}

Starts the thread execution. Should be called after
\helpref{Create}{wxthreadcreate}.

This function can only be called from another thread context.


\membersection{wxThread::SetPriority}\label{wxthreadsetpriority}

\func{void}{SetPriority}{\param{int}{ priority}}

Sets the priority of the thread, between $0$ and $100$. It can only be set
after calling \helpref{Create()}{wxthreadcreate} but before calling
\helpref{Run()}{wxthreadrun}.

The following priorities are already defined:

\twocolwidtha{7cm}
\begin{twocollist}\itemsep=0pt
\twocolitem{{\bf WXTHREAD\_MIN\_PRIORITY}}{0}
\twocolitem{{\bf WXTHREAD\_DEFAULT\_PRIORITY}}{50}
\twocolitem{{\bf WXTHREAD\_MAX\_PRIORITY}}{100}
\end{twocollist}


\membersection{wxThread::Sleep}\label{wxthreadsleep}

\func{static void}{Sleep}{\param{unsigned long }{milliseconds}}

Pauses the thread execution for the given amount of time.

This function should be used instead of \helpref{wxSleep}{wxsleep} by all worker
threads (i.e. all except the main one).


\membersection{wxThread::Resume}\label{wxthreadresume}

\func{wxThreadError}{Resume}{\void}

Resumes a thread suspended by the call to \helpref{Pause}{wxthreadpause}.

This function can only be called from another thread context.


\membersection{wxThread::SetConcurrency}\label{wxthreadsetconcurrency}

\func{static bool}{SetConcurrency}{\param{size\_t }{level}}

Sets the thread concurrency level for this process. This is, roughly, the
number of threads that the system tries to schedule to run in parallel.
The value of $0$ for {\it level} may be used to set the default one.

Returns \true on success or false otherwise (for example, if this function is
not implemented for this platform -- currently everything except Solaris).


\membersection{wxThread::TestDestroy}\label{wxthreadtestdestroy}

\func{virtual bool}{TestDestroy}{\void}

This function should be called periodically by the thread to ensure that calls
to \helpref{Pause}{wxthreadpause} and \helpref{Delete}{wxthreaddelete} will
work. If it returns \true, the thread should exit as soon as possible.

Notice that under some platforms (POSIX), implementation of 
\helpref{Pause}{wxthreadpause} also relies on this function being called, so
not calling it would prevent both stopping and suspending thread from working.


\membersection{wxThread::This}\label{wxthreadthis}

\func{static wxThread *}{This}{\void}

Return the thread object for the calling thread. NULL is returned if the calling thread
is the main (GUI) thread, but \helpref{IsMain}{wxthreadismain} should be used to test
whether the thread is really the main one because NULL may also be returned for the thread
not created with wxThread class. Generally speaking, the return value for such a thread
is undefined.


\membersection{wxThread::Yield}\label{wxthreadyield}

\func{void}{Yield}{\void}

Give the rest of the thread time slice to the system allowing the other threads to run.
See also \helpref{Sleep()}{wxthreadsleep}.


\membersection{wxThread::Wait}\label{wxthreadwait}

\constfunc{ExitCode}{Wait}{\void}

Waits for a joinable thread to terminate and returns the value the thread
returned from \helpref{wxThread::Entry}{wxthreadentry} or {\tt (ExitCode)-1} on
error. Notice that, unlike \helpref{Delete}{wxthreaddelete} doesn't cancel the
thread in any way so the caller waits for as long as it takes to the thread to
exit.

You can only Wait() for joinable (not detached) threads.

This function can only be called from another thread context.

See \helpref{wxThread deletion}{deletionwxthread} for a broader explanation of this routine.

