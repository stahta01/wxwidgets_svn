\section{\class{wxThread}}\label{wxthread}

A thread is basically a path of execution through a program. Threads are
sometimes called {\it light-weight processes}, but the fundamental difference
between threads and processes is that memory spaces of different processes are
separated while all threads share the same address space. While it makes it
much easier to share common data between several threads, it also makes much
easier to shoot oneself in the foot, so careful use of synchronization objects
such as \helpref{mutexes}{wxmutex} and/or \helpref{critical sections}{wxcriticalsection} is recommended.

There are two types of threads in wxWindows: {\it detached} and {\it joinable}
ones, just as in the POSIX thread API (but unlike Win32 threads where all threads
are joinable). The difference between the two is that only joinable threads
can return a return code - this is returned by the Wait() function. Detached
threads (the default type) cannot be waited for.

You shouldn't hurry to create all the threads joinable, however, because this
has a disadvantage as well: you {\bf must} Wait() for a joinable thread or the
system resources used by it will never be freed, and you also must delete the
corresponding wxThread object yourself. In contrast, detached threads are of the
"fire-and-forget" kind: you only have to start a detached thread and it will
terminate and destroy itself.

This means, of course, that all detached threads {\bf must} be created on the
heap because the thread will call {\tt delete this;} upon termination. Joinable
threads may be created on the stack although more usually they will be created
on the heap as well. Don't create global thread objects because they allocate
memory in their constructor, which will cause problems for the memory checking
system.

\wxheading{Derived from}

None.

\wxheading{Include files}

<wx/thread.h>

\wxheading{See also}

\helpref{wxMutex}{wxmutex}, \helpref{wxCondition}{wxcondition}, \helpref{wxCriticalSection}{wxcriticalsection}

\latexignore{\rtfignore{\wxheading{Members}}}

\membersection{wxThread::wxThread}\label{wxthreadctor}

\func{}{wxThread}{\param{wxThreadKind }{kind = wxTHREAD\_DETACHED}}

This constructor creates a new detached (default) or joinable C++ thread object. It
does not create or start execution of the real thread - for this you should
use the \helpref{Create}{wxthreadcreate} and \helpref{Run}{wxthreadrun} methods.

The possible values for {\it kind} parameters are:

\twocolwidtha{7cm}
\begin{twocollist}\itemsep=0pt
\twocolitem{{\bf wxTHREAD\_DETACHED}}{Create a detached thread.}
\twocolitem{{\bf wxTHREAD\_JOINABLE}}{Create a joinable thread}
\end{twocollist}

\membersection{wxThread::\destruct{wxThread}}

\func{}{\destruct{wxThread}}{\void}

The destructor frees the resources associated with the thread. Notice that you
should never delete a detached thread - you may only call 
\helpref{Delete}{wxthreaddelete} on it or wait until it terminates (and auto
destructs) itself. Because the detached threads delete themselves, they can
only be allocated on the heap.

Joinable threads should be deleted explicitly. The \helpref{Delete}{wxthreaddelete} and \helpref{Kill}{wxthreadkill} functions
will not delete the C++ thread object. It is also safe to allocate them on
stack.

\membersection{wxThread::Create}\label{wxthreadcreate}

\func{wxThreadError}{Create}{\void}

Creates a new thread. The thread object is created in the suspended state, and you
should call \helpref{Run}{wxthreadrun} to start running it.

\wxheading{Return value}

One of:

\twocolwidtha{7cm}
\begin{twocollist}\itemsep=0pt
\twocolitem{{\bf wxTHREAD\_NO\_ERROR}}{There was no error.}
\twocolitem{{\bf wxTHREAD\_NO\_RESOURCE}}{There were insufficient resources to create a new thread.}
\twocolitem{{\bf wxTHREAD\_RUNNING}}{The thread is already running.}
\end{twocollist}

\membersection{wxThread::Delete}\label{wxthreaddelete}

\func{void}{Delete}{\void}

Calling \helpref{Delete}{wxthreaddelete} is a graceful way to terminate the
thread. It asks the thread to terminate and, if the thread code is well
written, the thread will terminate after the next call to 
\helpref{TestDestroy}{wxthreadtestdestroy} which should happen quite soon.

However, if the thread doesn't call \helpref{TestDestroy}{wxthreadtestdestroy} 
often enough (or at all), the function will not return immediately, but wait
until the thread terminates. As it may take a long time, and the message processing
is not stopped during this function execution, message handlers may be
called from inside it!

Delete() may be called for thread in any state: running, paused or even not yet
created. Moreover, it must be called if \helpref{Create}{wxthreadcreate} or 
\helpref{Run}{wxthreadrun} failed for a detached thread to free the memory
occupied by the thread object. This cleanup will be done in the destructor for joinable
threads.

Delete() may be called for a thread in any state: running, paused or even not yet created. Moreover,
it must be called if \helpref{Create}{wxthreadcreate} or \helpref{Run}{wxthreadrun} fail to free
the memory occupied by the thread object. However, you should not call Delete()
on a detached thread which already terminated - doing so will probably result
in a crash because the thread object doesn't exist any more.

For detached threads Delete() will also delete the C++ thread object, but it
will not do this for joinable ones.

This function can only be called from another thread context.

\membersection{wxThread::Entry}\label{wxthreadentry}

\func{virtual ExitCode}{Entry}{\void}

This is the entry point of the thread. This function is pure virtual and must
be implemented by any derived class. The thread execution will start here.

The returned value is the thread exit code which is only useful for
joinable threads and is the value returned by \helpref{Wait}{wxthreadwait}.

This function is called by wxWindows itself and should never be called
directly.

\membersection{wxThread::Exit}\label{wxthreadexit}

\func{void}{Exit}{\param{ExitCode }{exitcode = 0}}

This is a protected function of the wxThread class and thus can only be called
from a derived class. It also can only be called in the context of this
thread, i.e. a thread can only exit from itself, not from another thread.

This function will terminate the OS thread (i.e. stop the associated path of
execution) and also delete the associated C++ object for detached threads. 
\helpref{wxThread::OnExit}{wxthreadonexit} will be called just before exiting.

\membersection{wxThread::GetCPUCount}\label{wxthreadgetcpucount}

\func{static int}{GetCPUCount}{\void}

Returns the number of system CPUs or -1 if the value is unknown.

\wxheading{See also}

\helpref{SetConcurrency}{wxthreadsetconcurrency}

\membersection{wxThread::GetId}\label{wxthreadgetid}

\constfunc{unsigned long}{GetId}{\void}

Gets the thread identifier: this is a platform dependent number that uniquely identifies the
thread throughout the system during its existence (i.e. the thread identifiers may be reused).

\membersection{wxThread::GetPriority}\label{wxthreadgetpriority}

\constfunc{int}{GetPriority}{\void}

Gets the priority of the thread, between zero and 100.

The following priorities are defined:

\twocolwidtha{7cm}
\begin{twocollist}\itemsep=0pt
\twocolitem{{\bf WXTHREAD\_MIN\_PRIORITY}}{0}
\twocolitem{{\bf WXTHREAD\_DEFAULT\_PRIORITY}}{50}
\twocolitem{{\bf WXTHREAD\_MAX\_PRIORITY}}{100}
\end{twocollist}

\membersection{wxThread::IsAlive}\label{wxthreadisalive}

\constfunc{bool}{IsAlive}{\void}

Returns TRUE if the thread is alive (i.e. started and not terminating).

\membersection{wxThread::IsDetached}\label{wxthreadisdetached}

\constfunc{bool}{IsDetached}{\void}

Returns TRUE if the thread is of the detached kind, FALSE if it is a joinable one.

\membersection{wxThread::IsMain}\label{wxthreadismain}

\func{static bool}{IsMain}{\void}

Returns TRUE if the calling thread is the main application thread.

\membersection{wxThread::IsPaused}\label{wxthreadispaused}

\constfunc{bool}{IsPaused}{\void}

Returns TRUE if the thread is paused.

\membersection{wxThread::IsRunning}\label{wxthreadisrunning}

\constfunc{bool}{IsRunning}{\void}

Returns TRUE if the thread is running.

\membersection{wxThread::Kill}\label{wxthreadkill}

\func{wxThreadError}{Kill}{\void}

Immediately terminates the target thread. {\bf This function is dangerous and should
be used with extreme care (and not used at all whenever possible)!} The resources
allocated to the thread will not be freed and the state of the C runtime library
may become inconsistent. Use \helpref{Delete()}{wxthreaddelete} instead.

For detached threads Kill() will also delete the associated C++ object.
However this will not happen for joinable threads and this means that you will
still have to delete the wxThread object yourself to avoid memory leaks.
In neither case \helpref{OnExit}{wxthreadonexit} of the dying thread will be
called, so no thread-specific cleanup will be performed.

This function can only be called from another thread context, i.e. a thread
cannot kill itself.

It is also an error to call this function for a thread which is not running or
paused (in the latter case, the thread will be resumed first) - if you do it, 
a {\tt wxTHREAD\_NOT\_RUNNING} error will be returned.

\membersection{wxThread::OnExit}\label{wxthreadonexit}

\func{void}{OnExit}{\void}

Called when the thread exits. This function is called in the context of the
thread associated with the wxThread object, not in the context of the main
thread. This function will not be called if the thread was 
\helpref{killed}{wxthreadkill}.

This function should never be called directly.

\membersection{wxThread::Pause}\label{wxthreadpause}

\func{wxThreadError}{Pause}{\void}

Suspends the thread. Under some implementations (Win32), the thread is
suspended immediately, under others it will only be suspended when it calls 
\helpref{TestDestroy}{wxthreadtestdestroy} for the next time (hence, if the
thread doesn't call it at all, it won't be suspended).

This function can only be called from another thread context.

\membersection{wxThread::Run}\label{wxthreadrun}

\func{wxThreadError}{Run}{\void}

Starts the thread execution. Should be called after 
\helpref{Create}{wxthreadcreate}.

This function can only be called from another thread context.

\membersection{wxThread::SetPriority}\label{wxthreadsetpriority}

\func{void}{SetPriority}{\param{int}{ priority}}

Sets the priority of the thread, between $0$ and $100$. It can only be set
after calling \helpref{Create()}{wxthreadcreate} but before calling 
\helpref{Run()}{wxthreadrun}.

The following priorities are already defined:

\twocolwidtha{7cm}
\begin{twocollist}\itemsep=0pt
\twocolitem{{\bf WXTHREAD\_MIN\_PRIORITY}}{0}
\twocolitem{{\bf WXTHREAD\_DEFAULT\_PRIORITY}}{50}
\twocolitem{{\bf WXTHREAD\_MAX\_PRIORITY}}{100}
\end{twocollist}

\membersection{wxThread::Sleep}\label{wxthreadsleep}

\func{static void}{Sleep}{\param{unsigned long }{milliseconds}}

Pauses the thread execution for the given amount of time.

This function should be used instead of \helpref{wxSleep}{wxsleep} by all worker
threads (i.e. all except the main one).

\membersection{wxThread::Resume}\label{wxthreadresume}

\func{wxThreadError}{Resume}{\void}

Resumes a thread suspended by the call to \helpref{Pause}{wxthreadpause}.

This function can only be called from another thread context.

\membersection{wxThread::SetConcurrency}\label{wxthreadsetconcurrency}

\func{static bool}{SetConcurrency}{\param{size\_t }{level}}

Sets the thread concurrency level for this process. This is, roughly, the
number of threads that the system tries to schedule to run in parallel.
The value of $0$ for {\it level} may be used to set the default one.

Returns TRUE on success or FALSE otherwise (for example, if this function is
not implemented for this platform - currently everything except Solaris).

\membersection{wxThread::TestDestroy}\label{wxthreadtestdestroy}

\func{bool}{TestDestroy}{\void}

This function should be called periodically by the thread to ensure that calls
to \helpref{Pause}{wxthreadpause} and \helpref{Delete}{wxthreaddelete} will
work. If it returns TRUE, the thread should exit as soon as possible.

\membersection{wxThread::This}\label{wxthreadthis}

\func{static wxThread *}{This}{\void}

Return the thread object for the calling thread. NULL is returned if the calling thread
is the main (GUI) thread, but \helpref{IsMain}{wxthreadismain} should be used to test
whether the thread is really the main one because NULL may also be returned for the thread
not created with wxThread class. Generally speaking, the return value for such a thread
is undefined.

\membersection{wxThread::Yield}\label{wxthreadyield}

\func{void}{Yield}{\void}

Give the rest of the thread time slice to the system allowing the other threads to run.
See also \helpref{Sleep()}{wxthreadsleep}.

\membersection{wxThread::Wait}\label{wxthreadwait}

\constfunc{ExitCode}{Wait}{\void}

Waits until the thread terminates and returns its exit code or {\tt (ExitCode)-1} on error.

You can only Wait() for joinable (not detached) threads.

This function can only be called from another thread context.

