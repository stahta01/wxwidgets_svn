%
% automatically generated by HelpGen from
% htmlhelp.h at 02/May/99 19:58:53
%


\section{\class{wxHtmlHelpController}}\label{wxhtmlhelpcontroller}

{\bf WARNING! This help controller has API incompatible with wxWindows
wxHelpController!}

This help controller provides easy way how to display HTML help in your
application (see {\it test} sample). Whole help system is based on {\bf books}
(see \helpref{AddBook}{wxhtmlhelpcontrolleraddbook}). A book is logical
part of documentation (for example "User's Guide" or "Programmer's Guide" or
"C++ Reference" or "wxWindows Reference"). Help controller can handle as
many books as you want.

wxHTML uses Microsoft's HTML Help Workshop project files (.hhp, .hhk, .hhc) as its
native format. The file format is described \helpref{here}{helpformat}.
Have a look at docs/html/ directory where sample project files are stored.

You can use tex2rtf to generate MHHW projects (see wxHTML homepage for details).

In order to use the controller in your application under Windows you must
have following line in your .rc file:

\begin{verbatim}
#include "wx/html/msw/wxhtml.rc"
\end{verbatim}


\wxheading{Derived from}

wxEvtHandler


\latexignore{\rtfignore{\wxheading{Members}}}


\membersection{wxHtmlHelpController::wxHtmlHelpController}\label{wxhtmlhelpcontrollerwxhtmlhelpcontroller}

\func{}{wxHtmlHelpController}{\void}


Constructor.


\membersection{wxHtmlHelpController::SetTitleFormat}\label{wxhtmlhelpcontrollersettitleformat}

\func{void}{SetTitleFormat}{\param{const wxString\& }{format}}

Sets format of title of the frame. Must contain exactly one "\%s"
(for title of displayed HTML page).


\membersection{wxHtmlHelpController::SetTempDir}\label{wxhtmlhelpcontrollersettempdir}

\func{void}{SetTempDir}{\param{const wxString\& }{path}}

Sets path for storing temporary files (cached binary versions of index and contents files. These binary
forms are much faster to read.) Default value is empty string (empty string means
that no cached data are stored). Note that these files are NOT
deleted when program exits!


\membersection{wxHtmlHelpController::AddBook}\label{wxhtmlhelpcontrolleraddbook}

\func{bool}{AddBook}{\param{const wxString\& }{book}, \param{bool }{show_wait_msg}}

Adds book (\helpref{.hhp file}{helpformat} - HTML Help Workshop project file) into the list of loaded books.
This must be called at least once before displaying  any help.

If {\it show_wait_msg} is TRUE then a decorationless window with progress message is displayed.


\membersection{wxHtmlHelpController::Display}\label{wxhtmlhelpcontrollerdisplay}

\func{void}{Display}{\param{const wxString\& }{x}}

Displays page {\it x}. This is THE important function - it is used to display
the help in application.

You can specify the page in many ways:

\begin{itemize}
\item as direct filename of HTML document
\item as chapter name (from contents) or as a book name
\item as some word from index
\item even as any word (will be searched) 
\end{itemize}

Looking for the page runs in these steps:

\begin{enumerate}
\item try to locate file named x (if x is for example "doc/howto.htm")
\item try to open starting page of book named x
\item try to find x in contents (if x is for example "How To ...")
\item try to find x in index (if x is for example "How To ...")
\item switch to Search panel and start searching
\end{enumerate}

\func{void}{Display}{\param{const int }{id}}

This alternative form is used to search help contents by numeric IDs.




\membersection{wxHtmlHelpController::DisplayContents}\label{wxhtmlhelpcontrollerdisplaycontents}

\func{void}{DisplayContents}{\void}

Displays help window and focuses contents panel.

\membersection{wxHtmlHelpController::DisplayIndex}\label{wxhtmlhelpcontrollerdisplayindex}

\func{void}{DisplayIndex}{\void}

Displays help window and focuses index panel.


\membersection{wxHtmlHelpController::KeywordSearch}\label{wxhtmlhelpcontrollerkeywordsearch}

\func{bool}{KeywordSearch}{\param{const wxString\& }{keyword}}

Displays help window, focuses search panel and starts searching.
Returns TRUE if the keyword was found.

IMPORTANT! KeywordSearch searches only pages listed in .htc file(s)!
(you should have all pages in contents file...)


\membersection{wxHtmlHelpController::UseConfig}\label{wxhtmlhelpcontrolleruseconfig}

\func{void}{UseConfig}{\param{wxConfigBase* }{config}, \param{const wxString\& }{rootpath = wxEmptyString}}

Associates {\it config} object with the controller.

If there is associated config object, wxHtmlHelpController automatically
reads and writes settings (including wxHtmlWindow's settings) when needed.

The only thing you must do is create wxConfig object and call UseConfig. 


\membersection{wxHtmlHelpController::ReadCustomization}\label{wxhtmlhelpcontrollerreadcustomization}

\func{void}{ReadCustomization}{\param{wxConfigBase* }{cfg}, \param{wxString }{path = wxEmptyString}}

Reads controllers setting (position of window etc.)


\membersection{wxHtmlHelpController::WriteCustomization}\label{wxhtmlhelpcontrollerwritecustomization}

\func{void}{WriteCustomization}{\param{wxConfigBase* }{cfg}, \param{wxString }{path = wxEmptyString}}

Stores controllers setting (position of window etc.)

