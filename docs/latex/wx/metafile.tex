\section{\class{wxMetafile}}\label{wxmetafile}

A {\bf wxMetafile} represents the MS Windows metafile object, so metafile
operations have no effect in X. In wxWindows, only sufficient functionality
has been provided for copying a graphic to the clipboard; this may be extended
in a future version. Presently, the only way of creating a metafile
is to use a wxMetafileDC.

\wxheading{Derived from}

\helpref{wxObject}{wxobject}

\wxheading{Include files}

<wx/metafile.h>

\wxheading{See also}

\helpref{wxMetafileDC}{wxmetafiledc}

\latexignore{\rtfignore{\wxheading{Members}}}

\membersection{wxMetafile::wxMetafile}

\func{}{wxMetafile}{\param{const wxString\& }{filename = ""}}

Constructor. If a filename is given, the Windows disk metafile is
read in. Check whether this was performed successfully by
using the \helpref{wxMetafile::Ok}{wxmetafileok} member.

\membersection{wxMetafile::\destruct{wxMetafile}}

\func{}{\destruct{wxMetafile}}{\void}

Destructor.

\membersection{wxMetafile::Ok}\label{wxmetafileok}

\func{bool}{Ok}{\void}

Returns TRUE if the metafile is valid.

\membersection{wxMetafile::Play}\label{wxmetafileplay}

\func{bool}{Play}{\param{wxDC *}{dc}}

Plays the metafile into the given device context, returning
TRUE if successful.

\membersection{wxMetafile::SetClipboard}

\func{bool}{SetClipboard}{\param{int}{ width = 0}, \param{int}{ height = 0}}

Passes the metafile data to the clipboard. The metafile can no longer be
used for anything, but the wxMetafile object must still be destroyed by
the application.

Below is a example of metafile, metafile device context and clipboard use
from the {\tt hello.cpp} example. Note the way the metafile dimensions
are passed to the clipboard, making use of the device context's ability
to keep track of the maximum extent of drawing commands.

\begin{verbatim}
  wxMetafileDC dc;
  if (dc.Ok())
  {
    Draw(dc, FALSE);
    wxMetafile *mf = dc.Close();
    if (mf)
    {
      bool success = mf->SetClipboard((int)(dc.MaxX() + 10), (int)(dc.MaxY() + 10));
      delete mf;
    }
  }
\end{verbatim}

\section{\class{wxMetafileDC}}\label{wxmetafiledc}

This is a type of device context that allows a metafile object to be
created (Windows only), and has most of the characteristics of a normal
\rtfsp{\bf wxDC}. The \helpref{wxMetafileDC::Close}{wxmetafiledcclose} member must be called after drawing into the
device context, to return a metafile. The only purpose for this at
present is to allow the metafile to be copied to the clipboard (see \helpref{wxMetafile}{wxmetafile}).

Adding metafile capability to an application should be easy if you
already write to a wxDC; simply pass the wxMetafileDC to your drawing
function instead. You may wish to conditionally compile this code so it
is not compiled under X (although no harm will result if you leave it
in).

Note that a metafile saved to disk is in standard Windows metafile format,
and cannot be imported into most applications. To make it importable,
call the function \helpref{::wxMakeMetafilePlaceable}{wxmakemetafileplaceable} after
closing your disk-based metafile device context.

\wxheading{Derived from}

\helpref{wxDC}{wxdc}\\
\helpref{wxObject}{wxobject}

\wxheading{Include files}

<wx/metafile.h>

\wxheading{See also}

\helpref{wxMetafile}{wxmetafile}, \helpref{wxDC}{wxdc}

\latexignore{\rtfignore{\wxheading{Members}}}

\membersection{wxMetafileDC::wxMetafileDC}

\func{}{wxMetafileDC}{\param{const wxString\& }{filename = ""}}

Constructor. If no filename is passed, the metafile is created
in memory.

\membersection{wxMetafileDC::\destruct{wxMetafileDC}}

\func{}{\destruct{wxMetafileDC}}{\void}

Destructor.

\membersection{wxMetafileDC::Close}\label{wxmetafiledcclose}

\func{wxMetafile *}{Close}{\void}

This must be called after the device context is finished with. A
metafile is returned, and ownership of it passes to the calling
application (so it should be destroyed explicitly).

