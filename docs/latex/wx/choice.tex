%%%%%%%%%%%%%%%%%%%%%%%%%%%%%%%%%%%%%%%%%%%%%%%%%%%%%%%%%%%%%%%%%%%%%%%%%%%%%%%
%% Name:        treectrl.tex
%% Purpose:     wxChoice documentation
%% Author:      wxWidgets Team
%% Modified by:
%% Created:
%% RCS-ID:      $Id$
%% Copyright:   (c) wxWidgets Team
%% License:     wxWindows license
%%%%%%%%%%%%%%%%%%%%%%%%%%%%%%%%%%%%%%%%%%%%%%%%%%%%%%%%%%%%%%%%%%%%%%%%%%%%%%%

\section{\class{wxChoice}}\label{wxchoice}

A choice item is used to select one of a list of strings. Unlike a
listbox, only the selection is visible until the user pulls down the
menu of choices.

\wxheading{Derived from}

\helpref{wxControlWithItems}{wxcontrolwithitems}\\
\helpref{wxControl}{wxcontrol}\\
\helpref{wxWindow}{wxwindow}\\
\helpref{wxEvtHandler}{wxevthandler}\\
\helpref{wxObject}{wxobject}

\wxheading{Include files}

<wx/choice.h>

\wxheading{Window styles}

There are no special styles for wxChoice.

See also \helpref{window styles overview}{windowstyles}.

\wxheading{Event handling}

\twocolwidtha{7cm}
\begin{twocollist}\itemsep=0pt
\twocolitem{{\bf EVT\_CHOICE(id, func)}}{Process a wxEVT\_COMMAND\_CHOICE\_SELECTED event,
when an item on the list is selected.}
\end{twocollist}

\wxheading{See also}

\helpref{wxListBox}{wxlistbox}, \helpref{wxComboBox}{wxcombobox},
\rtfsp\helpref{wxCommandEvent}{wxcommandevent}

\latexignore{\rtfignore{\wxheading{Members}}}


\membersection{wxChoice::wxChoice}\label{wxchoicector}

\func{}{wxChoice}{\void}

Default constructor.

\func{}{wxChoice}{\param{wxWindow *}{parent}, \param{wxWindowID}{ id},\rtfsp
\param{const wxPoint\& }{pos}, \param{const wxSize\&}{ size},\rtfsp
\param{int}{ n}, \param{const wxString }{choices[]},\rtfsp
\param{long}{ style = 0}, \param{const wxValidator\& }{validator = wxDefaultValidator}, \param{const wxString\& }{name = ``choice"}}

\func{}{wxChoice}{\param{wxWindow *}{parent}, \param{wxWindowID}{ id},\rtfsp
\param{const wxPoint\& }{pos}, \param{const wxSize\&}{ size},\rtfsp
\param{const wxArrayString\& }{choices},\rtfsp
\param{long}{ style = 0}, \param{const wxValidator\& }{validator = wxDefaultValidator}, \param{const wxString\& }{name = ``choice"}}

Constructor, creating and showing a choice.

\wxheading{Parameters}

\docparam{parent}{Parent window. Must not be NULL.}

\docparam{id}{Window identifier. A value of -1 indicates a default value.}

\docparam{pos}{Window position.}

\docparam{size}{Window size. If the default size (-1, -1) is specified then the choice is sized
appropriately.}

\docparam{n}{Number of strings with which to initialise the choice control.}

\docparam{choices}{An array of strings with which to initialise the choice control.}

\docparam{style}{Window style. See \helpref{wxChoice}{wxchoice}.}

\docparam{validator}{Window validator.}

\docparam{name}{Window name.}

\wxheading{See also}

\helpref{wxChoice::Create}{wxchoicecreate}, \helpref{wxValidator}{wxvalidator}

\pythonnote{The wxChoice constructor in wxPython reduces the {\tt n}
and {\tt choices} arguments are to a single argument, which is
a list of strings.}

\perlnote{In wxPerl there is just an array reference in place of {\tt n}
and {\tt choices}.}


\membersection{wxChoice::\destruct{wxChoice}}\label{wxchoicedtor}

\func{}{\destruct{wxChoice}}{\void}

Destructor, destroying the choice item.


\membersection{wxChoice::Create}\label{wxchoicecreate}

\func{bool}{Create}{\param{wxWindow *}{parent}, \param{wxWindowID}{ id},\rtfsp
\param{const wxPoint\& }{pos}, \param{const wxSize\&}{ size},\rtfsp
\param{int}{ n}, \param{const wxString }{choices[]},\rtfsp
\param{long}{ style = 0}, \param{const wxValidator\& }{validator = wxDefaultValidator},\rtfsp
\param{const wxString\& }{name = ``choice"}}

\func{bool}{Create}{\param{wxWindow *}{parent}, \param{wxWindowID}{ id},\rtfsp
\param{const wxPoint\& }{pos}, \param{const wxSize\&}{ size},\rtfsp
\param{const wxArrayString\& }{choices},\rtfsp
\param{long}{ style = 0}, \param{const wxValidator\& }{validator = wxDefaultValidator},\rtfsp
\param{const wxString\& }{name = ``choice"}}

Creates the choice for two-step construction. See \helpref{wxChoice::wxChoice}{wxchoicector}.


\membersection{wxChoice::GetColumns}\label{wxchoicegetcolumns}

\constfunc{int}{GetColumns}{\void}

Gets the number of columns in this choice item.

\wxheading{Remarks}

This is implemented for Motif only and always returns $1$ for the other
platforms.


\membersection{wxChoice::GetCurrentSelection}\label{wxchoicegetcurrentselection}

\constfunc{int}{GetCurrentSelection}{\void}

Unlike \helpref{GetSelection}{wxcontrolwithitemsgetselection} which only
returns the accepted selection value, i.e. the selection in the control once
the user closes the dropdown list, this function returns the current selection.
That is, while the dropdown list is shown, it returns the currently selected
item in it. When it is not shown, its result is the same as for the other
function.

\newsince{2.6.2} (before this version
\helpref{GetSelection}{wxcontrolwithitemsgetselection} itself behaved like
this).


\membersection{wxChoice::SetColumns}\label{wxchoicesetcolumns}

\func{void}{SetColumns}{\param{int}{ n = 1}}

Sets the number of columns in this choice item.

\wxheading{Parameters}

\docparam{n}{Number of columns.}

\wxheading{Remarks}

This is implemented for Motif only and doesn't do anything under other
platforms.

