\section{\class{wxRichTextBuffer}}\label{wxrichtextbuffer}

This class represents the whole buffer associated with a \helpref{wxRichTextCtrl}{wxrichtextctrl}.

\wxheading{Derived from}

wxRichTextParagraphLayoutBox

\wxheading{Include files}

<wx/richtext/richtextbuffer.h>

\wxheading{Data structures}

\wxheading{See also}

\helpref{wxTextAttr}{wxtextattr}, \helpref{wxTextAttrEx}{wxtextattrex}, \helpref{wxRichTextAttr}{wxrichtextattr}, \helpref{wxRichTextCtrl}{wxrichtextctrl}

\latexignore{\rtfignore{\wxheading{Members}}}

\membersection{wxRichTextBuffer::wxRichTextBuffer}\label{wxrichtextbufferwxrichtextbuffer}

\func{}{wxRichTextBuffer}{\param{const wxRichTextBuffer\& }{obj}}

Copy constructor.

\func{}{wxRichTextBuffer}{\void}

Default constructors.

\membersection{wxRichTextBuffer::\destruct{wxRichTextBuffer}}\label{wxrichtextbufferdtor}

\func{}{\destruct{wxRichTextBuffer}}{\void}

Destructor.

\membersection{wxRichTextBuffer::AddEventHandler}\label{wxrichtextbufferaddeventhandler}

\func{bool}{AddEventHandler}{\param{wxEvtHandler* }{handler}}

Adds an event handler to the buffer's list of handlers. A buffer associated with
a contol has the control as the only event handler, but the application is free
to add more if further notification is required. All handlers are notified
of an event originating from the buffer, such as the replacement of a style sheet
during loading. The buffer never deletes any of the event handlers, unless 
\helpref{wxRichTextBuffer::RemoveEventHandler}{wxrichtextbufferremoveeventhandler} is
called with \true as the second argument.

\membersection{wxRichTextBuffer::AddHandler}\label{wxrichtextbufferaddhandler}

\func{void}{AddHandler}{\param{wxRichTextFileHandler* }{handler}}

Adds a file handler.

\membersection{wxRichTextBuffer::AddParagraph}\label{wxrichtextbufferaddparagraph}

\func{wxRichTextRange}{AddParagraph}{\param{const wxString\& }{text}}

Adds a paragraph of text.

\membersection{wxRichTextBuffer::BatchingUndo}\label{wxrichtextbufferbatchingundo}

\constfunc{bool}{BatchingUndo}{\void}

Returns \true if the buffer is currently collapsing commands into a single notional command.

\membersection{wxRichTextBuffer::BeginAlignment}\label{wxrichtextbufferbeginalignment}

\func{bool}{BeginAlignment}{\param{wxTextAttrAlignment }{alignment}}

Begins using alignment.

\membersection{wxRichTextBuffer::BeginBatchUndo}\label{wxrichtextbufferbeginbatchundo}

\func{bool}{BeginBatchUndo}{\param{const wxString\& }{cmdName}}

Begins collapsing undo/redo commands. Note that this may not work properly
if combining commands that delete or insert content, changing ranges for
subsequent actions.

{\it cmdName} should be the name of the combined command that will appear
next to Undo and Redo in the edit menu.

\membersection{wxRichTextBuffer::BeginBold}\label{wxrichtextbufferbeginbold}

\func{bool}{BeginBold}{\void}

Begin applying bold.

\membersection{wxRichTextBuffer::BeginCharacterStyle}\label{wxrichtextbufferbegincharacterstyle}

\func{bool}{BeginCharacterStyle}{\param{const wxString\& }{characterStyle}}

Begins applying the named character style.

\membersection{wxRichTextBuffer::BeginFont}\label{wxrichtextbufferbeginfont}

\func{bool}{BeginFont}{\param{const wxFont\& }{font}}

Begins using this font.

\membersection{wxRichTextBuffer::BeginFontSize}\label{wxrichtextbufferbeginfontsize}

\func{bool}{BeginFontSize}{\param{int }{pointSize}}

Begins using the given point size.

\membersection{wxRichTextBuffer::BeginItalic}\label{wxrichtextbufferbeginitalic}

\func{bool}{BeginItalic}{\void}

Begins using italic.

\membersection{wxRichTextBuffer::BeginLeftIndent}\label{wxrichtextbufferbeginleftindent}

\func{bool}{BeginLeftIndent}{\param{int }{leftIndent}, \param{int }{leftSubIndent = 0}}

Begin using {\it leftIndent} for the left indent, and optionally {\it leftSubIndent} for
the sub-indent. Both are expressed in tenths of a millimetre.

The sub-indent is an offset from the left of the paragraph, and is used for all but the
first line in a paragraph. A positive value will cause the first line to appear to the left
of the subsequent lines, and a negative value will cause the first line to be indented
relative to the subsequent lines.

\membersection{wxRichTextBuffer::BeginLineSpacing}\label{wxrichtextbufferbeginlinespacing}

\func{bool}{BeginLineSpacing}{\param{int }{lineSpacing}}

Begins line spacing using the specified value. {\it spacing} is a multiple, where 10 means single-spacing,
15 means 1.5 spacing, and 20 means double spacing. The following constants are
defined for convenience:

{\small
\begin{verbatim}
#define wxTEXT_ATTR_LINE_SPACING_NORMAL         10
#define wxTEXT_ATTR_LINE_SPACING_HALF           15
#define wxTEXT_ATTR_LINE_SPACING_TWICE          20
\end{verbatim}
}

\membersection{wxRichTextBuffer::BeginListStyle}\label{wxrichtextbufferbeginliststyle}

\func{bool}{BeginListStyle}{\param{const wxString\&}{ listStyle}, \param{int}{ level=1}, \param{int}{ number=1}}

Begins using a specified list style. Optionally, you can also pass a level and a number.

\membersection{wxRichTextBuffer::BeginNumberedBullet}\label{wxrichtextbufferbeginnumberedbullet}

\func{bool}{BeginNumberedBullet}{\param{int }{bulletNumber}, \param{int }{leftIndent}, \param{int }{leftSubIndent}, \param{int }{bulletStyle = wxTEXT\_ATTR\_BULLET\_STYLE\_ARABIC|wxTEXT\_ATTR\_BULLET\_STYLE\_PERIOD}}

Begins a numbered bullet. This call will be needed for each item in the list, and the
application should take care of incrementing the numbering.

{\it bulletNumber} is a number, usually starting with 1.

{\it leftIndent} and {\it leftSubIndent} are values in tenths of a millimetre.

{\it bulletStyle} is a bitlist of the following values:

{\small
\begin{verbatim}
#define wxTEXT_ATTR_BULLET_STYLE_NONE               0x00000000
#define wxTEXT_ATTR_BULLET_STYLE_ARABIC             0x00000001
#define wxTEXT_ATTR_BULLET_STYLE_LETTERS_UPPER      0x00000002
#define wxTEXT_ATTR_BULLET_STYLE_LETTERS_LOWER      0x00000004
#define wxTEXT_ATTR_BULLET_STYLE_ROMAN_UPPER        0x00000008
#define wxTEXT_ATTR_BULLET_STYLE_ROMAN_LOWER        0x00000010
#define wxTEXT_ATTR_BULLET_STYLE_SYMBOL             0x00000020
#define wxTEXT_ATTR_BULLET_STYLE_BITMAP             0x00000040
#define wxTEXT_ATTR_BULLET_STYLE_PARENTHESES        0x00000080
#define wxTEXT_ATTR_BULLET_STYLE_PERIOD             0x00000100
#define wxTEXT_ATTR_BULLET_STYLE_STANDARD           0x00000200
#define wxTEXT_ATTR_BULLET_STYLE_RIGHT_PARENTHESIS  0x00000400
#define wxTEXT_ATTR_BULLET_STYLE_OUTLINE            0x00000800
#define wxTEXT_ATTR_BULLET_STYLE_ALIGN_LEFT         0x00000000
#define wxTEXT_ATTR_BULLET_STYLE_ALIGN_RIGHT        0x00001000
#define wxTEXT_ATTR_BULLET_STYLE_ALIGN_CENTRE       0x00002000
\end{verbatim}
}

wxRichTextBuffer uses indentation to render a bulleted item. The left indent is the distance between
the margin and the bullet. The content of the paragraph, including the first line, starts
at leftMargin + leftSubIndent. So the distance between the left edge of the bullet and the
left of the actual paragraph is leftSubIndent.

\membersection{wxRichTextBuffer::BeginParagraphSpacing}\label{wxrichtextbufferbeginparagraphspacing}

\func{bool}{BeginParagraphSpacing}{\param{int }{before}, \param{int }{after}}

Begins paragraph spacing; pass the before-paragraph and after-paragraph spacing in tenths of
a millimetre.

\membersection{wxRichTextBuffer::BeginParagraphStyle}\label{wxrichtextbufferbeginparagraphstyle}

\func{bool}{BeginParagraphStyle}{\param{const wxString\& }{paragraphStyle}}

Begins applying the named paragraph style.

\membersection{wxRichTextBuffer::BeginRightIndent}\label{wxrichtextbufferbeginrightindent}

\func{bool}{BeginRightIndent}{\param{int }{rightIndent}}

Begins a right indent, specified in tenths of a millimetre.

\membersection{wxRichTextBuffer::BeginStyle}\label{wxrichtextbufferbeginstyle}

\func{bool}{BeginStyle}{\param{const wxTextAttrEx\& }{style}}

Begins using a specified style.

\membersection{wxRichTextBuffer::BeginSuppressUndo}\label{wxrichtextbufferbeginsuppressundo}

\func{bool}{BeginSuppressUndo}{\void}

Begins suppressing undo/redo commands. The way undo is suppressed may be implemented
differently by each command. If not dealt with by a command implementation, then
it will be implemented automatically by not storing the command in the undo history
when the action is submitted to the command processor.

\membersection{wxRichTextBuffer::BeginStandardBullet}\label{wxrichtextbufferbeginstandardbullet}

\func{bool}{BeginStandardBullet}{\param{const wxString\&}{ bulletName}, \param{int }{leftIndent}, \param{int }{leftSubIndent}, \param{int }{bulletStyle = wxTEXT\_ATTR\_BULLET\_STYLE\_STANDARD}}

Begins applying a standard bullet, using one of the standard bullet names (currently {\tt standard/circle} or {\tt standard/square}.
See \helpref{BeginNumberedBullet}{wxrichtextbufferbeginnumberedbullet} for an explanation of how indentation is used to render the bulleted paragraph.

\membersection{wxRichTextBuffer::BeginSymbolBullet}\label{wxrichtextbufferbeginsymbolbullet}

\func{bool}{BeginSymbolBullet}{\param{wxChar }{symbol}, \param{int }{leftIndent}, \param{int }{leftSubIndent}, \param{int }{bulletStyle = wxTEXT\_ATTR\_BULLET\_STYLE\_SYMBOL}}

Begins applying a symbol bullet, using a character from the current font. See \helpref{BeginNumberedBullet}{wxrichtextbufferbeginnumberedbullet} for
an explanation of how indentation is used to render the bulleted paragraph.

\membersection{wxRichTextBuffer::BeginTextColour}\label{wxrichtextbufferbegintextcolour}

\func{bool}{BeginTextColour}{\param{const wxColour\& }{colour}}

Begins using the specified text foreground colour.

\membersection{wxRichTextBuffer::BeginUnderline}\label{wxrichtextbufferbeginunderline}

\func{bool}{BeginUnderline}{\void}

Begins using underline.

\membersection{wxRichTextBuffer::BeginURL}\label{wxrichtextbufferbeginurl}

\func{bool}{BeginURL}{\param{const wxString\&}{ url}, \param{const wxString\&}{ characterStyle = wxEmptyString}}

Begins applying wxTEXT\_ATTR\_URL to the content. Pass a URL and optionally, a character style to apply,
since it is common to mark a URL with a familiar style such as blue text with underlining.

\membersection{wxRichTextBuffer::CanPasteFromClipboard}\label{wxrichtextbuffercanpastefromclipboard}

\constfunc{bool}{CanPasteFromClipboard}{\void}

Returns \true if content can be pasted from the clipboard.

\membersection{wxRichTextBuffer::CleanUpHandlers}\label{wxrichtextbuffercleanuphandlers}

\func{void}{CleanUpHandlers}{\void}

Cleans up the file handlers.

\membersection{wxRichTextBuffer::Clear}\label{wxrichtextbufferclear}

\func{void}{Clear}{\void}

Clears the buffer.

\membersection{wxRichTextBuffer::ClearListStyle}\label{wxrichtextbufferclearliststyle}

\func{bool}{ClearListStyle}{\param{const wxRichTextRange\& }{range}, \param{int }{flags $=$ wxRICHTEXT\_SETSTYLE\_WITH\_UNDO}}

\func{bool}{ClearListStyle}{\param{const wxRichTextRange\& }{range}, \param{int }{flags $=$ wxRICHTEXT\_SETSTYLE\_WITH\_UNDO}}

Clears the list style from the given range, clearing list-related attributes and applying any named paragraph style associated with each paragraph.

{\it flags} is a bit list of the following:

\begin{itemize}\itemsep=0pt
\item wxRICHTEXT\_SETSTYLE\_WITH\_UNDO: specifies that this command will be undoable.
\end{itemize}

See also \helpref{wxRichTextBuffer::SetListStyle}{wxrichtextbuffersetliststyle}, \helpref{wxRichTextBuffer::PromoteList}{wxrichtextbufferpromotelist}, \helpref{wxRichTextBuffer::NumberList}{wxrichtextbuffernumberlist}.

\membersection{wxRichTextBuffer::ClearStyleStack}\label{wxrichtextbufferclearstylestack}

\func{void}{ClearStyleStack}{\void}

Clears the style stack.

\membersection{wxRichTextBuffer::Clone}\label{wxrichtextbufferclone}

\constfunc{wxRichTextObject*}{Clone}{\void}

Clones the object.

\membersection{wxRichTextBuffer::Copy}\label{wxrichtextbuffercopy}

\func{void}{Copy}{\param{const wxRichTextBuffer\& }{obj}}

Copies the given buffer.

\membersection{wxRichTextBuffer::CopyToClipboard}\label{wxrichtextbuffercopytoclipboard}

\func{bool}{CopyToClipboard}{\param{const wxRichTextRange\& }{range}}

Copy the given range to the clipboard.

\membersection{wxRichTextBuffer::DeleteRangeWithUndo}\label{wxrichtextbufferdeleterangewithundo}

\func{bool}{DeleteRangeWithUndo}{\param{const wxRichTextRange\& }{range}, \param{wxRichTextCtrl* }{ctrl}}

Submits a command to delete the given range.

\membersection{wxRichTextBuffer::Dump}\label{wxrichtextbufferdump}

\func{void}{Dump}{\void}

\func{void}{Dump}{\param{wxTextOutputStream\& }{stream}}

Dumps the contents of the buffer for debugging purposes.

\membersection{wxRichTextBuffer::EndAlignment}\label{wxrichtextbufferendalignment}

\func{bool}{EndAlignment}{\void}

Ends alignment.

\membersection{wxRichTextBuffer::EndAllStyles}\label{wxrichtextbufferendallstyles}

\func{bool}{EndAllStyles}{\void}

Ends all styles that have been started with a Begin... command.

\membersection{wxRichTextBuffer::EndBatchUndo}\label{wxrichtextbufferendbatchundo}

\func{bool}{EndBatchUndo}{\void}

Ends collapsing undo/redo commands, and submits the combined command.

\membersection{wxRichTextBuffer::EndBold}\label{wxrichtextbufferendbold}

\func{bool}{EndBold}{\void}

Ends using bold.

\membersection{wxRichTextBuffer::EndCharacterStyle}\label{wxrichtextbufferendcharacterstyle}

\func{bool}{EndCharacterStyle}{\void}

Ends using the named character style.

\membersection{wxRichTextBuffer::EndFont}\label{wxrichtextbufferendfont}

\func{bool}{EndFont}{\void}

Ends using a font.

\membersection{wxRichTextBuffer::EndFontSize}\label{wxrichtextbufferendfontsize}

\func{bool}{EndFontSize}{\void}

Ends using a point size.

\membersection{wxRichTextBuffer::EndItalic}\label{wxrichtextbufferenditalic}

\func{bool}{EndItalic}{\void}

Ends using italic.

\membersection{wxRichTextBuffer::EndLeftIndent}\label{wxrichtextbufferendleftindent}

\func{bool}{EndLeftIndent}{\void}

Ends using a left indent.

\membersection{wxRichTextBuffer::EndLineSpacing}\label{wxrichtextbufferendlinespacing}

\func{bool}{EndLineSpacing}{\void}

Ends using a line spacing.

\membersection{wxRichTextBuffer::EndListStyle}\label{wxrichtextbufferendliststyle}

\func{bool}{EndListStyle}{\void}

Ends using a specified list style.

\membersection{wxRichTextBuffer::EndNumberedBullet}\label{wxrichtextbufferendnumberedbullet}

\func{bool}{EndNumberedBullet}{\void}

Ends a numbered bullet.

\membersection{wxRichTextBuffer::EndParagraphSpacing}\label{wxrichtextbufferendparagraphspacing}

\func{bool}{EndParagraphSpacing}{\void}

Ends paragraph spacing.

\membersection{wxRichTextBuffer::EndParagraphStyle}\label{wxrichtextbufferendparagraphstyle}

\func{bool}{EndParagraphStyle}{\void}

Ends applying a named character style.

\membersection{wxRichTextBuffer::EndRightIndent}\label{wxrichtextbufferendrightindent}

\func{bool}{EndRightIndent}{\void}

Ends using a right indent.

\membersection{wxRichTextBuffer::EndStyle}\label{wxrichtextbufferendstyle}

\func{bool}{EndStyle}{\void}

Ends the current style.

\membersection{wxRichTextBuffer::EndSuppressUndo}\label{wxrichtextbufferendsuppressundo}

\func{bool}{EndSuppressUndo}{\void}

Ends suppressing undo/redo commands.

\membersection{wxRichTextBuffer::EndSymbolBullet}\label{wxrichtextbufferendsymbolbullet}

\func{bool}{EndSymbolBullet}{\void}

Ends using a symbol bullet.

\membersection{wxRichTextBuffer::EndStandardBullet}\label{wxrichtextbufferendstandardbullet}

\func{bool}{EndStandardBullet}{\void}

Ends using a standard bullet.

\membersection{wxRichTextBuffer::EndTextColour}\label{wxrichtextbufferendtextcolour}

\func{bool}{EndTextColour}{\void}

Ends using a text foreground colour.

\membersection{wxRichTextBuffer::EndUnderline}\label{wxrichtextbufferendunderline}

\func{bool}{EndUnderline}{\void}

Ends using underline.

\membersection{wxRichTextBuffer::EndURL}\label{wxrichtextbufferendurl}

\func{bool}{EndURL}{\void}

Ends applying a URL.

\membersection{wxRichTextBuffer::FindHandler}\label{wxrichtextbufferfindhandler}

\func{wxRichTextFileHandler*}{FindHandler}{\param{int }{imageType}}

Finds a handler by type.

\func{wxRichTextFileHandler*}{FindHandler}{\param{const wxString\& }{extension}, \param{int }{imageType}}

Finds a handler by extension and type.

\func{wxRichTextFileHandler*}{FindHandler}{\param{const wxString\& }{name}}

Finds a handler by name.

\membersection{wxRichTextBuffer::FindHandlerFilenameOrType}\label{wxrichtextbufferfindhandlerfilenameortype}

\func{wxRichTextFileHandler*}{FindHandlerFilenameOrType}{\param{const wxString\& }{filename}, \param{int }{imageType}}

Finds a handler by filename or, if supplied, type.

\membersection{wxRichTextBuffer::GetBasicStyle}\label{wxrichtextbuffergetbasicstyle}

\constfunc{const wxTextAttrEx\&}{GetBasicStyle}{\void}

Gets the basic (overall) style. This is the style of the whole
buffer before further styles are applied, unlike the default style, which
only affects the style currently being applied (for example, setting the default
style to bold will cause subsequently inserted text to be bold).

\membersection{wxRichTextBuffer::GetBatchedCommand}\label{wxrichtextbuffergetbatchedcommand}

\constfunc{wxRichTextCommand*}{GetBatchedCommand}{\void}

Gets the collapsed command.

\membersection{wxRichTextBuffer::GetCommandProcessor}\label{wxrichtextbuffergetcommandprocessor}

\constfunc{wxCommandProcessor*}{GetCommandProcessor}{\void}

Gets the command processor. A text buffer always creates its own command processor when it is
initialized.

\membersection{wxRichTextBuffer::GetDefaultStyle}\label{wxrichtextbuffergetdefaultstyle}

\constfunc{const wxTextAttrEx\&}{GetDefaultStyle}{\void}

Returns the current default style, affecting the style currently being applied (for example, setting the default
style to bold will cause subsequently inserted text to be bold).

\membersection{wxRichTextBuffer::GetExtWildcard}\label{wxrichtextbuffergetextwildcard}

\func{wxString}{GetExtWildcard}{\param{bool }{combine = false}, \param{bool }{save = false}, \param{wxArrayInt* }{types = NULL}}

Gets a wildcard incorporating all visible handlers. If {\it types} is present,
it will be filled with the file type corresponding to each filter. This can be
used to determine the type to pass to \helpref{LoadFile}{wxrichtextbuffergetextwildcard} given a selected filter.

\membersection{wxRichTextBuffer::GetHandlers}\label{wxrichtextbuffergethandlers}

\func{wxList\&}{GetHandlers}{\void}

Returns the list of file handlers.

\membersection{wxRichTextBuffer::GetRenderer}\label{wxrichtextbuffergetrenderer}

\func{static wxRichTextRenderer*}{GetRenderer}{\void}

Returns the object to be used to render certain aspects of the content, such as bullets.

\membersection{wxRichTextBuffer::GetStyle}\label{wxrichtextbuffergetstyle}

\func{bool}{GetStyle}{\param{long }{position}, \param{wxRichTextAttr\& }{style}}

\func{bool}{GetStyle}{\param{long }{position}, \param{wxTextAttrEx\& }{style}}

Gets the attributes at the given position.

This function gets the combined style - that is, the style you see on the screen as a result
of combining base style, paragraph style and character style attributes. To get the character
or paragraph style alone, use \helpref{GetUncombinedStyle}{wxrichtextbuffergetuncombinedstyle}.

\membersection{wxRichTextBuffer::GetStyleForRange}\label{wxrichtextbuffergetstyleforrange}

\func{bool}{GetStyleForRange}{\param{const wxRichTextRange\&}{ range}, \param{wxTextAttrEx\& }{style}}

This function gets a style representing the common, combined attributes in the given range.
Attributes which have different values within the specified range will not be included the style
flags.

The function is used to get the attributes to display in the formatting dialog: the user
can edit the attributes common to the selection, and optionally specify the values of further
attributes to be applied uniformly.

To apply the edited attributes, you can use \helpref{SetStyle}{wxrichtextbuffersetstyle} specifying
the wxRICHTEXT\_SETSTYLE\_OPTIMIZE flag, which will only apply attributes that are different
from the {\it combined} attributes within the range. So, the user edits the effective, displayed attributes
for the range, but his choice won't be applied unnecessarily to content. As an example,
say the style for a paragraph specifies bold, but the paragraph text doesn't specify a weight. The
combined style is bold, and this is what the user will see on-screen and in the formatting
dialog. The user now specifies red text, in addition to bold. When applying with
SetStyle, the content font weight attributes won't be changed to bold because this is already specified
by the paragraph. However the text colour attributes {\it will} be changed to
show red.

\membersection{wxRichTextBuffer::GetStyleSheet}\label{wxrichtextbuffergetstylesheet}

\constfunc{wxRichTextStyleSheet*}{GetStyleSheet}{\void}

Returns the current style sheet associated with the buffer, if any.

\membersection{wxRichTextBuffer::GetStyleStackSize}\label{wxrichtextbuffergetstylestacksize}

\constfunc{size\_t}{GetStyleStackSize}{\void}

Get the size of the style stack, for example to check correct nesting.

\membersection{wxRichTextBuffer::GetUncombinedStyle}\label{wxrichtextbuffergetuncombinedstyle}

\func{bool}{GetUncombinedStyle}{\param{long }{position}, \param{wxRichTextAttr\& }{style}}

\func{bool}{GetUncombinedStyle}{\param{long }{position}, \param{wxTextAttrEx\& }{style}}

Gets the attributes at the given position.

This function gets the {\it uncombined style} - that is, the attributes associated with the
paragraph or character content, and not necessarily the combined attributes you see on the
screen. To get the combined attributes, use \helpref{GetStyle}{wxrichtextbuffergetstyle}.

If you specify (any) paragraph attribute in {\it style}'s flags, this function will fetch
the paragraph attributes. Otherwise, it will return the character attributes.

\membersection{wxRichTextBuffer::HitTest}\label{wxrichtextbufferhittest}

\func{int}{HitTest}{\param{wxDC\& }{dc}, \param{const wxPoint\& }{pt}, \param{long\& }{textPosition}}

Finds the text position for the given position, putting the position in {\it textPosition} if
one is found. {\it pt} is in logical units (a zero y position is
at the beginning of the buffer).

The function returns one of the following values:

{\small
\begin{verbatim}
// The point was not on this object
#define wxRICHTEXT_HITTEST_NONE     0x01
// The point was before the position returned from HitTest
#define wxRICHTEXT_HITTEST_BEFORE   0x02
// The point was after the position returned from HitTest
#define wxRICHTEXT_HITTEST_AFTER    0x04
// The point was on the position returned from HitTest
#define wxRICHTEXT_HITTEST_ON       0x08
// The point was on space outside content
#define wxRICHTEXT_HITTEST_OUTSIDE  0x10
\end{verbatim}
}

\membersection{wxRichTextBuffer::Init}\label{wxrichtextbufferinit}

\func{void}{Init}{\void}

Initialisation.

\membersection{wxRichTextBuffer::InitStandardHandlers}\label{wxrichtextbufferinitstandardhandlers}

\func{void}{InitStandardHandlers}{\void}

Initialises the standard handlers. Currently, only the plain text loading/saving handler
is initialised by default.

\membersection{wxRichTextBuffer::InsertHandler}\label{wxrichtextbufferinserthandler}

\func{void}{InsertHandler}{\param{wxRichTextFileHandler* }{handler}}

Inserts a handler at the front of the list.

\membersection{wxRichTextBuffer::InsertImageWithUndo}\label{wxrichtextbufferinsertimagewithundo}

\func{bool}{InsertImageWithUndo}{\param{long }{pos}, \param{const wxRichTextImageBlock\& }{imageBlock}, \param{wxRichTextCtrl* }{ctrl}}

Submits a command to insert the given image.

\membersection{wxRichTextBuffer::InsertNewlineWithUndo}\label{wxrichtextbufferinsertnewlinewithundo}

\func{bool}{InsertNewlineWithUndo}{\param{long }{pos}, \param{wxRichTextCtrl* }{ctrl}}

Submits a command to insert a newline.

\membersection{wxRichTextBuffer::InsertTextWithUndo}\label{wxrichtextbufferinserttextwithundo}

\func{bool}{InsertTextWithUndo}{\param{long }{pos}, \param{const wxString\& }{text}, \param{wxRichTextCtrl* }{ctrl}}

Submits a command to insert the given text.

\membersection{wxRichTextBuffer::IsModified}\label{wxrichtextbufferismodified}

\constfunc{bool}{IsModified}{\void}

Returns \true if the buffer has been modified.

\membersection{wxRichTextBuffer::LoadFile}\label{wxrichtextbufferloadfile}

\func{bool}{LoadFile}{\param{wxInputStream\& }{stream}, \param{int }{type = wxRICHTEXT\_TYPE\_ANY}}

Loads content from a stream.

\func{bool}{LoadFile}{\param{const wxString\& }{filename}, \param{int }{type = wxRICHTEXT\_TYPE\_ANY}}

Loads content from a file.

\membersection{wxRichTextBuffer::Modify}\label{wxrichtextbuffermodify}

\func{void}{Modify}{\param{bool }{modify = true}}

Marks the buffer as modified or unmodified.

\membersection{wxRichTextBuffer::NumberList}\label{wxrichtextbuffernumberlist}

\func{bool}{NumberList}{\param{const wxRichTextRange\& }{range}, \param{const wxRichTextListStyleDefinition* }{style}, \param{int }{flags $=$ wxRICHTEXT\_SETSTYLE\_WITH\_UNDO}, \param{int}{ startFrom = -1}, \param{int}{ listLevel = -1}}

\func{bool}{Number}{\param{const wxRichTextRange\& }{range}, \param{const wxString\& }{styleName}, \param{int }{flags $=$ wxRICHTEXT\_SETSTYLE\_WITH\_UNDO}, \param{int}{ startFrom = -1}, \param{int}{ listLevel = -1}}

Numbers the paragraphs in the given range. Pass flags to determine how the attributes are set.
Either the style definition or the name of the style definition (in the current sheet) can be passed.

{\it flags} is a bit list of the following:

\begin{itemize}\itemsep=0pt
\item wxRICHTEXT\_SETSTYLE\_WITH\_UNDO: specifies that this command will be undoable.
\item wxRICHTEXT\_SETSTYLE\_RENUMBER: specifies that numbering should start from {\it startFrom}, otherwise existing attributes are used.
\item wxRICHTEXT\_SETSTYLE\_SPECIFY\_LEVEL: specifies that {\it listLevel} should be used as the level for all paragraphs, otherwise the current indentation will be used.
\end{itemize}

See also \helpref{wxRichTextBuffer::SetListStyle}{wxrichtextbuffersetliststyle}, \helpref{wxRichTextBuffer::PromoteList}{wxrichtextbufferpromotelist}, \helpref{wxRichTextBuffer::ClearListStyle}{wxrichtextbufferclearliststyle}.

\membersection{wxRichTextBuffer::PasteFromClipboard}\label{wxrichtextbufferpastefromclipboard}

\func{bool}{PasteFromClipboard}{\param{long }{position}}

Pastes the clipboard content to the buffer at the given position.

\membersection{wxRichTextBuffer::PromoteList}\label{wxrichtextbufferpromotelist}

\func{bool}{PromoteList}{\param{int}{ promoteBy}, \param{const wxRichTextRange\& }{range}, \param{const wxRichTextListStyleDefinition* }{style}, \param{int }{flags $=$ wxRICHTEXT\_SETSTYLE\_WITH\_UNDO}, \param{int}{ listLevel = -1}}

\func{bool}{PromoteList}{\param{int}{ promoteBy}, \param{const wxRichTextRange\& }{range}, \param{const wxString\& }{styleName}, \param{int }{flags $=$ wxRICHTEXT\_SETSTYLE\_WITH\_UNDO}, \param{int}{ listLevel = -1}}

Promotes or demotes the paragraphs in the given range. A positive {\it promoteBy} produces a smaller indent, and a negative number
produces a larger indent. Pass flags to determine how the attributes are set.
Either the style definition or the name of the style definition (in the current sheet) can be passed.

{\it flags} is a bit list of the following:

\begin{itemize}\itemsep=0pt
\item wxRICHTEXT\_SETSTYLE\_WITH\_UNDO: specifies that this command will be undoable.
\item wxRICHTEXT\_SETSTYLE\_RENUMBER: specifies that numbering should start from {\it startFrom}, otherwise existing attributes are used.
\item wxRICHTEXT\_SETSTYLE\_SPECIFY\_LEVEL: specifies that {\it listLevel} should be used as the level for all paragraphs, otherwise the current indentation will be used.
\end{itemize}

See also \helpref{wxRichTextBuffer::SetListStyle}{wxrichtextbuffersetliststyle}, See also \helpref{wxRichTextBuffer::SetListStyle}{wxrichtextbuffernumberlist}, \helpref{wxRichTextBuffer::ClearListStyle}{wxrichtextbufferclearliststyle}.

\membersection{wxRichTextBuffer::RemoveEventHandler}\label{wxrichtextbufferremoveeventhandler}

\func{bool}{RemoveEventHandler}{\param{wxEvtHandler* }{handler}, \param{bool}{ deleteHandler = false}}

Removes an event handler from the buffer's list of handlers, deleting the object if {\it deleteHandler} is \true.

\membersection{wxRichTextBuffer::RemoveHandler}\label{wxrichtextbufferremovehandler}

\func{bool}{RemoveHandler}{\param{const wxString\& }{name}}

Removes a handler.

\membersection{wxRichTextBuffer::ResetAndClearCommands}\label{wxrichtextbufferresetandclearcommands}

\func{void}{ResetAndClearCommands}{\void}

Clears the buffer, adds a new blank paragraph, and clears the command history.

\membersection{wxRichTextBuffer::SaveFile}\label{wxrichtextbuffersavefile}

\func{bool}{SaveFile}{\param{wxOutputStream\& }{stream}, \param{int }{type = wxRICHTEXT\_TYPE\_ANY}}

Saves content to a stream.

\func{bool}{SaveFile}{\param{const wxString\& }{filename}, \param{int }{type = wxRICHTEXT\_TYPE\_ANY}}

Saves content to a file.

\membersection{wxRichTextBuffer::SetBasicStyle}\label{wxrichtextbuffersetbasicstyle}

\func{void}{SetBasicStyle}{\param{const wxRichTextAttr\& }{style}}

\func{void}{SetBasicStyle}{\param{const wxTextAttrEx\& }{style}}

Sets the basic (overall) style. This is the style of the whole
buffer before further styles are applied, unlike the default style, which
only affects the style currently being applied (for example, setting the default
style to bold will cause subsequently inserted text to be bold).

\membersection{wxRichTextBuffer::SetDefaultStyle}\label{wxrichtextbuffersetdefaultstyle}

\func{void}{SetDefaultStyle}{\param{const wxTextAttrEx\& }{style}}

Sets the default style, affecting the style currently being applied (for example, setting the default
style to bold will cause subsequently inserted text to be bold).

This is not cumulative - setting the default style will replace the previous default style.

\membersection{wxRichTextBuffer::SetListStyle}\label{wxrichtextbuffersetliststyle}

\func{bool}{SetListStyle}{\param{const wxRichTextRange\& }{range}, \param{const wxRichTextListStyleDefinition* }{style}, \param{int }{flags $=$ wxRICHTEXT\_SETSTYLE\_WITH\_UNDO}, \param{int}{ startFrom = -1}, \param{int}{ listLevel = -1}}

\func{bool}{SetListStyle}{\param{const wxRichTextRange\& }{range}, \param{const wxString\& }{styleName}, \param{int }{flags $=$ wxRICHTEXT\_SETSTYLE\_WITH\_UNDO}, \param{int}{ startFrom = -1}, \param{int}{ listLevel = -1}}

Sets the list attributes for the given range, passing flags to determine how the attributes are set.
Either the style definition or the name of the style definition (in the current sheet) can be passed.

{\it flags} is a bit list of the following:

\begin{itemize}\itemsep=0pt
\item wxRICHTEXT\_SETSTYLE\_WITH\_UNDO: specifies that this command will be undoable.
\item wxRICHTEXT\_SETSTYLE\_RENUMBER: specifies that numbering should start from {\it startFrom}, otherwise existing attributes are used.
\item wxRICHTEXT\_SETSTYLE\_SPECIFY\_LEVEL: specifies that {\it listLevel} should be used as the level for all paragraphs, otherwise the current indentation will be used.
\end{itemize}

See also \helpref{wxRichTextBuffer::NumberList}{wxrichtextbuffernumberlist}, \helpref{wxRichTextBuffer::PromoteList}{wxrichtextbufferpromotelist}, \helpref{wxRichTextBuffer::ClearListStyle}{wxrichtextbufferclearliststyle}.

\membersection{wxRichTextBuffer::SetRenderer}\label{wxrichtextbuffersetrenderer}

\func{static void}{SetRenderer}{\param{wxRichTextRenderer* }{renderer}}

Sets {\it renderer} as the object to be used to render certain aspects of the content, such as bullets.
You can override default rendering by deriving a new class from wxRichTextRenderer or wxRichTextStdRenderer,
overriding one or more virtual functions, and setting an instance of the class using this function.

\membersection{wxRichTextBuffer::SetStyle}\label{wxrichtextbuffersetstyle}

\func{bool}{SetStyle}{\param{const wxRichTextRange\& }{range}, \param{const wxRichTextAttr\& }{style}, \param{int }{flags $=$ wxRICHTEXT\_SETSTYLE\_WITH\_UNDO}}

\func{bool}{SetStyle}{\param{const wxRichTextRange\& }{range}, \param{const wxTextAttrEx\& }{style}, \param{int }{flags $=$ wxRICHTEXT\_SETSTYLE\_WITH\_UNDO}}

Sets the attributes for the given range. Pass flags to determine how the attributes are set.

The end point of range is specified as the last character position of the span of text.
So, for example, to set the style for a character at position 5, use the range (5,5).
This differs from the wxRichTextCtrl API, where you would specify (5,6).

{\it flags} may contain a bit list of the following values:

\begin{itemize}\itemsep=0pt
\item wxRICHTEXT\_SETSTYLE\_NONE: no style flag.
\item wxRICHTEXT\_SETSTYLE\_WITH\_UNDO: specifies that this operation should be undoable.
\item wxRICHTEXT\_SETSTYLE\_OPTIMIZE: specifies that the style should not be applied if the
combined style at this point is already the style in question.
\item wxRICHTEXT\_SETSTYLE\_PARAGRAPHS\_ONLY: specifies that the style should only be applied to paragraphs,
and not the content. This allows content styling to be preserved independently from that of e.g. a named paragraph style.
\item wxRICHTEXT\_SETSTYLE\_CHARACTERS\_ONLY: specifies that the style should only be applied to characters,
and not the paragraph. This allows content styling to be preserved independently from that of e.g. a named paragraph style.
\item wxRICHTEXT\_SETSTYLE\_RESET: resets (clears) the existing style before applying the new style.
\item wxRICHTEXT\_SETSTYLE\_REMOVE: removes the specified style. Only the style flags are used in this operation.
\end{itemize}

\membersection{wxRichTextBuffer::SetStyleSheet}\label{wxrichtextbuffersetstylesheet}

\func{void}{SetStyleSheet}{\param{wxRichTextStyleSheet* }{styleSheet}}

Sets the current style sheet, if any. This will allow the application to use
named character and paragraph styles found in the style sheet.

\membersection{wxRichTextBuffer::SubmitAction}\label{wxrichtextbuffersubmitaction}

\func{bool}{SubmitAction}{\param{wxRichTextAction* }{action}}

Submit an action immediately, or delay it according to whether collapsing is on.

\membersection{wxRichTextBuffer::SuppressingUndo}\label{wxrichtextbuffersuppressingundo}

\constfunc{bool}{SuppressingUndo}{\void}

Returns \true if undo suppression is currently on.

