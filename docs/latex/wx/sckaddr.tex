% ----------------------------------------------------------------------------
% CLASS: wxSockAddress
% ----------------------------------------------------------------------------
\section{\class{wxSockAddress}}\label{wxsockaddress}

\wxheading{Derived from}

\helpref{wxObject}{wxobject}

{\bf Actually, you don't have to use these functions: only wxSocketBase use it.}

\wxheading{See also}

\helpref{wxSocketBase}{wxsocketbase}\\
\helpref{wxIPV4address}{wxipv4address}\\
\helpref{wxIPV6address}{wxipv6address}\\
\helpref{wxunixaddress}{wxunixaddress}

% ----------------------------------------------------------------------------
% Members
% ----------------------------------------------------------------------------

\latexignore{\rtfignore{\wxheading{Members}}}

%
% ctor/dtor
%

\membersection{wxSockAddress::wxSockAddress}
\func{}{wxSockAddress}{\void}

Default constructor.

\membersection{wxSockAddress::\destruct{wxSockAddress}}
\func{}{\destruct{wxSockAddress}}{\void}

Default destructor.

%
% Clear
%
\membersection{wxSockAddress::Clear}
\func{void}{Clear}{\void}

Delete all informations about the address.

%
% Build
%
\membersection{wxSockAddress::Build}
\func{void}{Build}{\param{struct sockaddr *\&}{ addr}, \param{size\_t\&}{ len}}

Build a coded socket address.

%
% Disassemble
%
\membersection{wxSockAddress::Disassemble}
\func{void}{Disassemble}{\param{struct sockaddr *}{addr}, \param{size\_t}{ len}}

Decode a socket address. {\bf Actually, you don't have to use this
function: only wxSocketBase use it.}

%
% SockAddrLen
%
\membersection{wxSockAddress::SockAddrLen}
\func{int}{SockAddrLen}{\void};

Returns the length of the socket address.

% ----------------------------------------------------------------------------
% CLASS: wxIPV4address
% ----------------------------------------------------------------------------
\section{\class{wxIPV4address}}\label{wxipv4address}

\wxheading{Derived from}

\helpref{wxSockAddress}{wxsockaddress}

% ----------------------------------------------------------------------------
% MEMBERS
% ----------------------------------------------------------------------------

\latexignore{\rtfignore{\wxheading{Members}}}

%
% Hostname
%

\membersection{wxIPV4address::Hostname}
\func{bool}{Hostname}{\param{const wxString\&}{ hostname}}

Use the specified {\it hostname} for the address.

\wxheading{Return value}

Returns FALSE if something bad happens (invalid hostname, invalid IP address). 

%
% Hostname
%

\membersection{wxIPV4address::Hostname}
\func{wxString}{Hostname}{\void}

Returns the hostname which matches the IP address.

%
% Service
%

\membersection{wxIPV4address::Service}
\func{bool}{Service}{\param{const wxString\&}{ service}}

Use the specified {\it service} string for the address.

\wxheading{Return value}

Returns FALSE if something bad happens (invalid service).

%
% Service
%

\membersection{wxIPV4address::Service}
\func{bool}{Service}{\param{unsigned short}{ service}}

Use the specified {\it service} for the address.

\wxheading{Return value}

Returns FALSE if something bad happens (invalid service).

%
% Service
%

\membersection{wxIPV4address::Service}
\func{unsigned short}{Service}{\void}

Returns the current service.

%
% LocalHost
%

\membersection{wxIPV4address::LocalHost}
\func{bool}{LocalHost}{\void}

Initialize peer host to local host.

\wxheading{Return value}

Returns FALSE if something bad happens.
