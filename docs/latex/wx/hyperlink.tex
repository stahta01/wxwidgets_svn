%%%%%%%%%%%%%%%%%%%%%%%%%%%%%%%%%%%%%%%%%%%%%%%%%%%%%%%%%%%%%%%%%%%%%%%%%%%%%%%
%% Name:        hyperlink.tex
%% Purpose:     wxHyperlinkCtrl documentation
%% Author:      Otto Wyss
%% Modified by: Francesco Montorsi
%% Created:     25.4.2004
%% RCS-ID:      $Id$
%% Copyright:   (c) 2004 wxCode, Francesco Montorsi
%% License:     wxWindows
%%%%%%%%%%%%%%%%%%%%%%%%%%%%%%%%%%%%%%%%%%%%%%%%%%%%%%%%%%%%%%%%%%%%%%%%%%%%%%%

\section{\class{wxHyperlinkCtrl}}\label{wxhyperlinkctrl}

This class shows a static text element which links to an URL.
Appearance and behaviour is completely customizable. In fact, when the user
clicks on the hyperlink, a \helpref{wxHyperlinkEvent}{wxhyperlinkevent} is
sent but if that event is not handled (or it's skipped; see
\helpref{wxEvent::Skip}{wxeventskip}), then a call to
\helpref{wxLaunchDefaultBrowser}{wxlaunchdefaultbrowser} is done with the
hyperlink's URL.

Note that standard \helpref{wxWindow}{wxwindow} functions like \helpref{SetBackgroundColour}{wxwindowsetbackgroundcolour}, \helpref{SetFont}{wxwindowsetfont}, \helpref{SetCursor}{wxwindowsetcursor}, \helpref{SetLabel}{wxwindowsetlabel} can be used to customize appearance of the hyperlink.


\wxheading{Derived from}

\helpref{wxControl}{wxcontrol}


\wxheading{Include files}

<wx/hyperlink.h>


\wxheading{Window styles}

\twocolwidtha{7cm}
\begin{twocollist}\itemsep=0pt
\twocolitem{\windowstyle{wxHL\_CONTEXTMENU}}{Pop up a context menu when the hyperlink is right-clicked. The context menu contains a \texttt{``Copy URL"} menu item which is automatically handled by the hyperlink and which just copies in the clipboard the URL (not the label) of the control.}
\twocolitem{\windowstyle{wxHL\_DEFAULT\_STYLE}}{The default style for wxHyperlinkCtrl: \texttt{wxNO\_BORDER|wxHL\_CONTEXTMENU}.}
\end{twocollist}

See also \helpref{window styles overview}{windowstyles}.


\wxheading{Event handling}

To process input from an hyperlink control, use these event handler macros to
direct input to member functions that take a
\helpref{wxHyperlinkEvent}{wxhyperlinkevent} argument.

\twocolwidtha{7cm}
\begin{twocollist}\itemsep=0pt
\twocolitem{{\bf EVT\_HYPERLINK(id, func)}}{The hyperlink was (left) clicked. If this event is not handled in user's code (or it's skipped; see \helpref{wxEvent::Skip}{wxeventskip}), then a call to \helpref{wxLaunchDefaultBrowser}{wxlaunchdefaultbrowser} is done with the hyperlink's URL.}
\end{twocollist}


\wxheading{See also}

\helpref{wxURL}{wxurl}, \helpref{wxHyperlinkEvent}{wxhyperlinkevent}


\latexignore{\rtfignore{\wxheading{Members}}}

\membersection{wxHyperlinkCtrl::wxHyperLink}\label{wxhyperlinkctrlctor}

\func{}{wxHyperLink}{\param{wxWindow*}{ parent}, \param{wxWindowID}{ id}, \param{const wxString \&}{ label}, \param{const wxString \&}{ url}, \param{const wxPoint\&}{ pos = wxDefaultPosition}, \param{const wxSize\&}{ size = wxDefaultSize}, \param{long}{ style}, \param{const wxString\& }{name = ``hyperlink"}}

Constructor. See \helpref{Create}{wxhyperlinkctrlcreate} for more info.


\membersection{wxHyperlinkCtrl::Create}\label{wxhyperlinkctrlcreate}

\func{bool}{Create}{\param{wxWindow*}{ parent}, \param{wxWindowID}{ id}, \param{const wxString \&}{ label}, \param{const wxString \&}{ url}, \param{const wxPoint\&}{ pos = wxDefaultPosition}, \param{const wxSize\&}{ size = wxDefaultSize}, \param{long}{ style}, \param{const wxString\& }{name = ``hyperlink"}}

Creates the hyperlink control.

\wxheading{Parameters}

\docparam{parent}{Parent window. Must not be \NULL.}

\docparam{id}{Window identifier. A value of wxID\_ANY indicates a default value.}

\docparam{label}{The label of the hyperlink.}

\docparam{url}{The URL associated with the given label.}

\docparam{pos}{Window position.}

\docparam{size}{Window size. If the wxDefaultSize is specified then the window is sized
appropriately.}

\docparam{style}{Window style. See \helpref{wxHyperlinkCtrl}{wxhyperlinkctrl}.}

\docparam{validator}{Window validator.}

\docparam{name}{Window name.}


\membersection{wxHyperlinkCtrl::GetHoverColour}\label{wxhyperlinkctrlgethovercolour}

\constfunc{wxColour}{GetHoverColour}{\void}

Returns the colour used to print the label of the hyperlink when the mouse is over the control.


\membersection{wxHyperlinkCtrl::SetHoverColour}\label{wxhyperlinkctrlsethovercolour}

\func{void}{SetHoverColour}{\param{const wxColour \&}{ colour}}

Sets the colour used to print the label of the hyperlink when the mouse is over the control.


\membersection{wxHyperlinkCtrl::GetNormalColour}\label{wxhyperlinkctrlgetnormalcolour}

\constfunc{wxColour}{GetNormalColour}{\void}

Returns the colour used to print the label when the link has never been clicked before
(i.e. the link has not been {\it visited}) and the mouse is not over the control.


\membersection{wxHyperlinkCtrl::SetNormalColour}\label{wxhyperlinkctrlsetnormalcolour}

\func{void}{SetNormalColour}{\param{const wxColour \&}{ colour}}

Sets the colour used to print the label when the link has never been clicked before
(i.e. the link has not been {\it visited}) and the mouse is not over the control.


\membersection{wxHyperlinkCtrl::GetVisitedColour}\label{wxhyperlinkctrlgetvisitedcolour}

\constfunc{wxColour}{GetVisitedColour}{\void}

Returns the colour used to print the label when the mouse is not over the control
and the link has already been clicked before (i.e. the link has been {\it visited}).


\membersection{wxHyperlinkCtrl::SetVisitedColour}\label{wxhyperlinkctrlsetvisitedcolour}

\func{void}{SetVisitedColour}{\param{const wxColour \&}{ colour}}

Sets the colour used to print the label when the mouse is not over the control
and the link has already been clicked before (i.e. the link has been {\it visited}).


\membersection{wxHyperlinkCtrl::GetVisited}\label{wxhyperlinkctrlgetvisited}

\constfunc{bool}{GetVisited}{\void}

Returns \true if the hyperlink has already been clicked by the user at least one time.


\membersection{wxHyperlinkCtrl::SetVisited}\label{wxhyperlinkctrlsetvisited}

\func{void}{SetVisited}{\param{bool}{ visited = true}}

Marks the hyperlink as visited (see \helpref{SetVisitedColour}{wxhyperlinkctrlsetvisitedcolour}).


\membersection{wxHyperlinkCtrl::GetURL}\label{wxhyperlinkctrlgeturl}

\constfunc{wxString}{GetURL}{\void}

Returns the URL associated with the hyperlink.


\membersection{wxHyperlinkCtrl::SetURL}\label{wxhyperlinkctrlseturl}

\func{void}{SetURL}{\param{const wxString \&}{ url}}

Sets the URL associated with the hyperlink.






\section{\class{wxHyperlinkEvent}}\label{wxhyperlinkevent}

This event class is used for the events generated by
\helpref{wxHyperlinkCtrl}{wxhyperlinkctrl}.

\wxheading{Derived from}

\helpref{wxCommandEvent}{wxcommandevent}\\
\helpref{wxEvent}{wxevent}\\
\helpref{wxObject}{wxobject}

\wxheading{Include files}

<wx/hyperlink.h>

\wxheading{Event handling}

To process input from a wxHyperlinkCtrl, use one of these event handler macros to
direct input to member function that take a
\helpref{wxHyperlinkEvent}{wxhyperlinkevent} argument:

\twocolwidtha{7cm}
\begin{twocollist}
\twocolitem{{\bf EVT\_HYPERLINK(id, func)}}{User clicked on an hyperlink.}
\end{twocollist}


\latexignore{\rtfignore{\wxheading{Members}}}

\membersection{wxHyperlinkEvent::wxHyperlinkEvent}\label{wxhyperlinkctrlctor}

\func{}{wxHyperlinkEvent}{\param{wxObject *}{ generator}, \param{int}{ id}, \param{const wxString \&}{ url}}

The constructor is not normally used by the user code.


\membersection{wxHyperlinkEvent::GetURL}\label{wxhyperlinkctrlgeturl}

\constfunc{wxString}{GetURL}{\void}

Returns the URL of the hyperlink where the user has just clicked.


\membersection{wxHyperlinkEvent::SetURL}\label{wxhyperlinkctrlseturl}

\func{void}{SetURL}{\param{const wxString \&}{ url}}

Sets the URL associated with the event.
