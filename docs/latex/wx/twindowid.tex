\section{Window IDs overview}\label{windowidsoverview}

\wxheading{See Also}

\helpref{wxIdManager}{wxidmanager}
\helpref{wxWindow::NewControlId}{wxwindownewcontrolid}
\helpref{wxWindow::UnreserveControlId}{wxwindowunreservecontrolid}

\subsection{Introduction}\label{windowidsoverviewintro}

Various contols and other parts of wxWidgets need an ID.  Sometimes the
ID may be directly provided by the use or have a predefined value, such as
wxID_OPEN.  Often, however, the value of the ID is unimportant and is created
automatically by calling \helpref{wxWindow::NewControlId}{wxwindownewcontrolid}
or by passing wxID_ANY as the ID of an object.

There are two ways to generate an ID.  One way, is to start at a negative number,
and for each new ID, return the next smallest number.  This is fine for systems
that can used the full range of negative numbers for an ID, as this provides
more than enough IDs and it would take a very very long time to run out and
wrap around.  However, some systems can not use the full range of the ID value.
Windows, for example, can only use 16 bit IDs, and only has about 32000 possible
automatic IDs that can be generated by \helpref{wxWindow::NewControlId}{wxwindownewcontrolid}.
If the program runs long enough, depending on the program itself, using this first
method would cause the IDs to wrap around into the positive ID range and cause possible
clashes with any directly specified ID values.

The other way is to keep track of the IDs returned by \helpref{wxWindow::NewControlId}{wxwindownewcontrolid}
and don't return them again until the ID is completely free and not being used by
any other objects.  This will make sure that the ID values do not clash with one
another.  This is accomplished by keeping a reference count for each of the IDs
that can possibly be returned by \helpref{wxWindow::NewControlId}{wxwindownewcontrolid}.
Other IDs are not reference counted.

\subsection{Data types}\label{windowidsoverviewtypes}

A wxWindowID is just the integer type for a window ID.  It should be used almost
everywhere.  To help keep track of the count for the automatically generated IDs,
a new type, wxWindowIDRef exists, that can take the place of wxWindowID where needed.
When an ID is first created, it is marked as reserved.  When assigning it to a
wxWindowIDRef, the usage count of the ID is increased, or set to 1 if it is currently
reserved.  Assigning the same ID to several wxWindowIDRefs will keep track of the count.
As the wxWindowIDRef gets destroyed or its value changes, it will decrease the count
of the used ID.  When there are no more wxWindowIDRef types with the created ID, the
ID is considered free and can then be used again by \helpref{wxWindow::NewControlId}{wxwindownewcontrolid}.

If a created ID is not assigned to a wxWindowIDRef, then it remains reserved until it
is unreserved manually with \helpref{wxWindow::UnreserveControlId}{wxwindowunreservecontrolid}.
However, if it is assigned to a wxWindowIDRef, then it will be unreserved automatically
and will be considered free when the count is 0, and should NOT be manually unreserved.

wxWindowIDRef can store both automatic IDs from \helpref{wxWindow::NewControlId}{wxwindownewcontrolid}
as well as normal IDs.  Reference counting is only done for the automatic IDs.  Also,
wxWindowIDRef has conversion operators that allow it to be treated just like a wxWindowID.

\subsection{Using wxWindowIDRef}\label{windowidsoverviewusing}

A wxWindowIDRef should be used in place of a wxWindowID where you want to make sure the
ID is not created again by \helpref{wxWindow::NewControlId}{wxwindownewcontrolid}
at least until the wxWindowIDRef is destroyed, usually when the associated object is destroyed.
This is done already for windows, menu items, and tool bar items.
It should only be used in the main thread, as it is not thread safe.

