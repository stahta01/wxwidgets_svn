%%%%%%%%%%%%%%%%%%%%%%%%%%%%%%%%%%%%%%%%%%%%%%%%%%%%%%%%%%%%%%%%%%%%%%%%%%%%%%%
%% Name:        log.tex
%% Purpose:     wxLog and related classes documentation
%% Author:      Vadim Zeitlin
%% Modified by:
%% Created:     some time ago
%% RCS-ID:      $Id$
%% Copyright:   (c) 1997-2001 Vadim Zeitlin
%% License:     wxWindows license
%%%%%%%%%%%%%%%%%%%%%%%%%%%%%%%%%%%%%%%%%%%%%%%%%%%%%%%%%%%%%%%%%%%%%%%%%%%%%%%

\section{\class{wxLog}}\label{wxlog}

wxLog class defines the interface for the {\it log targets} used by wxWidgets
logging functions as explained in the \helpref{wxLog overview}{wxlogoverview}.
The only situations when you need to directly use this class is when you want
to derive your own log target because the existing ones don't satisfy your
needs. Another case is if you wish to customize the behaviour of the standard
logging classes (all of which respect the wxLog settings): for example, set
which trace messages are logged and which are not or change (or even remove
completely) the timestamp on the messages.

Otherwise, it is completely hidden behind the {\it wxLogXXX()} functions and
you may not even know about its existence.

See \helpref{log overview}{wxlogoverview} for the descriptions of wxWidgets
logging facilities.

\wxheading{Derived from}

No base class

\wxheading{Include files}

<wx/log.h>

\latexignore{\rtfignore{\wxheading{Function groups}}}

\membersection{Global functions}

The functions in this section work with and manipulate the active log target.
The \helpref{OnLog()}{wxlogonlog} is called by the {\it wxLogXXX()} functions
and invokes the \helpref{DoLog()}{wxlogdolog} of the active log target if any.
Get/Set methods are used to install/query the current active target and,
finally, \helpref{DontCreateOnDemand()}{wxlogdontcreateondemand} disables the
automatic creation of a standard log target if none actually exists. It is
only useful when the application is terminating and shouldn't be used in other
situations because it may easily lead to a loss of messages.

\helpref{OnLog}{wxlogonlog}\\
\helpref{GetActiveTarget}{wxloggetactivetarget}\\
\helpref{SetActiveTarget}{wxlogsetactivetarget}\\
\helpref{DontCreateOnDemand}{wxlogdontcreateondemand}\\
\helpref{Suspend}{wxlogsuspend}\\
\helpref{Resume}{wxlogresume}

\membersection{Logging functions}\label{loggingfunctions}

There are two functions which must be implemented by any derived class to
actually process the log messages: \helpref{DoLog}{wxlogdolog} and
\helpref{DoLogString}{wxlogdologstring}. The second function receives a string
which just has to be output in some way and the easiest way to write a new log
target is to override just this function in the derived class. If more control
over the output format is needed, then the first function must be overridden
which allows to construct custom messages depending on the log level or even
do completely different things depending on the message severity (for example,
throw away all messages except warnings and errors, show warnings on the
screen and forward the error messages to the user's (or programmer's) cell
phone - maybe depending on whether the timestamp tells us if it is day or
night in the current time zone).

There also functions to support message buffering. Why are they needed?
Some of wxLog implementations, most notably the standard wxLogGui class,
buffer the messages (for example, to avoid showing the user a zillion of modal
message boxes one after another -- which would be really annoying).
\helpref{Flush()}{wxlogflush} shows them all and clears the buffer contents.
This function doesn't do anything if the buffer is already empty.

\helpref{Flush}{wxlogflush}\\
\helpref{FlushActive}{wxlogflushactive}

\membersection{Customization}\label{wxlogcustomization}

The functions below allow some limited customization of wxLog behaviour
without writing a new log target class (which, aside of being a matter of
several minutes, allows you to do anything you want).

The verbose messages are the trace messages which are not disabled in the
release mode and are generated by \helpref{wxLogVerbose}{wxlogverbose}. They
are not normally shown to the user because they present little interest, but
may be activated, for example, in order to help the user find some program
problem.

As for the (real) trace messages, their handling depends on the settings of
the (application global) {\it trace mask}. There are two ways to specify it:
either by using \helpref{SetTraceMask}{wxlogsettracemask} and
\helpref{GetTraceMask}{wxloggettracemask} and using
\helpref{wxLogTrace}{wxlogtrace} which takes an integer mask or by using
\helpref{AddTraceMask}{wxlogaddtracemask} for string trace masks.

The difference between bit-wise and string trace masks is that a message using
integer trace mask will only be logged if all bits of the mask are set in the
current mask while a message using string mask will be logged simply if the
mask had been added before to the list of allowed ones.

For example,

\begin{verbatim}
// wxTraceOleCalls is one of standard bit masks
wxLogTrace(wxTraceRefCount | wxTraceOleCalls, "Active object ref count: %d", nRef);
\end{verbatim}
will do something only if the current trace mask contains both
{\tt wxTraceRefCount} and {\tt wxTraceOle}, but

\begin{verbatim}
// wxTRACE_OleCalls is one of standard string masks
wxLogTrace(wxTRACE_OleCalls, "IFoo::Bar() called");
\end{verbatim}

will log the message if it was preceded by

\begin{verbatim}
wxLog::AddTraceMask(wxTRACE_OleCalls);
\end{verbatim}

Using string masks is simpler and allows to easily add custom ones, so this is
the preferred way of working with trace messages. The integer trace mask is
kept for compatibility and for additional (but very rarely needed) flexibility
only.

The standard trace masks are given in \helpref{wxLogTrace}{wxlogtrace}
documentation.

Finally, the {\it wxLog::DoLog()} function automatically prepends a time stamp
to all the messages. The format of the time stamp may be changed: it can be
any string with \% specifications fully described in the documentation of the
standard {\it strftime()} function. For example, the default format is
"[\%d/\%b/\%y \%H:\%M:\%S] " which gives something like "[17/Sep/98 22:10:16] "
(without quotes) for the current date. Setting an empty string as the time
format disables timestamping of the messages completely.

{\bf NB:} Timestamping is disabled for Visual C++ users in debug builds by
default because otherwise it would be impossible to directly go to the line
from which the log message was generated by simply clicking in the debugger
window on the corresponding error message. If you wish to enable it, please use
\helpref{SetTimestamp}{wxlogsettimestamp} explicitly.

\helpref{AddTraceMask}{wxlogaddtracemask}\\
\helpref{RemoveTraceMask}{wxlogremovetracemask}\\
\helpref{ClearTraceMasks}{wxlogcleartracemasks}\\
\helpref{GetTraceMasks}{wxloggettracemasks}\\
\helpref{IsAllowedTraceMask}{wxlogisallowedtracemask}\\
\helpref{SetVerbose}{wxlogsetverbose}\\
\helpref{GetVerbose}{wxloggetverbose}\\
\helpref{SetTimestamp}{wxlogsettimestamp}\\
\helpref{GetTimestamp}{wxloggettimestamp}\\
\helpref{SetTraceMask}{wxlogsettracemask}\\
\helpref{GetTraceMask}{wxloggettracemask}\\
\helpref{SetRepetitionCounting}{wxlogsetrepetitioncounting}\\
\helpref{GetRepetitionCounting}{wxloggetrepetitioncounting}

%%%%% MEMBERS HERE %%%%%
\helponly{\insertatlevel{2}{

\wxheading{Members}

}}

\membersection{wxLog::AddTraceMask}\label{wxlogaddtracemask}

\func{static void}{AddTraceMask}{\param{const wxString\& }{mask}}

Add the {\it mask} to the list of allowed masks for
\helpref{wxLogTrace}{wxlogtrace}.

\wxheading{See also}

\helpref{RemoveTraceMask}{wxlogremovetracemask}
\helpref{GetTraceMasks}{wxloggettracemasks}

\membersection{wxLog::ClearTraceMasks}\label{wxlogcleartracemasks}

\func{static void}{ClearTraceMasks}{\void}

Removes all trace masks previously set with
\helpref{AddTraceMask}{wxlogaddtracemask}.

\wxheading{See also}

\helpref{RemoveTraceMask}{wxlogremovetracemask}

\membersection{wxLog::GetTraceMasks}\label{wxloggettracemasks}

\func{static const wxArrayString \&}{GetTraceMasks}{\void}

Returns the currently allowed list of string trace masks.

\wxheading{See also}

\helpref{AddTraceMask}{wxlogaddtracemask}.

\membersection{wxLog::OnLog}\label{wxlogonlog}

\func{static void}{OnLog}{\param{wxLogLevel }{ level}, \param{const char * }{ message}}

Forwards the message at specified level to the {\it DoLog()} function of the
active log target if there is any, does nothing otherwise.

\membersection{wxLog::GetActiveTarget}\label{wxloggetactivetarget}

\func{static wxLog *}{GetActiveTarget}{\void}

Returns the pointer to the active log target (may be NULL).

\membersection{wxLog::SetActiveTarget}\label{wxlogsetactivetarget}

\func{static wxLog *}{SetActiveTarget}{\param{wxLog * }{ logtarget}}

Sets the specified log target as the active one. Returns the pointer to the
previous active log target (may be NULL).  To suppress logging use a new
instance of wxLogNull not NULL.  If the active log target is set to NULL a
new default log target will be created when logging occurs.

\membersection{wxLog::Suspend}\label{wxlogsuspend}

\func{static void}{Suspend}{\void}

Suspends the logging until \helpref{Resume}{wxlogresume} is called. Note that
the latter must be called the same number of times as the former to undo it,
i.e. if you call Suspend() twice you must call Resume() twice as well.

Note that suspending the logging means that the log sink won't be be flushed
periodically, it doesn't have any effect if the current log target does the
logging immediately without waiting for \helpref{Flush}{wxlogflush} to be
called (the standard GUI log target only shows the log dialog when it is
flushed, so Suspend() works as expected with it).

\wxheading{See also}

\helpref{Resume}{wxlogresume},\\
\helpref{wxLogNull}{wxlogoverview}

\membersection{wxLog::Resume}\label{wxlogresume}

\func{static void}{Resume}{\void}

Resumes logging previously suspended by a call to
\helpref{Suspend}{wxlogsuspend}. All messages logged in the meanwhile will be
flushed soon.

\membersection{wxLog::DoLog}\label{wxlogdolog}

\func{virtual void}{DoLog}{\param{wxLogLevel }{level}, \param{const wxChar }{*msg}, \param{time\_t }{timestamp}}

Called to process the message of the specified severity. {\it msg} is the text
of the message as specified in the call of {\it wxLogXXX()} function which
generated it and {\it timestamp} is the moment when the message was generated.

The base class version prepends the timestamp to the message, adds a prefix
corresponding to the log level and then calls
\helpref{DoLogString}{wxlogdologstring} with the resulting string.

\membersection{wxLog::DoLogString}\label{wxlogdologstring}

\func{virtual void}{DoLogString}{\param{const wxChar }{*msg}, \param{time\_t }{timestamp}}

Called to log the specified string. The timestamp is already included into the
string but still passed to this function.

A simple implementation may just send the string to {\tt stdout} or, better,
{\tt stderr}.

\membersection{wxLog::DontCreateOnDemand}\label{wxlogdontcreateondemand}

\func{static void}{DontCreateOnDemand}{\void}

Instructs wxLog to not create new log targets on the fly if there is none
currently. (Almost) for internal use only: it is supposed to be called by the
application shutdown code.

Note that this function also calls
\helpref{ClearTraceMasks}{wxlogcleartracemasks}.

\membersection{wxLog::Flush}\label{wxlogflush}

\func{virtual void}{Flush}{\void}

Shows all the messages currently in buffer and clears it. If the buffer
is already empty, nothing happens.

\membersection{wxLog::FlushActive}\label{wxlogflushactive}

\func{static void}{FlushActive}{\void}

Flushes the current log target if any, does nothing if there is none.

\wxheading{See also}

\helpref{Flush}{wxlogflush}

\membersection{wxLog::SetVerbose}\label{wxlogsetverbose}

\func{static void}{SetVerbose}{\param{bool }{ verbose = true}}

Activates or deactivates verbose mode in which the verbose messages are
logged as the normal ones instead of being silently dropped.

\membersection{wxLog::GetVerbose}\label{wxloggetverbose}

\func{static bool}{GetVerbose}{\void}

Returns whether the verbose mode is currently active.

\membersection{wxLog::SetLogLevel}\label{wxlogsetloglevel}

\func{static void}{SetLogLevel}{\param{wxLogLevel }{ logLevel}}

Specifies that log messages with $level > logLevel$ should be ignored
and not sent to the active log target.

\membersection{wxLog::GetLogLevel}\label{wxloggetloglevel}

\func{static wxLogLevel}{GetLogLevel}{\void}

Returns the current log level limit.

\membersection{wxLog::SetRepetitionCounting}\label{wxlogsetrepetitioncounting}

\func{static void}{SetRepetitionCounting}{\param{bool }{ repetCounting = true}}

Enables logging mode in which a log message is logged once, and in case exactly
the same message successively repeats one or more times, only the number of 
repetitions is logged.

\membersection{wxLog::GetRepetitionCounting}\label{wxloggetrepetitioncounting}

\func{static bool}{GetRepetitionCounting}{\void}

Returns whether the repetition counting mode is enabled.


\membersection{wxLog::SetTimestamp}\label{wxlogsettimestamp}

\func{void}{SetTimestamp}{\param{const char * }{ format}}

Sets the timestamp format prepended by the default log targets to all
messages. The string may contain any normal characters as well as \%
prefixed format specificators, see {\it strftime()} manual for details.
Passing a NULL value (not empty string) to this function disables message timestamping.

\membersection{wxLog::GetTimestamp}\label{wxloggettimestamp}

\constfunc{const char *}{GetTimestamp}{\void}

Returns the current timestamp format string.

\membersection{wxLog::SetTraceMask}\label{wxlogsettracemask}

\func{static void}{SetTraceMask}{\param{wxTraceMask }{ mask}}

Sets the trace mask, see \helpref{Customization}{wxlogcustomization}
section for details.

\membersection{wxLog::GetTraceMask}\label{wxloggettracemask}

Returns the current trace mask, see \helpref{Customization}{wxlogcustomization} section
for details.

\membersection{wxLog::IsAllowedTraceMask}\label{wxlogisallowedtracemask}

\func{static bool}{IsAllowedTraceMask}{\param{const wxChar *}{mask}}

Returns true if the {\it mask} is one of allowed masks for
\helpref{wxLogTrace}{wxlogtrace}.

See also: \helpref{AddTraceMask}{wxlogaddtracemask},
\helpref{RemoveTraceMask}{wxlogremovetracemask}

\membersection{wxLog::RemoveTraceMask}\label{wxlogremovetracemask}

\func{static void}{RemoveTraceMask}{\param{const wxString\& }{mask}}

Remove the {\it mask} from the list of allowed masks for
\helpref{wxLogTrace}{wxlogtrace}.

See also: \helpref{AddTraceMask}{wxlogaddtracemask}

%%%%%%%%%%%%%%%%%%%%%%%%%%%%%%%%% wxLogChain %%%%%%%%%%%%%%%%%%%%%%%%%%%%%%%%%

\section{\class{wxLogChain}}\label{wxlogchain}

This simple class allows to chain log sinks, that is to install a new sink but
keep passing log messages to the old one instead of replacing it completely as
\helpref{SetActiveTarget}{wxlogsetactivetarget} does.

It is especially useful when you want to divert the logs somewhere (for
example to a file or a log window) but also keep showing the error messages
using the standard dialogs as \helpref{wxLogGui}{wxlogoverview} does by default.

Example of usage:

\begin{verbatim}
wxLogChain *logChain = new wxLogChain(new wxLogStderr);

// all the log messages are sent to stderr and also processed as usually
...

// don't delete logChain directly as this would leave a dangling
// pointer as active log target, use SetActiveTarget() instead
delete wxLog::SetActiveTarget(...something else or NULL...);

\end{verbatim}

\wxheading{Derived from}

\helpref{wxLog}{wxlog}

\wxheading{Include files}

<wx/log.h>

\latexignore{\rtfignore{\wxheading{Members}}}

\membersection{wxLogChain::wxLogChain}\label{wxlogchainctor}

\func{}{wxLogChain}{\param{wxLog *}{logger}}

Sets the specified {\tt logger} (which may be {\tt NULL}) as the default log
target but the log messages are also passed to the previous log target if any.

\membersection{wxLogChain::\destruct{wxLogChain}}\label{wxlogchaindtor}

\func{}{\destruct{wxLogChain}}{\void}

Destroys the previous log target.

\membersection{wxLogChain::DetachOldLog}\label{wxlogchaindetacholdlog}

\func{void}{DetachOldLog}{\void}

Detaches the old log target so it won't be destroyed when the wxLogChain object
is destroyed.

\membersection{wxLogChain::GetOldLog}\label{wxlogchaingetoldlog}

\constfunc{wxLog *}{GetOldLog}{\void}

Returns the pointer to the previously active log target (which may be {\tt
NULL}).

\membersection{wxLogChain::IsPassingMessages}\label{wxlogchainispassingmessages}

\constfunc{bool}{IsPassingMessages}{\void}

Returns {\tt true} if the messages are passed to the previously active log
target (default) or {\tt false} if \helpref{PassMessages}{wxlogchainpassmessages}
had been called.

\membersection{wxLogChain::PassMessages}\label{wxlogchainpassmessages}

\func{void}{PassMessages}{\param{bool }{passMessages}}

By default, the log messages are passed to the previously active log target.
Calling this function with {\tt false} parameter disables this behaviour
(presumably temporarily, as you shouldn't use wxLogChain at all otherwise) and
it can be reenabled by calling it again with {\it passMessages} set to {\tt
true}.

\membersection{wxLogChain::SetLog}\label{wxlogchainsetlog}

\func{void}{SetLog}{\param{wxLog *}{logger}}

Sets another log target to use (may be {\tt NULL}). The log target specified
in the \helpref{constructor}{wxlogchainctor} or in a previous call to
this function is deleted.

This doesn't change the old log target value (the one the messages are
forwarded to) which still remains the same as was active when wxLogChain
object was created.

%%%%%%%%%%%%%%%%%%%%%%%%%%%%%%%%%% wxLogGui %%%%%%%%%%%%%%%%%%%%%%%%%%%%%%%%%%

\section{\class{wxLogGui}}\label{wxloggui}

This is the default log target for the GUI wxWidgets applications. It is passed
to \helpref{wxLog::SetActiveTarget}{wxlogsetactivetarget} at the program
startup and is deleted by wxWidgets during the program shut down.

\wxheading{Derived from}

\helpref{wxLog}{wxlog}

\wxheading{Include files}

<wx/log.h>

\latexignore{\rtfignore{\wxheading{Members}}}

\membersection{wxLogGui::wxLogGui}\label{wxlogguictor}

\func{}{wxLogGui}{\void}

Default constructor.

%%%%%%%%%%%%%%%%%%%%%%%%%%%%%%%%% wxLogNull %%%%%%%%%%%%%%%%%%%%%%%%%%%%%%%%%%

\section{\class{wxLogNull}}\label{wxlognull}

This class allows to temporarily suspend logging. All calls to the log
functions during the life time of an object of this class are just ignored.

In particular, it can be used to suppress the log messages given by wxWidgets
itself but it should be noted that it is rarely the best way to cope with this
problem as {\bf all} log messages are suppressed, even if they indicate a
completely different error than the one the programmer wanted to suppress.

For instance, the example of the overview:

{\small
\begin{verbatim}
  wxFile file;

  // wxFile.Open() normally complains if file can't be opened, we don't want it
  {
    wxLogNull logNo;
    if ( !file.Open("bar") )
      ... process error ourselves ...
  } // ~wxLogNull called, old log sink restored

  wxLogMessage("..."); // ok
\end{verbatim}
}%

would be better written as:

{\small
\begin{verbatim}
  wxFile file;

  // don't try to open file if it doesn't exist, we are prepared to deal with
  // this ourselves - but all other errors are not expected
  if ( wxFile::Exists("bar") )
  {
      // gives an error message if the file couldn't be opened
      file.Open("bar");
  }
  else
  {
      ...
  }
\end{verbatim}
}%

\wxheading{Derived from}

\helpref{wxLog}{wxlog}

\wxheading{Include files}

<wx/log.h>

\latexignore{\rtfignore{\wxheading{Members}}}

\membersection{wxLogNull::wxLogNull}\label{wxlognullctor}

\func{}{wxLogNull}{\void}

Suspends logging.

\membersection{wxLogNull::\destruct{wxLogNull}}\label{wxlognulldtor}

Resumes logging.

%%%%%%%%%%%%%%%%%%%%%%%%%%%%%% wxLogPassThrough %%%%%%%%%%%%%%%%%%%%%%%%%%%%%%

\section{\class{wxLogPassThrough}}\label{wxlogpassthrough}

A special version of \helpref{wxLogChain}{wxlogchain} which uses itself as the
new log target. Maybe more clearly, it means that this is a log target which
forwards the log messages to the previously installed one in addition to
processing them itself.

Unlike \helpref{wxLogChain}{wxlogchain} which is usually used directly as is,
this class must be derived from to implement \helpref{DoLog}{wxlogdolog}
and/or \helpref{DoLogString}{wxlogdologstring} methods.

\wxheading{Derived from}

\helpref{wxLogChain}{wxlogchain}

\wxheading{Include files}

<wx/log.h>

\latexignore{\rtfignore{\wxheading{Members}}}

\membersection{wxLogPassThrough::wxLogPassThrough}\label{wxlogpassthroughctor}

Default ctor installs this object as the current active log target.

%%%%%%%%%%%%%%%%%%%%%%%%%%%%%%%% wxLogStderr %%%%%%%%%%%%%%%%%%%%%%%%%%%%%%%%%

\section{\class{wxLogStderr}}\label{wxlogstderr}

This class can be used to redirect the log messages to a C file stream (not to
be confused with C++ streams). It is the default log target for the non-GUI
wxWidgets applications which send all the output to {\tt stderr}.

\wxheading{Derived from}

\helpref{wxLog}{wxlog}

\wxheading{Include files}

<wx/log.h>

\wxheading{See also}

\helpref{wxLogStream}{wxlogstream}

\latexignore{\rtfignore{\wxheading{Members}}}

\membersection{wxLogStderr::wxLogStderr}\label{wxlogstderrctor}

\func{}{wxLogStderr}{\param{FILE }{*fp = NULL}}

Constructs a log target which sends all the log messages to the given
{\tt FILE}. If it is {\tt NULL}, the messages are sent to {\tt stderr}.

%%%%%%%%%%%%%%%%%%%%%%%%%%%%%%%% wxLogStream %%%%%%%%%%%%%%%%%%%%%%%%%%%%%%%%%

\section{\class{wxLogStream}}\label{wxlogstream}

This class can be used to redirect the log messages to a C++ stream.

Please note that this class is only available if wxWidgets was compiled with
the standard iostream library support ({\tt wxUSE\_STD\_IOSTREAM} must be on).

\wxheading{Derived from}

\helpref{wxLog}{wxlog}

\wxheading{Include files}

<wx/log.h>

\wxheading{See also}

\helpref{wxLogStderr}{wxlogstderr},\\
\helpref{wxStreamToTextRedirector}{wxstreamtotextredirector}

\latexignore{\rtfignore{\wxheading{Members}}}

\membersection{wxLogStream::wxLogStream}\label{wxlogstreamctor}

\func{}{wxLogStream}{\param{std::ostream }{*ostr = NULL}}

Constructs a log target which sends all the log messages to the given
output stream. If it is {\tt NULL}, the messages are sent to {\tt cerr}.

%%%%%%%%%%%%%%%%%%%%%%%%%%%%%%% wxLogTextCtrl %%%%%%%%%%%%%%%%%%%%%%%%%%%%%%%%

\section{\class{wxLogTextCtrl}}\label{wxlogtextctrl}

Using these target all the log messages can be redirected to a text control.
The text control must have been created with {\tt wxTE\_MULTILINE} style by the
caller previously.

\wxheading{Derived from}

\helpref{wxLog}{wxlog}

\wxheading{Include files}

<wx/log.h>

\wxheading{See also}

\helpref{wxTextCtrl}{wxtextctrl},\\
\helpref{wxStreamToTextRedirector}{wxstreamtotextredirector}

\latexignore{\rtfignore{\wxheading{Members}}}

\membersection{wxLogTextCtrl::wxLogTextCtrl}\label{wxlogtextctrlctor}

\func{}{wxLogTextCtrl}{\param{wxTextCtrl }{*textctrl}}

Constructs a log target which sends all the log messages to the given text
control. The {\it textctrl} parameter cannot be {\tt NULL}.

%%%%%%%%%%%%%%%%%%%%%%%%%%%%%%%% wxLogWindow %%%%%%%%%%%%%%%%%%%%%%%%%%%%%%%%%

\section{\class{wxLogWindow}}\label{wxlogwindow}

This class represents a background log window: to be precise, it collects all
log messages in the log frame which it manages but also passes them on to the
log target which was active at the moment of its creation. This allows, for
example, to show all the log messages in a frame but still continue to process
them normally by showing the standard log dialog.

\wxheading{Derived from}

\helpref{wxLogPassThrough}{wxlogpassthrough}

\wxheading{Include files}

<wx/log.h>

\wxheading{See also}

\helpref{wxLogTextCtrl}{wxlogtextctrl}

\latexignore{\rtfignore{\wxheading{Members}}}

\membersection{wxLogWindow::wxLogWindow}\label{wxlogwindowctor}

\func{}{wxLogWindow}{\param{wxFrame }{*parent}, \param{const wxChar }{*title}, \param{bool }{show = {\tt true}}, \param{bool }{passToOld = {\tt true}}}

Creates the log frame window and starts collecting the messages in it.

\wxheading{Parameters}

\docparam{parent}{The parent window for the log frame, may be {\tt NULL}}

\docparam{title}{The title for the log frame}

\docparam{show}{{\tt true} to show the frame initially (default), otherwise
\helpref{wxLogWindow::Show}{wxlogwindowshow} must be called later.}

\docparam{passToOld}{{\tt true} to process the log messages normally in addition to
logging them in the log frame (default), {\tt false} to only log them in the
log frame.}

\membersection{wxLogWindow::Show}\label{wxlogwindowshow}

\func{void}{Show}{\param{bool }{show = {\tt true}}}

Shows or hides the frame.

\membersection{wxLogWindow::GetFrame}\label{wxlogwindowgetframe}

\constfunc{wxFrame *}{GetFrame}{\void}

Returns the associated log frame window. This may be used to position or resize
it but use \helpref{wxLogWindow::Show}{wxlogwindowshow} to show or hide it.

\membersection{wxLogWindow::OnFrameCreate}\label{wxlogwindowonframecreate}

\func{virtual void}{OnFrameCreate}{\param{wxFrame }{*frame}}

Called immediately after the log frame creation allowing for
any extra initializations.

\membersection{wxLogWindow::OnFrameClose}\label{wxlogwindowonframeclose}

\func{virtual bool}{OnFrameClose}{\param{wxFrame }{*frame}}

Called if the user closes the window interactively, will not be
called if it is destroyed for another reason (such as when program
exits).

Return {\tt true} from here to allow the frame to close, {\tt false} to
prevent this from happening.

\wxheading{See also}

\helpref{wxLogWindow::OnFrameDelete}{wxlogwindowonframedelete}

\membersection{wxLogWindow::OnFrameDelete}\label{wxlogwindowonframedelete}

\func{virtual void}{OnFrameDelete}{\param{wxFrame }{*frame}}

Called right before the log frame is going to be deleted: will
always be called unlike \helpref{OnFrameClose()}{wxlogwindowonframeclose}.

