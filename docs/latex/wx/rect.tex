\section{\class{wxRect}}\label{wxrect}

A class for manipulating rectangles.

\wxheading{Derived from}

None

\wxheading{Include files}

<wx/gdicmn.h>

\wxheading{See also}

\helpref{wxPoint}{wxpoint}, \helpref{wxSize}{wxsize}

\latexignore{\rtfignore{\wxheading{Members}}}

\membersection{wxRect::wxRect}

\func{}{wxRect}{\void}

Default constructor.

\func{}{wxRect}{\param{int}{ x}, \param{int}{ y}, \param{int}{ width}, \param{int}{ height}}

Creates a wxRect object from x, y, width and height values.

\func{}{wxRect}{\param{const wxPoint\&}{ topLeft}, \param{const wxPoint\&}{ bottomRight}}

Creates a wxRect object from top-left and bottom-right points.

\func{}{wxRect}{\param{const wxPoint\&}{ pos}, \param{const wxSize\&}{ size}}

Creates a wxRect object from position and size values.

\membersection{wxRect::x}

\member{int}{x}

x coordinate of the top-level corner of the rectangle.

\membersection{wxRect::y}

\member{int}{y}

y coordinate of the top-level corner of the rectangle.

\membersection{wxRect::width}

\member{int}{width}

Width member.

\membersection{wxRect::height}

\member{int}{height}

Height member.

\membersection{wxRect::Deflate}\label{wxrectdeflate}

\func{void}{Deflate}{\param{wxCoord }{dx}, \param{wxCoord }{dy}}

\func{void}{Deflate}{\param{wxCoord }{diff}}

\constfunc{wxRect}{Deflate}{\param{wxCoord }{dx}, \param{wxCoord }{dy}}

Decrease the rectangle size by {\it dx} in x direction and {\it dy} in y
direction. Both (or one of) parameters may be negative to increase the
rectngle size. This method is the opposite of \helpref{Inflate}{wxrectinflate}.

The second form uses the same {\it diff} for both {\it dx} and {\it dy}.

The first two versions modify the rectangle in place, the last one returns a
new rectangle leaving this one unchanged.

\wxheading{See also}

\helpref{Inflate}{wxrectinflate}

\membersection{wxRect::GetBottom}\label{wxrectgetbottom}

\constfunc{int}{GetBottom}{\void}

Gets the bottom point of the rectangle.

\membersection{wxRect::GetHeight}\label{wxrectgetheight}

\constfunc{int}{GetHeight}{\void}

Gets the height member.

\membersection{wxRect::GetLeft}\label{wxrectgetleft}

\constfunc{int}{GetLeft}{\void}

Gets the left point of the rectangle (the same as \helpref{wxRect::GetX}{wxrectgetx}).

\membersection{wxRect::GetPosition}\label{wxrectgetposition}

\constfunc{wxPoint}{GetPosition}{\void}

Gets the position.

\membersection{wxRect::GetRight}\label{wxrectgetright}

\constfunc{int}{GetRight}{\void}

Gets the right point of the rectangle.

\membersection{wxRect::GetSize}\label{wxrectgetsize}

\constfunc{wxSize}{GetSize}{\void}

Gets the size.

\membersection{wxRect::GetTop}\label{wxrectgettop}

\constfunc{int}{GetTop}{\void}

Gets the top point of the rectangle (the same as \helpref{wxRect::GetY}{wxrectgety}).

\membersection{wxRect::GetWidth}\label{wxrectgetwidth}

\constfunc{int}{GetWidth}{\void}

Gets the width member.

\membersection{wxRect::GetX}\label{wxrectgetx}

\constfunc{int}{GetX}{\void}

Gets the x member.

\membersection{wxRect::GetY}\label{wxrectgety}

\constfunc{int}{GetY}{\void}

Gets the y member.

\membersection{wxRect::Inflate}\label{wxrectinflate}

\func{void}{Inflate}{\param{wxCoord }{dx}, \param{wxCoord }{dy}}

\func{void}{Inflate}{\param{wxCoord }{diff}}

\constfunc{wxRect}{Inflate}{\param{wxCoord }{dx}, \param{wxCoord }{dy}}

Increase the rectangle size by {\it dx} in x direction and {\it dy} in y
direction. Both (or one of) parameters may be negative to decrease the
rectangle size.

The second form uses the same {\it diff} for both {\it dx} and {\it dy}.

The first two versions modify the rectangle in place, the last one returns a
new rectangle leaving this one unchanged.

\wxheading{See also}

\helpref{Deflate}{wxrectdeflate}

\membersection{wxRect:Inside}\label{wxrectinside}

\constfunc{bool}{Inside}{\param{int }{x}, \param{int }{y}}

\constfunc{bool}{Inside}{\param{const wxPoint\& }{pt}}

Returns {\tt TRUE} if the given point is inside the rectangle (or on its
boundary) and {\tt FALSE} otherwise.

\membersection{wxRect:Intersects}\label{wxrectintersects}

\constfunc{bool}{Intersects}{\param{const wxRect\& }{rect}}

Returns {\tt TRUE} if this rectangle has a non empty intersection with the
rectangle {\it rect} and {\tt FALSE} otherwise.

\membersection{wxRect::Offset}\label{wxrectoffset}

\func{void}{Offset}{\param{wxCoord }{dx}, \param{wxCoord }{dy}}

\func{void}{Offset}{\param{const wxPoint\& }{pt}}

Moves the rectangle by the specified offset. If {\it dx} is positive, the
rectangle is moved to the right, if {\it dy} is positive, it is moved to the
bottom, otherwise it is moved to the left or top respectively.

\membersection{wxRect::SetHeight}\label{wxrectsetheight}

\func{void}{SetHeight}{\param{int}{ height}}

Sets the height.

\membersection{wxRect::SetWidth}\label{wxrectsetwidth}

\func{void}{SetWidth}{\param{int}{ width}}

Sets the width.

\membersection{wxRect::SetX}\label{wxrectsetx}

\func{void}{SetX}{\param{int}{ x}}

Sets the x position.

\membersection{wxRect::SetY}\label{wxrectsety}

\func{void}{SetY}{\param{int}{ y}}

Sets the y position.

\membersection{wxRect::operator $=$}

\func{void}{operator $=$}{\param{const wxRect\& }{rect}}

Assignment operator.

\membersection{wxRect::operator $==$}

\func{bool}{operator $==$}{\param{const wxRect\& }{rect}}

Equality operator.

\membersection{wxRect::operator $!=$}

\func{bool}{operator $!=$}{\param{const wxRect\& }{rect}}

Inequality operator.

