\section{wxApp overview}\label{wxappoverview}

Classes: \helpref{wxApp}{wxapp}

A wxWindows application does not have a {\it main} procedure; the equivalent is the
\rtfsp\helpref{OnInit}{wxapponinit} member defined for a class derived from wxApp.\rtfsp
\rtfsp{\it OnInit} will usually create a top window as a bare minimum.

Unlike in earlier versions of wxWindows, OnInit does not return a frame. Instead it
returns a boolean value which indicates whether processing should continue (TRUE) or not (FALSE).
You call \helpref{wxApp::SetTopWindow}{wxappsettopwindow} to let wxWindows know
about the top window.

Note that the program's command line arguments, represented by {\it argc} 
and {\it argv}, are available from within wxApp member functions.

An application closes by destroying all windows. Because all frames must
be destroyed for the application to exit, it is advisable to use parent
frames wherever possible when creating new frames, so that deleting the
top level frame will automatically delete child frames. The alternative
is to explicitly delete child frames in the top-level frame's \helpref{wxCloseEvent}{wxcloseevent}\rtfsp
handler.

In emergencies the \helpref{wxExit}{wxexit} function can be called to kill the
application however normally the applications shuts down automatically, 
\helpref{see below}{wxappshutdownoverview}.

An example of defining an application follows:

\begin{verbatim}
class DerivedApp : public wxApp
{
public:
  virtual bool OnInit();
};

IMPLEMENT_APP(DerivedApp)

bool DerivedApp::OnInit()
{
  wxFrame *the_frame = new wxFrame(NULL, ID_MYFRAME, argv[0]);
  ...
  the_frame->Show(TRUE);
  SetTopWindow(the_frame);

  return TRUE;
}
\end{verbatim}

Note the use of IMPLEMENT\_APP(appClass), which allows wxWindows to dynamically create an instance of the application object
at the appropriate point in wxWindows initialization. Previous versions of wxWindows used
to rely on the creation of a global application object, but this is no longer recommended,
because required global initialization may not have been performed at application object
construction time.

You can also use DECLARE\_APP(appClass) in a header file to declare the wxGetApp function which returns
a reference to the application object.

\subsection{Application shutdown}\label{wxappshutdownoverview}

The application normally shuts down when the last of its top level windows is
closed. This is normally the expected behaviour and means that it is enough to
call \helpref{Close()}{wxwindowclose} in response to the {\tt "Exit"} menu
command if your program has a single top level window. If this behaviour is not
desirable \helpref{wxApp::SetExitOnFrameDelete}{wxappsetexitonframedelete} can
be called to change it. Note that starting from wxWindows 2.3.3 such logic
doesn't apply for the windows shown before the program enters the main loop: in
other words, you can safely show a dialog from 
\helpref{wxApp::OnInit}{wxapponinit} and not be afraid that your application
terminates when this dialog -- which is the last top level window for the
moment -- is closed.


Another aspect of the application shutdown is the \helpref{OnExit}{wxapponexit} 
which is called when the application exits but {\it before} wxWindows cleans up
its internal structures. Your should delete all wxWindows object that your
created by the time OnExit finishes. In particular, do {\bf not} destroy them
from application class' destructor!

For example, this code may crash:

\begin{verbatim}
class MyApp : public wxApp
{
 public:
    wxCHMHelpController m_helpCtrl;
    ...
};
\end{verbatim}

The reason for that is that {\tt m\_helpCtrl} is a member object and is 
thus destroyed from MyApp destructor. But MyApp object is deleted after 
wxWindows structures that wxCHMHelpController depends on were 
uninitialized! The solution is to destroy HelpCtrl in {\it OnExit}:

\begin{verbatim}
class MyApp : public wxApp
{
 public:
    wxCHMHelpController *m_helpCtrl;
    ...
};

bool MyApp::OnInit()
{
  ...
  m_helpCtrl = new wxCHMHelpController;
  ...
}

int MyApp::OnExit()
{
  delete m_helpCtrl;
  return 0;
}
\end{verbatim}
