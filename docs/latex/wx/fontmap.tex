%
% automatically generated by HelpGen from
% fontmap.h at 10/Mar/00 23:54:16
%

\section{\class{wxFontMapper}}\label{wxfontmapper}

wxFontMapper manages user-definable correspondence between logical font
names and the fonts present on the machine.

The default implementations of all functions will ask the user if they are
not capable of finding the answer themselves and store the answer in a
config file (configurable via SetConfigXXX functions). This behaviour may
be disabled by giving the value of false to "interactive" parameter.

However, the functions will always consult the config file to allow the
user-defined values override the default logic and there is no way to
disable this - which shouldn't be ever needed because if "interactive" was
never true, the config file is never created anyhow.

In case everything else fails (i.e. there is no record in config file
and "interactive" is false or user denied to choose any replacement), 
the class queries \helpref{wxEncodingConverter}{wxencodingconverter} 
for "equivalent" encodings (e.g. iso8859-2 and cp1250) and tries them.

\wxheading{Using wxFontMapper in conjunction with wxMBConv classes}

If you need to display text in encoding which is not available at
host system (see \helpref{IsEncodingAvailable}{wxfontmapperisencodingavailable}),
you may use these two classes to find font in some similar encoding
(see \helpref{GetAltForEncoding}{wxfontmappergetaltforencoding})
and convert the text to this encoding 
(\helpref{wxMBConv classes}{mbconvclasses}).

Following code snippet demonstrates it:

\begin{verbatim}
if (!wxFontMapper::Get()->IsEncodingAvailable(enc, facename))
{
   wxFontEncoding alternative;
   if (wxFontMapper::Get()->GetAltForEncoding(enc, &alternative,
                                              facename, false))
   {
       wxCSConv convFrom(wxFontMapper::Get()->GetEncodingName(enc));
       wxCSConv convTo(wxFontMapper::Get()->GetEncodingName(alternative));
       text = wxString(text.mb_str(convFrom), convTo);
   }
   else
       ...failure (or we may try iso8859-1/7bit ASCII)...
}
...display text...
\end{verbatim}


\wxheading{Derived from}

No base class

\wxheading{Include files}

<wx/fontmap.h>

\wxheading{See also}

\helpref{wxEncodingConverter}{wxencodingconverter}, 
\helpref{Writing non-English applications}{nonenglishoverview}

\latexignore{\rtfignore{\wxheading{Members}}}


\membersection{wxFontMapper::wxFontMapper}\label{wxfontmapperwxfontmapper}

\func{}{wxFontMapper}{\void}

Default ctor.

\wxheading{Note}

The preferred way of creating a wxFontMapper instance is to call 
\helpref{wxFontMapper::Get}{wxfontmapperget}.


\membersection{wxFontMapper::\destruct{wxFontMapper}}\label{wxfontmapperdtor}

\func{}{\destruct{wxFontMapper}}{\void}

Virtual dtor for a base class.


\membersection{wxFontMapper::CharsetToEncoding}\label{wxfontmappercharsettoencoding}

\func{wxFontEncoding}{CharsetToEncoding}{\param{const wxString\& }{charset}, \param{bool }{interactive = true}}

Returns the encoding for the given charset (in the form of RFC 2046) or
\texttt{wxFONTENCODING\_SYSTEM} if couldn't decode it.

Be careful when using this function with \arg{interactive} set to \true
(default value) as the function then may show a dialog box to the user which
may lead to unexpected reentrancies and may also take a significantly longer
time than a simple function call. For these reasons, it is almost always a bad
idea to call this function from the event handlers for repeatedly generated
events such as \texttt{EVT\_PAINT}.


\membersection{wxFontMapper::Get}\label{wxfontmapperget}

\func{static wxFontMapper *}{Get}{\void}

Get the current font mapper object. If there is no current object, creates
one.

\wxheading{See also}

\helpref{wxFontMapper::Set}{wxfontmapperset}


\membersection{wxFontMapper::GetAllEncodingNames}\label{wxfontmappergetallencodingnames}

\func{static const wxChar**}{GetAllEncodingNames}{\param{wxFontEncoding }{encoding}}

Returns the array of all possible names for the given encoding. The array is
\NULL-terminated. IF it isn't empty, the first name in it is the canonical
encoding name, i.e. the same string as returned by 
\helpref{GetEncodingName()}{wxfontmappergetencodingname}.


\membersection{wxFontMapper::GetAltForEncoding}\label{wxfontmappergetaltforencoding}

\func{bool}{GetAltForEncoding}{\param{wxFontEncoding }{encoding}, \param{wxNativeEncodingInfo* }{info}, \param{const wxString\& }{facename = wxEmptyString}, \param{bool }{interactive = true}}

\func{bool}{GetAltForEncoding}{\param{wxFontEncoding }{encoding}, \param{wxFontEncoding* }{alt\_encoding}, \param{const wxString\& }{facename = wxEmptyString}, \param{bool }{interactive = true}}

Find an alternative for the given encoding (which is supposed to not be
available on this system). If successful, return true and fill info
structure with the parameters required to create the font, otherwise
return false.

The first form is for wxWidgets' internal use while the second one
is better suitable for general use -- it returns wxFontEncoding which
can consequently be passed to wxFont constructor.


\membersection{wxFontMapper::GetEncoding}\label{wxfontmappergetencoding}

\func{static wxFontEncoding}{GetEncoding}{\param{size\_t }{n}}

Returns the {\it n}-th supported encoding. Together with 
\helpref{GetSupportedEncodingsCount()}{wxfontmappergetsupportedencodingscount} 
this method may be used to get all supported encodings.


\membersection{wxFontMapper::GetEncodingDescription}\label{wxfontmappergetencodingdescription}

\func{static wxString}{GetEncodingDescription}{\param{wxFontEncoding }{encoding}}

Return user-readable string describing the given encoding.


\membersection{wxFontMapper::GetEncodingFromName}\label{wxfontmappergetencodingfromname}

\func{static wxFontEncoding}{GetEncodingFromName}{\param{const wxString\& }{encoding}}

Return the encoding corresponding to the given internal name. This function is
the inverse of \helpref{GetEncodingName}{wxfontmappergetencodingname} and is
intentionally less general than 
\helpref{CharsetToEncoding}{wxfontmappercharsettoencoding}, i.e. it doesn't
try to make any guesses nor ever asks the user. It is meant just as a way of
restoring objects previously serialized using 
\helpref{GetEncodingName}{wxfontmappergetencodingname}.


\membersection{wxFontMapper::GetEncodingName}\label{wxfontmappergetencodingname}

\func{static wxString}{GetEncodingName}{\param{wxFontEncoding }{encoding}}

Return internal string identifier for the encoding (see also 
\helpref{GetEncodingDescription()}{wxfontmappergetencodingdescription})

\wxheading{See also}

\helpref{GetEncodingFromName}{wxfontmappergetencodingfromname}


\membersection{wxFontMapper::GetSupportedEncodingsCount}\label{wxfontmappergetsupportedencodingscount}

\func{static size\_t}{GetSupportedEncodingsCount}{\void}

Returns the number of the font encodings supported by this class. Together with 
\helpref{GetEncoding}{wxfontmappergetencoding} this method may be used to get
all supported encodings.


\membersection{wxFontMapper::IsEncodingAvailable}\label{wxfontmapperisencodingavailable}

\func{bool}{IsEncodingAvailable}{\param{wxFontEncoding }{encoding}, \param{const wxString\& }{facename = wxEmptyString}}

Check whether given encoding is available in given face or not.
If no facename is given, find {\it any} font in this encoding.


\membersection{wxFontMapper::SetDialogParent}\label{wxfontmappersetdialogparent}

\func{void}{SetDialogParent}{\param{wxWindow* }{parent}}

The parent window for modal dialogs.


\membersection{wxFontMapper::SetDialogTitle}\label{wxfontmappersetdialogtitle}

\func{void}{SetDialogTitle}{\param{const wxString\& }{title}}

The title for the dialogs (note that default is quite reasonable).


\membersection{wxFontMapper::Set}\label{wxfontmapperset}

\func{static wxFontMapper *}{Set}{\param{wxFontMapper *}{mapper}}

Set the current font mapper object and return previous one (may be NULL).
This method is only useful if you want to plug-in an alternative font mapper
into wxWidgets.

\wxheading{See also}

\helpref{wxFontMapper::Get}{wxfontmapperget}


\membersection{wxFontMapper::SetConfig}\label{wxfontmappersetconfig}

\func{void}{SetConfig}{\param{wxConfigBase* }{config}}

Set the config object to use (may be NULL to use default).

By default, the global one (from wxConfigBase::Get() will be used) 
and the default root path for the config settings is the string returned by
GetDefaultConfigPath().


\membersection{wxFontMapper::SetConfigPath}\label{wxfontmappersetconfigpath}

\func{void}{SetConfigPath}{\param{const wxString\& }{prefix}}

Set the root config path to use (should be an absolute path).

