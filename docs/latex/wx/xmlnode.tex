%%%%%%%%%%%%%%%%%%%%%%%%%%%%%%%%%%%%%%%%%%%%%%%%%%%%%%%%%%%%%%%%%%%%%%%%%%%%%%%
%% Name:        xmlnode.tex
%% Purpose:     wxXmlNode documentation
%% Author:      Francesco Montorsi
%% Created:     2006-04-18
%% RCS-ID:      $Id$
%% Copyright:   (c) 2006 Francesco Montorsi
%% License:     wxWindows license
%%%%%%%%%%%%%%%%%%%%%%%%%%%%%%%%%%%%%%%%%%%%%%%%%%%%%%%%%%%%%%%%%%%%%%%%%%%%%%%

\section{\class{wxXmlNode}}\label{wxxmlnode}

Represents a node in an XML document. See \helpref{wxXmlDocument}{wxxmldocument}.

Node has a name and may have content and attributes. Most common node types are 
{\tt wxXML\_TEXT\_NODE} (name and attributes are irrelevant) and 
{\tt wxXML\_ELEMENT\_NODE} (e.g. in {\tt <title>hi</title>} there is an element
with name="title", irrelevant content and one child ({\tt wxXML\_TEXT\_NODE}
with content="hi").

If \texttt{wxUSE\_UNICODE} is 0, all strings are encoded in the encoding given to
\helpref{wxXmlDocument::Load}{wxxmldocumentload} (default is UTF-8).


\wxheading{Derived from}

No base class

\wxheading{Include files}

<wx/xml/xml.h>

\wxheading{Constants}

The following are the node types supported by \helpref{wxXmlNode}{wxxmlnode}:

{\small
\begin{verbatim}
enum wxXmlNodeType
{
    wxXML_ELEMENT_NODE,
    wxXML_ATTRIBUTE_NODE,
    wxXML_TEXT_NODE,
    wxXML_CDATA_SECTION_NODE,
    wxXML_ENTITY_REF_NODE,
    wxXML_ENTITY_NODE,
    wxXML_PI_NODE,
    wxXML_COMMENT_NODE,
    wxXML_DOCUMENT_NODE,
    wxXML_DOCUMENT_TYPE_NODE,
    wxXML_DOCUMENT_FRAG_NODE,
    wxXML_NOTATION_NODE,
    wxXML_HTML_DOCUMENT_NODE
}
\end{verbatim}
}

\wxheading{See also}

\helpref{wxXmlDocument}{wxxmldocument}, \helpref{wxXmlAttribute}{wxxmlattribute}


\latexignore{\rtfignore{\wxheading{Members}}}


\membersection{wxXmlNode::wxXmlNode}\label{wxxmlnodewxxmlnode}

\func{}{wxXmlNode}{\param{wxXmlNode* }{parent}, \param{wxXmlNodeType }{type}, \param{const wxString\& }{name}, \param{const wxString\& }{content = wxEmptyString}, \param{wxXmlAttribute* }{attrs = \NULL}, \param{wxXmlNode* }{next = \NULL}}

\wxheading{Parameters}

\docparam{parent}{The parent node to which append this node instance.
If this argument is \NULL this new node will be {\it floating} and it can be appended later to 
another one using the \helpref{AddChild}{wxxmlnodeaddchild} or \helpref{InsertChild}{wxxmlnodeinsertchild}
functions. Otherwise the child is already added to the XML tree by this
constructor and it shouldn't be done again.}
\docparam{type}{One of the wxXmlNodeType enumeration value.}
\docparam{name}{The name of the node. This is the string which appears between angular brackets.}
\docparam{content}{The content of the node. Only meaningful when {\it type} is 
{\tt wxXML\_TEXT\_NODE} or {\tt wxXML\_CDATA\_SECTION\_NODE}.}
\docparam{attrs}{If not \NULL, this \helpref{wxXmlAttribute}{wxxmlattribute} object 
and its eventual siblings are attached to the node.}
\docparam{next}{If not \NULL, this node and its eventual siblings are attached to
the node.}

Creates this XML node and eventually insert it into an existing XML tree.

\func{}{wxXmlNode}{\param{const wxXmlNode\& }{node}}

Copy constructor. Note that this does NOT copy syblings
and parent pointer, i.e. \helpref{GetParent()}{wxxmlnodegetparent} and 
\helpref{GetNext()}{wxxmlnodegetnext} will return \NULL
after using copy ctor and are never unmodified by operator=.

On the other hand, it DOES copy children and attributes.


\func{}{wxXmlNode}{\param{wxXmlNodeType }{type}, \param{const wxString\& }{name}, \param{const wxString\& }{content = wxEmptyString}}

A simplified version of the first constructor form, assuming a \NULL parent.


\membersection{wxXmlNode::\destruct{wxXmlNode}}\label{wxxmlnodedtor}

\func{}{\destruct{wxXmlNode}}{\void}

The virtual destructor. Deletes attached children and attributes.

\membersection{wxXmlNode::AddChild}\label{wxxmlnodeaddchild}

\func{void}{AddChild}{\param{wxXmlNode* }{child}}

Adds the given node as child of this node. To attach a second children to this node, use the
\helpref{SetNext()}{wxxmlnodesetnext} function of the {\it child} node.

\membersection{wxXmlNode::AddAttribute}\label{wxxmlnodeaddattribute}

\func{void}{AddAttribute}{\param{const wxString\& }{name}, \param{const wxString\& }{value}}

Appends a attribute with given {\it name} and {\it value} to the list of attributes for this node.

\func{void}{AddAttribute}{\param{wxXmlAttribute* }{attr}}

Appends given attribute to the list of attributes for this node.

\membersection{wxXmlNode::DeleteAttribute}\label{wxxmlnodedeleteattribute}

\func{bool}{DeleteAttribute}{\param{const wxString\& }{name}}

Removes the first attributes which has the given {\it name} from the list of attributes for this node.

\membersection{wxXmlNode::GetChildren}\label{wxxmlnodegetchildren}

\constfunc{wxXmlNode*}{GetChildren}{\void}

Returns the first child of this node.
To get a pointer to the second child of this node (if it does exist), use the
\helpref{GetNext()}{wxxmlnodegetnext} function on the returned value.

\membersection{wxXmlNode::GetContent}\label{wxxmlnodegetcontent}

\constfunc{wxString}{GetContent}{\void}

Returns the content of this node. Can be an empty string.
Be aware that for nodes of type \texttt{wxXML\_ELEMENT\_NODE} (the most used node type) the
content is an empty string. See \helpref{GetNodeContent()}{wxxmlnodegetnodecontent} for more details.


\membersection{wxXmlNode::GetDepth}\label{wxxmlnodegetdepth}

\constfunc{int}{GetDepth}{\param{wxXmlNode* }{grandparent = NULL}}

Returns the number of nodes which separe this node from {\tt grandparent}.

This function searches only the parents of this node until it finds {\tt grandparent}
or the \NULL node (which is the parent of non-linked nodes or the parent of a
\helpref{wxXmlDocument}{wxxmldocument}'s root node).


\membersection{wxXmlNode::GetNodeContent}\label{wxxmlnodegetnodecontent}

\constfunc{wxString}{GetNodeContent}{\void}

Returns the content of the first child node of type \texttt{wxXML\_TEXT\_NODE} or \texttt{wxXML\_CDATA\_SECTION\_NODE}.
This function is very useful since the XML snippet \texttt{``<tagname>tagcontent</tagname>"} is represented by
expat with the following tag tree:

\begin{verbatim}
wxXML_ENTITY_NODE name="tagname", content=""
|-- wxXML_TEXT_NODE name="", content="tagcontent"
\end{verbatim}

or eventually:

\begin{verbatim}
wxXML_ENTITY_NODE name="tagname", content=""
|-- wxXML_CDATA_SECTION_NODE name="", content="tagcontent"
\end{verbatim}

An empty string is returned if the node has no children of type \texttt{wxXML\_TEXT\_NODE} or \texttt{wxXML\_CDATA\_SECTION\_NODE}, or if the content of the first child of such types is empty.


\membersection{wxXmlNode::GetName}\label{wxxmlnodegetname}

\constfunc{wxString}{GetName}{\void}

Returns the name of this node. Can be an empty string (e.g. for nodes of type {\tt wxXML\_TEXT\_NODE} or {\tt wxXML\_CDATA\_SECTION\_NODE}).

\membersection{wxXmlNode::GetNext}\label{wxxmlnodegetnext}

\constfunc{wxXmlNode*}{GetNext}{\void}

Returns a pointer to the sibling of this node or \NULL if there are no siblings.

\membersection{wxXmlNode::GetParent}\label{wxxmlnodegetparent}

\constfunc{wxXmlNode*}{GetParent}{\void}

Returns a pointer to the parent of this node or \NULL if this node has no parent.

\membersection{wxXmlNode::GetAttribute}\label{wxxmlnodegetattribute}

\constfunc{bool}{GetAttribute}{\param{const wxString\& }{attrName}, \param{wxString* }{value}}

Returns \true if a attribute named {\it attrName} could be found.
If the {\it value} pointer is not \NULL, the value of that attribute is saved there.

\constfunc{wxString}{GetAttribute}{\param{const wxString\& }{attrName}, \param{const wxString\& }{defaultVal}}

Returns the value of the attribute named {\it attrName} if it does exist.
If it does not exist, the {\it defaultVal} is returned.

\membersection{wxXmlNode::GetAttributes}\label{wxxmlnodegetattributes}

\constfunc{wxXmlAttribute *}{GetAttributes}{\void}

Return a pointer to the first attribute of this node.

\membersection{wxXmlNode::GetType}\label{wxxmlnodegettype}

\constfunc{wxXmlNodeType}{GetType}{\void}

Returns the type of this node.


\membersection{wxXmlNode::HasAttribute}\label{wxxmlnodehasattribute}

\constfunc{bool}{HasAttribute}{\param{const wxString\& }{attrName}}

Returns \true if this node has a attribute named {\it attrName}.

\membersection{wxXmlNode::InsertChild}\label{wxxmlnodeinsertchild}

\func{bool}{InsertChild}{\param{wxXmlNode* }{child}, \param{wxXmlNode* }{before\_node}}

Inserts the {\it child} node after {\it before\_node} in the children list.
If {\it before\_node} is \NULL, then {\it child} is prepended to the list of children and
becomes the first child of this node.
Returns \true if {\it before\_node} has been found and the {\it child} node has been inserted.

\membersection{wxXmlNode::IsWhitespaceOnly}\label{wxxmlnodecontainsiswhitespaceonly}

\constfunc{bool}{IsWhitespaceOnly}{\void}

Returns \true if the content of this node is a string containing only whitespaces (spaces,
tabs, new lines, etc). Note that this function is locale-independent since the parsing of XML
documents must always produce the exact same tree regardless of the locale it runs under.

\membersection{wxXmlNode::RemoveChild}\label{wxxmlnoderemovechild}

\func{bool}{RemoveChild}{\param{wxXmlNode* }{child}}

Removes the given node from the children list. Returns \true if the node was found and removed
or \false if the node could not be found.

Note that the caller is reponsible for deleting the removed node in order to avoid memory leaks.

\membersection{wxXmlNode::SetChildren}\label{wxxmlnodesetchildren}

\func{void}{SetChildren}{\param{wxXmlNode* }{child}}

Sets as first child the given node. The caller is responsible to delete any previously present
children node.

\membersection{wxXmlNode::SetContent}\label{wxxmlnodesetcontent}

\func{void}{SetContent}{\param{const wxString\& }{con}}

Sets the content of this node.

\membersection{wxXmlNode::SetName}\label{wxxmlnodesetname}

\func{void}{SetName}{\param{const wxString\& }{name}}

Sets the name of this node.

\membersection{wxXmlNode::SetNext}\label{wxxmlnodesetnext}

\func{void}{SetNext}{\param{wxXmlNode* }{next}}

Sets as sibling the given node. The caller is responsible to delete any previously present
sibling node.

\membersection{wxXmlNode::SetParent}\label{wxxmlnodesetparent}

\func{void}{SetParent}{\param{wxXmlNode* }{parent}}

Sets as parent the given node. The caller is responsible to delete any previously present
parent node.

\membersection{wxXmlNode::SetAttributes}\label{wxxmlnodesetattributes}

\func{void}{SetAttributes}{\param{wxXmlAttribute* }{attr}}

Sets as first attribute the given wxXmlAttribute object.
The caller is responsible to delete any previously present attributes attached to this node.

\membersection{wxXmlNode::SetType}\label{wxxmlnodesettype}

\func{void}{SetType}{\param{wxXmlNodeType }{type}}

Sets the type of this node.

\membersection{wxXmlNode::operator=}\label{wxxmlnodeoperatorassign}

\func{wxXmlNode\&}{operator=}{\param{const wxXmlNode\& }{node}}

See the copy constructor for more info.

