\documentstyle[a4,11pt,makeidx,verbatim,texhelp,fancyheadings,palatino]{report}
% JACS: doesn't make it through Tex2RTF, sorry. I will put it into texhelp.sty
% since Tex2RTF doesn't parse it.
% BTW, style MUST be report for it to work for Tex2RTF.
%KB:
%\addtolength{\textwidth}{1in}
%\addtolength{\oddsidemargin}{-0.5in}
%\addtolength{\topmargin}{-0.5in}
%\addtolength{\textheight}{1in}
%\sloppy
%end of my changes
\newcommand{\indexit}[1]{#1\index{#1}}%
\newcommand{\pipe}[0]{$\|$\ }%
\definecolour{black}{0}{0}{0}%
\definecolour{cyan}{0}{255}{255}%
\definecolour{green}{0}{255}{0}%
\definecolour{magenta}{255}{0}{255}%
\definecolour{red}{255}{0}{0}%
\definecolour{blue}{0}{0}{200}%
\definecolour{yellow}{255}{255}{0}%
\definecolour{white}{255}{255}{255}%
%
\input psbox.tex
% Remove this for processing with dvi2ps instead of dvips
%\special{!/@scaleunit 1 def}
\parskip=10pt
\parindent=0pt
\title{Multiplatform application development with wxWindows}
\winhelponly{\author{by Julian Smart et al
%\winhelponly{\\$$\image{1cm;0cm}{wxwin.wmf}$$}
}}
\winhelpignore{\author{Julian Smart, Robert Roebling, Vadim Zeitlin,
Robin Dunn, et al}
\date{August 13th 2000}
}
\makeindex
\begin{document}
\maketitle
\pagestyle{fancyplain}
\bibliographystyle{plain}
\setheader{{\it CONTENTS}}{}{}{}{}{{\it CONTENTS}}
\setfooter{\thepage}{}{}{}{}{\thepage}%
\pagenumbering{roman}
\tableofcontents

\input chap\_intro.tex              % Chapter 01: Introduction, advocacy, etc.
\input chap\_install.tex            % Chapter 02: Installing wxWindows (and what tools to use) 
\input chap\_cpp.tex                % Chapter 03: C++ and wxWindows. Summarises the sorts of constructs used/not used, plus wxString class, some conventions. Vadim suggests putting it in 1st chapter but I think it deserves a chapter of its own. 
\input chap\_getstart.tex           % Chapter 04: Getting started: Hello World. Introduces app class, frames, menus, status bar, message box 
\input chap\_basic\_events.tex      % Chapter 05: Basic event handling 
\input chap\_frames.tex             % Chapter 06: Frames and menubars. The components of a frame, menubars. 
\input chap\_toolbars.tex           % Chapter 07: Toolbars and status bars 
\input chap\_basic\_controls.tex    % Chapter 08: Basic controls 
\input chap\_common\_dialogs.tex    % Chapter 09: Common dialogs 
\input chap\_custom\_dialogs.tex    % Chapter 10: Custom dialogs and resources (XML + WXR) 
\input chap\_drawing.tex            % Chapter 11: Drawing on device contexts 
\input chap\_input.tex              % Chapter 12: Handling input (mouse, keyboard, joystick) 
\input chap\_sizers.tex             % Chapter 14: Sizers 
\input chap\_images.tex             % Chapter 15: Images and bitmaps 
\input chap\_clipboard\_dnd.tex     % Chapter 16: Clipboard and drag and drop 
\input chap\_advanced\_controls.tex % Chapter 17: Advanced controls (list,tree,notebook,splitter,wxWizard,wxCalCtrl...) 
\input chap\_docview.tex            % Chapter 18: Document/view classes 
\input chap\_scrolling.tex          % Chapter 19: Scrolling 
\input chap\_mdi.tex                % Chapter 20: MDI 
\input chap\_printing.tex           % Chapter 21: Printing 
\input chap\_help.tex               % Chapter 22: Providing help in your applications 
\input chap\_strings.tex            % Chapter 23: Strings and internationalization 
\input chap\_data\_classes.tex      % Chapter 24: Collection and container classes 
\input chap\_memory.tex             % Chapter 25: Memory management and debugging (including wxLog) 
\input chap\_runtime.tex            % Chapter 26: Run-time class information 
\input chap\_advanced\_events.tex   % Chapter 27: Advanced event handling (user-defined events, ...) 
\input chap\_comms.tex              % Chapter 28: Communication classes, including wxSocket 
\input chap\_database.tex           % Chapter 29: Database classes 
\input chap\_file\_stream.tex       % Chapter 30: File and stream classes 
\input chap\_config.tex             % Chapter 31: Configuration classes 
\input chap\_time.tex               % Chapter 32: Time, timers and idle processing 
\input chap\_multithreading.tex     % Chapter 33: Writing multithreading applications 
\input chap\_perfecting.tex         % Chapter 34: Perfecting your UI (Adapting to system settings, accelerators, ...) 
\input chap\_platform.tex           % Chapter 35: Platform-specific programming (metafiles, OLE automation, taskbar, ...) 
\input chap\_wxhtml.tex             % Chapter 36: Using wxHTML 
\input chap\_wxpython.tex           % Chapter 37: Using wxPython 
\input chap\_wxbase.tex             % Chapter 38: wxBase? 
\input chap\_comparison.tex         % Appendix: Comparison with other toolkits: MFC, Qt etc. 
\input chap\_resources.tex          % Appendix: a compendium of external resources, libraries etc.

\bibliography{refs}
\addcontentsline{toc}{chapter}{Bibliography}
\setheader{{\it REFERENCES}}{}{}{}{}{{\it REFERENCES}}%
\setfooter{\thepage}{}{}{}{}{\thepage}%

\newpage
% Note: In RTF, the \printindex must come before the
% change of header/footer, since the \printindex inserts
% the RTF \sect command which divides one chapter from
% the next.
\rtfonly{\printindex
\addcontentsline{toc}{chapter}{Index}
\setheader{{\it INDEX}}{}{}{}{}{{\it INDEX}}%
\setfooter{\thepage}{}{}{}{}{\thepage}
}
% In Latex, it must be this way around (I think)
\latexonly{\addcontentsline{toc}{chapter}{Index}
\setheader{{\it INDEX}}{}{}{}{}{{\it INDEX}}%
\setfooter{\thepage}{}{}{}{}{\thepage}
\printindex
}

\end{document}
