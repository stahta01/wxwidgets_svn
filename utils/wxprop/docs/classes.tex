\chapter{Alphabetical class reference}\label{classref}
\setheader{{\it CHAPTER \thechapter}}{}{}{}{}{{\it CHAPTER \thechapter}}%
\setfooter{\thepage}{}{}{}{}{\thepage}%

\overview{Property classes overview}{propertyoverview}


\section{\class{wxBoolFormValidator}: wxPropertyFormValidator}\label{wxboolformvalidator}

\overview{Validator classes}{validatorclasses}

This class validates a boolean value for a form view. The associated panel item must be a wxCheckBox.

\membersection{wxBoolFormValidator::wxBoolFormValidator}

\func{void}{wxBoolFormValidator}{\param{long }{flags=0}}

Constructor.

\section{\class{wxBoolListValidator}: wxPropertyListValidator}\label{wxboollistvalidator}

\overview{Validator classes}{validatorclasses}

This class validates a boolean value for a list view.

\membersection{wxBoolListValidator::wxBoolListValidator}

\func{void}{wxBoolListValidator}{\param{long }{flags=0}}

Constructor.

\section{\class{wxIntegerFormValidator}: wxPropertyFormValidator}\label{wxintegerformvalidator}

\overview{Validator classes}{validatorclasses}

This class validates a range of integer values for a form view. The associated panel item must be a wxText
or wxSlider.

\membersection{wxIntegerFormValidator::wxIntegerFormValidator}

\func{void}{wxIntegerFormValidator}{\param{long }{min=0}, \param{long }{max=0},
 \param{long}{ flags=0}}

Constructor. Assigning zero to minimum and maximum values indicates that there is no range to check.


\section{\class{wxIntegerListValidator}: wxPropertyListValidator}\label{wxintegerlistvalidator}

\overview{Validator classes}{validatorclasses}

This class validates a range of integer values for a list view.

\membersection{wxIntegerListValidator::wxIntegerListValidator}

\func{void}{wxIntegerListValidator}{\param{long }{min=0}, \param{long }{max=0},
 \param{long}{ flags=wxPROP\_ALLOW\_TEXT\_EDITING}}

Constructor. Assigning zero to minimum and maximum values indicates that there is no range to check.

\section{\class{wxFilenameListValidator}: wxPropertyListValidator}\label{wxfilenamelistvalidator}

\overview{Validator classes}{validatorclasses}

This class validates a filename for a list view, allowing the user to edit it textually and also popping up
a file selector in ``detailed editing" mode.

\membersection{wxFilenameListValidator::wxFilenameListValidator}

\func{void}{wxFilenameListValidator}{\param{wxString }{message = ``Select a file"}, \param{wxString }{wildcard = ``*.*"},
  \param{long}{ flags=0}}

Constructor. Supply an optional message and wildcard.

\section{\class{wxListOfStringsListValidator}: wxPropertyListValidator}\label{wxlistofstringslistvalidator}

\overview{Validator classes}{validatorclasses}

This class validates a list of strings for a list view. When editing the property,
a dialog box is presented for adding, deleting or editing entries in the list.
At present no constraints may be supplied.

You can construct a string list property value by constructing a wxStringList object.

For example:

\begin{verbatim}
  myListValidatorRegistry.RegisterValidator((wxString)"stringlist",
      new wxListOfStringsListValidator);

  wxStringList *strings = new wxStringList("earth", "fire", "wind", "water", NULL);

  sheet->AddProperty(new wxProperty("fred", strings, "stringlist"));
\end{verbatim}

\membersection{wxListOfStringsListValidator::wxListofStringsListValidator}

\func{void}{wxListOfStringsListValidator}{\param{long}{ flags=0}}

Constructor.

\section{\class{wxProperty}: wxObject}\label{wxproperty}

The {\bf wxProperty} class represents a property, with a \helpref{wxPropertyValue}{wxpropertyvalue}\rtfsp
containing the actual value, a name a role, an optional validator, and
an optional associated window. 

A property might correspond to an actual C++ data member, or it
might correspond to a conceptual property, such as the width of a window.
There is no explicit data member {\it wxWindow::width}, but it may be convenient
to invent such a property for the purposes of editing attributes of the window.
The properties in the property sheet can be mapped to ``reality" by
whatever means (in this case by calling wxWindow::SetSize when the user has
finished editing the property sheet).

A validator may be associated with the property in order to ensure that this and
only this validator will be used for editing and validating the property.
An alternative method is to use the {\it role} parameter to specify what kind
of validator would be appropriate; for example, specifying ``filename" for the role
would allow the property view to find an appropriate validator at edit time.


\membersection{wxProperty::wxProperty}

\func{void}{wxProperty}{\void}

\func{void}{wxProperty}{\param{wxProperty\& }{prop}}

\func{void}{wxProperty}{\param{wxString}{ name}, \param{wxString}{ role}, \param{wxPropertyValidator *}{validator=NULL}}

\func{void}{wxProperty}{\param{wxString}{ name}, \param{const wxPropertyValue\&}{ val}, \param{wxString}{ role}, \param{wxPropertyValidator *}{validator=NULL}}

Constructors.

\membersection{wxProperty::\destruct{wxProperty}}

\func{void}{\destruct{wxProperty}}{\void}

Destructor. Destroys the wxPropertyValue, and the property validator if there is one. However, if the
actual C++ value in the wxPropertyValue is a pointer, the data in that variable is not destroyed.

\membersection{wxProperty::GetValue}

\func{wxPropertyValue\&}{GetValue}{\void}

Returns a reference to the property value.

\membersection{wxProperty::GetValidator}

\func{wxPropertyValidator *}{GetValidator}{\void}

Returns a pointer to the associated property validator (if any).

\membersection{wxProperty::GetName}

\func{wxString\&}{GetName}{\void}

Returns the name of the property.

\membersection{wxProperty::GetRole}

\func{wxRole\&}{GetRole}{\void}

Returns the role of the property, to be used when choosing an appropriate validator.

\membersection{wxProperty::GetWindow}

\func{wxWindow *}{GetWindow}{\void}

Returns the window associated with the property (if any).

\membersection{wxProperty::SetValue}

\func{void}{SetValue}{\param{wxPropertyValue\&}{ val}}

Sets the value of the property.

\membersection{wxProperty::SetName}

\func{void}{SetName}{\param{wxString\&}{ name}}

Sets the name of the property.

\membersection{wxProperty::SetRole}

\func{void}{SetRole}{\param{wxString\&}{ role}}

Sets the role  of the property.

\membersection{wxProperty::SetValidator}

\func{void}{SetValidator}{\param{wxPropertyValidator *}{validator}}

Sets the validator: this will be deleted when the property is deleted.

\membersection{wxProperty::SetWindow}

\func{void}{SetWindow}{\param{wxWindow *}{win}}

Sets the window associated with the property.

\membersection{wxProperty::operator $=$}

\func{void}{operator $=$}{\param{const wxPropertyValue\&}{ val}}

Assignment operator.

\section{\class{wxPropertyFormValidator}: wxPropertyValidator}\label{wxpropertyformvalidator}

The {\bf wxPropertyFormValidator} abstract class is the root of classes that define validation
for a wxPropertyFormView.


\section{\class{wxPropertyFormDialog}: wxDialogBox}\label{wxpropertyformdialog}

The {\bf wxPropertyFormDialog} class is a prepackaged dialog which can
be used for viewing a form property sheet. Pass a property form view object, and the dialog
will pass OnClose and OnDefaultAction listbox messages to the view class for
processing.

\membersection{wxPropertyFormDialog::wxPropertyFormDialog}

\func{void}{wxPropertyFormDialog}{\param{wxPropertyFormView *}{view}, \param{wxWindow *}{parent}, \param{char *}{title},
 \param{Bool}{ modal=FALSE}, \param{int}{ x=-1}, \param{int}{ y=-1}, \param{int}{ width=-1}, \param{int}{height=-1},
 \param{long}{ style=wxDEFAULT\_DIALOG\_STYLE}, \param{char *}{name=``dialogBox"}}

Constructor.

\membersection{wxPropertyFormDialog::\destruct{wxPropertyFormDialog}}

\func{void}{\destruct{wxPropertyFormDialog}}{\void}

Destructor.


\section{\class{wxPropertyFormFrame}: wxFrame}\label{wxpropertyformframe}

The {\bf wxPropertyFormFrame} class is a prepackaged frame which can
be used for viewing a property form. Pass a property form view object, and the frame
will pass OnClose messages to the view class for processing.

Call Initialize to create the panel and associate the view; override OnCreatePanel
if you wish to use a panel class other than the default wxPropertyFormPanel.

\membersection{wxPropertyFormFrame::wxPropertyFormFrame}

\func{void}{wxPropertyFormFrame}{\param{wxPropertyFormView *}{view}, \param{wxFrame *}{parent}, \param{char *}{title},
 \param{int}{ x=-1}, \param{int}{ y=-1}, \param{int}{ width=-1}, \param{int}{height=-1},
 \param{long}{ style=wxSDI $\|$ wxDEFAULT\_FRAME}, \param{char *}{name=``frame"}}

Constructor.

\membersection{wxPropertyFormFrame::\destruct{wxPropertyFormFrame}}

\func{void}{\destruct{wxPropertyFormFrame}}{\void}

Destructor.

\membersection{wxPropertyFormFrame::GetPropertyPanel}

\func{wxPanel *}{GetPropertyPanel}{\void}

Returns the panel associated with the frame.

\membersection{wxPropertyFormFrame::Initialize}

\func{Bool}{Initialize}{\void}

Must be called to create the panel and associate the view with the panel and frame.

\membersection{wxPropertyFormFrame::OnCreatePanel}

\func{wxPanel *}{OnCreatePanel}{\param{wxFrame *}{parent}, \param{wxPropertyFormView *}{view}}

Creates a panel. Override this to create a panel type other than wxPropertyFormPanel.


\section{\class{wxPropertyFormPanel}: wxPanel}\label{wxpropertyformpanel}

The {\bf wxPropertyFormPanel} class is a prepackaged panel which can
be used for viewing a property form. Pass a property form view object, and the panel
will pass OnDefaultAction listbox messages to the view class for
processing.

\membersection{wxPropertyFormPanel::wxPropertyFormPanel}

\func{void}{wxPropertyFormPanel}{\param{wxPropertyFormView *}{view}, \param{wxWindow *}{parent},
 \param{int}{ x=-1}, \param{int}{ y=-1}, \param{int}{ width=-1}, \param{int}{height=-1},
 \param{long}{ style=0}, \param{char *}{name=``panel"}}

Constructor.

\membersection{wxPropertyFormPanel::\destruct{wxPropertyFormPanel}}

\func{void}{\destruct{wxPropertyFormPanel}}{\void}

Destructor.



\section{\class{wxPropertyFormValidator}: wxPropertyValidator}\label{wxpropertyformvalidatir}

\overview{wxPropertyFormValidator overview}{wxpropertyformvalidatoroverview}

The {\bf wxPropertyFormValidator} class defines a base class for form validators. By overriding virtual functions,
the programmer can create custom behaviour for kinds of property.

\membersection{wxPropertyFormValidator::wxPropertyFormValidator}

\func{void}{wxPropertyFormValidator}{\param{long}{ flags = 0}}

Constructor.

\membersection{wxPropertyFormValidator::\destruct{wxPropertyFormValidator}}

\func{void}{\destruct{wxPropertyFormValidator}}{\void}

Destructor.

\membersection{wxPropertyFormValidator::OnCommand}

\func{Bool}{OnCommand}{\param{wxProperty *}{property}, \param{wxPropertyFormView *}{view},
 \param{wxWindow *}{parentWindow}, \param{wxCommandEvent\& }{event}}

Called when the control corresponding to the property receives a command (if not intercepted
by a callback associated with the actual control).

\membersection{wxPropertyFormValidator::OnCheckValue}

\func{Bool}{OnCheckValue}{\param{wxProperty *}{property}, \param{wxPropertyFormView *}{view},
 \param{wxWindow *}{parentWindow}}
 
Called when the view checks the property value. The value checked by this validator should be taken from the
panel item corresponding to the property.

\membersection{wxPropertyFormValidator::OnDisplayValue}

\func{Bool}{OnDisplayValue}{\param{wxProperty *}{property}, \param{wxPropertyFormView *}{view},
 \param{wxWindow *}{parentWindow}}

Should display the property value in the appropriate control.
 
\membersection{wxPropertyFormValidator::OnDoubleClick}

\func{Bool}{OnDoubleClick}{\param{wxProperty *}{property}, \param{wxPropertyFormView *}{view},
 \param{wxWindow *}{parentWindow}}

Called when the control corresponding to the property is double clicked (listboxes only).

\membersection{wxPropertyFormValidator::OnRetrieveValue}

\func{Bool}{OnRetrieveValue}{\param{wxProperty *}{property}, \param{wxPropertyFormView *}{view},
 \param{wxWindow *}{parentWindow}}

Should do the transfer from the property editing area to the property itself.


\section{\class{wxPropertyFormView}: wxPropertyView}\label{wxpropertyformview}

\overview{wxPropertyFormView overview}{wxpropertyformviewoverview}

The {\bf wxPropertyFormView} class shows a wxPropertySheet as a view onto a panel or dialog
box which has already been created.

\membersection{wxPropertyFormView::wxPropertyFormView}

\func{void}{wxPropertyFormView}{\param{long}{ flags = 0}}

Constructor.

\membersection{wxPropertyFormView::\destruct{wxPropertyFormView}}

\func{void}{\destruct{wxPropertyFormView}}{\void}

Destructor.

\membersection{wxPropertyFormView::AssociateNames}\label{wxpropertyformviewassociatenames}

\func{void}{AssociateNames}{\void}

Associates the properties with the controls on the panel. For each panel item, if the
panel item name is the same as a property name, the two objects will be associated.
This function should be called manually since the programmer may wish to do the
association manually.

\membersection{wxPropertyFormView::Check}\label{wxpropertyformviewcheck}

\func{Bool}{Check}{\void}

Checks all properties by calling the appropriate validators; returns FALSE if a validation failed.

\membersection{wxPropertyFormView::GetPanel}\label{wxpropertyformviewgetpanel}

\func{wxPanel *}{GetPanel}{\void}

Returns the panel associated with the view.

\membersection{wxPropertyFormView::GetManagedWindow}\label{wxpropertyformviewgetmanagedwindow}

\func{wxWindow *}{GetManagedWindow}{\void}

Returns the managed window (a frame or dialog) associated with the view.

\membersection{wxPropertyFormView::OnOk}\label{wxpropertyformviewonok}

\func{void}{OnOk}{\void}

Virtual function that will be called when the OK button on the physical window is pressed.
By default, checks and updates the form values, closes and deletes the frame or dialog, then deletes the view.

\membersection{wxPropertyFormView::OnCancel}\label{wxpropertyformviewoncancel}

\func{void}{OnCancel}{\void}

Virtual function that will be called when the Cancel button on the physical window is pressed.
By default, closes and deletes the frame or dialog, then deletes the view.

\membersection{wxPropertyFormView::OnHelp}\label{wxpropertyformviewonhelp}

\func{void}{OnHelp}{\void}

Virtual function that will be called when the Help button on the physical window is pressed.
This needs to be overridden by the application for anything interesting to happen.

\membersection{wxPropertyFormView::OnRevert}\label{wxpropertyformviewonrevert}

\func{void}{OnRevert}{\void}

Virtual function that will be called when the Revert button on the physical window is pressed.
By default transfers the wxProperty values to the panel items (in effect
undoing any unsaved changes in the items).

\membersection{wxPropertyFormView::OnUpdate}\label{wxpropertyformviewonupdate}

\func{void}{OnUpdate}{\void}

Virtual function that will be called when the Update button on the physical window is pressed.
By defaults transfers the displayed values to the wxProperty objects.

\membersection{wxPropertyFormView::SetManagedWindow}\label{wxpropertyformviewsetmanagedwindow}

\func{void}{SetManagedWindow}{\param{wxWindow *}{win}}

Sets the managed window (a frame or dialog) associated with the view.

\membersection{wxPropertyFormView::TransferToDialog}\label{wxpropertyformviewtransfertodialog}

\func{Bool}{TransferToDialog}{\void}

Transfers property values to the controls in the dialog.

\membersection{wxPropertyFormView::TransferToPropertySheet}\label{wxpropertyformviewtransfertopropertysheet}

\func{Bool}{TransferToPropertySheet}{\void}

Transfers property values from the controls in the dialog to the property sheet.


\section{\class{wxPropertyListDialog}: wxDialogBox}\label{wxpropertylistdialog}

The {\bf wxPropertyListDialog} class is a prepackaged dialog which can
be used for viewing a property list. Pass a property list view object, and the dialog
will pass OnClose and OnDefaultAction listbox messages to the view class for
processing.

\membersection{wxPropertyListDialog::wxPropertyListDialog}

\func{void}{wxPropertyListDialog}{\param{wxPropertyListView *}{view}, \param{wxWindow *}{parent}, \param{char *}{title},
 \param{Bool}{ modal=FALSE}, \param{int}{ x=-1}, \param{int}{ y=-1}, \param{int}{ width=-1}, \param{int}{height=-1},
 \param{long}{ style=wxDEFAULT\_DIALOG\_STYLE}, \param{char *}{name=``dialogBox"}}

Constructor.

\membersection{wxPropertyListDialog::\destruct{wxPropertyListDialog}}

\func{void}{\destruct{wxPropertyListDialog}}{\void}

Destructor.


\section{\class{wxPropertyListFrame}: wxFrame}\label{wxpropertylistframe}

The {\bf wxPropertyListFrame} class is a prepackaged frame which can
be used for viewing a property list. Pass a property list view object, and the frame
will pass OnClose messages to the view class for processing.

Call Initialize to create the panel and associate the view; override OnCreatePanel
if you wish to use a panel class other than the default wxPropertyListPanel.

\membersection{wxPropertyListFrame::wxPropertyListFrame}

\func{void}{wxPropertyListFrame}{\param{wxPropertyListView *}{view}, \param{wxFrame *}{parent}, \param{char *}{title},
 \param{int}{ x=-1}, \param{int}{ y=-1}, \param{int}{ width=-1}, \param{int}{height=-1},
 \param{long}{ style=wxSDI $\|$ wxDEFAULT\_FRAME}, \param{char *}{name=``frame"}}

Constructor.

\membersection{wxPropertyListFrame::\destruct{wxPropertyListFrame}}

\func{void}{\destruct{wxPropertyListFrame}}{\void}

Destructor.

\membersection{wxPropertyListFrame::GetPropertyPanel}

\func{wxPanel *}{GetPropertyPanel}{\void}

Returns the panel associated with the frame.

\membersection{wxPropertyListFrame::Initialize}

\func{Bool}{Initialize}{\void}

Must be called to create the panel and associate the view with the panel and frame.

\membersection{wxPropertyListFrame::OnCreatePanel}

\func{wxPanel *}{OnCreatePanel}{\param{wxFrame *}{parent}, \param{wxPropertyListView *}{view}}

Creates a panel. Override this to create a panel type other than wxPropertyListPanel.


\section{\class{wxPropertyListPanel}: wxPanel}\label{wxpropertylistpanel}

The {\bf wxPropertyListPanel} class is a prepackaged panel which can
be used for viewing a property list. Pass a property list view object, and the panel
will pass OnDefaultAction listbox messages to the view class for
processing.

\membersection{wxPropertyListPanel::wxPropertyListPanel}

\func{void}{wxPropertyListPanel}{\param{wxPropertyListView *}{view}, \param{wxWindow *}{parent},
 \param{int}{ x=-1}, \param{int}{ y=-1}, \param{int}{ width=-1}, \param{int}{height=-1},
 \param{long}{ style=0}, \param{char *}{name=``panel"}}

Constructor.

\membersection{wxPropertyListPanel::\destruct{wxPropertyListPanel}}

\func{void}{\destruct{wxPropertyListPanel}}{\void}

Destructor.




\section{\class{wxPropertyListValidator}: wxPropertyValidator}\label{wxpropertylistvalidator}

\overview{wxPropertyListValidator overview}{wxpropertylistvalidatoroverview}

The {\bf wxPropertyListValidator} abstract class is the base class for
deriving validators for property lists.

\membersection{wxPropertyListValidator::wxPropertyListValidator}

\func{void}{wxPropertyListValidator}{\param{long}{ flags = wxPROP\_ALLOW\_TEXT\_EDITING}}

Constructor.

\membersection{wxPropertyListValidator::\destruct{wxPropertyListValidator}}

\func{void}{\destruct{wxPropertyListValidator}}{\void}

Destructor.

\membersection{wxPropertyListValidator::OnCheckValue}

\func{Bool}{OnCheckValue}{\param{wxProperty *}{property}, \param{wxPropertyListView *}{view},
 \param{wxWindow *}{parentWindow}}
 
Called when the Tick (Confirm) button is pressed or focus is list. Return FALSE if the value
was invalid, which is a signal restores the old value. Return TRUE if the value was valid.

\membersection{wxPropertyListValidator::OnClearControls}

\func{Bool}{OnClearControls}{\param{wxProperty *}{property}, \param{wxPropertyListView *}{view},
 \param{wxWindow *}{parentWindow}}
 
Allows the clearing (enabling, disabling) of property list controls, when the focus leaves the current property.

\membersection{wxPropertyListValidator::OnClearDetailControls}

\func{Bool}{OnClearDetailControls}{\param{wxProperty *}{property}, \param{wxPropertyListView *}{view},
 \param{wxWindow *}{parentWindow}}
 
Called when the focus is lost, if the validator is in detailed editing mode.

\membersection{wxPropertyListValidator::OnDisplayValue}

\func{Bool}{OnDisplayValue}{\param{wxProperty *}{property}, \param{wxPropertyListView *}{view},
 \param{wxWindow *}{parentWindow}}

Should display the value in the appropriate controls. The default implementation gets the
textual value from the property and inserts it into the text edit control.
 
\membersection{wxPropertyListValidator::OnDoubleClick}

\func{Bool}{OnDoubleClick}{\param{wxProperty *}{property}, \param{wxPropertyListView *}{view},
 \param{wxWindow *}{parentWindow}}

Called when the property is double clicked. Extra functionality can be provided,
such as cycling through possible values.

\membersection{wxPropertyListValidator::OnEdit}

\func{Bool}{OnEdit}{\param{wxProperty *}{property}, \param{wxPropertyListView *}{view},
 \param{wxWindow *}{parentWindow}}
 
Called when the Edit (detailed editing) button is pressed. The default implementation
calls wxPropertyListView::BeginDetailedEditing; a filename validator (for example) overrides
this function to show the file selector.

\membersection{wxPropertyListValidator::OnPrepareControls}

\func{Bool}{OnPrepareControls}{\param{wxProperty *}{property}, \param{wxPropertyListView *}{view},
 \param{wxWindow *}{parentWindow}}

Called to allow the validator to setup the display, such enabling or disabling buttons, and
setting the values and selection in the standard listbox control (the one optionally used for displaying
value options).

\membersection{wxPropertyListValidator::OnPrepareDetailControls}

\func{Bool}{OnPrepareDetailControls}{\param{wxProperty *}{property}, \param{wxPropertyListView *}{view},
 \param{wxWindow *}{parentWindow}}
 
Called when the property is edited `in detail', i.e. when the Edit button is pressed.

\membersection{wxPropertyListValidator::OnRetrieveValue}

\func{Bool}{OnRetrieveValue}{\param{wxProperty *}{property}, \param{wxPropertyListView *}{view},
 \param{wxWindow *}{parentWindow}}

Called when Tick (Confirm) is pressed or focus is lost or view wants to update
the property list. Should do the transfer from the property editing area to the property itself

\membersection{wxPropertyListValidator::OnSelect}

\func{Bool}{OnSelect}{\param{Bool}{ select}, \param{wxProperty *}{property}, \param{wxPropertyListView *}{view},
 \param{wxWindow *}{parentWindow}}

Called when the property is selected or deselected: typically displays the value
in the edit control (having chosen a suitable control to display: (non)editable text or listbox).

\membersection{wxPropertyListValidator::OnValueListSelect}

\func{Bool}{OnValueListSelect}{\param{wxProperty *}{property}, \param{wxPropertyListView *}{view},
 \param{wxWindow *}{parentWindow}}

Called when the value listbox is selected. The default behaviour is to copy
string to text control, and retrieve the value into the property.



\section{\class{wxPropertyListView}: wxPropertyView}\label{wxpropertylistview}

\overview{wxPropertyListView overview}{wxpropertylistviewoverview}

The {\bf wxPropertyListView} class shows a wxPropertySheet as a Visual Basic-style property list.

\membersection{wxPropertyListView::wxPropertyListView}

\func{void}{wxPropertyListView}{\param{long}{ flags = wxPROP\_BUTTON\_DEFAULT}}

Constructor.

The {\it flags} argument can be a bit list of the following:

\begin{itemize}\itemsep=0pt
\item wxPROP\_BUTTON\_CLOSE
\item wxPROP\_BUTTON\_OK
\item wxPROP\_BUTTON\_CANCEL
\item wxPROP\_BUTTON\_CHECK\_CROSS
\item wxPROP\_BUTTON\_HELP
\item wxPROP\_DYNAMIC\_VALUE\_FIELD
\item wxPROP\_PULLDOWN
\end{itemize}

\membersection{wxPropertyListView::\destruct{wxPropertyListView}}

\func{void}{\destruct{wxPropertyListView}}{\void}

Destructor.

\membersection{wxPropertyListView::AssociatePanel}\label{wxpropertylistviewassociatepanel}

\func{void}{AssociatePanel}{\param{wxPanel *}{panel}}

Associates the window on which the controls will be displayed, with the view (sets an internal pointer to the window).

\membersection{wxPropertyListView::BeginShowingProperty}\label{wxpropertylistviewbeginshowingproperty}

\func{Bool}{BeginShowingProperty}{\param{wxProperty *}{property}}

Finds the appropriate validator and loads the property into the controls, by calling
wxPropertyValidator::OnPrepareControls and then wxPropertyListView::DisplayProperty.

\membersection{wxPropertyListView::DisplayProperty}\label{wxpropertylistviewdisplayproperty}

\func{Bool}{DisplayProperty}{\param{wxProperty *}{property}}

Calls wxPropertyValidator::OnDisplayValue for the current property's validator. This function
gets called by wxPropertyListView::BeginShowingProperty, which is in turn called
from ShowProperty, called by OnPropertySelect, called by the listbox callback when selected.

\membersection{wxPropertyListView::EndShowingProperty}\label{wxpropertylistviewendshowingproperty}

\func{Bool}{EndShowingProperty}{\param{wxProperty *}{property}}

Finds the appropriate validator and unloads the property from the controls, by calling
wxPropertyListView::RetrieveProperty, wxPropertyValidator::OnClearControls and (if we're in
detailed editing mdoe) wxPropertyValidator::OnClearDetailControls.

\membersection{wxPropertyListView::GetPanel}\label{wxpropertylistviewgetpanel}

\func{wxPanel *}{GetPanel}{\void}

Returns the panel associated with the view.

\membersection{wxPropertyListView::GetManagedWindow}\label{wxpropertylistviewgetmanagedwindow}

\func{wxWindow *}{GetManagedWindow}{\void}

Returns the managed window (a frame or dialog) associated with the view.

\membersection{wxPropertyListView::GetWindowCancelButton}\label{wxpropertylistviewgetwindowcancelbutton}

\func{wxButton *}{GetWindowCancelButton}{\void}

Returns the window cancel button, if any.

\membersection{wxPropertyListView::GetWindowCloseButton}\label{wxpropertylistviewgetwindowclosebutton}

\func{wxButton *}{GetWindowCloseButton}{\void}

Returns the window close or OK button, if any.

\membersection{wxPropertyListView::GetWindowHelpButton}\label{wxpropertylistviewgetwindowhelpbutton}

\func{wxButton *}{GetWindowHelpButton}{\void}

Returns the window help button, if any.

\membersection{wxPropertyListView::SetManagedWindow}\label{wxpropertylistviewsetmanagedwindow}

\func{void}{SetManagedWindow}{\param{wxWindow *}{win}}

Sets the managed window (a frame or dialog) associated with the view.

\membersection{wxPropertyListView::UpdatePropertyDisplayInList}\label{wxpropertylistviewupdatepropdisplay}

\func{Bool}{UpdatePropertyDisplayInList}{\param{wxProperty *}{property}}

Updates the display for the given changed property.

\membersection{wxPropertyListView::UpdatePropertyList}\label{wxpropertylistviewupdateproplist}

\func{Bool}{UpdatePropertyList}{\param{Bool }{clearEditArea = TRUE}}

Updates the whole property list display.


\section{\class{wxPropertySheet}: wxObject}\label{wxpropertysheet}

\overview{wxPropertySheet overview}{wxpropertysheetoverview}

The {\bf wxPropertySheet} class is used for storing a number of
wxProperty objects (essentially names and values).

\membersection{wxPropertySheet::wxPropertySheet}

\func{void}{wxPropertySheet}{\void}

Constructor.

\membersection{wxPropertySheet::\destruct{wxPropertySheet}}

\func{void}{\destruct{wxPropertySheet}}{\void}

Destructor. Destroys all contained properties.

\membersection{wxPropertySheet::AddProperty}\label{wxpropertysheetaddproperty}

\func{void}{AddProperty}{\param{wxProperty *}{property}}

Adds a property to the sheet.

\membersection{wxPropertySheet::Clear}\label{wxpropertysheetclear}

\func{void}{Clear}{\void}

Clears all the properties from the sheet (deleting them).

\membersection{wxPropertySheet::GetProperties}\label{wxpropertysheetgetproperties}

\func{wxList\&}{GetProperties}{\void}

Returns a reference to the internal list of properties.

\membersection{wxPropertySheet::GetProperty}\label{wxpropertysheetgetproperty}

\func{wxProperty *}{GetProperty}{\param{char *}{name}}

Gets a property by name.

\membersection{wxPropertySheet::SetAllModified}

\func{void}{SetAllModified}{\param{Bool}{ flag}}

Sets the `modified' flag of each property value.



\section{\class{wxPropertyValidator}: wxEvtHandler}\label{wxpropertyvalidator}

\overview{wxPropertyValidator overview}{wxpropertyvalidatoroverview}

The {\bf wxPropertyValidator} abstract class is the base class for deriving
validators for properties.

\membersection{wxPropertyValidator::wxPropertyValidator}

\func{void}{wxPropertyValidator}{\param{long}{ flags = 0}}

Constructor.

\membersection{wxPropertyValidator::\destruct{wxPropertyValidator}}

\func{void}{\destruct{wxPropertyValidator}}{\void}

Destructor.

\membersection{wxPropertyValidator::GetFlags}

\func{long}{GetFlags}{\void}

Returns the flags for the validator.

\membersection{wxPropertyValidator::GetValidatorProperty}

\func{wxProperty *}{GetValidatorProperty}{\void}

Gets the property for the validator.

\membersection{wxPropertyValidator::SetValidatorProperty}

\func{void}{SetValidatorProperty}{\param{wxProperty *}{property}}

Sets the property for the validator.


\section{\class{wxPropertyValidatorRegistry}: wxHashTable}\label{wxpropertyvalidatorregistry}

The {\bf wxPropertyValidatorRegistry} class is used for storing validators,
indexed by the `role name' of the property, by which groups of property
can be identified for the purpose of validation and editing.

\membersection{wxPropertyValidatorRegistry::wxPropertyValidatorRegistry}

\func{void}{wxPropertyValidatorRegistry}{\void}

Constructor.

\membersection{wxPropertyValidatorRegistry::\destruct{wxPropertyValidatorRegistry}}

\func{void}{\destruct{wxPropertyValidatorRegistry}}{\void}

Destructor.

\membersection{wxPropertyValidatorRegistry::Clear}

\func{void}{ClearRegistry}{\void}

Clears the registry, deleting the validators.

\membersection{wxPropertyValidatorRegistry::GetValidator}

\func{wxPropertyValidator *}{GetValidator}{\param{wxString\& }{roleName}}

Retrieve a validator by the property role name.

\membersection{wxPropertyValidatorRegistry::RegisterValidator}\label{wxpropertyvalidatorregistervalidator}

\func{void}{RegisterValidator}{\param{wxString\& }{roleName}, \param{wxPropertyValidator *}{validator}}

Register a validator with the registry. {\it roleName} is a name indicating the
role of the property, such as ``filename''. Later, when a validator is chosen for
editing a property, this role name is matched against the class names of the property,
if the property does not already have a validator explicitly associated with it.


\section{\class{wxPropertyValue}: wxObject}\label{wxpropertyvalue}

The {\bf wxPropertyValue} class represents the value of a property,
and is normally associated with a wxProperty object.

A wxPropertyValue has one of the following types:

\begin{itemize}\itemsep=0pt
\item wxPropertyValueNull
\item wxPropertyValueInteger
\item wxPropertyValueReal
\item wxPropertyValueBool
\item wxPropertyValueString
\item wxPropertyValueList
\item wxPropertyValueIntegerPtr
\item wxPropertyValueRealPtr
\item wxPropertyValueBoolPtr
\item wxPropertyValueStringPtr
\end{itemize}

\membersection{wxPropertyValue::wxPropertyValue}

\func{void}{wxPropertyValue}{\void}

Default constructor.

\func{void}{wxPropertyValue}{\param{const wxPropertyValue\& }{copyFrom}}

Copy constructor.

\func{void}{wxPropertyValue}{\param{char *}{val}}

Construction from a string value.

\func{void}{wxPropertyValue}{\param{long}{ val}}

Construction from an integer value. You may need to cast to (long) to
avoid confusion with other constructors (such as the Bool constructor).

\func{void}{wxPropertyValue}{\param{Bool}{ val}}

Construction from a boolean value.

\func{void}{wxPropertyValue}{\param{float}{ val}}

Construction from a floating point value.

\func{void}{wxPropertyValue}{\param{double}{ val}}

Construction from a floating point value.

\func{void}{wxPropertyValue}{\param{wxList *}{ val}}

Construction from a list of wxPropertyValue objects. The
list, but not each contained wxPropertyValue, will be deleted
by the constructor. The wxPropertyValues will be assigned to
this wxPropertyValue list. In other words, so do not delete wxList or
its data after calling this constructor.

\func{void}{wxPropertyValue}{\param{wxStringList *}{ val}}

Construction from a list of strings. The list (including the strings
contained in it) will be deleted by the constructor, so do not
destroy {\it val} explicitly.

\func{void}{wxPropertyValue}{\param{char **}{val}}

Construction from a string pointer.

\func{void}{wxPropertyValue}{\param{long *}{val}}

Construction from an integer pointer.

\func{void}{wxPropertyValue}{\param{Bool *}{val}}

Construction from an boolean pointer.

\func{void}{wxPropertyValue}{\param{float *}{val}}

Construction from a floating point pointer.

The last four constructors use pointers to various C++ types, and do not
store the types themselves; this allows the values to stand in for actual
data values defined elsewhere.

\membersection{wxPropertyValue::\destruct{wxPropertyValue}}

\func{void}{\destruct{wxPropertyValue}}{\void}

Destructor.

\membersection{wxPropertyValue::Append}

\func{void}{Append}{\param{wxPropertyValue *}{expr}}

Appends a property value to the list.

\membersection{wxPropertyValue::BoolValue}

\func{Bool}{BoolValue}{\void}

Returns the boolean value.

\membersection{wxPropertyValue::BoolValuePtr}

\func{Bool *}{BoolValuePtr}{\void}

Returns the pointer to the boolean value.

\membersection{wxPropertyValue::ClearList}

\func{void}{ClearList}{\void}

Deletes the contents of the list.

\membersection{wxPropertyValue::Delete}

\func{void}{Delete}{\param{wxPropertyValue *}{expr}}

Deletes {\it expr} from this list.

\membersection{wxPropertyValue::GetFirst}

\func{wxPropertyValue *}{GetFirst}{\void}

Gets the first value in the list.

\membersection{wxPropertyValue::GetLast}

\func{wxPropertyValue *}{GetFirst}{\void}

Gets the last value in the list.

\membersection{wxPropertyValue::GetModified}

\func{Bool}{GetModified}{\void}

Returns TRUE if the value was modified since being created
(or since SetModified was called).

\membersection{wxPropertyValue::GetNext}

\func{wxPropertyValue *}{GetNext}{\void}

Gets the next value in the list (the one after `this').

\membersection{wxPropertyValue::GetStringRepresentation}

\func{wxString}{GetStringRepresentation}{\void}

Gets a string representation of the value.

\membersection{wxPropertyValue::IntegerValue}

\func{long}{IntegerValue}{\void}

Returns the integer value.

\membersection{wxPropertyValue::Insert}

\func{void}{Insert}{\param{wxPropertyValue *}{expr}}

Inserts a property value at the front of a list.

\membersection{wxPropertyValue::IntegerValuePtr}

\func{long *}{IntegerValuePtr}{\void}

Returns the pointer to the integer value.

\membersection{wxPropertyValue::Nth}

\func{wxPropertyValue *}{Nth}{\param{int}{ n}}

Returns the nth value of a list expression (starting from zero).

\membersection{wxPropertyValue::Number}

\func{int}{Number}{\void}

Returns the number of elements in a list expression.

\membersection{wxPropertyValue::RealValue}

\func{float}{RealValue}{\void}

Returns the floating point value.

\membersection{wxPropertyValue::RealValuePtr}

\func{float *}{RealValuePtr}{\void}

Returns the pointer to the floating point value.

\membersection{wxPropertyValue::SetModified}

\func{void}{SetModified}{\param{Bool}{ flag}}

Sets the `modified' flag.

\membersection{wxPropertyValue::StringValue}

\func{char *}{StringValue}{\void}

Returns the string value.

\membersection{wxPropertyValue::StringValuePtr}

\func{char **}{StringValuePtr}{\void}

Returns the pointer to the string value.

\membersection{wxPropertyValue::Type}

\func{wxPropertyValueType}{Type}{\void}

Returns the value type.

\membersection{wxPropertyValue::operator $=$}

\func{void}{operator $=$}{\param{const wxPropertyValue\& }{val}}

\func{void}{operator $=$}{\param{const char *}{val}}

\func{void}{operator $=$}{\param{const long }{val}}

\func{void}{operator $=$}{\param{const Bool }{val}}

\func{void}{operator $=$}{\param{const float }{val}}

\func{void}{operator $=$}{\param{const char **}{val}}

\func{void}{operator $=$}{\param{const long *}{val}}

\func{void}{operator $=$}{\param{const Bool *}{val}}

\func{void}{operator $=$}{\param{const float *}{val}}

Assignment operators.



\section{\class{wxPropertyView}: wxEvtHandler}\label{wxpropertyview}

\overview{wxPropertyView overview}{wxpropertyviewoverview}

The {\bf wxPropertyView} abstract class is the base class for views
of property sheets, acting as intermediaries between properties and
actual windows.

\membersection{wxPropertyView::wxPropertyView}

\func{void}{wxPropertyView}{\param{long}{ flags = wxPROP\_BUTTON\_DEFAULT}}

Constructor.

The {\it flags} argument can be a bit list of the following:

\begin{itemize}\itemsep=0pt
\item wxPROP\_BUTTON\_CLOSE
\item wxPROP\_BUTTON\_OK
\item wxPROP\_BUTTON\_CANCEL
\item wxPROP\_BUTTON\_CHECK\_CROSS
\item wxPROP\_BUTTON\_HELP
\item wxPROP\_DYNAMIC\_VALUE\_FIELD
\item wxPROP\_PULLDOWN
\end{itemize}

\membersection{wxPropertyView::\destruct{wxPropertyView}}

\func{void}{\destruct{wxPropertyView}}{\void}

Destructor.

\membersection{wxPropertyView::AddRegistry}\label{wxpropertyviewaddregistry}

\func{void}{AddRegistry}{\param{wxPropertyValidatorRegistry *}{registry}}

Adds a registry (list of property validators) the view's list of registries, which is initially empty.

\membersection{wxPropertyView::FindPropertyValidator}\label{wxpropertyviewfindpropertyvalidator}

\func{wxPropertyValidator *}{FindPropertyValidator}{\param{wxProperty *}{property}}

Finds the property validator that is most appropriate to this property.

\membersection{wxPropertyView::GetPropertySheet}\label{wxpropertyviewgetpropertysheet}

\func{wxPropertySheet *}{GetPropertySheet}{\void}

Gets the property sheet for this view.

\membersection{wxPropertyView::GetRegistryList}\label{wxpropertyviewgetregistrylist}

\func{wxList\&}{GetRegistryList}{\void}

Returns a reference to the list of property validator registries.

\membersection{wxPropertyView::OnOk}\label{wxpropertyviewonok}

\func{void}{OnOk}{\void}

Virtual function that will be called when the OK button on the physical window is pressed (if it exists).

\membersection{wxPropertyView::OnCancel}\label{wxpropertyviewoncancel}

\func{void}{OnCancel}{\void}

Virtual function that will be called when the Cancel button on the physical window is pressed (if it exists).

\membersection{wxPropertyView::OnClose}\label{wxpropertyviewonclose}

\func{Bool}{OnClose}{\void}

Virtual function that will be called when the physical window is closed. The default implementation returns FALSE.

\membersection{wxPropertyView::OnHelp}\label{wxpropertyviewonhelp}

\func{void}{OnHelp}{\void}

Virtual function that will be called when the Help button on the physical window is pressed (if it exists).

\membersection{wxPropertyView::OnPropertyChanged}\label{wxpropertyviewonpropertychanged}

\func{void}{OnPropertyChanged}{\param{wxProperty *}{property}}

Virtual function called by a view or validator when a property's value changed. Validators
must be written correctly for this to be called. You can override this function
to respond immediately to property value changes.

\membersection{wxPropertyView::OnUpdateView}\label{wxpropertyviewonupdateview}

\func{Bool}{OnUpdateView}{\void}

Called by the viewed object to update the view. The default implementation just returns
FALSE.

\membersection{wxPropertyView::SetPropertySheet}\label{wxpropertyviewsetpropertysheet}

\func{void}{SetPropertySheet}{\param{wxPropertySheet *}{sheet}}

Sets the property sheet for this view.

\membersection{wxPropertyView::ShowView}\label{wxpropertyviewshowview}

\func{void}{ShowView}{\param{wxPropertySheet *}{sheet}, \param{wxPanel *}{panel}}

Associates this view with the given panel, and shows the view.

\section{\class{wxRealFormValidator}: wxPropertyFormValidator}\label{wxrealformvalidator}

\overview{Validator classes}{validatorclasses}

This class validates a range of real values for form views. The associated panel item must be a wxText.

\membersection{wxRealFormValidator::wxRealFormValidator}

\func{void}{wxRealFormValidator}{\param{float }{min=0.0}, \param{float }{max=0.0},
 \param{long}{ flags=0}}

Constructor. Assigning zero to minimum and maximum values indicates that there is no range to check.


\section{\class{wxStringFormValidator}: wxPropertyFormValidator}\label{wxstringformvalidator}

\overview{Validator classes}{validatorclasses}

This class validates a string value for a form view, with an optional choice of possible values.
The associated panel item must be a wxText, wxListBox or wxChoice. For wxListBox and wxChoice items,
if the item is empty, the validator attempts to initialize the item from the strings in
the validator. Note that this does not happen for XView wxChoice items since XView cannot reinitialize a wxChoice.
 
\membersection{wxStringFormValidator::wxStringFormValidator}

\func{void}{wxStringFormValidator}{\param{wxStringList *}{list=NULL}, \param{long}{ flags=0}}

Constructor. Supply a list of strings to indicate a choice, or no strings to allow the
user to freely edit the string. The string list will be deleted when the validator is deleted.


\section{\class{wxRealListValidator}: wxPropertyListValidator}\label{wxreallistvalidator}

\overview{Validator classes}{validatorclasses}

This class validates a range of real values for property lists.

\membersection{wxRealListValidator::wxreallistvalidator}

\func{void}{wxRealListValidator}{\param{float }{min=0.0}, \param{float }{max=0.0},
 \param{long}{ flags=wxPROP\_ALLOW\_TEXT\_EDITING}}

Constructor. Assigning zero to minimum and maximum values indicates that there is no range to check.


\section{\class{wxStringListValidator}: wxPropertyListValidator}\label{wxstringlistvalidator}

\overview{Validator classes}{validatorclasses}

This class validates a string value, with an optional choice of possible values.

\membersection{wxStringListValidator::wxStringListValidator}

\func{void}{wxStringListValidator}{\param{wxStringList *}{list=NULL}, \param{long}{ flags=0}}

Constructor. Supply a list of strings to indicate a choice, or no strings to allow the
user to freely edit the string. The string list will be deleted when the validator is deleted.


\chapter{Classes by category}\label{classesbycat}

A classification of property sheet classes by category.

\section{Data classes}

\begin{itemize}\itemsep=0pt
\item \helpref{wxProperty}{wxproperty}
\item \helpref{wxPropertyValue}{wxpropertyvalue}
\item \helpref{wxPropertySheet}{wxpropertysheet}
\end{itemize}


\section{Validator classes}\label{validatorclasses}

Validators check that the values the user has entered for a property are
valid. They can also define specific ways of entering data, such as a
file selector for a filename, and they are responsible for transferring
values between the wxProperty and the physical display. 

Base classes:

\begin{itemize}\itemsep=0pt
\item \helpref{wxPropertyValidator}{wxproperty}
\item \helpref{wxPropertyListValidator}{wxpropertylistvalidator}
\item \helpref{wxPropertyFormValidator}{wxpropertyformvalidator}
\end{itemize}

List view validators:

\begin{itemize}\itemsep=0pt
\item \helpref{wxBoolListValidator}{wxboollistvalidator}
\item \helpref{wxFilenameListValidator}{wxfilenamelistvalidator}
\item \helpref{wxIntegerListValidator}{wxintegerlistvalidator}
\item \helpref{wxListOfStringsListValidator}{wxlistofstringslistvalidator}
\item \helpref{wxRealListValidator}{wxreallistvalidator}
\item \helpref{wxStringListValidator}{wxstringlistvalidator}
\end{itemize}

Form view validators:

\begin{itemize}\itemsep=0pt
\item \helpref{wxBoolFormValidator}{wxboolformvalidator}
\item \helpref{wxIntegerFormValidator}{wxintegerformvalidator}
\item \helpref{wxRealFormValidator}{wxrealformvalidator}
\item \helpref{wxStringFormValidator}{wxstringformvalidator}
\end{itemize}

\section{View classes}\label{viewclasses}

View classes mediate between a property sheet and a physical window.

\begin{itemize}\itemsep=0pt
\item \helpref{wxPropertyView}{wxpropertyview}
\item \helpref{wxPropertyListView}{wxpropertylistview}
\item \helpref{wxPropertyFormView}{wxpropertyformview}
\end{itemize}

\section{Window classes}\label{windowclasses}

The class library defines some window classes that can be used as-is with a suitable
view class and property sheet.

\begin{itemize}\itemsep=0pt
\item \helpref{wxPropertyFormFrame}{wxpropertyformframe}
\item \helpref{wxPropertyFormDialog}{wxpropertyformdialog}
\item \helpref{wxPropertyFormPanel}{wxpropertyformpanel}
\item \helpref{wxPropertyListFrame}{wxpropertylistframe}
\item \helpref{wxPropertyListDialog}{wxpropertylistdialog}
\item \helpref{wxPropertyListPanel}{wxpropertylistpanel}
\end{itemize}

\section{Registry classes}

A validator registry is a list of validators that can be applied to properties in a property sheet.
There may be one or more registries per property view.

\begin{itemize}\itemsep=0pt
\item \helpref{wxPropertyValidatorRegistry}{wxpropertyvalidatorregistry}
\end{itemize}


\chapter{Topic overviews}\label{overviews}

This chapter contains a selection of topic overviews.

\section{Property classes overview}\label{propertyoverview}

The property classes help a programmer to express relationships between
data and physical windows, in particular:

\begin{itemize}\itemsep=0pt
\item the transfer of data to and from the physical controls;
\item the behaviour of various controls and custom windows for particular
types of data;
\item the validation of data, notifying the user when incorrect data is entered,
or even better, constraining the input so only valid data can be entered.
\end{itemize}

With a consistent framework, the programmer should be able to use existing
components and design new ones in a principled manner, to solve many data entry
requirements.

Each datum is represented in a \helpref{wxProperty}{wxproperty}, which has a name and a value.
Various C++ types are permitted in the value of a property, and the property can store a pointer
to the data instead of a copy of the data. A \helpref{wxPropertySheet}{wxpropertysheet} represents a number of these properties.

These two classes are independent from the way in which the data is visually manipulated. To
mediate between property sheets and windows, the abstract class \helpref{wxPropertyView}{wxpropertyview} is
available for programmers to derive new kinds of view. One kind of view that is available is the \helpref{wxPropertyListView}{wxpropertylistview},
which displays the data in a Visual Basic-style list, with a small number of controls for editing
the currently selected property. Another is \helpref{wxPropertyFormView}{wxpropertyformview} which
mediates between an existing dialog or panel and the property sheet.

The hard work of mediation is actually performed by validators, which are instances of classes
derived from \helpref{wxPropertyValidator}{wxpropertyvalidator}. A validator is associated with
a particular property and is responsible for
responding to user interface events, and displaying, updating and checking the property value.
Because a validator's behaviour depends largely on the kind of view being used, there has to be
a separate hierarchy of validators for each class of view. So for wxPropertyListView, there is
an abstract class \helpref{wxPropertyListValidator}{wxpropertylistvalidator} from which concrete
classes are derived, such as \helpref{wxRealListValidator}{wxreallistvalidator} and
\rtfsp\helpref{wxStringListValidator}{wxstringlistvalidator}.

A validator can be explicitly set for a property, so there is no doubt which validator
should be used to edit that property. However, it is also possible to define a registry
of validators, and have the validator chosen on the basis of the {\it role} of the property.
So a property with a ``filename" role would match the ``filename" validator, which pops
up a file selector when the user double clicks on the property.

You don't have to define your own frame or window classes: there are some predefined
that will work with the property list view. See \helpref{Window classes}{windowclasses} for
further details.

\subsection{Example 1: Property list view}

The following code fragment shows the essentials of creating a registry of
standard validators, a property sheet containing some properties, and
a property list view and dialog or frame. RegisterValidators will be
called on program start, and PropertySheetTest is called in response to a
menu command.

Note how some properties are created with an explicit reference to
a validator, and others are provided with a ``role'' which can be matched
against a validator in the registry.

The interface generated by this test program is shown in the section \helpref{Appearance and
behaviour of a property list view}{appearance}.

\begin{verbatim}
void RegisterValidators(void)
{
  myListValidatorRegistry.RegisterValidator((wxString)"real", new wxRealListValidator);
  myListValidatorRegistry.RegisterValidator((wxString)"string", new wxStringListValidator);
  myListValidatorRegistry.RegisterValidator((wxString)"integer", new wxIntegerListValidator);
  myListValidatorRegistry.RegisterValidator((wxString)"bool", new wxBoolListValidator);
}

void PropertyListTest(Bool useDialog)
{
  wxPropertySheet *sheet = new wxPropertySheet;

  sheet->AddProperty(new wxProperty("fred", 1.0, "real"));
  sheet->AddProperty(new wxProperty("tough choice", (Bool)TRUE, "bool"));
  sheet->AddProperty(new wxProperty("ian", (long)45, "integer", new wxIntegerListValidator(-50, 50)));
  sheet->AddProperty(new wxProperty("bill", 25.0, "real", new wxRealListValidator(0.0, 100.0)));
  sheet->AddProperty(new wxProperty("julian", "one", "string"));
  sheet->AddProperty(new wxProperty("bitmap", "none", "string", new wxFilenameListValidator("Select a bitmap file", "*.bmp")));
  wxStringList *strings = new wxStringList("one", "two", "three", NULL);
  sheet->AddProperty(new wxProperty("constrained", "one", "string", new wxStringListValidator(strings)));

  wxPropertyListView *view =
    new wxPropertyListView(NULL,
     wxPROP_BUTTON_CHECK_CROSS|wxPROP_DYNAMIC_VALUE_FIELD|wxPROP_PULLDOWN);

  wxDialogBox *propDialog = NULL;
  wxPropertyListFrame *propFrame = NULL;
  if (useDialog)
  {
    propDialog = new wxPropertyListDialog(view, NULL, "Property Sheet Test", TRUE, -1, -1, 400, 500);
  }
  else
  {
    propFrame = new wxPropertyListFrame(view, NULL, "Property Sheet Test", -1, -1, 400, 500);
  }
  
  view->AddRegistry(&myListValidatorRegistry);

  if (useDialog)
  {
    view->ShowView(sheet, propDialog);
    propDialog->Centre(wxBOTH);
    propDialog->Show(TRUE);
  }
  else
  {
    propFrame->Initialize();
    view->ShowView(sheet, propFrame->GetPropertyPanel());
    propFrame->Centre(wxBOTH);
    propFrame->Show(TRUE);
  }
}
\end{verbatim}

\subsection{Example 2: Property form view}

This example is similar to Example 1, but uses a property form view to
edit a property sheet using a predefined dialog box.

\begin{verbatim}
void RegisterValidators(void)
{
  myFormValidatorRegistry.RegisterValidator((wxString)"real", new wxRealFormValidator);
  myFormValidatorRegistry.RegisterValidator((wxString)"string", new wxStringFormValidator);
  myFormValidatorRegistry.RegisterValidator((wxString)"integer", new wxIntegerFormValidator);
  myFormValidatorRegistry.RegisterValidator((wxString)"bool", new wxBoolFormValidator);
}

void PropertyFormTest(Bool useDialog)
{
  wxPropertySheet *sheet = new wxPropertySheet;

  sheet->AddProperty(new wxProperty("fred", 25.0, "real", new wxRealFormValidator(0.0, 100.0)));
  sheet->AddProperty(new wxProperty("tough choice", (Bool)TRUE, "bool"));
  sheet->AddProperty(new wxProperty("ian", (long)45, "integer", new wxIntegerFormValidator(-50, 50)));
  sheet->AddProperty(new wxProperty("julian", "one", "string"));
  wxStringList *strings = new wxStringList("one", "two", "three", NULL);
  sheet->AddProperty(new wxProperty("constrained", "one", "string", new wxStringFormValidator(strings)));

  wxPropertyFormView *view = new wxPropertyFormView(NULL);

  wxDialogBox *propDialog = NULL;
  wxPropertyFormFrame *propFrame = NULL;
  if (useDialog)
  {
    propDialog = new wxPropertyFormDialog(view, NULL, "Property Form Test", TRUE, -1, -1, 400, 300);
  }
  else
  {
    propFrame = new wxPropertyFormFrame(view, NULL, "Property Form Test", -1, -1, 400, 300);
    propFrame->Initialize();
  }
  
  wxPanel *panel = propDialog ? propDialog : propFrame->GetPropertyPanel();
  panel->SetLabelPosition(wxVERTICAL);
  
  // Add items to the panel
  
  (void) new wxButton(panel, (wxFunction)NULL, "OK", -1, -1, -1, -1, 0, "ok");
  (void) new wxButton(panel, (wxFunction)NULL, "Cancel", -1, -1, 80, -1, 0, "cancel");
  (void) new wxButton(panel, (wxFunction)NULL, "Update", -1, -1, 80, -1, 0, "update");
  (void) new wxButton(panel, (wxFunction)NULL, "Revert", -1, -1, -1, -1, 0, "revert");
  panel->NewLine();
  
  // The name of this text item matches the "fred" property
  (void) new wxText(panel, (wxFunction)NULL, "Fred", "", -1, -1, 90, -1, 0, "fred");
  (void) new wxCheckBox(panel, (wxFunction)NULL, "Yes or no", -1, -1, -1, -1, 0, "tough choice");
  (void) new wxSlider(panel, (wxFunction)NULL, "Sliding scale", 0, -50, 50, 100, -1, -1, wxHORIZONTAL, "ian");
  panel->NewLine();
  (void) new wxListBox(panel, (wxFunction)NULL, "Constrained", wxSINGLE, -1, -1, 100, 90, 0, NULL, 0, "constrained");

  view->AddRegistry(&myFormValidatorRegistry);

  if (useDialog)
  {
    view->ShowView(sheet, propDialog);
    view->AssociateNames();
    view->TransferToDialog();
    propDialog->Centre(wxBOTH);
    propDialog->Show(TRUE);
  }
  else
  {
    view->ShowView(sheet, propFrame->GetPropertyPanel());
    view->AssociateNames();
    view->TransferToDialog();
    propFrame->Centre(wxBOTH);
    propFrame->Show(TRUE);
  }
}
\end{verbatim}

\section{Validator classes overview}\label{validatoroverview}

Classes: \helpref{Validator classes}{validatorclasses}

The validator classes provide functionality for mediating between a wxProperty and
the actual display. There is a separate family of validator classes for each
class of view, since the differences in user interface for these views implies
that little common functionality can be shared amongst validators.

\subsection{wxPropertyValidator overview}\label{wxpropertyvalidatoroverview}

Class: \helpref{wxPropertyValidator}{wxpropertyvalidator}

This class is the root of all property validator classes. It contains a small
amount of common functionality, including functions to convert between
strings and C++ values.

A validator is notionally an object which sits between a property and its displayed
value, and checks that the value the user enters is correct, giving an error message
if the validation fails. In fact, the validator does more than that, and is akin to
a view class but at a finer level of detail. It is also responsible for
loading the dialog box control with the value from the property, putting it back
into the property, preparing special controls for editing the value, and
may even invoke special dialogs for editing the value in a convenient way.

In a property list dialog, there is quite a lot of scope for supplying custom dialogs,
such as file or colour selectors. For a form dialog, there is less scope because
there is no concept of `detailed editing' of a value: one control is associated with
one property, and there is no provision for invoking further dialogs. The reader
may like to work out how the form view could be extended to provide some of the
functionality of the property list!

Validator objects may be associated explictly with a wxProperty, or they may be
indirectly associated by virtue of a property `kind' that matches validators having
that kind. In the latter case, such validators are stored in a validator registry
which is passed to the view before the dialog is shown. If the validator takes
arguments, such as minimum and maximum values in the case of a wxIntegerListValidator,
then the validator must be associated explicitly with the property. The validator
will be deleted when the property is deleted.

\subsection{wxPropertyListValidator overview}\label{wxpropertylistvalidatoroverview}

Class: \helpref{wxPropertyListValidator}{wxpropertylistvalidator}

This class is the abstract base class for property list view validators.
The list view acts upon a user interface containing a list of properties,
a text item for direct property value editing, confirm/cancel buttons for the value,
a pulldown list for making a choice between values, and OK/Cancel/Help buttons
for the dialog (see \helpref{property list appearance}{appearance}).

By overriding virtual functions, the programmer can create custom
behaviour for different kinds of property. Custom behaviour can use just the
available controls on the property list dialog, or the validator can
invoke custom editors with quite different controls, which pop up in
`detailed editing' mode.

See the detailed class documentation for the members you should override
to give your validator appropriate behaviour.

\subsection{wxPropertyFormValidator overview}\label{wxpropertyformvalidatoroverview}

This class is the abstract base class for property form view validators.
The form view acts upon an existing dialog box or panel, where either the
panel item names correspond to property names, or the programmer has explicitly
associated the panel item with the property.

By overriding virtual functions, the programmer determines how
values are displayed or retrieved, and the checking that the validator does.

See the detailed class documentation for the members you should override
to give your validator appropriate behaviour.

\section{View classes overview}\label{viewoverview}

Classes: \helpref{View classes}{viewclasses}

An instance of a view class relates a property sheet with an actual window.
Currently, there are two classes of view: wxPropertyListView and wxPropertyFormView.

\subsection{wxPropertyView overview}\label{wxpropertyviewoverview}

Class: \helpref{wxPropertyView}{wxpropertyview}

This is the abstract base class for property views.

\subsection{wxPropertyListView overview}\label{wxpropertylistviewoverview}

Class: \helpref{wxPropertyListView}{wxpropertylistview}

The property list view defines the relationship between a property sheet and
a property list dialog or panel. It manages user interface events such as
clicking on a property, pressing return in the text edit field, and clicking
on Confirm or Cancel. These events cause member functions of the
class to be called, and these in turn may call member functions of
the appropriate validator to be called, to prepare controls, check the property value,
invoke detailed editing, etc.

\subsection{wxPropertyFormView overview}\label{wxpropertyformviewoverview}

Class: \helpref{wxPropertyFormView}{wxpropertyformview}

The property form view manages the relationship between a property sheet
and an existing dialog or panel.

You must first create a panel or dialog box for the view to work on.
The panel should contain panel items with names that correspond to
properties in your property sheet; or you can explicitly set the
panel item for each property.

Apart from any custom panel items that you wish to control independently
of the property-editing items, wxPropertyFormView takes over the
processing of item events. It can also control normal dialog behaviour such
as OK, Cancel, so you should also create some standard buttons that the property view
can recognise. Just create the buttons with standard names and the view
will do the rest. The following button names are recognised:

\begin{itemize}\itemsep=0pt
\item {\bf ok}: indicates the OK button. Calls wxPropertyFormView::OnOk. By default,
checks and updates the form values, closes and deletes the frame or dialog, then deletes the view.
\item {\bf cancel}: indicates the Cancel button. Calls wxPropertyFormView::OnCancel. By default,
closes and deletes the frame or dialog, then deletes the view.
\item {\bf help}: indicates the Help button. Calls wxPropertyFormView::OnHelp. This needs
to be overridden by the application for anything interesting to happen.
\item {\bf revert}: indicates the Revert button. Calls wxPropertyFormView::OnRevert,
which by default transfers the wxProperty values to the panel items (in effect
undoing any unsaved changes in the items).
\item {\bf update}: indicates the Revert button. Calls wxPropertyFormView::OnUpdate, which
by defaults transfers the displayed values to the wxProperty objects.
\end{itemize}

\section{wxPropertySheet overview}\label{wxpropertysheetoverview}

Classes: \helpref{wxPropertySheet}{wxpropertysheet}, \helpref{wxProperty}{wxproperty}, \helpref{wxPropertyValue}{wxpropertyvalue}

A property sheet defines zero or more properties. This is a bit like an explicit representation of
a C++ object. wxProperty objects can have values which are pointers to C++ values, or they
can allocate their own storage for values.

Because the property sheet representation is explicit and can be manipulated by
a program, it is a convenient form to be used for a variety of
editing purposes. wxPropertyListView and wxPropertyFormView are two classes that
specify the relationship between a property sheet and a user interface. You could imagine
other uses for wxPropertySheet, for example to generate a form-like user interface without
the need for GUI programming. Or for storing the names and values of command-line switches, with the
option to subsequently edit these values using a wxPropertyListView.

A typical use for a property sheet is to represent values of an object
which are only implicit in the current representation of it. For
example, in Visual Basic and similar programming environments, you can
`edit a button', or rather, edit the button's properties.  One of the
properties you can edit is {\it width} - but there is no explicit
representation of width in a wxWindows button; instead, you call SetSize
and GetSize members. To translate this into a consisent,
property-oriented scheme, we could derive a new class
wxButtonWithProperties, which has two new functions: SetProperty and
GetProperty.  SetProperty accepts a property name and a value, and calls
an appropriate function for the property that is being passed.
GetProperty accepts a property name, returning a property value. So
instead of having to use the usual arbitrary set of C++ member functions
to set or access attributes of a window, programmer deals merely with
SetValue/GetValue, and property names and values.
We now have a single point at which we can modify or query an object by specifying
names and values at run-time. (The implementation of SetProperty and GetProperty
is probably quite messy and involves a large if-then-else statement to
test the property name and act accordingly.)

When the user invokes the property editor for a wxButtonWithProperties, the system
creates a wxPropertySheet with `imaginary' properties such as width, height, font size
and so on. For each property, wxButtonWithProperties::GetProperty is called, and the result is
passed to the corresponding wxProperty. The wxPropertySheet is passed to a wxPropertyListView
as described elsewhere, and the user edits away. When the user has finished editing, the system calls
wxButtonWithProperties::SetProperty to transfer the wxProperty value back into the button
by way of an appropriate call, wxWindow::SetSize in the case of width and height properties.



