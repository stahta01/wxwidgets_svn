\section{\class{wxFileType}}\label{wxfiletype}

This class holds information about a given {\it file type}. File type is the same as
MIME type under Unix, but under Windows it corresponds more to an extension than
to MIME type (in fact, several extensions may correspond to a file type). This
object may be created in several different ways: the program might know the file
extension and wish to find out the corresponding MIME type or, conversely, it
might want to find the right extension for the file to which it writes the
contents of given MIME type. Depending on how it was created some fields may be
unknown so the return value of all the accessors {\bf must} be checked: {\tt false}
will be returned if the corresponding information couldn't be found.

The objects of this class are never created by the application code but are
returned by \helpref{wxMimeTypesManager::GetFileTypeFromMimeType}{wxmimetypesmanagergetfiletypefrommimetype} and 
\helpref{wxMimeTypesManager::GetFileTypeFromExtension}{wxmimetypesmanagergetfiletypefromextension} methods.
But it is your responsibility to delete the returned pointer when you're done
with it!

% TODO describe MIME types better than this...
A brief reminder about what the MIME types are (see the RFC 1341 for more
information): basically, it is just a pair category/type (for example,
"text/plain") where the category is a basic indication of what a file is.
Examples of categories are "application", "image", "text", "binary", and
type is a precise definition of the document format: "plain" in the example
above means just ASCII text without any formatting, while "text/html" is the
HTML document source.

A MIME type may have one or more associated extensions: "text/plain" will
typically correspond to the extension ".txt", but may as well be associated with
".ini" or ".conf".

\wxheading{Derived from}

None

\wxheading{Include files}

<wx/mimetype.h>

\wxheading{See also}

\helpref{wxMimeTypesManager}{wxmimetypesmanager}

\latexignore{\rtfignore{\wxheading{Members}}}

\membersection{MessageParameters class}\label{wxfiletypemessageparameters}

One of the most common usages of MIME is to encode an e-mail message. The MIME
type of the encoded message is an example of a {\it message parameter}. These
parameters are found in the message headers ("Content-XXX"). At the very least,
they must specify the MIME type and the version of MIME used, but almost always
they provide additional information about the message such as the original file
name or the charset (for the text documents).

These parameters may be useful to the program used to open, edit, view or print
the message, so, for example, an e-mail client program will have to pass them to
this program. Because wxFileType itself can not know about these parameters,
it uses MessageParameters class to query them. The default implementation only
requires the caller to provide the file name (always used by the program to be
called - it must know which file to open) and the MIME type and supposes that
there are no other parameters. If you wish to supply additional parameters, you
must derive your own class from MessageParameters and override GetParamValue()
function, for example:

\begin{verbatim}
// provide the message parameters for the MIME type manager
class MailMessageParameters : public wxFileType::MessageParameters
{
public:
   MailMessageParameters(const wxString& filename,
                         const wxString& mimetype)
      : wxFileType::MessageParameters(filename, mimetype)
   {
   }

   virtual wxString GetParamValue(const wxString& name) const
   {
       // parameter names are not case-sensitive
       if ( name.CmpNoCase("charset") == 0 )
           return "US-ASCII";
       else
           return wxFileType::MessageParameters::GetParamValue(name);
   }
};
\end{verbatim}

Now you only need to create an object of this class and pass it to, for example,
\rtfsp\helpref{GetOpenCommand}{wxfiletypegetopencommand} like this:

\begin{verbatim}
wxString command;
if ( filetype->GetOpenCommand(&command,
                              MailMessageParameters("foo.txt", "text/plain")) )
{
    // the full command for opening the text documents is in 'command'
    // (it might be "notepad foo.txt" under Windows or "cat foo.txt" under Unix)
}
else
{
    // we don't know how to handle such files...
}
\end{verbatim}

{\bf Windows:} As only the file name is used by the program associated with the
given extension anyhow (but no other message parameters), there is no need to
ever derive from MessageParameters class for a Windows-only program.

\membersection{wxFileType::wxFileType}\label{wxfiletypewxfiletype}

\func{}{wxFileType}{\void}

The default constructor is private because you should never create objects of
this type: they are only returned by \helpref{wxMimeTypesManager}{wxmimetypesmanager} methods.

\membersection{wxFileType::\destruct{wxFileType}}\label{wxfiletypedtor}

\func{}{\destruct{wxFileType}}{\void}

The destructor of this class is not virtual, so it should not be derived from.

\membersection{wxFileType::GetMimeType}\label{wxfiletypegetmimetype}

\func{bool}{GetMimeType}{\param{wxString*}{ mimeType}}

If the function returns {\tt true}, the string pointed to by {\it mimeType} is filled
with full MIME type specification for this file type: for example, "text/plain".

\membersection{wxFileType::GetMimeTypes}\label{wxfiletypegetmimetypes}

\func{bool}{GetMimeType}{\param{wxArrayString\&}{ mimeTypes}}

Same as \helpref{GetMimeType}{wxfiletypegetmimetype} but returns array of MIME
types. This array will contain only one item in most cases but sometimes,
notably under Unix with KDE, may contain more MIME types. This happens when
one file extension is mapped to different MIME types by KDE, mailcap and
mime.types.

\membersection{wxFileType::GetExtensions}\label{wxfiletypegetextensions}

\func{bool}{GetExtensions}{\param{wxArrayString\&}{ extensions}}

If the function returns {\tt true}, the array {\it extensions} is filled
with all extensions associated with this file type: for example, it may
contain the following two elements for the MIME type "text/html" (notice the
absence of the leading dot): "html" and "htm".

{\bf Windows:} This function is currently not implemented: there is no
(efficient) way to retrieve associated extensions from the given MIME type on
this platform, so it will only return {\tt true} if the wxFileType object was created
by \helpref{GetFileTypeFromExtension}{wxmimetypesmanagergetfiletypefromextension} 
function in the first place.

\membersection{wxFileType::GetIcon}\label{wxfiletypegeticon}

\func{bool}{GetIcon}{\param{wxIconLocation *}{ iconLoc}}

If the function returns {\tt true}, the {\tt iconLoc} is filled with the
location of the icon for this MIME type. A \helpref{wxIcon}{wxicon} may be
created from {\it iconLoc} later.

{\bf Windows:} The function returns the icon shown by Explorer for the files of
the specified type.

{\bf Mac:} This function is not implemented and always returns {\tt false}.

{\bf Unix:} MIME manager gathers information about icons from GNOME
and KDE settings and thus GetIcon's success depends on availability
of these desktop environments.

\membersection{wxFileType::GetDescription}\label{wxfiletypegetdescription}

\func{bool}{GetDescription}{\param{wxString*}{ desc}}

If the function returns {\tt true}, the string pointed to by {\it desc} is filled
with a brief description for this file type: for example, "text document" for
the "text/plain" MIME type.

\membersection{wxFileType::GetOpenCommand}\label{wxfiletypegetopencommand}

\func{bool}{GetOpenCommand}{\param{wxString*}{ command}, \param{MessageParameters\&}{ params}}

\func{wxString}{GetOpenCommand}{\param{const wxString\&}{ filename}}

With the first version of this method, if the {\tt true} is returned, the
string pointed to by {\it command} is filled with the command which must be
executed (see \helpref{wxExecute}{wxexecute}) in order to open the file of the
given type. In this case, the name of the file as well as any other parameters
is retrieved from \helpref{MessageParameters}{wxfiletypemessageparameters} 
class.

In the second case, only the filename is specified and the command to be used
to open this kind of file is returned directly. An empty string is returned to
indicate that an error occurred (typically meaning that there is no standard way
to open this kind of files).

\membersection{wxFileType::GetPrintCommand}\label{wxfiletypegetprintcommand}

\func{bool}{GetPrintCommand}{\param{wxString*}{ command},\param{MessageParameters\&}{ params}}

If the function returns {\tt true}, the string pointed to by {\it command} is filled
with the command which must be executed (see \helpref{wxExecute}{wxexecute}) in
order to print the file of the given type. The name of the file is
retrieved from \helpref{MessageParameters}{wxfiletypemessageparameters} class.

\membersection{wxFileType::ExpandCommand}\label{wxfiletypeexpandcommand}

\func{static wxString}{ExpandCommand}{\param{const wxString\&}{ command}, \param{MessageParameters\&}{ params}}

This function is primarily intended for GetOpenCommand and GetPrintCommand
usage but may be also used by the application directly if, for example, you want
to use some non default command to open the file.

The function replaces all occurrences of

\twocolwidtha{7cm}
\begin{twocollist}\itemsep=0pt
\twocolitem{format specification}{with}
\twocolitem{\%s}{the full file name}
\twocolitem{\%t}{the MIME type}
\twocolitem{\%\{param\}}{the value of the parameter {\it param}}
\end{twocollist}

using the MessageParameters object you pass to it.

If there is no '\%s' in the command string (and the string is not empty), it is
assumed that the command reads the data on stdin and so the effect is the same
as "< \%s" were appended to the string.

Unlike all other functions of this class, there is no error return for this
function.

