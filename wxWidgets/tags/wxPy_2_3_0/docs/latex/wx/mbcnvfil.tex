%
% automatically generated by HelpGen from
% ../include/wx/strconv.h at 25/Mar/00 10:20:56
%

\section{\class{wxMBConvFile}}\label{wxmbconvfile}

This class converts file names between filesystem multibyte encoding and
Unicode. It has one predefined instance, {\bf wxConvFile}.
Since some platforms (e.g. Win32) use Unicode in the filenames,
and others (e.g. Unix) use multibyte encodings, this class should only
be used directly if wxMBFILES is defined to 1. A convenience macro,
wxFNCONV, is defined to wxConvFile.cWX2MB in this case. You could use it
like this:

\begin{verbatim}
wxChar *name = wxT("rawfile.doc");
FILE *fil = fopen(wxFNCONV(name), "r");
\end{verbatim}

(although it would be better to use wxFopen(name, wxT("r")) in this case.)

\wxheading{Derived from}

\helpref{wxMBConv}{wxmbconv}

\wxheading{Include files}

<wx/strconv.h>

\wxheading{See also}

\helpref{wxMBConv classes overview}{mbconvclasses}

\latexignore{\rtfignore{\wxheading{Members}}}


\membersection{wxMBConvFile::MB2WC}\label{wxmbconvfilemb2wc}

\constfunc{size\_t}{MB2WC}{\param{wchar\_t* }{buf}, \param{const char* }{psz}, \param{size\_t }{n}}

Converts from multibyte filename encoding to Unicode. Returns the size of the destination buffer.

\membersection{wxMBConvFile::WC2MB}\label{wxmbconvfilewc2mb}

\constfunc{size\_t}{WC2MB}{\param{char* }{buf}, \param{const wchar\_t* }{psz}, \param{size\_t }{n}}

Converts from Unicode to multibyte filename encoding. Returns the size of the destination buffer.

