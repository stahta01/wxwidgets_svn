\section{\class{wxIdleEvent}}\label{wxidleevent}

This class is used for idle events, which are generated when the system becomes
idle. Note that, unless you do something specifically, the idle events are not
sent if the system remains idle once it has become it, e.g. only a single idle
event will be generated until something else resulting in more normal events
happens and only then is the next idle event sent again. If you need to ensure
a continuous stream of idle events, you can either use 
\helpref{RequestMore}{wxidleeventrequestmore} method in your handler or call 
\helpref{wxWakeUpIdle}{wxwakeupidle} periodically (for example from timer
event), but note that both of these approaches (and especially the first one)
increase the system load and so should be avoided if possible.

By default, idle events are sent to all windows (and also 
\helpref{wxApp}{wxapp}, as usual). If this is causing a significant
overhead in your application, you can call \helpref{wxIdleEvent::SetMode}{wxidleeventsetmode} with
the value wxIDLE\_PROCESS\_SPECIFIED, and set the wxWS\_EX\_PROCESS\_IDLE extra
window style for every window which should receive idle events.

\wxheading{Derived from}

\helpref{wxEvent}{wxevent}\\
\helpref{wxObject}{wxobject}

\wxheading{Include files}

<wx/event.h>

\wxheading{Event table macros}

To process an idle event, use this event handler macro to direct input to a member
function that takes a wxIdleEvent argument.

\twocolwidtha{7cm}
\begin{twocollist}\itemsep=0pt
\twocolitem{{\bf EVT\_IDLE(func)}}{Process a wxEVT\_IDLE event.}
\end{twocollist}%

\wxheading{See also}

\helpref{Event handling overview}{eventhandlingoverview}, \helpref{wxUpdateUIEvent}{wxupdateuievent}, 
\helpref{wxWindow::OnInternalIdle}{wxwindowoninternalidle}

\latexignore{\rtfignore{\wxheading{Members}}}

\membersection{wxIdleEvent::wxIdleEvent}\label{wxidleeventctor}

\func{}{wxIdleEvent}{\void}

Constructor.

\membersection{wxIdleEvent::CanSend}\label{wxidleeventcansend}

\func{static bool}{CanSend}{\param{wxWindow*}{ window}}

Returns {\tt true} if it is appropriate to send idle events to
this window.

This function looks at the mode used (see \helpref{wxIdleEvent::SetMode}{wxidleeventsetmode}),
and the wxWS\_EX\_PROCESS\_IDLE style in {\it window} to determine whether idle events should be sent to
this window now. By default this will always return {\tt true} because
the update mode is initially wxIDLE\_PROCESS\_ALL. You can change the mode
to only send idle events to windows with the wxWS\_EX\_PROCESS\_IDLE extra window style set.

\wxheading{See also}

\helpref{wxIdleEvent::SetMode}{wxidleeventsetmode}

\membersection{wxIdleEvent::GetMode}\label{wxidleeventgetmode}

\func{static wxIdleMode}{GetMode}{\void}

Static function returning a value specifying how wxWidgets
will send idle events: to all windows, or only to those which specify that they
will process the events.

See \helpref{wxIdleEvent::SetMode}{wxidleeventsetmode}.

\membersection{wxIdleEvent::RequestMore}\label{wxidleeventrequestmore}

\func{void}{RequestMore}{\param{bool}{ needMore = true}}

Tells wxWidgets that more processing is required. This function can be called by an OnIdle
handler for a window or window event handler to indicate that wxApp::OnIdle should
forward the OnIdle event once more to the application windows. If no window calls this function
during OnIdle, then the application will remain in a passive event loop (not calling OnIdle) until a
new event is posted to the application by the windowing system.

\wxheading{See also}

\helpref{wxIdleEvent::MoreRequested}{wxidleeventmorerequested}

\membersection{wxIdleEvent::MoreRequested}\label{wxidleeventmorerequested}

\constfunc{bool}{MoreRequested}{\void}

Returns true if the OnIdle function processing this event requested more processing time.

\wxheading{See also}

\helpref{wxIdleEvent::RequestMore}{wxidleeventrequestmore}

\membersection{wxIdleEvent::SetMode}\label{wxidleeventsetmode}

\func{static void}{SetMode}{\param{wxIdleMode }{mode}}

Static function for specifying how wxWidgets will send idle events: to
all windows, or only to those which specify that they
will process the events.

{\it mode} can be one of the following values.
The default is wxIDLE\_PROCESS\_ALL.

\begin{verbatim}
enum wxIdleMode
{
        // Send idle events to all windows
    wxIDLE_PROCESS_ALL,

        // Send idle events to windows that have
        // the wxWS_EX_PROCESS_IDLE flag specified
    wxIDLE_PROCESS_SPECIFIED
};
\end{verbatim}

