%%%%%%%%%%%%%%%%%%%%%%%%%%%%%%%%%%%%%%%%%%%%%%%%%%%%%%%%%%%%%%%%%%%%%%%%%%%%%
%% Name:        colour.tex
%% Purpose:     wxColour docs
%% Author:
%% Modified by: Francesco Montorsi
%% Created:
%% RCS-ID:      $Id$
%% Copyright:   (c) wxWidgets
%% License:     wxWindows license
%%%%%%%%%%%%%%%%%%%%%%%%%%%%%%%%%%%%%%%%%%%%%%%%%%%%%%%%%%%%%%%%%%%%%%%%%%%%%

\section{\class{wxColour}}\label{wxcolour}

A colour is an object representing a combination of Red, Green, and Blue (RGB) intensity values,
and is used to determine drawing colours. See the
entry for \helpref{wxColourDatabase}{wxcolourdatabase} for how a pointer to a predefined,
named colour may be returned instead of creating a new colour.

Valid RGB values are in the range 0 to 255.

You can retrieve the current system colour settings with \helpref{wxSystemSettings}{wxsystemsettings}.

\wxheading{Derived from}

\helpref{wxObject}{wxobject}

\wxheading{Include files}

<wx/colour.h>

\wxheading{Predefined objects}

Objects:

{\bf wxNullColour}

Pointers:

{\bf wxBLACK\\
wxWHITE\\
wxRED\\
wxBLUE\\
wxGREEN\\
wxCYAN\\
wxLIGHT\_GREY}

\wxheading{See also}

\helpref{wxColourDatabase}{wxcolourdatabase}, \helpref{wxPen}{wxpen}, \helpref{wxBrush}{wxbrush},\rtfsp
\helpref{wxColourDialog}{wxcolourdialog}, \helpref{wxSystemSettings}{wxsystemsettings}

\latexignore{\rtfignore{\wxheading{Members}}}


\membersection{wxColour::wxColour}\label{wxcolourconstr}

\func{}{wxColour}{\void}

Default constructor.

\func{}{wxColour}{\param{unsigned char}{ red}, \param{unsigned char}{ green}, \param{unsigned char}{ blue}, \param{unsigned char}{ alpha=wxALPHA\_OPAQUE}}

Constructs a colour from red, green, blue and alpha values.

\func{}{wxColour}{\param{const wxString\& }{colourNname}}

Constructs a colour using the given string. See \helpref{Set}{wxcolourset} for more info.

\func{}{wxColour}{\param{const wxColour\&}{ colour}}

Copy constructor.

\wxheading{Parameters}

\docparam{red}{The red value.}

\docparam{green}{The green value.}

\docparam{blue}{The blue value.}

\docparam{alpha}{The alpha value. Alpha values range from 0 (wxALPHA\_TRANSPARENT) to 255 (wxALPHA\_OPAQUE).}

\docparam{colourName}{The colour name.}

\docparam{colour}{The colour to copy.}

\wxheading{See also}

\helpref{wxColourDatabase}{wxcolourdatabase}

\pythonnote{Constructors supported by wxPython are:\par
\indented{2cm}{\begin{twocollist}
\twocolitem{{\bf wxColour(red=0, green=0, blue=0)}}{}
\twocolitem{{\bf wxNamedColour(name)}}{}
\end{twocollist}}
}



\membersection{wxColour::Alpha}\label{wxcolouralpha}

\constfunc{unsigned char}{Alpha}{\void}

Returns the alpha value, on platforms where alpha is not yet supported, this always returns wxALPHA\_OPAQUE.


\membersection{wxColour::Blue}\label{wxcolourblue}

\constfunc{unsigned char}{Blue}{\void}

Returns the blue intensity.


\membersection{wxColour::GetAsString}\label{wxcolourgetasstring}

\constfunc{wxString}{GetAsString}{\param{long}{ flags}}

Converts this colour to a \helpref{wxString}{wxstring}
using the given {\it flags}.

The supported flags are {\bf wxC2S\_NAME}, to obtain the colour
name (e.g. wxColour(255,0,0) -> \texttt{``red"}), {\bf wxC2S\_CSS\_SYNTAX}, to obtain
the colour in the \texttt{``rgb(r,g,b)"} syntax
(e.g. wxColour(255,0,0) -> \texttt{``rgb(255,0,0)"}), and {\bf wxC2S\_HTML\_SYNTAX}, to obtain
the colour as  \texttt{``\#"} followed by 6 hexadecimal digits
(e.g. wxColour(255,0,0) -> \texttt{``\#FF0000"}).

This function never fails and always returns a non-empty string.

\newsince{2.7.0}

\membersection{wxColour::GetPixel}\label{wxcolourgetpixel}

\constfunc{long}{GetPixel}{\void}

Returns a pixel value which is platform-dependent. On Windows, a COLORREF is returned.
On X, an allocated pixel value is returned.

-1 is returned if the pixel is invalid (on X, unallocated).


\membersection{wxColour::Green}\label{wxcolourgreen}

\constfunc{unsigned char}{Green}{\void}

Returns the green intensity.


\membersection{wxColour::IsOk}\label{wxcolourisok}

\constfunc{bool}{IsOk}{\void}

Returns \true if the colour object is valid (the colour has been initialised with RGB values).


\membersection{wxColour::Red}\label{wxcolourred}

\constfunc{unsigned char}{Red}{\void}

Returns the red intensity.


\membersection{wxColour::Set}\label{wxcolourset}

\func{void}{Set}{\param{unsigned char}{ red}, \param{unsigned char}{ green}, \param{unsigned char}{ blue}, \param{unsigned char}{ alpha=wxALPHA\_OPAQUE}}

\func{void}{Set}{\param{unsigned long}{ RGB}}

\func{bool}{Set}{\param{const wxString \&}{ str}}

Sets the RGB intensity values using the given values (first overload), extracting them from the packed long (second overload), using the given string (third overloard).

When using third form, Set() accepts: colour names (those listed in \helpref{wxTheColourDatabase}{wxcolourdatabase}), the CSS-like \texttt{``RGB(r,g,b)"} syntax (case insensitive) and the HTML-like syntax (i.e. \texttt{``\#"} followed by 6 hexadecimal digits for red, green, blue components).

Returns \true if the conversion was successful, \false otherwise.

\newsince{2.7.0}


\membersection{wxColour::operator $=$}\label{wxcolourassign}

\func{wxColour\&}{operator $=$}{\param{const wxColour\&}{ colour}}

Assignment operator, taking another colour object.

\func{wxColour\&}{operator $=$}{\param{const wxString\&}{ colourName}}

Assignment operator, using a colour name to be found in the colour database.

\wxheading{See also}

\helpref{wxColourDatabase}{wxcolourdatabase}


\membersection{wxColour::operator $==$}\label{wxcolourequality}

\func{bool}{operator $==$}{\param{const wxColour\&}{ colour}}

Tests the equality of two colours by comparing individual red, green, blue colours and alpha values.


\membersection{wxColour::operator $!=$}\label{wxcolourinequality}

\func{bool}{operator $!=$}{\param{const wxColour\&}{ colour}}

Tests the inequality of two colours by comparing individual red, green, blue colours and alpha values.

\section{\class{wxColourData}}\label{wxcolourdata}

This class holds a variety of information related to colour dialogs.

\wxheading{Derived from}

\helpref{wxObject}{wxobject}

\wxheading{Include files}

<wx/cmndata.h>

\wxheading{See also}

\helpref{wxColour}{wxcolour}, \helpref{wxColourDialog}{wxcolourdialog}, \helpref{wxColourDialog overview}{wxcolourdialogoverview}

\latexignore{\rtfignore{\wxheading{Members}}}


\membersection{wxColourData::wxColourData}\label{wxcolourdatactor}

\func{}{wxColourData}{\void}

Constructor. Initializes the custom colours to {\tt wxNullColour},
the {\it data colour} setting
to black, and the {\it choose full} setting to true.


\membersection{wxColourData::\destruct{wxColourData}}\label{wxcolourdatadtor}

\func{}{\destruct{wxColourData}}{\void}

Destructor.


\membersection{wxColourData::GetChooseFull}\label{wxcolourdatagetchoosefull}

\constfunc{bool}{GetChooseFull}{\void}

Under Windows, determines whether the Windows colour dialog will display the full dialog
with custom colour selection controls. Under PalmOS, determines whether colour dialog
will display full rgb colour picker or only available palette indexer.
Has no meaning under other platforms.

The default value is true.


\membersection{wxColourData::GetColour}\label{wxcolourdatagetcolour}

\constfunc{wxColour\&}{GetColour}{\void}

Gets the current colour associated with the colour dialog.

The default colour is black.


\membersection{wxColourData::GetCustomColour}\label{wxcolourdatagetcustomcolour}

\constfunc{wxColour\&}{GetCustomColour}{\param{int}{ i}}

Gets the {\it i}th custom colour associated with the colour dialog. {\it i} should
be an integer between 0 and 15.

The default custom colours are invalid colours.


\membersection{wxColourData::SetChooseFull}\label{wxcolourdatasetchoosefull}

\func{void}{SetChooseFull}{\param{const bool }{flag}}

Under Windows, tells the Windows colour dialog to display the full dialog
with custom colour selection controls. Under other platforms, has no effect.

The default value is true.


\membersection{wxColourData::SetColour}\label{wxcolourdatasetcolour}

\func{void}{SetColour}{\param{const wxColour\&}{ colour}}

Sets the default colour for the colour dialog.

The default colour is black.


\membersection{wxColourData::SetCustomColour}\label{wxcolourdatasetcustomcolour}

\func{void}{SetCustomColour}{\param{int}{ i}, \param{const wxColour\&}{ colour}}

Sets the {\it i}th custom colour for the colour dialog. {\it i} should
be an integer between 0 and 15.

The default custom colours are invalid colours.


\membersection{wxColourData::operator $=$}\label{wxcolourdataassign}

\func{void}{operator $=$}{\param{const wxColourData\&}{ data}}

Assignment operator for the colour data.




\section{\class{wxColourDatabase}}\label{wxcolourdatabase}

wxWidgets maintains a database of standard RGB colours for a predefined
set of named colours (such as ``BLACK'', ``LIGHT GREY''). The
application may add to this set if desired by using
\helpref{AddColour}{wxcolourdatabaseaddcolour} and may use it to look up
colours by names using \helpref{Find}{wxcolourdatabasefind} or find the names
for the standard colour suing \helpref{FindName}{wxcolourdatabasefindname}.

There is one predefined instance of this class called
{\bf wxTheColourDatabase}.

\wxheading{Derived from}

None

\wxheading{Include files}

<wx/gdicmn.h>

\wxheading{Remarks}

The standard database contains at least the following colours:

AQUAMARINE, BLACK, BLUE, BLUE VIOLET, BROWN, CADET BLUE, CORAL,
CORNFLOWER BLUE, CYAN, DARK GREY, DARK GREEN, DARK OLIVE GREEN, DARK
ORCHID, DARK SLATE BLUE, DARK SLATE GREY DARK TURQUOISE, DIM GREY,
FIREBRICK, FOREST GREEN, GOLD, GOLDENROD, GREY, GREEN, GREEN YELLOW,
INDIAN RED, KHAKI, LIGHT BLUE, LIGHT GREY, LIGHT STEEL BLUE, LIME GREEN,
MAGENTA, MAROON, MEDIUM AQUAMARINE, MEDIUM BLUE, MEDIUM FOREST GREEN,
MEDIUM GOLDENROD, MEDIUM ORCHID, MEDIUM SEA GREEN, MEDIUM SLATE BLUE,
MEDIUM SPRING GREEN, MEDIUM TURQUOISE, MEDIUM VIOLET RED, MIDNIGHT BLUE,
NAVY, ORANGE, ORANGE RED, ORCHID, PALE GREEN, PINK, PLUM, PURPLE, RED,
SALMON, SEA GREEN, SIENNA, SKY BLUE, SLATE BLUE, SPRING GREEN, STEEL
BLUE, TAN, THISTLE, TURQUOISE, VIOLET, VIOLET RED, WHEAT, WHITE, YELLOW,
YELLOW GREEN.

\wxheading{See also}

\helpref{wxColour}{wxcolour}

\latexignore{\rtfignore{\wxheading{Members}}}


\membersection{wxColourDatabase::wxColourDatabase}\label{wxcolourdatabaseconstr}

\func{}{wxColourDatabase}{\void}

Constructs the colour database. It will be initialized at the first use.


\membersection{wxColourDatabase::AddColour}\label{wxcolourdatabaseaddcolour}

\func{void}{AddColour}{\param{const wxString\& }{colourName}, \param{const wxColour\&}{colour}}

\func{void}{AddColour}{\param{const wxString\& }{colourName}, \param{wxColour* }{colour}}

Adds a colour to the database. If a colour with the same name already exists,
it is replaced.

Please note that the overload taking a pointer is deprecated and will be
removed in the next wxWidgets version, please don't use it.


\membersection{wxColourDatabase::Find}\label{wxcolourdatabasefind}

\func{wxColour}{Find}{\param{const wxString\& }{colourName}}

Finds a colour given the name. Returns an invalid colour object (that is, such
that its \helpref{Ok()}{wxcolourisok} method returns \false) if the colour wasn't
found in the database.


\membersection{wxColourDatabase::FindName}\label{wxcolourdatabasefindname}

\constfunc{wxString}{FindName}{\param{const wxColour\&}{ colour}}

Finds a colour name given the colour. Returns an empty string if the colour is
not found in the database.

