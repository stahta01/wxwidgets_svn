\section{\class{wxThreadHelper}}\label{wxthreadhelper}

The wxThreadHelper class is a mix-in class that manages a single background
thread.  By deriving from wxThreadHelper, a class can implement the thread
code in its own \helpref{wxThreadHelper::Entry}{wxthreadhelperentry} method
and easily share data and synchronization objects between the main thread
and the worker thread.  Doing this prevents the awkward passing of pointers
that is needed when the original object in the main thread needs to
synchronize with its worker thread in its own wxThread derived object.

For example, \helpref{wxFrame}{wxframe} may need to make some calculations
in a background thread and then display the results of those calculations in
the main window.

Ordinarily, a \helpref{wxThread}{wxthread} derived object would be created
with the calculation code implemented in
\helpref{wxThread::Entry}{wxthreadentry}.  To access the inputs to the
calculation, the frame object would often to pass a pointer to itself to the
thread object.  Similiarly, the frame object would hold a pointer to the
thread object.  Shared data and synchronization objects could be stored in
either object though the object without the data would have to access the
data through a pointer.

However, with wxThreadHelper, the frame object and the thread object are
treated as the same object.  Shared data and synchronization variables are
stored in the single object, eliminating a layer of indirection and the
associated pointers.

\wxheading{Derived from}

None.

\wxheading{Include files}

<wx/thread.h>

\wxheading{See also}

\helpref{wxThread}{wxthread}

\latexignore{\rtfignore{\wxheading{Members}}}

\membersection{wxThreadHelper::wxThreadHelper}\label{wxthreadhelperctor}

\func{}{wxThreadHelper}{\void}

This constructor simply initializes a member variable.

\membersection{wxThreadHelper::m\_thread}

\member{wxThread *}{m\_thread}

the actual \helpref{wxThread}{wxthread} object.

\membersection{wxThread::\destruct{wxThreadHelper}}

\func{}{\destruct{wxThreadHelper}}{\void}

The destructor frees the resources associated with the thread.

\membersection{wxThreadHelper::Create}\label{wxthreadhelpercreate}

\func{wxThreadError}{Create}{\param{unsigned int }{stackSize = 0}}

Creates a new thread. The thread object is created in the suspended state, and you
should call \helpref{GetThread()->Run()}{wxthreadrun} to start running
it.  You may optionally specify the stack size to be allocated to it (Ignored on
platforms that don't support setting it explicitly, eg. Unix).

\wxheading{Return value}

One of:

\twocolwidtha{7cm}
\begin{twocollist}\itemsep=0pt
\twocolitem{{\bf wxTHREAD\_NO\_ERROR}}{There was no error.}
\twocolitem{{\bf wxTHREAD\_NO\_RESOURCE}}{There were insufficient resources to create a new thread.}
\twocolitem{{\bf wxTHREAD\_RUNNING}}{The thread is already running.}
\end{twocollist}

\membersection{wxThreadHelper::Entry}\label{wxthreadhelperentry}

\func{virtual ExitCode}{Entry}{\void}

This is the entry point of the thread. This function is pure virtual and must
be implemented by any derived class. The thread execution will start here.

The returned value is the thread exit code which is only useful for
joinable threads and is the value returned by
\helpref{GetThread()->Wait()}{wxthreadwait}.

This function is called by wxWindows itself and should never be called
directly.

\membersection{wxThreadHelper::GetThread}\label{wxthreadhelpergetthread}

\func{wxThread *}{GetThread}{\void}

This is a public function that returns the \helpref{wxThread}{wxthread} object
associated with the thread.

