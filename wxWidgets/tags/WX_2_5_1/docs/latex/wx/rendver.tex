%%%%%%%%%%%%%%%%%%%%%%%%%%%%%%%%%%%%%%%%%%%%%%%%%%%%%%%%%%%%%%%%%%%%%%%%%%%%%%%
%% Name:        rendver.tex
%% Purpose:     wxRendererVersion documentation
%% Author:      Vadim Zeitlin
%% Modified by:
%% Created:     11.08.03
%% RCS-ID:      $Id$
%% Copyright:   (c) 2003 Vadim Zeitlin <vadim@wxwindows.org>
%% License:     wxWindows license
%%%%%%%%%%%%%%%%%%%%%%%%%%%%%%%%%%%%%%%%%%%%%%%%%%%%%%%%%%%%%%%%%%%%%%%%%%%%%%%

\section{\class{wxRendererVersion}}\label{wxrendererversion}

This simple struct represents the \helpref{wxRendererNative}{wxrenderernative} 
interface version and is only used as the return value of 
\helpref{wxRendererNative::GetVersion}{wxrenderernativegetversion}.

The version has two components: the version itself and the age. If the main
program and the renderer have different versions they are never compatible with
each other because the version is only changed when an existing virtual
function is modified or removed. The age, on the other hand, is incremented
each time a new virtual method is added and so, at least for the compilers
using a common C++ object model, the calling program is compatible with any
renderer which has the age greater or equal to its age. This verification is
done by \helpref{IsCompatible}{wxrenderernativeiscompatible} method.


\wxheading{Derived from}

No base class

\wxheading{Include files}

<wx/renderer.h>


\latexignore{\rtfignore{\wxheading{Members}}}

\membersection{wxRendererVersion::IsCompatible}\label{wxrenderernativeiscompatible}

\func{static bool}{IsCompatible}{\param{const wxRendererVersion\& }{ver}}

Checks if the main program is compatible with the renderer having the version 
\arg{ver}, returns \true if it is and \false otherwise.

This method is used by 
\helpref{wxRendererNative::Load}{wxrenderernativeload} to determine whether a
renderer can be used.


\membersection{wxRendererVersion::version}

\member{const int}{version}

The version component.


\membersection{wxRendererVersion::age}

\member{const int}{age}

The age component.

