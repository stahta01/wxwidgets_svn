\section{\class{wxToolBar}}\label{wxtoolbar}

The name wxToolBar is defined to be a synonym for one of the following classes:

\begin{itemize}\itemsep=0pt
\item {\bf wxToolBar95} The native Windows 95 toolbar. Used on Windows 95, NT 4 and above.
\item {\bf wxToolBarMSW} A Windows implementation. Used on 16-bit Windows.
\item {\bf wxToolBarGTK} The GTK toolbar.
\item {\bf wxToolBarSimple} A simple implementation, with scrolling.
Used on platforms with no native toolbar control, or where scrolling is required.
\end{itemize}

Note that the base class {\bf wxToolBarBase} defines
automatic scrolling management functionality which is similar
to \helpref{wxScrolledWindow}{wxscrolledwindow}, so please refer to this class also.
Not all toolbars support scrolling, but wxToolBarSimple does.

\wxheading{Derived from}

wxToolBarBase\\
\helpref{wxControl}{wxcontrol}\\
\helpref{wxWindow}{wxwindow}\\
\helpref{wxEvtHandler}{wxevthandler}\\
\helpref{wxObject}{wxobject}

\wxheading{Include files}

<wx/toolbar.h> (to allow wxWidgets to select an appropriate toolbar class)\\
<wx/tbarbase.h> (the base class)\\
<wx/tbarmsw.h> (the non-Windows 95 Windows toolbar class)\\
<wx/tbar95.h> (the Windows 95/98 toolbar class)\\
<wx/tbarsmpl.h> (the generic simple toolbar class)

\wxheading{Remarks}

You may also create a toolbar that is managed by the frame, by
calling \helpref{wxFrame::CreateToolBar}{wxframecreatetoolbar}.

Due to the use of native toolbars on the various platforms, certain adaptions will
often have to be made in order to get optimal look on all platforms as some platforms
ignore the values for explicit placement and use their own layout and the meaning
of a "separator" is a vertical line under Windows95 vs. simple space under GTK etc.

{\bf wxToolBar95:} Note that this toolbar paints tools to reflect system-wide colours.
If you use more than 16 colours in your tool bitmaps, you may wish to suppress
this behaviour, otherwise system colours in your bitmaps will inadvertently be
mapped to system colours. To do this, set the {\tt msw.remap} system option
before creating the toolbar:

\begin{verbatim}
  wxSystemOptions::SetOption(wxT("msw.remap"), 0);
\end{verbatim}

If you wish to use 32-bit images (which include an alpha channel for transparency)
use:

\begin{verbatim}
  wxSystemOptions::SetOption(wxT("msw.remap"), 2);
\end{verbatim}

then colour remapping is switched off, and a transparent background used. But only
use this option under Windows XP with true colour:

\begin{verbatim}
  (wxTheApp->GetComCtl32Version() >= 600 && ::wxDisplayDepth() >= 32)
\end{verbatim}

\wxheading{Window styles}

\twocolwidtha{5cm}
\begin{twocollist}\itemsep=0pt
\twocolitem{\windowstyle{wxTB\_FLAT}}{Gives the toolbar a flat look (Windows and GTK only).}
\twocolitem{\windowstyle{wxTB\_DOCKABLE}}{Makes the toolbar floatable and dockable (GTK only).}
\twocolitem{\windowstyle{wxTB\_HORIZONTAL}}{Specifies horizontal layout (default).}
\twocolitem{\windowstyle{wxTB\_VERTICAL}}{Specifies vertical layout.}
\twocolitem{\windowstyle{wxTB\_TEXT}}{Shows the text in the toolbar buttons; by default only icons are shown.}
\twocolitem{\windowstyle{wxTB\_NOICONS}}{Specifies no icons in the toolbar buttons; by default they are shown.}
\twocolitem{\windowstyle{wxTB\_NODIVIDER}}{Specifies no divider (border) above the toolbar (Windows only).}
\twocolitem{\windowstyle{wxTB\_NOALIGN}}{Specifies no alignment with the parent window (Windows only, not very useful).}
\twocolitem{\windowstyle{wxTB\_HORZ\_LAYOUT}}{Shows the text and the icons alongside, not vertically stacked (Windows and GTK
2 only). This style must be used with wxTB\_TEXT.}
\twocolitem{\windowstyle{wxTB\_HORZ\_TEXT}}{Combination of wxTB\_HORZ\_LAYOUT and wxTB\_TEXT.}
\end{twocollist}

See also \helpref{window styles overview}{windowstyles}. Note that the Win32
native toolbar ignores {\tt wxTB\_NOICONS} style. Also, toggling the 
{\tt wxTB\_TEXT} works only if the style was initially on.

\wxheading{Event handling}

The toolbar class emits menu commands in the same way that a frame menubar does,
so you can use one EVT\_MENU macro for both a menu item and a toolbar button.
The event handler functions take a wxCommandEvent argument. For most event macros,
the identifier of the tool is passed, but for EVT\_TOOL\_ENTER the toolbar
window identifier is passed and the tool identifier is retrieved from the wxCommandEvent.
This is because the identifier may be -1 when the mouse moves off a tool, and -1 is not
allowed as an identifier in the event system.

\twocolwidtha{7cm}
\begin{twocollist}\itemsep=0pt
\twocolitem{{\bf EVT\_TOOL(id, func)}}{Process a wxEVT\_COMMAND\_TOOL\_CLICKED event
(a synonym for wxEVT\_COMMAND\_MENU\_SELECTED). Pass the id of the tool.}
\twocolitem{{\bf EVT\_MENU(id, func)}}{The same as EVT\_TOOL.}
\twocolitem{{\bf EVT\_TOOL\_RANGE(id1, id2, func)}}{Process a wxEVT\_COMMAND\_TOOL\_CLICKED event
for a range of identifiers. Pass the ids of the tools.}
\twocolitem{{\bf EVT\_MENU\_RANGE(id1, id2, func)}}{The same as EVT\_TOOL\_RANGE.}
\twocolitem{{\bf EVT\_TOOL\_RCLICKED(id, func)}}{Process a wxEVT\_COMMAND\_TOOL\_RCLICKED event.
Pass the id of the tool.}
\twocolitem{{\bf EVT\_TOOL\_RCLICKED\_RANGE(id1, id2, func)}}{Process a wxEVT\_COMMAND\_TOOL\_RCLICKED event
for a range of ids. Pass the ids of the tools.}
\twocolitem{{\bf EVT\_TOOL\_ENTER(id, func)}}{Process a wxEVT\_COMMAND\_TOOL\_ENTER event.
Pass the id of the toolbar itself. The value of wxCommandEvent::GetSelection is the tool id, or -1 if the mouse cursor has moved off a tool.}
\end{twocollist}

\wxheading{See also}

\overview{Toolbar overview}{wxtoolbaroverview},\rtfsp
\helpref{wxScrolledWindow}{wxscrolledwindow}

\latexignore{\rtfignore{\wxheading{Members}}}

\membersection{wxToolBar::wxToolBar}\label{wxtoolbarconstr}

\func{}{wxToolBar}{\void}

Default constructor.

\func{}{wxToolBar}{\param{wxWindow*}{ parent}, \param{wxWindowID }{id},
 \param{const wxPoint\& }{pos = wxDefaultPosition},
 \param{const wxSize\& }{size = wxDefaultSize},
 \param{long }{style = wxTB\_HORIZONTAL \pipe wxNO\_BORDER},
 \param{const wxString\& }{name = wxPanelNameStr}}

Constructs a toolbar.

\wxheading{Parameters}

\docparam{parent}{Pointer to a parent window.}

\docparam{id}{Window identifier. If -1, will automatically create an identifier.}

\docparam{pos}{Window position. wxDefaultPosition is (-1, -1) which indicates that wxWidgets
should generate a default position for the window. If using the wxWindow class directly, supply
an actual position.}

\docparam{size}{Window size. wxDefaultSize is (-1, -1) which indicates that wxWidgets
should generate a default size for the window.}

\docparam{style}{Window style. See \helpref{wxToolBar}{wxtoolbar} for details.}

\docparam{name}{Window name.}

\wxheading{Remarks}

After a toolbar is created, you use \helpref{wxToolBar::AddTool}{wxtoolbaraddtool} and
perhaps \helpref{wxToolBar::AddSeparator}{wxtoolbaraddseparator}, and then you
must call \helpref{wxToolBar::Realize}{wxtoolbarrealize} to construct and display the toolbar
tools.

You may also create a toolbar that is managed by the frame, by
calling \helpref{wxFrame::CreateToolBar}{wxframecreatetoolbar}.

\membersection{wxToolBar::\destruct{wxToolBar}}\label{wxtoolbardtor}

\func{void}{\destruct{wxToolBar}}{\void}

Toolbar destructor.

\membersection{wxToolBar::AddControl}\label{wxtoolbaraddcontrol}

\func{bool}{AddControl}{\param{wxControl*}{ control}}

Adds any control to the toolbar, typically e.g. a combobox.

\docparam{control}{The control to be added.}

\membersection{wxToolBar::AddSeparator}\label{wxtoolbaraddseparator}

\func{void}{AddSeparator}{\void}

Adds a separator for spacing groups of tools.

\wxheading{See also}

\helpref{wxToolBar::AddTool}{wxtoolbaraddtool}, \helpref{wxToolBar::SetToolSeparation}{wxtoolbarsettoolseparation}

\membersection{wxToolBar::AddTool}\label{wxtoolbaraddtool}

\func{wxToolBarTool*}{AddTool}{\param{int}{ toolId},\rtfsp
\param{const wxString\&}{ label},\rtfsp
\param{const wxBitmap\&}{ bitmap1},\rtfsp
\param{const wxString\& }{shortHelpString = ""},\rtfsp
\param{wxItemKind}{ kind = wxITEM\_NORMAL}}

\func{wxToolBarTool*}{AddTool}{\param{int}{ toolId},\rtfsp
\param{const wxString\&}{ label},\rtfsp
\param{const wxBitmap\&}{ bitmap1},\rtfsp
\param{const wxBitmap\&}{ bitmap2 = wxNullBitmap},\rtfsp
\param{wxItemKind}{ kind = wxITEM\_NORMAL},\rtfsp
\param{const wxString\& }{shortHelpString = ""},
\param{const wxString\& }{longHelpString = ""},\rtfsp
\param{wxObject* }{clientData = NULL}}

\func{wxToolBarTool*}{AddTool}{\param{wxToolBarTool* }{tool}}

Adds a tool to the toolbar. The first (short and most commonly used) version
has fewer parameters than the full version at the price of not being able to
specify some of the more rarely used button features. The last version allows
to add an existing tool.

\wxheading{Parameters}

\docparam{toolId}{An integer by which
the tool may be identified in subsequent operations.}

\docparam{kind}{May be wxITEM\_NORMAL for a normal button (default),
wxITEM\_CHECK for a checkable tool (such tool stays pressed after it had been
toggled) or wxITEM\_RADIO for a checkable tool which makes part of a radio
group of tools each of which is automatically unchecked whenever another button
in the group is checked}

\docparam{bitmap1}{The primary tool bitmap for toggle and button tools.}

\docparam{bitmap2}{The second bitmap specifies the on-state bitmap for a toggle
tool. If this is wxNullBitmap, either an inverted version of the primary bitmap is
used for the on-state of a toggle tool (monochrome displays) or a black
border is drawn around the tool (colour displays) or the pixmap is shown
as a pressed button (GTK). }

\docparam{shortHelpString}{This string is used for the tools tooltip}

\docparam{longHelpString}{This string is shown in the statusbar (if any) of the
parent frame when the mouse pointer is inside the tool}

\docparam{clientData}{An optional pointer to client data which can be
retrieved later using \helpref{wxToolBar::GetToolClientData}{wxtoolbargettoolclientdata}.}

\docparam{tool}{The tool to be added.}

\wxheading{Remarks}

After you have added tools to a toolbar, you must call \helpref{wxToolBar::Realize}{wxtoolbarrealize} in
order to have the tools appear.

\wxheading{See also}

\helpref{wxToolBar::AddSeparator}{wxtoolbaraddseparator},\rtfsp
\helpref{wxToolBar::AddCheckTool}{wxtoolbaraddchecktool},\rtfsp
\helpref{wxToolBar::AddRadioTool}{wxtoolbaraddradiotool},\rtfsp
\helpref{wxToolBar::InsertTool}{wxtoolbarinserttool},\rtfsp
\helpref{wxToolBar::DeleteTool}{wxtoolbardeletetool},\rtfsp
\helpref{wxToolBar::Realize}{wxtoolbarrealize}

\membersection{wxToolBar::AddCheckTool}\label{wxtoolbaraddchecktool}

\func{wxToolBarTool*}{AddCheckTool}{\param{int}{ toolId},\rtfsp
\param{const wxString\&}{ label},\rtfsp
\param{const wxBitmap\&}{ bitmap1},\rtfsp
\param{const wxBitmap\&}{ bitmap2},\rtfsp
\param{const wxString\& }{shortHelpString = ""},\rtfsp
\param{const wxString\& }{longHelpString = ""},\rtfsp
\param{wxObject* }{clientData = NULL}}

Adds a new check (or toggle) tool to the toolbar. The parameters are the same
as in \helpref{wxToolBar::AddTool}{wxtoolbaraddtool}.

\wxheading{See also}

\helpref{wxToolBar::AddTool}{wxtoolbaraddtool}

\membersection{wxToolBar::AddRadioTool}\label{wxtoolbaraddradiotool}

\func{wxToolBarTool*}{AddRadioTool}{\param{int}{ toolId},\rtfsp
\param{const wxString\&}{ label},\rtfsp
\param{const wxBitmap\&}{ bitmap1},\rtfsp
\param{const wxBitmap\&}{ bitmap2},\rtfsp
\param{const wxString\& }{shortHelpString = ""},\rtfsp
\param{const wxString\& }{longHelpString = ""},\rtfsp
\param{wxObject* }{clientData = NULL}}

Adds a new radio tool to the toolbar. Consecutive radio tools form a radio
group such that exactly one button in the group is pressed at any moment, in
other words whenever a button in the group is pressed the previously pressed
button is automatically released. You should avoid having the radio groups of
only one element as it would be impossible for the user to use such button.

By default, the first button in the radio group is initially pressed, the
others are not.

\wxheading{See also}

\helpref{wxToolBar::AddTool}{wxtoolbaraddtool}

\membersection{wxToolBar::DeleteTool}\label{wxtoolbardeletetool}

\func{bool}{DeleteTool}{\param{int }{toolId}}

Removes the specified tool from the toolbar and deletes it. If you don't want
to delete the tool, but just to remove it from the toolbar (to possibly add it
back later), you may use \helpref{RemoveTool}{wxtoolbarremovetool} instead.

Note that it is unnecessary to call \helpref{Realize}{wxtoolbarrealize} for the
change to take place, it will happen immediately.

Returns true if the tool was deleted, false otherwise.

\wxheading{See also}

\helpref{DeleteToolByPos}{wxtoolbardeletetoolbypos}

\membersection{wxToolBar::DeleteToolByPos}\label{wxtoolbardeletetoolbypos}

\func{bool}{DeleteToolByPos}{\param{size\_t }{pos}}

This function behaves like \helpref{DeleteTool}{wxtoolbardeletetool} but it
deletes the tool at the specified position and not the one with the given id.

\membersection{wxToolBar::EnableTool}\label{wxtoolbarenabletool}

\func{void}{EnableTool}{\param{int }{toolId}, \param{const bool}{ enable}}

Enables or disables the tool.

\wxheading{Parameters}

\docparam{toolId}{Tool to enable or disable.}

\docparam{enable}{If true, enables the tool, otherwise disables it.}

{\bf NB:} This function should only be called after 
\helpref{Realize}{wxtoolbarrealize}.

\wxheading{Remarks}

For wxToolBarSimple, does nothing. Some other implementations
will change the visible state of the tool to indicate that it is disabled.

\wxheading{See also}

\helpref{wxToolBar::GetToolEnabled}{wxtoolbargettoolenabled},\rtfsp
\helpref{wxToolBar::ToggleTool}{wxtoolbartoggletool}

\membersection{wxToolBar::FindById}\label{wxtoolbarfindbyid}

\func{wxToolBarTool*}{FindById}{\param{int }{id}}

Returns a pointer to the tool identified by {\it id} or
NULL if no corresponding tool is found.

\membersection{wxToolBar::FindControl}\label{wxtoolbarfindcontrol}

\func{wxControl*}{FindControl}{\param{int }{id}}

Returns a pointer to the control identified by {\it id} or
NULL if no corresponding control is found.

\membersection{wxToolBar::FindToolForPosition}\label{wxtoolbarfindtoolforposition}

\constfunc{wxToolBarTool*}{FindToolForPosition}{\param{const float}{ x}, \param{const float}{ y}}

Finds a tool for the given mouse position.

\wxheading{Parameters}

\docparam{x}{X position.}

\docparam{y}{Y position.}

\wxheading{Return value}

A pointer to a tool if a tool is found, or NULL otherwise.

\wxheading{Remarks}

Used internally, and should not need to be used by the programmer.

\membersection{wxToolBar::GetToolSize}\label{wxtoolbargettoolsize}

\func{wxSize}{GetToolSize}{\void}

Returns the size of a whole button, which is usually larger than a tool bitmap because
of added 3D effects.

\wxheading{See also}

\helpref{wxToolBar::SetToolBitmapSize}{wxtoolbarsettoolbitmapsize},\rtfsp
\helpref{wxToolBar::GetToolBitmapSize}{wxtoolbargettoolbitmapsize}

\membersection{wxToolBar::GetToolBitmapSize}\label{wxtoolbargettoolbitmapsize}

\func{wxSize}{GetToolBitmapSize}{\void}

Returns the size of bitmap that the toolbar expects to have. The default bitmap size is 16 by 15 pixels.

\wxheading{Remarks}

Note that this is the size of the bitmap you pass to \helpref{wxToolBar::AddTool}{wxtoolbaraddtool},
and not the eventual size of the tool button.

\wxheading{See also}

\helpref{wxToolBar::SetToolBitmapSize}{wxtoolbarsettoolbitmapsize},\rtfsp
\helpref{wxToolBar::GetToolSize}{wxtoolbargettoolsize}

\membersection{wxToolBar::GetMargins}\label{wxtoolbargetmargins}

\constfunc{wxSize}{GetMargins}{\void}

Returns the left/right and top/bottom margins, which are also used for inter-toolspacing.

\wxheading{See also}

\helpref{wxToolBar::SetMargins}{wxtoolbarsetmargins}

\membersection{wxToolBar::GetToolClientData}\label{wxtoolbargettoolclientdata}

\constfunc{wxObject*}{GetToolClientData}{\param{int }{toolId}}

Get any client data associated with the tool.

\wxheading{Parameters}

\docparam{toolId}{Id of the tool, as passed to \helpref{wxToolBar::AddTool}{wxtoolbaraddtool}.}

\wxheading{Return value}

Client data, or NULL if there is none.

\membersection{wxToolBar::GetToolEnabled}\label{wxtoolbargettoolenabled}

\constfunc{bool}{GetToolEnabled}{\param{int }{toolId}}

Called to determine whether a tool is enabled (responds to user input).

\wxheading{Parameters}

\docparam{toolId}{Id of the tool in question.}

\wxheading{Return value}

true if the tool is enabled, false otherwise.

\wxheading{See also}

\helpref{wxToolBar::EnableTool}{wxtoolbarenabletool}

\membersection{wxToolBar::GetToolLongHelp}\label{wxtoolbargettoollonghelp}

\constfunc{wxString}{GetToolLongHelp}{\param{int }{toolId}}

Returns the long help for the given tool.

\wxheading{Parameters}

\docparam{toolId}{The tool in question.}

\wxheading{See also}

\helpref{wxToolBar::SetToolLongHelp}{wxtoolbarsettoollonghelp},\rtfsp
\helpref{wxToolBar::SetToolShortHelp}{wxtoolbarsettoolshorthelp}\rtfsp

\membersection{wxToolBar::GetToolPacking}\label{wxtoolbargettoolpacking}

\constfunc{int}{GetToolPacking}{\void}

Returns the value used for packing tools.

\wxheading{See also}

\helpref{wxToolBar::SetToolPacking}{wxtoolbarsettoolpacking}

\membersection{wxToolBar::GetToolPos}\label{wxtoolbargettoolpos}

\constfunc{int}{GetToolPos}{\param{int }{toolId}}

Returns the tool position in the toolbar, or {\tt wxNOT\_FOUND} if the tool is not found.

\membersection{wxToolBar::GetToolSeparation}\label{wxtoolbargettoolseparation}

\constfunc{int}{GetToolSeparation}{\void}

Returns the default separator size.

\wxheading{See also}

\helpref{wxToolBar::SetToolSeparation}{wxtoolbarsettoolseparation}

\membersection{wxToolBar::GetToolShortHelp}\label{wxtoolbargettoolshorthelp}

\constfunc{wxString}{GetToolShortHelp}{\param{int }{toolId}}

Returns the short help for the given tool.

\wxheading{Parameters}

\docparam{toolId}{The tool in question.}

\wxheading{See also}

\helpref{wxToolBar::GetToolLongHelp}{wxtoolbargettoollonghelp},\rtfsp
\helpref{wxToolBar::SetToolShortHelp}{wxtoolbarsettoolshorthelp}\rtfsp

\membersection{wxToolBar::GetToolState}\label{wxtoolbargettoolstate}

\constfunc{bool}{GetToolState}{\param{int }{toolId}}

Gets the on/off state of a toggle tool.

\wxheading{Parameters}

\docparam{toolId}{The tool in question.}

\wxheading{Return value}

true if the tool is toggled on, false otherwise.

\wxheading{See also}

\helpref{wxToolBar::ToggleTool}{wxtoolbartoggletool}

\membersection{wxToolBar::InsertControl}\label{wxtoolbarinsertcontrol}

\func{wxToolBarTool *}{InsertControl}{\param{size\_t }{pos}, \param{wxControl *}{control}}

Inserts the control into the toolbar at the given position.

You must call \helpref{Realize}{wxtoolbarrealize} for the change to take place.

\wxheading{See also}

\helpref{AddControl}{wxtoolbaraddcontrol},\\
\helpref{InsertTool}{wxtoolbarinserttool}

\membersection{wxToolBar::InsertSeparator}\label{wxtoolbarinsertseparator}

\func{wxToolBarTool *}{InsertSeparator}{\param{size\_t }{pos}}

Inserts the separator into the toolbar at the given position.

You must call \helpref{Realize}{wxtoolbarrealize} for the change to take place.

\wxheading{See also}

\helpref{AddSeparator}{wxtoolbaraddseparator},\\
\helpref{InsertTool}{wxtoolbarinserttool}

\membersection{wxToolBar::InsertTool}\label{wxtoolbarinserttool}

\func{wxToolBarTool *}{InsertTool}{\param{size\_t }{pos},\rtfsp
\param{int}{ toolId}, \param{const wxBitmap\&}{ bitmap1},\rtfsp
\param{const wxBitmap\&}{ bitmap2 = wxNullBitmap}, \param{bool}{ isToggle = false},\rtfsp
\param{wxObject* }{clientData = NULL}, \param{const wxString\& }{shortHelpString = ""}, \param{const wxString\& }{longHelpString = ""}}

\func{wxToolBarTool *}{InsertTool}{\param{size\_t }{pos},\rtfsp
\param{wxToolBarTool* }{tool}}

Inserts the tool with the specified attributes into the toolbar at the given
position.

You must call \helpref{Realize}{wxtoolbarrealize} for the change to take place.

\wxheading{See also}

\helpref{AddTool}{wxtoolbaraddtool},\\
\helpref{InsertControl}{wxtoolbarinsertcontrol},\\
\helpref{InsertSeparator}{wxtoolbarinsertseparator}

\membersection{wxToolBar::OnLeftClick}\label{wxtoolbaronleftclick}

\func{bool}{OnLeftClick}{\param{int}{ toolId}, \param{bool}{ toggleDown}}

Called when the user clicks on a tool with the left mouse button.

This is the old way of detecting tool clicks; although it will still work,
you should use the EVT\_MENU or EVT\_TOOL macro instead.

\wxheading{Parameters}

\docparam{toolId}{The identifier passed to \helpref{wxToolBar::AddTool}{wxtoolbaraddtool}.}

\docparam{toggleDown}{true if the tool is a toggle and the toggle is down, otherwise is false.}

\wxheading{Return value}

If the tool is a toggle and this function returns false, the toggle
toggle state (internal and visual) will not be changed. This provides a way of
specifying that toggle operations are not permitted in some circumstances.

\wxheading{See also}

\helpref{wxToolBar::OnMouseEnter}{wxtoolbaronmouseenter},\rtfsp
\helpref{wxToolBar::OnRightClick}{wxtoolbaronrightclick}

\membersection{wxToolBar::OnMouseEnter}\label{wxtoolbaronmouseenter}

\func{void}{OnMouseEnter}{\param{int}{ toolId}}

This is called when the mouse cursor moves into a tool or out of
the toolbar.

This is the old way of detecting mouse enter events; although it will still work,
you should use the EVT\_TOOL\_ENTER macro instead.

\wxheading{Parameters}

\docparam{toolId}{Greater than -1 if the mouse cursor has moved into the tool,
or -1 if the mouse cursor has moved. The
programmer can override this to provide extra information about the tool,
such as a short description on the status line.}

\wxheading{Remarks}

With some derived toolbar classes, if the mouse moves quickly out of the toolbar, wxWidgets may not be able to
detect it. Therefore this function may not always be called when expected.

\membersection{wxToolBar::OnRightClick}\label{wxtoolbaronrightclick}

\func{void}{OnRightClick}{\param{int}{ toolId}, \param{float}{ x}, \param{float}{ y}}

Called when the user clicks on a tool with the right mouse button. The
programmer should override this function to detect right tool clicks.

This is the old way of detecting tool right clicks; although it will still work,
you should use the EVT\_TOOL\_RCLICKED macro instead.

\wxheading{Parameters}

\docparam{toolId}{The identifier passed to \helpref{wxToolBar::AddTool}{wxtoolbaraddtool}.}

\docparam{x}{The x position of the mouse cursor.}

\docparam{y}{The y position of the mouse cursor.}

\wxheading{Remarks}

A typical use of this member might be to pop up a menu.

\wxheading{See also}

\helpref{wxToolBar::OnMouseEnter}{wxtoolbaronmouseenter},\rtfsp
\helpref{wxToolBar::OnLeftClick}{wxtoolbaronleftclick}

\membersection{wxToolBar::Realize}\label{wxtoolbarrealize}

\func{bool}{Realize}{\void}

This function should be called after you have added tools.

\membersection{wxToolBar::RemoveTool}\label{wxtoolbarremovetool}

\func{wxToolBarTool *}{RemoveTool}{\param{int }{id}}

Removes the given tool from the toolbar but doesn't delete it. This allows to
insert/add this tool back to this (or another) toolbar later.

Note that it is unnecessary to call \helpref{Realize}{wxtoolbarrealize} for the
change to take place, it will happen immediately.

\wxheading{See also}

\helpref{DeleteTool}{wxtoolbardeletetool}

\membersection{wxToolBar::SetMargins}\label{wxtoolbarsetmargins}

\func{void}{SetMargins}{\param{const wxSize\&}{ size}}

\func{void}{SetMargins}{\param{int}{ x}, \param{int}{ y}}

Set the values to be used as margins for the toolbar.

\wxheading{Parameters}

\docparam{size}{Margin size.}

\docparam{x}{Left margin, right margin and inter-tool separation value.}

\docparam{y}{Top margin, bottom margin and inter-tool separation value.}

\wxheading{Remarks}

This must be called before the tools are added if absolute positioning is to be used, and the
default (zero-size) margins are to be overridden.

\wxheading{See also}

\helpref{wxToolBar::GetMargins}{wxtoolbargetmargins}, \helpref{wxSize}{wxsize}

\membersection{wxToolBar::SetToolBitmapSize}\label{wxtoolbarsettoolbitmapsize}

\func{void}{SetToolBitmapSize}{\param{const wxSize\&}{ size}}

Sets the default size of each tool bitmap. The default bitmap size is 16 by 15 pixels.

\wxheading{Parameters}

\docparam{size}{The size of the bitmaps in the toolbar.}

\wxheading{Remarks}

This should be called to tell the toolbar what the tool bitmap size is. Call
it before you add tools.

Note that this is the size of the bitmap you pass to \helpref{wxToolBar::AddTool}{wxtoolbaraddtool},
and not the eventual size of the tool button.

\wxheading{See also}

\helpref{wxToolBar::GetToolBitmapSize}{wxtoolbargettoolbitmapsize},\rtfsp
\helpref{wxToolBar::GetToolSize}{wxtoolbargettoolsize}

\membersection{wxToolBar::SetToolClientData}\label{wxtoolbarsettoolclientdata}

\func{void}{SetToolClientData}{\param{int }{id}, \param{wxObject* }{clientData}}

Sets the client data associated with the tool.

\membersection{wxToolBar::SetToolLongHelp}\label{wxtoolbarsettoollonghelp}

\func{void}{SetToolLongHelp}{\param{int }{toolId}, \param{const wxString\& }{helpString}}

Sets the long help for the given tool.

\wxheading{Parameters}

\docparam{toolId}{The tool in question.}

\docparam{helpString}{A string for the long help.}

\wxheading{Remarks}

You might use the long help for displaying the tool purpose on the status line.

\wxheading{See also}

\helpref{wxToolBar::GetToolLongHelp}{wxtoolbargettoollonghelp},\rtfsp
\helpref{wxToolBar::SetToolShortHelp}{wxtoolbarsettoolshorthelp},\rtfsp

\membersection{wxToolBar::SetToolPacking}\label{wxtoolbarsettoolpacking}

\func{void}{SetToolPacking}{\param{int}{ packing}}

Sets the value used for spacing tools. The default value is 1.

\wxheading{Parameters}

\docparam{packing}{The value for packing.}

\wxheading{Remarks}

The packing is used for spacing in the vertical direction if the toolbar is horizontal,
and for spacing in the horizontal direction if the toolbar is vertical.

\wxheading{See also}

\helpref{wxToolBar::GetToolPacking}{wxtoolbargettoolpacking}

\membersection{wxToolBar::SetToolShortHelp}\label{wxtoolbarsettoolshorthelp}

\func{void}{SetToolShortHelp}{\param{int }{toolId}, \param{const wxString\& }{helpString}}

Sets the short help for the given tool.

\wxheading{Parameters}

\docparam{toolId}{The tool in question.}

\docparam{helpString}{The string for the short help.}

\wxheading{Remarks}

An application might use short help for identifying the tool purpose in a tooltip.

\wxheading{See also}

\helpref{wxToolBar::GetToolShortHelp}{wxtoolbargettoolshorthelp}, \helpref{wxToolBar::SetToolLongHelp}{wxtoolbarsettoollonghelp}

\membersection{wxToolBar::SetToolSeparation}\label{wxtoolbarsettoolseparation}

\func{void}{SetToolSeparation}{\param{int}{ separation}}

Sets the default separator size. The default value is 5.

\wxheading{Parameters}

\docparam{separation}{The separator size.}

\wxheading{See also}

\helpref{wxToolBar::AddSeparator}{wxtoolbaraddseparator}

\membersection{wxToolBar::ToggleTool}\label{wxtoolbartoggletool}

\func{void}{ToggleTool}{\param{int }{toolId}, \param{const bool}{ toggle}}

Toggles a tool on or off. This does not cause any event to get emitted.

\wxheading{Parameters}

\docparam{toolId}{Tool in question.}

\docparam{toggle}{If true, toggles the tool on, otherwise toggles it off.}

\wxheading{Remarks}

Only applies to a tool that has been specified as a toggle tool.

\wxheading{See also}

\helpref{wxToolBar::GetToolState}{wxtoolbargettoolstate}

