\section{\class{wxStreamBuffer}}\label{wxstreambuffer}

\wxheading{Derived from}

None

\wxheading{Include files}

<wx/stream.h>

\wxheading{See also}

\helpref{wxStreamBase}{wxstreambase}

% ---------------------------------------------------------------------------
% Members
% ---------------------------------------------------------------------------
\latexignore{\rtfignore{\wxheading{Members}}}

% -----------
% ctor & dtor
% -----------
\membersection{wxStreamBuffer::wxStreamBuffer}\label{wxstreambufferctor}

\func{}{wxStreamBuffer}{\param{wxStreamBase\&}{ stream}, \param{BufMode}{ mode}}

Constructor, creates a new stream buffer using {\it stream} as a parent stream
and {\it mode} as the IO mode. {\it mode} can be: wxStreamBuffer::read,
wxStreamBuffer::write, wxStreamBuffer::read\_write.

One stream can have many stream buffers but only one is used internally to
pass IO call (e.g. wxInputStream::Read() -> wxStreamBuffer::Read()), but you
can call directly wxStreamBuffer::Read without any problems. Note that
all errors and messages linked to the stream are stored in the stream, not
the stream buffers:

\begin{verbatim}
  streambuffer.Read(...);
  streambuffer2.Read(...); /* This call erases previous error messages set by 
                              ``streambuffer'' */
\end{verbatim}

\func{}{wxStreamBuffer}{\param{BufMode}{ mode}}

Constructor, creates a new empty stream buffer which won't flush any data
to a stream. {\it mode} specifies the type of the buffer (read, write, read\_write).
This stream buffer has the advantage to be stream independent and to
work only on memory buffers but it is still compatible with the rest of the
wxStream classes. You can write, read to this special stream and it will
grow (if it is allowed by the user) its internal buffer. Briefly, it has all
functionality of a ``normal'' stream.

\wxheading{Warning}

The "read\_write" mode doesn't currently work for standalone stream buffers.

\func{}{wxStreamBuffer}{\param{const wxStreamBuffer\&}{buffer}}

Constructor. It initializes the stream buffer with the data of the specified
stream buffer. The new stream buffer has the same attributes, size, position
and they share the same buffer. This will cause problems if the stream to
which the stream buffer belong is destroyed and the newly cloned stream
buffer continues to be used, trying to call functions in the (destroyed)
stream. It is advised to use this feature only in very local area of the
program.

\wxheading{See also}

\helpref{wxStreamBuffer:SetBufferIO}{wxstreambuffersetbufferio}

\membersection{wxStreamBuffer::\destruct{wxStreamBuffer}}\label{wxstreambufferdtor}

\func{}{wxStreamBuffer}{\destruct{wxStreamBuffer}}

Destructor. It finalizes all IO calls and frees all internal buffers if
necessary.

% -----------
% Filtered IO
% -----------
\membersection{wxStreamBuffer::Read}\label{wxstreambufferread}

\func{size\_t}{Read}{\param{void *}{buffer}, \param{size\_t }{size}}

Reads a block of the specified {\it size} and stores the data in {\it buffer}.
This function tries to read from the buffer first and if more data has been
requested, reads more data from the associated stream and updates the buffer
accordingly until all requested data is read.

\wxheading{Return value}

It returns the size of the data read. If the returned size is different of the specified 
{\it size}, an error has occurred and should be tested using 
\helpref{GetLastError}{wxstreambasegetlasterror}.

\func{size\_t}{Read}{\param{wxStreamBuffer *}{buffer}}

Reads a {\it buffer}. The function returns when {\it buffer} is full or when there isn't
data anymore in the current buffer.

\wxheading{See also}

\helpref{wxStreamBuffer::Write}{wxstreambufferwrite}

\membersection{wxStreamBuffer::Write}\label{wxstreambufferwrite}

\func{size\_t}{Write}{\param{const void *}{buffer}, \param{size\_t }{size}}

Writes a block of the specified {\it size} using data of {\it buffer}. The data
are cached in a buffer before being sent in one block to the stream.

\func{size\_t}{Write}{\param{wxStreamBuffer *}{buffer}}

See \helpref{Read}{wxstreambufferread}.

\membersection{wxStreamBuffer::GetChar}\label{wxstreambuffergetchar}

\func{char}{GetChar}{\void}

Gets a single char from the stream buffer. It acts like the Read call.

\wxheading{Problem}

You aren't directly notified if an error occurred during the IO call.

\wxheading{See also}

\helpref{wxStreamBuffer::Read}{wxstreambufferread}

\membersection{wxStreamBuffer::PutChar}\label{wxstreambufferputchar}

\func{void}{PutChar}{\param{char }{c}}

Puts a single char to the stream buffer.

\wxheading{Problem}

You aren't directly notified if an error occurred during the IO call.

\wxheading{See also}

\helpref{wxStreamBuffer::Read}{wxstreambufferwrite}

\membersection{wxStreamBuffer::Tell}\label{wxstreambuffertell}

\constfunc{off\_t}{Tell}{\void}

Gets the current position in the stream. This position is calculated from
the {\it real} position in the stream and from the internal buffer position: so
it gives you the position in the {\it real} stream counted from the start of
the stream.

\wxheading{Return value}

Returns the current position in the stream if possible, wxInvalidOffset in the
other case.

\membersection{wxStreamBuffer::Seek}\label{wxstreambufferseek}

\func{off\_t}{Seek}{\param{off\_t }{pos}, \param{wxSeekMode }{mode}}

Changes the current position.

{\it mode} may be one of the following:

\twocolwidtha{5cm}
\begin{twocollist}\itemsep=0pt
\twocolitem{{\bf wxFromStart}}{The position is counted from the start of the stream.}
\twocolitem{{\bf wxFromCurrent}}{The position is counted from the current position of the stream.}
\twocolitem{{\bf wxFromEnd}}{The position is counted from the end of the stream.}
\end{twocollist}

\wxheading{Return value}

Upon successful completion, it returns the new offset as measured in bytes from
the beginning of the stream. Otherwise, it returns wxInvalidOffset.

% --------------
% Buffer control
% --------------
\membersection{wxStreamBuffer::ResetBuffer}\label{wxstreambufferresetbuffer}

\func{void}{ResetBuffer}{\void}

Resets to the initial state variables concerning the buffer.

\membersection{wxStreamBuffer::SetBufferIO}\label{wxstreambuffersetbufferio}

\func{void}{SetBufferIO}{\param{char*}{ buffer\_start}, \param{char*}{ buffer\_end}}

Specifies which pointers to use for stream buffering. You need to pass a pointer on the
start of the buffer end and another on the end. The object will use this buffer
to cache stream data. It may be used also as a source/destination buffer when
you create an empty stream buffer (See \helpref{wxStreamBuffer::wxStreamBuffer}{wxstreambufferctor}).

\wxheading{Remarks}

When you use this function, you will have to destroy the IO buffers yourself
after the stream buffer is destroyed or don't use it anymore.
In the case you use it with an empty buffer, the stream buffer will not resize
it when it is full.

\wxheading{See also}

\helpref{wxStreamBuffer constructor}{wxstreambufferctor}\\
\helpref{wxStreamBuffer::Fixed}{wxstreambufferfixed}\\
\helpref{wxStreamBuffer::Flushable}{wxstreambufferflushable}

\func{void}{SetBufferIO}{\param{size\_t}{ bufsize}}

Destroys or invalidates the previous IO buffer and allocates a new one of the
specified size.

\wxheading{Warning}

All previous pointers aren't valid anymore.

\wxheading{Remark}

The created IO buffer is growable by the object.

\wxheading{See also}

\helpref{wxStreamBuffer::Fixed}{wxstreambufferfixed}\\
\helpref{wxStreamBuffer::Flushable}{wxstreambufferflushable}

\membersection{wxStreamBuffer::GetBufferStart}\label{wxstreambuffergetbufferstart}

\constfunc{char *}{GetBufferStart}{\void}

Returns a pointer on the start of the stream buffer.

\membersection{wxStreamBuffer::GetBufferEnd}\label{wxstreambuffergetbufferend}

\constfunc{char *}{GetBufferEnd}{\void}

Returns a pointer on the end of the stream buffer.

\membersection{wxStreamBuffer::GetBufferPos}\label{wxstreambuffergetbufferpos}

\constfunc{char *}{GetBufferPos}{\void}

Returns a pointer on the current position of the stream buffer.

\membersection{wxStreamBuffer::GetIntPosition}\label{wxstreambuffergetintposition}

\constfunc{off\_t}{GetIntPosition}{\void}

Returns the current position (counted in bytes) in the stream buffer.

\membersection{wxStreamBuffer::SetIntPosition}\label{wxstreambuffersetintposition}

\func{void}{SetIntPosition}{\param{size\_t}{ pos}}

Sets the current position (in bytes) in the stream buffer.

\wxheading{Warning}

Since it is a very low-level function, there is no check on the position:
specifying an invalid position can induce unexpected results.

\membersection{wxStreamBuffer::GetLastAccess}\label{wxstreambuffergetlastaccess}

\constfunc{size\_t}{GetLastAccess}{\void}

Returns the amount of bytes read during the last IO call to the parent stream.

\membersection{wxStreamBuffer::Fixed}\label{wxstreambufferfixed}

\func{void}{Fixed}{\param{bool}{ fixed}}

Toggles the fixed flag. Usually this flag is toggled at the same time as 
{\it flushable}. This flag allows (when it has the false value) or forbids
(when it has the true value) the stream buffer to resize dynamically the IO buffer.

\wxheading{See also}

\helpref{wxStreamBuffer::SetBufferIO}{wxstreambuffersetbufferio}

\membersection{wxStreamBuffer::Flushable}\label{wxstreambufferflushable}

\func{void}{Flushable}{\param{bool}{ flushable}}

Toggles the flushable flag. If {\it flushable} is disabled, no data are sent
to the parent stream.

\membersection{wxStreamBuffer::FlushBuffer}\label{wxstreambufferflushbuffer}

\func{bool}{FlushBuffer}{\void}

Flushes the IO buffer.

\membersection{wxStreamBuffer::FillBuffer}\label{wxstreambufferfillbuffer}

\func{bool}{FillBuffer}{\void}

Fill the IO buffer.

\membersection{wxStreamBuffer::GetDataLeft}\label{wxstreambuffergetdataleft}

\func{size\_t}{GetDataLeft}{\void}

Returns the amount of available data in the buffer.

% --------------
% Administration
% --------------
\membersection{wxStreamBuffer::Stream}\label{wxstreambufferstream}

\func{wxStreamBase*}{Stream}{\void}

Returns the parent stream of the stream buffer.

