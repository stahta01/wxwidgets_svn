\section{\class{wxIdleEvent}}\label{wxidleevent}

This class is used for idle events, which are generated when the system is idle.

\wxheading{Derived from}

\helpref{wxEvent}{wxevent}\\
\helpref{wxObject}{wxobject}

\wxheading{Include files}

<wx/event.h>

\wxheading{Event table macros}

To process an idle event, use this event handler macro to direct input to a member
function that takes a wxIdleEvent argument.

\twocolwidtha{7cm}
\begin{twocollist}\itemsep=0pt
\twocolitem{{\bf EVT\_IDLE(func)}}{Process a wxEVT\_IDLE event.}
\end{twocollist}%

\wxheading{Remarks}

Idle events can be caught by the wxApp class, or by top-level window classes.

\wxheading{See also}

\helpref{Event handling overview}{eventhandlingoverview}

\latexignore{\rtfignore{\wxheading{Members}}}

\membersection{wxIdleEvent::wxIdleEvent}

\func{}{wxIdleEvent}{\void}

Constructor.

\membersection{wxIdleEvent::RequestMore}\label{wxidleeventrequestmore}

\func{void}{RequestMore}{\param{bool}{ needMore = true}}

Tells wxWindows that more processing is required. This function can be called by an OnIdle
handler for a window or window event handler to indicate that wxApp::OnIdle should
forward the OnIdle event once more to the application windows. If no window calls this function
during OnIdle, then the application will remain in a passive event loop (not calling OnIdle) until a
new event is posted to the application by the windowing system.

\wxheading{See also}

\helpref{wxIdleEvent::MoreRequested}{wxidleeventmorerequested}

\membersection{wxIdleEvent::MoreRequested}\label{wxidleeventmorerequested}

\constfunc{bool}{MoreRequested}{\void}

Returns true if the OnIdle function processing this event requested more processing time.

\wxheading{See also}

\helpref{wxIdleEvent::RequestMore}{wxidleeventrequestmore}

