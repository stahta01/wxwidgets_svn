%
% automatically generated by HelpGen from
% /home/vasek/fs\_mem.h at 27/Feb/00 19:23:10
%

\section{\class{wxMemoryFSHandler}}\label{wxmemoryfshandler}

This \helpref{wxFileSystem}{wxfilesystem} handler can store arbitrary 
data in memory stream and make them accessible via URL. It is particularly
suitable for storing bitmaps from resources or included XPM files so that
they can be used with wxHTML.

Filenames are prefixed with "memory:", e.g. "memory:myfile.html".

Example:

\begin{verbatim}
#ifndef __WXMSW__
#include "logo.xpm"
#endif

void MyFrame::OnAbout(wxCommandEvent&)
{
    wxBusyCursor bcur;
    wxFileSystem::AddHandler(new wxMemoryFSHandler);
    wxMemoryFSHandler::AddFile("logo.pcx", wxBITMAP(logo), wxBITMAP_TYPE_PCX);
    wxMemoryFSHandler::AddFile("about.htm", 
                               "<html><body>About: "
                               "<img src=\"memory:logo.pcx\"></body></html>");

    wxDialog dlg(this, -1, wxString(_("About")));
    wxBoxSizer *topsizer;
    wxHtmlWindow *html;
    topsizer = new wxBoxSizer(wxVERTICAL);
    html = new wxHtmlWindow(&dlg, -1, wxDefaultPosition, 
                            wxSize(380, 160), wxHW_SCROLLBAR_NEVER);
    html->SetBorders(0);
    html->LoadPage("memory:about.htm");
    html->SetSize(html->GetInternalRepresentation()->GetWidth(), 
                  html->GetInternalRepresentation()->GetHeight());
    topsizer->Add(html, 1, wxALL, 10);
    topsizer->Add(new wxStaticLine(&dlg, -1), 0, wxEXPAND | wxLEFT | wxRIGHT, 10);
    topsizer->Add(new wxButton(&dlg, wxID_OK, "Ok"), 
                  0, wxALL | wxALIGN_RIGHT, 15);
    dlg.SetAutoLayout(true);
    dlg.SetSizer(topsizer);
    topsizer->Fit(&dlg);
    dlg.Centre();
    dlg.ShowModal();
    
    wxMemoryFSHandler::RemoveFile("logo.pcx");
    wxMemoryFSHandler::RemoveFile("about.htm");
}
\end{verbatim}


\wxheading{Derived from}

\helpref{wxFileSystemHandler}{wxfilesystemhandler}

\wxheading{Include files}

<wx/fs\_mem.h>


\latexignore{\rtfignore{\wxheading{Members}}}


\membersection{wxMemoryFSHandler::wxMemoryFSHandler}\label{wxmemoryfshandlerwxmemoryfshandler}

\func{}{wxMemoryFSHandler}{\void}

Constructor.

\membersection{wxMemoryFSHandler::AddFile}\label{wxmemoryfshandleraddfile}

\func{static void}{AddFile}{\param{const wxString\& }{filename}, \param{wxImage\& }{image}, \param{long }{type}}

\func{static void}{AddFile}{\param{const wxString\& }{filename}, \param{const wxBitmap\& }{bitmap}, \param{long }{type}}

\func{static void}{AddFile}{\param{const wxString\& }{filename}, \param{const wxString\& }{textdata}}

\func{static void}{AddFile}{\param{const wxString\& }{filename}, \param{const void* }{binarydata}, \param{size\_t }{size}}

Add file to list of files stored in memory. Stored 
data (bitmap, text or raw data)
will be copied into private memory stream and available under 
name "memory:" + filename.

Note that when storing image/bitmap, you must use image format that wxWindows
can write (e.g. JPG, PNG, see \helpref{wxImage documentation}{wximage})!


\membersection{wxMemoryFSHandler::RemoveFile}\label{wxmemoryfshandlerremovefile}

\func{static void}{RemoveFile}{\param{const wxString\& }{filename}}

Remove file from memory FS and free occupied memory.

