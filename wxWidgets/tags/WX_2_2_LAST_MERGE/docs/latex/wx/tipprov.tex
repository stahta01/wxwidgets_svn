\section{\class{wxTipProvider}}\label{wxtipprovider}

This is the class used together with \helpref{wxShowTip}{wxshowtip} function.
It must implement \helpref{GetTip}{wxtipprovidergettip} function and return the
current tip from it (different tip each time it is called).

You will never use this class yourself, but you need it to show startup tips
with wxShowTip. Also, if you want to get the tips text from elsewhere than a
simple text file, you will want to derive a new class from wxTipProvider and
use it instead of the one returned by \helpref{wxCreateFileTipProvider}{wxcreatefiletipprovider}.

\wxheading{Derived from}

None.

\wxheading{Include files}

<wx/tipdlg.h>

\wxheading{See also}

\helpref{Startup tips overview}{tipsoverview}, \helpref{::wxShowTip}{wxshowtip}

\latexignore{\rtfignore{\wxheading{Members}}}

\membersection{wxTipProvider::wxTipProvider}\label{wxtipproviderctor}

\func{}{wxTipProvider}{\param{size\_t }{currentTip}}

Constructor.

\docparam{currentTip}{The starting tip index.}

\membersection{wxTipProvider::GetTip}\label{wxtipprovidergettip}

\func{wxString}{GetTip}{\void}

Return the text of the current tip and pass to the next one. This function is
pure virtual, it should be implemented in the derived classes.

\membersection{wxCurrentTipProvider::GetCurrentTip}\label{wxtipprovidergetcurrenttip}

\constfunc{size\_t}{GetCurrentTip}{\void}

Returns the index of the current tip (i.e. the one which would be returned by
GetTip).

The program usually remembers the value returned by this function after calling 
\helpref{wxShowTip}{wxshowtip}. Note that it is not the same as the value which
was passed to wxShowTip $+ 1$ because the user might have pressed the "Next"
button in the tip dialog.

