\section{\class{wxTreeCtrl}}\label{wxtreectrl}

A tree control presents information as a hierarchy, with items that may be expanded
to show further items. Items in a tree control are referenced by long integer handles.

To intercept events from a tree control, use the event table macros described in \helpref{wxTreeEvent}{wxtreeevent}.

\wxheading{Derived from}

\helpref{wxControl}{wxcontrol}\\
\helpref{wxWindow}{wxwindow}\\
\helpref{wxEvtHandler}{wxevthandler}\\
\helpref{wxObject}{wxobject}

\wxheading{Window styles}

\twocolwidtha{5cm}
\begin{twocollist}\itemsep=0pt
\twocolitem{\windowstyle{wxTR\_HAS\_BUTTONS}}{Use this style to show + and - buttons to the
left of parent items.}
\twocolitem{\windowstyle{wxTR\_EDIT\_LABELS}}{Use this style if you wish the user to be
able to edit labels in the tree control.}
\end{twocollist}

See also \helpref{window styles overview}{windowstyles}.

\wxheading{Event handling}

To process input from a tree control, use these event handler macros to direct input to member
functions that take a \helpref{wxTreeEvent}{wxtreeevent} argument.

\twocolwidtha{7cm}
\begin{twocollist}\itemsep=0pt
\twocolitem{{\bf EVT\_TREE\_BEGIN\_DRAG(id, func)}}{Begin dragging with the left mouse button.}
\twocolitem{{\bf EVT\_TREE\_BEGIN\_RDRAG(id, func)}}{Begin dragging with the right mouse button.}
\twocolitem{{\bf EVT\_TREE\_BEGIN\_LABEL\_EDIT(id, func)}}{Begin editing a label.}
\twocolitem{{\bf EVT\_TREE\_END\_LABEL\_EDIT(id, func)}}{Finish editing a label.}
\twocolitem{{\bf EVT\_TREE\_DELETE\_ITEM(id, func)}}{Delete an item.}
\twocolitem{{\bf EVT\_TREE\_GET\_INFO(id, func)}}{Request information from the application.}
\twocolitem{{\bf EVT\_TREE\_SET\_INFO(id, func)}}{Information is being supplied.}
\twocolitem{{\bf EVT\_TREE\_ITEM\_EXPANDED(id, func)}}{Parent has been expanded.}
\twocolitem{{\bf EVT\_TREE\_ITEM\_EXPANDING(id, func)}}{Parent is being expanded.}
\twocolitem{{\bf EVT\_TREE\_SEL\_CHANGED(id, func)}}{Selection has changed.}
\twocolitem{{\bf EVT\_TREE\_SEL\_CHANGING(id, func)}}{Selection is changing.}
\twocolitem{{\bf EVT\_TREE\_KEY\_DOWN(id, func)}}{A key has been pressed.}
\end{twocollist}%

\wxheading{See also}

\helpref{wxTreeCtrl overview}{wxtreectrloverview}, \helpref{wxListBox}{wxlistbox}, \helpref{wxListCtrl}{wxlistctrl},\rtfsp
\helpref{wxImageList}{wximagelist}, \helpref{wxTreeEvent}{wxtreeevent}

\latexignore{\rtfignore{\wxheading{Members}}}

\membersection{wxTreeCtrl::wxTreeCtrl}\label{wxtreectrlconstr}

\func{}{wxTreeCtrl}{\void}

Default constructor.

\func{}{wxTreeCtrl}{\param{wxWindow*}{ parent}, \param{wxWindowID}{ id},\rtfsp
\param{const wxPoint\&}{ pos = wxDefaultPosition}, \param{const wxSize\&}{ size = wxDefaultSize},\rtfsp
\param{long}{ style = wxTR\_HAS\_BUTTONS}, \param{const wxValidator\& }{validator = wxDefaultValidator}, \param{const wxString\& }{name = ``listCtrl"}}

Constructor, creating and showing a tree control.

\wxheading{Parameters}

\docparam{parent}{Parent window. Must not be NULL.}

\docparam{id}{Window identifier. A value of -1 indicates a default value.}

\docparam{pos}{Window position.}

\docparam{size}{Window size. If the default size (-1, -1) is specified then the window is sized
appropriately.}

\docparam{style}{Window style. See \helpref{wxTreeCtrl}{wxtreectrl}.}

\docparam{validator}{Window validator.}

\docparam{name}{Window name.}

\wxheading{See also}

\helpref{wxTreeCtrl::Create}{wxtreectrlcreate}, \helpref{wxValidator}{wxvalidator}

\membersection{wxTreeCtrl::\destruct{wxTreeCtrl}}

\func{void}{\destruct{wxTreeCtrl}}{\void}

Destructor, destroying the list control.

\membersection{wxTreeCtrl::Create}\label{wxtreectrlcreate}

\func{bool}{wxTreeCtrl}{\param{wxWindow*}{ parent}, \param{wxWindowID}{ id},\rtfsp
\param{const wxPoint\&}{ pos = wxDefaultPosition}, \param{const wxSize\&}{ size = wxDefaultSize},\rtfsp
\param{long}{ style = wxTR\_HAS\_BUTTONS}, \param{const wxValidator\& }{validator = wxDefaultValidator}, \param{const wxString\& }{name = ``listCtrl"}}

Creates the tree control. See \helpref{wxTreeCtrl::wxTreeCtrl}{wxtreectrlconstr} for further details.

\membersection{wxTreeCtrl::DeleteAllItems}\label{wxtreectrldeleteallitems}

\func{bool}{DeleteAllItems}{\void}

Deletes all the items in the control.

\membersection{wxTreeCtrl::DeleteItem}\label{wxtreectrldeleteitem}

\func{bool}{DeleteItem}{\param{long }{item}}

Deletes the specified item.

\membersection{wxTreeCtrl::EditLabel}\label{wxtreectrleditlabel}

\func{wxTextCtrl*}{EditLabel}{\param{long }{item}, \param{wxClassInfo*}{ textControlClass = CLASSINFO(wxTextCtrl)}}

Starts editing the label of the given item, returning the text control that the tree control uses for editing.

Pass another {\it textControlClass} if a derived class is required. It usually will be, in order for
the application to detect when editing has finished and to call \helpref{wxTreeCtrl::EndEditLabel}{wxtreectrlendeditlabel}.

Do not delete the text control yourself.

This function is currently supported under Windows only.

\wxheading{See also}

\helpref{wxTreeCtrl::EndEditLabel}{wxtreectrlendeditlabel}

\membersection{wxTreeCtrl::EndEditLabel}\label{wxtreectrlendeditlabel}

\func{bool}{EndEditLabel}{\param{bool }{cancelEdit}}

Ends label editing. If {\it cancelEdit} is TRUE, the edit will be cancelled.

This function is currently supported under Windows only.

\wxheading{See also}

\helpref{wxTreeCtrl::EditLabel}{wxtreectrleditlabel}

\membersection{wxTreeCtrl::EnsureVisible}\label{wxtreectrlensurevisible}

\func{bool}{EnsureVisible}{\param{long }{item}}

Scrolls and/or expands items to ensure that the given item is visible.

\membersection{wxTreeCtrl::ExpandItem}\label{wxtreectrlexpanditem}

\func{bool}{ExpandItem}{\param{long }{item}, \param{int }{action}}

Expands the given item.

{\it action} may be one of:

\twocolwidtha{5cm}
\begin{twocollist}\itemsep=0pt
\twocolitem{\windowstyle{wxTREE\_EXPAND\_EXPAND}}{Expands the item.}
\twocolitem{\windowstyle{wxTREE\_EXPAND\_COLLAPSE}}{Collapses the item.}
\twocolitem{\windowstyle{wxTREE\_EXPAND\_COLLAPSE\_RESET}}{Collapses the item and removes the child items.}
\twocolitem{\windowstyle{wxTREE\_EXPAND\_TOGGLE}}{Expands if the item is collapsed, collapses if the item is expanded.}
\end{twocollist}

\membersection{wxTreeCtrl::GetChild}\label{wxtreectrlgetchild}

\constfunc{long}{GetChild}{\param{long }{item}}

Call this function to retrieve the tree view item that is the first child of the item specified by {\it item}.

\membersection{wxTreeCtrl::GetCount}\label{wxtreectrlgetcount}

\constfunc{int}{GetCount}{\void}

Returns the number of items in the control.

\membersection{wxTreeCtrl::GetEditControl}\label{wxtreectrlgeteditcontrol}

\constfunc{wxTextCtrl\&}{GetEditControl}{\void}

Returns the edit control used to edit a label.

\membersection{wxTreeCtrl::GetFirstVisibleItem}\label{wxtreectrlgetfirstvisibleitem}

\constfunc{long}{GetFirstVisibleItem}{\void}

Returns the first visible item.

\membersection{wxTreeCtrl::GetImageList}\label{wxtreectrlgetimagelist}

\constfunc{wxImageList*}{GetImageList}{\param{int }{which = wxIMAGE\_LIST\_NORMAL}}

Returns the specified image list. {\it which} may be one of:

\twocolwidtha{5cm}
\begin{twocollist}\itemsep=0pt
\twocolitem{\windowstyle{wxIMAGE\_LIST\_NORMAL}}{The normal (large icon) image list.}
\twocolitem{\windowstyle{wxIMAGE\_LIST\_SMALL}}{The small icon image list.}
\twocolitem{\windowstyle{wxIMAGE\_LIST\_STATE}}{The user-defined state image list (unimplemented).}
\end{twocollist}

\membersection{wxTreeCtrl::GetIndent}\label{wxtreectrlgetindent}

\constfunc{int}{GetIndent}{\void}

Returns the current tree control indentation.

\membersection{wxTreeCtrl::GetItem}\label{wxtreectrlgetitem}

\constfunc{bool}{GetItem}{\param{wxTreeItem\& }{info}}

Gets information about the item. See \helpref{wxTreeCtrl::SetItem}{wxtreectrlsetitem} for more
information.

\membersection{wxTreeCtrl::GetItemData}\label{wxtreectrlgetitemdata}

\constfunc{long}{GetItemData}{\param{long }{item}}

Returns the client data associated with the item, if any.

\membersection{wxTreeCtrl::GetItemRect}\label{wxtreectrlgetitemrect}

\constfunc{bool}{GetItemRect}{\param{long }{item}, \param{wxRect\& }{rect}, \param{bool }{textOnly = FALSE}}

Returns the position and size of the rectangle bounding the item.

\membersection{wxTreeCtrl::GetItemState}\label{wxtreectrlgetitemstate}

\constfunc{int}{GetItemState}{\param{long }{item}, \param{long }{stateMask}}

Gets the item state. For a list of state flags, see \helpref{wxTreeCtrl::SetItem}{wxtreectrlsetitem}.

\membersection{wxTreeCtrl::GetItemText}\label{wxtreectrlgetitemtext}

\constfunc{wxString}{GetItemText}{\param{long }{item}}

Returns the item label.

\membersection{wxTreeCtrl::GetNextItem}\label{wxtreectrlgetnextitem}

\constfunc{long}{GetNextItem}{\param{long }{item}, \param{int }{code}}

Searches for an item using the given criterion, starting from {\it item}.

Returns the item or 0 if unsuccessful.

{\it code} can be one of:

\twocolwidtha{5cm}
\begin{twocollist}\itemsep=0pt
\twocolitem{wxTREE\_NEXT\_CARET}{Retrieves the currently selected item.}
\twocolitem{wxTREE\_NEXT\_CHILD}{Retrieves the first child item. The hItem parameter must be NULL.}
\twocolitem{wxTREE\_NEXT\_DROPHILITE}{Retrieves the item that is the target of a drag-and-drop operation.}
\twocolitem{wxTREE\_NEXT\_FIRSTVISIBLE}{Retrieves the first visible item.}
\twocolitem{wxTREE\_NEXT\_NEXT}{Retrieves the next sibling item.}
\twocolitem{wxTREE\_NEXT\_NEXTVISIBLE}{Retrieves the next visible item that follows the specified item.}
\twocolitem{wxTREE\_NEXT\_PARENT}{Retrieves the parent of the specified item.}
\twocolitem{wxTREE\_NEXT\_PREVIOUS}{Retrieves the previous sibling item.}
\twocolitem{wxTREE\_NEXT\_PREVIOUSVISIBLE}{Retrieves the first visible item that precedes the specified item.}
\twocolitem{wxTREE\_NEXT\_ROOT}{Retrieves the first child item of the root item of which the specified item is a part.}
\end{twocollist}

\membersection{wxTreeCtrl::GetNextVisibleItem}\label{wxtreectrlgetnextvisibleitem}

\constfunc{long}{GetNextVisibleItem}{\param{long }{item}}

Returns the next visible item.

\membersection{wxTreeCtrl::GetParent}\label{wxtreectrlgetparent}

\constfunc{long}{GetParent}{\param{long }{item}}

Returns the item's parent.

\membersection{wxTreeCtrl::GetRootItem}\label{wxtreectrlgetrootitem}

\constfunc{long}{GetRootItem}{\void}

Returns the root item for the tree control.

\membersection{wxTreeCtrl::GetSelection}\label{wxtreectrlgetselection}

\constfunc{long}{GetSelection}{\void}

Returns the selection, or 0 if there is no selection.

\membersection{wxTreeCtrl::HitTest}\label{wxtreectrlhittest}

\func{long}{HitTest}{\param{const wxPoint\& }{point}, \param{int\& }{flags}}

Calculates which (if any) item is under the given point, returning extra information
in {\it flags}. {\it flags} is a bitlist of the following:

\twocolwidtha{5cm}
\begin{twocollist}\itemsep=0pt
\twocolitem{wxTREE\_HITTEST\_ABOVE}{Above the client area.}
\twocolitem{wxTREE\_HITTEST\_BELOW}{Below the client area.}
\twocolitem{wxTREE\_HITTEST\_NOWHERE}{In the client area but below the last item.}
\twocolitem{wxTREE\_HITTEST\_ONITEMBUTTON}{On the button associated with an item.}
\twocolitem{wxTREE\_HITTEST\_ONITEMICON}{On the bitmap associated with an item.}
\twocolitem{wxTREE\_HITTEST\_ONITEMINDENT}{In the indentation associated with an item.}
\twocolitem{wxTREE\_HITTEST\_ONITEMLABEL}{On the label (string) associated with an item.}
\twocolitem{wxTREE\_HITTEST\_ONITEMRIGHT}{In the area to the right of an item.}
\twocolitem{wxTREE\_HITTEST\_ONITEMSTATEICON}{On the state icon for a tree view item that is in a user-defined state.}
\twocolitem{wxTREE\_HITTEST\_TOLEFT}{To the right of the client area.}
\twocolitem{wxTREE\_HITTEST\_TORIGHT}{To the left of the client area.}
\end{twocollist}

\membersection{wxTreeCtrl::InsertItem}\label{wxtreectrlinsertitem}

\func{long}{InsertItem}{\param{long }{parent}, \param{wxTreeItem\& }{info}, \param{long }{insertAfter = wxTREE\_INSERT\_LAST}}

Inserts an item. For more information on {\it info}, see \helpref{wxTreeCtrl::SetItem}{wxtreectrlsetitem}.

\func{long}{InsertItem}{\param{long }{parent}, \param{const wxString\& }{label}, \param{int }{image = -1}, \param{int }{selImage = -1}, \param{long }{insertAfter = wxTREE\_INSERT\_LAST}}

Inserts an item.

If {\it image} > -1 and {\it selImage} is -1, the same image is used for
both selected and unselected items.

\membersection{wxTreeCtrl::ItemHasChildren}\label{wxtreectrlitemhaschildren}

\constfunc{bool}{ItemHasChildren}{\param{long }{item}}

Returns TRUE if the item has children.

\membersection{wxTreeCtrl::ScrollTo}\label{wxtreectrlscrollto}

\func{bool}{ScrollTo}{\param{long }{item}}

selects the specified item and scrolls the item into view,

\membersection{wxTreeCtrl::SelectItem}\label{wxtreectrlselectitem}

\func{bool}{SelectItem}{\param{long }{item}}

Selects the given item.

\membersection{wxTreeCtrl::SetIndent}\label{wxtreectrlsetindent}

\func{void}{SetIndent}{\param{int }{indent}}

Sets the indentation for the tree control.

\membersection{wxTreeCtrl::SetImageList}\label{wxtreectrlsetimagelist}

\func{void}{SetImageList}{\param{wxImageList*}{ imageList}, \param{int }{which = wxIMAGE\_LIST\_NORMAL}}

Sets the image list. {\it which} should be one of wxIMAGE\_LIST\_NORMAL, wxIMAGE\_LIST\_SMALL and
wxIMAGE\_LIST\_STATE.

\membersection{wxTreeCtrl::SetItem}\label{wxtreectrlsetitem}

\func{bool}{SetItem}{\param{wxTreeItem\& }{info}}

Sets the properties of the item.

The members of wxTreeItem are as follows:

\twocolwidtha{5cm}
\begin{twocollist}\itemsep=0pt
\twocolitem{m\_mask}{A bitlist specifying the valid members. See below for mask flags.}
\twocolitem{m\_itemId}{The item identifier.}
\twocolitem{m\_state}{The item state. See below for state flags.}
\twocolitem{m\_stateMask}{A bitlist specifying the valid contents of {\it m\_state}. These flags
are taken from the same set of symbols as {\it m\_state}.}
\twocolitem{m\_text}{The item label.}
\twocolitem{m\_image}{The item image index (an index into the appropriate image list).}
\twocolitem{m\_selectedImage}{The item selected index (an index into the appropriate image list).}
\twocolitem{m\_children}{The number of child items that this item has.}
\twocolitem{m\_data}{The application-defined data associated with this item.}
\end{twocollist}

Valid mask flags are:

\twocolwidtha{5cm}
\begin{twocollist}\itemsep=0pt
\twocolitem{wxTREE\_MASK\_HANDLE}{The {\it m\_itemId} member is valid.}
\twocolitem{wxTREE\_MASK\_STATE}{The {\it m\_state} member is valid.}
\twocolitem{wxTREE\_MASK\_TEXT}{The {\it m\_text} member is valid.}
\twocolitem{wxTREE\_MASK\_IMAGE}{The {\it m\_image} member is valid.}
\twocolitem{wxTREE\_MASK\_SELECTED\_IMAGE}{The {\it m\_selectedImage} member is valid.}
\twocolitem{wxTREE\_MASK\_CHILDREN}{The {\it m\_children} member is valid.}
\twocolitem{wxTREE\_MASK\_DATA}{The {\it m\_data} member is valid.}
\end{twocollist}

Valid state and state mask flags are:

\twocolwidtha{5cm}
\begin{twocollist}\itemsep=0pt
\twocolitem{wxTREE\_STATE\_BOLD}{The label is emboldened.}
\twocolitem{wxTREE\_STATE\_DROPHILITED}{The item indicates it is a drop target.}
\twocolitem{wxTREE\_STATE\_EXPANDED}{The item is expanded.}
\twocolitem{wxTREE\_STATE\_EXPANDEDONCE}{The item's list of child items has been expanded at least once.}
\twocolitem{wxTREE\_STATE\_FOCUSED}{The item has the focus, so it is surrounded by a standard focus rectangle.
Only one item can have the focus.}
\twocolitem{wxTREE\_STATE\_SELECTED}{The item is selected.}
\twocolitem{wxTREE\_STATE\_CUT}{The item is selected as part of a cut and paste operation.}
\end{twocollist}

\membersection{wxTreeCtrl::SetItemImage}\label{wxtreectrlsetitemimage}

\func{bool}{SetItemImage}{\param{long }{item}, \param{int }{image}, \param{int }{selImage}}

Sets the item image and selected image. These are indices into the assciated image list.

\membersection{wxTreeCtrl::SetItemState}\label{wxtreectrlsetitemstate}

\func{bool}{SetItemState}{\param{long }{item}, \param{long }{state}, \param{long }{stateMask}}

Sets the item state. See \helpref{wxTreeCtrl::SetItem}{wxtreectrlsetitem} for valid state and state mask flags.

\membersection{wxTreeCtrl::SetItemText}\label{wxtreectrlsetitemtext}

\func{void}{SetItemText}{\param{long }{item}, \param{const wxString\& }{text}}

Sets the item label.

\membersection{wxTreeCtrl::SetItemData}\label{wxtreectrlsetitemdata}

\func{bool}{SetItemData}{\param{long }{item}, \param{long }{data}}

Sets the item client data.

\membersection{wxTreeCtrl::SortChildren}\label{wxtreectrlsortchildren}

\func{bool}{SortChildren}{\param{long }{item}}

Sorts the children of the given item in ascending alphabetical order.

