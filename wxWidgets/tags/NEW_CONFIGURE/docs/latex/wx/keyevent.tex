\section{\class{wxKeyEvent}}\label{wxkeyevent}

This event class contains information about keypress (character) events. See \helpref{wxWindow::OnChar}{wxwindowonchar}.

\wxheading{Derived from}

\helpref{wxEvent}{wxevent}

\wxheading{Event table macros}

To process a key event, use these event handler macros to direct input to member
functions that take a wxKeyEvent argument.

\twocolwidtha{7cm}
\begin{twocollist}\itemsep=0pt
\twocolitem{{\bf EVT\_CHAR(func)}}{Process a wxEVT\_CHAR event.}
\twocolitem{{\bf EVT\_CHAR\_HOOK(func)}}{Process a wxEVT\_CHAR\_HOOK event.}
\end{twocollist}%

\latexignore{\rtfignore{\wxheading{Members}}}

\membersection{wxKeyEvent::m\_altDown}

\member{bool}{m\_altDown}

TRUE if the Alt key is pressed down.

\membersection{wxKeyEvent::m\_controlDown}

\member{bool}{m\_controlDown}

TRUE if control is pressed down.

\membersection{wxKeyEvent::m\_keyCode}

\member{long}{m\_keyCode}

Virtual keycode. An enumerated type, one of:

\begin{verbatim}
 WXK_BACK    = 8
 WXK_TAB     = 9
 WXK_RETURN  = 13
 WXK_ESCAPE  = 27
 WXK_SPACE   = 32
 WXK_DELETE  = 127

 WXK_START   = 300
 WXK_LBUTTON
 WXK_RBUTTON
 WXK_CANCEL
 WXK_MBUTTON
 WXK_CLEAR
 WXK_SHIFT
 WXK_CONTROL
 WXK_MENU
 WXK_PAUSE
 WXK_CAPITAL
 WXK_PRIOR
 WXK_NEXT
 WXK_END
 WXK_HOME
 WXK_LEFT
 WXK_UP
 WXK_RIGHT
 WXK_DOWN
 WXK_SELECT
 WXK_PRINT
 WXK_EXECUTE
 WXK_SNAPSHOT
 WXK_INSERT
 WXK_HELP
 WXK_NUMPAD0
 WXK_NUMPAD1
 WXK_NUMPAD2
 WXK_NUMPAD3
 WXK_NUMPAD4
 WXK_NUMPAD5
 WXK_NUMPAD6
 WXK_NUMPAD7
 WXK_NUMPAD8
 WXK_NUMPAD9
 WXK_MULTIPLY
 WXK_ADD
 WXK_SEPARATOR
 WXK_SUBTRACT
 WXK_DECIMAL
 WXK_DIVIDE
 WXK_F1
 WXK_F2
 WXK_F3
 WXK_F4
 WXK_F5
 WXK_F6
 WXK_F7
 WXK_F8
 WXK_F9
 WXK_F10
 WXK_F11
 WXK_F12
 WXK_F13
 WXK_F14
 WXK_F15
 WXK_F16
 WXK_F17
 WXK_F18
 WXK_F19
 WXK_F20
 WXK_F21
 WXK_F22
 WXK_F23
 WXK_F24
 WXK_NUMLOCK
 WXK_SCROLL 
\end{verbatim}

\membersection{wxKeyEvent::m\_metaDown}

\member{bool}{m\_metaDown}

TRUE if the Meta key is pressed down.

\membersection{wxKeyEvent::m\_shiftDown}

\member{bool}{m\_shiftDown}

TRUE if shift is pressed down.

\membersection{wxKeyEvent::m\_x}

\member{int}{m\_x}

X position of the event.

\membersection{wxKeyEvent::m\_y}

\member{int}{m\_y}

Y position of the event.

\membersection{wxKeyEvent::wxKeyEvent}

\func{}{wxKeyEvent}{\param{WXTYPE}{ keyEventType}}

Constructor. Currently, the only valid event types are wxEVT\_CHAR and wxEVT\_CHAR\_HOOK.

\membersection{wxKeyEvent::AltDown}

\func{bool}{AltDown}{\void}

Returns TRUE if the Alt key was down at the time of the key event.

\membersection{wxKeyEvent::ControlDown}

\func{bool}{ControlDown}{\void}

Returns TRUE if the control key was down at the time of the key event.

\membersection{wxKeyEvent::GetX}

\func{float}{GetX}{\void}

Returns the X position of the event.

\membersection{wxKeyEvent::GetY}

\func{float}{GetY}{\void}

Returns the Y position of the event.

\membersection{wxKeyEvent::KeyCode}

\func{long}{KeyCode}{\void}

Returns the virtual key code. ASCII events return normal ASCII values,
while non-ASCII events return values such as {\bf WXK\_LEFT} for the
left cursor key. See {\tt wx\_defs.h} for a full list of the virtual key codes.

\membersection{wxKeyEvent::MetaDown}

\func{bool}{MetaDown}{\void}

Returns TRUE if the Meta key was down at the time of the key event.

\membersection{wxKeyEvent::Position}

\func{void}{Position}{\param{float *}{x}, \param{float *}{y}}

Obtains the position at which the key was pressed.

\membersection{wxKeyEvent::ShiftDown}

\func{bool}{ShiftDown}{\void}

Returns TRUE if the shift key was down at the time of the key event.


