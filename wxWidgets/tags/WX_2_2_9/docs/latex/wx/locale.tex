\section{\class{wxLocale}}\label{wxlocale}

wxLocale class encapsulates all language-dependent settings and is a
generalization of the C locale concept.

In wxWindows this class manages message catalogs which contain the translations
of the strings used to the current language.

\wxheading{Derived from}

No base class

\wxheading{See also}

\helpref{I18n overview}{internationalization}

\wxheading{Include files}

<wx/intl.h>

\latexignore{\rtfignore{\wxheading{Members}}}

\membersection{wxLocale::wxLocale}\label{wxlocaledefctor}

\func{}{wxLocale}{\void}

This is the default constructor and it does nothing to initialize the object: 
\helpref{Init()}{wxlocaleinit} must be used to do that.

\func{}{wxLocale}{\param{const char }{*szName}, \param{const char }{*szShort = NULL}, \param{const char }{*szLocale = NULL}, \param{bool }{bLoadDefault = TRUE}, \param{bool }{bConvertEncoding = FALSE}}

The parameters have the following meaning:
\begin{itemize}\itemsep=0pt
\item szName is the name of the locale and is only used in diagnostic messages
\item szShort is the standard 2 letter locale abbreviation and is used as the
directory prefix when looking for the message catalog files
\item szLocale is the parameter for the call to setlocale()
\item bLoadDefault may be set to FALSE to prevent loading of the message catalog
for the given locale containing the translations of standard wxWindows messages.
This parameter would be rarely used in normal circumstances.
\item bConvertEncoding may be set to TRUE to do automatic conversion of message
catalogs to platform's native encoding. Note that it will do only basic 
conversion between well-known pair like iso8859-1 and windows-1252 or
iso8859-2 and windows-1250. 
See \helpref{Writing non-English applications}{nonenglishoverview} for detailed
description of this behaviour.
\end{itemize}

The call of this function has several global side effects which you should
understand: first of all, the application locale is changed - note that this
will affect many of standard C library functions such as printf() or strftime().
Second, this wxLocale object becomes the new current global locale for the
application and so all subsequent calls to wxGetTranslation() will try to
translate the messages using the message catalogs for this locale. Finally,
unless bLoadDefault parameter is FALSE, the method also loads the wxstd.mo
catalog (which is looked for in all the usual places and, additionally, under
the location specified bythe environment variable WXDIR if it is set) which
will allow to translate all the messages generated by the library itself.

\membersection{wxLocale::\destruct{wxLocale}}\label{wxlocaledtor}

\func{}{\destruct{wxLocale}}{\void}

The destructor, like the constructor, also has global side effects: the previously
set locale is restored and so the changes described in 
\helpref{Init}{wxlocaleinit} documentation are rolled back.

\membersection{wxLocale::AddCatalog}\label{wxlocaleaddcatalog}

\func{bool}{AddCatalog}{\param{const char }{*szDomain}}

Add a catalog for use with the current locale: it is searched for in standard
places (current directory first, then the system one), but you may also prepend
additional directories to the search path with 
\helpref{AddCatalogLookupPathPrefix()}{wxlocaleaddcataloglookuppathprefix}.

All loaded catalogs will be used for message lookup by GetString() for the
current locale.

Returns TRUE if catalog was successfully loaded, FALSE otherwise (which might
mean that the catalog is not found or that it isn't in the correct format).

\membersection{wxLocale::AddCatalogLookupPathPrefix}\label{wxlocaleaddcataloglookuppathprefix}

\func{void}{AddCatalogLookupPathPrefix}{\param{const wxString\& }{prefix}}

Add a prefix to the catalog lookup path: the message catalog files will be
looked up under prefix/<lang>/LC\_MESSAGES, prefix/LC\_MESSAGES and prefix
(in this order).

This only applies to subsequent invocations of AddCatalog()!

\membersection{wxLocale::GetLocale}\label{wxlocalegetlocale}

\constfunc{const char*}{GetLocale}{\void}

Returns the locale name as passed to the constructor or 
\helpref{Init()}{wxlocaleinit}.

\membersection{wxLocale::GetName}\label{wxlocalegetname}

\constfunc{const wxString\&}{GetName}{\void}

Returns the current short name for the locale (as given to the constructor or
the Init() function).

\membersection{wxLocale::GetString}\label{wxlocalegetstring}

\constfunc{const char*}{GetString}{\param{const char }{*szOrigString}, \param{const char }{*szDomain = NULL}}

Retrieves the translation for a string in all loaded domains unless the szDomain
parameter is specified (and then only this catalog/domain is searched).

Returns original string if translation is not available
(in this case an error message is generated the first time
a string is not found; use \helpref{wxLogNull}{wxlogoverview} to suppress it).

\wxheading{Remarks}

Domains are searched in the last to first order, i.e. catalogs
added later override those added before.

\membersection{wxLocale::Init}\label{wxlocaleinit}

\func{bool}{Init}{\param{const char }{*szName}, \param{const char }{*szShort = NULL}, \param{const char }{*szLocale = NULL}, \param{bool }{bLoadDefault = TRUE}}

The parameters have the following meaning:

\begin{itemize}\itemsep=0pt
\item szName is the name of the locale and is only used in diagnostic messages
\item szShort is the standard 2 letter locale abbreviation and is used as the
directory prefix when looking for the message catalog files
\item szLocale is the parameter for the call to setlocale()
\item bLoadDefault may be set to FALSE to prevent loading of the message catalog
for the given locale containing the translations of standard wxWindows messages.
This parameter would be rarely used in normal circumstances.
\end{itemize}

The call of this function has several global side effects which you should
understand: first of all, the application locale is changed - note that this
will affect many of standard C library functions such as printf() or strftime().
Second, this wxLocale object becomes the new current global locale for the
application and so all subsequent calls to wxGetTranslation() will try to
translate the messages using the message catalogs for this locale.

Returns TRUE on success or FALSE if the given locale couldn't be set.

\membersection{wxLocale::IsLoaded}\label{wxlocaleisloaded}

\constfunc{bool}{IsLoaded}{\param{const char* }{domain}}

Check if the given catalog is loaded, and returns TRUE if it is.

According to GNU gettext tradition, each catalog
normally corresponds to 'domain' which is more or less the application name.

See also: \helpref{AddCatalog}{wxlocaleaddcatalog}

\membersection{wxLocale::IsOk}\label{wxlocaleisok}

\constfunc{bool}{IsOk}{\void}

Returns TRUE if the locale could be set successfully.

