%%%%%%%%%%%%%%%%%%%%%%%%%%%%%%%%%%%%%%%%%%%%%%%%%%%%%%%%%%%%%%%%%%%%%%%%%%%%%%%
%% Name:        mbconv.tex
%% Purpose:     wxMBConv documentation
%% Author:      Ove Kaaven, Vadim Zeitlin
%% Created:     2000-03-25
%% RCS-ID:      $Id$
%% Copyright:   (c) 2000 Ove Kaaven
%%              (c) 2003-2006 Vadim Zeitlin
%% License:     wxWindows license
%%%%%%%%%%%%%%%%%%%%%%%%%%%%%%%%%%%%%%%%%%%%%%%%%%%%%%%%%%%%%%%%%%%%%%%%%%%%%%%


\section{\class{wxMBConv}}\label{wxmbconv}

This class is the base class of a hierarchy of classes capable of converting
text strings between multibyte (SBCS or DBCS) encodings and Unicode.

In the documentation for this and related classes please notice that 
\emph{length} of the string refers to the number of characters in the string
not counting the terminating \NUL, if any. While the \emph{size} of the string
is the total number of bytes in the string, including any trailing \NUL.
Thus, length of wide character string \texttt{L"foo"} is $3$ while its size can
be either $8$ or $16$ depending on whether \texttt{wchar\_t} is $2$ bytes (as
under Windows) or $4$ (Unix).

\wxheading{Global variables}

There are several predefined instances of this class:
\begin{twocollist}
\twocolitem{\textbf{wxConvLibc}}{Uses the standard ANSI C \texttt{mbstowcs()} and
\texttt{wcstombs()} functions to perform the conversions; thus depends on the
current locale.}
\twocolitem{\textbf{wxConvLocal}}{Another conversion corresponding to the
current locale but this one uses the best available conversion.}
\twocolitem{\textbf{wxConvUI}}{The conversion used for hte standard UI elements
such as menu items and buttons. This is a pointer which is initially set to
\texttt{wxConvLocal} as the program uses the current locale by default but can
be set to some specific conversion if the program needs to use a specific
encoding for its UI.}
\twocolitem{\textbf{wxConvISO8859\_1}}{Conversion to and from ISO-8859-1 (Latin I)
encoding.}
\twocolitem{\textbf{wxConvUTF8}}{Conversion to and from UTF-8 encoding.}
\twocolitem{\textbf{wxConvFile}}{The appropriate conversion for the file names,
depends on the system.}
% \twocolitem{\textbf{wxConvCurrent}}{Not really clear what is it for...}
\end{twocollist}


\wxheading{Constants}

\texttt{wxCONV\_FAILED} value is defined as \texttt{(size\_t)$-1$} and is
returned by the conversion functions instead of the length of the converted
string if the conversion fails.


\wxheading{Derived from}

No base class

\wxheading{Include files}

<wx/strconv.h>

\wxheading{See also}

\helpref{wxCSConv}{wxcsconv}, 
\helpref{wxEncodingConverter}{wxencodingconverter}, 
\helpref{wxMBConv classes overview}{mbconvclasses}


\latexignore{\rtfignore{\wxheading{Members}}}


\membersection{wxMBConv::wxMBConv}\label{wxmbconvwxmbconv}

\func{}{wxMBConv}{\void}

Trivial default constructor.


\membersection{wxMBConv::MB2WC}\label{wxmbconvmb2wc}

\constfunc{virtual size\_t}{MB2WC}{\param{wchar\_t *}{out}, \param{const char *}{in}, \param{size\_t }{outLen}}

\deprecated{\helpref{ToWChar}{wxmbconvtowchar}}

Converts from a string \arg{in} in multibyte encoding to Unicode putting up to 
\arg{outLen} characters into the buffer \arg{out}.

If \arg{out} is \NULL, only the length of the string which would result from
the conversion is calculated and returned. Note that this is the length and not
size, i.e. the returned value does \emph{not} include the trailing \NUL. But
when the function is called with a non-\NULL \arg{out} buffer, the \arg{outLen} 
parameter should be one more to allow to properly \NUL-terminate the string.

\wxheading{Parameters}

\docparam{out}{The output buffer, may be \NULL if the caller is only
interested in the length of the resulting string}

\docparam{in}{The \NUL-terminated input string, cannot be \NULL}

\docparam{outLen}{The length of the output buffer but \emph{including} 
\NUL, ignored if \arg{out} is \NULL}

\wxheading{Return value}

The length of the converted string \emph{excluding} the trailing \NUL.


\membersection{wxMBConv::WC2MB}\label{wxmbconvwc2mb}

\constfunc{virtual size\_t}{WC2MB}{\param{char* }{buf}, \param{const wchar\_t* }{psz}, \param{size\_t }{n}}

\deprecated{\helpref{FromWChar}{wxmbconvfromwchar}}

Converts from Unicode to multibyte encoding. The semantics of this function
(including the return value meaning) is the same as for 
\helpref{MB2WC}{wxmbconvmb2wc}.

Notice that when the function is called with a non-\NULL buffer, the 
{\it n} parameter should be the size of the buffer and so it \emph{should} take
into account the trailing \NUL, which might take two or four bytes for some
encodings (UTF-16 and UTF-32) and not one.


\membersection{wxMBConv::cMB2WC}\label{wxmbconvcmb2wc}

\constfunc{const wxWCharBuffer}{cMB2WC}{\param{const char *}{in}}

\constfunc{const wxWCharBuffer}{cMB2WC}{\param{const char *}{in}, \param{size\_t }{inLen}, \param{size\_t }{*outLen}}

Converts from multibyte encoding to Unicode by calling 
\helpref{MB2WC}{wxmbconvmb2wc}, allocating a temporary wxWCharBuffer to hold
the result.

The first overload takes a \NUL-terminated input string. The second one takes a
string of exactly the specified length and the string may include or not the
trailing \NUL character(s). If the string is not \NUL-terminated, a temporary 
\NUL-terminated copy of it suitable for passing to \helpref{MB2WC}{wxmbconvmb2wc} 
is made, so it is more efficient to ensure that the string is does have the
appropriate number of \NUL bytes (which is usually $1$ but may be $2$ or $4$
for UTF-16 or UTF-32, see \helpref{GetMBNulLen}{wxmbconvgetmbnullen}),
especially for long strings.

If \arg{outLen} is not-\NULL, it receives the length of the converted
string.


\membersection{wxMBConv::cWC2MB}\label{wxmbconvcwc2mb}

\constfunc{const wxCharBuffer}{cWC2MB}{\param{const wchar\_t* }{in}}

\constfunc{const wxCharBuffer}{cWC2MB}{\param{const wchar\_t* }{in}, \param{size\_t }{inLen}, \param{size\_t }{*outLen}}

Converts from Unicode to multibyte encoding by calling WC2MB,
allocating a temporary wxCharBuffer to hold the result.

The second overload of this function allows to convert a string of the given
length \arg{inLen}, whether it is \NUL-terminated or not (for wide character
strings, unlike for the multibyte ones, a single \NUL is always enough).
But notice that just as with \helpref{cMB2WC}{wxmbconvmb2wc}, it is more
efficient to pass an already terminated string to this function as otherwise a
copy is made internally.

If \arg{outLen} is not-\NULL, it receives the length of the converted
string.


\membersection{wxMBConv::cMB2WX}\label{wxmbconvcmb2wx}

\constfunc{const char*}{cMB2WX}{\param{const char* }{psz}}

\constfunc{const wxWCharBuffer}{cMB2WX}{\param{const char* }{psz}}

Converts from multibyte encoding to the current wxChar type
(which depends on whether wxUSE\_UNICODE is set to 1). If wxChar is char,
it returns the parameter unaltered. If wxChar is wchar\_t, it returns the
result in a wxWCharBuffer. The macro wxMB2WXbuf is defined as the correct
return type (without const).


\membersection{wxMBConv::cWX2MB}\label{wxmbconvcwx2mb}

\constfunc{const char*}{cWX2MB}{\param{const wxChar* }{psz}}

\constfunc{const wxCharBuffer}{cWX2MB}{\param{const wxChar* }{psz}}

Converts from the current wxChar type to multibyte encoding. If wxChar is char,
it returns the parameter unaltered. If wxChar is wchar\_t, it returns the
result in a wxCharBuffer. The macro wxWX2MBbuf is defined as the correct
return type (without const).


\membersection{wxMBConv::cWC2WX}\label{wxmbconvcwc2wx}

\constfunc{const wchar\_t*}{cWC2WX}{\param{const wchar\_t* }{psz}}

\constfunc{const wxCharBuffer}{cWC2WX}{\param{const wchar\_t* }{psz}}

Converts from Unicode to the current wxChar type. If wxChar is wchar\_t,
it returns the parameter unaltered. If wxChar is char, it returns the
result in a wxCharBuffer. The macro wxWC2WXbuf is defined as the correct
return type (without const).


\membersection{wxMBConv::cWX2WC}\label{wxmbconvcwx2wc}

\constfunc{const wchar\_t*}{cWX2WC}{\param{const wxChar* }{psz}}

\constfunc{const wxWCharBuffer}{cWX2WC}{\param{const wxChar* }{psz}}

Converts from the current wxChar type to Unicode. If wxChar is wchar\_t,
it returns the parameter unaltered. If wxChar is char, it returns the
result in a wxWCharBuffer. The macro wxWX2WCbuf is defined as the correct
return type (without const).


\membersection{wxMBConv::FromWChar}\label{wxmbconvfromwchar}

\constfunc{virtual size\_t}{FromWChar}{\param{wchar\_t *}{dst}, \param{size\_t }{dstLen}, \param{const char *}{src}, \param{size\_t }{srcLen = $-1$}}

The most general function for converting a multibyte string to a wide string.
The main case is when \arg{dst} is not \NULL and \arg{srcLen} is not $-1$: then
the function converts exactly \arg{srcLen} bytes starting at \arg{src} into
wide string which it output to \arg{dst}. If the length of the resulting wide
string is greater than \arg{dstLen}, an error is returned. Note that if 
\arg{srcLen} bytes don't include \NUL characters, the resulting wide string is
not \NUL-terminated neither.

If \arg{srcLen} is $-1$, the function supposes that the string is properly
(i.e. as necessary for the encoding handled by this conversion) \NUL-terminated
and converts the entire string, including any trailing \NUL bytes. In this case
the wide string is also \NUL-terminated.

Finally, if \arg{dst} is \NULL, the function returns the length of the needed
buffer.

\wxheading{Return value}

The number of characters written to \arg{dst} (or the number of characters
which would have been written to it if it were non-\NULL) on success or 
\texttt{wxCONV\_FAILED} on error.


\membersection{wxMBConv::GetMaxMBNulLen}\label{wxmbconvgetmaxmbnullen}

\func{const size\_t}{GetMaxMBNulLen}{\void}

Returns the maximal value which can be returned by 
\helpref{GetMBNulLen}{wxmbconvgetmbnullen} for any conversion object. Currently
this value is $4$.

This method can be used to allocate the buffer with enough space for the
trailing \NUL characters for any encoding.


\membersection{wxMBConv::GetMBNulLen}\label{wxmbconvgetmbnullen}

\constfunc{size\_t}{GetMBNulLen}{\void}

This function returns $1$ for most of the multibyte encodings in which the
string is terminated by a single \NUL, $2$ for UTF-16 and $4$ for UTF-32 for
which the string is terminated with $2$ and $4$ \NUL characters respectively.
The other cases are not currently supported and $-1$ is returned for them.


\membersection{wxMBConv::ToWChar}\label{wxmbconvtowchar}

\constfunc{virtual size\_t}{ToWChar}{\param{char\_t *}{dst}, \param{size\_t }{dstLen}, \param{const wchar\_t *}{src}, \param{size\_t }{srcLen = $-1$}}

This function has the same semantics as \helpref{FromWChar}{wxmbconvfromwchar} 
except that it converts a wide string to multibyte one.


