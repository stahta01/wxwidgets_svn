\section{\class{wxStatusBar}}\label{wxstatusbar}

A status bar is a narrow window that can be placed along the bottom of a frame to give
small amounts of status information. It can contain one or more fields, one or more of which can
be variable length according to the size of the window.

\helpref{wxWindow}{wxwindow}\\
\helpref{wxEvtHandler}{wxevthandler}\\
\helpref{wxObject}{wxobject}

\wxheading{Derived from}

\helpref{wxWindow}{wxwindow}\\
\helpref{wxEvtHandler}{wxevthandler}\\
\helpref{wxObject}{wxobject}

\wxheading{Include files}

<wx/statusbr.h>

\wxheading{Window styles}

\twocolwidtha{5cm}
\begin{twocollist}\itemsep=0pt
\twocolitem{\windowstyle{wxST\_SIZEGRIP}}{On Windows 95, displays a gripper at right-hand side of
the status bar.}
\end{twocollist}

See also \helpref{window styles overview}{windowstyles}.

\wxheading{Remarks}

It is possible to create controls and other windows on the status bar. Position these windows
from an {\bf OnSize} event handler.

\wxheading{See also}

\helpref{wxFrame}{wxframe}, \helpref{Status bar sample}{samplestatbar}

\latexignore{\rtfignore{\wxheading{Members}}}

\membersection{wxStatusBar::wxStatusBar}\label{wxstatusbarctor}

\func{}{wxStatusBar}{\void}

Default constructor.

\func{}{wxStatusBar}{\param{wxWindow* }{parent}, \param{wxWindowID }{id = wxID\_ANY},\rtfsp
\param{long}{ style = wxST\_SIZEGRIP},\rtfsp
\param{const wxString\& }{name = ``statusBar"}}

Constructor, creating the window.

\wxheading{Parameters}

\docparam{parent}{The window parent, usually a frame.}

\docparam{id}{The window identifier. It may take a value of -1 to indicate a default value.}

\docparam{style}{The window style. See \helpref{wxStatusBar}{wxstatusbar}.}

\docparam{name}{The name of the window. This parameter is used to associate a name with the item,
allowing the application user to set Motif resource values for
individual windows.}

\wxheading{See also}

\helpref{wxStatusBar::Create}{wxstatusbarcreate}

\membersection{wxStatusBar::\destruct{wxStatusBar}}\label{wxstatusbardtor}

\func{void}{\destruct{wxStatusBar}}{\void}

Destructor.

\membersection{wxStatusBar::Create}\label{wxstatusbarcreate}

\func{bool}{Create}{\param{wxWindow* }{parent}, \param{wxWindowID }{id = wxID\_ANY},\rtfsp
\param{long}{ style = wxST\_SIZEGRIP},\rtfsp
\param{const wxString\& }{name = ``statusBar"}}

Creates the window, for two-step construction.

See \helpref{wxStatusBar::wxStatusBar}{wxstatusbarctor} for details.

\membersection{wxStatusBar::GetFieldRect}\label{wxstatusbargetfieldrect}

\constfunc{virtual bool}{GetFieldRect}{\param{int}{ i}, \param{wxRect\&}{ rect}}

Returns the size and position of a field's internal bounding rectangle.

\wxheading{Parameters}

\docparam{i}{The field in question.}

\docparam{rect}{The rectangle values are placed in this variable.}

\wxheading{Return value}

true if the field index is valid, false otherwise.

\wxheading{See also}

\helpref{wxRect}{wxrect}

\perlnote{In wxPerl this function returns a {\tt Wx::Rect} if the field
  index is valid, {\tt undef} otherwise.}

\membersection{wxStatusBar::GetFieldsCount}\label{wxstatusbargetfieldscount}

\constfunc{int}{GetFieldsCount}{\void}

Returns the number of fields in the status bar.

\membersection{wxStatusBar::GetStatusText}\label{wxstatusbargetstatustext}

\constfunc{virtual wxString}{GetStatusText}{\param{int}{ i = 0}}

Returns the string associated with a status bar field.

\wxheading{Parameters}

\docparam{i}{The number of the status field to retrieve, starting from zero.}

\wxheading{Return value}

The status field string if the field is valid, otherwise the empty string.

\wxheading{See also}

\helpref{wxStatusBar::SetStatusText}{wxstatusbarsetstatustext}

\membersection{wxStatusBar::PopStatusText}\label{wxstatusbarpopstatustext}

\func{void}{PopStatusText}{\param{int}{ field = 0}}

Sets the field text to the top of the stack, and pops the stack of saved
strings.

\wxheading{See also}

\helpref{wxStatusBar::PushStatusText}{wxstatusbarpushstatustext}

\membersection{wxStatusBar::PushStatusText}\label{wxstatusbarpushstatustext}

\func{void}{PushStatusText}{\param{const wxString\&}{ string}, \param{int}{ field = 0}}

Saves the current field text in a per field stack, and sets the field text
to the string passed as argument.

\membersection{wxStatusBar::SetFieldsCount}\label{wxstatusbarsetfieldscount}

\func{virtual void}{SetFieldsCount}{\param{int}{ number = 1}, \param{int* }{widths = NULL}}

Sets the number of fields, and optionally the field widths.

\pythonnote{Only the first parameter is accepted.  Use SetStatusWidths
to set the widths of the fields.}

\perlnote{In wxPerl this function accepts only the {\tt n} parameter.
Use SetStatusWidths to set the field widths.}

\wxheading{Parameters}

\docparam{number}{The number of fields.}

\docparam{widths}{An array of {\it n} integers interpreted in the same way as
in \helpref{SetStatusWidths}{wxstatusbarsetstatuswidths}}

\membersection{wxStatusBar::SetMinHeight}\label{wxstatusbarsetminheight}

\func{void}{SetMinHeight}{\param{int}{ height}}

Sets the minimal possible height for the status bar. The real height may be
bigger than the height specified here depending on the size of the font used by
the status bar.

\membersection{wxStatusBar::SetStatusText}\label{wxstatusbarsetstatustext}

\func{virtual void}{SetStatusText}{\param{const wxString\& }{text}, \param{int}{ i = 0}}

Sets the text for one field.

\wxheading{Parameters}

\docparam{text}{The text to be set. Use an empty string (``") to clear the field.}

\docparam{i}{The field to set, starting from zero.}

\wxheading{See also}

\helpref{wxStatusBar::GetStatusText}{wxstatusbargetstatustext}, \helpref{wxFrame::SetStatusText}{wxframesetstatustext}

\membersection{wxStatusBar::SetStatusWidths}\label{wxstatusbarsetstatuswidths}

\func{virtual void}{SetStatusWidths}{\param{int}{ n}, \param{int *}{widths}}

Sets the widths of the fields in the status line. There are two types of
fields: fixed widths one and variable width fields. For the fixed width fields
you should specify their (constant) width in pixels. For the variable width
fields, specify a negative number which indicates how the field should expand:
the space left for all variable width fields is divided between them according
to the absolute value of this number. A variable width field with width of $-2$
gets twice as much of it as a field with width $-1$ and so on.

For example, to create one fixed width field of width $100$ in the right part of
the status bar and two more fields which get $66$\% and $33$\% of the remaining
space correspondingly, you should use an array containing $-2$, $-1$ and $100$.

\wxheading{Parameters}

\docparam{n}{The number of fields in the status bar. Must be equal to the
number passed to \helpref{SetFieldsCount}{wxstatusbarsetfieldscount} the last
time it was called.}

\docparam{widths}{Contains an array of {\it n} integers, each of which is
either an absolute status field width in pixels if positive or indicates a
variable width field if negative.}

\wxheading{Remarks}

The widths of the variable fields are calculated from the total width of all fields,
minus the sum of widths of the non-variable fields, divided by the number of
variable fields.

\wxheading{See also}

\helpref{wxStatusBar::SetFieldsCount}{wxstatusbarsetfieldscount}, \helpref{wxFrame::SetStatusWidths}{wxframesetstatuswidths}

\pythonnote{Only a single parameter is required, a Python list of
integers.}

\perlnote{In wxPerl this method takes as parameters the field widths.}

\membersection{wxStatusBar::SetStatusStyles}\label{wxstatusbarsetstatusstyles}

\func{virtual void}{SetStatusStyles}{\param{int}{ n}, \param{int *}{styles}}

Sets the styles of the fields in the status line which can make fields appear flat
or raised instead of the standard sunken 3D border.

\wxheading{Parameters}

\docparam{n}{The number of fields in the status bar. Must be equal to the
number passed to \helpref{SetFieldsCount}{wxstatusbarsetfieldscount} the last
time it was called.}

\docparam{styles}{Contains an array of {\it n} integers with the styles for each field. There 
are three possible styles:

\twocolwidtha{5cm}
\begin{twocollist}\itemsep=0pt
\twocolitem{\windowstyle{wxSB\_NORMAL}}{(default) The field appears sunken with a standard 3D border.}
\twocolitem{\windowstyle{wxSB\_FLAT}}{No border is painted around the field so that it appears flat.}
\twocolitem{\windowstyle{wxSB\_RAISED}}{A raised 3D border is painted around the field.}
\end{twocollist}
}


