\section{\class{wxCloseEvent}}\label{wxcloseevent}

This event class contains information about window and session close events.

The handler function for EVT\_CLOSE is called when the user has tried to close a a frame
or dialog box using the window manager (X) or system menu (Windows). It can
also be invoked by the application itself programmatically, for example by
calling the \helpref{wxWindow::Close}{wxwindowclose} function.

You should check whether the application is forcing the deletion of the window
using \helpref{wxCloseEvent::CanVeto}{wxcloseeventcanveto}. If this is {\tt false},
you {\it must} destroy the window using \helpref{wxWindow::Destroy}{wxwindowdestroy}.
If the return value is true, it is up to you whether you respond by destroying the window.

If you don't destroy the window, you should call \helpref{wxCloseEvent::Veto}{wxcloseeventveto} to
let the calling code know that you did not destroy the window. This allows the \helpref{wxWindow::Close}{wxwindowclose} function
to return {\tt true} or {\tt false} depending on whether the close instruction was honoured or not.

\wxheading{Derived from}

\helpref{wxEvent}{wxevent}

\wxheading{Include files}

<wx/event.h>

\wxheading{Event table macros}

To process a close event, use these event handler macros to direct input to member
functions that take a wxCloseEvent argument.

\twocolwidtha{7cm}
\begin{twocollist}\itemsep=0pt
\twocolitem{{\bf EVT\_CLOSE(func)}}{Process a close event, supplying the member function. This
event applies to wxFrame and wxDialog classes.}
\twocolitem{{\bf EVT\_QUERY\_END\_SESSION(func)}}{Process a query end session event, supplying the member function.
This event applies to wxApp only.}
\twocolitem{{\bf EVT\_END\_SESSION(func)}}{Process an end session event, supplying the member function.
This event applies to wxApp only.}
\end{twocollist}%

\wxheading{See also}

\helpref{wxWindow::Close}{wxwindowclose},\rtfsp
%% GD: OnXXX functions are not documented
%%\helpref{wxApp::OnEndSession}{wxapponendsession},\rtfsp
\helpref{Window deletion overview}{windowdeletionoverview}

\latexignore{\rtfignore{\wxheading{Members}}}


\membersection{wxCloseEvent::wxCloseEvent}\label{wxcloseeventctor}

\func{}{wxCloseEvent}{\param{WXTYPE}{ commandEventType = 0}, \param{int}{ id = 0}}

Constructor.


\membersection{wxCloseEvent::CanVeto}\label{wxcloseeventcanveto}

\func{bool}{CanVeto}{\void}

Returns true if you can veto a system shutdown or a window close event.
Vetoing a window close event is not possible if the calling code wishes to
force the application to exit, and so this function must be called to check this.


\membersection{wxCloseEvent::GetLoggingOff}\label{wxcloseeventgetloggingoff}

\constfunc{bool}{GetLoggingOff}{\void}

Returns true if the user is just logging off or false if the system is
shutting down. This method can only be called for end session and query end
session events, it doesn't make sense for close window event.


\membersection{wxCloseEvent::SetCanVeto}\label{wxcloseeventsetcanveto}

\func{void}{SetCanVeto}{\param{bool}{ canVeto}}

Sets the 'can veto' flag.


\membersection{wxCloseEvent::SetForce}\label{wxcloseeventsetforce}

\constfunc{void}{SetForce}{\param{bool}{ force}}

Sets the 'force' flag.


\membersection{wxCloseEvent::SetLoggingOff}\label{wxcloseeventsetloggingoff}

\constfunc{void}{SetLoggingOff}{\param{bool}{ loggingOff}}

Sets the 'logging off' flag.


\membersection{wxCloseEvent::Veto}\label{wxcloseeventveto}

\func{void}{Veto}{\param{bool}{ veto = true}}

Call this from your event handler to veto a system shutdown or to signal
to the calling application that a window close did not happen.

You can only veto a shutdown if \helpref{wxCloseEvent::CanVeto}{wxcloseeventcanveto} returns
true.


