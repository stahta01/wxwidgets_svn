%
% automatically generated by HelpGen from
% fontmap.h at 10/Mar/00 23:54:16
%

\section{\class{wxFontMapper}}\label{wxfontmapper}

wxFontMapper manages user-definable correspondence between logical font
names and the fonts present on the machine.

The default implementations of all functions will ask the user if they are
not capable of finding the answer themselves and store the answer in a
config file (configurable via SetConfigXXX functions). This behaviour may
be disabled by giving the value of FALSE to "interactive" parameter.

However, the functions will always consult the config file to allow the
user-defined values override the default logic and there is no way to
disable this - which shouldn't be ever needed because if "interactive" was
never TRUE, the config file is never created anyhow.

In case everything else fails (i.e. there is no record in config file
and "interactive" is FALSE or user denied to choose any replacement), 
the class queries \helpref{wxEncodingConverter}{wxencodingconverter} 
for "equivalent" encodings (e.g. iso8859-2 and cp1250) and tries them.

\wxheading{Global variables}

{\tt wxFontMapper *wxTheFontMapper} is defined.

\wxheading{Using wxFontMapper in conjunction with wxEncodingConverter}

If you need to display text in encoding which is not available at
host system (see \helpref{IsEncodingAvailable}{wxfontmapperisencodingavailable}),
you may use these two classes to a) find font in some similar encoding
(see \helpref{GetAltForEncoding}{wxfontmappergetaltforencoding})
and
b) convert the text to this encoding 
(\helpref{wxEncodingConverter::Convert}{wxencodingconverterconvert}).

Following code snippet demonstrates it:

\begin{verbatim}
if (!wxTheFontMapper->IsEncodingAvailable(enc, facename))
{
   wxFontEncoding alternative;
   if (wxTheFontMapper->GetAltForEncoding(enc, &alternative,
                                          facename, FALSE))
   {
       wxEncodingConverter encconv;
       if (!encconv.Init(enc, alternative))
           ...failure...
       else
           text = encconv.Convert(text);
   }
   else
       ...failure (or we may try iso8859-1/7bit ASCII)...
}
...display text...
\end{verbatim}


\wxheading{Derived from}

No base class

\wxheading{Include files}

<wx/fontmap.h>

\wxheading{See also}

\helpref{wxEncodingConverter}{wxencodingconverter}, 
\helpref{Writing non-English applications}{nonenglishoverview}

\latexignore{\rtfignore{\wxheading{Members}}}

\membersection{wxFontMapper::wxFontMapper}\label{wxfontmapperwxfontmapper}

\func{}{wxFontMapper}{\void}

Default ctor.

\membersection{wxFontMapper::\destruct{wxFontMapper}}\label{wxfontmapperdtor}

\func{}{\destruct{wxFontMapper}}{\void}

Virtual dtor for a base class.

\membersection{wxFontMapper::GetAltForEncoding}\label{wxfontmappergetaltforencoding}

\func{bool}{GetAltForEncoding}{\param{wxFontEncoding }{encoding}, \param{wxNativeEncodingInfo* }{info}, \param{const wxString\& }{facename = wxEmptyString}, \param{bool }{interactive = TRUE}}

\func{bool}{GetAltForEncoding}{\param{wxFontEncoding }{encoding}, \param{wxFontEncoding* }{alt\_encoding}, \param{const wxString\& }{facename = wxEmptyString}, \param{bool }{interactive = TRUE}}

Find an alternative for the given encoding (which is supposed to not be
available on this system). If successful, return TRUE and fill info
structure with the parameters required to create the font, otherwise
return FALSE.

The first form is for wxWindows' internal use while the second one
is better suitable for general use -- it returns wxFontEncoding which
can consequently be passed to wxFont constructor.

\membersection{wxFontMapper::IsEncodingAvailable}\label{wxfontmapperisencodingavailable}

\func{bool}{IsEncodingAvailable}{\param{wxFontEncoding }{encoding}, \param{const wxString\& }{facename = wxEmptyString}}

Check whether given encoding is available in given face or not.
If no facename is given, find {\it any} font in this encoding.

\membersection{wxFontMapper::CharsetToEncoding}\label{wxfontmappercharsettoencoding}

\func{wxFontEncoding}{CharsetToEncoding}{\param{const wxString\& }{charset}, \param{bool }{interactive = TRUE}}

Returns the encoding for the given charset (in the form of RFC 2046) or
wxFONTENCODING\_SYSTEM if couldn't decode it.

\membersection{wxFontMapper::GetEncodingName}\label{wxfontmappergetencodingname}

\func{static wxString}{GetEncodingName}{\param{wxFontEncoding }{encoding}}

Return internal string identifier for the encoding (see also 
\helpref{GetEncodingDescription()}{wxfontmappergetencodingdescription})

\membersection{wxFontMapper::GetEncodingDescription}\label{wxfontmappergetencodingdescription}

\func{static wxString}{GetEncodingDescription}{\param{wxFontEncoding }{encoding}}

Return user-readable string describing the given encoding.

\membersection{wxFontMapper::SetDialogParent}\label{wxfontmappersetdialogparent}

\func{void}{SetDialogParent}{\param{wxWindow* }{parent}}

The parent window for modal dialogs.

\membersection{wxFontMapper::SetDialogTitle}\label{wxfontmappersetdialogtitle}

\func{void}{SetDialogTitle}{\param{const wxString\& }{title}}

The title for the dialogs (note that default is quite reasonable).

\membersection{wxFontMapper::SetConfig}\label{wxfontmappersetconfig}

\func{void}{SetConfig}{\param{wxConfigBase* }{config}}

Set the config object to use (may be NULL to use default).

By default, the global one (from wxConfigBase::Get() will be used) 
and the default root path for the config settings is the string returned by
GetDefaultConfigPath().

\membersection{wxFontMapper::SetConfigPath}\label{wxfontmappersetconfigpath}

\func{void}{SetConfigPath}{\param{const wxString\& }{prefix}}

Set the root config path to use (should be an absolute path).

