% ----------------------------------------------------------------------------
% CLASS: wxIPV4address
% ----------------------------------------------------------------------------
\section{\class{wxIPV4address}}\label{wxipv4address}

\wxheading{Derived from}

\helpref{wxSockAddress}{wxsockaddress}

\wxheading{Include files}

<wx/socket.h>

% ----------------------------------------------------------------------------
% MEMBERS
% ----------------------------------------------------------------------------
\latexignore{\rtfignore{\wxheading{Members}}}

%
% Hostname
%

\membersection{wxIPV4address::Hostname}

\func{bool}{Hostname}{\param{const wxString\&}{ hostname}}

Set the address to {\it hostname}, which can be a host name
or an IP-style address in dot notation (a.b.c.d)

\wxheading{Return value}

Returns TRUE on success, FALSE if something goes wrong
(invalid hostname or invalid IP address).

%
% Hostname
%

\membersection{wxIPV4address::Hostname}

\func{wxString}{Hostname}{\void}

Returns the hostname which matches the IP address.

%
% Service
%

\membersection{wxIPV4address::Service}

\func{bool}{Service}{\param{const wxString\&}{ service}}

Set the port to that corresponding to the specified {\it service}.

\wxheading{Return value}

Returns TRUE on success, FALSE if something goes wrong
(invalid service).

%
% Service
%

\membersection{wxIPV4address::Service}

\func{bool}{Service}{\param{unsigned short}{ service}}

Set the port to that corresponding to the specified {\it service}.

\wxheading{Return value}

Returns TRUE on success, FALSE if something goes wrong
(invalid service).

%
% Service
%

\membersection{wxIPV4address::Service}

\func{unsigned short}{Service}{\void}

Returns the current service.

%
% AnyAddress
%

\membersection{wxIPV4address::AnyAddress}\label{wxipv4addressanyaddress}

\func{bool}{AnyAddress}{\void}

Set address to any of the addresses of the current machine. Whenever
possible, use this function instead of \helpref{wxIPV4address::LocalHost}{wxipv4addresslocalhost},
as this correctly handles multi-homed hosts and avoids other small
problems. Internally, this is the same as setting the IP address
to {\bf INADDR\_ANY}.

\wxheading{Return value}

Returns TRUE on success, FALSE if something went wrong.

%
% LocalHost
%

\membersection{wxIPV4address::LocalHost}\label{wxipv4addresslocalhost}

\func{bool}{LocalHost}{\void}

Set address to localhost (127.0.0.1). Whenever possible, use the 
\helpref{wxIPV4address::AnyAddress}{wxipv4addressanyaddress},
function instead of this one, as this will correctly handle multi-homed
hosts and avoid other small problems.

\wxheading{Return value}

Returns TRUE on success, FALSE if something went wrong.

