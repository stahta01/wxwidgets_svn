\section{\class{wxBrush}}\label{wxbrush}

A brush is a drawing tool for filling in areas. It is used for painting
the background of rectangles, ellipses, etc. It has a colour and a
style.

\wxheading{Derived from}

\helpref{wxGDIObject}{wxgdiobject}\\
\helpref{wxObject}{wxobject}

\wxheading{Include files}

<wx/brush.h>

\wxheading{Predefined objects}

Objects:

{\bf wxNullBrush}

Pointers:

{\bf wxBLUE\_BRUSH\\
wxGREEN\_BRUSH\\
wxWHITE\_BRUSH\\
wxBLACK\_BRUSH\\
wxGREY\_BRUSH\\
wxMEDIUM\_GREY\_BRUSH\\
wxLIGHT\_GREY\_BRUSH\\
wxTRANSPARENT\_BRUSH\\
wxCYAN\_BRUSH\\
wxRED\_BRUSH}

\wxheading{Remarks}

On a monochrome display, wxWidgets shows
all brushes as white unless the colour is really black.

Do not initialize objects on the stack before the program commences,
since other required structures may not have been set up yet. Instead,
define global pointers to objects and create them in \helpref{wxApp::OnInit}{wxapponinit} or
when required.

An application may wish to create brushes with different
characteristics dynamically, and there is the consequent danger that a
large number of duplicate brushes will be created. Therefore an
application may wish to get a pointer to a brush by using the global
list of brushes {\bf wxTheBrushList}, and calling the member function
\rtfsp{\bf FindOrCreateBrush}.

wxBrush uses a reference counting system, so assignments between brushes are very
cheap. You can therefore use actual wxBrush objects instead of pointers without
efficiency problems. Once one wxBrush object changes its data it will create its
own brush data internally so that other brushes, which previously shared the
data using the reference counting, are not affected.

%TODO: an overview for wxBrush.
\wxheading{See also}

\helpref{wxBrushList}{wxbrushlist}, \helpref{wxDC}{wxdc}, \helpref{wxDC::SetBrush}{wxdcsetbrush}

\latexignore{\rtfignore{\wxheading{Members}}}


\membersection{wxBrush::wxBrush}\label{wxbrushctor}

\func{}{wxBrush}{\void}

Default constructor. The brush will be uninitialised, and \helpref{wxBrush::Ok}{wxbrushok} will
return false.

\func{}{wxBrush}{\param{const wxColour\&}{ colour}, \param{int}{ style = {\tt wxSOLID}}}

Constructs a brush from a colour object and style.

\func{}{wxBrush}{\param{const wxString\& }{colourName}, \param{int}{ style}}

Constructs a brush from a colour name and style.

\func{}{wxBrush}{\param{const wxBitmap\& }{stippleBitmap}}

Constructs a stippled brush using a bitmap.

\func{}{wxBrush}{\param{const wxBrush\&}{ brush}}

Copy constructor. This uses reference counting so is a cheap operation.

\wxheading{Parameters}

\docparam{colour}{Colour object.}

\docparam{colourName}{Colour name. The name will be looked up in the colour database.}

\docparam{style}{One of:

\begin{twocollist}\itemsep=0pt
\twocolitem{{\bf wxTRANSPARENT}}{Transparent (no fill).}
\twocolitem{{\bf wxSOLID}}{Solid.}
\twocolitem{{\bf wxSTIPPLE}}{Uses a bitmap as a stipple.}
\twocolitem{{\bf wxBDIAGONAL\_HATCH}}{Backward diagonal hatch.}
\twocolitem{{\bf wxCROSSDIAG\_HATCH}}{Cross-diagonal hatch.}
\twocolitem{{\bf wxFDIAGONAL\_HATCH}}{Forward diagonal hatch.}
\twocolitem{{\bf wxCROSS\_HATCH}}{Cross hatch.}
\twocolitem{{\bf wxHORIZONTAL\_HATCH}}{Horizontal hatch.}
\twocolitem{{\bf wxVERTICAL\_HATCH}}{Vertical hatch.}
\end{twocollist}}

\docparam{brush}{Pointer or reference to a brush to copy.}

\docparam{stippleBitmap}{A bitmap to use for stippling.}

\wxheading{Remarks}

If a stipple brush is created, the brush style will be set to wxSTIPPLE.

\wxheading{See also}

\helpref{wxBrushList}{wxbrushlist}, \helpref{wxColour}{wxcolour}, \helpref{wxColourDatabase}{wxcolourdatabase}


\membersection{wxBrush::\destruct{wxBrush}}\label{wxbrushdtor}

\func{}{\destruct{wxBrush}}{\void}

Destructor.

\wxheading{Remarks}

The destructor may not delete the underlying brush object of the native windowing
system, since wxBrush uses a reference counting system for efficiency.

Although all remaining brushes are deleted when the application exits,
the application should try to clean up all brushes itself. This is because
wxWidgets cannot know if a pointer to the brush object is stored in an
application data structure, and there is a risk of double deletion.


\membersection{wxBrush::GetColour}\label{wxbrushgetcolour}

\constfunc{wxColour\&}{GetColour}{\void}

Returns a reference to the brush colour.

\wxheading{See also}

\helpref{wxBrush::SetColour}{wxbrushsetcolour}


\membersection{wxBrush::GetStipple}\label{wxbrushgetstipple}

\constfunc{wxBitmap *}{GetStipple}{\void}

Gets a pointer to the stipple bitmap. If the brush does not have a wxSTIPPLE style,
this bitmap may be non-NULL but uninitialised (\helpref{wxBitmap::Ok}{wxbitmapok} returns false).

\wxheading{See also}

\helpref{wxBrush::SetStipple}{wxbrushsetstipple}


\membersection{wxBrush::GetStyle}\label{wxbrushgetstyle}

\constfunc{int}{GetStyle}{\void}

Returns the brush style, one of:

\begin{twocollist}\itemsep=0pt
\twocolitem{{\bf wxTRANSPARENT}}{Transparent (no fill).}
\twocolitem{{\bf wxSOLID}}{Solid.}
\twocolitem{{\bf wxBDIAGONAL\_HATCH}}{Backward diagonal hatch.}
\twocolitem{{\bf wxCROSSDIAG\_HATCH}}{Cross-diagonal hatch.}
\twocolitem{{\bf wxFDIAGONAL\_HATCH}}{Forward diagonal hatch.}
\twocolitem{{\bf wxCROSS\_HATCH}}{Cross hatch.}
\twocolitem{{\bf wxHORIZONTAL\_HATCH}}{Horizontal hatch.}
\twocolitem{{\bf wxVERTICAL\_HATCH}}{Vertical hatch.}
\twocolitem{{\bf wxSTIPPLE}}{Stippled using a bitmap.}
\twocolitem{{\bf wxSTIPPLE\_MASK\_OPAQUE}}{Stippled using a bitmap's mask.}
\end{twocollist}

\wxheading{See also}

\helpref{wxBrush::SetStyle}{wxbrushsetstyle}, \helpref{wxBrush::SetColour}{wxbrushsetcolour},\rtfsp
\helpref{wxBrush::SetStipple}{wxbrushsetstipple}


\membersection{wxBrush::IsHatch}\label{wxbrushishatch}

\constfunc{bool}{IsHatch}{\void}

Returns true if the style of the brush is any of hatched fills.

\wxheading{See also}

\helpref{wxBrush::GetStyle}{wxbrushgetstyle}


\membersection{wxBrush::Ok}\label{wxbrushok}

\constfunc{bool}{Ok}{\void}

Returns true if the brush is initialised. It will return false if the default
constructor has been used (for example, the brush is a member of a class, or
NULL has been assigned to it).


\membersection{wxBrush::SetColour}\label{wxbrushsetcolour}

\func{void}{SetColour}{\param{wxColour\& }{colour}}

Sets the brush colour using a reference to a colour object.

\func{void}{SetColour}{\param{const wxString\& }{colourName}}

Sets the brush colour using a colour name from the colour database.

\func{void}{SetColour}{\param{const unsigned char}{ red}, \param{const unsigned char}{ green}, \param{const unsigned char}{ blue}}

Sets the brush colour using red, green and blue values.

\wxheading{See also}

\helpref{wxBrush::GetColour}{wxbrushgetcolour}


\membersection{wxBrush::SetStipple}\label{wxbrushsetstipple}

\func{void}{SetStipple}{\param{const wxBitmap\&}{ bitmap}}

Sets the stipple bitmap.

\wxheading{Parameters}

\docparam{bitmap}{The bitmap to use for stippling.}

\wxheading{Remarks}

The style will be set to wxSTIPPLE, unless the bitmap has a mask associated
to it, in which case the style will be set to wxSTIPPLE\_MASK\_OPAQUE.

If the wxSTIPPLE variant is used, the bitmap will be used to fill out the
area to be drawn. If the wxSTIPPLE\_MASK\_OPAQUE is used, the current
text foreground and text background determine what colours are used for
displaying and the bits in the mask (which is a mono-bitmap actually)
determine where to draw what.

Note that under Windows 95, only 8x8 pixel large stipple bitmaps are
supported, Windows 98 and NT as well as GTK support arbitrary bitmaps.

\wxheading{See also}

\helpref{wxBitmap}{wxbitmap}


\membersection{wxBrush::SetStyle}\label{wxbrushsetstyle}

\func{void}{SetStyle}{\param{int}{ style}}

Sets the brush style.

\docparam{style}{One of:

\begin{twocollist}\itemsep=0pt
\twocolitem{{\bf wxTRANSPARENT}}{Transparent (no fill).}
\twocolitem{{\bf wxSOLID}}{Solid.}
\twocolitem{{\bf wxBDIAGONAL\_HATCH}}{Backward diagonal hatch.}
\twocolitem{{\bf wxCROSSDIAG\_HATCH}}{Cross-diagonal hatch.}
\twocolitem{{\bf wxFDIAGONAL\_HATCH}}{Forward diagonal hatch.}
\twocolitem{{\bf wxCROSS\_HATCH}}{Cross hatch.}
\twocolitem{{\bf wxHORIZONTAL\_HATCH}}{Horizontal hatch.}
\twocolitem{{\bf wxVERTICAL\_HATCH}}{Vertical hatch.}
\twocolitem{{\bf wxSTIPPLE}}{Stippled using a bitmap.}
\twocolitem{{\bf wxSTIPPLE\_MASK\_OPAQUE}}{Stippled using a bitmap's mask.}
\end{twocollist}}

\wxheading{See also}

\helpref{wxBrush::GetStyle}{wxbrushgetstyle}


\membersection{wxBrush::operator $=$}\label{wxbrushassignment}

\func{wxBrush\&}{operator $=$}{\param{const wxBrush\& }{brush}}

Assignment operator, using reference counting. Returns a reference
to `this'.


\membersection{wxBrush::operator $==$}\label{wxbrushequals}

\func{bool}{operator $==$}{\param{const wxBrush\& }{brush}}

Equality operator. Two brushes are equal if they contain pointers
to the same underlying brush data. It does not compare each attribute,
so two independently-created brushes using the same parameters will
fail the test.


\membersection{wxBrush::operator $!=$}\label{wxbrushnotequals}

\func{bool}{operator $!=$}{\param{const wxBrush\& }{brush}}

Inequality operator. Two brushes are not equal if they contain pointers
to different underlying brush data. It does not compare each attribute.

\section{\class{wxBrushList}}\label{wxbrushlist}

A brush list is a list containing all brushes which have been created.

\wxheading{Derived from}

\helpref{wxList}{wxlist}\\
\helpref{wxObject}{wxobject}

\wxheading{Include files}

<wx/gdicmn.h>

\wxheading{Remarks}

There is only one instance of this class: {\bf wxTheBrushList}.  Use
this object to search for a previously created brush of the desired
type and create it if not already found. In some windowing systems,
the brush may be a scarce resource, so it can pay to reuse old
resources if possible. When an application finishes, all brushes will
be deleted and their resources freed, eliminating the possibility of
`memory leaks'. However, it is best not to rely on this automatic
cleanup because it can lead to double deletion in some circumstances.

There are two mechanisms in recent versions of wxWidgets which make the
brush list less useful than it once was. Under Windows, scarce resources
are cleaned up internally if they are not being used. Also, a reference
counting mechanism applied to all GDI objects means that some sharing
of underlying resources is possible. You don't have to keep track of pointers,
working out when it is safe delete a brush, because the reference counting does
it for you. For example, you can set a brush in a device context, and then
immediately delete the brush you passed, because the brush is `copied'.

So you may find it easier to ignore the brush list, and instead create
and copy brushes as you see fit. If your Windows resource meter suggests
your application is using too many resources, you can resort to using
GDI lists to share objects explicitly.

The only compelling use for the brush list is for wxWidgets to keep
track of brushes in order to clean them up on exit. It is also kept for
backward compatibility with earlier versions of wxWidgets.

\wxheading{See also}

\helpref{wxBrush}{wxbrush}

\latexignore{\rtfignore{\wxheading{Members}}}


\membersection{wxBrushList::wxBrushList}\label{wxbrushlistconstr}

\func{void}{wxBrushList}{\void}

Constructor. The application should not construct its own brush list:
use the object pointer {\bf wxTheBrushList}.


\membersection{wxBrushList::AddBrush}\label{wxbrushlistaddbrush}

\func{void}{AddBrush}{\param{wxBrush *}{brush}}

Used internally by wxWidgets to add a brush to the list.


\membersection{wxBrushList::FindOrCreateBrush}\label{wxbrushlistfindorcreatebrush}

\func{wxBrush *}{FindOrCreateBrush}{\param{const wxColour\& }{colour}, \param{int}{ style = wxSOLID}}

Finds a brush with the specified attributes and returns it, else creates a new brush, adds it
to the brush list, and returns it.

\wxheading{Parameters}

\docparam{colour}{Colour object.}

\docparam{style}{Brush style. See \helpref{wxBrush::SetStyle}{wxbrushsetstyle} for a list of styles.}


\membersection{wxBrushList::RemoveBrush}\label{wxbrushlistremovebrush}

\func{void}{RemoveBrush}{\param{wxBrush *}{brush}}

Used by wxWidgets to remove a brush from the list.


