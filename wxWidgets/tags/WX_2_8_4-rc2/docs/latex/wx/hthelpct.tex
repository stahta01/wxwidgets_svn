%
% automatically generated by HelpGen from
% htmlhelp.h at 02/May/99 19:58:53
%

\section{\class{wxHtmlHelpController}}\label{wxhtmlhelpcontroller}

This help controller provides an easy way of displaying HTML help in your
application (see {\it test} sample). The help system is based on {\bf books} 
(see \helpref{AddBook}{wxhtmlhelpcontrolleraddbook}). A book is a logical
section of documentation (for example "User's Guide" or "Programmer's Guide" or
"C++ Reference" or "wxWidgets Reference"). The help controller can handle as
many books as you want.

Although this class has an API compatible with other wxWidgets
help controllers as documented by \helpref{wxHelpController}{wxhelpcontroller}, it
is recommended that you use the enhanced capabilities of wxHtmlHelpController's API.

wxHTML uses Microsoft's HTML Help Workshop project files (.hhp, .hhk, .hhc) as its
native format. The file format is described \helpref{here}{helpformat}.
Have a look at docs/html/ directory where sample project files are stored.

You can use Tex2RTF to produce these files when generating HTML, if you set {\bf htmlWorkshopFiles} to {\bf true} in
your tex2rtf.ini file. The commercial tool HelpBlocks (www.helpblocks.com) can also create these files.

\wxheading{Note}

It is strongly recommended to use preprocessed {\bf .hhp.cached} version of
projects. It can be either created on-the-fly (see 
\helpref{SetTempDir}{wxhtmlhelpcontrollersettempdir}) or you can use 
{\bf hhp2cached} utility from {\it utils/hhp2cached} to create it and
distribute the cached version together with helpfiles. See {\it samples/html/help} 
sample for demonstration of its use.

\wxheading{See also}

\helpref{Information about wxBestHelpController}{wxhelpcontroller}, 
\helpref{wxHtmlHelpFrame}{wxhtmlhelpframe}, 
\helpref{wxHtmlHelpDialog}{wxhtmlhelpdialog}, 
\helpref{wxHtmlHelpWindow}{wxhtmlhelpwindow},
\helpref{wxHtmlModalHelp}{wxhtmlmodalhelp}

\wxheading{Derived from}

wxHelpControllerBase

\wxheading{Include files}

<wx/html/helpctrl.h>

\latexignore{\rtfignore{\wxheading{Members}}}

\membersection{wxHtmlHelpController::wxHtmlHelpController}\label{wxhtmlhelpcontrollerwxhtmlhelpcontroller}

\func{}{wxHtmlHelpController}{\param{int }{style = wxHF\_DEFAULT\_STYLE}, \param{wxWindow* }{parentWindow = NULL}}

Constructor.

\wxheading{Parameters}

{\it style} is a combination of these flags:

\begin{twocollist}\itemsep=0pt
\twocolitem{\windowstyle{wxHF\_TOOLBAR}}{The help window has a toolbar.}
\twocolitem{\windowstyle{wxHF\_FLAT\_TOOLBAR}}{The help window has a toolbar with flat buttons (aka coolbar).}
\twocolitem{\windowstyle{wxHF\_CONTENTS}}{The help window has a contents panel.}
\twocolitem{\windowstyle{wxHF\_INDEX}}{The help window has an index panel.}
\twocolitem{\windowstyle{wxHF\_SEARCH}}{The help window has a search panel.}
\twocolitem{\windowstyle{wxHF\_BOOKMARKS}}{The help window has bookmarks controls.}
\twocolitem{\windowstyle{wxHF\_OPEN\_FILES}}{Allows user to open arbitrary HTML document.}
\twocolitem{\windowstyle{wxHF\_PRINT}}{The toolbar contains "print" button.}
\twocolitem{\windowstyle{wxHF\_MERGE\_BOOKS}}{The contents pane does not show
book nodes. All books are merged together and appear as single book to the
user.}
\twocolitem{\windowstyle{wxHF\_ICONS\_BOOK}}{All nodes in contents pane
have a book icon. This is how Microsoft's HTML help viewer behaves.}
\twocolitem{\windowstyle{wxHF\_ICONS\_FOLDER}}{Book nodes in contents pane have
a book icon, book's sections have a folder icon. This is the default.}
\twocolitem{\windowstyle{wxHF\_ICONS\_BOOK\_CHAPTER}}{Both book nodes and
nodes of top-level sections of a book (i.e. chapters) have a book icon,
all other sections (sections, subsections, ...) have a folder icon.}
\twocolitem{\windowstyle{wxHF\_EMBEDDED}}{Specifies that the help controller controls an embedded window of class \helpref{wxHtmlHelpWindow}{wxhtmlhelpwindow} that
should not be destroyed when the controller is destroyed.}
\twocolitem{\windowstyle{wxHF\_DIALOG}}{Specifies that the help controller should create a dialog containing the help window.}
\twocolitem{\windowstyle{wxHF\_FRAME}}{Specifies that the help controller should create a frame containing the help window. This is the default if neither wxHF\_DIALOG nor wxHF\_EMBEDDED is specified.}
\twocolitem{\windowstyle{wxHF\_MODAL}}{Specifies that the help controller should create a modal dialog containing the help window (used with the wxHF\_DIALOG style).}
\twocolitem{\windowstyle{wxHF\_DEFAULT\_STYLE}}{{\tt wxHF\_TOOLBAR | wxHF\_CONTENTS
| wxHF\_INDEX | wxHF\_SEARCH | wxHF\_BOOKMARKS | wxHF\_PRINT}}
\end{twocollist}

{\it parentWindow} is an optional window to be used as the parent for the help window.

\membersection{wxHtmlHelpController::AddBook}\label{wxhtmlhelpcontrolleraddbook}

\func{bool}{AddBook}{\param{const wxFileName\& }{bookFile}, \param{bool }{showWaitMsg}}

\func{bool}{AddBook}{\param{const wxString\& }{bookUrl}, \param{bool }{showWaitMsg}}

Adds book (\helpref{.hhp file}{helpformat} - HTML Help Workshop project file) into the list of loaded books.
This must be called at least once before displaying  any help.

{\it bookFile} or {\it bookUrl}  may be either .hhp file or ZIP archive
that contains arbitrary number of .hhp files in 
top-level directory. This ZIP archive must have .zip or .htb extension
(the latter stands for "HTML book"). In other words, {\tt AddBook(wxFileName("help.zip"))} 
is possible and is the recommended way.

\wxheading{Parameters}

\docparam{showWaitMsg}{If true then a decoration-less window with progress message is displayed.}
\docparam{bookFile}{Help book filename. It is recommended to use this prototype
instead of the one taking URL, because it is less error-prone.}
\docparam{bookUrl}{Help book URL (note that syntax of filename and URL is 
different on most platforms)}

\wxheading{Note}

Don't forget to install the archive wxFileSystem handler with
{\tt wxFileSystem::AddHandler(new wxArchiveFSHandler);} before calling this method
on a .zip or .htb file!

\membersection{wxHtmlHelpController::CreateHelpDialog}\label{wxhtmlhelpcontrollercreatehelpdialog}

\func{virtual wxHtmlHelpDialog*}{CreateHelpDialog}{\param{wxHtmlHelpData * }{data}}

This protected virtual method may be overridden so that when specifying the wxHF\_DIALOG style, the controller
uses a different dialog.

\membersection{wxHtmlHelpController::CreateHelpFrame}\label{wxhtmlhelpcontrollercreatehelpframe}

\func{virtual wxHtmlHelpFrame*}{CreateHelpFrame}{\param{wxHtmlHelpData * }{data}}

This protected virtual method may be overridden so that the controller
uses a different frame.

\membersection{wxHtmlHelpController::Display}\label{wxhtmlhelpcontrollerdisplay}

\func{void}{Display}{\param{const wxString\& }{x}}

Displays page {\it x}. This is THE important function - it is used to display
the help in application.

You can specify the page in many ways:

\begin{itemize}\itemsep=0pt
\item as direct filename of HTML document
\item as chapter name (from contents) or as a book name
\item as some word from index
\item even as any word (will be searched)
\end{itemize}

Looking for the page runs in these steps:

\begin{enumerate}\itemsep=0pt
\item try to locate file named x (if x is for example "doc/howto.htm")
\item try to open starting page of book named x
\item try to find x in contents (if x is for example "How To ...")
\item try to find x in index (if x is for example "How To ...")
\item switch to Search panel and start searching
\end{enumerate}

\func{void}{Display}{\param{const int }{id}}

This alternative form is used to search help contents by numeric IDs.

\pythonnote{The second form of this method is named DisplayId in
wxPython.}

\membersection{wxHtmlHelpController::DisplayContents}\label{wxhtmlhelpcontrollerdisplaycontents}

\func{void}{DisplayContents}{\void}

Displays help window and focuses contents panel.

\membersection{wxHtmlHelpController::DisplayIndex}\label{wxhtmlhelpcontrollerdisplayindex}

\func{void}{DisplayIndex}{\void}

Displays help window and focuses index panel.

\membersection{wxHtmlHelpController::KeywordSearch}\label{wxhtmlhelpcontrollerkeywordsearch}

\func{bool}{KeywordSearch}{\param{const wxString\& }{keyword}, \param{wxHelpSearchMode }{mode = wxHELP\_SEARCH\_ALL}}

Displays help window, focuses search panel and starts searching.  Returns true
if the keyword was found. Optionally it searches through the index (mode =
wxHELP\_SEARCH\_INDEX), default the content (mode = wxHELP\_SEARCH\_ALL).

{\bf Important:} KeywordSearch searches only pages listed in .hhc file(s).
You should list all pages in the contents file.

\membersection{wxHtmlHelpController::ReadCustomization}\label{wxhtmlhelpcontrollerreadcustomization}

\func{void}{ReadCustomization}{\param{wxConfigBase* }{cfg}, \param{wxString }{path = wxEmptyString}}

Reads the controller's setting (position of window, etc.)

\membersection{wxHtmlHelpController::SetTempDir}\label{wxhtmlhelpcontrollersettempdir}

\func{void}{SetTempDir}{\param{const wxString\& }{path}}

Sets the path for storing temporary files - cached binary versions of index and contents files. These binary
forms are much faster to read. Default value is empty string (empty string means
that no cached data are stored). Note that these files are {\it not} 
deleted when program exits.

Once created these cached files will be used in all subsequent executions 
of your application. If cached files become older than corresponding .hhp
file (e.g. if you regenerate documentation) it will be refreshed.

\membersection{wxHtmlHelpController::SetTitleFormat}\label{wxhtmlhelpcontrollersettitleformat}

\func{void}{SetTitleFormat}{\param{const wxString\& }{format}}

Sets format of title of the frame. Must contain exactly one "\%s"
(for title of displayed HTML page).

\membersection{wxHtmlHelpController::UseConfig}\label{wxhtmlhelpcontrolleruseconfig}

\func{void}{UseConfig}{\param{wxConfigBase* }{config}, \param{const wxString\& }{rootpath = wxEmptyString}}

Associates {\it config} object with the controller.

If there is associated config object, wxHtmlHelpController automatically
reads and writes settings (including wxHtmlWindow's settings) when needed.

The only thing you must do is create wxConfig object and call UseConfig.

If you do not use {\it UseConfig}, wxHtmlHelpController will use 
default wxConfig object if available (for details see 
\helpref{wxConfigBase::Get}{wxconfigbaseget} and 
\helpref{wxConfigBase::Set}{wxconfigbaseset}).

\membersection{wxHtmlHelpController::WriteCustomization}\label{wxhtmlhelpcontrollerwritecustomization}

\func{void}{WriteCustomization}{\param{wxConfigBase* }{cfg}, \param{wxString }{path = wxEmptyString}}

Stores controllers setting (position of window etc.)

