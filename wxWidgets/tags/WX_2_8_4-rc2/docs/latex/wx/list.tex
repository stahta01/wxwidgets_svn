%%%%%%%%%%%%%%%%%%%%%%%%%%%%%%%%%%%%%%%%%%%%%%%%%%%%%%%%%%%%%%%%%%%%%%%%%%%%%%%
%% Name:        list.tex
%% Purpose:     wxList
%% Author:      wxWidgets Team
%% Modified by:
%% Created:
%% RCS-ID:      $Id$
%% Copyright:   (c) wxWidgets Team
%% License:     wxWindows license
%%%%%%%%%%%%%%%%%%%%%%%%%%%%%%%%%%%%%%%%%%%%%%%%%%%%%%%%%%%%%%%%%%%%%%%%%%%%%%%

\section{\class{wxList}}\label{wxlist}

wxList classes provide linked list functionality for wxWidgets, and for an
application if it wishes.  Depending on the form of constructor used, a list
can be keyed on integer or string keys to provide a primitive look-up ability,
but please note that this feature is {\bf deprecated}.
See \helpref{wxHashMap}{wxhashmap}\rtfsp for a faster method of storage
when random access is required.

While wxList class in the previous versions of wxWidgets only could contain
elements of type wxObject and had essentially untyped interface (thus allowing
you to put apples in the list and read back oranges from it), the new wxList
classes family may contain elements of any type and has much more strict type
checking. Unfortunately, it also requires an additional line to be inserted in
your program for each list class you use (which is the only solution short of
using templates which is not done in wxWidgets because of portability issues).

The general idea is to have the base class wxListBase working with {\it void *}
data but make all of its dangerous (because untyped) functions protected, so
that they can only be used from derived classes which, in turn, expose a type
safe interface. With this approach a new wxList-like class must be defined for
each list type (i.e. list of ints, of wxStrings or of MyObjects). This is done
with {\it WX\_DECLARE\_LIST} and {\it WX\_DEFINE\_LIST} macros like this
(notice the similarity with WX\_DECLARE\_OBJARRAY and WX\_IMPLEMENT\_OBJARRAY
macros):

\wxheading{Example}

\begin{verbatim}
    // this part might be in a header or source (.cpp) file
    class MyListElement
    {
        ... // whatever
    };

    // declare our list class: this macro declares and partly implements MyList
    // class (which derives from wxListBase)
    WX_DECLARE_LIST(MyListElement, MyList);

    ...

    // the only requirement for the rest is to be AFTER the full declaration of
    // MyListElement (for WX_DECLARE_LIST forward declaration is enough), but
    // usually it will be found in the source file and not in the header

    #include <wx/listimpl.cpp>
    WX_DEFINE_LIST(MyList);

    // now MyList class may be used as a usual wxList, but all of its methods
    // will take/return the objects of the right (i.e. MyListElement) type. You
    // also have MyList::Node type which is the type-safe version of wxNode.
    MyList list;
    MyListElement element;
    list.Append(&element);     // ok
    list.Append(17);           // error: incorrect type

    // let's iterate over the list
    for ( MyList::Node *node = list.GetFirst(); node; node = node->GetNext() )
    {
        MyListElement *current = node->GetData();

        ...process the current element...
    }
\end{verbatim}

For compatibility with previous versions wxList and wxStringList classes are
still defined, but their usage is deprecated and they will disappear in the
future versions completely. The use of the latter is especially discouraged as
it is not only unsafe but is also much less efficient than
\helpref{wxArrayString}{wxarraystring} class.

In the documentation of the list classes below, the template notations are
used even though these classes are not really templates at all -- but it helps
to think about them as if they were. You should replace wxNode<T> with
wxListName::Node and T itself with the list element type (i.e. the first
parameter of WX\_DECLARE\_LIST).

\wxheading{Derived from}

\helpref{wxObject}{wxobject}

\wxheading{Include files}

<wx/list.h>

\wxheading{Example}

It is very common to iterate on a list as follows:

\begin{verbatim}
  ...
  wxWindow *win1 = new wxWindow(...);
  wxWindow *win2 = new wxWindow(...);

  wxList SomeList;
  SomeList.Append(win1);
  SomeList.Append(win2);

  ...

  wxNode *node = SomeList.GetFirst();
  while (node)
  {
    wxWindow *win = node->GetData();
    ...
    node = node->GetNext();
  }
\end{verbatim}

To delete nodes in a list as the list is being traversed, replace

\begin{verbatim}
    ...
    node = node->GetNext();
    ...
\end{verbatim}

with

\begin{verbatim}
    ...
    delete win;
    delete node;
    node = SomeList.GetFirst();
    ...
\end{verbatim}

See \helpref{wxNode}{wxnode} for members that retrieve the data associated with a node, and
members for getting to the next or previous node.

\wxheading{See also}

\helpref{wxNode}{wxnode},
\helpref{wxArray}{wxarray}

\latexignore{\rtfignore{\wxheading{Members}}}

\membersection{wxList::wxList}\label{wxlistctor}

\func{}{wxList}{\void}

\func{}{wxList}{\param{int}{ n}, \param{T *}{objects[]}}

\func{}{wxList}{\param{T *}{object}, ...}

{\bf Note}: keyed lists are deprecated and should not be used in new code.

\func{}{wxList}{\param{unsigned int}{ key\_type}}

Constructors. {\it key\_type} is one of wxKEY\_NONE, wxKEY\_INTEGER, or wxKEY\_STRING,
and indicates what sort of keying is required (if any).

{\it objects} is an array of {\it n} objects with which to initialize the list.

The variable-length argument list constructor must be supplied with a
terminating NULL.

\membersection{wxList::\destruct{wxList}}\label{wxlistdtor}

\func{}{\destruct{wxList}}{\void}

Destroys the list.  Also destroys any remaining nodes, but does not destroy
client data held in the nodes.

\membersection{wxList::Append}\label{wxlistappend}

\func{wxNode<T> *}{Append}{\param{T *}{object}}

{\bf Note}: keyed lists are deprecated and should not be used in new code.

\func{wxNode<T> *}{Append}{\param{long}{ key}, \param{T *}{object}}

\func{wxNode<T> *}{Append}{\param{const wxString\& }{key}, \param{T *}{object}}

Appends a new \helpref{wxNode}{wxnode} to the end of the list and puts a
pointer to the \rtfsp{\it object} in the node.  The last two forms store a key
with the object for later retrieval using the key. The new node is returned in
each case.

The key string is copied and stored by the list implementation.

\membersection{wxList::Clear}\label{wxlistclear}

\func{void}{Clear}{\void}

Clears the list (but does not delete the client data stored with each node
unless you called DeleteContents({\tt true}), in which case it deletes data).

\membersection{wxList::DeleteContents}\label{wxlistdeletecontents}

\func{void}{DeleteContents}{\param{bool}{ destroy}}

If {\it destroy} is {\tt true}, instructs the list to call {\it delete} on the client contents of
a node whenever the node is destroyed. The default is {\tt false}.

\membersection{wxList::DeleteNode}\label{wxlistdeletenode}

\func{bool}{DeleteNode}{\param{wxNode<T> *}{node}}

Deletes the given node from the list, returning {\tt true} if successful.

\membersection{wxList::DeleteObject}\label{wxlistdeleteobject}

\func{bool}{DeleteObject}{\param{T *}{object}}

Finds the given client {\it object} and deletes the appropriate node from the list, returning
{\tt true} if successful. The application must delete the actual object separately.

\membersection{wxList::Erase}\label{wxlisterase}

\func{void}{Erase}{\param{wxNode<T> *}{node}}

Removes element at given position.

\membersection{wxList::Find}\label{wxlistfind}

\func{wxNode<T> *}{Find}{\param{T *}{ object}}

Returns the node whose client data is {\it object} or NULL if none found.

{\bf Note}: keyed lists are deprecated and should not be used in new code.

\func{wxNode<T> *}{Find}{\param{long}{ key}}

\func{wxNode<T> *}{Find}{\param{const wxString\& }{key}}

Returns the node whose stored key matches {\it key}. Use on a keyed list only.

\membersection{wxList::GetCount}\label{wxlistgetcount}

\constfunc{size\_t}{GetCount}{\void}

Returns the number of elements in the list.

\membersection{wxList::GetFirst}\label{wxlistgetfirst}

\func{wxNode<T> *}{GetFirst}{\void}

Returns the first node in the list (NULL if the list is empty).

\membersection{wxList::GetLast}\label{wxlistgetlast}

\func{wxNode<T> *}{GetLast}{\void}

Returns the last node in the list (NULL if the list is empty).

\membersection{wxList::IndexOf}\label{wxlistindexof}

\func{int}{IndexOf}{\param{T*}{ obj }}

Returns the index of {\it obj} within the list or {\tt wxNOT\_FOUND} if {\it obj}
is not found in the list.

\membersection{wxList::Insert}\label{wxlistinsert}

\func{wxNode<T> *}{Insert}{\param{T *}{object}}

Insert object at front of list.

\func{wxNode<T> *}{Insert}{\param{size\_t }{position}, \param{T *}{object}}

Insert object before {\it position}, i.e. the index of the new item in the
list will be equal to {\it position}. {\it position} should be less than or
equal to \helpref{GetCount}{wxlistgetcount}; if it is equal to it, this is the
same as calling \helpref{Append}{wxlistappend}.

\func{wxNode<T> *}{Insert}{\param{wxNode<T> *}{node}, \param{T *}{object}}

Inserts the object before the given {\it node}.

\membersection{wxList::IsEmpty}\label{wxlistisempty}

\constfunc{bool}{IsEmpty}{\void}

Returns {\tt true} if the list is empty, {\tt false} otherwise.

% Use different label name to avoid clashing with wxListItem label
\membersection{wxList::Item}\label{wxlistitemfunc}

\constfunc{wxNode<T> *}{Item}{\param{size\_t }{index}}

Returns the node at given position in the list.

\membersection{wxList::Member}\label{wxlistmember}

\func{wxNode<T> *}{Member}{\param{T *}{object}}

{\bf NB:} This function is deprecated, use \helpref{Find}{wxlistfind} instead.

Returns the node associated with {\it object} if it is in the list, NULL otherwise.

\membersection{wxList::Nth}\label{wxlistnth}

\func{wxNode<T> *}{Nth}{\param{int}{ n}}

{\bf NB:} This function is deprecated, use \helpref{Item}{wxlistitemfunc} instead.

Returns the {\it nth} node in the list, indexing from zero (NULL if the list is empty
or the nth node could not be found).

\membersection{wxList::Number}\label{wxlistnumber}

\func{int}{Number}{\void}

{\bf NB:} This function is deprecated, use \helpref{GetCount}{wxlistgetcount} instead.

Returns the number of elements in the list.

\membersection{wxList::Sort}\label{wxlistsort}

\func{void}{Sort}{\param{wxSortCompareFunction}{ compfunc}}

\begin{verbatim}
  // Type of compare function for list sort operation (as in 'qsort')
  typedef int (*wxSortCompareFunction)(const void *elem1, const void *elem2);
\end{verbatim}

Allows the sorting of arbitrary lists by giving
a function to compare two list elements. We use the system {\bf qsort} function
for the actual sorting process.

If you use untyped wxList the sort function receives pointers to wxObject
pointers (wxObject **), so be careful to dereference appropriately - but,
of course, a better solution is to use list of appropriate type defined with
{\tt WX\_DECLARE\_LIST}.

Example:

\begin{verbatim}
  int listcompare(const void *arg1, const void *arg2)
  {
    return(compare(**(wxString **)arg1,    // use the wxString 'compare'
                   **(wxString **)arg2));  // function
  }

  void main()
  {
    wxList list;

    list.Append(new wxString("DEF"));
    list.Append(new wxString("GHI"));
    list.Append(new wxString("ABC"));
    list.Sort(listcompare);
  }
\end{verbatim}
