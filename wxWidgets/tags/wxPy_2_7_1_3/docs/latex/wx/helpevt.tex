\section{\class{wxHelpEvent}}\label{wxhelpevent}

A help event is sent when the user has requested context-sensitive help.
This can either be caused by the application requesting
context-sensitive help mode via \helpref{wxContextHelp}{wxcontexthelp}, or
(on MS Windows) by the system generating a WM\_HELP message when the user pressed F1 or clicked
on the query button in a dialog caption.

A help event is sent to the window that the user clicked on, and is propagated up the
window hierarchy until the event is processed or there are no more event handlers.
The application should call wxEvent::GetId to check the identity of the clicked-on window,
and then either show some suitable help or call wxEvent::Skip if the identifier is unrecognised.
Calling Skip is important because it allows wxWidgets to generate further events for ancestors
of the clicked-on window. Otherwise it would be impossible to show help for container windows,
since processing would stop after the first window found.

\wxheading{Derived from}

\helpref{wxCommandEvent}{wxcommandevent}\\
\helpref{wxEvent}{wxevent}\\
\helpref{wxObject}{wxobject}

\wxheading{Include files}

<wx/event.h>

\wxheading{Event table macros}

To process an activate event, use these event handler macros to direct input to a member
function that takes a wxHelpEvent argument.

\twocolwidtha{7cm}
\begin{twocollist}\itemsep=0pt
\twocolitem{{\bf EVT\_HELP(id, func)}}{Process a wxEVT\_HELP event.}
\twocolitem{{\bf EVT\_HELP\_RANGE(id1, id2, func)}}{Process a wxEVT\_HELP event for a range of ids.}
\end{twocollist}%

\wxheading{See also}

\helpref{wxContextHelp}{wxcontexthelp},\rtfsp
\helpref{wxDialog}{wxdialog},\rtfsp
\helpref{Event handling overview}{eventhandlingoverview}

\latexignore{\rtfignore{\wxheading{Members}}}

\membersection{wxHelpEvent::wxHelpEvent}\label{wxhelpeventctor}

\func{}{wxHelpEvent}{\param{WXTYPE }{eventType = 0}, \param{wxWindowID }{id = 0},
 \param{const wxPoint\& }{point}}

Constructor.

\membersection{wxHelpEvent::GetOrigin}\label{wxhelpeventgetorigin}

\constfunc{wxHelpEvent::Origin }{GetOrigin}{\void}

Returns the origin of the help event which is one of the following values:

\twocolwidtha{7cm}
\begin{twocollist}\itemsep=0pt
\twocolitem{{\bf Origin\_Unknown}}{Unrecognized event source.}
\twocolitem{{\bf Origin\_Keyboard}}{Event generated by \texttt{F1} key press.}
\twocolitem{{\bf Origin\_HelpButton}}{Event generated by 
\helpref{wxContextHelp}{wxcontexthelp} or using the "?" title bur button under
MS Windows.}
\end{twocollist}

The application may handle events generated using the keyboard or mouse
differently, e.g. by using \helpref{wxGetMousePosition()}{wxgetmouseposition} 
for the mouse events.

\wxheading{See also}

\helpref{wxHelpEvent::SetOrigin}{wxhelpeventsetorigin}


\membersection{wxHelpEvent::GetPosition}\label{wxhelpeventgetposition}

\constfunc{const wxPoint\&}{GetPosition}{\void}

Returns the left-click position of the mouse, in screen coordinates. This allows
the application to position the help appropriately.

\membersection{wxHelpEvent::SetOrigin}\label{wxhelpeventsetorigin}

\func{void}{SetOrigin}{\param{wxHelpEvent::Origin }{origin}}

Set the help event origin, only used internally by wxWidgets normally.

\wxheading{See also}

\helpref{wxHelpEvent::GetOrigin}{wxhelpeventgetorigin}


\membersection{wxHelpEvent::SetPosition}\label{wxhelpeventsetposition}

\func{void}{SetPosition}{\param{const wxPoint\&}{ pt}}

Sets the left-click position of the mouse, in screen coordinates.

