\section{\class{wxDialog}}\label{wxdialog}

A dialog box is a window with a title bar and sometimes a system menu, which
can be moved around the screen. It can contain controls and other windows and
is usually used to allow the user to make some choice or to answer a question.

\wxheading{Derived from}

\helpref{wxWindow}{wxwindow}\\
\helpref{wxEvtHandler}{wxevthandler}\\
\helpref{wxObject}{wxobject}

\wxheading{Include files}

<wx/dialog.h>

\wxheading{Remarks}

There are two kinds of dialog -- {\it modal}\ and {\it modeless}. A modal dialog
blocks program flow and user input on other windows until it is dismissed,
whereas a modeless dialog behaves more like a frame in that program flow
continues, and input on other windows is still possible. To show a modal dialog
you should use \helpref{ShowModal}{wxdialogshowmodal} method while to show
dialog modelessly you simply use \helpref{Show}{wxdialogshow}, just as with the
frames.

Note that the modal dialogs are one of the very few examples of
wxWindow-derived objects which may be created on the stack and not on the heap.
In other words, although this code snippet
\begin{verbatim}
    void AskUser()
    {
        MyAskDialog *dlg = new MyAskDialog(...);
        if ( dlg->ShowModal() == wxID_OK )
            ...
        //else: dialog was cancelled or some another button pressed

        dlg->Destroy();
    }
\end{verbatim}
works, you can also achieve the same result by using a simpler code fragment
below:
\begin{verbatim}
    void AskUser()
    {
        MyAskDialog dlg(...);
        if ( dlg.ShowModal() == wxID_OK )
            ...

        // no need to call Destroy() here
    }
\end{verbatim}

A dialog may be loaded from a wxWindows resource file (extension {\tt wxr}),
which may itself be created by Dialog Editor. For details, see 
\helpref{The wxWindows resource system}{resourceformats}, 
\helpref{wxWindows resource functions}{resourcefuncs} 
and the resource sample.

An application can define an \helpref{wxCloseEvent}{wxcloseevent} handler for
the dialog to respond to system close events.

\wxheading{Window styles}

\twocolwidtha{5cm}
\begin{twocollist}\itemsep=0pt
\twocolitem{\windowstyle{wxCAPTION}}{Puts a caption on the dialog box.}
\twocolitem{\windowstyle{wxDEFAULT\_DIALOG\_STYLE}}{Equivalent to a combination of wxCAPTION, wxSYSTEM\_MENU and wxTHICK\_FRAME}
\twocolitem{\windowstyle{wxRESIZE\_BORDER}}{Display a resizeable frame around the window.}
\twocolitem{\windowstyle{wxSYSTEM\_MENU}}{Display a system menu.}
\twocolitem{\windowstyle{wxTHICK\_FRAME}}{Display a thick frame around the window.}
\twocolitem{\windowstyle{wxSTAY\_ON\_TOP}}{The dialog stays on top of all other windows (Windows only).}
\twocolitem{\windowstyle{wxNO\_3D}}{Under Windows, specifies that the child controls
should not have 3D borders unless specified in the control.}
\twocolitem{\windowstyle{wxDIALOG\_NO\_PARENT}}{By default, the dialogs created
with {\tt NULL} parent window will be given the 
\helpref{applications top level window}{wxappgettopwindow} as parent. Use this
style to prevent this from happening and create a really orphan dialog (note
that this is not recommended for modal dialogs).}
\twocolitem{\windowstyle{wxDIALOG\_EX\_CONTEXTHELP}}{Under Windows, puts a query button on the
caption. When pressed, Windows will go into a context-sensitive help mode and wxWindows will send
a wxEVT\_HELP event if the user clicked on an application window. {\it Note}\ that this is an extended
style and must be set by calling \helpref{SetExtraStyle}{wxwindowsetextrastyle} before Create is called (two-step construction).}
\end{twocollist}

Under Unix or Linux, MWM (the Motif Window Manager) or other window managers
recognizing the MHM hints should be running for any of these styles to have an
effect.

See also \helpref{Generic window styles}{windowstyles}.

\wxheading{See also}

\helpref{wxDialog overview}{wxdialogoverview}, \helpref{wxFrame}{wxframe}, \helpref{Resources}{resources},\rtfsp
\helpref{Validator overview}{validatoroverview}

\latexignore{\rtfignore{\wxheading{Members}}}

\membersection{wxDialog::wxDialog}\label{wxdialogconstr}

\func{}{wxDialog}{\void}

Default constructor.

\func{}{wxDialog}{\param{wxWindow* }{parent}, \param{wxWindowID }{id},\rtfsp
\param{const wxString\& }{title},\rtfsp
\param{const wxPoint\& }{pos = wxDefaultPosition},\rtfsp
\param{const wxSize\& }{size = wxDefaultSize},\rtfsp
\param{long}{ style = wxDEFAULT\_DIALOG\_STYLE},\rtfsp
\param{const wxString\& }{name = ``dialogBox"}}

Constructor.

\wxheading{Parameters}

\docparam{parent}{Can be NULL, a frame or another dialog box.}

\docparam{id}{An identifier for the dialog. A value of -1 is taken to mean a default.}

\docparam{title}{The title of the dialog.}

\docparam{pos}{The dialog position. A value of (-1, -1) indicates a default position, chosen by
either the windowing system or wxWindows, depending on platform.}

\docparam{size}{The dialog size. A value of (-1, -1) indicates a default size, chosen by
either the windowing system or wxWindows, depending on platform.}

\docparam{style}{The window style. See \helpref{wxDialog}{wxdialog}.}

\docparam{name}{Used to associate a name with the window,
allowing the application user to set Motif resource values for
individual dialog boxes.}

\wxheading{See also}

\helpref{wxDialog::Create}{wxdialogcreate}

\membersection{wxDialog::\destruct{wxDialog}}

\func{}{\destruct{wxDialog}}{\void}

Destructor. Deletes any child windows before deleting the physical window.

\membersection{wxDialog::Centre}\label{wxdialogcentre}

\func{void}{Centre}{\param{int}{ direction = wxBOTH}}

Centres the dialog box on the display.

\wxheading{Parameters}

\docparam{direction}{May be {\tt wxHORIZONTAL}, {\tt wxVERTICAL} or {\tt wxBOTH}.}

\membersection{wxDialog::Create}\label{wxdialogcreate}

\func{bool}{Create}{\param{wxWindow* }{parent}, \param{wxWindowID }{id},\rtfsp
\param{const wxString\& }{title},\rtfsp
\param{const wxPoint\& }{pos = wxDefaultPosition},\rtfsp
\param{const wxSize\& }{size = wxDefaultSize},\rtfsp
\param{long}{ style = wxDEFAULT\_DIALOG\_STYLE},\rtfsp
\param{const wxString\& }{name = ``dialogBox"}}

Used for two-step dialog box construction. See \helpref{wxDialog::wxDialog}{wxdialogconstr}\rtfsp
for details.

\membersection{wxDialog::EndModal}\label{wxdialogendmodal}

\func{void}{EndModal}{\param{int }{retCode}}

Ends a modal dialog, passing a value to be returned from the \helpref{wxDialog::ShowModal}{wxdialogshowmodal}\rtfsp
invocation.

\wxheading{Parameters}

\docparam{retCode}{The value that should be returned by {\bf ShowModal}.}

\wxheading{See also}

\helpref{wxDialog::ShowModal}{wxdialogshowmodal},\rtfsp
\helpref{wxDialog::GetReturnCode}{wxdialoggetreturncode},\rtfsp
\helpref{wxDialog::SetReturnCode}{wxdialogsetreturncode}

\membersection{wxDialog::GetReturnCode}\label{wxdialoggetreturncode}

\func{int}{GetReturnCode}{\void}

Gets the return code for this window.

\wxheading{Remarks}

A return code is normally associated with a modal dialog, where \helpref{wxDialog::ShowModal}{wxdialogshowmodal} returns
a code to the application.

\wxheading{See also}

\helpref{wxDialog::SetReturnCode}{wxdialogsetreturncode}, \helpref{wxDialog::ShowModal}{wxdialogshowmodal},\rtfsp
\helpref{wxDialog::EndModal}{wxdialogendmodal}

\membersection{wxDialog::GetTitle}\label{wxdialoggettitle}

\constfunc{wxString}{GetTitle}{\void}

Returns the title of the dialog box.

\membersection{wxDialog::Iconize}\label{wxdialogiconized}

\func{void}{Iconize}{\param{const bool}{ iconize}}

Iconizes or restores the dialog. Windows only.

\wxheading{Parameters}

\docparam{iconize}{If TRUE, iconizes the dialog box; if FALSE, shows and restores it.}

\wxheading{Remarks}

Note that in Windows, iconization has no effect since dialog boxes cannot be
iconized. However, applications may need to explicitly restore dialog
boxes under Motif which have user-iconizable frames, and under Windows
calling {\tt Iconize(FALSE)} will bring the window to the front, as does
\rtfsp{\tt Show(TRUE)}.

\membersection{wxDialog::IsIconized}\label{wxdialogisiconized}

\constfunc{bool}{IsIconized}{\void}

Returns TRUE if the dialog box is iconized. Windows only.

\wxheading{Remarks}

Always returns FALSE under Windows since dialogs cannot be iconized.

\membersection{wxDialog::IsModal}\label{wxdialogismodal}

\constfunc{bool}{IsModal}{\void}

Returns TRUE if the dialog box is modal, FALSE otherwise.

\membersection{wxDialog::OnCharHook}\label{wxdialogoncharhook}

\func{void}{OnCharHook}{\param{wxKeyEvent\&}{ event}}

This member is called to allow the window to intercept keyboard events
before they are processed by child windows.

%For more information, see \helpref{wxWindow::OnCharHook}{wxwindowoncharhook}

\wxheading{Remarks}

wxDialog implements this handler to fake a cancel command if the escape key has been
pressed. This will dismiss the dialog.

\membersection{wxDialog::OnApply}\label{wxdialogonapply}

\func{void}{OnApply}{\param{wxCommandEvent\& }{event}}

The default handler for the wxID\_APPLY identifier.

\wxheading{Remarks}

This function calls \helpref{wxWindow::Validate}{wxwindowvalidate} and \helpref{wxWindow::TransferDataToWindow}{wxwindowtransferdatatowindow}.

\wxheading{See also}

\helpref{wxDialog::OnOK}{wxdialogonok}, \helpref{wxDialog::OnCancel}{wxdialogoncancel}

\membersection{wxDialog::OnCancel}\label{wxdialogoncancel}

\func{void}{OnCancel}{\param{wxCommandEvent\& }{event}}

The default handler for the wxID\_CANCEL identifier.

\wxheading{Remarks}

The function either calls {\bf EndModal(wxID\_CANCEL)} if the dialog is modal, or
sets the return value to wxID\_CANCEL and calls {\bf Show(FALSE)} if the dialog is modeless.

\wxheading{See also}

\helpref{wxDialog::OnOK}{wxdialogonok}, \helpref{wxDialog::OnApply}{wxdialogonapply}

\membersection{wxDialog::OnOK}\label{wxdialogonok}

\func{void}{OnOK}{\param{wxCommandEvent\& }{event}}

The default handler for the wxID\_OK identifier.

\wxheading{Remarks}

The function calls
\rtfsp\helpref{wxWindow::Validate}{wxwindowvalidate}, then \helpref{wxWindow::TransferDataFromWindow}{wxwindowtransferdatafromwindow}.
If this returns TRUE, the function either calls {\bf EndModal(wxID\_OK)} if the dialog is modal, or
sets the return value to wxID\_OK and calls {\bf Show(FALSE)} if the dialog is modeless.

\wxheading{See also}

\helpref{wxDialog::OnCancel}{wxdialogoncancel}, \helpref{wxDialog::OnApply}{wxdialogonapply}

\membersection{wxDialog::OnSysColourChanged}\label{wxdialogonsyscolourchanged}

\func{void}{OnSysColourChanged}{\param{wxSysColourChangedEvent\& }{event}}

The default handler for wxEVT\_SYS\_COLOUR\_CHANGED.

\wxheading{Parameters}

\docparam{event}{The colour change event.}

\wxheading{Remarks}

Changes the dialog's colour to conform to the current settings (Windows only).
Add an event table entry for your dialog class if you wish the behaviour
to be different (such as keeping a user-defined
background colour). If you do override this function, call wxEvent::Skip to
propagate the notification to child windows and controls.

\wxheading{See also}

\helpref{wxSysColourChangedEvent}{wxsyscolourchangedevent}

\membersection{wxDialog::SetIcon}\label{wxdialogseticon}

\func{void}{SetIcon}{\param{const wxIcon\& }{icon}}

Sets the icon for this dialog.

\wxheading{Parameters}

\docparam{icon}{The icon to associate with this dialog.}

See also \helpref{wxIcon}{wxicon}.

\membersection{wxDialog::SetIcons}\label{wxdialogseticons}

\func{void}{SetIcons}{\param{const wxIconBundle\& }{icons}}

Sets the icons for this dialog.

\wxheading{Parameters}

\docparam{icons}{The icons to associate with this dialog.}

See also \helpref{wxIconBundle}{wxiconbundle}.

\membersection{wxDialog::SetModal}\label{wxdialogsetmodal}

\func{void}{SetModal}{\param{const bool}{ flag}}

{\bf NB:} This function is deprecated and doesn't work for all ports, just use 
\helpref{ShowModal}{wxdialogshowmodal} to show a modal dialog instead.

Allows the programmer to specify whether the dialog box is modal (wxDialog::Show blocks control
until the dialog is hidden) or modeless (control returns immediately).

\wxheading{Parameters}

\docparam{flag}{If TRUE, the dialog will be modal, otherwise it will be modeless.}

\membersection{wxDialog::SetReturnCode}\label{wxdialogsetreturncode}

\func{void}{SetReturnCode}{\param{int }{retCode}}

Sets the return code for this window.

\wxheading{Parameters}

\docparam{retCode}{The integer return code, usually a control identifier.}

\wxheading{Remarks}

A return code is normally associated with a modal dialog, where \helpref{wxDialog::ShowModal}{wxdialogshowmodal} returns
a code to the application. The function \helpref{wxDialog::EndModal}{wxdialogendmodal} calls {\bf SetReturnCode}.

\wxheading{See also}

\helpref{wxDialog::GetReturnCode}{wxdialoggetreturncode}, \helpref{wxDialog::ShowModal}{wxdialogshowmodal},\rtfsp
\helpref{wxDialog::EndModal}{wxdialogendmodal}

\membersection{wxDialog::SetTitle}\label{wxdialogsettitle}

\func{void}{SetTitle}{\param{const wxString\& }{ title}}

Sets the title of the dialog box.

\wxheading{Parameters}

\docparam{title}{The dialog box title.}

\membersection{wxDialog::Show}\label{wxdialogshow}

\func{bool}{Show}{\param{const bool}{ show}}

Hides or shows the dialog.

\wxheading{Parameters}

\docparam{show}{If TRUE, the dialog box is shown and brought to the front;
otherwise the box is hidden. If FALSE and the dialog is
modal, control is returned to the calling program.}

\wxheading{Remarks}

The preferred way of dismissing a modal dialog is to use \helpref{wxDialog::EndModal}{wxdialogendmodal}.

\membersection{wxDialog::ShowModal}\label{wxdialogshowmodal}

\func{int}{ShowModal}{\void}

Shows a modal dialog. Program flow does not return until the dialog has been dismissed with\rtfsp
\helpref{wxDialog::EndModal}{wxdialogendmodal}.

\wxheading{Return value}

The return value is the value set with \helpref{wxDialog::SetReturnCode}{wxdialogsetreturncode}.

\wxheading{See also}

\helpref{wxDialog::EndModal}{wxdialogendmodal},\rtfsp
\helpref{wxDialog:GetReturnCode}{wxdialoggetreturncode},\rtfsp
\helpref{wxDialog::SetReturnCode}{wxdialogsetreturncode}

