%%%%%%%%%%%%%%%%%%%%%%%%%%%%%%%%%%%%%%%%%%%%%%%%%%%%%%%%%%%%%%%%%%%%%%%%%%%%%%%
%% Name:        tstyles.tex
%% Purpose:     Window styles documenation
%% Author:      wxWidgets Team
%% Modified by:
%% Created:
%% RCS-ID:      $Id$
%% Copyright:   (c) wxWidgets Team
%% License:     wxWindows license
%%%%%%%%%%%%%%%%%%%%%%%%%%%%%%%%%%%%%%%%%%%%%%%%%%%%%%%%%%%%%%%%%%%%%%%%%%%%%%%

\section{Window styles}\label{windowstyles}

Window styles are used to specify alternative behaviour and appearances for windows, when they are
created. The symbols are defined in such a way that they can be combined in a `bit-list' using the
C++ {\it bitwise-or} operator. For example:

\begin{verbatim}
  wxCAPTION | wxMINIMIZE_BOX | wxMAXIMIZE_BOX | wxRESIZE_BORDER
\end{verbatim}

For the window styles specific to each window class, please see the documentation
for the window. Most windows can use the generic styles listed for \helpref{wxWindow}{wxwindow} in
addition to their own styles.

