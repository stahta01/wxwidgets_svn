% -----------------------------------------------------------------------------
% wxFilterOutputStream
% -----------------------------------------------------------------------------
\section{\class{wxFilterOutputStream}}\label{wxfilteroutputstream}

A filter stream has the capability of a normal
stream but it can be placed on top of another stream. So, for example, it
can compress, encrypt the data which are passed to it and write them to another
stream.

\wxheading{Derived from}

\helpref{wxOutputStream}{wxoutputstream}\\
\helpref{wxStreamBase}{wxstreambase}

\wxheading{Include files}

<wx/stream.h>

\wxheading{Note}

The use of this class is exactly the same as of wxOutputStream. Only a constructor
differs and it is documented below.

\wxheading{See also}

\helpref{wxFilterClassFactory}{wxfilterclassfactory}\\
\helpref{wxFilterInputStream}{wxfilterinputstream}

\latexignore{\rtfignore{\wxheading{Members}}}

% -----------
% ctor & dtor
% -----------
\membersection{wxFilterOutputStream::wxFilterOutputStream}\label{wxfilteroutputstreamctor}

\func{}{wxFilterOutputStream}{\param{wxOutputStream\&}{ stream}}

\func{}{wxFilterOutputStream}{\param{wxOutputStream*}{ stream}}

Initializes a "filter" stream.

If the parent stream is passed as a pointer then the new filter stream
takes ownership of it. If it is passed by reference then it does not.

