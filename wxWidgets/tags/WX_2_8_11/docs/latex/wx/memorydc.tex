\section{\class{wxMemoryDC}}\label{wxmemorydc}

A memory device context provides a means to draw graphics onto a bitmap. When
drawing in to a mono-bitmap, using {\tt wxWHITE}, {\tt wxWHITE\_PEN} and
{\tt wxWHITE\_BRUSH}
will draw the background colour (i.e. 0) whereas all other colours will draw the
foreground colour (i.e. 1).

\wxheading{Derived from}

\helpref{wxDC}{wxdc}\\
\helpref{wxObject}{wxobject}

\wxheading{Include files}

<wx/dcmemory.h>

\wxheading{Remarks}

A bitmap must be selected into the new memory DC before it may be used
for anything.  Typical usage is as follows:

\begin{verbatim}
  // Create a memory DC
  wxMemoryDC temp_dc;
  temp_dc.SelectObject(test_bitmap);

  // We can now draw into the memory DC...
  // Copy from this DC to another DC.
  old_dc.Blit(250, 50, BITMAP_WIDTH, BITMAP_HEIGHT, temp_dc, 0, 0);
\end{verbatim}

Note that the memory DC {\it must} be deleted (or the bitmap selected out of it) before a bitmap
can be reselected into another memory DC.

\wxheading{See also}

\helpref{wxBitmap}{wxbitmap}, \helpref{wxDC}{wxdc}

\latexignore{\rtfignore{\wxheading{Members}}}

\membersection{wxMemoryDC::wxMemoryDC}\label{wxmemorydcctor}

\func{}{wxMemoryDC}{\void}

Constructs a new memory device context.

Use the \helpref{IsOk}{wxdcisok} member to test whether the constructor was successful
in creating a usable device context.
Don't forget to select a bitmap into the DC before drawing on it.

\func{}{wxMemoryDC}{\param{wxBitmap\& }{bitmap}}

Constructs a new memory device context and calls \helpref{SelectObject}{wxmemorydcselectobject}
with the given bitmap.
Use the \helpref{IsOk}{wxdcisok} member to test whether the constructor was successful
in creating a usable device context.


\membersection{wxMemoryDC::SelectObject}\label{wxmemorydcselectobject}

\func{void}{SelectObject}{\param{wxBitmap\& }{bitmap}}

Works exactly like \helpref{SelectObjectAsSource}{wxmemorydcselectobjectassource} but
this is the function you should use when you select a bitmap because you want to modify
it, e.g. drawing on this DC.

Be careful to use this function and not \helpref{SelectObjectAsSource}{wxmemorydcselectobjectassource}
when you want to modify the bitmap you are selecting otherwise you may incurr in some
problems related to wxBitmap being a reference counted object
(see \helpref{reference counting overview}{trefcount}).

\wxheading{See also}

\helpref{wxDC::DrawBitmap}{wxdcdrawbitmap}



\membersection{wxMemoryDC::SelectObjectAsSource}\label{wxmemorydcselectobjectassource}

\func{void}{SelectObjectAsSource}{\param{const wxBitmap\& }{bitmap}}

Selects the given bitmap into the device context, to use as the memory
bitmap. Selecting the bitmap into a memory DC allows you to draw into
the DC (and therefore the bitmap) and also to use \helpref{wxDC::Blit}{wxdcblit} to copy
the bitmap to a window. For this purpose, you may find \helpref{wxDC::DrawIcon}{wxdcdrawicon}\rtfsp
easier to use instead.

If the argument is wxNullBitmap (or some other uninitialised wxBitmap) the current bitmap is
selected out of the device context, and the original bitmap restored, allowing the current bitmap to
be destroyed safely.

\wxheading{See also}

\helpref{wxMemoryDC::SelectObject}{wxmemorydcselectobject}

