%
% automatically generated by HelpGen from
% htmltag.tex at 14/Mar/99 20:13:37
%

\section{\class{wxHtmlTag}}\label{wxhtmltag}

This class represents a single HTML tag. 
It is used by \helpref{tag handlers}{handlers}.

\wxheading{Derived from}

wxObject

\wxheading{Include files}

<wx/html/htmltag.h>

\latexignore{\rtfignore{\wxheading{Members}}}

\membersection{wxHtmlTag::wxHtmlTag}\label{wxhtmltagwxhtmltag}

\func{}{wxHtmlTag}{\param{wxHtmlTag *}{parent}, \param{const wxString\& }{source}, 
\param{int }{pos}, \param{int }{end\_pos}, 
\param{wxHtmlTagsCache* }{cache}, \param{wxHtmlEntitiesParser *}{entParser}}

Constructor. You will probably never have to construct a wxHtmlTag object
yourself. Feel free to ignore the constructor parameters.
Have a look at src/html/htmlpars.cpp if you're interested in creating it.

\membersection{wxHtmlTag::GetAllParams}\label{wxhtmltaggetallparams}

\constfunc{const wxString\&}{GetAllParams}{\void}

Returns a string containing all parameters.

Example : tag contains {\tt <FONT SIZE=+2 COLOR="\#000000">}. Call to
tag.GetAllParams() would return {\tt SIZE=+2 COLOR="\#000000"}.


\membersection{wxHtmlTag::GetBeginPos}\label{wxhtmltaggetbeginpos}

\constfunc{int}{GetBeginPos}{\void}

Returns beginning position of the text {\it between} this tag and paired
ending tag. 
See explanation (returned position is marked with `|'):

\begin{verbatim}
bla bla bla <MYTAG> bla bla internal text</MYTAG> bla bla
                   |
\end{verbatim}


\membersection{wxHtmlTag::GetEndPos1}\label{wxhtmltaggetendpos1}

\constfunc{int}{GetEndPos1}{\void}

Returns ending position of the text {\it between} this tag and paired
ending tag.
See explanation (returned position is marked with `|'):

\begin{verbatim}
bla bla bla <MYTAG> bla bla internal text</MYTAG> bla bla
                                        |
\end{verbatim}


\membersection{wxHtmlTag::GetEndPos2}\label{wxhtmltaggetendpos2}

\constfunc{int}{GetEndPos2}{\void}

Returns ending position 2 of the text {\it between} this tag and paired
ending tag.
See explanation (returned position is marked with `|'):

\begin{verbatim}
bla bla bla <MYTAG> bla bla internal text</MYTAG> bla bla
                                               |
\end{verbatim}

\membersection{wxHtmlTag::GetName}\label{wxhtmltaggetname}

\constfunc{wxString}{GetName}{\void}

Returns tag's name. The name is always in uppercase and it doesn't contain
'<' or '/' characters. (So the name of {\tt <FONT SIZE=+2>} tag is "FONT"
and name of {\tt </table>} is "TABLE")


\membersection{wxHtmlTag::GetParam}\label{wxhtmltaggetparam}

\constfunc{wxString}{GetParam}{\param{const wxString\& }{par}, \param{bool }{with\_commas = false}}

Returns the value of the parameter. You should check whether the
parameter exists or not (use \helpref{HasParam}{wxhtmltaghasparam}) first.

\wxheading{Parameters}

\docparam{par}{The parameter's name.}

\docparam{with\_commas}{true if you want to get commas as well. See example.}

\wxheading{Example}

\begin{verbatim}
...
/* you have wxHtmlTag variable tag which is equal to
   HTML tag <FONT SIZE=+2 COLOR="#0000FF"> */
dummy = tag.GetParam("SIZE");
   // dummy == "+2"
dummy = tag.GetParam("COLOR");
   // dummy == "#0000FF"
dummy = tag.GetParam("COLOR", true);
   // dummy == "\"#0000FF\"" -- see the difference!!
\end{verbatim}

\membersection{wxHtmlTag::GetParamAsColour}\label{wxhtmltaggetparamascolour}

\constfunc{bool}{GetParamAsColour}{\param{const wxString\& }{par}, \param{wxColour *}{clr}}

Interprets tag parameter {\it par} as colour specification and saves its value
into wxColour variable pointed by {\it clr}.

Returns true on success and false if {\it par} is not colour specification or
if the tag has no such parameter.

\membersection{wxHtmlTag::GetParamAsInt}\label{wxhtmltaggetparamasint}

\constfunc{bool}{GetParamAsInt}{\param{const wxString\& }{par}, \param{int *}{value}}

Interprets tag parameter {\it par} as an integer and saves its value
into int variable pointed by {\it value}.

Returns true on success and false if {\it par} is not an integer or
if the tag has no such parameter.

\membersection{wxHtmlTag::HasEnding}\label{wxhtmltaghasending}

\constfunc{bool}{HasEnding}{\void}

Returns true if this tag is paired with ending tag, false otherwise.

See the example of HTML document:

\begin{verbatim}
<html><body>
Hello<p>
How are you?
<p align=center>This is centered...</p>
Oops<br>Oooops!
</body></html>
\end{verbatim}

In this example tags HTML and BODY have ending tags, first P and BR 
doesn't have ending tag while the second P has. The third P tag (which
is ending itself) of course doesn't have ending tag.

\membersection{wxHtmlTag::HasParam}\label{wxhtmltaghasparam}

\constfunc{bool}{HasParam}{\param{const wxString\& }{par}}

Returns true if the tag has a parameter of the given name. 
Example : {\tt <FONT SIZE=+2 COLOR="\#FF00FF">} has two parameters named
"SIZE" and "COLOR".

\wxheading{Parameters}

\docparam{par}{the parameter you're looking for.}

\membersection{wxHtmlTag::ScanParam}\label{wxhtmltagscanparam}

\constfunc{wxString}{ScanParam}{\param{const wxString\& }{par}, \param{const wxChar *}{format}, \param{void *}{value}}

This method scans the given parameter. Usage is exactly the same as sscanf's 
usage except that you don't pass a string but a parameter name as the first
argument
and you can only retrieve one value (i.e. you can use only one "\%" element
in {\it format}).


\wxheading{Parameters}

\docparam{par}{The name of the tag you want to query}

\docparam{format}{scanf()-like format string.}

\docparam{value}{pointer to a variable to store the value in }

