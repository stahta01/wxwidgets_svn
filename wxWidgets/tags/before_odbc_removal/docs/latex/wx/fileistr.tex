\section{\class{wxFileInputStream}}\label{wxfileinputstream}

This class represents data read in from a file. There are actually
two such groups of classes: this one is based on \helpref{wxFile}{wxfile} 
whereas \helpref{wxFFileInputStream}{wxffileinputstream} is based in
the \helpref{wxFFile}{wxffile} class.

Note that \helpref{SeekI()}{wxinputstreamseeki} 
can seek beyond the end of the stream (file) and will thus not return 
{\it wxInvalidOffset} for that.

\wxheading{Derived from}

\helpref{wxInputStream}{wxinputstream}\\
\helpref{wxStreamBase}{wxstreambase}

\wxheading{Include files}

<wx/wfstream.h>

\wxheading{Library}

\helpref{wxBase}{librarieslist}

\wxheading{See also}

\helpref{wxBufferedInputStream}{wxbufferedinputstream}, \helpref{wxFileOutputStream}{wxfileoutputstream}, \helpref{wxFFileOutputStream}{wxffileoutputstream}

% ----------
% Members
% ----------
\latexignore{\rtfignore{\wxheading{Members}}}

\membersection{wxFileInputStream::wxFileInputStream}\label{wxfileinputstreamctor}

\func{}{wxFileInputStream}{\param{const wxString\&}{ ifileName}}

Opens the specified file using its {\it ifilename} name in read-only mode.

\func{}{wxFileInputStream}{\param{wxFile\&}{ file}}

Initializes a file stream in read-only mode using the file I/O object {\it file}.

\func{}{wxFileInputStream}{\param{int}{ fd}}

Initializes a file stream in read-only mode using the specified file descriptor.

\membersection{wxFileInputStream::\destruct{wxFileInputStream}}\label{wxfileinputstreamdtor}

\func{}{\destruct{wxFileInputStream}}{\void}

Destructor.

\membersection{wxFileInputStream::IsOk}\label{wxfileinputstreamisok}

\constfunc{bool}{IsOk}{\void}

Returns true if the stream is initialized and ready.

