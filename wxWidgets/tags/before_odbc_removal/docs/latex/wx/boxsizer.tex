\section{\class{wxBoxSizer}}\label{wxboxsizer}

The basic idea behind a box sizer is that windows will most often be laid out in rather
simple basic geometry, typically in a row or a column or several hierarchies of either.

For more information, please see \helpref{Programming with wxBoxSizer}{boxsizerprogramming}.

\wxheading{Derived from}

\helpref{wxSizer}{wxsizer}\\
\helpref{wxObject}{wxobject}

\wxheading{Include files}

<wx/sizer.h>

\wxheading{Library}

\helpref{wxCore}{librarieslist}

\wxheading{See also}

\helpref{wxSizer}{wxsizer}, \helpref{Sizer overview}{sizeroverview}


\latexignore{\rtfignore{\wxheading{Members}}}

\membersection{wxBoxSizer::wxBoxSizer}\label{wxboxsizerwxboxsizer}

\func{}{wxBoxSizer}{\param{int }{orient}}

Constructor for a wxBoxSizer. {\it orient} may be either of wxVERTICAL
or wxHORIZONTAL for creating either a column sizer or a row sizer.

\membersection{wxBoxSizer::RecalcSizes}\label{wxboxsizerrecalcsizes}

\func{void}{RecalcSizes}{\void}

Implements the calculation of a box sizer's dimensions and then sets
the size of its children (calling \helpref{wxWindow::SetSize}{wxwindowsetsize} 
if the child is a window). It is used internally only and must not be called
by the user (call Layout() if you want to resize). Documented for information.

\membersection{wxBoxSizer::CalcMin}\label{wxboxsizercalcmin}

\func{wxSize}{CalcMin}{\void}

Implements the calculation of a box sizer's minimal. It is used internally
only and must not be called by the user. Documented for information.

\membersection{wxBoxSizer::GetOrientation}\label{wxboxsizergetorientation}

\func{int}{GetOrientation}{\void}

Returns the orientation of the box sizer, either wxVERTICAL
or wxHORIZONTAL.

