\section{\class{wxMutexLocker}}\label{wxmutexlocker}

This is a small helper class to be used with \helpref{wxMutex}{wxmutex} 
objects. A wxMutexLocker acquires a mutex lock in the constructor and releases
(or unlocks) the mutex in the destructor making it much more difficult to
forget to release a mutex (which, in general, will promptly lead to the serious
problems). See \helpref{wxMutex}{wxmutex} for an example of wxMutexLocker
usage.

\wxheading{Derived from}

None.

\wxheading{Include files}

<wx/thread.h>

\wxheading{See also}

\helpref{wxMutex}{wxmutex}, \helpref{wxCriticalSectionLocker}{wxcriticalsectionlocker}

\latexignore{\rtfignore{\wxheading{Members}}}

\membersection{wxMutexLocker::wxMutexLocker}\label{wxmutexlockerctor}

\func{}{wxMutexLocker}{\param{wxMutex *}{mutex}}

Constructs a wxMutexLocker object associated with mutex which must be non-NULL
and locks it. Call \helpref{IsLocked}{wxmutexlockerisok} to check if the mutex was
successfully locked.

\membersection{wxMutexLocker::\destruct{wxMutexLocker}}\label{wxmutexlockerdtor}

\func{}{\destruct{wxMutexLocker}}{\void}

Destuctor releases the mutex if it was successfully acquired in the ctor.

\membersection{wxMutexLocker::IsOk}\label{wxmutexlockerisok}

\constfunc{bool}{IsOk}{\void}

Returns TRUE if mutex was acquired in the constructor, FALSE otherwise.

