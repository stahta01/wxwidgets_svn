\section{\class{wxEvent}}\label{wxevent}

An event is a structure holding information about an event passed to a
callback or member function. {\bf wxEvent} used to be a multipurpose
event object, and is an abstract base class for other event classes (see below).

\wxheading{Derived from}

\helpref{wxObject}{wxobject}

\wxheading{Include files}

<wx/event.h>

\wxheading{See also}

\helpref{wxCommandEvent}{wxcommandevent},\rtfsp
\helpref{wxMouseEvent}{wxmouseevent}

\latexignore{\rtfignore{\wxheading{Members}}}

\membersection{wxEvent::wxEvent}

\func{}{wxEvent}{\param{int }{id = 0}}

Constructor. Should not need to be used directly by an application.

\membersection{wxEvent::m\_eventObject}

\member{wxObject*}{m\_eventObject}

The object (usually a window) that the event was generated from,
or should be sent to.

\membersection{wxEvent::m\_eventType}

\member{WXTYPE}{m\_eventType}

The type of the event, such as wxEVENT\_TYPE\_BUTTON\_COMMAND.

\membersection{wxEvent::m\_id}

\member{int}{m\_id}

Identifier for the window.

\membersection{wxEvent::m\_skipped}

\member{bool}{m\_skipped}

Set to TRUE by {\bf Skip} if this event should be skipped.

\membersection{wxEvent::m\_timeStamp}

\member{long}{m\_timeStamp}

Timestamp for this event.

\membersection{wxEvent::Clone}\label{wxeventclone}

\func{virtual wxEvent*}{Clone}{\void} const

Returns a copy of the event.

Any event that is posted to the wxWindows event system for later action (via
\helpref{wxEvtHandler::AddPendingEvent}{wxevthandleraddpendingevent} or
\helpref{wxPostEvent}{wxpostevent}) must implement this method. All wxWindows
events fully implement this method, but any derived events implemented by the
user should also implement this method just in case they (or some event
derived from them) are ever posted.

All wxWindows events implement a copy constructor, so the easiest way of
implementing the Clone function is to implement a copy constructor for
a new event (call it MyEvent) and then define the Clone function like this:
\begin{verbatim}
    wxEvent *Clone(void) const { return new MyEvent(*this); }
\end{verbatim}

\membersection{wxEvent::GetEventObject}

\func{wxObject*}{GetEventObject}{\void}

Returns the object associated with the
event, if any.

\membersection{wxEvent::GetEventType}

\func{WXTYPE}{GetEventType}{\void}

Returns the identifier of the given event type,
such as wxEVENT\_TYPE\_BUTTON\_COMMAND.

\membersection{wxEvent::GetId}

\func{int}{GetId}{\void}

Returns the identifier associated with this event, such as a button command id.

\membersection{wxEvent::GetObjectType}

\func{WXTYPE}{GetObjectType}{\void}

Returns the type of the object associated with the
event, such as wxTYPE\_BUTTON.

\membersection{wxEvent::GetSkipped}

\func{bool}{GetSkipped}{\void}

Returns TRUE if the event handler should be skipped, FALSE otherwise.

\membersection{wxEvent::GetTimestamp}

\func{long}{GetTimestamp}{\void}

Gets the timestamp for the event.

\membersection{wxEvent::SetEventObject}

\func{void}{SetEventObject}{\param{wxObject* }{object}}

Sets the originating object.

\membersection{wxEvent::SetEventType}

\func{void}{SetEventType}{\param{WXTYPE }{typ}}

Sets the event type.

\membersection{wxEvent::SetId}

\func{void}{SetId}{\param{int}{ id}}

Sets the identifier associated with this event, such as a button command id.

\membersection{wxEvent::SetTimestamp}

\func{void}{SetTimestamp}{\param{long }{timeStamp}}

Sets the timestamp for the event.

Sets the originating object.

\membersection{wxEvent::Skip}\label{wxeventskip}

\func{void}{Skip}{\param{bool}{ skip = TRUE}}

Called by an event handler to tell the event system that the
event handler should be skipped, and the next valid handler used
instead.

