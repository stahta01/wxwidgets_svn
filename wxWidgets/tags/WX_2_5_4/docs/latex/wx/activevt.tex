\section{\class{wxActivateEvent}}\label{wxactivateevent}

An activate event is sent when a window or application is being activated
or deactivated.

\wxheading{Derived from}

\helpref{wxEvent}{wxevent}\\
\helpref{wxObject}{wxobject}

\wxheading{Include files}

<wx/event.h>

\wxheading{Event table macros}

To process an activate event, use these event handler macros to direct input to a member
function that takes a wxActivateEvent argument.

\twocolwidtha{7cm}
\begin{twocollist}\itemsep=0pt
\twocolitem{{\bf EVT\_ACTIVATE(func)}}{Process a wxEVT\_ACTIVATE event.}
\twocolitem{{\bf EVT\_ACTIVATE\_APP(func)}}{Process a wxEVT\_ACTIVATE\_APP event.}
\end{twocollist}%

\wxheading{Remarks}

A top-level window (a dialog or frame) receives an activate event when is
being activated or deactivated. This is indicated visually by the title
bar changing colour, and a subwindow gaining the keyboard focus.

An application is activated or deactivated when one of its frames becomes activated,
or a frame becomes inactivate resulting in all application frames being inactive. (Windows only)

Please note that usually you should call \helpref{event.Skip()}{wxeventskip} in
your handlers for these events as not doing so can result in strange effects,
especially on Mac platform.

\wxheading{See also}

\helpref{Event handling overview}{eventhandlingoverview},\rtfsp
\helpref{wxApp::IsActive}{wxappisactive}

\latexignore{\rtfignore{\wxheading{Members}}}

\membersection{wxActivateEvent::wxActivateEvent}\label{wxactivateeventctor}

\func{}{wxActivateEvent}{\param{WXTYPE }{eventType = 0}, \param{bool}{ active = true}, \param{int }{id = 0}}

Constructor.

\membersection{wxActivateEvent::GetActive}\label{wxactivateeventgetactive}

\constfunc{bool}{GetActive}{\void}

Returns true if the application or window is being activated, false otherwise.

