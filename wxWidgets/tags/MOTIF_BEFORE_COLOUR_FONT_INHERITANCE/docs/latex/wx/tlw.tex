%%%%%%%%%%%%%%%%%%%%%%%%%%%%%%%%%%%%%%%%%%%%%%%%%%%%%%%%%%%%%%%%%%%%%%%%%%%%%%%
%% Name:        tlw.tex
%% Purpose:     wxTopLevelWindow documentation
%% Author:      Vadim Zeitlin
%% Modified by:
%% Created:     2004-09-07 (partly extracted from frame.tex)
%% RCS-ID:      $Id$
%% Copyright:   (c) 2004 Vadim Zeitlin
%% License:     wxWindows license
%%%%%%%%%%%%%%%%%%%%%%%%%%%%%%%%%%%%%%%%%%%%%%%%%%%%%%%%%%%%%%%%%%%%%%%%%%%%%%%

\section{\class{wxTopLevelWindow}}\label{wxtoplevelwindow}

wxTopLevelWindow is a common base class for \helpref{wxDialog}{wxdialog} and
\helpref{wxFrame}{wxframe}. It is an abstract base class meaning that you never
work with objects of this class directly, but all of its methods are also
applicable for the two classes above.

\wxheading{Derived from}

\helpref{wxWindow}{wxwindow}\\
\helpref{wxEvtHandler}{wxevthandler}\\
\helpref{wxObject}{wxobject}

\wxheading{Include files}

<wx/toplevel.h>


\latexignore{\rtfignore{\wxheading{Members}}}

\membersection{wxTopLevelWindow::CanSetTransparent}\label{wxtoplevelwindowcansettransparent}

\func{virtual bool}{CanSetTransparent}{\void}

Returns \true if the platform supports making the window translucent.

\wxheading{See also}

\helpref{wxTopLevelWindow::SetTransparent}{wxtoplevelwindowsettransparent}


\membersection{wxTopLevelWindow::EnableCloseButton}\label{wxtoplevelenableclosebutton}

\func{bool}{EnableCloseButton}{\param{bool}{ enable = true}}

Enables or disables the Close button (most often in the right
upper corner of a dialog) and the Close entry of the system
menu (most often in the left upper corner of the dialog).
Currently only implemented for wxMSW and wxGTK. Returns
true if operation was successful. This may be wrong on
X11 (including GTK+) where the window manager may not support
this operation and there is no way to find out.

\membersection{wxTopLevelWindow::GetDefaultItem}\label{wxtoplevelwindowgetdefaultitem}

\constfunc{wxWindow *}{GetDefaultItem}{\void}

Returns a pointer to the button which is the default for this window, or \NULL.
The default button is the one activated by pressing the Enter key.


\membersection{wxTopLevelWindow::GetIcon}\label{wxtoplevelwindowgeticon}

\constfunc{const wxIcon\&}{GetIcon}{\void}

Returns the standard icon of the window. The icon will be invalid if it hadn't
been previously set by \helpref{SetIcon}{wxtoplevelwindowseticon}.

\wxheading{See also}

\helpref{GetIcons}{wxtoplevelwindowgeticons}


\membersection{wxTopLevelWindow::GetIcons}\label{wxtoplevelwindowgeticons}

\constfunc{const wxIconBundle\&}{GetIcons}{\void}

Returns all icons associated with the window, there will be none of them if
neither \helpref{SetIcon}{wxtoplevelwindowseticon} nor
\helpref{SetIcons}{wxtoplevelwindowseticons} had been called before.

Use \helpref{GetIcon}{wxtoplevelwindowgeticon} to get the main icon of the
window.

\wxheading{See also}

\helpref{wxIconBundle}{wxiconbundle}


\membersection{wxTopLevelWindow::GetTitle}\label{wxtoplevelwindowgettitle}

\constfunc{wxString}{GetTitle}{\void}

Gets a string containing the window title.

\wxheading{See also}

\helpref{wxTopLevelWindow::SetTitle}{wxtoplevelwindowsettitle}


\membersection{wxTopLevelWindow::HandleSettingChange}\label{wxtoplevelwindowhandlesettingchange}

\func{virtual bool}{HandleSettingChange}{\param{WXWPARAM}{ wParam}, \param{WXLPARAM}{ lParam}}

Unique to the wxWinCE port. Responds to showing/hiding SIP (soft input panel) area and resize
window accordingly. Override this if you want to avoid resizing or do additional
operations.


\membersection{wxTopLevelWindow::IsActive}\label{wxtoplevelwindowisactive}

\constfunc{bool}{IsActive}{\void}

Returns \true if this window is currently active, i.e. if the user is currently
working with it.


\membersection{wxTopLevelWindow::IsAlwaysMaximized}\label{wxtoplevelwindowisalwaysmaximized}

\constfunc{virtual bool}{IsAlwaysMaximized}{\void}

Returns \true if this window is expected to be always maximized, either due to platform policy
or due to local policy regarding particular class.


\membersection{wxTopLevelWindow::Iconize}\label{wxtoplevelwindowiconize}

\func{void}{Iconize}{\param{bool}{ iconize}}

Iconizes or restores the window.

\wxheading{Parameters}

\docparam{iconize}{If \true, iconizes the window; if \false, shows and restores it.}

\wxheading{See also}

\helpref{wxTopLevelWindow::IsIconized}{wxtoplevelwindowisiconized}, \helpref{wxTopLevelWindow::Maximize}{wxtoplevelwindowmaximize}.


\membersection{wxTopLevelWindow::IsFullScreen}\label{wxtoplevelwindowisfullscreen}

\func{bool}{IsFullScreen}{\void}

Returns \true if the window is in fullscreen mode.

\wxheading{See also}

\helpref{wxTopLevelWindow::ShowFullScreen}{wxtoplevelwindowshowfullscreen}


\membersection{wxTopLevelWindow::IsIconized}\label{wxtoplevelwindowisiconized}

\constfunc{bool}{IsIconized}{\void}

Returns \true if the window is iconized.


\membersection{wxTopLevelWindow::IsMaximized}\label{wxtoplevelwindowismaximized}

\constfunc{bool}{IsMaximized}{\void}

Returns \true if the window is maximized.


\membersection{wxTopLevelWindow::IsUsingNativeDecorations}\label{wxtoplevelwindowisusingnativedecorations}

\constfunc{bool}{IsUsingNativeDecorations}{\void}

\bftt{This method is specific to wxUniversal port}

Returns \true if this window is using native decorations, \false if we draw
them ourselves.

\wxheading{See also}

\helpref{UseNativeDecorations}{wxtoplevelwindowusenativedecorations},\\
\helpref{UseNativeDecorationsByDefault}{wxtoplevelwindowusenativedecorationsbydefault}


\membersection{wxTopLevelWindow::Maximize}\label{wxtoplevelwindowmaximize}

\func{void}{Maximize}{\param{bool }{maximize}}

Maximizes or restores the window.

\wxheading{Parameters}

\docparam{maximize}{If \true, maximizes the window, otherwise it restores it.}

\wxheading{See also}

\helpref{wxTopLevelWindow::Iconize}{wxtoplevelwindowiconize}


\membersection{wxTopLevelWindow::RequestUserAttention}\label{wxtoplevelwindowrequestuserattention}

\func{void}{RequestUserAttention}{\param{int }{flags = wxUSER\_ATTENTION\_INFO}}

Use a system-dependent way to attract users attention to the window when it is
in background.

\arg{flags} may have the value of either \texttt{wxUSER\_ATTENTION\_INFO}
(default) or \texttt{wxUSER\_ATTENTION\_ERROR} which results in a more drastic
action. When in doubt, use the default value.

Note that this function should normally be only used when the application is
not already in foreground.

This function is currently implemented for Win32 where it flashes the
window icon in the taskbar, and for wxGTK with task bars supporting it.


\membersection{wxTopLevelWindow::SetDefaultItem}\label{wxtoplevelwindowsetdefaultitem}

\func{void}{SetDefaultItem}{\param{wxWindow }{*win}}

Changes the default item for the panel, usually \arg{win} is a button.

\wxheading{See also}

\helpref{GetDefaultItem}{wxtoplevelwindowgetdefaultitem}


\membersection{wxTopLevelWindow::SetIcon}\label{wxtoplevelwindowseticon}

\func{void}{SetIcon}{\param{const wxIcon\& }{icon}}

Sets the icon for this window.

\wxheading{Parameters}

\docparam{icon}{The icon to associate with this window.}

\wxheading{Remarks}

The window takes a `copy' of {\it icon}, but since it uses reference
counting, the copy is very quick. It is safe to delete {\it icon} after
calling this function.

See also \helpref{wxIcon}{wxicon}.


\membersection{wxTopLevelWindow::SetIcons}\label{wxtoplevelwindowseticons}

\func{void}{SetIcons}{\param{const wxIconBundle\& }{icons}}

Sets several icons of different sizes for this window: this allows to use
different icons for different situations (e.g. task switching bar, taskbar,
window title bar) instead of scaling, with possibly bad looking results, the
only icon set by \helpref{SetIcon}{wxtoplevelwindowseticon}.

\wxheading{Parameters}

\docparam{icons}{The icons to associate with this window.}

\wxheading{See also}

\helpref{wxIconBundle}{wxiconbundle}.


\membersection{wxTopLevelWindow::SetLeftMenu}\label{wxtoplevelwindowsetleftmenu}

\func{void}{SetLeftMenu}{\param{int}{ id = wxID\_ANY}, \param{const wxString\&}{ label = wxEmptyString}, \param{wxMenu *}{ subMenu = NULL}}

Sets action or menu activated by pressing left hardware button on the smart phones.
Unavailable on full keyboard machines.

\wxheading{Parameters}

\docparam{id}{Identifier for this button.}

\docparam{label}{Text to be displayed on the screen area dedicated to this hardware button.}

\docparam{subMenu}{The menu to be opened after pressing this hardware button.}

\wxheading{See also}

\helpref{wxTopLevelWindow::SetRightMenu}{wxtoplevelwindowsetrightmenu}.


\membersection{wxTopLevelWindow::SetMaxSize}\label{wxtoplevelwindowsetmaxsize}

\func{void}{SetMaxSize}{\param{const wxSize\& }{size}}

A simpler interface for setting the size hints than
\helpref{SetSizeHints}{wxtoplevelwindowsetsizehints}.

\membersection{wxTopLevelWindow::SetMinSize}\label{wxtoplevelwindowsetminsize}

\func{void}{SetMinSize}{\param{const wxSize\& }{size}}

A simpler interface for setting the size hints than
\helpref{SetSizeHints}{wxtoplevelwindowsetsizehints}.

\membersection{wxTopLevelWindow::SetSizeHints}\label{wxtoplevelwindowsetsizehints}

\func{virtual void}{SetSizeHints}{\param{int}{ minW}, \param{int}{ minH}, \param{int}{ maxW=-1}, \param{int}{ maxH=-1},
 \param{int}{ incW=-1}, \param{int}{ incH=-1}}

\func{void}{SetSizeHints}{\param{const wxSize\&}{ minSize},
\param{const wxSize\&}{ maxSize=wxDefaultSize}, \param{const wxSize\&}{ incSize=wxDefaultSize}}

Allows specification of minimum and maximum window sizes, and window size increments.
If a pair of values is not set (or set to -1), the default values will be used.

\docparam{incW}{Specifies the increment for sizing the width (GTK/Motif/Xt only).}

\docparam{incH}{Specifies the increment for sizing the height (GTK/Motif/Xt only).}

\docparam{incSize}{Increment size (GTK/Motif/Xt only).}

\wxheading{Remarks}

If this function is called, the user will not be able to size the window outside
the given bounds. The resizing increments are only significant under GTK, Motif or Xt.


\membersection{wxTopLevelWindow::SetRightMenu}\label{wxtoplevelwindowsetrightmenu}

\func{void}{SetRightMenu}{\param{int}{ id = wxID\_ANY}, \param{const wxString\&}{ label = wxEmptyString}, \param{wxMenu *}{ subMenu = NULL}}

Sets action or menu activated by pressing right hardware button on the smart phones.
Unavailable on full keyboard machines.

\wxheading{Parameters}

\docparam{id}{Identifier for this button.}

\docparam{label}{Text to be displayed on the screen area dedicated to this hardware button.}

\docparam{subMenu}{The menu to be opened after pressing this hardware button.}

\wxheading{See also}

\helpref{wxTopLevelWindow::SetLeftMenu}{wxtoplevelwindowsetleftmenu}.


\membersection{wxTopLevelWindow::SetShape}\label{wxtoplevelwindowsetshape}

\func{bool}{SetShape}{\param{const wxRegion\&}{ region}}

If the platform supports it, sets the shape of the window to that
depicted by {\it region}.  The system will not display or
respond to any mouse event for the pixels that lie outside of the
region.  To reset the window to the normal rectangular shape simply
call {\it SetShape} again with an empty region.  Returns true if the
operation is successful.


\membersection{wxTopLevelWindow::SetTitle}\label{wxtoplevelwindowsettitle}

\func{virtual void}{SetTitle}{\param{const wxString\& }{ title}}

Sets the window title.

\wxheading{Parameters}

\docparam{title}{The window title.}

\wxheading{See also}

\helpref{wxTopLevelWindow::GetTitle}{wxtoplevelwindowgettitle}


\membersection{wxTopLevelWindow::SetTransparent}\label{wxtoplevelwindowsettransparent}

\func{virtual bool}{SetTransparent}{\param{int }{ alpha}}

If the platform supports it will set the window to be translucent

\wxheading{Parameters}

\docparam{alpha}{Determines how opaque or transparent the window will
  be, if the platform supports the opreration.  A value of 0 sets the
  window to be fully transparent, and a value of 255 sets the window
  to be fully opaque.}

Returns \true if the transparency was successfully changed.



\membersection{wxTopLevelWindow::ShouldPreventAppExit}\label{wxtoplevelwindowshouldpreventappexit}

\constfunc{virtual bool}{ShouldPreventAppExit}{\void}

This virtual function is not meant to be called directly but can be overridden
to return \false (it returns \true by default) to allow the application to
close even if this, presumably not very important, window is still opened.
By default, the application stays alive as long as there are any open top level
windows.


\membersection{wxTopLevelWindow::ShowFullScreen}\label{wxtoplevelwindowshowfullscreen}

\func{bool}{ShowFullScreen}{\param{bool}{ show}, \param{long}{ style = wxFULLSCREEN\_ALL}}

Depending on the value of {\it show} parameter the window is either shown full
screen or restored to its normal state. {\it style} is a bit list containing
some or all of the following values, which indicate what elements of the window
to hide in full-screen mode:

\begin{itemize}\itemsep=0pt
\item wxFULLSCREEN\_NOMENUBAR
\item wxFULLSCREEN\_NOTOOLBAR
\item wxFULLSCREEN\_NOSTATUSBAR
\item wxFULLSCREEN\_NOBORDER
\item wxFULLSCREEN\_NOCAPTION
\item wxFULLSCREEN\_ALL (all of the above)
\end{itemize}

This function has not been tested with MDI frames.

Note that showing a window full screen also actually
\helpref{Show()s}{wxwindowshow} if it hadn't been shown yet.

\wxheading{See also}

\helpref{wxTopLevelWindow::IsFullScreen}{wxtoplevelwindowisfullscreen}


\membersection{wxTopLevelWindow::UseNativeDecorations}\label{wxtoplevelwindowusenativedecorations}

\func{void}{UseNativeDecorations}{\param{bool }{native = \true}}

\bftt{This method is specific to wxUniversal port}

Use native or custom-drawn decorations for this window only. Notice that to
have any effect this method must be called before really creating the window,
i.e. two step creation must be used:
\begin{verbatim}
    MyFrame *frame = new MyFrame;           // use default ctor
    frame->UseNativeDecorations(false);     // change from default "true"
    frame->Create(parent, title, ...);      // really create the frame
\end{verbatim}

\wxheading{See also}

\helpref{UseNativeDecorationsByDefault}{wxtoplevelwindowusenativedecorationsbydefault},\\
\helpref{IsUsingNativeDecorations}{wxtoplevelwindowisusingnativedecorations}


\membersection{wxTopLevelWindow::UseNativeDecorationsByDefault}\label{wxtoplevelwindowusenativedecorationsbydefault}

\func{void}{UseNativeDecorationsByDefault}{\param{bool }{native = \true}}

\bftt{This method is specific to wxUniversal port}

Top level windows in wxUniversal port can use either system-provided window
decorations (i.e. title bar and various icons, buttons and menus in it) or draw
the decorations themselves. By default the system decorations are used if they
are available, but this method can be called with \arg{native} set to \false to
change this for all windows created after this point.

Also note that if \texttt{WXDECOR} environment variable is set, then custom
decorations are used by default and so it may make sense to call this method
with default argument if the application can't use custom decorations at all
for some reason.

\wxheading{See also}

\helpref{UseNativeDecorations}{wxtoplevelwindowusenativedecorations}

