%%%%%%%%%%%%%%%%%%%%%%%%%%%%%%%%%%%%%%%%%%%%%%%%%%%%%%%%%%%%%%%%%%%%%%%%%%%%%%%
%% Name:        renderer.tex
%% Purpose:     wxRenderer and wxRendererNative documentation
%% Author:      Vadim Zeitlin
%% Modified by:
%% Created:     06.08.03
%% RCS-ID:      $Id$
%% Copyright:   (c) 2003 Vadim Zeitlin <vadim@wxwindows.org>
%% License:     wxWindows license
%%%%%%%%%%%%%%%%%%%%%%%%%%%%%%%%%%%%%%%%%%%%%%%%%%%%%%%%%%%%%%%%%%%%%%%%%%%%%%%

\section{\class{wxRendererNative}}\label{wxrenderernative}

First, a brief introduction to wxRenderer and why it is needed.

Usually wxWindows uses the underlying low level GUI system to draw all the
controls -- this is what we mean when we say that it is a ``native'' framework.
However not all controls exist under all (or even any) platforms and in this
case wxWindows provides a default, generic, implementation of them written in
wxWindows itself.

These controls don't have the native appearance if only the standard
line drawing and other graphics primitives are used, because the native
appearance is different under different platforms while the lines are always
drawn in the same way.

This is why we have renderers: wxRenderer is a class which virtualizes the
drawing, i.e. it abstracts the drawing operations and allows you to draw say, a
button, without caring about exactly how this is done. Of course, as we
can draw the button differently in different renderers, this also allows us to
emulate the native look and feel.

So the renderers work by exposing a large set of high-level drawing functions
which are used by the generic controls. There is always a default global
renderer but it may be changed or extended by the user, see 
\helpref{Render sample}{samplerender}.

All drawing functions take some standard parameters:
\begin{itemize}
    \item \arg{win} is the window being drawn. It is normally not used and when
          it is it should only be used as a generic \helpref{wxWindow}{wxwindow} 
          (in order to get its low level handle, for example), but you should
          \emph{not} assume that it is of some given type as the same renderer
          function may be reused for drawing different kinds of control.
    \item \arg{dc} is the \helpref{wxDC}{wxdc} to draw on. Only this device
          context should be used for drawing. It is not necessary to restore
          pens and brushes for it on function exit but, on the other hand, you
          shouldn't assume that it is in any specific state on function entry:
          the rendering functions should always prepare it.
    \item \arg{rect} the bounding rectangle for the element to be drawn.
    \item \arg{flags} the optional flags (none by default) which can be a
          combination of the \texttt{wxCONTROL\_XXX} constants below.
\end{itemize}

\wxheading{Constants}

The following rendering flags are defined:
\begin{verbatim}
enum
{
    wxCONTROL_DISABLED   = 0x00000001,  // control is disabled
    wxCONTROL_FOCUSED    = 0x00000002,  // currently has keyboard focus
    wxCONTROL_PRESSED    = 0x00000004,  // (button) is pressed
    wxCONTROL_ISDEFAULT  = 0x00000008,  // only applies to the buttons
    wxCONTROL_ISSUBMENU  = wxCONTROL_ISDEFAULT, // only for menu items
    wxCONTROL_EXPANDED   = wxCONTROL_ISDEFAULT, // only for the tree items
    wxCONTROL_CURRENT    = 0x00000010,  // mouse is currently over the control
    wxCONTROL_SELECTED   = 0x00000020,  // selected item in e.g. listbox
    wxCONTROL_CHECKED    = 0x00000040,  // (check/radio button) is checked
    wxCONTROL_CHECKABLE  = 0x00000080   // (menu) item can be checked
};
\end{verbatim}


\wxheading{Derived from}

No base class

\wxheading{Include files}

<wx/renderer.h>


\latexignore{\rtfignore{\wxheading{Members}}}


\membersection{wxRendererNative::\destruct{wxRendererNative}}\label{wxrenderernativedtor}

\func{}{\destruct{wxRendererNative}}{\void}

Virtual destructor as for any base class.


\membersection{wxRendererNative::DrawHeaderButton}\label{wxrenderernativedrawheaderbutton}

\func{void}{DrawHeaderButton}{\param{wxWindow* }{win}, \param{wxDC\& }{dc}, \param{const wxRect\& }{rect}, \param{int }{flags = 0}}

Draw the header control button (used by \helpref{wxListCtrl}{wxlistctrl}).


\membersection{wxRendererNative::DrawSplitterBorder}\label{wxrenderernativedrawsplitterborder}

\func{void}{DrawSplitterBorder}{\param{wxWindow* }{win}, \param{wxDC\& }{dc}, \param{const wxRect\& }{rect}, \param{int }{flags = 0}}

Draw the border for sash window: this border must be such that the sash
drawn by \helpref{DrawSash}{wxrenderernativedrawsplittersash} blends into it
well.


\membersection{wxRendererNative::DrawSplitterSash}\label{wxrenderernativedrawsplittersash}

\func{void}{DrawSplitterSash}{\param{wxWindow* }{win}, \param{wxDC\& }{dc}, \param{const wxSize\& }{size}, \param{wxCoord }{position}, \param{wxOrientation }{orient}, \param{int }{flags = 0}}

Draw a sash. The \arg{orient} parameter defines whether the sash should be
vertical or horizontal and how should the \arg{position} be interpreted.


\membersection{wxRendererNative::DrawTreeItemButton}\label{wxrenderernativedrawtreeitembutton}

\func{void}{DrawTreeItemButton}{\param{wxWindow* }{win}, \param{wxDC\& }{dc}, \param{const wxRect\& }{rect}, \param{int }{flags = 0}}

Draw the expanded/collapsed icon for a tree control item. To draw an expanded
button the \arg{flags} parameter must contain {\tt wxCONTROL\_EXPANDED} bit.


\membersection{wxRendererNative::Get}\label{wxrenderernativeget}

\func{wxRendererNative\&}{Get}{\void}

Return the currently used renderer.


\membersection{wxRendererNative::GetDefault}\label{wxrenderernativegetdefault}

\func{wxRendererNative\&}{GetDefault}{\void}

Return the default (native) implementation for this platform -- this is also
the one used by default but this may be changed by calling 
\helpref{Set}{wxrenderernativeset} in which case the return value of this
method may be different from the return value of \helpref{Get}{wxrenderernativeget}.


\membersection{wxRendererNative::GetGeneric}\label{wxrenderernativegetgeneric}

\func{wxRendererNative\&}{GetGeneric}{\void}

Return the generic implementation of the renderer. Under some platforms, this
is the default renderer implementation, others have platform-specific default
renderer which can be retrieved by calling \helpref{GetDefault}{wxrenderernativegetdefault}.


\membersection{wxRendererNative::GetSplitterParams}\label{wxrenderernativegetsplitterparams}

\func{wxSplitterRenderParams}{GetSplitterParams}{\param{const wxWindow* }{win}}

Get the splitter parameters, see 
\helpref{wxSplitterRenderParams}{wxsplitterrenderparams}.


\membersection{wxRendererNative::GetVersion}\label{wxrenderernativegetversion}

\constfunc{wxRendererVersion}{GetVersion}{\void}

This function is used for version checking: \helpref{Load}{wxrenderernativeload} 
refuses to load any shared libraries implementing an older or incompatible
version.

The implementation of this method is always the same in all renderers (simply
construct \helpref{wxRendererVersion}{wxrendererversion} using the 
{\tt wxRendererVersion::Current\_XXX} values), but it has to be in the derived,
not base, class, to detect mismatches between the renderers versions and so you
have to implement it anew in all renderers.


\membersection{wxRendererNative::Load}\label{wxrenderernativeload}

\func{wxRendererNative*}{Load}{\param{const wxString\& }{name}}

Load the renderer from the specified DLL, the returned pointer must be
deleted by caller if not {\tt NULL} when it is not used any more.

The \arg{name} should be just the base name of the renderer and not the full
name of the DLL file which is constructed differently (using 
\helpref{wxDynamicLibrary::CanonicalizePluginName}{wxdynamiclibrarycanonicalizepluginname}) 
on different systems.


\membersection{wxRendererNative::Set}\label{wxrenderernativeset}

\func{wxRendererNative*}{Set}{\param{wxRendererNative* }{renderer}}

Set the renderer to use, passing {\tt NULL} reverts to using the default
renderer (the global renderer must always exist).

Return the previous renderer used with Set() or {\tt NULL} if none.

