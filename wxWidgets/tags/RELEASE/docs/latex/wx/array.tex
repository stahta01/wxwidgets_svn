\section{\class{wxArray}}\label{wxarray}

This section describes the so called {\it dynamic arrays}. This is a C
array-like data structure i.e. the member access time is constant (and not
linear according to the number of container elements as for linked lists). However, these
arrays are dynamic in the sense that they will automatically allocate more
memory if there is not enough of it for adding a new element. They also perform
range checking on the index values but in debug mode only, so please be sure to
compile your application in debug mode to use it (see \helpref{debugging overview}{debuggingoverview} for
details). So, unlike the arrays in some other
languages, attempt to access an element beyond the arrays bound doesn't
automatically expand the array but provokes an assertion failure instead in
debug build and does nothing (except possibly crashing your program) in the
release build.

The array classes were designed to be reasonably efficient, both in terms of
run-time speed and memory consumption and the executable size. The speed of
array item access is, of course, constant (independent of the number of elements)
making them much more efficient than linked lists (\helpref{wxList}{wxlist}).
Adding items to the arrays is also implemented in more or less constant time -
but the price is preallocating the memory in advance. In the \helpref{memory management}{wxarraymemorymanagement} section
you may find some useful hints about optimizing wxArray memory usage. As for executable size, all
wxArray functions are inline, so they do not take {\it any space at all}.

wxWidgets has three different kinds of array. All of them derive from
wxBaseArray class which works with untyped data and can not be used directly.
The standard macros WX\_DEFINE\_ARRAY(), WX\_DEFINE\_SORTED\_ARRAY() and
WX\_DEFINE\_OBJARRAY() are used to define a new class deriving from it. The
classes declared will be called in this documentation wxArray, wxSortedArray and
wxObjArray but you should keep in mind that no classes with such names actually
exist, each time you use one of WX\_DEFINE\_XXXARRAY macro you define a class
with a new name. In fact, these names are "template" names and each usage of one
of the macros mentioned above creates a template specialization for the given
element type.

wxArray is suitable for storing integer types and pointers which it does not
treat as objects in any way, i.e. the element pointed to by the pointer is not
deleted when the element is removed from the array. It should be noted that
all of wxArray's functions are inline, so it costs strictly nothing to define as
many array types as you want (either in terms of the executable size or the
speed) as long as at least one of them is defined and this is always the case
because wxArrays are used by wxWidgets internally. This class has one serious
limitation: it can only be used for storing integral types (bool, char, short,
int, long and their unsigned variants) or pointers (of any kind). An attempt
to use with objects of sizeof() greater than sizeof(long) will provoke a
runtime assertion failure, however declaring a wxArray of floats will not (on
the machines where sizeof(float) <= sizeof(long)), yet it will {\bf not} work,
please use wxObjArray for storing floats and doubles (NB: a more efficient
wxArrayDouble class is scheduled for the next release of wxWidgets).

wxSortedArray is a wxArray variant which should be used when searching in the
array is a frequently used operation. It requires you to define an additional
function for comparing two elements of the array element type and always stores
its items in the sorted order (according to this function). Thus, it is 
 \helpref{Index()}{wxarrayindex} function execution time is $O(log(N))$ instead of
$O(N)$ for the usual arrays but the \helpref{Add()}{wxarrayadd} method is
slower: it is $O(log(N))$ instead of constant time (neglecting time spent in
memory allocation routine). However, in a usual situation elements are added to
an array much less often than searched inside it, so wxSortedArray may lead to
huge performance improvements compared to wxArray. Finally, it should be
noticed that, as wxArray, wxSortedArray can be only used for storing integral
types or pointers.

wxObjArray class treats its elements like "objects". It may delete them when
they are removed from the array (invoking the correct destructor) and copies
them using the objects copy constructor. In order to implement this behaviour
the definition of the wxObjArray arrays is split in two parts: first, you should
declare the new wxObjArray class using WX\_DECLARE\_OBJARRAY() macro and then
you must include the file defining the implementation of template type:
<wx/arrimpl.cpp> and define the array class with WX\_DEFINE\_OBJARRAY() macro
from a point where the full (as opposed to `forward') declaration of the array
elements class is in scope. As it probably sounds very complicated here is an
example:

\begin{verbatim}
#include <wx/dynarray.h>

// we must forward declare the array because it is used inside the class
// declaration
class MyDirectory;
class MyFile;

// this defines two new types: ArrayOfDirectories and ArrayOfFiles which can be
// now used as shown below
WX_DECLARE_OBJARRAY(MyDirectory, ArrayOfDirectories);
WX_DECLARE_OBJARRAY(MyFile,      ArrayOfFiles);

class MyDirectory
{
...
    ArrayOfDirectories m_subdirectories; // all subdirectories
    ArrayOfFiles       m_files;          // all files in this directory
};

...

// now that we have MyDirectory declaration in scope we may finish the
// definition of ArrayOfDirectories -- note that this expands into some C++
// code and so should only be compiled once (i.e., don't put this in the
// header, but into a source file or you will get linking errors)
#include <wx/arrimpl.cpp> // this is a magic incantation which must be done!
WX_DEFINE_OBJARRAY(ArrayOfDirectories);

// that's all!
\end{verbatim}

It is not as elegant as writing

\begin{verbatim} 
typedef std::vector<MyDirectory> ArrayOfDirectories;
\end{verbatim}

but is not that complicated and allows the code to be compiled with any, however
dumb, C++ compiler in the world.

Things are much simpler for wxArray and wxSortedArray however: it is enough
just to write

\begin{verbatim} 
WX_DEFINE_ARRAY(int, ArrayOfDirectories);
WX_DEFINE_SORTED_ARRAY(int, ArrayOfFiles);
\end{verbatim}

i.e. there is only one {\tt DEFINE} macro and no need for separate 
{\tt DECLARE} one.


\wxheading{See also:}

\helpref{Container classes overview}{wxcontaineroverview}, \helpref{wxList}{wxlist}

\wxheading{Include files}

<wx/dynarray.h> for wxArray and wxSortedArray and additionally <wx/arrimpl.cpp>
for wxObjArray.

\latexignore{\rtfignore{\wxheading{Function groups}}}

\membersection{Macros for template array definition}\label{arraymacros}

To use an array you must first define the array class. This is done with the
help of the macros in this section. The class of array elements must be (at
least) forward declared for WX\_DEFINE\_ARRAY, WX\_DEFINE\_SORTED\_ARRAY and
WX\_DECLARE\_OBJARRAY macros and must be fully declared before you use
WX\_DEFINE\_OBJARRAY macro.

\helpref{WX\_DEFINE\_ARRAY}{wxdefinearray}\\
\helpref{WX\_DEFINE\_EXPORTED\_ARRAY}{wxdefinearray}\\
\helpref{WX\_DEFINE\_USER\_EXPORTED\_ARRAY}{wxdefinearray}\\
\helpref{WX\_DEFINE\_SORTED\_ARRAY}{wxdefinesortedarray}\\
\helpref{WX\_DEFINE\_SORTED\_EXPORTED\_ARRAY}{wxdefinesortedarray}\\
\helpref{WX\_DEFINE\_SORTED\_USER\_EXPORTED\_ARRAY}{wxdefinesortedarray}\\
\helpref{WX\_DECLARE\_EXPORTED\_OBJARRAY}{wxdeclareobjarray}\\
\helpref{WX\_DECLARE\_USER\_EXPORTED\_OBJARRAY}{wxdeclareobjarray}\\
\helpref{WX\_DEFINE\_OBJARRAY}{wxdefineobjarray}\\
\helpref{WX\_DEFINE\_EXPORTED\_OBJARRAY}{wxdefineobjarray}\\
\helpref{WX\_DEFINE\_USER\_EXPORTED\_OBJARRAY}{wxdefineobjarray}

To slightly complicate the matters even further, the operator $->$ defined by
default for the array iterators by these macros only makes sense if the array
element type is not a pointer itself and, although it still works, this
provokes warnings from some compilers and to avoid them you should use the 
{\tt \_PTR} versions of the macros above. For example, to define an array of
pointers to {\tt double} you should use.

Note that the above macros are generally only useful for 
wxObject types.  There are separate macros for declaring an array of a simple type, 
such as an int.

The following simple types are supported:\\
int\\
long\\
size\_t\\
double

To create an array of a simple type, simply append the type you want in CAPS to
the array definition.

For example, for an integer array, you'd use one of the following variants:

\helpref{WX\_DEFINE\_ARRAY\_INT}{wxdefinearray}\\
\helpref{WX\_DEFINE\_EXPORTED\_ARRAY\_INT}{wxdefinearray}\\
\helpref{WX\_DEFINE\_USER\_EXPORTED\_ARRAY\_INT}{wxdefinearray}\\
\helpref{WX\_DEFINE\_SORTED\_ARRAY\_INT}{wxdefinesortedarray}\\
\helpref{WX\_DEFINE\_SORTED\_EXPORTED\_ARRAY\_INT}{wxdefinesortedarray}\\
\helpref{WX\_DEFINE\_SORTED\_USER\_EXPORTED\_ARRAY\_INT}{wxdefinesortedarray}\\

\begin{verbatim} 
WX_DEFINE_ARRAY_PTR(double *, MyArrayOfDoublePointers);
\end{verbatim}

\membersection{Constructors and destructors}\label{arrayconstructorsdestructors}

Array classes are 100\% C++ objects and as such they have the appropriate copy
constructors and assignment operators. Copying wxArray just copies the elements
but copying wxObjArray copies the arrays items. However, for memory-efficiency
sake, neither of these classes has virtual destructor. It is not very important
for wxArray which has trivial destructor anyhow, but it does mean that you
should avoid deleting wxObjArray through a wxBaseArray pointer (as you would
never use wxBaseArray anyhow it shouldn't be a problem) and that you should not
derive your own classes from the array classes.

\helpref{wxArray default constructor}{wxarrayctordef}\\
\helpref{wxArray copy constructors and assignment operators}{wxarrayctorcopy}\\
\helpref{\destruct{wxArray}}{wxarraydtor}

\membersection{Memory management}\label{wxarraymemorymanagement}

Automatic array memory management is quite trivial: the array starts by
preallocating some minimal amount of memory (defined by
WX\_ARRAY\_DEFAULT\_INITIAL\_SIZE) and when further new items exhaust already
allocated memory it reallocates it adding 50\% of the currently allocated
amount, but no more than some maximal number which is defined by
ARRAY\_MAXSIZE\_INCREMENT constant. Of course, this may lead to some memory
being wasted (ARRAY\_MAXSIZE\_INCREMENT in the worst case, i.e. 4Kb in the
current implementation), so the \helpref{Shrink()}{wxarrayshrink} function is
provided to deallocate the extra memory. The \helpref{Alloc()}{wxarrayalloc} 
function can also be quite useful if you know in advance how many items you are
going to put in the array and will prevent the array code from reallocating the
memory more times than needed.

\helpref{Alloc}{wxarrayalloc}\\
\helpref{Shrink}{wxarrayshrink}

\membersection{Number of elements and simple item access}\label{arrayelementsaccess}

Functions in this section return the total number of array elements and allow to
retrieve them - possibly using just the C array indexing $[]$ operator which
does exactly the same as \helpref{Item()}{wxarrayitem} method.

\helpref{Count}{wxarraycount}\\
\helpref{GetCount}{wxarraygetcount}\\
\helpref{IsEmpty}{wxarrayisempty}\\
\helpref{Item}{wxarrayitem}\\
\helpref{Last}{wxarraylast}

\membersection{Adding items}\label{arrayadding}

\helpref{Add}{wxarrayadd}\\
\helpref{Insert}{wxarrayinsert}\\
\helpref{SetCount}{wxarraysetcount}\\
\helpref{WX\_APPEND\_ARRAY}{wxappendarray}

\membersection{Removing items}\label{arrayremoving}

\helpref{WX\_CLEAR\_ARRAY}{wxcleararray}\\
\helpref{Empty}{wxarrayempty}\\
\helpref{Clear}{wxarrayclear}\\
\helpref{RemoveAt}{wxarrayremoveat}\\
\helpref{Remove}{wxarrayremove}

\membersection{Searching and sorting}\label{arraysearchingandsorting}

\helpref{Index}{wxarrayindex}\\
\helpref{Sort}{wxarraysort}

%%%%% MEMBERS HERE %%%%%
\helponly{\insertatlevel{2}{

\wxheading{Members}

}}

\membersection{WX\_DEFINE\_ARRAY}\label{wxdefinearray}

\func{}{WX\_DEFINE\_ARRAY}{\param{}{T}, \param{}{name}}

\func{}{WX\_DEFINE\_EXPORTED\_ARRAY}{\param{}{T}, \param{}{name}}

\func{}{WX\_DEFINE\_USER\_EXPORTED\_ARRAY}{\param{}{T}, \param{}{name}, \param{}{exportspec}}

This macro defines a new array class named {\it name} and containing the
elements of type {\it T}. The second form is used when compiling wxWidgets as
a DLL under Windows and array needs to be visible outside the DLL.  The third is
needed for exporting an array from a user DLL.

Example:

\begin{verbatim}
WX_DEFINE_ARRAY_INT(wxArrayInt);

class MyClass;
WX_DEFINE_ARRAY(MyClass *, wxArrayOfMyClass);
\end{verbatim}

Note that wxWidgets predefines the following standard array classes: wxArrayInt,
wxArrayLong and wxArrayPtrVoid.

\membersection{WX\_DEFINE\_SORTED\_ARRAY}\label{wxdefinesortedarray}

\func{}{WX\_DEFINE\_SORTED\_ARRAY}{\param{}{T}, \param{}{name}}

\func{}{WX\_DEFINE\_SORTED\_EXPORTED\_ARRAY}{\param{}{T}, \param{}{name}}

\func{}{WX\_DEFINE\_SORTED\_USER\_EXPORTED\_ARRAY}{\param{}{T}, \param{}{name}}

This macro defines a new sorted array class named {\it name} and containing
the elements of type {\it T}. The second form is used when compiling wxWidgets as
a DLL under Windows and array needs to be visible outside the DLL.  The third is
needed for exporting an array from a user DLL.

Example:

\begin{verbatim}
WX_DEFINE_SORTED_ARRAY_INT(wxSortedArrayInt);

class MyClass;
WX_DEFINE_SORTED_ARRAY(MyClass *, wxArrayOfMyClass);
\end{verbatim}

You will have to initialize the objects of this class by passing a comparison
function to the array object constructor like this:

\begin{verbatim}
int CompareInts(int n1, int n2)
{
    return n1 - n2;
}

wxSortedArrayInt sorted(CompareInts);

int CompareMyClassObjects(MyClass *item1, MyClass *item2)
{
    // sort the items by their address...
    return Stricmp(item1->GetAddress(), item2->GetAddress());
}

wxArrayOfMyClass another(CompareMyClassObjects);
\end{verbatim}

\membersection{WX\_DECLARE\_OBJARRAY}\label{wxdeclareobjarray}

\func{}{WX\_DECLARE\_OBJARRAY}{\param{}{T}, \param{}{name}}

\func{}{WX\_DECLARE\_EXPORTED\_OBJARRAY}{\param{}{T}, \param{}{name}}

\func{}{WX\_DECLARE\_USER\_EXPORTED\_OBJARRAY}{\param{}{T}, \param{}{name}}

This macro declares a new object array class named {\it name} and containing
the elements of type {\it T}. The second form is used when compiling wxWidgets as
a DLL under Windows and array needs to be visible outside the DLL.  The third is
needed for exporting an array from a user DLL.

Example:

\begin{verbatim}
class MyClass;
WX_DECLARE_OBJARRAY(MyClass, wxArrayOfMyClass); // note: not "MyClass *"!
\end{verbatim}

You must use \helpref{WX\_DEFINE\_OBJARRAY()}{wxdefineobjarray} macro to define
the array class - otherwise you would get link errors.

\membersection{WX\_DEFINE\_OBJARRAY}\label{wxdefineobjarray}

\func{}{WX\_DEFINE\_OBJARRAY}{\param{}{name}}

\func{}{WX\_DEFINE\_EXPORTED\_OBJARRAY}{\param{}{name}}

\func{}{WX\_DEFINE\_USER\_EXPORTED\_OBJARRAY}{\param{}{name}}

This macro defines the methods of the array class {\it name} not defined by the 
\helpref{WX\_DECLARE\_OBJARRAY()}{wxdeclareobjarray} macro. You must include the
file <wx/arrimpl.cpp> before using this macro and you must have the full
declaration of the class of array elements in scope! If you forget to do the 
first, the error will be caught by the compiler, but, unfortunately, many
compilers will not give any warnings if you forget to do the second - but the
objects of the class will not be copied correctly and their real destructor will
not be called.  The latter two forms are merely aliases of the first to satisfy
some people's sense of symmetry when using the exported declarations.

Example of usage:

\begin{verbatim}
// first declare the class!
class MyClass
{
public:
   MyClass(const MyClass&);

   ...

   virtual ~MyClass();
};

#include <wx/arrimpl.cpp>
WX_DEFINE_OBJARRAY(wxArrayOfMyClass);
\end{verbatim}

\membersection{WX\_APPEND\_ARRAY}\label{wxappendarray}

\func{void}{WX\_APPEND\_ARRAY}{\param{wxArray\& }{array}, \param{wxArray\& }{other}}

This macro may be used to append all elements of the {\it other} array to the 
{\it array}. The two arrays must be of the same type.

\membersection{WX\_CLEAR\_ARRAY}\label{wxcleararray}

\func{void}{WX\_CLEAR\_ARRAY}{\param{wxArray\& }{array}}

This macro may be used to delete all elements of the array before emptying it.
It can not be used with wxObjArrays - but they will delete their elements anyhow
when you call Empty().

\membersection{Default constructors}\label{wxarrayctordef}

\func{}{wxArray}{\void}

\func{}{wxObjArray}{\void}

Default constructor initializes an empty array object.

\func{}{wxSortedArray}{\param{int (*)(T first, T second)}{compareFunction}}

There is no default constructor for wxSortedArray classes - you must initialize it
with a function to use for item comparison. It is a function which is passed
two arguments of type {\it T} where {\it T} is the array element type and which
should return a negative, zero or positive value according to whether the first
element passed to it is less than, equal to or greater than the second one.

\membersection{wxArray copy constructor and assignment operator}\label{wxarrayctorcopy}

\func{}{wxArray}{\param{const wxArray\& }{array}}

\func{}{wxSortedArray}{\param{const wxSortedArray\& }{array}}

\func{}{wxObjArray}{\param{const wxObjArray\& }{array}}

\func{wxArray\&}{operator$=$}{\param{const wxArray\& }{array}}

\func{wxSortedArray\&}{operator$=$}{\param{const wxSortedArray\& }{array}}

\func{wxObjArray\&}{operator$=$}{\param{const wxObjArray\& }{array}}

The copy constructors and assignment operators perform a shallow array copy
(i.e. they don't copy the objects pointed to even if the source array contains
the items of pointer type) for wxArray and wxSortedArray and a deep copy (i.e.
the array element are copied too) for wxObjArray.

\membersection{wxArray::\destruct{wxArray}}\label{wxarraydtor}

\func{}{\destruct{wxArray}}{\void}

\func{}{\destruct{wxSortedArray}}{\void}

\func{}{\destruct{wxObjArray}}{\void}

The wxObjArray destructor deletes all the items owned by the array. This is not
done by wxArray and wxSortedArray versions - you may use 
\helpref{WX\_CLEAR\_ARRAY}{wxcleararray} macro for this.

\membersection{wxArray::Add}\label{wxarrayadd}

\func{void}{Add}{\param{T }{item}, \param{size\_t}{ copies = $1$}}

\func{void}{Add}{\param{T *}{item}}

\func{void}{Add}{\param{T \&}{item}, \param{size\_t}{ copies = $1$}}

Appends the given number of {\it copies} of the {\it item} to the array
consisting of the elements of type {\it T}.

The first version is used with wxArray and wxSortedArray. The second and the
third are used with wxObjArray. There is an important difference between
them: if you give a pointer to the array, it will take ownership of it, i.e.
will delete it when the item is deleted from the array. If you give a reference
to the array, however, the array will make a copy of the item and will not take
ownership of the original item. Once again, it only makes sense for wxObjArrays
because the other array types never take ownership of their elements. Also note
that you cannot append more than one pointer as reusing it would lead to
deleting it twice (or more) and hence to a crash.

You may also use \helpref{WX\_APPEND\_ARRAY}{wxappendarray} macro to append all
elements of one array to another one but it is more efficient to use 
{\it copies} parameter and modify the elements in place later if you plan to
append a lot of items.

\membersection{wxArray::Alloc}\label{wxarrayalloc}

\func{void}{Alloc}{\param{size\_t }{count}}

Preallocates memory for a given number of array elements. It is worth calling
when the number of items which are going to be added to the array is known in
advance because it will save unneeded memory reallocation. If the array already
has enough memory for the given number of items, nothing happens.

\membersection{wxArray::Clear}\label{wxarrayclear}

\func{void}{Clear}{\void}

This function does the same as \helpref{Empty()}{wxarrayempty} and additionally
frees the memory allocated to the array.

\membersection{wxArray::Count}\label{wxarraycount}

\constfunc{size\_t}{Count}{\void}

Same as \helpref{GetCount()}{wxarraygetcount}. This function is deprecated -
it exists only for compatibility.

\membersection{wxObjArray::Detach}\label{wxobjarraydetach}

\func{T *}{Detach}{\param{size\_t }{index}}

Removes the element from the array, but, unlike, 
\helpref{Remove()}{wxarrayremove} doesn't delete it. The function returns the
pointer to the removed element.

\membersection{wxArray::Empty}\label{wxarrayempty}

\func{void}{Empty}{\void}

Empties the array. For wxObjArray classes, this destroys all of the array
elements. For wxArray and wxSortedArray this does nothing except marking the
array of being empty - this function does not free the allocated memory, use 
\helpref{Clear()}{wxarrayclear} for this.

\membersection{wxArray::GetCount}\label{wxarraygetcount}

\constfunc{size\_t}{GetCount}{\void}

Return the number of items in the array.

\membersection{wxArray::Index}\label{wxarrayindex}

\func{int}{Index}{\param{T\& }{item}, \param{bool }{searchFromEnd = false}}

\func{int}{Index}{\param{T\& }{item}}

The first version of the function is for wxArray and wxObjArray, the second is
for wxSortedArray only.

Searches the element in the array, starting from either beginning or the end
depending on the value of {\it searchFromEnd} parameter. {\tt wxNOT\_FOUND} is
returned if the element is not found, otherwise the index of the element is
returned.

Linear search is used for the wxArray and wxObjArray classes but binary search
in the sorted array is used for wxSortedArray (this is why searchFromEnd
parameter doesn't make sense for it).

{\bf NB:} even for wxObjArray classes, the operator==() of the elements in the
array is {\bf not} used by this function. It searches exactly the given
element in the array and so will only succeed if this element had been
previously added to the array, but fail even if another, identical, element is
in the array.

\membersection{wxArray::Insert}\label{wxarrayinsert}

\func{void}{Insert}{\param{T }{item}, \param{size\_t }{n}, \param{size\_t }{copies = $1$}}

\func{void}{Insert}{\param{T *}{item}, \param{size\_t }{n}}

\func{void}{Insert}{\param{T \&}{item}, \param{size\_t }{n}, \param{size\_t }{copies = $1$}}

Insert the given number of {\it copies} of the {\it item} into the array before
the existing item {\it n} - thus, {\it Insert(something, 0u)} will insert an
item in such way that it will become the first array element.

Please see \helpref{Add()}{wxarrayadd} for explanation of the differences
between the overloaded versions of this function.

\membersection{wxArray::IsEmpty}\label{wxarrayisempty}

\constfunc{bool}{IsEmpty}{\void}

Returns true if the array is empty, false otherwise.

\membersection{wxArray::Item}\label{wxarrayitem}

\constfunc{T\&}{Item}{\param{size\_t }{index}}

Returns the item at the given position in the array. If {\it index} is out of
bounds, an assert failure is raised in the debug builds but nothing special is
done in the release build.

The returned value is of type "reference to the array element type" for all of
the array classes.

\membersection{wxArray::Last}\label{wxarraylast}

\constfunc{T\&}{Last}{\void}

Returns the last element in the array, i.e. is the same as Item(GetCount() - 1).
An assert failure is raised in the debug mode if the array is empty.

The returned value is of type "reference to the array element type" for all of
the array classes.

\membersection{wxArray::Remove}\label{wxarrayremove}

\func{\void}{Remove}{\param{T }{item}}

Removes an element from the array by value: the first item of the
array equal to {\it item} is removed, an assert failure will result from an
attempt to remove an item which doesn't exist in the array.

When an element is removed from wxObjArray it is deleted by the array - use 
\helpref{Detach()}{wxobjarraydetach} if you don't want this to happen. On the
other hand, when an object is removed from a wxArray nothing happens - you
should delete it manually if required:

\begin{verbatim}
T *item = array[n];
delete item;
array.Remove(n)
\end{verbatim}

See also \helpref{WX\_CLEAR\_ARRAY}{wxcleararray} macro which deletes all
elements of a wxArray (supposed to contain pointers).

\membersection{wxArray::RemoveAt}\label{wxarrayremoveat}

\func{\void}{RemoveAt}{\param{size\_t }{index}, \param{size\_t }{count = $1$}}

Removes {\it count} elements starting at {\it index} from the array. When an
element is removed from wxObjArray it is deleted by the array - use
\helpref{Detach()}{wxobjarraydetach} if you don't want this to happen. On
the other hand, when an object is removed from a wxArray nothing happens -
you should delete it manually if required:

\begin{verbatim}
T *item = array[n];
delete item;
array.RemoveAt(n)
\end{verbatim}

See also \helpref{WX\_CLEAR\_ARRAY}{wxcleararray} macro which deletes all
elements of a wxArray (supposed to contain pointers).

\membersection{wxArray::SetCount}\label{wxarraysetcount}

\func{void}{SetCount}{\param{size\_t }{count}, \param{T }{defval = T($0$)}}

This function ensures that the number of array elements is at least 
{\it count}. If the array has already {\it count} or more items, nothing is
done. Otherwise, {\tt count - GetCount()} elements are added and initialized to
the value {\it defval}.

\wxheading{See also}

\helpref{GetCount}{wxarraygetcount}

\membersection{wxArray::Shrink}\label{wxarrayshrink}

\func{void}{Shrink}{\void}

Frees all memory unused by the array. If the program knows that no new items
will be added to the array it may call Shrink() to reduce its memory usage.
However, if a new item is added to the array, some extra memory will be
allocated again.

\membersection{wxArray::Sort}\label{wxarraysort}

\func{void}{Sort}{\param{CMPFUNC<T> }{compareFunction}}

The notation CMPFUNC<T> should be read as if we had the following declaration:

\begin{verbatim}
template int CMPFUNC(T *first, T *second);
\end{verbatim}

where {\it T} is the type of the array elements. I.e. it is a function returning 
{\it int} which is passed two arguments of type {\it T *}.

Sorts the array using the specified compare function: this function should
return a negative, zero or positive value according to whether the first element
passed to it is less than, equal to or greater than the second one.

wxSortedArray doesn't have this function because it is always sorted.

