\section{\class{wxModule}}\label{wxmodule}

The module system is a very simple mechanism to allow applications (and parts of wxWidgets itself) to
define initialization and cleanup functions that are automatically called on wxWidgets
startup and exit.

To define a new kind of module, derive a class from wxModule, override the OnInit and OnExit functions,
and add the DECLARE\_DYNAMIC\_CLASS and IMPLEMENT\_DYNAMIC\_CLASS to header and implementation files
(which can be the same file). On initialization, wxWidgets will find all classes derived from wxModule,
create an instance of each, and call each OnInit function. On exit, wxWidgets will call the OnExit
function for each module instance.

Note that your module class does not have to be in a header file.

For example:

\begin{verbatim}
  // A module to allow DDE initialization/cleanup
  // without calling these functions from app.cpp or from
  // the user's application.

  class wxDDEModule: public wxModule
  {
  DECLARE_DYNAMIC_CLASS(wxDDEModule)
  public:
      wxDDEModule() {}
      bool OnInit() { wxDDEInitialize(); return true; };
      void OnExit() { wxDDECleanUp(); };
  };

  IMPLEMENT_DYNAMIC_CLASS(wxDDEModule, wxModule)
\end{verbatim}

\wxheading{Derived from}

\helpref{wxObject}{wxobject}

\wxheading{Include files}

<wx/module.h>

\latexignore{\rtfignore{\wxheading{Members}}}

\membersection{wxModule::wxModule}\label{wxmodulector}

\func{}{wxModule}{\void}

Constructs a wxModule object.

\membersection{wxModule::\destruct{wxModule}}\label{wxmoduledtor}

\func{}{\destruct{wxModule}}{\void}

Destructor.

\membersection{wxModule::OnExit}\label{wxmoduleonexit}

\func{virtual void}{OnExit}{\void}

Provide this function with appropriate cleanup for your module.

\membersection{wxModule::OnInit}\label{wxmoduleoninit}

\func{virtual bool}{OnInit}{\void}

Provide this function with appropriate initialization for your module. If the function
returns false, wxWidgets will exit immediately.

