\section{\class{wxPathList}}\label{wxpathlist}

The path list is a convenient way of storing a number of directories, and
when presented with a filename without a directory, searching for an existing file
in those directories.

Be sure to look also at \helpref{wxStandardPaths}{wxstandardpaths} if you only
want to search files in some standard paths.

\wxheading{Derived from}

\helpref{wxArrayString}{wxarraystring}

\wxheading{Include files}

<wx/filefn.h>

\wxheading{See also}

\helpref{wxArrayString}{wxarraystring}, \helpref{wxStandardPaths}{wxstandardpaths}, \helpref{wxFileName}{wxfilename}

\latexignore{\rtfignore{\wxheading{Members}}}


\membersection{wxPathList::wxPathList}\label{wxpathlistctor}

\func{}{wxPathList}{\void}

Empty constructor.

\func{}{wxPathList}{\param{const wxArrayString\& }{arr}}

Constructs the object calling the \helpref{Add}{wxpathlistadd} function.


\membersection{wxPathList::AddEnvList}\label{wxpathlistaddenvlist}

\func{void}{AddEnvList}{\param{const wxString\& }{env\_variable}}

Finds the value of the given environment variable, and adds all paths
to the path list. Useful for finding files in the {\tt PATH} variable, for
example.


\membersection{wxPathList::Add}\label{wxpathlistadd}

\func{bool}{Add}{\param{const wxString\& }{path}}

\func{void}{Add}{\param{const wxArrayString\& }{arr}}

The first form adds the given directory to the path list, if the path is not already in the list.
If the path cannot be normalized for some reason, it returns \false.

The second form just calls the first form on all elements of the given array.

The {\it path} is always considered a directory but no existence checks will be done on it
(because if it doesn't exist, it could be created later and thus result a valid path when
\helpref{FindValidPath}{wxpathlistfindvalidpath} is called).

{\bf Note:} if the given path is relative, it won't be made absolute before adding it
(this is why \helpref{FindValidPath}{wxpathlistfindvalidpath} may return relative paths).


\membersection{wxPathList::EnsureFileAccessible}\label{wxpathlistensurefileaccessible}

\func{bool}{EnsureFileAccessible}{\param{const wxString\& }{filename}}

Given a full filename (with path), calls \helpref{Add}{wxpathlistadd} with the path
of the file.


\membersection{wxPathList::FindAbsoluteValidPath}\label{wxpathlistfindabsolutepath}

\constfunc{wxString}{FindAbsoluteValidPath}{\param{const wxString\& }{file}}

Like \helpref{FindValidPath}{wxpathlistfindvalidpath} but this function always
returns an absolute path (eventually prepending the current working directory
to the value returned \helpref{FindValidPath}{wxpathlistfindvalidpath}) or an
empty string.


\membersection{wxPathList::FindValidPath}\label{wxpathlistfindvalidpath}

\constfunc{wxString}{FindValidPath}{\param{const wxString\& }{file}}

Searches the given file in all paths stored in this class.
The first path which concatenated to the given string points to an existing
file (see \helpref{wxFileExists}{wxfileexists}) is returned.

If the file wasn't found in any of the stored paths, an empty string is returned.

The given string must be a file name, eventually with a path prefix (if the path
prefix is absolute, only its name will be searched); i.e. it must not end with
a directory separator (see \helpref{wxFileName::GetPathSeparator}{wxfilenamegetpathseparator})
otherwise an assertion will fail.

The returned path may be relative to the current working directory.
Note in fact that wxPathList can be used to store both relative and absolute paths so that
if you \helpref{Add()}{wxpathlistadd}ed relative paths, then the current working directory
(see \helpref{wxGetCwd}{wxgetcwd} and \helpref{wxSetWorkingDirectory}{wxsetworkingdirectory})
may affect the value returned by this function!

