%%%%%%%%%%%%%%%%%%%%%%%%%%%%%%%%%%%%%%%%%%%%%%%%%%%%%%%%%%%%%%%%%%%%%%%%%%%%%%%
%% Name:        regkey.tex
%% Purpose:     wxRegKey docs
%% Author:      Ryan Norton <wxprojects@comcast.net>, C.C.Chakkaradeep
%% Modified by:
%% Created:     2/5/2005
%% RCS-ID:      $Id$
%% Copyright:   (c) Ryan Norton (C.C.Chakkaradeep?)
%% License:     wxWindows license
%%%%%%%%%%%%%%%%%%%%%%%%%%%%%%%%%%%%%%%%%%%%%%%%%%%%%%%%%%%%%%%%%%%%%%%%%%%%%%%

\section{\class{wxRegKey}}\label{wxregkey}

wxRegKey is a class representing the Windows registry (it is only available
under Windows). One can create, query and delete registry keys using this
class.

The Windows registry is easy to understand. There are five registry keys,
namely:

\begin{enumerate}\itemsep=0pt
\item HKEY\_CLASSES\_ROOT (HKCR)
\item HKEY\_CURRENT\_USER (HKCU)
\item HKEY\_LOCAL\_MACHINE (HKLM)
\item HKEY\_CURRENT\_CONFIG (HKCC)
\item HKEY\_USERS (HKU)
\end{enumerate}

After creating a key, it can hold a value. The values can be:

\begin{enumerate}\itemsep=0pt
\item String Value
\item Binary Value
\item DWORD Value
\item Multi String Value
\item Expandable String Value
\end{enumerate}

\wxheading{Derived from}

None

\wxheading{Include files}

<wx/msw/registry.h>

\wxheading{Example}

\begin{verbatim}
wxRegKey *pRegKey = new wxRegKey("HKEY_LOCAL_MACHINE\\Software\\MyKey");

//will create the Key if it does not exist
if( !pRegKey->Exists() )
    pRegKey->Create();

//will create a new value MYVALUE and set it to 12
pRegKey->SetValue("MYVALUE",12);

//Query for the Value and Retrieve it
long lMyVal;
wxString strTemp;
pRegKey->QueryValue("MYVALUE",&lMyVal); 
strTemp.Printf("%d",lMyVal);
wxMessageBox(strTemp,"Registry Value",0,this);

//Retrive the number of SubKeys and enumerate them
size_t nSubKeys;
pRegKey->GetKeyInfo(&nSubKeys,NULL,NULL,NULL);

pRegKey->GetFirstKey(strTemp,1);
for(int i=0;i<nSubKeys;i++)
{
     wxMessageBox(strTemp,"SubKey Name",0,this);
     pRegKey->GetNextKey(strTemp,1);
}
\end{verbatim}

\latexignore{\rtfignore{\wxheading{Members}}}


\membersection{wxRegKey::wxRegKey}\label{wxregkeyctor}

\func{}{wxRegKey}{\void}

The Constructor to set to HKCR

\func{}{wxRegKey}{\param{const wxString\&}{ strKey}}

The constructor to set the full name of the key.

\func{}{wxRegKey}{\param{const wxRegKey\&}{ keyParent}, \param{const wxString\&}{ strKey}}

The constructor to set the full name of the key under a previously created parent.


\membersection{wxRegKey::Close}\label{wxregkeyclose}

\func{void}{Close}{\void}

Closes the key.


\membersection{wxRegKey::Create}\label{wxregkeycreate}

\func{bool}{Create}{\param{bool }{bOkIfExists = true}}

Creates the key. Will fail if the key already exists and {\it bOkIfExists} is false.


\membersection{wxRegKey::DeleteSelf}\label{wxregkeydeleteself}

\func{void}{DeleteSelf}{\void}

Deletes this key and all of its subkeys and values recursively.


\membersection{wxRegKey::DeleteKey}\label{wxregkeydeletekey}

\func{void}{DeleteKey}{\param{const wxChar *}{szKey}}

Deletes the subkey with all of its subkeys/values recursively.


\membersection{wxRegKey::DeleteValue}\label{wxregkeydeletevalue}

\func{void}{DeleteValue}{\param{const wxChar *}{szKey}}

Deletes the named value.


\membersection{wxRegKey::Exists}\label{wxregkeyexists}

\constfunc{static bool}{Exists}{\void}

Returns true if the key exists.


\membersection{wxRegKey::GetName}\label{wxregkeygetname}

\constfunc{wxString}{GetName}{\param{bool }{bShortPrefix = true}}

Gets the name of the registry key.


\membersection{wxRegKey::GetFirstKey}\label{wxregkeygetfirstkey}

\func{bool}{GetFirstKey}{\param{wxString\&}{ strKeyName}, \param{long\&}{ lIndex}}

Gets the first key.


\membersection{wxRegKey::GetFirstValue}\label{wxregkeygetfirstvalue}

\func{bool}{GetFirstValue}{\param{wxString\&}{ strValueName}, \param{long\&}{ lIndex}}

Gets the first value of this key.


\membersection{wxRegKey::GetKeyInfo}\label{wxregkeygetkeyinfo}

\constfunc{bool}{Exists}{\param{size\_t *}{pnSubKeys}, \param{size\_t *}{pnValues}, \param{size\_t *}{pnMaxValueLen}}

Gets information about the key.

\wxheading{Parameters}

\docparam{pnSubKeys}{The number of subkeys.}

\docparam{pnMaxKeyLen}{The maximum length of the subkey name.}

\docparam{pnValues}{The number of values.}


\membersection{wxRegKey::GetNextKey}\label{wxregkeygetnextkey}

\constfunc{bool}{GetNextKey}{\param{wxString\&}{ strKeyName}, \param{long\&}{ lIndex}}

Gets the next key.


\membersection{wxRegKey::GetNextValue}\label{wxregkeygetnextvalue}

\constfunc{bool}{GetNextValue}{\param{wxString\&}{ strValueName}, \param{long\&}{ lIndex}}

Gets the next key value for this key.


\membersection{wxRegKey::HasValue}\label{wxregkeyhasvalue}

\constfunc{bool}{HasValue}{\param{const wxChar *}{szValue}}

Returns true if the value exists.


\membersection{wxRegKey::HasValues}\label{wxregkeyhasvalues}

\constfunc{bool}{HasValues}{\void}

Returns true if any values exist.


\membersection{wxRegKey::HasSubKey}\label{wxregkeyhassubkey}

\constfunc{bool}{HasSubKey}{\param{const wxChar *}{szKey}}

Returns true if given subkey exists.


\membersection{wxRegKey::HasSubKeys}\label{wxregkeyhassubkeys}

\constfunc{bool}{HasSubKeys}{\void}

Returns true if any subkeys exist.


\membersection{wxRegKey::IsEmpty}\label{wxregkeyisempty}

\constfunc{bool}{IsEmpty}{\void}

Returns true if this key is empty, nothing under this key.


\membersection{wxRegKey::IsOpened}\label{wxregkeyisopened}

\constfunc{bool}{IsOpened}{\void}

Returns true if the key is opened.


\membersection{wxRegKey::Open}\label{wxregkeyopen}

\func{bool}{Open}{\param{AccessMode }{mode = Write}}

Explicitly opens the key. This method also allows the key to be opened in
read-only mode by passing \texttt{wxRegKey::Read} instead of default 
\texttt{wxRegKey::Write} parameter.


\membersection{wxRegKey::QueryValue}\label{wxregkeyqueryvalue}

\constfunc{bool}{QueryValue}{\param{const wxChar *}{szValue}, \param{wxString\&}{ strValue}}

Retrieves the string value.

\constfunc{bool}{QueryValue}{\param{const wxChar *}{szValue}, \param{long *}{plValue}}

Retrieves the numeric value.


\membersection{wxRegKey::Rename}\label{wxregkeyrename}

\func{bool}{Rename}{\param{const wxChar *}{ szNewName}}

Renames the key.


\membersection{wxRegKey::RenameValue}\label{wxregkeyrenamevalue}

\func{bool}{RenameValue}{\param{const wxChar *}{szValueOld}, \param{const wxChar *}{szValueNew}}

Renames a value.


\membersection{wxRegKey::SetValue}\label{wxregkeysetvalue}

\func{bool}{SetValue}{\param{const wxChar *}{szValue}, \param{long}{ lValue}}

\func{bool}{SetValue}{\param{const wxChar *}{szValue}, \param{const wxString\&}{ strValue}}

\func{bool}{SetValue}{\param{const wxChar *}{szValue}, \param{const wxMemoryBuffer\&}{ buf}}

Sets the given \arg{szValue} which must be numeric, string or binary depending
on the overload used. If the value doesn't exist, it is created.

