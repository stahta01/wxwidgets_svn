%%%%%%%%%%%%%%%%%%%%%%%%%%%%%%%%%%%%%%%%%%%%%%%%%%%%%%%%%%%%%%%%%%%%%%%%%%%%%%%
%% Name:        recguard.tex
%% Purpose:     wxRecursionGuard documentation
%% Author:      Vadim Zeitlin
%% Modified by:
%% Created:     14.08.03
%% RCS-ID:      $Id$
%% Copyright:   (c) Vadim Zeitlin
%% License:     wxWindows license
%%%%%%%%%%%%%%%%%%%%%%%%%%%%%%%%%%%%%%%%%%%%%%%%%%%%%%%%%%%%%%%%%%%%%%%%%%%%%%%

\section{\class{wxRecursionGuard}}\label{wxrecursionguard}

wxRecursionGuard is a very simple class which can be used to prevent reentrancy
problems in a function. It is not thread-safe and so should be used only in
single-threaded programs or in combination with some thread synchronization
mechanisms.

wxRecursionGuard is always used together with the 
\helpref{wxRecursionGuardFlag}{wxrecursionguardflag} like in this example:
\begin{verbatim}
    void Foo()
    {
        static wxRecursionGuardFlag s_flag;
        wxRecursionGuard guard(s_flag);
        if ( guard.IsInside() )
        {
            // don't allow reentrancy
            return;
        }

        ...
    }
\end{verbatim}

As you can see, wxRecursionGuard simply tests the flag value and sets it to
true if it hadn't been already set. 
\helpref{IsInside()}{wxrecursionguardisinside} allows testing the old flag
value. The advantage of using this class compared to directly manipulating the
flag is that the flag is always reset in the wxRecursionGuard destructor and so
you don't risk to forget to do it even if the function returns in an unexpected
way (for example because an exception has been thrown).

\wxheading{Derived from}

No base class

\wxheading{Include files}

<wx/recguard.h>


\latexignore{\rtfignore{\wxheading{Members}}}

\membersection{wxRecursionGuard::wxRecursionGuard}\label{wxrecursionguardctor}

\func{}{wxRecursionGuard}{\param{wxRecursionGuardFlag\& }{flag}}

A wxRecursionGuard object must always be initialized with a (static) 
\helpref{wxRecursionGuardFlag}{wxrecursionguardflag}. The constructor saves the
value of the flag to be able to return the correct value from 
\helpref{IsInside}{wxrecursionguardisinside}.


\membersection{wxRecursionGuard::\destruct{wxRecursionGuard}}\label{wxrecursionguarddtor}

\func{}{\destruct{wxRecursionGuard}}{\void}

The destructor resets the flag value so that the function can be entered again
the next time.

Note that it is not virtual and so this class is not meant to be derived from
(besides, there is absolutely no reason to do it anyhow).


\membersection{wxRecursionGuard::IsInside}\label{wxrecursionguardisinside}

\constfunc{bool}{IsInside}{\void}

Returns \true if we're already inside the code block ``protected'' by this
wxRecursionGuard (i.e. between this line and the end of current scope). Usually
the function using wxRecursionGuard takes some specific actions in such case
(may be simply returning) to prevent reentrant calls to itself.

If this method returns \false, it is safe to continue.



\section{\class{wxRecursionGuardFlag}}\label{wxrecursionguardflag}

This is a completely opaque class which exists only to be used with 
\helpref{wxRecursionGuard}{wxrecursionguard}, please see the example in that
class documentation.

Please notice that wxRecursionGuardFlag object \emph{must} be declared 
\texttt{static} or the recursion would never be detected.

\wxheading{Derived from}

No base class

\wxheading{Include files}

<wx/recguard.h>

