\section{\class{wxLayoutConstraints}}\label{wxlayoutconstraints}

{\bf Note:} constraints are now deprecated and you should use \helpref{sizers}{sizeroverview} instead.

Objects of this class can be associated with a window to define its
layout constraints, with respect to siblings or its parent.

The class consists of the following eight constraints of class wxIndividualLayoutConstraint,
some or all of which should be accessed directly to set the appropriate
constraints.

\begin{itemize}\itemsep=0pt
\item {\bf left:} represents the left hand edge of the window
\item {\bf right:} represents the right hand edge of the window
\item {\bf top:} represents the top edge of the window
\item {\bf bottom:} represents the bottom edge of the window
\item {\bf width:} represents the width of the window
\item {\bf height:} represents the height of the window
\item {\bf centreX:} represents the horizontal centre point of the window
\item {\bf centreY:} represents the vertical centre point of the window
\end{itemize}

Most constraints are initially set to have the relationship wxUnconstrained,
which means that their values should be calculated by looking at known constraints.
The exceptions are {\it width} and {\it height}, which are set to wxAsIs to
ensure that if the user does not specify a constraint, the existing
width and height will be used, to be compatible with panel items which often
have take a default size. If the constraint is wxAsIs, the dimension will
not be changed.

\perlnote{In wxPerl the constraints are accessed as
\begin{verbatim}
  constraint = Wx::LayoutConstraints->new();
  constraint->centreX->AsIs();
  constraint->centreY->Unconstrained();
\end{verbatim}
}

\wxheading{Derived from}

\helpref{wxObject}{wxobject}

\wxheading{Include files}

<wx/layout.h>

\wxheading{See also}

\helpref{Overview and examples}{constraintsoverview},\rtfsp
\helpref{wxIndividualLayoutConstraint}{wxindividuallayoutconstraint}, \helpref{wxWindow::SetConstraints}{wxwindowsetconstraints}

\latexignore{\rtfignore{\wxheading{Members}}}

\membersection{wxLayoutConstraints::wxLayoutConstraints}\label{wxlayoutconstraintsctor}

\func{}{wxLayoutConstraints}{\void}

Constructor.

\membersection{wxLayoutConstraints::bottom}\label{wxlayoutconstraintsbottom}

\member{wxIndividualLayoutConstraint}{bottom}

Constraint for the bottom edge.

\membersection{wxLayoutConstraints::centreX}\label{wxlayoutconstraintscentrex}

\member{wxIndividualLayoutConstraint}{centreX}

Constraint for the horizontal centre point.

\membersection{wxLayoutConstraints::centreY}\label{wxlayoutconstraintscentrey}

\member{wxIndividualLayoutConstraint}{centreY}

Constraint for the vertical centre point.

\membersection{wxLayoutConstraints::height}\label{wxlayoutconstraintsheight}

\member{wxIndividualLayoutConstraint}{height}

Constraint for the height.

\membersection{wxLayoutConstraints::left}\label{wxlayoutconstraintsleft}

\member{wxIndividualLayoutConstraint}{left}

Constraint for the left-hand edge.

\membersection{wxLayoutConstraints::right}\label{wxlayoutconstraintsright}

\member{wxIndividualLayoutConstraint}{right}

Constraint for the right-hand edge.

\membersection{wxLayoutConstraints::top}\label{wxlayoutconstraintstop}

\member{wxIndividualLayoutConstraint}{top}

Constraint for the top edge.

\membersection{wxLayoutConstraints::width}\label{wxlayoutconstraintswidth}

\member{wxIndividualLayoutConstraint}{width}

Constraint for the width.


