%%%%%%%%%%%%%%%%%%%%%%%%%%%%%%%%%%%%%%%%%%%%%%%%%%%%%%%%%%%%%%%%%%%%%%%%%%%%%%%
%% Name:        dynlib.tex
%% Purpose:     wxDynamicLibrary documentation
%% Author:      Vadim Zeitlin
%% Modified by:
%% Created:     14.01.02 (extracted from dllload.tex)
%% RCS-ID:      $Id$
%% Copyright:   (c) Vadim Zeitlin
%% License:     wxWindows license
%%%%%%%%%%%%%%%%%%%%%%%%%%%%%%%%%%%%%%%%%%%%%%%%%%%%%%%%%%%%%%%%%%%%%%%%%%%%%%%

\section{\class{wxDynamicLibrary}}\label{wxdynamiclibrary}

wxDynamicLibrary is a class representing dynamically loadable library
(Windows DLL, shared library under Unix etc.). Just create an object of
this class to load a library and don't worry about unloading it -- it will be
done in the objects destructor automatically.

% deprecated now...
%
%\wxheading{See also}
%
%\helpref{wxDllLoader}{wxdllloader}


\membersection{wxDynamicLibrary::wxDynamicLibrary}\label{wxdynamiclibrarywxdynamiclibrary}

\func{}{wxDynamicLibrary}{\void}

\func{}{wxDynamicLibrary}{\param{const wxString\& }{name}, \param{int }{flags = wxDL\_DEFAULT}}

Constructor. Second form calls \helpref{Load}{wxdynamiclibraryload}.

\membersection{wxDynamicLibrary::CanonicalizeName}\label{wxdynamiclibrarycanonicalizename}

\func{wxString}{CanonicalizeName}{\param{const wxString\& }{name}, \param{wxDynamicLibraryCategory}{ cat = wxDL\_LIBRARY}}

Returns the platform-specific full name for the library called \arg{name}. E.g.
it adds a {\tt ".dll"} extension under Windows and {\tt "lib"} prefix and 
{\tt ".so"}, {\tt ".sl"} or maybe {\tt ".dylib"} extension under Unix.

The possible values for \arg{cat} are:

\begin{twocollist}
    \twocolitem{wxDL\_LIBRARY}{normal library}
    \twocolitem{wxDL\_MODULE}{a loadable module or plugin}
\end{twocollist}

\wxheading{See also}

\helpref{CanonicalizePluginName}{wxdynamiclibrarycanonicalizepluginname}


\membersection{wxDynamicLibrary::CanonicalizePluginName}\label{wxdynamiclibrarycanonicalizepluginname}

\func{wxString}{CanonicalizePluginName}{\param{const wxString\& }{name}, \param{wxPluginCategory}{ cat = wxDL\_PLUGIN\_GUI}}

This function does the same thing as 
\helpref{CanonicalizeName}{wxdynamiclibrarycanonicalizename} but for wxWindows
plugins. The only difference is that compiler and version information are added
to the name to ensure that the plugin which is going to be loaded will be
compatible with the main program.

The possible values for \arg{cat} are:

\begin{twocollist}
    \twocolitem{wxDL\_PLUGIN\_GUI}{plugin which uses GUI classes (default)}
    \twocolitem{wxDL\_PLUGIN\_BASE}{plugin which only uses wxBase}
\end{twocollist}

\membersection{wxDynamicLibrary::Detach}\label{wxdynamiclibrarydetach}

\func{wxDllType}{Detach}{\void}

Detaches this object from its library handle, i.e. the object will not unload
the library any longer in its destructor but it is now the callers
responsability to do this using \helpref{Unload}{wxdynamiclibraryunload}.

\membersection{wxDynamicLibrary::GetSymbol}\label{wxdynamiclibrarygetsymbol}

\constfunc{void*}{GetSymbol}{\param{const wxString\& }{name}}

Returns pointer to symbol {\it name} in the library or NULL if the library
contains no such symbol.

\wxheading{See also}

\helpref{wxDYNLIB\_FUNCTION}{wxdynlibfunction}

\membersection{wxDynamicLibrary::IsLoaded}\label{wxdynamiclibraryisloaded}

\constfunc{bool}{IsLoaded}{\void}

Returns \true if the library was successfully loaded, \false otherwise.

\membersection{wxDynamicLibrary::Load}\label{wxdynamiclibraryload}

\func{bool}{Load}{\param{const wxString\& }{name}, \param{int }{flags = wxDL\_DEFAULT}}

Loads DLL with the given \arg{name} into memory. The \arg{flags} argument can
be a combination of the following bits:

\begin{twocollist}
\twocolitem{wxDL\_LAZY}{equivalent of RTLD\_LAZY under Unix, ignored elsewhere}
\twocolitem{wxDL\_NOW}{equivalent of RTLD\_NOW under Unix, ignored elsewhere}
\twocolitem{wxDL\_GLOBAL}{equivalent of RTLD\_GLOBAL under Unix, ignored elsewhere}
\twocolitem{wxDL\_VERBATIM}{don't try to append the appropriate extension to
the library name (this is done by default).}
\end{twocollist}

Returns \true if the library was successfully loaded, \false otherwise.

\membersection{wxDynamicLibrary::Unload}\label{wxdynamiclibraryunload}

\func{void}{Unload}{\void}

\func{static void}{Unload}{\param{wxDllType }{handle}}

Unloads the library from memory. wxDynamicLibrary object automatically calls
this method from its destructor if it had been successfully loaded.

The second version is only used if you need to keep the library in memory
during a longer period of time than the scope of the wxDynamicLibrary object.
In this case you may call \helpref{Detach}{wxdynamiclibrarydetach} and store
the handle somewhere and call this static method later to unload it.

