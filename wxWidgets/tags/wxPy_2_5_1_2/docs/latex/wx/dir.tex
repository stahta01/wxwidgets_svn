%%%%%%%%%%%%%%%%%%%%%%%%%%%%%%%%%%%%%%%%%%%%%%%%%%%%%%%%%%%%%%%%%%%%%%%%%%%%%%%
%% Name:        dir.tex
%% Purpose:     wxDir documentation
%% Author:      Vadim Zeitlin
%% Modified by:
%% Created:     04.04.00
%% RCS-ID:      $Id$
%% Copyright:   (c) Vadim Zeitlin
%% License:     wxWindows license
%%%%%%%%%%%%%%%%%%%%%%%%%%%%%%%%%%%%%%%%%%%%%%%%%%%%%%%%%%%%%%%%%%%%%%%%%%%%%%%

\section{\class{wxDir}}\label{wxdir}

wxDir is a portable equivalent of Unix {open/read/close}dir functions which
allow enumerating of the files in a directory. wxDir allows enumerate files as
well as directories.

wxDir also provides a flexible way to enumerate files recursively using 
\helpref{Traverse}{wxdirtraverse} or a simpler 
\helpref{GetAllFiles}{wxdirgetallfiles} function.

Example of use:

\begin{verbatim}
    wxDir dir(wxGetCwd());

    if ( !dir.IsOpened() )
    {
        // deal with the error here - wxDir would already log an error message
        // explaining the exact reason of the failure
        return;
    }

    puts("Enumerating object files in current directory:");

    wxString filename;

    bool cont = dir.GetFirst(&filename, filespec, flags);
    while ( cont )
    {
        printf("%s\n", filename.c_str());

        cont = dir.GetNext(&filename);
    }
\end{verbatim}

\wxheading{Derived from}

No base class

\wxheading{Constants}

These flags define what kind of filename is included in the list of files
enumerated by GetFirst/GetNext.

{\small
\begin{verbatim}
enum
{
    wxDIR_FILES     = 0x0001,       // include files
    wxDIR_DIRS      = 0x0002,       // include directories
    wxDIR_HIDDEN    = 0x0004,       // include hidden files
    wxDIR_DOTDOT    = 0x0008,       // include '.' and '..'

    // by default, enumerate everything except '.' and '..'
    wxDIR_DEFAULT   = wxDIR_FILES | wxDIR_DIRS | wxDIR_HIDDEN
}
\end{verbatim}
}

\wxheading{Include files}

<wx/dir.h>

\latexignore{\rtfignore{\wxheading{Members}}}

\membersection{wxDir::wxDir}\label{wxdirwxdir}

\func{}{wxDir}{\void}

Default constructor, use \helpref{Open()}{wxdiropen} afterwards.

\func{}{wxDir}{\param{const wxString\& }{dir}}

Opens the directory for enumeration, use \helpref{IsOpened()}{wxdirisopened} 
to test for errors.


\membersection{wxDir::\destruct{wxDir}}\label{wxdirdtor}

\func{}{\destruct{wxDir}}{\void}

Destructor cleans up the associated resources. It is not virtual and so this
class is not meant to be used polymorphically.


\membersection{wxDir::Exists}\label{wxdirexists}

\func{static bool}{Exists}{\param{const wxString\& }{dir}}

Test for existence of a directory with the given name


\membersection{wxDir::GetAllFiles}\label{wxdirgetallfiles}

\func{static size\_t}{GetAllFiles}{\param{const wxString\& }{dirname}, \param{wxArrayString *}{files}, \param{const wxString\& }{filespec = wxEmptyString}, \param{int }{flags = wxDIR\_DEFAULT}}

The function appends the names of all the files under directory {\it dirname} 
to the array {\it files} (note that its old contents is preserved). Only files
matching the {\it filespec} are taken, with empty spec matching all the files.

The {\it flags} parameter should always include {\tt wxDIR\_FILES} or the array
would be unchanged and should include {\tt wxDIR\_DIRS} flag to recurse into
subdirectories (both flags are included in the value by default).

See also: \helpref{Traverse}{wxdirtraverse}


\membersection{wxDir::GetFirst}\label{wxdirgetfirst}

\constfunc{bool}{GetFirst}{\param{wxString* }{filename}, \param{const wxString\& }{filespec = wxEmptyString}, \param{int }{flags = wxDIR\_DEFAULT}}

Start enumerating all files matching {\it filespec} (or all files if it is
empty) and flags, return true on success.


\membersection{wxDir::GetName}\label{wxdirgetname}

\constfunc{wxString}{GetName}{\void}

Returns the name of the directory itself. The returned string does not have the
trailing path separator (slash or backslash).


\membersection{wxDir::GetNext}\label{wxdirgetnext}

\constfunc{bool}{GetNext}{\param{wxString* }{filename}}

Continue enumerating files satisfying the criteria specified by the last call
to \helpref{GetFirst}{wxdirgetfirst}.


\membersection{wxDir::HasFiles}\label{wxdirhasfiles}

\func{bool}{HasFiles}{\param{const wxString\& }{filespec = wxEmptyString}}

Returns {\tt true} if the directory contains any files matching the given 
{\it filespec}. If {\it filespec} is empty, look for any files at all. In any
case, even hidden files are taken into account.


\membersection{wxDir::HasSubDirs}\label{wxdirhassubdirs}

\func{bool}{HasSubDirs}{\param{const wxString\& }{dirspec = wxEmptyString}}

Returns {\tt true} if the directory contains any subdirectories (if a non
empty {\it filespec} is given, only check for directories matching it).
The hidden subdirectories are taken into account as well.


\membersection{wxDir::IsOpened}\label{wxdirisopened}

\constfunc{bool}{IsOpened}{\void}

Returns true if the directory was successfully opened by a previous call to 
\helpref{Open}{wxdiropen}.


\membersection{wxDir::Open}\label{wxdiropen}

\func{bool}{Open}{\param{const wxString\& }{dir}}

Open the directory for enumerating, returns true on success or false if an
error occurred.


\membersection{wxDir::Traverse}\label{wxdirtraverse}

\func{size\_t}{Traverse}{\param{wxDirTraverser\& }{sink}, \param{const wxString\& }{filespec = wxEmptyString}, \param{int }{flags = wxDIR\_DEFAULT}}

Enumerate all files and directories under the given directory recursively
calling the element of the provided \helpref{wxDirTraverser}{wxdirtraverser} 
object for each of them.

More precisely, the function will really recurse into subdirectories if 
{\it flags} contains {\tt wxDIR\_DIRS} flag. It will ignore the files (but
still possibly recurse into subdirectories) if {\tt wxDIR\_FILES} flag is
given.

For each found directory, \helpref{sink.OnDir()}{wxdirtraverserondir} is called
and \helpref{sink.OnFile()}{wxdirtraverseronfile} is called for every file.
Depending on the return value, the enumeration may continue or stop.

The function returns the total number of files found or {\tt (size\_t)-1} on
error.

See also: \helpref{GetAllFiles}{wxdirgetallfiles}

