\section{\class{wxTreeCtrl}}\label{wxtreectrl}

A tree control presents information as a hierarchy, with items that may be expanded
to show further items. Items in a tree control are referenced by wxTreeItemId handles,
which may be tested for validity by calling wxTreeItemId::IsOk.

To intercept events from a tree control, use the event table macros described in \helpref{wxTreeEvent}{wxtreeevent}.

\wxheading{Derived from}

\helpref{wxControl}{wxcontrol}\\
\helpref{wxWindow}{wxwindow}\\
\helpref{wxEvtHandler}{wxevthandler}\\
\helpref{wxObject}{wxobject}

\wxheading{Include files}

<wx/treectrl.h>

\wxheading{Window styles}

\twocolwidtha{5cm}
\begin{twocollist}\itemsep=0pt
\twocolitem{\windowstyle{wxTR\_EDIT\_LABELS}}{Use this style
if you wish the user to be able to edit labels in the tree control.}
\twocolitem{\windowstyle{wxTR\_NO\_BUTTONS}}{For convenience
to document that no buttons are to be drawn.}
\twocolitem{\windowstyle{wxTR\_HAS\_BUTTONS}}{Use this style
to show + and - buttons to the left of parent items.}
\twocolitem{\windowstyle{wxTR\_TWIST\_BUTTONS}}{Use this style
to show Mac-style twister buttons to the left of parent items.
If both wxTR\_HAS\_BUTTONS and wxTR\_TWIST\_BUTTONS are given,
twister buttons are generated.  Generic only.}
\twocolitem{\windowstyle{wxTR\_NO\_LINES}}{Use this style
to hide vertical level connectors.}
\twocolitem{\windowstyle{wxTR\_FULL\_ROW\_HIGHLIGHT}}{Use this style to have the background
colour and the selection highlight extend over the entire horizontal
row of the tree control window. (This flag is ignored under Windows unless you 
specify wxTR\_NO\_LINES as well.) }
\twocolitem{\windowstyle{wxTR\_LINES\_AT\_ROOT}}{Use this style
to show lines between root nodes.
Only applicable if wxTR\_HIDE\_ROOT is set and wxTR\_NO\_LINES is not set.}
\twocolitem{\windowstyle{wxTR\_HIDE\_ROOT}}{Use this style
to suppress the display of the root node,
effectively causing the first-level nodes
to appear as a series of root nodes.}
\twocolitem{\windowstyle{wxTR\_ROW\_LINES}}{Use this style
to draw a contrasting border between displayed rows.}
\twocolitem{\windowstyle{wxTR\_HAS\_VARIABLE\_ROW\_HEIGHT}}{Use this style
to cause row heights to be just big enough to fit the content.
If not set, all rows use the largest row height.
The default is that this flag is unset.
Generic only.}
\twocolitem{\windowstyle{wxTR\_SINGLE}}{For convenience
to document that only one item may be selected at a time.
Selecting another item causes the current selection, if any,
to be deselected.  This is the default.}
\twocolitem{\windowstyle{wxTR\_MULTIPLE}}{Use this style
to allow a range of items to be selected.
If a second range is selected, the current range, if any, is deselected.}
\twocolitem{\windowstyle{wxTR\_EXTENDED}}{Use this style
to allow disjoint items to be selected.  (Only partially implemented; may not work in all cases.)}
\twocolitem{\windowstyle{wxTR\_DEFAULT\_STYLE}}{The set of flags that are
closest to the defaults for the native control for a particular toolkit.}
\end{twocollist}

See also \helpref{window styles overview}{windowstyles}.

\wxheading{Event handling}

To process input from a tree control, use these event handler macros to direct input to member
functions that take a \helpref{wxTreeEvent}{wxtreeevent} argument.

\twocolwidtha{7cm}
\begin{twocollist}\itemsep=0pt
\twocolitem{{\bf EVT\_TREE\_BEGIN\_DRAG(id, func)}}{Begin dragging with the left mouse button.}
\twocolitem{{\bf EVT\_TREE\_BEGIN\_RDRAG(id, func)}}{Begin dragging with the right mouse button.}
\twocolitem{{\bf EVT\_TREE\_BEGIN\_LABEL\_EDIT(id, func)}}{Begin editing a label. This can be prevented by calling \helpref{Veto()}{wxnotifyeventveto}.}
\twocolitem{{\bf EVT\_TREE\_END\_LABEL\_EDIT(id, func)}}{Finish editing a label. This can be prevented by calling \helpref{Veto()}{wxnotifyeventveto}.}
\twocolitem{{\bf EVT\_TREE\_DELETE\_ITEM(id, func)}}{Delete an item.}
\twocolitem{{\bf EVT\_TREE\_GET\_INFO(id, func)}}{Request information from the application.}
\twocolitem{{\bf EVT\_TREE\_SET\_INFO(id, func)}}{Information is being supplied.}
\twocolitem{{\bf EVT\_TREE\_ITEM\_ACTIVATED(id, func)}}{The item has been activated, i.e. chosen by double clicking it with mouse or from keyboard}
\twocolitem{{\bf EVT\_TREE\_ITEM\_COLLAPSED(id, func)}}{The item has been collapsed.}
\twocolitem{{\bf EVT\_TREE\_ITEM\_COLLAPSING(id, func)}}{The item is being collapsed. This can be prevented by calling \helpref{Veto()}{wxnotifyeventveto}.}
\twocolitem{{\bf EVT\_TREE\_ITEM\_EXPANDED(id, func)}}{The item has been expanded.}
\twocolitem{{\bf EVT\_TREE\_ITEM\_EXPANDING(id, func)}}{The item is being expanded. This can be prevented by calling \helpref{Veto()}{wxnotifyeventveto}.}
\twocolitem{{\bf EVT\_TREE\_SEL\_CHANGED(id, func)}}{Selection has changed.}
\twocolitem{{\bf EVT\_TREE\_SEL\_CHANGING(id, func)}}{Selection is changing. This can be prevented by calling \helpref{Veto()}{wxnotifyeventveto}.}
\twocolitem{{\bf EVT\_TREE\_KEY\_DOWN(id, func)}}{A key has been pressed.}
\twocolitem{{\bf EVT\_TREE\_ITEM\_GETTOOLTIP(id, func)}}{The opportunity to set the item tooltip
is being given to the application (call wxTreeEvent::SetToolTip). Windows only.}
\end{twocollist}

\wxheading{See also}

\helpref{wxTreeItemData}{wxtreeitemdata}, \helpref{wxTreeCtrl overview}{wxtreectrloverview}, \helpref{wxListBox}{wxlistbox}, \helpref{wxListCtrl}{wxlistctrl},\rtfsp
\helpref{wxImageList}{wximagelist}, \helpref{wxTreeEvent}{wxtreeevent}

\wxheading{Win32 notes}

wxTreeCtrl class uses the standard common treeview control under Win32
implemented in the system library {\tt comctl32.dll}. Some versions of this
library are known to have bugs with handling the tree control colours: the
usual symptom is that the expanded items leave black (or otherwise incorrectly
coloured) background behind them, especially for the controls using non
default background colour. The recommended solution is to upgrade the {\tt comctl32.dll}
to a newer version: see 
\urlref{http://www.microsoft.com/msdownload/ieplatform/ie/comctrlx86.asp}{http://www.microsoft.com/msdownload/ieplatform/ie/comctrlx86.asp}.

\latexignore{\rtfignore{\wxheading{Members}}}


\membersection{wxTreeCtrl::wxTreeCtrl}\label{wxtreectrlconstr}

\func{}{wxTreeCtrl}{\void}

Default constructor.

\func{}{wxTreeCtrl}{\param{wxWindow*}{ parent}, \param{wxWindowID}{ id},\rtfsp
\param{const wxPoint\&}{ pos = wxDefaultPosition}, \param{const wxSize\&}{ size = wxDefaultSize},\rtfsp
\param{long}{ style = wxTR\_HAS\_BUTTONS}, \param{const wxValidator\& }{validator = wxDefaultValidator}, \param{const wxString\& }{name = ``listCtrl"}}

Constructor, creating and showing a tree control.

\wxheading{Parameters}

\docparam{parent}{Parent window. Must not be {\tt NULL}.}

\docparam{id}{Window identifier. A value of -1 indicates a default value.}

\docparam{pos}{Window position.}

\docparam{size}{Window size. If the default size (-1, -1) is specified then the window is sized
appropriately.}

\docparam{style}{Window style. See \helpref{wxTreeCtrl}{wxtreectrl}.}

\docparam{validator}{Window validator.}

\docparam{name}{Window name.}

\wxheading{See also}

\helpref{wxTreeCtrl::Create}{wxtreectrlcreate}, \helpref{wxValidator}{wxvalidator}


\membersection{wxTreeCtrl::\destruct{wxTreeCtrl}}

\func{void}{\destruct{wxTreeCtrl}}{\void}

Destructor, destroying the list control.


\membersection{wxTreeCtrl::AddRoot}\label{wxtreectrladdroot}

\func{wxTreeItemId}{AddRoot}{\param{const wxString\&}{ text},
 \param{int}{ image = -1}, \param{int}{ selImage = -1}, \param{wxTreeItemData*}{ data = {\tt NULL}}}

Adds the root node to the tree, returning the new item.

The {\it image} and {\it selImage} parameters are an index within
the normal image list specifying the image to use for unselected and
selected items, respectively.
If {\it image} > -1 and {\it selImage} is -1, the same image is used for
both selected and unselected items.


\membersection{wxTreeCtrl::AppendItem}\label{wxtreectrlappenditem}

\func{wxTreeItemId}{AppendItem}{\param{const wxTreeItemId\& }{parent}, \param{const wxString\&}{ text},
 \param{int}{ image = -1}, \param{int}{ selImage = -1}, \param{wxTreeItemData*}{ data = {\tt NULL}}}

Appends an item to the end of the branch identified by {\it parent}, return a new item id.

The {\it image} and {\it selImage} parameters are an index within
the normal image list specifying the image to use for unselected and
selected items, respectively.
If {\it image} > -1 and {\it selImage} is -1, the same image is used for
both selected and unselected items.


\membersection{wxTreeCtrl::AssignButtonsImageList}\label{wxtreectrlassignbuttonsimagelist}

\func{void}{AssignButtonsImageList}{\param{wxImageList*}{ imageList}}

Sets the buttons image list. The button images assigned with this method will
be automatically deleted by wxTreeCtrl as appropriate
(i.e. it takes ownership of the list).

Setting or assigning the button image list enables the display of image buttons.
Once enabled, the only way to disable the display of button images is to set
the button image list to {\tt NULL}.

This function is only available in the generic version.

See also \helpref{SetButtonsImageList}{wxtreectrlsetbuttonsimagelist}.


\membersection{wxTreeCtrl::AssignImageList}\label{wxtreectrlassignimagelist}

\func{void}{AssignImageList}{\param{wxImageList*}{ imageList}}

Sets the normal image list. Image list assigned with this method will
be automatically deleted by wxTreeCtrl as appropriate
(i.e. it takes ownership of the list).

See also \helpref{SetImageList}{wxtreectrlsetimagelist}.


\membersection{wxTreeCtrl::AssignStateImageList}\label{wxtreectrlassignstateimagelist}

\func{void}{AssignStateImageList}{\param{wxImageList*}{ imageList}}

Sets the state image list. Image list assigned with this method will
be automatically deleted by wxTreeCtrl as appropriate
(i.e. it takes ownership of the list).

See also \helpref{SetStateImageList}{wxtreectrlsetstateimagelist}.



\membersection{wxTreeCtrl::Collapse}\label{wxtreectrlcollapse}

\func{void}{Collapse}{\param{const wxTreeItemId\&}{ item}}

Collapses the given item.


\membersection{wxTreeCtrl::CollapseAndReset}\label{wxtreectrlcollapseandreset}

\func{void}{CollapseAndReset}{\param{const wxTreeItemId\&}{ item}}

Collapses the given item and removes all children.


\membersection{wxTreeCtrl::Create}\label{wxtreectrlcreate}

\func{bool}{wxTreeCtrl}{\param{wxWindow*}{ parent}, \param{wxWindowID}{ id},\rtfsp
\param{const wxPoint\&}{ pos = wxDefaultPosition}, \param{const wxSize\&}{ size = wxDefaultSize},\rtfsp
\param{long}{ style = wxTR\_HAS\_BUTTONS}, \param{const wxValidator\& }{validator = wxDefaultValidator}, \param{const wxString\& }{name = ``listCtrl"}}

Creates the tree control. See \helpref{wxTreeCtrl::wxTreeCtrl}{wxtreectrlconstr} for further details.


\membersection{wxTreeCtrl::Delete}\label{wxtreectrldelete}

\func{void}{Delete}{\param{const wxTreeItemId\&}{ item}}

Deletes the specified item. A {\tt EVT\_TREE\_DELETE\_ITEM} event will be
generated.

This function may cause a subsequent call to GetNextChild to fail.


\membersection{wxTreeCtrl::DeleteAllItems}\label{wxtreectrldeleteallitems}

\func{void}{DeleteAllItems}{\void}

Deletes all the items in the control. Note that this may not generate 
{\tt EVT\_TREE\_DELETE\_ITEM} events under some Windows versions although
normally such event is generated for each removed item.


\membersection{wxTreeCtrl::DeleteChildren}\label{wxtreectrldeletechildren}

\func{void}{DeleteChildren}{\param{const wxTreeItemId\& }{item}}

Deletes all children of the given item (but not the item itself). Note that
this will {\bf not} generate any events unlike 
\helpref{Delete}{wxtreectrldelete} method.

If you have called \helpref{wxTreeCtrl::SetItemHasChildren}{wxtreectrlsetitemhaschildren}, you
may need to call it again since {\it DeleteChildren} does not automatically
clear the setting.


\membersection{wxTreeCtrl::EditLabel}\label{wxtreectrleditlabel}

\func{void}{EditLabel}{\param{const wxTreeItemId\&}{ item}}

Starts editing the label of the given item. This function generates a
EVT\_TREE\_BEGIN\_LABEL\_EDIT event which can be vetoed so that no
text control will appear for in-place editing.

If the user changed the label (i.e. s/he does not press ESC or leave
the text control without changes, a EVT\_TREE\_END\_LABEL\_EDIT event
will be sent which can be vetoed as well.

\wxheading{See also}

\helpref{wxTreeCtrl::EndEditLabel}{wxtreectrlendeditlabel},
\helpref{wxTreeEvent}{wxtreeevent}


\membersection{wxTreeCtrl::EndEditLabel}\label{wxtreectrlendeditlabel}

\func{void}{EndEditLabel}{\param{bool }{cancelEdit}}

Ends label editing. If {\it cancelEdit} is {\tt true}, the edit will be cancelled.

This function is currently supported under Windows only.

\wxheading{See also}

\helpref{wxTreeCtrl::EditLabel}{wxtreectrleditlabel}


\membersection{wxTreeCtrl::EnsureVisible}\label{wxtreectrlensurevisible}

\func{void}{EnsureVisible}{\param{const wxTreeItemId\&}{ item}}

Scrolls and/or expands items to ensure that the given item is visible.


\membersection{wxTreeCtrl::Expand}\label{wxtreectrlexpand}

\func{void}{Expand}{\param{const wxTreeItemId\&}{ item}}

Expands the given item.


\membersection{wxTreeCtrl::GetBoundingRect}\label{wxtreectrlgetitemrect}

\constfunc{bool}{GetBoundingRect}{\param{const wxTreeItemId\&}{ item}, \param{wxRect\& }{rect}, \param{bool }{textOnly = {\tt false}}}

Retrieves the rectangle bounding the {\it item}. If {\it textOnly} is {\tt true},
only the rectangle around the item's label will be returned, otherwise the
item's image is also taken into account.

The return value is {\tt true} if the rectangle was successfully retrieved or {\tt false}
if it was not (in this case {\it rect} is not changed) - for example, if the
item is currently invisible.

\pythonnote{The wxPython version of this method requires only the
{\tt item} and {\tt textOnly} parameters.  The return value is either a
{\tt wxRect} object or {\tt None}.}

\perlnote{In wxPerl this method only takes the parameters {\tt item} and 
  {\tt textOnly}, and returns a Wx::Rect ( or undef ).}


\membersection{wxTreeCtrl::GetButtonsImageList}\label{wxtreectrlgetbuttonsimagelist}

\constfunc{wxImageList*}{GetButtonsImageList}{\void}

Returns the buttons image list (from which application-defined button images are taken).

This function is only available in the generic version.


\membersection{wxTreeCtrl::GetChildrenCount}\label{wxtreectrlgetchildrencount}

\constfunc{size\_t}{GetChildrenCount}{\param{const wxTreeItemId\&}{ item}, \param{bool}{ recursively = {\tt true}}}

Returns the number of items in the branch. If {\it recursively} is {\tt true}, returns the total number
of descendants, otherwise only one level of children is counted.


\membersection{wxTreeCtrl::GetCount}\label{wxtreectrlgetcount}

\constfunc{int}{GetCount}{\void}

Returns the number of items in the control.


\membersection{wxTreeCtrl::GetEditControl}\label{wxtreectrlgeteditcontrol}

\constfunc{wxTextCtrl *}{GetEditControl}{\void}

Returns the edit control being currently used to edit a label. Returns {\tt NULL}
if no label is being edited.

{\bf NB:} It is currently only implemented for wxMSW.


\membersection{wxTreeCtrl::GetFirstChild}\label{wxtreectrlgetfirstchild}

\constfunc{wxTreeItemId}{GetFirstChild}{\param{const wxTreeItemId\&}{ item}, \param{wxTreeItemIdValue \& }{cookie}}

Returns the first child; call \helpref{wxTreeCtrl::GetNextChild}{wxtreectrlgetnextchild} for the next child.

For this enumeration function you must pass in a `cookie' parameter
which is opaque for the application but is necessary for the library
to make these functions reentrant (i.e. allow more than one
enumeration on one and the same object simultaneously). The cookie passed to
GetFirstChild and GetNextChild should be the same variable.

Returns an invalid tree item if there are no further children.

\wxheading{See also}

\helpref{wxTreeCtrl::GetNextChild}{wxtreectrlgetnextchild},
\helpref{wxTreeCtrl::GetNextSibling}{wxtreectrlgetnextsibling}

\pythonnote{In wxPython the returned wxTreeItemId and the new cookie
value are both returned as a tuple containing the two values.}

\perlnote{In wxPerl this method only takes the {\tt item} parameter, and
  returns a 2-element list {\tt ( item, cookie )}.}


\membersection{wxTreeCtrl::GetFirstVisibleItem}\label{wxtreectrlgetfirstvisibleitem}

\constfunc{wxTreeItemId}{GetFirstVisibleItem}{\void}

Returns the first visible item.


\membersection{wxTreeCtrl::GetImageList}\label{wxtreectrlgetimagelist}

\constfunc{wxImageList*}{GetImageList}{\void}

Returns the normal image list.


\membersection{wxTreeCtrl::GetIndent}\label{wxtreectrlgetindent}

\constfunc{int}{GetIndent}{\void}

Returns the current tree control indentation.


\membersection{wxTreeCtrl::GetItemBackgroundColour}\label{wxtreectrlgetitembackgroundcolour}

\constfunc{wxColour}{GetItemBackgroundColour}{\param{const wxTreeItemId\&}{ item}}

Returns the background colour of the item.


\membersection{wxTreeCtrl::GetItemData}\label{wxtreectrlgetitemdata}

\constfunc{wxTreeItemData*}{GetItemData}{\param{const wxTreeItemId\&}{ item}}

Returns the tree item data associated with the item.

\wxheading{See also}

\helpref{wxTreeItemData}{wxtreeitemdata}

\pythonnote{wxPython provides the following shortcut method:

\indented{2cm}{\begin{twocollist}\itemsep=0pt
\twocolitem{{\bf GetPyData(item)}}{Returns the Python Object
associated with the wxTreeItemData for the given item Id.}
\end{twocollist}}
}%

\perlnote{wxPerl provides the following shortcut method:
\indented{2cm}{
\begin{twocollist}\itemsep=0pt
\twocolitem{{\bf GetPlData( item )}}{Returns the Perl data
associated with the Wx::TreeItemData. It is just the same as
tree->GetItemData(item)->GetData().}
\end{twocollist}}
}%

\membersection{wxTreeCtrl::GetItemFont}\label{wxtreectrlgetitemfont}

\constfunc{wxFont}{GetItemFont}{\param{const wxTreeItemId\&}{ item}}

Returns the font of the item label.


\membersection{wxTreeCtrl::GetItemImage}\label{wxtreectrlgetitemimage}

\constfunc{int}{GetItemImage}{\param{const wxTreeItemId\& }{item},
 \param{wxTreeItemIcon }{which = wxTreeItemIcon\_Normal}}

Gets the specified item image. The value of {\it which} may be:

\begin{itemize}\itemsep=0pt
\item{wxTreeItemIcon\_Normal} to get the normal item image
\item{wxTreeItemIcon\_Selected} to get the selected item image (i.e. the image
which is shown when the item is currently selected)
\item{wxTreeItemIcon\_Expanded} to get the expanded image (this only
makes sense for items which have children - then this image is shown when the
item is expanded and the normal image is shown when it is collapsed)
\item{wxTreeItemIcon\_SelectedExpanded} to get the selected expanded image
(which is shown when an expanded item is currently selected)
\end{itemize}


\membersection{wxTreeCtrl::GetItemText}\label{wxtreectrlgetitemtext}

\constfunc{wxString}{GetItemText}{\param{const wxTreeItemId\&}{ item}}

Returns the item label.


\membersection{wxTreeCtrl::GetItemTextColour}\label{wxtreectrlgetitemtextcolour}

\constfunc{wxColour}{GetItemTextColour}{\param{const wxTreeItemId\&}{ item}}

Returns the colour of the item label.


\membersection{wxTreeCtrl::GetLastChild}\label{wxtreectrlgetlastchild}

\constfunc{wxTreeItemId}{GetLastChild}{\param{const wxTreeItemId\&}{ item}}

Returns the last child of the item (or an invalid tree item if this item has no children).

\wxheading{See also}

\helpref{GetFirstChild}{wxtreectrlgetfirstchild},
\helpref{wxTreeCtrl::GetNextSibling}{wxtreectrlgetnextsibling},
\helpref{GetLastChild}{wxtreectrlgetlastchild}


\membersection{wxTreeCtrl::GetNextChild}\label{wxtreectrlgetnextchild}

\constfunc{wxTreeItemId}{GetNextChild}{\param{const wxTreeItemId\&}{ item}, \param{wxTreeItemIdValue \& }{cookie}}

Returns the next child; call \helpref{wxTreeCtrl::GetFirstChild}{wxtreectrlgetfirstchild} for the first child.

For this enumeration function you must pass in a `cookie' parameter
which is opaque for the application but is necessary for the library
to make these functions reentrant (i.e. allow more than one
enumeration on one and the same object simultaneously). The cookie passed to
GetFirstChild and GetNextChild should be the same.

Returns an invalid tree item if there are no further children.

\wxheading{See also}

\helpref{wxTreeCtrl::GetFirstChild}{wxtreectrlgetfirstchild}

\pythonnote{In wxPython the returned wxTreeItemId and the new cookie
value are both returned as a tuple containing the two values.}

\perlnote{In wxPerl this method returns a 2-element list
  {\tt ( item, cookie )}, instead of modifying its parameters.}


\membersection{wxTreeCtrl::GetNextSibling}\label{wxtreectrlgetnextsibling}

\constfunc{wxTreeItemId}{GetNextSibling}{\param{const wxTreeItemId\&}{ item}}

Returns the next sibling of the specified item; call \helpref{wxTreeCtrl::GetPrevSibling}{wxtreectrlgetprevsibling} for the previous sibling.

Returns an invalid tree item if there are no further siblings.

\wxheading{See also}

\helpref{wxTreeCtrl::GetPrevSibling}{wxtreectrlgetprevsibling}


\membersection{wxTreeCtrl::GetNextVisible}\label{wxtreectrlgetnextvisible}

\constfunc{wxTreeItemId}{GetNextVisible}{\param{const wxTreeItemId\&}{ item}}

Returns the next visible item.


\membersection{wxTreeCtrl::GetItemParent}\label{wxtreectrlgetitemparent}

\constfunc{wxTreeItemId}{GetItemParent}{\param{const wxTreeItemId\&}{ item}}

Returns the item's parent.


\membersection{wxTreeCtrl::GetParent}\label{wxtreectrlgetparent}

\constfunc{wxTreeItemId}{GetParent}{\param{const wxTreeItemId\&}{ item}}

{\bf NOTE:} This function is deprecated and will only work if {\tt WXWIN\_COMPATIBILITY\_2\_2}
is defined.  Use \helpref{wxTreeCtrl::GetItemParent}{wxtreectrlgetitemparent} instead.

Returns the item's parent.

\pythonnote{This method is named {\tt GetItemParent} to avoid a name
clash with wxWindow::GetParent.}


\membersection{wxTreeCtrl::GetPrevSibling}\label{wxtreectrlgetprevsibling}

\constfunc{wxTreeItemId}{GetPrevSibling}{\param{const wxTreeItemId\&}{ item}}

Returns the previous sibling of the specified item; call \helpref{wxTreeCtrl::GetNextSibling}{wxtreectrlgetnextsibling} for the next sibling.

Returns an invalid tree item if there are no further children.

\wxheading{See also}

\helpref{wxTreeCtrl::GetNextSibling}{wxtreectrlgetnextsibling}


\membersection{wxTreeCtrl::GetPrevVisible}\label{wxtreectrlgetprevvisible}

\constfunc{wxTreeItemId}{GetPrevVisible}{\param{const wxTreeItemId\&}{ item}}

Returns the previous visible item.


\membersection{wxTreeCtrl::GetRootItem}\label{wxtreectrlgetrootitem}

\constfunc{wxTreeItemId}{GetRootItem}{\void}

Returns the root item for the tree control.


\membersection{wxTreeCtrl::GetItemSelectedImage}\label{wxtreectrlgetitemselectedimage}

\constfunc{int}{GetItemSelectedImage}{\param{const wxTreeItemId\& }{item}}

Gets the selected item image (this function is obsolete, use
{\tt GetItemImage(item, wxTreeItemIcon\_Selected}) instead).


\membersection{wxTreeCtrl::GetSelection}\label{wxtreectrlgetselection}

\constfunc{wxTreeItemId}{GetSelection}{\void}

Returns the selection, or an invalid item if there is no selection.
This function only works with the controls without wxTR\_MULTIPLE style, use
\helpref{GetSelections}{wxtreectrlgetselections} for the controls which do have
this style.


\membersection{wxTreeCtrl::GetSelections}\label{wxtreectrlgetselections}

\constfunc{size\_t}{GetSelections}{\param{wxArrayTreeItemIds\& }{selection}}

Fills the array of tree items passed in with the currently selected items. This
function can be called only if the control has the wxTR\_MULTIPLE style.

Returns the number of selected items.

\pythonnote{The wxPython version of this method accepts no parameters
and returns a Python list of {\tt wxTreeItemId}s.}

\perlnote{In wxPerl this method takes no parameters and returns a list of
 {\tt Wx::TreeItemId}s.}


\membersection{wxTreeCtrl::GetStateImageList}\label{wxtreectrlgetstateimagelist}

\constfunc{wxImageList*}{GetStateImageList}{\void}

Returns the state image list (from which application-defined state images are taken).


\membersection{wxTreeCtrl::HitTest}\label{wxtreectrlhittest}

\func{wxTreeItemId}{HitTest}{\param{const wxPoint\& }{point}, \param{int\& }{flags}}

Calculates which (if any) item is under the given point, returning the tree item
id at this point plus extra information {\it flags}. {\it flags} is a bitlist of the following:

\twocolwidtha{5cm}
\begin{twocollist}\itemsep=0pt
\twocolitem{wxTREE\_HITTEST\_ABOVE}{Above the client area.}
\twocolitem{wxTREE\_HITTEST\_BELOW}{Below the client area.}
\twocolitem{wxTREE\_HITTEST\_NOWHERE}{In the client area but below the last item.}
\twocolitem{wxTREE\_HITTEST\_ONITEMBUTTON}{On the button associated with an item.}
\twocolitem{wxTREE\_HITTEST\_ONITEMICON}{On the bitmap associated with an item.}
\twocolitem{wxTREE\_HITTEST\_ONITEMINDENT}{In the indentation associated with an item.}
\twocolitem{wxTREE\_HITTEST\_ONITEMLABEL}{On the label (string) associated with an item.}
\twocolitem{wxTREE\_HITTEST\_ONITEMRIGHT}{In the area to the right of an item.}
\twocolitem{wxTREE\_HITTEST\_ONITEMSTATEICON}{On the state icon for a tree view item that is in a user-defined state.}
\twocolitem{wxTREE\_HITTEST\_TOLEFT}{To the right of the client area.}
\twocolitem{wxTREE\_HITTEST\_TORIGHT}{To the left of the client area.}
\end{twocollist}

\pythonnote{in wxPython both the wxTreeItemId and the flags are
returned as a tuple.}

\perlnote{In wxPerl this method only takes the {\tt point} parameter
  and returns a 2-element list {\tt ( item, flags )}.}


\membersection{wxTreeCtrl::InsertItem}\label{wxtreectrlinsertitem}

\func{wxTreeItemId}{InsertItem}{\param{const wxTreeItemId\& }{parent}, \param{const wxTreeItemId\& }{previous}, \param{const wxString\&}{ text},
 \param{int}{ image = -1}, \param{int}{ selImage = -1}, \param{wxTreeItemData*}{ data = {\tt NULL}}}

\func{wxTreeItemId}{InsertItem}{\param{const wxTreeItemId\& }{parent}, \param{size\_t}{ before}, \param{const wxString\&}{ text},
 \param{int}{ image = -1}, \param{int}{ selImage = -1}, \param{wxTreeItemData*}{ data = {\tt NULL}}}

Inserts an item after a given one ({\it previous}) or before one identified by its position ({\it before}).
{\it before} must be less than the number of children.

The {\it image} and {\it selImage} parameters are an index within
the normal image list specifying the image to use for unselected and
selected items, respectively.
If {\it image} > -1 and {\it selImage} is -1, the same image is used for
both selected and unselected items.

\pythonnote{The second form of this method is called
{\tt InsertItemBefore} in wxPython.}


\membersection{wxTreeCtrl::IsBold}\label{wxtreectrlisbold}

\constfunc{bool}{IsBold}{\param{const wxTreeItemId\& }{item}}

Returns {\tt true} if the given item is in bold state.

See also: \helpref{SetItemBold}{wxtreectrlsetitembold}


\membersection{wxTreeCtrl::IsExpanded}\label{wxtreectrlisexpanded}

\constfunc{bool}{IsExpanded}{\param{const wxTreeItemId\&}{ item}}

Returns {\tt true} if the item is expanded (only makes sense if it has children).


\membersection{wxTreeCtrl::IsSelected}\label{wxtreectrlisselected}

\constfunc{bool}{IsSelected}{\param{const wxTreeItemId\&}{ item}}

Returns {\tt true} if the item is selected.


\membersection{wxTreeCtrl::IsVisible}\label{wxtreectrlisvisible}

\constfunc{bool}{IsVisible}{\param{const wxTreeItemId\&}{ item}}

Returns {\tt true} if the item is visible (it might be outside the view, or not expanded).


\membersection{wxTreeCtrl::ItemHasChildren}\label{wxtreectrlitemhaschildren}

\constfunc{bool}{ItemHasChildren}{\param{const wxTreeItemId\&}{ item}}

Returns {\tt true} if the item has children.


\membersection{wxTreeCtrl::OnCompareItems}\label{wxtreectrloncompareitems}

\func{int}{OnCompareItems}{\param{const wxTreeItemId\& }{item1}, \param{const wxTreeItemId\& }{item2}}

Override this function in the derived class to change the sort order of the
items in the tree control. The function should return a negative, zero or
positive value if the first item is less than, equal to or greater than the
second one.

The base class version compares items alphabetically.

See also: \helpref{SortChildren}{wxtreectrlsortchildren}


\membersection{wxTreeCtrl::PrependItem}\label{wxtreectrlprependitem}

\func{wxTreeItemId}{PrependItem}{\param{const wxTreeItemId\& }{parent}, \param{const wxString\&}{ text},
 \param{int}{ image = -1}, \param{int}{ selImage = -1}, \param{wxTreeItemData*}{ data = {\tt NULL}}}

Appends an item as the first child of {\it parent}, return a new item id.

The {\it image} and {\it selImage} parameters are an index within
the normal image list specifying the image to use for unselected and
selected items, respectively.
If {\it image} > -1 and {\it selImage} is -1, the same image is used for
both selected and unselected items.


\membersection{wxTreeCtrl::ScrollTo}\label{wxtreectrlscrollto}

\func{void}{ScrollTo}{\param{const wxTreeItemId\&}{ item}}

Scrolls the specified item into view.


\membersection{wxTreeCtrl::SelectItem}\label{wxtreectrlselectitem}

\func{bool}{SelectItem}{\param{const wxTreeItemId\&}{ item}, \param{bool }{select = \true}}

Selects the given item. In multiple selection controls, can be also used to
deselect a currently selected item if the value of \arg{select} is false.


\membersection{wxTreeCtrl::SetButtonsImageList}\label{wxtreectrlsetbuttonsimagelist}

\func{void}{SetButtonsImageList}{\param{wxImageList*}{ imageList}}

Sets the buttons image list (from which application-defined button images are taken).
The button images assigned with this method will
{\bf not} be deleted by wxTreeCtrl's destructor, you must delete it yourself.

Setting or assigning the button image list enables the display of image buttons.
Once enabled, the only way to disable the display of button images is to set
the button image list to {\tt NULL}.

This function is only available in the generic version.

See also \helpref{AssignButtonsImageList}{wxtreectrlassignbuttonsimagelist}.


\membersection{wxTreeCtrl::SetIndent}\label{wxtreectrlsetindent}

\func{void}{SetIndent}{\param{int }{indent}}

Sets the indentation for the tree control.


\membersection{wxTreeCtrl::SetImageList}\label{wxtreectrlsetimagelist}

\func{void}{SetImageList}{\param{wxImageList*}{ imageList}}

Sets the normal image list. Image list assigned with this method will
{\bf not} be deleted by wxTreeCtrl's destructor, you must delete it yourself.

See also \helpref{AssignImageList}{wxtreectrlassignimagelist}.



\membersection{wxTreeCtrl::SetItemBackgroundColour}\label{wxtreectrlsetitembackgroundcolour}

\func{void}{SetItemBackgroundColour}{\param{const wxTreeItemId\&}{ item}, \param{const wxColour\& }{col}}

Sets the colour of the item's background.


\membersection{wxTreeCtrl::SetItemBold}\label{wxtreectrlsetitembold}

\func{void}{SetItemBold}{\param{const wxTreeItemId\& }{item}, \param{bool}{ bold = {\tt true}}}

Makes item appear in bold font if {\it bold} parameter is {\tt true} or resets it to
the normal state.

See also: \helpref{IsBold}{wxtreectrlisbold}


\membersection{wxTreeCtrl::SetItemData}\label{wxtreectrlsetitemdata}

\func{void}{SetItemData}{\param{const wxTreeItemId\&}{ item}, \param{wxTreeItemData* }{data}}

Sets the item client data.

\pythonnote{wxPython provides the following shortcut method:\par
\indented{2cm}{\begin{twocollist}\itemsep=0pt
\twocolitem{{\bf SetPyData(item, obj)}}{Associate the given Python
Object with the wxTreeItemData for the given item Id.}
\end{twocollist}}
}%

\perlnote{wxPerl provides the following shortcut method:
\indented{2cm}{
\begin{twocollist}\itemsep=0pt
\twocolitem{{\bf SetPlData( item, data )}}{Sets the Perl data
associated with the Wx::TreeItemData. It is just the same as
tree->GetItemData(item)->SetData(data).}
\end{twocollist}}
}%

\membersection{wxTreeCtrl::SetItemFont}\label{wxtreectrlsetitemfont}

\func{void}{SetItemFont}{\param{const wxTreeItemId\&}{ item}, \param{const wxFont\& }{font}}

Sets the item's font. All items in the tree should have the same height to avoid
text clipping, so the fonts height should be the same for all of them,
although font attributes may vary.

\wxheading{See also}

\helpref{SetItemBold}{wxtreectrlsetitembold}


\membersection{wxTreeCtrl::SetItemHasChildren}\label{wxtreectrlsetitemhaschildren}

\func{void}{SetItemHasChildren}{\param{const wxTreeItemId\&}{ item}, \param{bool }{hasChildren = {\tt true}}}

Force appearance of the button next to the item. This is useful to
allow the user to expand the items which don't have any children now,
but instead adding them only when needed, thus minimizing memory
usage and loading time.


\membersection{wxTreeCtrl::SetItemImage}\label{wxtreectrlsetitemimage}

\func{void}{SetItemImage}{\param{const wxTreeItemId\&}{ item},
 \param{int }{image}, \param{wxTreeItemIcon }{which = wxTreeItemIcon\_Normal}}

Sets the specified item image. See \helpref{GetItemImage}{wxtreectrlgetitemimage}
for the description of the {\it which} parameter.


\membersection{wxTreeCtrl::SetItemSelectedImage}\label{wxtreectrlsetitemselectedimage}

\func{void}{SetItemSelectedImage}{\param{const wxTreeItemId\&}{ item}, \param{int }{selImage}}

Sets the selected item image (this function is obsolete, use {\tt SetItemImage(item, wxTreeItemIcon\_Selected}) instead).


\membersection{wxTreeCtrl::SetItemText}\label{wxtreectrlsetitemtext}

\func{void}{SetItemText}{\param{const wxTreeItemId\&}{ item}, \param{const wxString\& }{text}}

Sets the item label.


\membersection{wxTreeCtrl::SetItemTextColour}\label{wxtreectrlsetitemtextcolour}

\func{void}{SetItemTextColour}{\param{const wxTreeItemId\&}{ item}, \param{const wxColour\& }{col}}

Sets the colour of the item's text.


\membersection{wxTreeCtrl::SetStateImageList}\label{wxtreectrlsetstateimagelist}

\func{void}{SetStateImageList}{\param{wxImageList*}{ imageList}}

Sets the state image list (from which application-defined state images are taken).
Image list assigned with this method will
{\bf not} be deleted by wxTreeCtrl's destructor, you must delete it yourself.

See also \helpref{AssignStateImageList}{wxtreectrlassignstateimagelist}.

\membersection{wxTreeCtrl::SetWindowStyle}\label{wxtreectrlsetwindowstyle}

\func{void}{SetWindowStyle}{\param{long}{styles}}

Sets the mode flags associated with the display of the tree control.
The new mode takes effect immediately.
(Generic only; MSW ignores changes.)


\membersection{wxTreeCtrl::SortChildren}\label{wxtreectrlsortchildren}

\func{void}{SortChildren}{\param{const wxTreeItemId\&}{ item}}

Sorts the children of the given item using
\helpref{OnCompareItems}{wxtreectrloncompareitems} method of wxTreeCtrl. You
should override that method to change the sort order (the default is ascending
case-sensitive alphabetical order).

\wxheading{See also}

\helpref{wxTreeItemData}{wxtreeitemdata}, \helpref{OnCompareItems}{wxtreectrloncompareitems}


\membersection{wxTreeCtrl::Toggle}\label{wxtreectrltoggle}

\func{void}{Toggle}{\param{const wxTreeItemId\&}{ item}}

Toggles the given item between collapsed and expanded states.


\membersection{wxTreeCtrl::ToggleItemSelection}\label{wxtreectrltoggleitemselection}

\func{void}{ToggleItemSelection}{\param{const wxTreeItemId\&}{ item}}

Toggles the given item between selected and unselected states. For
multiselection controls only.


\membersection{wxTreeCtrl::Unselect}\label{wxtreectrlunselect}

\func{void}{Unselect}{\void}

Removes the selection from the currently selected item (if any).


\membersection{wxTreeCtrl::UnselectAll}\label{wxtreectrlunselectall}

\func{void}{UnselectAll}{\void}

This function either behaves the same as \helpref{Unselect}{wxtreectrlunselect}
if the control doesn't have wxTR\_MULTIPLE style, or removes the selection from
all items if it does have this style.


\membersection{wxTreeCtrl::UnselectItem}\label{wxtreectrlunselectitem}

\func{void}{UnselectItem}{\param{const wxTreeItemId\& }{item}}

Unselects the given item. This works in multiselection controls only.


