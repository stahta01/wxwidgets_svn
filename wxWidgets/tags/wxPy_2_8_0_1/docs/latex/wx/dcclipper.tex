%%%%%%%%%%%%%%%%%%%%%%%%%%%%%%%%%%%%%%%%%%%%%%%%%%%%%%%%%%%%%%%%%%%%%%%%%%%%%%%
%% Name:        dcclipper.tex
%% Purpose:     wxDCClipper documentation
%% Author:      Vadim Zeitlin
%% Created:     2006-04-10
%% RCS-ID:      $Id$
%% Copyright:   (c) 2006 Vadim Zeitlin
%% License:     wxWindows license
%%%%%%%%%%%%%%%%%%%%%%%%%%%%%%%%%%%%%%%%%%%%%%%%%%%%%%%%%%%%%%%%%%%%%%%%%%%%%%%

\section{\class{wxDCClipper}}\label{wxdcclipper}

wxDCClipper is a small helper class for setting a clipping region on a 
\helpref{wxDC}{wxdc} and unsetting it automatically. An object of wxDCClipper
class is typically created on the stack so that it is automatically destroyed
when the object goes out of scope. A typical usage example:

\begin{verbatim}
    void MyFunction(wxDC& dc)
    {
        wxDCClipper clip(rect);
        ... drawing functions here are affected by clipping rect ...
    }

    void OtherFunction()
    {
        wxDC dc;
        MyFunction(dc);
        ... drawing functions here are not affected by clipping rect ...
    }
\end{verbatim}

\wxheading{Derived from}

No base class

\wxheading{Include files}

<wx/dc.h>

\wxheading{See also}

\helpref{wxDC::SetClippingRegion}{wxdcsetclippingregion}



\latexignore{\rtfignore{\wxheading{Members}}}

\membersection{wxDCClipper::wxDCClipper}\label{wxdcclipperctor}

\func{}{wxDCClipper}{\param{wxDC\& }{dc}, \param{const wxRegion\& }{r}}

\func{}{wxDCClipper}{\param{wxDC\& }{dc}, \param{const wxRect\& }{rect}}

\func{}{wxDCClipper}{\param{wxDC\& }{dc}, \param{int }{x}, \param{int }{y}, \param{int }{w}, \param{int }{h}}

Sets the clipping region to the specified region \arg{r} or rectangle specified
by either a single \arg{rect} parameter or its position (\arg{x} and \arg{y})
and size (\arg{w} ad \arg{h}).

The clipping region is automatically unset when this object is destroyed.

