%%%%%%%%%%%%%%%%%%%%%%%%%%%%%%%%%%%%%%%%%%%%%%%%%%%%%%%%%%%%%%%%%%%%%%%%%%%%%%%
%% Name:        dgramsocket.tex
%% Purpose:     wxSocket docs
%% Author:      Angel Vidal Veiga (kry@amule.org)
%% Modified by:
%% Created:     2006
%% RCS-ID:      $Id$
%% Copyright:   (c) wxWidgets team
%% License:     wxWindows license
%%%%%%%%%%%%%%%%%%%%%%%%%%%%%%%%%%%%%%%%%%%%%%%%%%%%%%%%%%%%%%%%%%%%%%%%%%%%%%%

% ---------------------------------------------------------------------------
% CLASS wxDatagramSocket
% ---------------------------------------------------------------------------

\section{\class{wxDatagramSocket}}\label{wxdatagramsocket}

\wxheading{Derived from}

\helpref{wxSocketBase}{wxsocketbase}

\wxheading{Include files}

<wx/socket.h>

\latexignore{\rtfignore{\wxheading{Members}}}

% ---------------------------------------------------------------------------
% Members
% ---------------------------------------------------------------------------
%
% wxDatagramSocket
%
\membersection{wxDatagramSocket::wxDatagramSocket}\label{wxdatagramsocketctor}

\func{}{wxDatagramSocket}{\param{wxSocketFlags}{ flags = wxSOCKET\_NONE}}

Constructor.

\wxheading{Parameters}

\docparam{flags}{Socket flags (See \helpref{wxSocketBase::SetFlags}{wxsocketbasesetflags})}

%
% ~wxDatagramSocket
%
\membersection{wxDatagramSocket::\destruct{wxDatagramSocket}}\label{wxdatagramsocketdtor}

\func{}{\destruct{wxDatagramSocket}}{\void}

Destructor. Please see \helpref{wxSocketBase::Destroy}{wxsocketbasedestroy}.

%
% ReceiveFrom
%
\membersection{wxDatagramSocket::ReceiveFrom}\label{wxdatagramsocketreceivefrom}

\func{wxDatagramSocket\&}{ReceiveFrom}{\param{wxSockAddress\&}{ address}, \param{void *}{ buffer}, \param{wxUint32}{ nbytes}}

This function reads a buffer of {\it nbytes} bytes from the socket.

Use \helpref{LastCount}{wxsocketbaselastcount} to verify the number of bytes actually read.

Use \helpref{Error}{wxsocketbaseerror} to determine if the operation succeeded.

\wxheading{Parameters}

\docparam{address}{Any address - will be overwritten with the address of the peer that sent that data.}

\docparam{buffer}{Buffer where to put read data.}

\docparam{nbytes}{Number of bytes.}

\wxheading{Return value}

Returns a reference to the current object, and the address of the peer that sent the data on address param.

\wxheading{See also}

\helpref{wxSocketBase::Error}{wxsocketbaseerror}, 
\helpref{wxSocketBase::LastError}{wxsocketbaselasterror}, 
\helpref{wxSocketBase::LastCount}{wxsocketbaselastcount}, 
\helpref{wxSocketBase::SetFlags}{wxsocketbasesetflags},

%
% SendTo
%
\membersection{wxDatagramSocket::SendTo}\label{wxdatagramsocketsendto}

\func{wxDatagramSocket\&}{SendTo}{\param{const wxSockAddress\&}{ address}, \param{const void *}{ buffer}, \param{wxUint32}{ nbytes}}

This function writes a buffer of {\it nbytes} bytes to the socket.

Use \helpref{LastCount}{wxsocketbaselastcount} to verify the number of bytes actually wrote.

Use \helpref{Error}{wxsocketbaseerror} to determine if the operation succeeded.

\wxheading{Parameters}

\docparam{address}{The address of the destination peer for this data.}

\docparam{buffer}{Buffer where read data is.}

\docparam{nbytes}{Number of bytes.}

\wxheading{Return value}

Returns a reference to the current object.

\wxheading{See also}

\helpref{wxSocketBase::Error}{wxsocketbaseerror}, 
\helpref{wxSocketBase::LastError}{wxsocketbaselasterror}, 
\helpref{wxSocketBase::LastCount}{wxsocketbaselastcount}, 
\helpref{wxSocketBase::SetFlags}{wxsocketbasesetflags}

