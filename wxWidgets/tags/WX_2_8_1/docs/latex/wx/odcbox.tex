%%%%%%%%%%%%%%%%%%%%%%%%%%%%%%%%%%%%%%%%%%%%%%%%%%%%%%%%%%%%%%%%%%%%%%%%%%%%%
%% Name:        odcbox.tex
%% Purpose:     wxOwnerDrawnComboBox docs
%% Author:      Jaakko Salli
%% Modified by:
%% Created:
%% RCS-ID:      $Id$
%% Copyright:   (c) Jaakko Salli
%% License:     wxWindows license
%%%%%%%%%%%%%%%%%%%%%%%%%%%%%%%%%%%%%%%%%%%%%%%%%%%%%%%%%%%%%%%%%%%%%%%%%%%%%

\section{\class{wxOwnerDrawnComboBox}}\label{wxownerdrawncombobox}

wxOwnerDrawnComboBox is a combobox with owner-drawn list items.
In essence, it is a \helpref{wxComboCtrl}{wxcomboctrl} with
\helpref{wxVListBox}{wxvlistbox} popup and \helpref{wxControlWithItems}{wxcontrolwithitems}
interface.

Implementing item drawing and measuring is similar to \helpref{wxVListBox}{wxvlistbox}.
Application needs to subclass wxOwnerDrawnComboBox and implement
\helpref{OnDrawItem()}{wxownerdrawncomboboxondrawitem}, \helpref{OnMeasureItem()}{wxownerdrawncomboboxonmeasureitem}
and \helpref{OnMeasureItemWidth()}{wxownerdrawncomboboxonmeasureitemwidth}.


\wxheading{Derived from}

\helpref{wxComboCtrl}{wxcomboctrl}\\
\helpref{wxControlWithItems}{wxcontrolwithitems}\\
\helpref{wxControl}{wxcontrol}\\
\helpref{wxWindow}{wxwindow}\\
\helpref{wxEvtHandler}{wxevthandler}\\
\helpref{wxObject}{wxobject}

\wxheading{Include files}

<odcombo.h>

\wxheading{Window styles}

\begin{twocollist}\itemsep=0pt
\twocolitem{\windowstyle{wxODCB\_DCLICK\_CYCLES}}{Double-clicking cycles item
if wxCB\_READONLY is also used. Synonymous with wxCC\_SPECIAL\_DCLICK.}
\twocolitem{\windowstyle{wxODCB\_STD\_CONTROL\_PAINT}}{Control itself is not
custom painted using OnDrawItem. Even if this style is not used, writable
wxOwnerDrawnComboBox is never custom painted unless SetCustomPaintWidth is
called.}
\end{twocollist}

See also \helpref{wxComboCtrl window styles}{wxcomboctrl} and
base \helpref{window styles overview}{windowstyles}.

\wxheading{Event handling}

\twocolwidtha{7cm}
\begin{twocollist}\itemsep=0pt
\twocolitem{{\bf EVT\_COMBOBOX(id, func)}}{Process a wxEVT\_COMMAND\_COMBOBOX\_SELECTED event,
when an item on the list is selected. Note that calling
\helpref{GetValue}{wxcomboctrlgetvalue} returns the new value of selection.}
\end{twocollist}

See also events emitted by \helpref{wxComboCtrl}{wxcomboctrl}.

\wxheading{See also}

\helpref{wxComboCtrl}{wxcomboctrl}, \helpref{wxComboBox}{wxcombobox}, \helpref{wxVListBox}{wxvlistbox},
\rtfsp\helpref{wxCommandEvent}{wxcommandevent}


\latexignore{\rtfignore{\wxheading{Members}}}


\membersection{wxOwnerDrawnComboBox::wxOwnerDrawnComboBox}\label{wxownerdrawncomboboxctor}

\func{}{wxOwnerDrawnComboBox}{\void}

Default constructor.

\func{}{wxOwnerDrawnComboBox}{\param{wxWindow*}{ parent}, \param{wxWindowID}{ id},\rtfsp
\param{const wxString\& }{value = ``"}, \param{const wxPoint\&}{ pos = wxDefaultPosition}, \param{const wxSize\&}{ size = wxDefaultSize},\rtfsp
\param{int}{ n = 0}, \param{const wxString }{choices[] = NULL},\rtfsp
\param{long}{ style = 0}, \param{const wxValidator\& }{validator = wxDefaultValidator}, \param{const wxString\& }{name = ``comboBox"}}

\func{}{wxOwnerDrawnComboBox}{\param{wxWindow*}{ parent}, \param{wxWindowID}{ id},\rtfsp
\param{const wxString\& }{value}, \param{const wxPoint\&}{ pos}, \param{const wxSize\&}{ size},\rtfsp
\param{const wxArrayString\& }{choices},\rtfsp
\param{long}{ style = 0}, \param{const wxValidator\& }{validator = wxDefaultValidator}, \param{const wxString\& }{name = ``comboBox"}}

Constructor, creating and showing a owner-drawn combobox.

\wxheading{Parameters}

\docparam{parent}{Parent window. Must not be NULL.}

\docparam{id}{Window identifier. A value of -1 indicates a default value.}

\docparam{value}{Initial selection string. An empty string indicates no selection.}

\docparam{pos}{Window position.}

\docparam{size}{Window size. If the default size (-1, -1) is specified then the window is sized
appropriately.}

\docparam{n}{Number of strings with which to initialise the control.}

\docparam{choices}{An array of strings with which to initialise the control.}

\docparam{style}{Window style. See \helpref{wxOwnerDrawnComboBox}{wxownerdrawncombobox}.}

\docparam{validator}{Window validator.}

\docparam{name}{Window name.}

\wxheading{See also}

\helpref{wxOwnerDrawnComboBox::Create}{wxownerdrawncomboboxcreate}, \helpref{wxValidator}{wxvalidator}


\membersection{wxOwnerDrawnComboBox::\destruct{wxOwnerDrawnComboBox}}\label{wxownerdrawncomboboxdtor}

\func{}{\destruct{wxOwnerDrawnComboBox}}{\void}

Destructor, destroying the owner-drawn combobox.


\membersection{wxOwnerDrawnComboBox::Create}\label{wxownerdrawncomboboxcreate}

\func{bool}{Create}{\param{wxWindow*}{ parent}, \param{wxWindowID}{ id},\rtfsp
\param{const wxString\& }{value = ``"}, \param{const wxPoint\&}{ pos = wxDefaultPosition}, \param{const wxSize\&}{ size = wxDefaultSize},\rtfsp
\param{int}{ n}, \param{const wxString }{choices[]},\rtfsp
\param{long}{ style = 0}, \param{const wxValidator\& }{validator = wxDefaultValidator}, \param{const wxString\& }{name = ``comboBox"}}

\func{bool}{Create}{\param{wxWindow*}{ parent}, \param{wxWindowID}{ id},\rtfsp
\param{const wxString\& }{value}, \param{const wxPoint\&}{ pos}, \param{const wxSize\&}{ size},\rtfsp
\param{const wxArrayString\& }{choices},\rtfsp
\param{long}{ style = 0}, \param{const wxValidator\& }{validator = wxDefaultValidator}, \param{const wxString\& }{name = ``comboBox"}}

Creates the combobox for two-step construction. Derived classes
should call or replace this function. See
\helpref{wxOwnerDrawnComboBox::wxOwnerDrawnComboBox}{wxownerdrawncomboboxctor}\rtfsp
for further details.


\membersection{wxOwnerDrawnComboBox::GetWidestItem}\label{wxownerdrawncomboboxgetwidestitem}

\constfunc{int}{GetWidestItem}{\void}

Returns index to the widest item in the list.


\membersection{wxOwnerDrawnComboBox::GetWidestItemWidth}\label{wxownerdrawncomboboxgetwidestitemwidth}

\constfunc{int}{GetWidestItemWidth}{\void}

Returns width of the widest item in the list.


\membersection{wxOwnerDrawnComboBox::OnDrawBackground}\label{wxownerdrawncomboboxondrawbackground}

\constfunc{void}{OnDrawBackground}{\param{wxDC\& }{dc}, \param{const wxRect\& }{rect}, \param{int }{item}, \param{int }{flags}}

This method is used to draw the items background and, maybe, a border around it.

The base class version implements a reasonable default behaviour which consists
in drawing the selected item with the standard background colour and drawing a
border around the item if it is either selected or current.

\wxheading{Remarks}

\arg{flags} has the same meaning as with \helpref{OnDrawItem}{wxownerdrawncomboboxondrawitem}.


\membersection{wxOwnerDrawnComboBox::OnDrawItem}\label{wxownerdrawncomboboxondrawitem}

\constfunc{void}{OnDrawItem}{\param{wxDC\& }{dc}, \param{const wxRect\& }{rect}, \param{int }{item}, \param{int }{flags}}

The derived class may implement this function to actually draw the item
with the given index on the provided DC. If function is not implemented,
the item text is simply drawn, as if the control was a normal combobox.

\wxheading{Parameters}

\docparam{dc}{The device context to use for drawing}

\docparam{rect}{The bounding rectangle for the item being drawn (DC clipping
region is set to this rectangle before calling this function)}

\docparam{item}{The index of the item to be drawn}

\docparam{flags}{Combines any of the following flag values:}

\twocolwidtha{7cm}%
\begin{twocollist}\itemsep=0pt
\twocolitem{{\tt wxODCB\_PAINTING\_CONTROL}}{Combo control is being
painted, instead of a list item. Argument item may be wxNOT\_FOUND in this
case.}
\twocolitem{{\tt wxODCB\_PAINTING\_SELECTED}}{An item with selection
background is being painted. DC text colour should already be correct.}
\end{twocollist}


\membersection{wxOwnerDrawnComboBox::OnMeasureItem}\label{wxownerdrawncomboboxonmeasureitem}

\constfunc{wxCoord}{OnMeasureItem}{\param{size\_t }{item}}

The derived class may implement this method to return the height of the
specified item (in pixels).

The default implementation returns text height, as if this control was
a normal combobox.


\membersection{wxOwnerDrawnComboBox::OnMeasureItemWidth}\label{wxownerdrawncomboboxonmeasureitemwidth}

\constfunc{wxCoord}{OnMeasureItemWidth}{\param{size\_t }{item}}

The derived class may implement this method to return the width of the
specified item (in pixels). If -1 is returned, then the item text width
is used.

The default implementation returns -1.

