\section{\class{wxDataInputStream}}\label{wxdatainputstream}

This class provides functions that read binary data types in a
portable way. Data can be read in either big-endian or little-endian
format, little-endian being the default on all architectures.

If you want to read data from text files (or streams) use 
\helpref{wxTextInputStream}{wxtextinputstream} instead.

The >> operator is overloaded and you can use this class like a standard C++ iostream.
Note, however, that the arguments are the fixed size types wxUint32, wxInt32 etc
and on a typical 32-bit computer, none of these match to the "long" type (wxInt32
is defined as signed int on 32-bit architectures) so that you cannot use long. To avoid
problems (here and elsewhere), make use of the wxInt32, wxUint32, etc types.

For example:

\begin{verbatim}
  wxFileInputStream input( "mytext.dat" );
  wxDataInputStream store( input );
  wxUint8 i1;
  float f2;
  wxString line;

  store >> i1;       // read a 8 bit integer.
  store >> i1 >> f2; // read a 8 bit integer followed by float.
  store >> line;     // read a text line
\end{verbatim}

See also \helpref{wxDataOutputStream}{wxdataoutputstream}. 

\wxheading{Derived from}

None

\wxheading{Include files}

<wx/datstrm.h>

\latexignore{\rtfignore{\wxheading{Members}}}

\membersection{wxDataInputStream::wxDataInputStream}\label{wxdatainputstreamconstr}

\func{}{wxDataInputStream}{\param{wxInputStream\&}{ stream}}

\func{}{wxDataInputStream}{\param{wxInputStream\&}{ stream}, \param{wxMBConv\&}{ conv = wxMBConvUTF8}}

Constructs a datastream object from an input stream. Only read methods will
be available. The second form is only available in Unicode build of wxWindows.

\wxheading{Parameters}

\docparam{stream}{The input stream.}

\docparam{conv}{Charset conversion object object used to decode strings in Unicode 
mode (see \helpref{wxDataInputStream::ReadString}{wxdatainputstreamreadstring}
documentation for detailed description). Note that you must not destroy
{\it conv} before you destroy this wxDataInputStream instance!}

\membersection{wxDataInputStream::\destruct{wxDataInputStream}}

\func{}{\destruct{wxDataInputStream}}{\void}

Destroys the wxDataInputStream object.

\membersection{wxDataInputStream::BigEndianOrdered}

\func{void}{BigEndianOrdered}{\param{bool}{ be\_order}}

If {\it be\_order} is TRUE, all data will be read in big-endian
order, such as written by programs on a big endian architecture 
(e.g. Sparc) or written by Java-Streams (which always use 
big-endian order).
  
\membersection{wxDataInputStream::Read8}

\func{wxUint8}{Read8}{\void}

Reads a single byte from the stream.

\membersection{wxDataInputStream::Read16}

\func{wxUint16}{Read16}{\void}

Reads a 16 bit unsigned integer from the stream.

\membersection{wxDataInputStream::Read32}

\func{wxUint32}{Read32}{\void}

Reads a 32 bit unsigned integer from the stream.

\membersection{wxDataInputStream::Read64}

\func{wxUint64}{Read64}{\void}

Reads a 64 bit unsigned integer from the stream.

\membersection{wxDataInputStream::ReadDouble}

\func{double}{ReadDouble}{\void}

Reads a double (IEEE encoded) from the stream.

\membersection{wxDataInputStream::ReadString}\label{wxdatainputstreamreadstring}

\func{wxString}{ReadString}{\void}

Reads a string from a stream. Actually, this function first reads a long 
integer specifying the length of the string (without the last null character) 
and then reads the string.

In Unicode build of wxWindows, the fuction first reads multibyte (char*)
string from the stream and then converts it to Unicode using the {\it conv}
object passed to constructor and returns the result as wxString. You are
responsible for using the same convertor as when writing the stream.

See also \helpref{wxDataOutputStream::WriteString}{wxdataoutputstreamwritestring}.

