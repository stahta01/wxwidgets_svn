\section{\class{wxFile}}\label{wxfile}

A wxFile performs raw file I/O. This is a very small class designed to
minimize the overhead of using it - in fact, there is hardly any overhead at
all, but using it brings you automatic error checking and hides differences
between platforms and compilers. wxFile also automatically closes the file in
its destructor making it unnecessary to worry about forgetting to do it.
wxFile is a wrapper around {\tt file descriptor.} - see also 
\helpref{wxFFile}{wxffile} for a wrapper around {\tt FILE} structure.

\wxheading{Derived from}

None.

\wxheading{Include files}

<wx/file.h>

\wxheading{Constants}

wx/file.h defines the following constants:

\begin{verbatim}
#define wxS_IRUSR 00400
#define wxS_IWUSR 00200
#define wxS_IXUSR 00100

#define wxS_IRGRP 00040
#define wxS_IWGRP 00020
#define wxS_IXGRP 00010

#define wxS_IROTH 00004
#define wxS_IWOTH 00002
#define wxS_IXOTH 00001

// default mode for the new files: corresponds to umask 022
#define wxS_DEFAULT  (wxS_IRUSR | wxS_IWUSR | wxS_IRGRP | wxS_IWGRP | wxS_IROTH | wxS_IWOTH)
\end{verbatim}

These constants define the file access rights and are used with 
\helpref{wxFile::Create}{wxfilecreate} and \helpref{wxFile::Open}{wxfileopen}.

The {\it OpenMode} enumeration defines the different modes for opening a file,
it is defined inside wxFile class so its members should be specified with {\it wxFile::} scope
resolution prefix. It is also used with \helpref{wxFile::Access}{wxfileaccess} function.

\twocolwidtha{7cm}
\begin{twocollist}\itemsep=0pt%
\twocolitem{{\bf wxFile::read}}{Open file for reading or test if it can be opened for reading with Access()}
\twocolitem{{\bf wxFile::write}}{Open file for writing deleting the contents of the file if it already exists
or test if it can be opened for writing with Access()}
\twocolitem{{\bf wxFile::read\_write}}{Open file for reading and writing; can not be used with Access()}
\twocolitem{{\bf wxFile::write\_append}}{Open file for appending: the file is opened for writing, but the old
contents of the file is not erased and the file pointer is initially placed at the end of the file;
can not be used with Access(). This is the same as {\bf wxFile::write} if the
file doesn't exist.}
\end{twocollist}

Other constants defined elsewhere but used by wxFile functions are wxInvalidOffset which represents an
invalid value of type {\it off\_t} and is returned by functions returning {\it off\_t} on error and the seek
mode constants used with \helpref{Seek()}{wxfileseek}:

\twocolwidtha{7cm}
\begin{twocollist}\itemsep=0pt%
\twocolitem{{\bf wxFromStart}}{Count offset from the start of the file}
\twocolitem{{\bf wxFromCurrent}}{Count offset from the current position of the file pointer}
\twocolitem{{\bf wxFromEnd}}{Count offset from the end of the file (backwards)}
\end{twocollist}

\latexignore{\rtfignore{\wxheading{Members}}}

\membersection{wxFile::wxFile}\label{wxfileconstr}

\func{}{wxFile}{\void}

Default constructor.

\func{}{wxFile}{\param{const char*}{ filename}, \param{wxFile::OpenMode}{ mode = wxFile::read}}

Opens a file with the given mode. As there is no way to return whether the
operation was successful or not from the constructor you should test the
return value of \helpref{IsOpened}{wxfileisopened} to check that it didn't
fail.

\func{}{wxFile}{\param{int}{ fd}}

Associates the file with the given file descriptor, which has already been opened.

\wxheading{Parameters}

\docparam{filename}{The filename.}

\docparam{mode}{The mode in which to open the file. May be one of {\bf wxFile::read}, {\bf wxFile::write} and {\bf wxFile::read\_write}.}

\docparam{fd}{An existing file descriptor (see \helpref{Attach()}{wxfileattach} for the list of predefined descriptors)}

\membersection{wxFile::\destruct{wxFile}}

\func{}{\destruct{wxFile}}{\void}

Destructor will close the file.

NB: it is not virtual so you should use wxFile polymorphically.

\membersection{wxFile::Access}\label{wxfileaccess}

\func{static bool}{Access}{\param{const char *}{ name}, \param{OpenMode}{ mode}}

This function verifies if we may access the given file in specified mode. Only
values of wxFile::read or wxFile::write really make sense here.

\membersection{wxFile::Attach}\label{wxfileattach}

\func{void}{Attach}{\param{int}{ fd}}

Attaches an existing file descriptor to the wxFile object. Example of predefined
file descriptors are 0, 1 and 2 which correspond to stdin, stdout and stderr (and
have symbolic names of {\bf wxFile::fd\_stdin}, {\bf wxFile::fd\_stdout} and {\bf wxFile::fd\_stderr}).

The descriptor should be already opened and it will be closed by wxFile
object.

\membersection{wxFile::Close}\label{wxfileclose}

\func{void}{Close}{\void}

Closes the file.

\membersection{wxFile::Create}\label{wxfilecreate}

\func{bool}{Create}{\param{const char*}{ filename}, \param{bool}{ overwrite = FALSE}, \param{int }{access = wxS\_DEFAULT}}

Creates a file for writing. If the file already exists, setting {\bf overwrite} to TRUE
will ensure it is overwritten.

\membersection{wxFile::Detach}\label{wxfiledetach}

\func{void}{Detach}{\void}

Get back a file descriptor from wxFile object - the caller is responsible for closing the file if this
descriptor is opened. \helpref{IsOpened()}{wxfileisopened} will return FALSE after call to Detach().

\membersection{wxFile::fd}\label{wxfilefd}

\constfunc{int}{fd}{\void}

Returns the file descriptor associated with the file.

\membersection{wxFile::Eof}\label{wxfileeof}

\constfunc{bool}{Eof}{\void}

Returns TRUE if the end of the file has been reached.

Note that the behaviour of the file pointer based class 
\helpref{wxFFile}{wxffile} is different as \helpref{wxFFile::Eof}{wxffileeof} 
will return TRUE here only if an attempt has been made to read 
{\it past} the last byte of the file, while wxFile::Eof() will return TRUE
even before such attempt is made if the file pointer is at the last position
in the file.

Note also that this function doesn't work on unseekable file descriptors
(examples include pipes, terminals and sockets under Unix) and an attempt to
use it will result in an error message in such case. So, to read the entire
file into memory, you should write a loop which uses 
\helpref{Read}{wxfileread} repeatedly and tests its return condition instead
of using Eof() as this will not work for special files under Unix.

\membersection{wxFile::Exists}\label{wxfileexists}

\func{static bool}{Exists}{\param{const char*}{ filename}}

Returns TRUE if the given name specifies an existing regular file (not a
directory or a link)

\membersection{wxFile::Flush}\label{wxfileflush}

\func{bool}{Flush}{\void}

Flushes the file descriptor.

Note that wxFile::Flush is not implemented on some Windows compilers
due to a missing fsync function, which reduces the usefulness of this function
(it can still be called but it will do nothing on unsupported compilers).

\membersection{wxFile::IsOpened}\label{wxfileisopened}

\constfunc{bool}{IsOpened}{\void}

Returns TRUE if the file has been opened.

\membersection{wxFile::Length}\label{wxfilelength}

\constfunc{off\_t}{Length}{\void}

Returns the length of the file.

\membersection{wxFile::Open}\label{wxfileopen}

\func{bool}{Open}{\param{const char*}{ filename}, \param{wxFile::OpenMode}{ mode = wxFile::read}}

Opens the file, returning TRUE if successful.

\wxheading{Parameters}

\docparam{filename}{The filename.}

\docparam{mode}{The mode in which to open the file. May be one of {\bf wxFile::read}, {\bf wxFile::write} and {\bf wxFile::read\_write}.}

\membersection{wxFile::Read}\label{wxfileread}

\func{off\_t}{Read}{\param{void*}{ buffer}, \param{off\_t}{ count}}

Reads the specified number of bytes into a buffer, returning the actual number read.

\wxheading{Parameters}

\docparam{buffer}{A buffer to receive the data.}

\docparam{count}{The number of bytes to read.}

\wxheading{Return value}

The number of bytes read, or the symbol {\bf wxInvalidOffset} (-1) if there was an error.

\membersection{wxFile::Seek}\label{wxfileseek}

\func{off\_t}{Seek}{\param{off\_t }{ofs}, \param{wxSeekMode }{mode = wxFromStart}}

Seeks to the specified position.

\wxheading{Parameters}

\docparam{ofs}{Offset to seek to.}

\docparam{mode}{One of {\bf wxFromStart}, {\bf wxFromEnd}, {\bf wxFromCurrent}.}

\wxheading{Return value}

The actual offset position achieved, or wxInvalidOffset on failure.

\membersection{wxFile::SeekEnd}\label{wxfileseekend}

\func{off\_t}{SeekEnd}{\param{off\_t }{ofs = 0}}

Moves the file pointer to the specified number of bytes before the end of the file.

\wxheading{Parameters}

\docparam{ofs}{Number of bytes before the end of the file.}

\wxheading{Return value}

The actual offset position achieved, or wxInvalidOffset on failure.

\membersection{wxFile::Tell}\label{wxfiletell}

\constfunc{off\_t}{Tell}{\void}

Returns the current position or wxInvalidOffset if file is not opened or if another
error occurred.

\membersection{wxFile::Write}\label{wxfilewrite}

\func{size\_t}{Write}{\param{const void*}{ buffer}, \param{off\_t}{ count}}

Writes the specified number of bytes from a buffer.

\wxheading{Parameters}

\docparam{buffer}{A buffer containing the data.}

\docparam{count}{The number of bytes to write.}

\wxheading{Return value}

the number of bytes actually written

\membersection{wxFile::Write}\label{wxfilewrites}

\func{bool}{Write}{\param{const wxString\& }{s}}

Writes the contents of the string to the file, returns TRUE on success.

\section{\class{wxFFile}}\label{wxffile}

wxFFile implements buffered file I/O. This is a very small class designed to
minimize the overhead of using it - in fact, there is hardly any overhead at
all, but using it brings you automatic error checking and hides differences
between platforms and compilers. It wraps inside it a {\tt FILE *} handle used
by standard C IO library (also known as {\tt stdio}).

\wxheading{Derived from}

None.

\wxheading{Include files}

<wx/ffile.h>

\twocolwidtha{7cm}
\begin{twocollist}\itemsep=0pt%
\twocolitem{{\bf wxFromStart}}{Count offset from the start of the file}
\twocolitem{{\bf wxFromCurrent}}{Count offset from the current position of the file pointer}
\twocolitem{{\bf wxFromEnd}}{Count offset from the end of the file (backwards)}
\end{twocollist}

\latexignore{\rtfignore{\wxheading{Members}}}

\membersection{wxFFile::wxFFile}\label{wxffileconstr}

\func{}{wxFFile}{\void}

Default constructor.

\func{}{wxFFile}{\param{const char*}{ filename}, \param{const char*}{ mode = "r"}}

Opens a file with the given mode. As there is no way to return whether the
operation was successful or not from the constructor you should test the
return value of \helpref{IsOpened}{wxffileisopened} to check that it didn't
fail.

\func{}{wxFFile}{\param{FILE*}{ fp}}

Opens a file with the given file pointer, which has already been opened.

\wxheading{Parameters}

\docparam{filename}{The filename.}

\docparam{mode}{The mode in which to open the file using standard C strings.}

\docparam{fp}{An existing file descriptor, such as stderr.}

\membersection{wxFFile::\destruct{wxFFile}}

\func{}{\destruct{wxFFile}}{\void}

Destructor will close the file.

NB: it is not virtual so you should {\it not} derive from wxFFile!

\membersection{wxFFile::Attach}\label{wxffileattach}

\func{void}{Attach}{\param{FILE*}{ fp}}

Attaches an existing file pointer to the wxFFile object.

The descriptor should be already opened and it will be closed by wxFFile
object.

\membersection{wxFFile::Close}\label{wxffileclose}

\func{bool}{Close}{\void}

Closes the file and returns TRUE on success.

\membersection{wxFFile::Detach}\label{wxffiledetach}

\func{void}{Detach}{\void}

Get back a file pointer from wxFFile object - the caller is responsible for closing the file if this
descriptor is opened. \helpref{IsOpened()}{wxffileisopened} will return FALSE after call to Detach().

\membersection{wxFFile::fp}\label{wxffilefp}

\constfunc{FILE *}{fp}{\void}

Returns the file pointer associated with the file.

\membersection{wxFFile::Eof}\label{wxffileeof}

\constfunc{bool}{Eof}{\void}

Returns TRUE if the an attempt has been made to read {\it past}
the end of the file. 

Note that the behaviour of the file descriptor based class
\helpref{wxFile}{wxfile} is different as \helpref{wxFile::Eof}{wxfileeof}
will return TRUE here as soon as the last byte of the file has been
read.

\membersection{wxFFile::Flush}\label{wxffileflush}

\func{bool}{Flush}{\void}

Flushes the file and returns TRUE on success.

\membersection{wxFFile::IsOpened}\label{wxffileisopened}

\constfunc{bool}{IsOpened}{\void}

Returns TRUE if the file has been opened.

\membersection{wxFFile::Length}\label{wxffilelength}

\constfunc{size\_t}{Length}{\void}

Returns the length of the file.

\membersection{wxFFile::Open}\label{wxffileopen}

\func{bool}{Open}{\param{const char*}{ filename}, \param{const char*}{ mode = "r"}}

Opens the file, returning TRUE if successful.

\wxheading{Parameters}

\docparam{filename}{The filename.}

\docparam{mode}{The mode in which to open the file.}

\membersection{wxFFile::Read}\label{wxffileread}

\func{size\_t}{Read}{\param{void*}{ buffer}, \param{off\_t}{ count}}

Reads the specified number of bytes into a buffer, returning the actual number read.

\wxheading{Parameters}

\docparam{buffer}{A buffer to receive the data.}

\docparam{count}{The number of bytes to read.}

\wxheading{Return value}

The number of bytes read.

\membersection{wxFFile::Seek}\label{wxffileseek}

\func{bool}{Seek}{\param{long }{ofs}, \param{wxSeekMode }{mode = wxFromStart}}

Seeks to the specified position and returs TRUE on success.

\wxheading{Parameters}

\docparam{ofs}{Offset to seek to.}

\docparam{mode}{One of {\bf wxFromStart}, {\bf wxFromEnd}, {\bf wxFromCurrent}.}

\membersection{wxFFile::SeekEnd}\label{wxffileseekend}

\func{bool}{SeekEnd}{\param{long }{ofs = 0}}

Moves the file pointer to the specified number of bytes before the end of the file
and returns TRUE on success.

\wxheading{Parameters}

\docparam{ofs}{Number of bytes before the end of the file.}

\membersection{wxFFile::Tell}\label{wxffiletell}

\constfunc{size\_t}{Tell}{\void}

Returns the current position.

\membersection{wxFFile::Write}\label{wxffilewrite}

\func{size\_t}{Write}{\param{const void*}{ buffer}, \param{size\_t}{ count}}

Writes the specified number of bytes from a buffer.

\wxheading{Parameters}

\docparam{buffer}{A buffer containing the data.}

\docparam{count}{The number of bytes to write.}

\wxheading{Return value}

Number of bytes written.

\membersection{wxFFile::Write}\label{wxffilewrites}

\func{bool}{Write}{\param{const wxString\& }{s}}

Writes the contents of the string to the file, returns TRUE on success.

