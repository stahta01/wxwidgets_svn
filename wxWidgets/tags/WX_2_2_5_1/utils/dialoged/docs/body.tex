\chapter{Introduction}\label{introduction}
\pagenumbering{arabic}%
\setheader{{\it CHAPTER \thechapter}}{}{}{}{}{{\it CHAPTER \thechapter}}%
\setfooter{\thepage}{}{}{}{}{\thepage}%

The wxWindows Dialog Editor is a tool for creating dialog resource files, in {\tt .wxr} format.
It differs from wxBuilder in the following respectes:

\begin{enumerate}\itemsep=0pt
\item Scope. It is written for dialog editing only, and is therefore more convenient than wxBuilder for this purpose.
\item File format. Dialog editor reads and writes wxWindows resource files (extension {\tt .wxr}) and has
no independent file format.
\item Robustness. It is written in a more principled way than wxBuilder, and is less ambitious.
\item Ease of use. Windows are edited using the mouse or via consistent {\it property editors}, which
provide immediate visual feedback of changed properties.
\end{enumerate}

Dialog Editor 2.0 should be compiled and used with wxWindows 2.0.

\section{Current status}

Dialog Editor currently runs under wxMSW and wxGTK. It has yet to
be tested under wxMotif.

\section{Future developments}

\begin{itemize}\itemsep=0pt
\item Motif compilation.
\item It would be nice to have a dialog browser, showing thumbnails of
all dialogs in a particular directory.
\item Maybe add a menubar editor (from wxBuilder).
\item Maybe convert Windows .rc files.
\end{itemize}

\chapter{Commands}\label{commands}
\setheader{{\it CHAPTER \thechapter}}{}{}{}{}{{\it CHAPTER \thechapter}}%
\setfooter{\thepage}{}{}{}{}{\thepage}%

\section{Dialog editor menu bar}

\subsection{File menu}

\begin{twocollist}\itemsep=0pt
\twocolitem{New Dialog}{Creates a new dialog resource.}
\twocolitem{New Project}{Creates a new project (clears index and resets project name).}
\twocolitem{Open...}{Opens an existing resource file.}
\twocolitem{Save}{Saves the current resources.}
\twocolitem{Save As...}{Saves the current resources in a named file.}
\twocolitem{Clear}{Clears the current resources.}
\twocolitem{Convert Old Resources...}{Takes a directory of wxWindows 1.68 dialog resources,
and converts them to wxWindows 2 resources, in a separate directory. See 
\helpref{Converting old files}{convertingoldfiles}.}
\twocolitem{Exit}{Exits the program.}
\end{twocollist}

\subsection{Edit menu}

\begin{twocollist}\itemsep=0pt
\twocolitem{Test Dialog}{Creates the current dialog for test purposes.}
\twocolitem{Recreate}{Recreates the currently selected control from the underlying resource. This may be necessary
to regenerate items that cannot be changed dynamically, and which have got out of sync with the displayed
item.}
\twocolitem{Delete}{Deletes the currently selected resource.}
\end{twocollist}

\subsection{Help menu}

\begin{twocollist}\itemsep=0pt
\twocolitem{Help Topics}{Displays on-line help at the contents page.}
\twocolitem{About}{Displays an dialog showing the Dialog Editor version and author.}
\end{twocollist}

\latexonly{\newpage}
\section{Command toolbar}

The command toolbar consists of the following tools:

\begin{twocollist}%\itemsep=0pt
\twocolitem{\icon{new.eps}{New}}{Clears the project.}
\twocolitem{\icon{open.eps}{Open}}{Opens an existing resource file.}
\twocolitem{\icon{save.eps}{Save}}{Saves the current resources.}
\twocolitem{\icon{vert.eps}{Horizontal align}}{Aligns the centre of the selected controls horizontally.}
\twocolitem{\icon{alignt.eps}{Horizontal top-align}}{Aligns the top sides of the selected controls horizontally.}
\twocolitem{\icon{alignb.eps}{Horizontal bottom-align}}{Aligns the bottom sides of the selected controls horizontally.}
\twocolitem{\icon{horiz.eps}{Vertical align}}{Aligns the centre of the selected controls vertically.}
\twocolitem{\icon{alignl.eps}{Vertical left-align}}{Aligns the left sides of the selected controls vertically.}
\twocolitem{\icon{alignr.eps}{Vertical right-align}}{Aligns the right sides of the selected controls vertically.}
\twocolitem{\icon{copysize.eps}{Copy size}}{Copies the size of the first selected control to the subsequently selected control(s).}
\twocolitem{\icon{copywdth.eps}{Copy width}}{Copies the width of the first selected control to the subsequently selected control(s).}
\twocolitem{\icon{copyhght.eps}{Copy height}}{Copies the height of the first selected control to the subsequently selected control(s).}
\twocolitem{\icon{disthor.eps}{Distribute horizontally}}{Evenly distributes the space between the selected controls, horizontally. Note that the controls
should be selected in order from left to right.}
\twocolitem{\icon{distvert.eps}{Distribute vertically}}{Evenly distributes the space between the selected controls, vertically. Note that the controls
should be selected in order from top to bottom.}
\twocolitem{\icon{tofront.eps}{To front}}{Puts the selected control(s) to the front of the display list.}
\twocolitem{\icon{toback.eps}{To back}}{Puts the selected control(s) to the back of the display list.}
\twocolitem{\icon{help.eps}{Help}}{Invokes Dialog Editor help.}
\end{twocollist}

\latexonly{\newpage}
\section{Tool palette}

The tool palette is used to select a type of control to create on the dialog.
To create a new control, select a tool with left-click, then left-click on the dialog.
Select the pointer tool to use left-click for selecting and deselecting
items.

\section{Resource tree}

The resource tree shows a list of the dialogs, controls and bitmaps currently loaded
in Dialog Editor. Double-clicking on an item shows the associated resource.

\chapter{Procedures}\label{procedures}
\setheader{{\it CHAPTER \thechapter}}{}{}{}{}{{\it CHAPTER \thechapter}}%
\setfooter{\thepage}{}{}{}{}{\thepage}%

\section{Running Dialog Editor}

To run Dialog Editor under Windows, click on the Program Manager or Explorer icon.
Under UNIX, run from the command line.

The main window shows a menu bar, command toolbar, tool palette, resource list, and
status line.

\section{Creating a dialog}

To create a new dialog, click on the {\bf File: New} menu item, or equivalent
toolbar button. A dialog will appear. To put a control on the dialog, left-click
on the appropriate palette icon and then left-click on the dialog. A new item
will appear at the place you clicked.

You can edit any control or dialog by control-left clicking. A property editor
will appear, allowing any property to be selected and edited (see \helpref{Using property editors}{propeditors}).
You can also edit items by right-clicking to show a menu, and then selecting {\it Edit properties}.

To move a control, drag the item with the left mouse button, or edit
the position values in the property editor. To resize a control, you
can either select it by left-clicking and then dragging on a selection
handle, or edit the size values in the property editor. 

You can delete items from the right-click menu, or by selecting the item and
choosing {\bf Edit: Delete} from the menu bar.

\section{Using property editors}\label{propeditors}

Property editors consist of a list of properties and current values, plus controls at the top of
the editor. If the property is of an appropriate type, you can edit the value directly in the
text field, and confirm or cancel the value using the two buttons to the left of it.
If the property has a predefined range of values, such as labelFontFamily, you can
see a list of permissable values by clicking on the button labelled with an ellipsis symbol ({\bf ...}).
This will show a listbox with possible values and current selection. You may also be able
to cycle through values by double-clicking the value in the listbox.

Properties may have special editors appropriate to the type. Filename properties invoke
the file selector, and properties containing list of user-definable strings use a
string editor.

When you change a property value, this value is immediately reflected in
the dialog or control.  If the item allows this value to be changed
dynamically, the relevant wxWindows function will be called internally
to effect the change.  If the value cannot be changed dynamically, the
item will be destroyed and re-created, which means that there will be
more flickering associated with some kinds of property changes than
others.

\section{Saving and loading files}

Use {\it File: Save} and {\it File: Save as} or the equivalent toolbar button
to save the current dialog(s) in a wxWindows resource file (extension {\tt .wxr}).

The {\tt .wxr} file can be used directly in a wxWindows program, if
wxWindows resources have been enabled when building the wxWindows library.
These files can be loaded dynamically, or included directly into program source
with a \verb$#include$ directive. See the wxWindows user manual for further details.

\section{Working with identifiers}

Dialog Editor keeps track of identifiers in your resources, and reads and writes an include file of the
form {\tt name.h} where 'name' is the root name of your {\tt .wxr} file. Dialog Editor
knows about the predefined identifiers such as wxID\_OK.

When you create a dialog or control, the identifier is initially generated. When you
edit the identifier via a property editor, you can choose a new name, such as a predefined
symbol and optionally change the integer assigned to the name (assuming it's not a
predefined symbol).

When you save the project, the identifier include file is saved as well. Include this file
in your project so that you can refer to controls and dialogs by identifier rather than
obscure integers. Note that the {\tt .wxr} file itself can only contain integer ids and not the symbols,
due to way in which the resource file is loaded.

\section{Multi-platform development}

{\tt .wxr} files generated on one environment (e.g. Windows) can be used in another (e.g. GTK).
If you use default fonts and colouring (set {\bf useSystemDefaults} to True in the dialog properties)
then the dialog fonts and colours will take on the native values, rather than ones specified in the
resource. Without this, colours in the dialog resource may not match system colours.

Also, set {\bf useDialogUnits} to True whenever possible since this will cause the dialog
to be created using a scale based on the current system font size, and will result in dialogs that are
portable between screen resolutions as well as platforms.

Because the same control can have different sizes on different GUIs,
the user should be cautious in assuming that one resource file will work for all
platforms. It may be better to plan to conditionally include or load different
resource files for different platforms, with spacing modified to suit each
environment. The best thing is to try your dialog resource on several platforms
and see whether tweaking is required for some platforms.

\section{Converting old files}\label{convertingoldfiles}

Dialog Editor can make an attempt at converting dialog resources created with Dialog Editor for wxWindows 1.68.
The command is {\bf Convert Old Resources...} on the {\bf File} menu.

You need to specify two directories, an input and an output directory. Dialog Editor will
do the following conversions:

\begin{enumerate}\itemsep=0pt
\item wxMultiText becomes a wxTextCtrl with wxTE\_MULTILINE style.
\item wxText becomes a wxTextCtrl.
\item wxMessage becomes either a wxStaticText or wxStaticBitmap.
\item wxButton becomes a wxBitmapButton if necessary.
\item wxGroupBox becomes wxStaticBox.
\item Controls that no longer have labels, such as wxTextCtrl and wxListBox,
have a separate wxStaticText control created for them at approximately the correct
position. The label's window name becomes ControlName_Label where ControlName is
the name of the control that formerly had the label.
\item Identifiers are allocated.
\item Font sizes are reduced to counter the decreased font size now created by wxWindows
for a given point size.
\item The dialog height is reduced slightly to compensate for the fact that the dialog caption
is no longer included in the size.
\end{enumerate}

