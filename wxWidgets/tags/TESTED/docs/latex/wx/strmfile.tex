% -----------------------------------------------------------------------------
% wxFileInputStream
% -----------------------------------------------------------------------------
\section{\class{wxFileInputStream}}\label{wxfileinputstream}

This class represents data read in from a file. There are actually
two such groups of classes: those documented here, and another group called
wxFFileInputStream, wxFFileOutputStream and wxFFileStream which are not
based on file descriptors (and their wxWindows equivalent wxFile) but the
FILE* type (and wxFFile). Apart from the different constructor ("FILE *file"
instead if "int fd") their interface is identical.

\wxheading{Derived from}

\helpref{wxInputStream}{wxinputstream}

\wxheading{Include files}

<wx/wfstream.h>

\wxheading{See also}

\helpref{wxStreamBuffer}{wxstreamBuffer}, \helpref{wxFileOutputStream}{wxfileoutputstream}

% ----------
% Members
% ----------
\latexignore{\rtfignore{\wxheading{Members}}}

\membersection{wxFileInputStream::wxFileInputStream}

\func{}{wxFileInputStream}{\param{const wxString\&}{ ifileName}}

Opens the specified file using its \it{ifilename} name in read-only mode.

\func{}{wxFileInputStream}{\param{wxFile\&}{ file}}

Initializes a file stream in read-only mode using the file I/O object \it{file}.

\func{}{wxFileInputStream}{\param{int}{ fd}}

Initializes a file stream in read-only mode using the specified file descriptor.

\membersection{wxFileInputStream::\destruct{wxFileInputStream}}

\func{}{\destruct{wxFileInputStream}}{\void}

Destructor.

\membersection{wxFileInputStream::Ok}

\constfunc{bool}{Ok}{\void}

Returns TRUE if the stream is initialized and ready.

% -----------------------------------------------------------------------------
% wxFileOutputStream
% -----------------------------------------------------------------------------
\section{\class{wxFileOutputStream}}\label{wxfileoutputstream}

This class represents data written to a file. There are actually
two such groups of classes: those documented here, and another group called
wxFFileInputStream, wxFFileOutputStream and wxFFileStream which are not
based on file descriptors (and their wxWindows equivalent wxFile) but the
FILE* type (and wxFFile). Apart from the different constructor ("FILE *file"
instead if "int fd") their interface is identical.

\wxheading{Derived from}

\helpref{wxOutputStream}{wxoutputstream}

\wxheading{Include files}

<wx/wfstream.h>

\wxheading{See also}

\helpref{wxStreamBuffer}{wxstreamBuffer}, \helpref{wxFileInputStream}{wxfileinputstream}

% ----------
% Members
% ----------
\latexignore{\rtfignore{\wxheading{Members}}}

\membersection{wxFileOutputStream::wxFileOutputStream}

\func{}{wxFileOutputStream}{\param{const wxString\&}{ ofileName}}

Creates a new file with \it{ofilename} name and initializes the stream in
write-only mode. 

\func{}{wxFileOutputStream}{\param{wxFile\&}{ file}}

Initializes a file stream in write-only mode using the file I/O object \it{file}.

\func{}{wxFileOutputStream}{\param{int}{ fd}}

Initializes a file stream in write-only mode using the file descriptor \it{fd}.

\membersection{wxFileOutputStream::\destruct{wxFileOutputStream}}

\func{}{\destruct{wxFileOutputStream}}{\void}

Destructor.

\membersection{wxFileOutputStream::Ok}

\constfunc{bool}{Ok}{\void}

Returns TRUE if the stream is initialized and ready.

% -----------------------------------------------------------------------------
% wxFileStream
% -----------------------------------------------------------------------------
\section{\class{wxFileStream}}

\wxheading{Derived from}

\helpref{wxFileOutputStream}{wxFileOutputStream}, \helpref{wxFileInputStream}{wxfileinputstream}

\wxheading{Include files}

<wx/wfstream.h>

\wxheading{See also}

\helpref{wxStreamBuffer}{wxstreamBuffer}

\latexignore{\rtfignore{\wxheading{Members}}}

\membersection{wxFileStream::wxFileStream}

\func{}{wxFileStream}{\param{const wxString\&}{ iofileName}}

Initializes a new file stream in read-write mode using the specified 
\it{iofilename} name.

