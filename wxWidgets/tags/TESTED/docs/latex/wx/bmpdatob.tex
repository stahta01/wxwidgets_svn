\section{\class{wxBitmapDataObject}}\label{wxbitmapdataobject}

wxBitmapDataObject is a specialization of wxDataObject for bitmap data. It can
be used without change to paste data into the
\helpref{wxClipboard}{wxclipboard} or a \helpref{wxDropSource}{wxdropsource}. A
user may wish to derive a new class from this class for providing a bitmap
on-demand in order to minimize memory consumption when offering data in several
formats, such as a bitmap and GIF.

\pythonnote{If you wish to create a derived wxBitmapDataObject class in
wxPython you should derive the class from wxPyBitmapDataObject
in order to get Python-aware capabilities for the various virtual
methods.}

\wxheading{Virtual functions to override}

This class may be used as is, but
\helpref{GetBitmap}{wxbitmapdataobjectgetbitmap} may be overridden to increase
efficiency.

\wxheading{Derived from}

\helpref{wxDataObjectSimple}{wxdataobjectsimple}\\
\helpref{wxDataObject}{wxdataobject}

\wxheading{Include files}

<wx/dataobj.h>

\wxheading{See also}

\helpref{Clipboard and drag and drop overview}{wxdndoverview},
\helpref{wxDataObject}{wxdataobject},
\helpref{wxDataObjectSimple}{wxdataobjectsimple},
\helpref{wxFileDataObject}{wxfiledataobject},
\helpref{wxTextDataObject}{wxtextdataobject},
\helpref{wxDataObject}{wxdataobject}

\func{}{wxBitmapDataObject}{\param{const wxBitmap\& }{bitmap = wxNullBitmap}}

Constructor, optionally passing a bitmap (otherwise use
\helpref{SetBitmap}{wxbitmapdataobjectsetbitmap} later)

\membersection{wxBitmapDataObject::GetBitmap}\label{wxbitmapdataobjectgetbitmap}

\constfunc{virtual wxBitmap}{GetBitmap}{\void}

Returns the bitmap associated with the data object. You may wish to override
this method when offering data on-demand, but this is not required by
wxWindows' internals. Use this method to get data in bitmap form from
the \helpref{wxClipboard}{wxclipboard}.

\membersection{wxBitmapDataObject::SetBitmap}\label{wxbitmapdataobjectsetbitmap}

\func{virtual void}{SetBitmap}{\param{const wxBitmap\& }{bitmap}}

Sets the bitmap associated with the data object. This method is called when the
data object receives data. Usually there will be no reason to override this
function.

