\section{\class{wxProcessEvent}}\label{wxprocessevent}

A process event is sent when a process is terminated.

\wxheading{Derived from}

\helpref{wxEvent}{wxevent}\\
\helpref{wxObject}{wxobject}

\wxheading{Include files}

<wx/process.h>

\wxheading{Event table macros}

To process a wxProcessEvent, use these event handler macros to direct input to a member
function that takes a wxProcessEvent argument.

\twocolwidtha{7cm}
\begin{twocollist}\itemsep=0pt
\twocolitem{{\bf EVT\_END\_PROCESS(id, func)}}{Process a wxEVT\_END\_PROCESS event.
{\it id} is the identifier of the process object (the id passed to the wxProcess constructor)
or a window to receive the event.}
\end{twocollist}%

\wxheading{See also}

\helpref{wxProcess}{wxprocess},\rtfsp
\helpref{Event handling overview}{eventhandlingoverview}

\latexignore{\rtfignore{\wxheading{Members}}}

\membersection{wxProcessEvent::wxProcessEvent}

\func{}{wxProcessEvent}{\param{int }{id = 0}, \param{int }{pid = 0}}

Constructor. Takes a wxProcessObject or window id, and a process id.

\membersection{wxProcessEvent::m\_pid}

\member{int}{m\_pid}

Contains the process id.

\membersection{wxProcessEvent::GetPid}\label{wxprocesseventgetpid}

\constfunc{int}{GetPid}{\void}

Returns the process id.

\membersection{wxProcessEvent::SetPid}\label{wxprocesseventsetpid}

\func{void}{SetPid}{\param{int}{ pid}}

Sets the process id.

