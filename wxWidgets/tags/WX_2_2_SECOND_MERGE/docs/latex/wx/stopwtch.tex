\section{\class{wxStopWatch}}\label{wxstopwatch}

The wxStopWatch class allow you to measure time intervals.

\wxheading{Include files}

<wx/timer.h>

\wxheading{See also}

\helpref{::wxStartTimer}{wxstarttimer}, \helpref{::wxGetElapsedTime}{wxgetelapsedtime}, \helpref{wxTimer}{wxtimer}

\latexignore{\rtfignore{\wxheading{Members}}}

\membersection{wxStopWatch::wxStopWatch}

\func{}{wxStopWatch}{\void}

Constructor. This starts the stop watch.

\membersection{wxStopWatch::Pause}\label{wxstopwatchpause}

\func{void}{Pause}{\void}

Pauses the stop watch. Call \helpref{wxStopWatch::Resume}{wxstopwatchresume} to resume 
time measuring again.

\membersection{wxStopWatch::Start}

\func{void}{Start}{\param{long}{ milliseconds = 0}}

(Re)starts the stop watch with a given initial value.

\membersection{wxStopWatch::Resume}\label{wxstopwatchresume}

\func{void}{Resume}{\void}

Resumes the stop watch after having been paused with \helpref{wxStopWatch::Pause}{wxstopwatchpause}.

\membersection{wxStopWatch::Time}

\func{long}{Time}{\void}\label{wxstopwatchtime}

Returns the time in milliseconds since the start (or restart) or the last call of 
\helpref{wxStopWatch::Pause}{wxstopwatchpause}.

