\subsection{Help Files Format}\label{helpformat}

wxHTML library uses a reduced version of MS HTML Workshop format.
Tex2RTF can produce these files when generating HTML, if you set {\bf htmlWorkshopFiles} to {\bf true} in
your tex2rtf.ini file.

(See \helpref{wxHtmlHelpController}{wxhtmlhelpcontroller} for help controller description.)

A {\bf book} consists of three files: header file, contents file and index file.
You can make a regular zip archive of these files, plus the HTML and any image files,
for wxHTML (or helpview) to read; and the .zip file can optionally be renamed to .htb.

\wxheading{Header file (.hhp)}

Header file must contain these lines (and may contain additional lines which are ignored) :

\begin{verbatim}
Contents file=<filename.hhc>
Index file=<filename.hhk>
Title=<title of your book>
Default topic=<default page to be displayed.htm>
\end{verbatim}

All filenames (including the Default topic) are relative to the 
location of .hhp file.

{\bf Localization note:} In addition, .hhp file may contain line

\begin{verbatim}
Charset=<rfc_charset>
\end{verbatim}

which specifies what charset (e.g. "iso8859\_1") was used in contents
and index files. Please note that this line is incompatible with
MS HTML Help Workshop and it would either silently remove it or complain
with some error. See also 
\helpref{Writing non-English applications}{nonenglishoverview}.

\wxheading{Contents file (.hhc)}

Contents file has HTML syntax and it can be parsed by regular HTML parser. It contains exactly one list
({\tt <ul>}....{\tt </ul>} statement):

\begin{verbatim}
<ul>

  <li> <object type="text/sitemap">
           <param name="Name" value="@topic name@">
           <param name="ID" value=@numeric_id@>
           <param name="Local" value="@filename.htm@">
       </object>
  <li> <object type="text/sitemap">
           <param name="Name" value="@topic name@">
           <param name="ID" value=@numeric_id@>
           <param name="Local" value="@filename.htm@">
       </object>
  ...    

</ul>
\end{verbatim}

You can modify value attributes of param tags. {\it topic name} is name of chapter/topic as is displayed in
contents, {\it filename.htm} is HTML page name (relative to .hhp file) and {\it numeric\_id} is optional 
- it is used only when you use \helpref{wxHtmlHelpController::Display(int)}{wxhtmlhelpcontrollerdisplay}

Items in the list may be nested - one {\tt <li>} statement may contain a {\tt <ul>} sub-statement:

\begin{verbatim}
<ul>

  <li> <object type="text/sitemap">
           <param name="Name" value="Top node">
           <param name="Local" value="top.htm">
       </object>
       <ul>
         <li> <object type="text/sitemap">
              <param name="Name" value="subnode in topnode">
              <param name="Local" value="subnode1.htm">
              </object>
      ...
       </ul>
       
  <li> <object type="text/sitemap">
           <param name="Name" value="Another Top">
           <param name="Local" value="top2.htm">
       </object>
  ...    

</ul>
\end{verbatim}

\wxheading{Index file (.hhk)}

Index files have same format as contents file except that ID params are ignored and sublists are {\bf not} 
allowed.

