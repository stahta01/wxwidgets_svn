\section{\class{wxBusyInfo}}\label{wxbusyinfo}

This class makes it easy to tell your user that the program is temporarily busy.
Just create a wxBusyInfo object on the stack, and within the current scope,
a message window will be shown.

For example:

\begin{verbatim}
    wxBusyInfo wait("Please wait, working...");

    for (int i = 0; i < 100000; i++)
    {
        DoACalculation();
    }
\end{verbatim}

It works by creating a window in the constructor,
and deleting it in the destructor.

You may also want to call wxTheApp->Yield() to refresh the window
periodically (in case it had been obscured by other windows, for
example) like this:

\begin{verbatim}
    wxWindowDisabler disableAll;

    wxBusyInfo wait("Please wait, working...");

    for (int i = 0; i < 100000; i++)
    {
        DoACalculation();

        if ( !(i % 1000) )
            wxTheApp->Yield();
    }
\end{verbatim}

but take care to not cause undesirable reentrancies when doing it (see 
\helpref{wxApp::Yield()}{wxappyield} for more details). The simplest way to do
it is to use \helpref{wxWindowDisabler}{wxwindowdisabler} class as illustrated
in the above example.

\wxheading{Derived from}

None

\wxheading{Include files}

<wx/busyinfo.h>

\latexignore{\rtfignore{\wxheading{Members}}}

\membersection{wxBusyInfo::wxBusyInfo}\label{wxbusyinfoctor}

\func{}{wxBusyInfo}{\param{const wxString\&}{ msg}, \param{wxWindow* }{parent = NULL}}

Constructs a busy info window as child of {\it parent} and displays {\it msg}
in it.

{\bf NB:} If {\it parent} is not {\tt NULL} you must ensure that it is not
closed while the busy info is shown.

\membersection{wxBusyInfo::\destruct{wxBusyInfo}}\label{wxbusyinfodtor}

\func{}{\destruct{wxBusyInfo}}{\void}

Hides and closes the window containing the information text.

