\section{wxTipProvider overview}\label{tipsoverview}

Many "modern" Windows programs have a feature (some would say annoyance) of
presenting the user tips at program startup. While this is probably useless to
the advanced users of the program, the experience shows that the tips may be
quite helpful for the novices and so more and more programs now do this.

For a wxWindows programmer, implementing this feature is extremely easy. To
show a tip, it is enough to just call \helpref{wxShowTip}{wxshowtip} function
like this:

\begin{verbatim}
    if ( ...show tips at startup?... )
    {
        wxTipProvider *tipProvider = wxCreateFileTipProvider("tips.txt", 0);
        wxShowTip(windowParent, tipProvider);
        delete tipProvider;
    }
\end{verbatim}

Of course, you need to get the text of the tips from somewhere - in the example
above, the text is supposed to be in the file tips.txt from where it is read by
the {\it tip provider}. The tip provider is just an object of a class deriving
from \helpref{wxTipProvider}{wxtipprovider}. It has to implement one pure
virtual function of the base class: \helpref{GetTip}{wxtipprovidergettip}.
In the case of the tip provider created by 
\helpref{wxCreateFileTipProvider}{wxcreatefiletipprovider}, the tips are just
the lines of the text file.

If you want to implement your own tip provider (for example, if you wish to
hardcode the tips inside your program), you just have to derive another class
from wxTipProvider and pass a pointer to the object of this class to wxShowTip
- then you don't need wxCreateFileTipProvider at all.

You will probably want to save somewhere the index of the tip last
shown - so that the program doesn't always show the same tip on startup. As you
also need to remember whether to show tips or not (you shouldn't do it if the
user unchecked "Show tips on startup" checkbox in the dialog), you will
probably want to store both the index of the
last shown tip (as returned by 
\helpref{wxTipProvider::GetCurrentTip}{wxtipprovidergetcurrenttip} and the flag
telling whether to show the tips at startup at all.

In a tips.txt file, lines that begin with a \# character are considered comments 
and are automatically skipped. Blank lines and lines only having spaces are also 
skipped.

You can easily add runtime-translation capacity by placing each line of the 
tips.txt file inside the usual translation macro. For example, your tips.txt 
file would look like this:

\begin{verbatim}
_("This is my first tip")
_("This is my second tip")
\end{verbatim}

Now add your tips.txt file into the list of files that gettext searches 
for translatable strings. The tips will thus get included into your 
generated .po file catalog and be translated at runtime along with the rest of 
your aplication's translatable strings. 
Note1: Each line in the tips.txt file needs to strictly begin with exactly the 
3 characters of underscore-parenthesis-doublequote, and end with 
doublequote-parenthesis, as shown above. 
Note2: Remember to escape any doublequote characters within the tip string with
a backslash-doublequote.

See the dialogs program in your samples folder for a working example inside a 
program.

