\section{\class{wxEraseEvent}}\label{wxeraseevent}

An erase event is sent when a window's background needs to be repainted.

\wxheading{Derived from}

\helpref{wxEvent}{wxevent}\\
\helpref{wxObject}{wxobject}

\wxheading{Include files}

<wx/event.h>

\wxheading{Event table macros}

To process an erase event, use this event handler macro to direct input to a member
function that takes a wxEraseEvent argument.

\twocolwidtha{7cm}
\begin{twocollist}\itemsep=0pt
\twocolitem{{\bf EVT\_ERASE\_BACKGROUND(func)}}{Process a wxEVT\_ERASE\_BACKGROUND event.}
\end{twocollist}%

\wxheading{Remarks}

If the {\bf m\_DC} member is non-NULL, draw into this device context.

\wxheading{See also}

\helpref{wxWindow::OnEraseBackground}{wxwindowonerasebackground}, \helpref{Event handling overview}{eventhandlingoverview}

\latexignore{\rtfignore{\wxheading{Members}}}

\membersection{wxEraseEvent::wxEraseEvent}

\func{}{wxEraseEvent}{\param{int }{id = 0}, \param{wxDC* }{dc = NULL}}

Constructor.

\membersection{wxEraseEvent::m\_dc}

\member{wxDC*}{m\_dc}

The device context associated with the erase event (may be NULL).

\membersection{wxEraseEvent::GetDC}\label{wxeraseeventgetdc}

\constfunc{wxDC*}{GetDC}{\void}

Returns the device context to draw into. If this is non-NULL, you should draw
into it to perform the erase operation.

