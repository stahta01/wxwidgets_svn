\section{\class{wxSpinButton}}\label{wxspinbutton}

A wxSpinButton has two small up and down (or left and right) arrow buttons. It
is often used next to a text control for increment and decrementing a value.
Portable programs should try to use \helpref{wxSpinCtrl}{wxspinctrl} instead
as wxSpinButton is not implemented for all platforms (Win32 and GTK only
currently).

{\bf NB:} the range supported by this control (and wxSpinCtrl) depends on the
platform but is at least {\tt SHRT\_MIN} to {\tt SHRT\_MAX}.

\wxheading{Derived from}

\helpref{wxControl}{wxcontrol}\\
\helpref{wxWindow}{wxwindow}\\
\helpref{wxEvtHandler}{wxevthandler}\\
\helpref{wxObject}{wxobject}

\wxheading{See also}

\helpref{wxSpinCtrl}{wxspinctrl} 

\wxheading{Include files}

<wx/spinbutt.h>

\wxheading{Window styles}

\twocolwidtha{5cm}
\begin{twocollist}\itemsep=0pt
\twocolitem{\windowstyle{wxSP\_HORIZONTAL}}{Specifies a horizontal spin button (note that this style is not supported in wxGTK).}
\twocolitem{\windowstyle{wxSP\_VERTICAL}}{Specifies a vertical spin button.}
\twocolitem{\windowstyle{wxSP\_ARROW\_KEYS}}{The user can use arrow keys.}
\twocolitem{\windowstyle{wxSP\_WRAP}}{The value wraps at the minimum and maximum.}
\end{twocollist}

See also \helpref{window styles overview}{windowstyles}.

\wxheading{Event handling}

To process input from a spin button, use one of these event handler macros to
direct input to member functions that take a 
\helpref{wxSpinEvent}{wxspinevent} argument:

\twocolwidtha{7cm}
\begin{twocollist}
\twocolitem{{\bf EVT\_SPIN(id, func)}}{Generated whenever an arros is pressed.}
\twocolitem{{\bf EVT\_SPIN\_UP(id, func)}}{Generated when left/up arrow is pressed.}
\twocolitem{{\bf EVT\_SPIN\_DOWN(id, func)}}{Generated when right/down arrow is pressed.}
\end{twocollist}%

\wxheading{See also}

\helpref{Event handling overview}{eventhandlingoverview}

\latexignore{\rtfignore{\wxheading{Members}}}

\membersection{wxSpinButton::wxSpinButton}\label{wxspinbuttonconstr}

\func{}{wxSpinButton}{\void}

Default constructor.

\func{}{wxSpinButton}{\param{wxWindow*}{ parent}, \param{wxWindowID }{id},\rtfsp
\param{const wxPoint\& }{pos = wxDefaultPosition}, \param{const wxSize\& }{size = wxDefaultSize},\rtfsp
\param{long}{ style = wxSP\_HORIZONTAL}, \param{const wxValidator\& }{validator = wxDefaultValidator},\rtfsp
\param{const wxString\& }{name = ``spinButton"}}

Constructor, creating and showing a spin button.

\wxheading{Parameters}

\docparam{parent}{Parent window. Must not be NULL.}

\docparam{id}{Window identifier. A value of -1 indicates a default value.}

\docparam{pos}{Window position. If the position (-1, -1) is specified then a default position is chosen.}

\docparam{size}{Window size. If the default size (-1, -1) is specified then a default size is chosen.}

\docparam{style}{Window style. See \helpref{wxSpinButton}{wxspinbutton}.}

\docparam{validator}{Window validator.}

\docparam{name}{Window name.}

\wxheading{See also}

\helpref{wxSpinButton::Create}{wxspinbuttoncreate}, \helpref{wxValidator}{wxvalidator}

\membersection{wxSpinButton::\destruct{wxSpinButton}}

\func{void}{\destruct{wxSpinButton}}{\void}

Destructor, destroying the spin button.

\membersection{wxSpinButton::Create}\label{wxspinbuttoncreate}

\func{bool}{Create}{\param{wxWindow*}{ parent}, \param{wxWindowID }{id},\rtfsp
\param{const wxPoint\& }{pos = wxDefaultPosition}, \param{const wxSize\& }{size = wxDefaultSize},\rtfsp
\param{long}{ style = wxSP\_HORIZONTAL}, \param{const wxValidator\& }{validator = wxDefaultValidator},\rtfsp
\param{const wxString\& }{name = ``spinButton"}}

Scrollbar creation function called by the spin button constructor.
See \helpref{wxSpinButton::wxSpinButton}{wxspinbuttonconstr} for details.

\membersection{wxSpinButton::GetMax}\label{wxspinbuttongetmax}

\constfunc{int}{GetMax}{\void}

Returns the maximum permissible value.

\wxheading{See also}

\helpref{wxSpinButton::SetRange}{wxspinbuttonsetrange}

\membersection{wxSpinButton::GetMin}\label{wxspinbuttongetmin}

\constfunc{int}{GetMin}{\void}

Returns the minimum permissible value.

\wxheading{See also}

\helpref{wxSpinButton::SetRange}{wxspinbuttonsetrange}

\membersection{wxSpinButton::GetValue}\label{wxspinbuttongetvalue}

\constfunc{int}{GetValue}{\void}

Returns the current spin button value.

\wxheading{See also}

\helpref{wxSpinButton::SetValue}{wxspinbuttonsetvalue}

\membersection{wxSpinButton::SetRange}\label{wxspinbuttonsetrange}

\func{void}{SetRange}{\param{int}{ min}, \param{int}{ max}}

Sets the range of the spin button.

\wxheading{Parameters}

\docparam{min}{The minimum value for the spin button.}

\docparam{max}{The maximum value for the spin button.}

\wxheading{See also}

\helpref{wxSpinButton::GetMin}{wxspinbuttongetmin}, \helpref{wxSpinButton::GetMax}{wxspinbuttongetmax}

\membersection{wxSpinButton::SetValue}\label{wxspinbuttonsetvalue}

\func{void}{SetValue}{\param{int}{ value}}

Sets the value of the spin button.

\wxheading{Parameters}

\docparam{value}{The value for the spin button.}

\wxheading{See also}

\helpref{wxSpinButton::GetValue}{wxspinbuttongetvalue}

