\section{Sizer overview}\label{sizeroverview}

Classes: \helpref{wxSizer}{wxsizer}, \helpref{wxGridSizer}{wxgridsizer}, 
\helpref{wxFlexGridSizer}{wxflexgridsizer}, \helpref{wxBoxSizer}{wxboxsizer}, 
\helpref{wxStaticBoxSizer}{wxstaticboxsizer}, 
\helpref{wxNotebookSizer}{wxnotebooksizer},
\helpref{CreateButtonSizer}{createbuttonsizer}

Sizers, as represented by the wxSizer class and its descendants in
the wxWindows class hierarchy, have become the method of choice to
define the layout of controls in dialogs in wxWindows because of
their ability to create visually appealing dialogs independent of the
platform, taking into account the differences in size and style of
the individual controls. Unlike the original wxWindows Dialog Editor,
editors such as wxDesigner, wxrcedit, XRCed and wxWorkshop create dialogs based exclusively on sizers,
practically forcing the user to create platform independent layouts without compromises.

The next section describes and shows what can be done with sizers.
The following sections briefly describe how to program with individual sizer classes.

For information about the new wxWindows resource system, which can describe
sizer-based dialogs, see the \helpref{XML-based resource system overview}{xrcoverview}.

\subsection{The idea behind sizers}\label{ideabehindsizers}

The layout algorithm used by sizers in wxWindows is closely related to layout
systems in other GUI toolkits, such as Java's AWT, the GTK toolkit or the Qt toolkit. It is
based upon the idea of individual subwindows reporting their minimal required
size and their ability to get stretched if the size of the parent window has changed.
This will most often mean that the programmer does not set the start-up size of
a dialog, the dialog will rather be assigned a sizer and this sizer
will be queried about the recommended size. This sizer in turn will query its
children (which can be normal windows, empty space or other sizers) so that
a hierarchy of sizers can be constructed. Note that wxSizer does not derive from wxWindow
and thus does not interfere with tab ordering and requires very few resources compared
to a real window on screen.

What makes sizers so well fitted for use in wxWindows is the fact that every control
reports its own minimal size and the algorithm can handle differences in font sizes
or different window (dialog item) sizes on different platforms without problems. For example, if
the standard font as well as the overall design of Linux/GTK widgets requires more space than
on Windows, the initial dialog size will automatically be bigger on Linux/GTK than on Windows.

There are currently five different kinds of sizers available in wxWindows. Each represents
either a certain way to lay out dialog items in a dialog or it fulfils a special task
such as wrapping a static box around a dialog item (or another sizer). These sizers will
be discussed one by one in the text below. For more detailed information on how to use sizers
programmatically, please refer to the section \helpref{Programming with Sizers}{boxsizerprogramming}.

\subsubsection{Common features}\label{sizerscommonfeatures}

All sizers are containers, that is, they are used to lay out one dialog item (or several
dialog items), which they contain. Such items are sometimes referred to as the children
of the sizer. Independent of how the individual sizers lay out their children, all children
have certain features in common:

{\bf A minimal size:} This minimal size is usually identical to
the initial size of the controls and may either be set explicitly in the wxSize field
of the control constructor or may be calculated by wxWindows, typically by setting
the height and/or the width of the item to -1. Note that only some controls can
calculate their size (such as a checkbox) whereas others (such as a listbox)
don't have any natural width or height and thus require an explicit size. Some controls
can calculate their height, but not their width (e.g. a single line text control):

\newcommand{\myimage}[1]{\mbox{\image{0cm;0cm}{#1}}}

\begin{center}
\myimage{sizer03.eps}\gifsep
\myimage{sizer04.eps}\gifsep
\myimage{sizer05.eps}
\end{center}

{\bf A border:} The border is just empty space and is used to separate dialog items
in a dialog. This border can either be all around, or at any combination of sides
such as only above and below the control. The thickness of this border must be set
explicitly, typically 5 points. The following samples show dialogs with only one
dialog item (a button) and a border of 0, 5, and 10 pixels around the button:

\begin{center}
\myimage{sizer00.eps}\gifsep
\myimage{sizer01.eps}\gifsep
\myimage{sizer02.eps}
\end{center}

{\bf An alignment:} Often, a dialog item is given more space than its minimal size
plus its border. Depending on what flags are used for the respective dialog
item, the dialog item can be made to fill out the available space entirely, i.e.
it will grow to a size larger than the minimal size, or it will be moved to either
the centre of the available space or to either side of the space. The following
sample shows a listbox and three buttons in a horizontal box sizer; one button
is centred, one is aligned at the top, one is aligned at the bottom:

\begin{center}
\myimage{sizer06.eps}
\end{center}

{\bf A stretch factor:} If a sizer contains more than one child and it is offered
more space than its children and their borders need, the question arises how to
distribute the surplus space among the children. For this purpose, a stretch
factor may be assigned to each child, where the default value of 0 indicates that the child
will not get more space than its requested minimum size. A value of more than zero
is interpreted in relation to the sum of all stretch factors in the children
of the respective sizer, i.e. if two children get a stretch factor of 1, they will
get half the extra space each {\it independent of whether one control has a minimal
sizer inferior to the other or not}. The following sample shows a dialog with
three buttons, the first one has a stretch factor of 1 and thus gets stretched,
whereas the other two buttons have a stretch factor of zero and keep their
initial width:

\begin{center}
\myimage{sizer07.eps}
\end{center}

Within wxDesigner, this stretch factor gets set from the {\it Option} menu.

\wxheading{wxBoxSizer}

\helpref{wxBoxSizer}{wxboxsizer} can lay out its children either vertically
or horizontally, depending on what flag is being used in its constructor.
When using a vertical sizer, each child can be centered, aligned to the
right or aligned to the left. Correspondingly, when using a horizontal
sizer, each child can be centered, aligned at the bottom or aligned at
the top. The stretch factor described in the last paragraph is used
for the main orientation, i.e. when using a horizontal box sizer, the
stretch factor determines how much the child can be stretched horizontally.
The following sample shows the same dialog as in the last sample,
only the box sizer is a vertical box sizer now:

\begin{center}
\myimage{sizer08.eps}
\end{center}

\wxheading{wxStaticBoxSizer}

\helpref{wxStaticBoxSixer}{wxstaticboxsizer} is the same as a wxBoxSizer, but surrounded by a
static box. Here is a sample:

\begin{center}
\myimage{sizer09.eps}
\end{center}

\wxheading{wxGridSizer}

\helpref{wxGridSizer}{wxgridsizer} is a two-dimensional sizer. All children are given the
same size, which is the minimal size required by the biggest child, in
this case the text control in the left bottom border. Either the number
of columns or the number or rows is fixed and the grid sizer will grow
in the respectively other orientation if new children are added:

\begin{center}
\myimage{sizer10.eps}
\end{center}

For programming information, see \helpref{wxGridSizer}{wxgridsizer}.

\wxheading{wxFlexGridSizer}

Another two-dimensional sizer derived from
wxGridSizer. The width of each column and the height of each row
are calculated individually according the minimal requirements
from the respectively biggest child. Additionally, columns and
rows can be declared to be stretchable if the sizer is assigned
a size different from that which it requested. The following sample shows
the same dialog as the one above, but using a flex grid sizer:

\begin{center}
\myimage{sizer11.eps}
\end{center}

\wxheading{wxNotebookSizer}

\helpref{wxNotebookSizer}{wxnotebooksizer} can be used
with notebooks. It calculates the size of each
notebook page and sets the size of the notebook to the size
of the biggest page plus some extra space required for the
notebook tabs and decorations.

\subsection{Programming with wxBoxSizer}\label{boxsizerprogramming}

The basic idea behind a \helpref{wxBoxSizer}{wxboxsizer} is that windows will most often be laid out in rather
simple basic geometry, typically in a row or a column or several hierarchies of either.

As an example, we will construct a dialog that will contain a text field at the top and
two buttons at the bottom. This can be seen as a top-hierarchy column with the text at
the top and buttons at the bottom and a low-hierarchy row with an OK button to the left
and a Cancel button to the right. In many cases (particularly dialogs under Unix and
normal frames) the main window will be resizable by the user and this change of size
will have to get propagated to its children. In our case, we want the text area to grow
with the dialog, whereas the button shall have a fixed size. In addition, there will be
a thin border around all controls to make the dialog look nice and - to make matter worse -
the buttons shall be centred as the width of the dialog changes.

It is the unique feature of a box sizer, that it can grow in both directions (height and
width) but can distribute its growth in the main direction (horizontal for a row) {\it unevenly} 
among its children. In our example case, the vertical sizer is supposed to propagate all its
height changes to only the text area, not to the button area. This is determined by the {\it proportion} parameter
when adding a window (or another sizer) to a sizer. It is interpreted
as a weight factor, i.e. it can be zero, indicating that the window may not be resized
at all, or above zero. If several windows have a value above zero, the value is interpreted
relative to the sum of all weight factors of the sizer, so when adding two windows with
a value of 1, they will both get resized equally much and each half as much as the sizer
owning them. Then what do we do when a column sizer changes its width? This behaviour is
controlled by {\it flags} (the second parameter of the Add() function): Zero or no flag
indicates that the window will preserve it is original size, wxGROW flag (same as wxEXPAND)
forces the window to grow with the sizer, and wxSHAPED flag tells the window to change it is
size proportionally, preserving original aspect ratio.  When wxGROW flag is not used,
the item can be aligned within available space.  wxALIGN\_LEFT, wxALIGN\_TOP, wxALIGN\_RIGHT,
wxALIGN\_BOTTOM, wxALIGN\_CENTER\_HORIZONTAL and wxALIGN\_CENTER\_VERTICAL do what they say.
wxALIGN\_CENTRE (same as wxALIGN\_CENTER) is defined as (wxALIGN\_CENTER\_HORIZONTAL |
wxALIGN\_CENTER\_VERTICAL).  Default alignment is wxALIGN\_LEFT | wxALIGN\_TOP.

As mentioned above, any window belonging to a sizer may have border, and it can be specified
which of the four sides may have this border, using the wxTOP, wxLEFT, wxRIGHT and wxBOTTOM
constants or wxALL for all directions (and you may also use wxNORTH, wxWEST etc instead). These
flags can be used in combination with the alignment flags above as the second parameter of the
Add() method using the binary or operator |. The sizer of the border also must be made known,
and it is the third parameter in the Add() method. This means, that the entire behaviour of
a sizer and its children can be controlled by the three parameters of the Add() method.

\begin{verbatim}
// we want to get a dialog that is stretchable because it
// has a text ctrl at the top and two buttons at the bottom

MyDialog::MyDialog(wxFrame *parent, wxWindowID id, const wxString &title )
        : wxDialog(parent, id, title, wxDefaultPosition, wxDefaultSize,
                   wxDEFAULT_DIALOG_STYLE | wxRESIZE_BORDER)
{
  wxBoxSizer *topsizer = new wxBoxSizer( wxVERTICAL );

  // create text ctrl with minimal size 100x60
  topsizer->Add(
    new wxTextCtrl( this, -1, "My text.", wxDefaultPosition, wxSize(100,60), wxTE_MULTILINE),
    1,            // make vertically stretchable
    wxEXPAND |    // make horizontally stretchable
    wxALL,        //   and make border all around
    10 );         // set border width to 10


  wxBoxSizer *button_sizer = new wxBoxSizer( wxHORIZONTAL );
  button_sizer->Add(
     new wxButton( this, wxID_OK, "OK" ),
     0,           // make horizontally unstretchable
     wxALL,       // make border all around (implicit top alignment)
     10 );        // set border width to 10
  button_sizer->Add(
     new wxButton( this, wxID_CANCEL, "Cancel" ),
     0,           // make horizontally unstretchable
     wxALL,       // make border all around (implicit top alignment)
     10 );        // set border width to 10

  topsizer->Add(
     button_sizer,
     0,                // make vertically unstretchable
     wxALIGN_CENTER ); // no border and centre horizontally

  SetSizer( topsizer );      // use the sizer for layout

  topsizer->SetSizeHints( this );   // set size hints to honour minimum size
}
\end{verbatim}

\subsection{Programming with wxGridSizer}\label{gridsizerprogramming}

\helpref{wxGridSizer}{wxgridsizer} is a sizer which lays out its children in a two-dimensional
table with all table fields having the same size,
i.e. the width of each field is the width of the widest child,
the height of each field is the height of the tallest child.

\subsection{Programming with wxFlexGridSizer}\label{flexgridsizerprogramming}

\helpref{wxFlexGridSizer}{wxflexgridsizer} is a sizer which lays out its children in a two-dimensional
table with all table fields in one row having the same
height and all fields in one column having the same width, but all
rows or all columns are not necessarily the same height or width as in
the \helpref{wxGridSizer}{wxgridsizer}.

\subsection{Programming with wxNotebookSizer}\label{notebooksizerprogramming}

\helpref{wxNotebookSizer}{wxnotebooksizer} is a specialized sizer to make sizers work in connection
with using notebooks. This sizer is different from any other sizer as 
you must not add any children to it - instead, it queries the notebook class itself.
The only thing this sizer does is to determine the size of the biggest
page of the notebook and report an adjusted minimal size to a more toplevel
sizer.

In order to query the size of notebook page, this page needs to have its
own sizer, otherwise the wxNotebookSizer will ignore it. Notebook pages
get their sizer by assigning one to them using \helpref{wxWindow::SetSizer}{wxwindowsetsizer} 
and setting the auto-layout option to TRUE using 
\helpref{wxWindow::SetAutoLayout}{wxwindowsetautolayout}. Here is one
example showing how to add a notebook page that the notebook sizer is
aware of:

\begin{verbatim}
    wxNotebook *notebook = new wxNotebook( &dialog, -1 );
    wxNotebookSizer *nbs = new wxNotebookSizer( notebook );

    // Add panel as notebook page
    wxPanel *panel = new wxPanel( notebook, -1 );
    notebook->AddPage( panel, "My Notebook Page" );

    wxBoxSizer *panelsizer = new wxBoxSizer( wxVERTICAL );

    // Add controls to panel and panelsizer here...

    panel->SetAutoLayout( TRUE );
    panel->SetSizer( panelsizer );
\end{verbatim}

\subsection{Programming with wxStaticBoxSizer}\label{staticboxsizerprogramming}

\helpref{wxStaticBoxSizer}{wxstaticboxsizer} is a sizer derived from wxBoxSizer but adds a static
box around the sizer. Note that this static box has to be created 
separately.

\subsection{CreateButtonSizer}\label{createbuttonsizer}

As a convenience, CreateButtonSizer ( long flags ) can be used to create a standard button sizer
in which standard buttons are displayed. The following flags can be passed to this function:


\begin{verbatim}
    wxYES_NO // Add Yes/No subpanel
    wxYES    // return wxID_YES
    wxNO     // return wxID_NO
    wxNO_DEFAULT // make the wxNO button the default, otherwise wxYES or wxOK button will be default
    
    wxOK     // return wxID_OK
    wxCANCEL // return wxID_CANCEL
    wxHELP   // return wxID_HELP
    
    wxFORWARD   // return wxID_FORWARD  
    wxBACKWARD  // return wxID_BACKWARD 
    wxSETUP     // return wxID_SETUP    
    wxMORE      // return wxID_MORE     

\end{verbatim}
