%%%%%%%%%%%%%%%%%%%%%%%%%%%%%%%%%%%%%%%%%%%%%%%%%%%%%%%%%%%%%%%%%%%%%%%%%%%%%%%
%% Name:        size.tex
%% Purpose:     wxSize documentation
%% Author:      wxTeam
%% Created:
%% RCS-ID:      $Id$
%% Copyright:   (c) wxTeam
%% License:     wxWindows license
%%%%%%%%%%%%%%%%%%%%%%%%%%%%%%%%%%%%%%%%%%%%%%%%%%%%%%%%%%%%%%%%%%%%%%%%%%%%%%%

\section{\class{wxSize}}\label{wxsize}

A {\bf wxSize} is a useful data structure for graphics operations.
It simply contains integer {\it width} and {\it height} members.

wxSize is used throughout wxWidgets as well as wxPoint which, although almost
equivalent to wxSize, has a different meaning: wxPoint represents a position
while wxSize - the size.

\pythonnote{wxPython defines aliases for the {\tt x} and {\tt y} members
named {\tt width} and {\tt height} since it makes much more sense for
sizes.
}

\wxheading{Derived from}

None

\wxheading{Include files}

<wx/gdicmn.h>

\wxheading{See also}

\helpref{wxPoint}{wxpoint}, \helpref{wxRealPoint}{wxrealpoint}

\latexignore{\rtfignore{\wxheading{Members}}}


\membersection{wxSize::wxSize}\label{wxsizector}

\func{}{wxSize}{\void}

\func{}{wxSize}{\param{int}{ width}, \param{int}{ height}}

Creates a size object.



\membersection{wxSize::DecTo}\label{wxsizedecto}

\func{void}{DecTo}{\param{const wxSize\& }{size}}

Decrements this object so that both of its dimensions are not greater than the
corresponding dimensions of the \arg{size}.

\wxheading{See also}

\helpref{IncTo}{wxsizeincto}


\membersection{wxSize::IsFullySpecified}\label{wxsizeisfullyspecified}

\constfunc{bool}{IsFullySpecified}{\void}

Returns \true if neither of the size object components is equal to $-1$, which
is used as default for the size values in wxWidgets (hence the predefined
\texttt{wxDefaultSize} has both of its components equal to $-1$).

This method is typically used before calling
\helpref{SetDefaults}{wxsizesetdefaults}.


\membersection{wxSize::GetWidth}\label{wxsizegetwidth}

\constfunc{int}{GetWidth}{\void}

Gets the width member.


\membersection{wxSize::GetHeight}\label{wxsizegetheight}

\constfunc{int}{GetHeight}{\void}

Gets the height member.



\membersection{wxSize::IncTo}\label{wxsizeincto}

\func{void}{IncTo}{\param{const wxSize\& }{size}}

Increments this object so that both of its dimensions are not less than the
corresponding dimensions of the \arg{size}.

\wxheading{See also}

\helpref{DecTo}{wxsizedecto}



\membersection{wxSize::Scale}\label{wxsizescale}

\func{void}{Scale}{\param{float}{ xscale}, \param{float}{ yscale}}

Scales the dimensions of this object by the given factors.
If you want to scale both dimensions by the same factor you can also use
the \helpref{operator *=}{wxsizeoperators}



\membersection{wxSize::Set}\label{wxsizeset}

\func{void}{Set}{\param{int}{ width}, \param{int}{ height}}

Sets the width and height members.


\membersection{wxSize::SetDefaults}\label{wxsizesetdefaults}

\func{void}{SetDefaults}{\param{const wxSize\& }{sizeDefault}}

Combine this size object with another one replacing the default (i.e. equal
to $-1$) components of this object with those of the other. It is typically
used like this:
\begin{verbatim}
    if ( !size.IsFullySpecified() )
    {
        size.SetDefaults(GetDefaultSize());
    }
\end{verbatim}

\wxheading{See also}

\helpref{IsFullySpecified}{wxsizeisfullyspecified}


\membersection{wxSize::SetHeight}\label{wxsizesetheight}

\func{void}{SetHeight}{\param{int}{ height}}

Sets the height.


\membersection{wxSize::SetWidth}\label{wxsizesetwidth}

\func{void}{SetWidth}{\param{int}{ width}}

Sets the width.


\membersection{Operators}\label{wxsizeoperators}

\func{void}{operator $=$}{\param{const wxSize\& }{sz}}

Assignment operator.


\func{bool}{operator $==$}{\param{const wxSize\& }{sz}}

\func{bool}{operator $!=$}{\param{const wxSize\& }{sz}}

\func{wxSize}{operator $+$}{\param{const wxSize\& }{sz}}

\func{wxSize}{operator $-$}{\param{const wxSize\& }{sz}}

\func{wxSize\&}{operator $+=$}{\param{const wxSize\& }{sz}}

\func{wxSize\&}{operator $-=$}{\param{const wxSize\& }{sz}}

Operators for comparison, sum and subtraction between \helpref{wxSize}{wxsize} objects.


\func{wxSize}{operator $/$}{\param{int }{factor}}

\func{wxSize}{operator $*$}{\param{int }{factor}}

\func{wxSize\&}{operator $/=$}{\param{int }{factor}}

\func{wxSize\&}{operator $*=$}{\param{int }{factor}}

Operators for division and multiplication between a \helpref{wxSize}{wxsize} object and an integer.
