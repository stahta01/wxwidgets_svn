%
% automatically generated by HelpGen from
% helpdata.h at 24/Oct/99 18:03:10
%

\section{\class{wxHtmlHelpData}}\label{wxhtmlhelpdata}

This class is used by \helpref{wxHtmlHelpController}{wxhtmlhelpcontroller} 
and \helpref{wxHtmlHelpFrame}{wxhtmlhelpframe} to access HTML help items.
It is internal class and should not be used directly - except for the case 
you're writing your own HTML help controller.

\wxheading{Derived from}

\helpref{wxObject}{wxobject}

\wxheading{Include files}

<wx/html/helpdata.h>

\latexignore{\rtfignore{\wxheading{Members}}}

\membersection{wxHtmlHelpData::wxHtmlHelpData}\label{wxhtmlhelpdatawxhtmlhelpdata}

\func{}{wxHtmlHelpData}{\void}

Constructor.

\membersection{wxHtmlHelpData::AddBook}\label{wxhtmlhelpdataaddbook}

\func{bool}{AddBook}{\param{const wxString\& }{book\_url}}

Adds new book. {\it book} is URL (not filename!) of HTML help project (hhp)
or ZIP file that contains arbitrary number of .hhp projects (this zip
file can have either .zip or .htb extension, htb stands for "html book").
Returns success.

\membersection{wxHtmlHelpData::FindPageById}\label{wxhtmlhelpdatafindpagebyid}

\func{wxString}{FindPageById}{\param{int }{id}}

Returns page's URL based on integer ID stored in project.

\membersection{wxHtmlHelpData::FindPageByName}\label{wxhtmlhelpdatafindpagebyname}

\func{wxString}{FindPageByName}{\param{const wxString\& }{page}}

Returns page's URL based on its (file)name.

\membersection{wxHtmlHelpData::GetBookRecArray}\label{wxhtmlhelpdatagetbookrecarray}

\func{const wxHtmlBookRecArray\&}{GetBookRecArray}{\void}

Returns array with help books info.

\membersection{wxHtmlHelpData::GetContentsArray}\label{wxhtmlhelpdatagetcontentsarray}

\func{const wxHtmlHelpDataItems\&}{GetContentsArray}{\void}

Returns reference to array with contents entries.

\membersection{wxHtmlHelpData::GetIndexArray}\label{wxhtmlhelpdatagetindexarray}

\func{const wxHtmlHelpDataItems\&}{GetIndexArray}{\void}

Returns reference to array with index entries.

\membersection{wxHtmlHelpData::SetTempDir}\label{wxhtmlhelpdatasettempdir}

\func{void}{SetTempDir}{\param{const wxString\& }{path}}

Sets temporary directory where binary cached versions of MS HTML Workshop
files will be stored. (This is turned off by default and you can enable
this feature by setting non-empty temp dir.)

