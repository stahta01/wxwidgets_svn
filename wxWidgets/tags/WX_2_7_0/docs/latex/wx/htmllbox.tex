%%%%%%%%%%%%%%%%%%%%%%%%%%%%%%%%%%%%%%%%%%%%%%%%%%%%%%%%%%%%%%%%%%%%%%%%%%%%%%%
%% Name:        htmllbox.tex
%% Purpose:     wxHtmlListBox documentation
%% Author:      Vadim Zeitlin
%% Modified by:
%% Created:     01.06.03
%% RCS-ID:      $Id$
%% Copyright:   (c) 2003 Vadim Zeitlin <vadim@wxwindows.org>
%% License:     wxWindows license
%%%%%%%%%%%%%%%%%%%%%%%%%%%%%%%%%%%%%%%%%%%%%%%%%%%%%%%%%%%%%%%%%%%%%%%%%%%%%%%

\section{\class{wxHtmlListBox}}\label{wxhtmllistbox}

wxHtmlListBox is an implementation of \helpref{wxVListBox}{wxvlistbox} which
shows HTML content in the listbox rows. This is still an abstract base class
and you will need to derive your own class from it (see htlbox sample for the
example) but you will only need to override a single
\helpref{OnGetItem()}{wxhtmllistboxongetitem} function.

\wxheading{Derived from}

\helpref{wxVListBox}{wxvlistbox}\\
\helpref{wxVScrolledWindow}{wxvscrolledwindow}\\
\helpref{wxPanel}{wxpanel}\\
\helpref{wxWindow}{wxwindow}\\
\helpref{wxEvtHandler}{wxevthandler}\\
\helpref{wxObject}{wxobject}

\wxheading{Include files}

<wx/htmllbox.h>


\latexignore{\rtfignore{\wxheading{Members}}}


\membersection{wxHtmlListBox::wxHtmlListBox}\label{wxhtmllistboxwxhtmllistbox}

\func{}{wxHtmlListBox}{\param{wxWindow* }{parent}, \param{wxWindowID }{id = wxID\_ANY}, \param{const wxPoint\& }{pos = wxDefaultPosition}, \param{const wxSize\& }{size = wxDefaultSize}, \param{long }{style = 0}, \param{const wxString\& }{name = wxVListBoxNameStr}}

Normal constructor which calls \helpref{Create()}{wxhtmllistboxcreate}
internally.

\func{}{wxHtmlListBox}{\void}

Default constructor, you must call \helpref{Create()}{wxhtmllistboxcreate}
later.


\membersection{wxHtmlListBox::\destruct{wxHtmlListBox}}\label{wxhtmllistboxdtor}

\func{}{\destruct{wxHtmlListBox}}{\void}

Destructor cleans up whatever resources we use.


\membersection{wxHtmlListBox::Create}\label{wxhtmllistboxcreate}

\func{bool}{Create}{\param{wxWindow* }{parent}, \param{wxWindowID }{id = wxID\_ANY}, \param{const wxPoint\& }{pos = wxDefaultPosition}, \param{const wxSize\& }{size = wxDefaultSize}, \param{long }{style = 0}, \param{const wxString\& }{name = wxVListBoxNameStr}}

Creates the control and optionally sets the initial number of items in it
(it may also be set or changed later with
\helpref{SetItemCount()}{wxvlistboxsetitemcount}).

There are no special styles defined for wxHtmlListBox, in particular the
wxListBox styles can not be used here.

Returns {\tt true} on success or {\tt false} if the control couldn't be created


\membersection{wxHtmlListBox::GetFileSystem}\label{wxhtmllistboxgetfilesystem}

\func{wxFileSystem\&}{GetFileSystem}{\void}

\constfunc{const wxFileSystem\&}{GetFileSystem}{\void}

Returns the \helpref{wxFileSystem}{wxfilesystem} used by the HTML parser of
this object. The file system object is used to resolve the paths in HTML
fragments displayed in the control and you should use
\helpref{wxFileSystem::ChangePathTo}{wxfilesystemchangepathto} if you use
relative paths for the images or other resources embedded in your HTML.


\membersection{wxHtmlListBox::GetSelectedTextBgColour}\label{wxhtmllistboxgetselectedtextbgcolour}

\constfunc{wxColour}{GetSelectedTextBgColour}{\param{const wxColour\& }{colBg}}

This virtual function may be overridden to change the appearance of the
background of the selected cells in the same way as
\helpref{GetSelectedTextColour}{wxhtmllistboxgetselectedtextcolour}.

It should be rarely, if ever, used because
\helpref{SetSelectionBackground}{wxvlistboxsetselectionbackground} allows to
change the selection background for all cells at once and doing anything more
fancy is probably going to look strangely.

\wxheading{See also}

\helpref{GetSelectedTextColour}{wxhtmllistboxgetselectedtextcolour}


\membersection{wxHtmlListBox::GetSelectedTextColour}\label{wxhtmllistboxgetselectedtextcolour}

\constfunc{wxColour}{GetSelectedTextColour}{\param{const wxColour\& }{colFg}}

This virtual function may be overridden to customize the appearance of the
selected cells. It is used to determine how the colour {\it colFg} is going to
look inside selection. By default all original colours are completely ignored
and the standard, system-dependent, selection colour is used but the program
may wish to override this to achieve some custom appearance.

\wxheading{See also}

\helpref{GetSelectedTextBgColour}{wxhtmllistboxgetselectedtextbgcolour},\\
\helpref{SetSelectionBackground}{wxvlistboxsetselectionbackground},\\
\helpref{wxSystemSettings::GetColour}{wxsystemsettingsgetcolour}


\membersection{wxHtmlListBox::OnGetItem}\label{wxhtmllistboxongetitem}

\constfunc{wxString}{OnGetItem}{\param{size\_t }{n}}

This method must be implemented in the derived class and should return
the body (i.e. without {\tt <html>} nor {\tt <body>} tags) of the HTML fragment
for the given item.


\membersection{wxHtmlListBox::OnGetItemMarkup}\label{wxhtmllistboxongetitemmarkup}

\constfunc{wxString}{OnGetItemMarkup}{\param{size\_t }{n}}

This function may be overridden to decorate HTML returned by
\helpref{OnGetItem()}{wxhtmllistboxongetitem}.

\membersection{wxHtmlListBox::OnLinkClicked}\label{wxhtmlistboxonlinkclicked}

\func{virtual void}{OnLinkClicked}{\param{size\_t }{n}, \param{const wxHtmlLinkInfo\& }{link}}

Called when the user clicks on hypertext link. Does nothing by default.

\wxheading{Parameters}

\docparam{n}{Index of the item containing the link.}

\docparam{link}{Description of the link.}

\wxheading{See also}

See also \helpref{wxHtmlLinkInfo}{wxhtmllinkinfo}.
