\section{\class{wxTextDataObject}}\label{wxtextdataobject}

wxTextDataObject is a specialization of wxDataObject for text data. It can be
used without change to paste data into the \helpref{wxClipboard}{wxclipboard} 
or a \helpref{wxDropSource}{wxdropsource}. A user may wish to derive a new
class from this class for providing text on-demand in order to minimize memory
consumption when offering data in several formats, such as plain text and RTF
because by default the text is stored in a string in this class, but it might
as well be generated when requested. For this, 
\helpref{GetTextLength}{wxtextdataobjectgettextlength} and 
\helpref{GetText}{wxtextdataobjectgettext} will have to be overridden.

Note that if you already have the text inside a string, you will not achieve
any efficiency gain by overriding these functions because copying wxStrings is
already a very efficient operation (data is not actually copied because
wxStrings are reference counted).

\pythonnote{If you wish to create a derived wxTextDataObject class in
wxPython you should derive the class from wxPyTextDataObject
in order to get Python-aware capabilities for the various virtual
methods.}

\wxheading{Virtual functions to override}

This class may be used as is, but all of the data transfer functions may be
overridden to increase efficiency.

\wxheading{Derived from}

\helpref{wxDataObjectSimple}{wxdataobjectsimple}\\
\helpref{wxDataObject}{wxdataobject}

\wxheading{Include files}

<wx/dataobj.h>

\wxheading{See also}

\helpref{Clipboard and drag and drop overview}{wxdndoverview}, 
\helpref{wxDataObject}{wxdataobject}, 
\helpref{wxDataObjectSimple}{wxdataobjectsimple}, 
\helpref{wxFileDataObject}{wxfiledataobject}, 
\helpref{wxBitmapDataObject}{wxbitmapdataobject}

\latexignore{\rtfignore{\wxheading{Members}}}

\membersection{wxTextDataObject::wxTextDataObject}\label{wxtextdataobjectwxtextdataobject}

\func{}{wxTextDataObject}{\param{const wxString\& }{text = wxEmptyString}}

Constructor, may be used to initialise the text (otherwise 
\helpref{SetText}{wxtextdataobjectsettext} should be used later).

\membersection{wxTextDataObject::GetTextLength}\label{wxtextdataobjectgettextlength}

\constfunc{virtual size\_t}{GetTextLength}{\void}

Returns the data size. By default, returns the size of the text data
set in the constructor or using \helpref{SetText}{wxtextdataobjectsettext}.
This can be overridden to provide text size data on-demand. It is recommended
to return the text length plus 1 for a trailing zero, but this is not
strictly required.

\membersection{wxTextDataObject::GetText}\label{wxtextdataobjectgettext}

\constfunc{virtual wxString}{GetText}{\void}

Returns the text associated with the data object. You may wish to override
this method when offering data on-demand, but this is not required by
wxWidgets' internals. Use this method to get data in text form from
the \helpref{wxClipboard}{wxclipboard}.

\membersection{wxTextDataObject::SetText}\label{wxtextdataobjectsettext}

\func{virtual void}{SetText}{\param{const wxString\& }{strText}}

Sets the text associated with the data object. This method is called
when the data object receives the data and, by default, copies the text into
the member variable. If you want to process the text on the fly you may wish to
override this function.

