\section{\class{wxEraseEvent}}\label{wxeraseevent}

An erase event is sent when a window's background needs to be repainted.

On some platforms, such as GTK+, this event is simulated (simply generated just before the
paint event) and may cause flicker. It is therefore recommended that
you set the text background colour explicitly in order to prevent flicker.
The default background colour under GTK+ is grey.

To intercept this event, use the EVT\_ERASE\_BACKGROUND macro in an event table definition.

You must call wxEraseEvent::GetDC and use the returned device context if it is non-NULL.
If it is NULL, create your own temporary wxClientDC object. 

\wxheading{Derived from}

\helpref{wxEvent}{wxevent}\\
\helpref{wxObject}{wxobject}

\wxheading{Include files}

<wx/event.h>

\wxheading{Event table macros}

To process an erase event, use this event handler macro to direct input to a member
function that takes a wxEraseEvent argument.

\twocolwidtha{7cm}
\begin{twocollist}\itemsep=0pt
\twocolitem{{\bf EVT\_ERASE\_BACKGROUND(func)}}{Process a wxEVT\_ERASE\_BACKGROUND event.}
\end{twocollist}%

\wxheading{Remarks}

Use the device context returned by \helpref{GetDC}{wxeraseeventgetdc} to draw on,
don't create a wxPaintDC in the event handler.

\wxheading{See also}

%\helpref{wxWindow::OnEraseBackground}{wxwindowonerasebackground},
\helpref{Event handling overview}{eventhandlingoverview}

\latexignore{\rtfignore{\wxheading{Members}}}

\membersection{wxEraseEvent::wxEraseEvent}\label{wxeraseeventctor}

\func{}{wxEraseEvent}{\param{int }{id = 0}, \param{wxDC* }{dc = NULL}}

Constructor.

\membersection{wxEraseEvent::GetDC}\label{wxeraseeventgetdc}

\constfunc{wxDC*}{GetDC}{\void}

Returns the device context associated with the erase event to draw on.

