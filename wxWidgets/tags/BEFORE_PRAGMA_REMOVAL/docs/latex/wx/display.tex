\section{\class{wxDisplay}}\label{wxdisplay}

Determines the sizes and locations of displays connected to the system.

\wxheading{Derived from}

None

\wxheading{Include files}

<wx/display.h>

\wxheading{See also}

\helpref{wxClientDisplayRect}{wxclientdisplayrect}, \helpref{wxDisplaySize}{wxdisplaysize}, \helpref{wxDisplaySizeMM}{wxdisplaysizemm}

\latexignore{\rtfignore{\wxheading{Members}}}


\membersection{wxDisplay::wxDisplay}\label{wxdisplayctor}

\func{}{wxDisplay}{\param{size\_t }{index = 0}}

Constructor, setting up a wxDisplay instance with the specified display.

\wxheading{Parameters}

\docparam{index}{The index of the display to use.  This must be non-negative
and lower than the value returned by \helpref{GetCount()}{wxdisplaygetcount}.}


\membersection{wxDisplay::\destruct{wxDisplay}}\label{wxdisplaydtor}

\func{void}{\destruct{wxDisplay}}{\void}

Destructor.


\membersection{wxDisplay::ChangeMode}\label{wxdisplaychangemode}

\func{bool }{ChangeMode}{\param{const wxVideoMode\& }{mode = wxDefaultVideoMode}}

Changes the video mode of this display to the mode specified
in the mode parameter.

If wxDefaultVideoMode is passed in as the mode parameter,
the defined behaviour is that wxDisplay will reset the video
mode to the default mode used by the display.  On Windows, 
the behavior is normal.  However, there are differences on other
platforms. On Unix variations using X11 extensions it should
behave as defined, but some irregularities may occur.  

On wxMac passing in wxDefaultVideoMode as the mode
parameter does nothing.  This happens because carbon 
no longer has access to DMUseScreenPrefs, an undocumented
function that changed the video mode to the system
default by using the system's 'scrn' resource.


\membersection{wxDisplay::GetCount}\label{wxdisplaygetcount}

\func{static size\_t}{GetCount}{\void}

Returns the number of connected displays.


\membersection{wxDisplay::GetCurrentMode}\label{wxdisplaygetcurrentmode}

\constfunc{wxVideoMode }{GetCurrentMode}{\void}

Returns the current video mode that this display is in. 


\membersection{wxDisplay::GetDepth}\label{wxdisplaygetdepth}

\constfunc{int }{GetDepth}{\void}

Returns the bit depth of the display whose index was passed to the constructor.


\membersection{wxDisplay::GetFromPoint}\label{wxdisplaygetfrompoint}

\func{static int}{GetFromPoint}{\param{const wxPoint\& }{pt}}

Returns the index of the display on which the given point lies.  Returns -1 if
the point is not on any connected display.

\wxheading{Parameters}

\docparam{pt}{The point to locate.}


\membersection{wxDisplay::GetFromWindow}\label{wxdisplaygetfromwindow}

\func{static int}{GetFromWindow}{\param{wxWindow* }{win}}

Returns the index of the display on which the given window lies.

If the window is on more than one display it gets the display that overlaps the window the most.

Returns -1 if the window is not on any connected display.

Currently wxMSW only.

\wxheading{Parameters}

\docparam{win}{The window to locate.}


\membersection{wxDisplay::GetGeometry}\label{wxdisplaygetgeometry}

\constfunc{wxRect }{GetGeometry}{\void}

Returns the bounding rectangle of the display whose index was passed to the
constructor.


\membersection{wxDisplay::GetModes}\label{wxdisplaygetmodes}

\constfunc{wxArrayVideoModes }{GetModes}{\param{const wxVideoMode\& }{mode = wxDefaultVideoMode}}

Fills and returns an array with all the video modes that
are supported by this display, or video modes that are 
supported by this display and match the mode parameter
(if mode is not wxDefaultVideoMode).


\membersection{wxDisplay::GetName}\label{wxdisplaygetname}

\constfunc{wxString }{GetName}{\void}

Returns the display's name.  A name is not available on all platforms.


\membersection{wxDisplay::IsPrimary}\label{wxdisplayisprimary}

\func{bool }{IsPrimary}{\void}

Returns true if the display is the primary display.  The primary display is the
one whose index is 0.

