\section{\class{wxScrollEvent}}\label{wxscrollevent}

A scroll event holds information about events sent from stand-alone
\helpref{scrollbars}{wxscrollbar} and \helpref{sliders}{wxslider}. Note that
starting from wxWindows 2.1, scrolled windows send the 
\helpref{wxScrollWinEvent}{wxscrollwinevent} which does not derive from
wxCommandEvent, but from wxEvent directly - don't confuse these two kinds of
events and use the event table macros mentioned below only for the
scrollbar-like controls.

\wxheading{Derived from}

\helpref{wxCommandEvent}{wxcommandevent}\\
\helpref{wxEvent}{wxevent}\\
\helpref{wxObject}{wxobject}

\wxheading{Include files}

<wx/event.h>

\input scrolevt.inc

\wxheading{Remarks}

Note that unless specifying a scroll control identifier, you will need to test for scrollbar
orientation with \helpref{wxScrollEvent::GetOrientation}{wxscrolleventgetorientation}, since
horizontal and vertical scroll events are processed using the same event handler.

\wxheading{See also}

\helpref{wxScrollBar}{wxscrollbar}, \helpref{wxSlider}{wxslider}, \helpref{wxSpinButton}{wxspinbutton}, \\
\helpref{wxScrollWinEvent}{wxscrollwinevent}, \helpref{Event handling overview}{eventhandlingoverview}

\latexignore{\rtfignore{\wxheading{Members}}}

\membersection{wxScrollEvent::wxScrollEvent}

\func{}{wxScrollEvent}{\param{WXTYPE }{commandType = 0}, \param{int }{id = 0}, \param{int}{ pos = 0},
\rtfsp\param{int}{ orientation = 0}}

Constructor.

\membersection{wxScrollEvent::GetOrientation}\label{wxscrolleventgetorientation}

\constfunc{int}{GetOrientation}{\void}

Returns wxHORIZONTAL or wxVERTICAL, depending on the orientation of the scrollbar.

\membersection{wxScrollEvent::GetPosition}\label{wxscrolleventgetposition}

\constfunc{int}{GetPosition}{\void}

Returns the position of the scrollbar.

