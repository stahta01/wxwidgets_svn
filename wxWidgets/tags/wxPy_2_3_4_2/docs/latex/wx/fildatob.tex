\section{\class{wxFileDataObject}}\label{wxfiledataobject}

wxFileDataObject is a specialization of \helpref{wxDataObject}{wxdataobject} 
for file names. The program works with it just as if it were a list of absolute file
names, but internally it uses the same format as
Explorer and other compatible programs under Windows or GNOME/KDE filemanager
under Unix which makes it possible to receive files from them using this
class.

{\bf Warning:} Under all non-Windows platforms this class is currently
"input-only", i.e. you can receive the files from another application, but
copying (or dragging) file(s) from a wxWindows application is not currently
supported.

\wxheading{Virtual functions to override}

None.

\wxheading{Derived from}

\helpref{wxDataObjectSimple}{wxdataobjectsimple}\\
\helpref{wxDataObject}{wxdataobject}

\wxheading{Include files}

<wx/dataobj.h>

\wxheading{See also}

\helpref{wxDataObject}{wxdataobject}, 
\helpref{wxDataObjectSimple}{wxdataobjectsimple}, 
\helpref{wxTextDataObject}{wxtextdataobject}, 
\helpref{wxBitmapDataObject}{wxbitmapdataobject}, 
\helpref{wxDataObject}{wxdataobject}

\latexignore{\rtfignore{\wxheading{Members}}}

\membersection{wxFileDataObject}\label{wxfiledataobjectwxfiledataobject}

\func{}{wxFileDataObject}{\void}

Constructor.

\membersection{wxFileDataObject::AddFile}\label{wxfiledataobjectaddfile}

\func{virtual void}{AddFile}{\param{const wxString\& }{file}}

{\bf MSW only:} adds a file to the file list represented by this data object.

\membersection{wxFileDataObject::GetFilenames}\label{wxfiledataobjectgetfilenames}

\constfunc{const wxArrayString\& }{GetFilenames}{\void}

Returns the \helpref{array}{wxarraystring} of file names.

