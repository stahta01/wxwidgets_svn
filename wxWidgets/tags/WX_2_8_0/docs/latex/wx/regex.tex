%%%%%%%%%%%%%%%%%%%%%%%%%%%%%%%%%%%%%%%%%%%%%%%%%%%%%%%%%%%%%%%%%%%%%%%%%%%%%%%
%% Name:        regex.tex
%% Purpose:     wxRegEx documentation
%% Author:      Vadim Zeitlin
%% Modified by:
%% Created:     14.07.01
%% RCS-ID:      $Id$
%% Copyright:   (c) 2001 Vadim Zeitlin
%% License:     wxWindows license
%%%%%%%%%%%%%%%%%%%%%%%%%%%%%%%%%%%%%%%%%%%%%%%%%%%%%%%%%%%%%%%%%%%%%%%%%%%%%%%

\section{\class{wxRegEx}}\label{wxregex}

wxRegEx represents a regular expression.  This class provides support
for regular expressions matching and also replacement.

It is built on top of either the system library (if it has support
for POSIX regular expressions - which is the case of the most modern
Unices) or uses the built in Henry Spencer's library.  Henry Spencer
would appreciate being given credit in the documentation of software
which uses his library, but that is not a requirement.

Regular expressions, as defined by POSIX, come in two flavours: {\it extended}
and {\it basic}.  The builtin library also adds a third flavour
of expression \helpref{advanced}{wxresyn}, which is not available
when using the system library.

Unicode is fully supported only when using the builtin library.
When using the system library in Unicode mode, the expressions and data
are translated to the default 8-bit encoding before being passed to
the library.

On platforms where a system library is available, the default is to use
the builtin library for Unicode builds, and the system library otherwise.
It is possible to use the other if preferred by selecting it when building
the wxWidgets.

\wxheading{Derived from}

No base class

\wxheading{Data structures}

Flags for regex compilation to be used with \helpref{Compile()}{wxregexcompile}:

\begin{verbatim}
enum
{
    // use extended regex syntax
    wxRE_EXTENDED = 0,
    
    // use advanced RE syntax (built-in regex only)
#ifdef wxHAS_REGEX_ADVANCED
    wxRE_ADVANCED = 1,
#endif

    // use basic RE syntax
    wxRE_BASIC    = 2,

    // ignore case in match
    wxRE_ICASE    = 4,

    // only check match, don't set back references
    wxRE_NOSUB    = 8,

    // if not set, treat '\n' as an ordinary character, otherwise it is
    // special: it is not matched by '.' and '^' and '$' always match
    // after/before it regardless of the setting of wxRE_NOT[BE]OL
    wxRE_NEWLINE  = 16,

    // default flags
    wxRE_DEFAULT  = wxRE_EXTENDED
}
\end{verbatim}

Flags for regex matching to be used with \helpref{Matches()}{wxregexmatches}.

These flags are mainly useful when doing several matches in a long string
to prevent erroneous matches for {\tt '\textasciicircum'} and {\tt '\$'}:

\begin{verbatim}
enum
{
    // '^' doesn't match at the start of line
    wxRE_NOTBOL = 32,

    // '$' doesn't match at the end of line
    wxRE_NOTEOL = 64
}
\end{verbatim}

\wxheading{Examples}

A bad example of processing some text containing email addresses (the example
is bad because the real email addresses can have more complicated form than
{\tt user@host.net}):

\begin{verbatim}
wxString text;
...
wxRegEx reEmail = wxT("([^@]+)@([[:alnum:].-_].)+([[:alnum:]]+)");
if ( reEmail.Matches(text) )
{
    wxString text = reEmail.GetMatch(email);
    wxString username = reEmail.GetMatch(email, 1);
    if ( reEmail.GetMatch(email, 3) == wxT("com") ) // .com TLD?
    {
        ...
    }
}

// or we could do this to hide the email address
size_t count = reEmail.ReplaceAll(text, wxT("HIDDEN@\\2\\3"));
printf("text now contains %u hidden addresses", count);
\end{verbatim}

\wxheading{Include files}

<wx/regex.h>

\latexignore{\rtfignore{\wxheading{Members}}}

\membersection{wxRegEx::wxRegEx}\label{wxregexwxregex}

\func{}{wxRegEx}{\void}

Default ctor: use \helpref{Compile()}{wxregexcompile} later.

\func{}{wxRegEx}{\param{const wxString\& }{expr}, \param{int }{flags = wxRE\_DEFAULT}}

Create and compile the regular expression, use 
\helpref{IsValid}{wxregexisvalid} to test for compilation errors.

\membersection{wxRegEx::\destruct{wxRegEx}}\label{wxregexdtor}

\func{}{\destruct{wxRegEx}}{\void}

dtor not virtual, don't derive from this class

\membersection{wxRegEx::Compile}\label{wxregexcompile}

\func{bool}{Compile}{\param{const wxString\& }{pattern}, \param{int }{flags = wxRE\_DEFAULT}}

Compile the string into regular expression, return {\tt true} if ok or {\tt false} 
if string has a syntax error.

\membersection{wxRegEx::IsValid}\label{wxregexisvalid}

\constfunc{bool}{IsValid}{\void}

Return {\tt true} if this is a valid compiled regular expression, {\tt false} 
otherwise.

\membersection{wxRegEx::GetMatch}\label{wxregexgetmatch}

\constfunc{bool}{GetMatch}{\param{size\_t* }{start}, \param{size\_t* }{len}, \param{size\_t }{index = 0}}

Get the start index and the length of the match of the expression
(if {\it index} is $0$) or a bracketed subexpression ({\it index} different
from $0$).

May only be called after successful call to \helpref{Matches()}{wxregexmatches} 
and only if {\tt wxRE\_NOSUB} was {\bf not} used in 
\helpref{Compile()}{wxregexcompile}.

Returns {\tt false} if no match or if an error occurred.

\constfunc{wxString}{GetMatch}{\param{const wxString\& }{text}, \param{size\_t }{index = 0}}

Returns the part of string corresponding to the match where {\it index} is
interpreted as above. Empty string is returned if match failed

May only be called after successful call to \helpref{Matches()}{wxregexmatches} 
and only if {\tt wxRE\_NOSUB} was {\bf not} used in 
\helpref{Compile()}{wxregexcompile}.

\membersection{wxRegEx::GetMatchCount}\label{wxregexgetmatchcount}

\constfunc{size\_t}{GetMatchCount}{\void}

Returns the size of the array of matches, i.e. the number of bracketed
subexpressions plus one for the expression itself, or $0$ on error.

May only be called after successful call to \helpref{Compile()}{wxregexcompile}.
and only if {\tt wxRE\_NOSUB} was {\bf not} used.

\membersection{wxRegEx::Matches}\label{wxregexmatches}

\constfunc{bool}{Matches}{\param{const wxChar* }{text}, \param{int }{flags = 0}}

\constfunc{bool}{Matches}{\param{const wxChar* }{text}, \param{int }{flags}, \param{size\_t }{len}}

\constfunc{bool}{Matches}{\param{const wxString\& }{text}, \param{int }{flags = 0}}

Matches the precompiled regular expression against the string {\it text},
returns {\tt true} if matches and {\tt false} otherwise.

{\it Flags} may be combination of {\tt wxRE\_NOTBOL} and {\tt wxRE\_NOTEOL}.

Some regex libraries assume that the text given is null terminated, while
others require the length be given as a separate parameter. Therefore for
maximum portability assume that {\it text} cannot contain embedded nulls.

When the {\it Matches(const wxChar *text, int flags = 0)} form is used,
a {\it wxStrlen()} will be done internally if the regex library requires the
length. When using {\it Matches()} in a loop
the {\it Matches(text, flags, len)} form can be used instead, making it
possible to avoid a {\it wxStrlen()} inside the loop.

May only be called after successful call to \helpref{Compile()}{wxregexcompile}.

\membersection{wxRegEx::Replace}\label{wxregexreplace}

\constfunc{int}{Replace}{\param{wxString* }{text}, \param{const wxString\& }{replacement}, \param{size\_t }{maxMatches = 0}}

Replaces the current regular expression in the string pointed to by
{\it text}, with the text in {\it replacement} and return number of matches
replaced (maybe $0$ if none found) or $-1$ on error.

The replacement text may contain back references {\tt $\backslash$number} which will be
replaced with the value of the corresponding subexpression in the
pattern match. {\tt $\backslash$0} corresponds to the entire match and {\tt \&} is a
synonym for it. Backslash may be used to quote itself or {\tt \&} character.

{\it maxMatches} may be used to limit the number of replacements made, setting
it to $1$, for example, will only replace first occurrence (if any) of the
pattern in the text while default value of $0$ means replace all.

\membersection{wxRegEx::ReplaceAll}\label{wxregexreplaceall}

\constfunc{int}{ReplaceAll}{\param{wxString* }{text}, \param{const wxString\& }{replacement}}

Replace all occurrences: this is actually a synonym for 
\helpref{Replace()}{wxregexreplace}.

\wxheading{See also}

\helpref{ReplaceFirst}{wxregexreplacefirst}

\membersection{wxRegEx::ReplaceFirst}\label{wxregexreplacefirst}

\constfunc{int}{ReplaceFirst}{\param{wxString* }{text}, \param{const wxString\& }{replacement}}

Replace the first occurrence.

\wxheading{See also}

\helpref{Replace}{wxregexreplace}

