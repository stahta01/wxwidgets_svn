\section{wxDataObject overview}\label{wxdataobjectoverview}

Classes: \helpref{wxDataObject}{wxdataobject},
 \helpref{wxClipboard}{wxclipboard},
 \helpref{wxDataFormat}{wxdataformat},
 \helpref{wxDropSource}{wxdropsource},
 \helpref{wxDropTarget}{wxdroptarget}

See also: \helpref{Drag and drop overview}{wxdndoverview} and \helpref{DnD sample}{samplednd}

This overview discusses data transfer through clipboard or drag and drop. In
wxWidgets, these two ways to transfer data (either between different
applications or inside one and the same) are very similar which allows to
implement both of them using almost the same code - or, in other
words, if you implement drag and drop support for your application, you get
clipboard support for free and vice versa.

At the heart of both clipboard and drag and drop operations lies the 
\helpref{wxDataObject}{wxdataobject} class. The objects of this class (or, to
be precise, classes derived from it) represent the data which is being carried
by the mouse during drag and drop operation or copied to or pasted from the
clipboard. wxDataObject is a "smart" piece of data because it knows which
formats it supports (see GetFormatCount and GetAllFormats) and knows how to
render itself in any of them (see GetDataHere). It can also receive its value
from the outside in a format it supports if it implements the SetData method.
Please see the documentation of this class for more details.

Both clipboard and drag and drop operations have two sides: the source and
target, the data provider and the data receiver. These which may be in the same
application and even the same window when, for example, you drag some text from
one position to another in a word processor. Let us describe what each of them
should do.

\subsection{The data provider (source) duties}\label{wxdataobjectsource}

The data provider is responsible for creating a 
\helpref{wxDataObject}{wxdataobject} containing the data to be
transferred. Then it should either pass it to the clipboard using 
\helpref{SetData}{wxclipboardsetdata} function or to 
\helpref{wxDropSource}{wxdropsource} and call 
\helpref{DoDragDrop}{wxdropsourcedodragdrop} function.

The only (but important) difference is that the object for the clipboard
transfer must always be created on the heap (i.e. using {\tt new}) and it will
be freed by the clipboard when it is no longer needed (indeed, it is not known
in advance when, if ever, the data will be pasted from the clipboard). On the
other hand, the object for drag and drop operation must only exist while 
\helpref{DoDragDrop}{wxdropsourcedodragdrop} executes and may be safely deleted
afterwards and so can be created either on heap or on stack (i.e. as a local
variable).

Another small difference is that in the case of clipboard operation, the
application usually knows in advance whether it copies or cuts (i.e. copies and
deletes) data - in fact, this usually depends on which menu item the user
chose. But for drag and drop it can only know it after 
\helpref{DoDragDrop}{wxdropsourcedodragdrop} returns (from its return value).

\subsection{The data receiver (target) duties}\label{wxdataobjecttarget}

To receive (paste in usual terminology) data from the clipboard, you should
create a \helpref{wxDataObject}{wxdataobject} derived class which supports the
data formats you need and pass it as argument to 
\helpref{wxClipboard::GetData}{wxclipboardgetdata}. If it returns {\tt false},
no data in (any of) the supported format(s) is available. If it returns {\tt
true}, the data has been successfully transferred to wxDataObject.

For drag and drop case, the \helpref{wxDropTarget::OnData}{wxdroptargetondata} 
virtual function will be called when a data object is dropped, from which the
data itself may be requested by calling 
\helpref{wxDropTarget::GetData}{wxdroptargetwxdroptarget} method which fills
the data object.

