\section{\class{wxFont}}\label{wxfont}

A font is an object which determines the appearance of text. Fonts are
used for drawing text to a device context, and setting the appearance of
a window's text.

\wxheading{Derived from}

\helpref{wxGDIObject}{wxgdiobject}\\
\helpref{wxObject}{wxobject}

\wxheading{Include files}

<wx/font.h>

\wxheading{Constants}

\begin{verbatim}
enum wxFontEncoding
{
    wxFONTENCODING_SYSTEM = -1,     // system default
    wxFONTENCODING_DEFAULT,         // current default encoding

    // ISO8859 standard defines a number of single-byte charsets
    wxFONTENCODING_ISO8859_1,       // West European (Latin1)
    wxFONTENCODING_ISO8859_2,       // Central and East European (Latin2)
    wxFONTENCODING_ISO8859_3,       // Esperanto (Latin3)
    wxFONTENCODING_ISO8859_4,       // Baltic (old) (Latin4)
    wxFONTENCODING_ISO8859_5,       // Cyrillic
    wxFONTENCODING_ISO8859_6,       // Arabic
    wxFONTENCODING_ISO8859_7,       // Greek
    wxFONTENCODING_ISO8859_8,       // Hebrew
    wxFONTENCODING_ISO8859_9,       // Turkish (Latin5)
    wxFONTENCODING_ISO8859_10,      // Variation of Latin4 (Latin6)
    wxFONTENCODING_ISO8859_11,      // Thai
    wxFONTENCODING_ISO8859_12,      // doesn't exist currently, but put it
                                    // here anyhow to make all ISO8859
                                    // consecutive numbers
    wxFONTENCODING_ISO8859_13,      // Baltic (Latin7)
    wxFONTENCODING_ISO8859_14,      // Latin8
    wxFONTENCODING_ISO8859_15,      // Latin9 (a.k.a. Latin0, includes euro)
    wxFONTENCODING_ISO8859_MAX,

    // Cyrillic charset soup (see http://czyborra.com/charsets/cyrillic.html)
    wxFONTENCODING_KOI8,            // we don't support any of KOI8 variants
    wxFONTENCODING_ALTERNATIVE,     // same as MS-DOS CP866
    wxFONTENCODING_BULGARIAN,       // used under Linux in Bulgaria

    // what would we do without Microsoft? They have their own encodings
        // for DOS
    wxFONTENCODING_CP437,           // original MS-DOS codepage
    wxFONTENCODING_CP850,           // CP437 merged with Latin1
    wxFONTENCODING_CP852,           // CP437 merged with Latin2
    wxFONTENCODING_CP855,           // another cyrillic encoding
    wxFONTENCODING_CP866,           // and another one
        // and for Windows
    wxFONTENCODING_CP874,           // WinThai
    wxFONTENCODING_CP1250,          // WinLatin2
    wxFONTENCODING_CP1251,          // WinCyrillic
    wxFONTENCODING_CP1252,          // WinLatin1
    wxFONTENCODING_CP1253,          // WinGreek (8859-7)
    wxFONTENCODING_CP1254,          // WinTurkish
    wxFONTENCODING_CP1255,          // WinHebrew
    wxFONTENCODING_CP1256,          // WinArabic
    wxFONTENCODING_CP1257,          // WinBaltic (same as Latin 7)
    wxFONTENCODING_CP12_MAX,

    wxFONTENCODING_UTF7,            // UTF-7 Unicode encoding
    wxFONTENCODING_UTF8,            // UTF-8 Unicode encoding

    wxFONTENCODING_UNICODE,         // Unicode - currently used only by
                                    // wxEncodingConverter class

    wxFONTENCODING_MAX
};
\end{verbatim}

\wxheading{Predefined objects}

Objects:

{\bf wxNullFont}

Pointers:

{\bf wxNORMAL\_FONT\\
wxSMALL\_FONT\\
wxITALIC\_FONT\\
wxSWISS\_FONT}

\wxheading{See also}

\helpref{wxFont overview}{wxfontoverview}, \helpref{wxDC::SetFont}{wxdcsetfont},\rtfsp
\helpref{wxDC::DrawText}{wxdcdrawtext}, \helpref{wxDC::GetTextExtent}{wxdcgettextextent},\rtfsp
\helpref{wxFontDialog}{wxfontdialog}

\latexignore{\rtfignore{\wxheading{Members}}}

\membersection{wxFont::wxFont}\label{wxfontconstr}

\func{}{wxFont}{\void}

Default constructor.

\func{}{wxFont}{\param{int}{ pointSize}, \param{int}{ family}, \param{int}{ style}, \param{int}{ weight},
 \param{const bool}{ underline = FALSE}, \param{const wxString\& }{faceName = ""},
 \param{wxFontEncoding }{encoding = wxFONTENCODING\_DEFAULT}}

Creates a font object (see \helpref{font encoding
overview}{wxfontencodingoverview} for the meaning of the last parameter).

\wxheading{Parameters}

\docparam{pointSize}{Size in points.}

\docparam{family}{Font family, a generic way of referring to fonts without specifying actual facename. One of:

\twocolwidtha{5cm}
\begin{twocollist}\itemsep=0pt
\twocolitem{{\bf wxDEFAULT}}{Chooses a default font.}
\twocolitem{{\bf wxDECORATIVE}}{A decorative font.}
\twocolitem{{\bf wxROMAN}}{A formal, serif font.}
\twocolitem{{\bf wxSCRIPT}}{A handwriting font.}
\twocolitem{{\bf wxSWISS}}{A sans-serif font.}
\twocolitem{{\bf wxMODERN}}{A fixed pitch font.}
\end{twocollist}}

\docparam{style}{One of {\bf wxNORMAL}, {\bf wxSLANT} and {\bf wxITALIC}.}

\docparam{weight}{One of {\bf wxNORMAL}, {\bf wxLIGHT} and {\bf wxBOLD}.}

\docparam{underline}{The value can be TRUE or FALSE. At present this has an effect on Windows only.}

\docparam{faceName}{An optional string specifying the actual typeface to be used. If the empty string,
a default typeface will chosen based on the family.}

\docparam{encoding}{An encoding which may be one of
\twocolwidtha{5cm}
\begin{twocollist}\itemsep=0pt
\twocolitem{{\bf wxFONTENCODING\_SYSTEM}}{Default system encoding.}
\twocolitem{{\bf wxFONTENCODING\_DEFAULT}}{Default application encoding: this
is the encoding set by calls to 
\helpref{SetDefaultEncoding}{wxfontsetdefaultencoding} and which may be set to,
say, KOI8 to create all fonts by default with KOI8 encoding. Initially, the
default application encoding is the same as default system encoding.}
\twocolitem{{\bf wxFONTENCODING\_ISO8859\_1...15}}{ISO8859 encodings.}
\twocolitem{{\bf wxFONTENCODING\_KOI8}}{The standard russian encoding for Internet.}
\twocolitem{{\bf wxFONTENCODING\_CP1250...1252}}{Windows encodings similar to ISO8859 (but not identical).}
\end{twocollist}
If the specified encoding isn't available, no font is created.
}

\wxheading{Remarks}

If the desired font does not exist, the closest match will be chosen.
Under Windows, only scaleable TrueType fonts are used.

See also \helpref{wxDC::SetFont}{wxdcsetfont}, \helpref{wxDC::DrawText}{wxdcdrawtext}
and \helpref{wxDC::GetTextExtent}{wxdcgettextextent}.

\membersection{wxFont::\destruct{wxFont}}

\func{}{\destruct{wxFont}}{\void}

Destructor.

\wxheading{Remarks}

The destructor may not delete the underlying font object of the native windowing
system, since wxFont uses a reference counting system for efficiency.

Although all remaining fonts are deleted when the application exits,
the application should try to clean up all fonts itself. This is because
wxWindows cannot know if a pointer to the font object is stored in an
application data structure, and there is a risk of double deletion.

\membersection{wxFont::IsFixedWidth}\label{wxfontisfixedwidth}

\constfunc{bool}{IsFixedWidth}{\void}

Returns {\tt TRUE} if the font is a fixed width (or monospaced) font, 
{\tt FALSE} if it is a proportional one or font is invalid.

\membersection{wxFont::GetDefaultEncoding}\label{wxfontgetdefaultencoding}

\func{static wxFontEncoding}{GetDefaultEncoding}{\void}

Returns the current applications default encoding.

\wxheading{See also}

\helpref{Font encoding overview}{wxfontencodingoverview}, 
\helpref{SetDefaultEncoding}{wxfontsetdefaultencoding}

\membersection{wxFont::GetFaceName}\label{wxfontgetfacename}

\constfunc{wxString}{GetFaceName}{\void}

Returns the typeface name associated with the font, or the empty string if there is no
typeface information.

\wxheading{See also}

\helpref{wxFont::SetFaceName}{wxfontsetfacename}

\membersection{wxFont::GetFamily}\label{wxfontgetfamily}

\constfunc{int}{GetFamily}{\void}

Gets the font family. See \helpref{wxFont::wxFont}{wxfontconstr} for a list of valid
family identifiers.

\wxheading{See also}

\helpref{wxFont::SetFamily}{wxfontsetfamily}

\membersection{wxFont::GetNativeFontInfoDesc}\label{wxfontgetnativefontinfodesc}

\constfunc{wxString}{GetNativeFontInfoDesc}{\void}

Returns the platform-dependent string completely describing this font or an
empty string if the font wasn't constructed using the native font description.

\wxheading{See also}

\helpref{wxFont::SetNativeFontInfo}{wxfontsetnativefontinfo}

\membersection{wxFont::GetPointSize}\label{wxfontgetpointsize}

\constfunc{int}{GetPointSize}{\void}

Gets the point size.

\wxheading{See also}

\helpref{wxFont::SetPointSize}{wxfontsetpointsize}

\membersection{wxFont::GetStyle}\label{wxfontgetstyle}

\constfunc{int}{GetStyle}{\void}

Gets the font style. See \helpref{wxFont::wxFont}{wxfontconstr} for a list of valid
styles.

\wxheading{See also}

\helpref{wxFont::SetStyle}{wxfontsetstyle}

\membersection{wxFont::GetUnderlined}\label{wxfontgetunderlined}

\constfunc{bool}{GetUnderlined}{\void}

Returns TRUE if the font is underlined, FALSE otherwise.

\wxheading{See also}

\helpref{wxFont::SetUnderlined}{wxfontsetunderlined}

\membersection{wxFont::GetWeight}\label{wxfontgetweight}

\constfunc{int}{GetWeight}{\void}

Gets the font weight. See \helpref{wxFont::wxFont}{wxfontconstr} for a list of valid
weight identifiers.

\wxheading{See also}

\helpref{wxFont::SetWeight}{wxfontsetweight}

\membersection{wxFont::Ok}\label{wxfontok}

\constfunc{bool}{Ok}{\void}

Returns {\tt TRUE} if this object is a valid font, {\tt FALSE} otherwise.

\membersection{wxFont::SetDefaultEncoding}\label{wxfontsetdefaultencoding}

\func{static void}{SetDefaultEncoding}{\param{wxFontEncoding }{encoding}}

Sets the default font encoding.

\wxheading{See also}

\helpref{Font encoding overview}{wxfontencodingoverview}, 
\helpref{GetDefaultEncoding}{wxfontgetdefaultencoding}

\membersection{wxFont::SetFaceName}\label{wxfontsetfacename}

\func{void}{SetFaceName}{\param{const wxString\& }{faceName}}

Sets the facename for the font.

\wxheading{Parameters}

\docparam{faceName}{A valid facename, which should be on the end-user's system.}

\wxheading{Remarks}

To avoid portability problems, don't rely on a specific face, but specify the font family
instead or as well. A suitable font will be found on the end-user's system. If both the
family and the facename are specified, wxWindows will first search for the specific face,
and then for a font belonging to the same family.

\wxheading{See also}

\helpref{wxFont::GetFaceName}{wxfontgetfacename}, \helpref{wxFont::SetFamily}{wxfontsetfamily}

\membersection{wxFont::SetFamily}\label{wxfontsetfamily}

\func{void}{SetFamily}{\param{int}{ family}}

Sets the font family.

\wxheading{Parameters}

\docparam{family}{One of:

\twocolwidtha{5cm}
\begin{twocollist}\itemsep=0pt
\twocolitem{{\bf wxDEFAULT}}{Chooses a default font.}
\twocolitem{{\bf wxDECORATIVE}}{A decorative font.}
\twocolitem{{\bf wxROMAN}}{A formal, serif font.}
\twocolitem{{\bf wxSCRIPT}}{A handwriting font.}
\twocolitem{{\bf wxSWISS}}{A sans-serif font.}
\twocolitem{{\bf wxMODERN}}{A fixed pitch font.}
\end{twocollist}}

\wxheading{See also}

\helpref{wxFont::GetFamily}{wxfontgetfamily}, \helpref{wxFont::SetFaceName}{wxfontsetfacename}

\membersection{wxFont::SetNativeFontInfo}\label{wxfontsetnativefontinfo}

\func{void}{SetNativeFontInfo}{\param{const wxString\& }{info}}

Creates the font corresponding to the given native font description string
which must have been previously returned by 
\helpref{GetNativeFontInfoDesc}{wxfontgetnativefontinfodesc}. If the string is
invalid, font is unchanged.

\membersection{wxFont::SetPointSize}\label{wxfontsetpointsize}

\func{void}{SetPointSize}{\param{int}{ pointSize}}

Sets the point size.

\wxheading{Parameters}

\docparam{pointSize}{Size in points.}

\wxheading{See also}

\helpref{wxFont::GetPointSize}{wxfontgetpointsize}

\membersection{wxFont::SetStyle}\label{wxfontsetstyle}

\func{void}{SetStyle}{\param{int}{ style}}

Sets the font style.

\wxheading{Parameters}

\docparam{style}{One of {\bf wxNORMAL}, {\bf wxSLANT} and {\bf wxITALIC}.}

\wxheading{See also}

\helpref{wxFont::GetStyle}{wxfontgetstyle}

\membersection{wxFont::SetUnderlined}\label{wxfontsetunderlined}

\func{void}{SetUnderlined}{\param{const bool}{ underlined}}

Sets underlining.

\wxheading{Parameters}

\docparam{underlining}{TRUE to underline, FALSE otherwise.}

\wxheading{See also}

\helpref{wxFont::GetUnderlined}{wxfontgetunderlined}

\membersection{wxFont::SetWeight}\label{wxfontsetweight}

\func{void}{SetWeight}{\param{int}{ weight}}

Sets the font weight.

\wxheading{Parameters}

\docparam{weight}{One of {\bf wxNORMAL}, {\bf wxLIGHT} and {\bf wxBOLD}.}

\wxheading{See also}

\helpref{wxFont::GetWeight}{wxfontgetweight}

\membersection{wxFont::operator $=$}\label{wxfontassignment}

\func{wxFont\&}{operator $=$}{\param{const wxFont\& }{font}}

Assignment operator, using reference counting. Returns a reference
to `this'.

\membersection{wxFont::operator $==$}\label{wxfontequals}

\func{bool}{operator $==$}{\param{const wxFont\& }{font}}

Equality operator. Two fonts are equal if they contain pointers
to the same underlying font data. It does not compare each attribute,
so two indefontdently-created fonts using the same parameters will
fail the test.

\membersection{wxFont::operator $!=$}\label{wxfontnotequals}

\func{bool}{operator $!=$}{\param{const wxFont\& }{font}}

Inequality operator. Two fonts are not equal if they contain pointers
to different underlying font data. It does not compare each attribute.


