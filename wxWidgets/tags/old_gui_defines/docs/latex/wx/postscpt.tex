\section{\class{wxPostScriptDC}}\label{wxpostscriptdc}

This defines the wxWindows Encapsulated PostScript device context,
which can write PostScript files on any platform. See \helpref{wxDC}{wxdc} for
descriptions of the member functions.

\wxheading{Derived from}

\helpref{wxDC}{wxdc}\\
\helpref{wxObject}{wxobject}

\membersection{wxPostScriptDC::wxPostScriptDC}

\func{}{wxPostScriptDC}{\param{const wxString\& }{output}, \param{bool }{interactive = TRUE},\\
  \param{wxWindow *}{parent}}

Constructor. {\it output} is an optional file for printing to, and if
\rtfsp{\it interactive} is TRUE a dialog box will be displayed for adjusting
various parameters. {\it parent} is the parent of the printer dialog box.

Use the {\it Ok} member to test whether the constructor was successful
in creating a useable device context.

See \helpref{Printer settings}{printersettings} for functions to set and
get PostScript printing settings.

\membersection{wxPostScriptDC::GetStream}

\func{ostream *}{GetStream}{\void}

Returns the stream currently being used to write PostScript output. Use this
to insert any PostScript code that is outside the scope of wxPostScriptDC.


