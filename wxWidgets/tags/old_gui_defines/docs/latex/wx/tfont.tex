\section{Font overview}\label{wxfontoverview}

Class: \helpref{wxFont}{wxfont}

A font is an object which determines the appearance of text, primarily
when drawing text to a window or device context. A font is determined by
up to six parameters:

\begin{twocollist}\itemsep=0pt
\twocolitem{Point size}{This is the standard way of referring to text size.}
\twocolitem{Family}{Supported families are:
  {\bf wxDEFAULT, wxDECORATIVE, wxROMAN, wxSCRIPT, wxSWISS, wxMODERN}.
  {\bf wxMODERN} is a fixed pitch font; the others are either fixed or variable pitch.}
\twocolitem{Style}{The value can be {\bf wxNORMAL, wxSLANT} or {\bf wxITALIC}.}
\twocolitem{Weight}{The value can be {\bf wxNORMAL, wxLIGHT} or {\bf wxBOLD}.}
\twocolitem{Underlining}{The value can be TRUE or FALSE.}
\twocolitem{Face name}{An optional string specifying the actual typeface to be used. If NULL,
a default typeface will chosen based on the family.}
\end{twocollist}

Specifying a family, rather than a specific typeface name, ensures a degree of portability
across platforms because a suitable font will be chosen for the given font family.

Under Windows, the face name can be one of the installed fonts on the user's system. Since
the choice of fonts differs from system to system, either choose standard Windows fonts,
or if allowing the user to specify a face name, store the family id with any file that
might be transported to a different Windows machine or other platform.

\normalbox{{\bf Note:} There is currently a difference between the appearance of fonts on the
two platforms, if the mapping mode is anything other than MM\_TEXT.
Under X, font size is always specified in points. Under MS Windows, the
unit for text is points but the text is scaled according to the
current mapping mode. However, user scaling on a device context will
also scale fonts under both environments.}


