\section{\class{wxRadioBox}}\label{wxradiobox}

A radio box item is used to select one of number of mutually exclusive
choices.  It is displayed as a vertical column or horizontal row of
labelled buttons.

\wxheading{Derived from}

\helpref{wxControl}{wxcontrol}\\
\helpref{wxWindow}{wxwindow}\\
\helpref{wxEvtHandler}{wxevthandler}\\
\helpref{wxObject}{wxobject}

\wxheading{Include files}

<wx/radiobox.h>

\wxheading{Window styles}

\twocolwidtha{5cm}
\begin{twocollist}\itemsep=0pt
\twocolitem{\windowstyle{wxRA\_SPECIFY\_ROWS}}{The major dimension parameter refers to the
maximum number of rows.}
\twocolitem{\windowstyle{wxRA\_SPECIFY\_COLS}}{The major dimension parameter refers to the
maximum number of columns.}
\end{twocollist}

See also \helpref{window styles overview}{windowstyles}.

\wxheading{Event handling}

\twocolwidtha{7cm}
\begin{twocollist}\itemsep=0pt
\twocolitem{{\bf EVT\_RADIOBOX(id, func)}}{Process a wxEVT\_COMMAND\_RADIOBOX\_SELECTED event,
when a radiobutton is clicked.}
\end{twocollist}

\wxheading{See also}

\helpref{Event handling overview}{eventhandlingoverview}, \helpref{wxRadioButton}{wxradiobutton},\rtfsp
\helpref{wxCheckBox}{wxcheckbox}

\latexignore{\rtfignore{\wxheading{Members}}}

\membersection{wxRadioBox::wxRadioBox}\label{wxradioboxconstr}

\func{}{wxRadioBox}{\void}

Default constructor.

\func{}{wxRadioBox}{\param{wxWindow* }{parent}, \param{wxWindowID }{id}, \param{const wxString\& }{label},\rtfsp
\param{const wxPoint\& }{point = wxDefaultPosition}, \param{const wxSize\& }{size = wxDefaultSize},\rtfsp
\param{int}{ n = 0}, \param{const wxString}{ choices[] = NULL},\rtfsp
\param{int}{ majorDimension = 0}, \param{long}{ style = wxRA\_SPECIFY\_COLS},\rtfsp
\param{const wxValidator\& }{validator = wxDefaultValidator},\rtfsp
\param{const wxString\& }{name = ``radioBox"}}

\func{}{wxRadioBox}{\param{wxWindow* }{parent}, \param{wxWindowID }{id}, \param{const wxString\& }{label},\rtfsp
\param{const wxPoint\& }{point}, \param{const wxSize\& }{size},\rtfsp
\param{const wxArrayString\&}{ choices},\rtfsp
\param{int}{ majorDimension = 0}, \param{long}{ style = wxRA\_SPECIFY\_COLS},\rtfsp
\param{const wxValidator\& }{validator = wxDefaultValidator},\rtfsp
\param{const wxString\& }{name = ``radioBox"}}

Constructor, creating and showing a radiobox.

\wxheading{Parameters}

\docparam{parent}{Parent window. Must not be NULL.}

\docparam{id}{Window identifier. A value of -1 indicates a default value.}

\docparam{label}{Label for the static box surrounding the radio buttons.}

\docparam{pos}{Window position. If the position (-1, -1) is specified then a default position is chosen.}

\docparam{size}{Window size. If the default size (-1, -1) is specified then a default size is chosen.}

\docparam{n}{Number of choices with which to initialize the radiobox.}

\docparam{choices}{An array of choices with which to initialize the radiobox.}

\docparam{majorDimension}{Specifies the maximum number of rows (if style contains wxRA\_SPECIFY\_ROWS) or columns (if style contains wxRA\_SPECIFY\_COLS) for a two-dimensional
radiobox.}

\docparam{style}{Window style. See \helpref{wxRadioBox}{wxradiobox}.}

\docparam{validator}{Window validator.}

\docparam{name}{Window name.}

\wxheading{See also}

\helpref{wxRadioBox::Create}{wxradioboxcreate}, \helpref{wxValidator}{wxvalidator}

\pythonnote{The wxRadioBox constructor in wxPython reduces the {\tt n}
and {\tt choices} arguments are to a single argument, which is
a list of strings.}

\perlnote{In wxPerl there is just an array reference in place of {\tt n}
and {\tt choices}.}

\membersection{wxRadioBox::\destruct{wxRadioBox}}

\func{}{\destruct{wxRadioBox}}{\void}

Destructor, destroying the radiobox item.

\membersection{wxRadioBox::Create}\label{wxradioboxcreate}

\func{bool}{Create}{\param{wxWindow* }{parent}, \param{wxWindowID }{id}, \param{const wxString\& }{label},\rtfsp
\param{const wxPoint\& }{point = wxDefaultPosition}, \param{const wxSize\& }{size = wxDefaultSize},\rtfsp
\param{int}{ n = 0}, \param{const wxString}{ choices[] = NULL},\rtfsp
\param{int}{ majorDimension = 0}, \param{long}{ style = wxRA\_SPECIFY\_COLS},\rtfsp
\param{const wxValidator\& }{validator = wxDefaultValidator},\rtfsp
\param{const wxString\& }{name = ``radioBox"}}

\func{bool}{Create}{\param{wxWindow* }{parent}, \param{wxWindowID }{id}, \param{const wxString\& }{label},\rtfsp
\param{const wxPoint\& }{point}, \param{const wxSize\& }{size},\rtfsp
\param{const wxArrayString\&}{ choices},\rtfsp
\param{int}{ majorDimension = 0}, \param{long}{ style = wxRA\_SPECIFY\_COLS},\rtfsp
\param{const wxValidator\& }{validator = wxDefaultValidator},\rtfsp
\param{const wxString\& }{name = ``radioBox"}}

Creates the radiobox for two-step construction. See \helpref{wxRadioBox::wxRadioBox}{wxradioboxconstr}\rtfsp
for further details.

\membersection{wxRadioBox::Enable}\label{wxradioboxenable}

\func{void}{Enable}{\param{bool}{ enable = {\tt true}}}

Enables or disables the entire radiobox.

\func{void}{Enable}{\param{int}{ n}, \param{bool}{ enable = {\tt true}}}

Enables or disables an individual button in the radiobox.

\wxheading{Parameters}

\docparam{enable}{true to enable, false to disable.}

\docparam{n}{The zero-based button to enable or disable.}

\pythonnote{In place of a single overloaded method name, wxPython
implements the following methods:\par
\indented{2cm}{\begin{twocollist}
\twocolitem{{\bf Enable(flag)}}{Enables or disables the entire radiobox.}
\twocolitem{{\bf EnableItem(n, flag)}}{Enables or disables an
individual button in the radiobox.}
\end{twocollist}}
}


\membersection{wxRadioBox::FindString}\label{wxradioboxfindstring}

\constfunc{int}{FindString}{\param{const wxString\& }{string}}

Finds a button matching the given string, returning the position if found, or
-1 if not found.

\wxheading{Parameters}

\docparam{string}{The string to find.}

\membersection{wxRadioBox::GetCount}\label{wxradioboxgetcount}

\constfunc{int}{GetCount}{\void}

Returns the number of items in the radiobox.

\membersection{wxRadioBox::GetLabel}\label{wxradioboxgetlabel}

\constfunc{wxString}{GetLabel}{\void}

Returns the radiobox label.

\constfunc{wxString}{GetLabel}{\param{int }{n}}

Returns the label for the given button.

\wxheading{Parameters}

\docparam{n}{The zero-based button index.}

\wxheading{See also}

\helpref{wxRadioBox::SetLabel}{wxradioboxsetlabel}

\pythonnote{In place of a single overloaded method name, wxPython
implements the following methods:\par
\indented{2cm}{\begin{twocollist}
\twocolitem{{\bf GetLabel()}}{Returns the radiobox label.}
\twocolitem{{\bf GetItemLabel(n)}}{Returns the label for the given button.}
\end{twocollist}}
}


\membersection{wxRadioBox::GetSelection}\label{wxradioboxgetselection}

\constfunc{int}{GetSelection}{\void}

Returns the zero-based position of the selected button.

\membersection{wxRadioBox::GetStringSelection}\label{wxradioboxgetstringselection}

\constfunc{wxString}{GetStringSelection}{\void}

Returns the selected string.

\membersection{wxRadioBox::GetString}\label{wxradioboxgetstring}

\constfunc{wxString}{GetString}{\param{int}{ n}}

Returns the label for the button at the given position.

\wxheading{Parameters}

\docparam{n}{The zero-based button position.}

\membersection{wxRadioBox::Number}\label{wxradioboxnumber}

\constfunc{int}{Number}{\void}

{\bf Obsolescence note:} This method is obsolete and was replaced with 
\helpref{GetCount}{wxradioboxgetcount}, please use the new method in the new
code. This method is only available if wxWindows was compiled with 
{\tt WXWIN\_COMPATIBILITY\_2\_2} defined and will disappear completely in
future versions.

Returns the number of buttons in the radiobox.

\membersection{wxRadioBox::SetLabel}\label{wxradioboxsetlabel}

\func{void}{SetLabel}{\param{const wxString\&}{ label}}

Sets the radiobox label.

\func{void}{SetLabel}{\param{int }{n}, \param{const wxString\&}{ label}}

Sets a label for a radio button.

\wxheading{Parameters}

\docparam{label}{The label to set.}

\docparam{n}{The zero-based button index.}

\pythonnote{In place of a single overloaded method name, wxPython
implements the following methods:\par
\indented{2cm}{\begin{twocollist}
\twocolitem{{\bf SetLabel(string)}}{Sets the radiobox label.}
\twocolitem{{\bf SetItemLabel(n, string)}}{Sets a label for a radio button.}
\end{twocollist}}
}

\membersection{wxRadioBox::SetSelection}\label{wxradioboxsetselection}

\func{void}{SetSelection}{\param{int}{ n}}

Sets a button by passing the desired string position. This does not cause
a wxEVT\_COMMAND\_RADIOBOX\_SELECTED event to get emitted.

\wxheading{Parameters}

\docparam{n}{The zero-based button position.}

\membersection{wxRadioBox::SetStringSelection}\label{wxradioboxsetstringselection}

\func{void}{SetStringSelection}{\param{const wxString\& }{string}}

Sets the selection to a button by passing the desired string. This does not cause
a wxEVT\_COMMAND\_RADIOBOX\_SELECTED event to get emitted.

\wxheading{Parameters}

\docparam{string}{The label of the button to select.}

\membersection{wxRadioBox::Show}\label{wxradioboxshow}

\func{void}{Show}{\param{const bool}{ show}}

Shows or hides the entire radiobox.

\func{void}{Show}{\param{int }{item}, \param{const bool}{ show}}

Shows or hides individual buttons.

\wxheading{Parameters}

\docparam{show}{true to show, false to hide.}

\docparam{item}{The zero-based position of the button to show or hide.}

\pythonnote{In place of a single overloaded method name, wxPython
implements the following methods:\par
\indented{2cm}{\begin{twocollist}
\twocolitem{{\bf Show(flag)}}{Shows or hides the entire radiobox.}
\twocolitem{{\bf ShowItem(n, flag)}}{Shows or hides individual buttons.}
\end{twocollist}}
}

