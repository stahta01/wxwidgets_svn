%
% automatically generated by HelpGen $Revision$ from
% access.h at 11/Apr/03 12:31:15
%

\section{\class{wxAccessible}}\label{wxaccessible}

The wxAccessible class allows wxWindows applications, and
wxWindows itself, to return extended information about user interface elements
to client applications such as screen readers. This is the
main way in which wxWindows implements accessibility features.

At present, only Microsoft Active Accessibility is supported
by this class.

To use this class, derive from wxAccessible, implement appropriate
functions, and associate an object of the class with a
window using \helpref{wxWindow::SetAccessible}{wxwindowsetaccessible}.

All functions return an indication of success, failure, or not implemented
using values of the wxAccStatus enum type.

If you return wxACC\_NOT\_IMPLEMENTED from any functioon, the system will try to
implement the appropriate functionality. However this will not work with
all functions.

Most functions work with an {\it object id}, which can be zero to refer to
'this' UI element, or greater than zero to refer to the nth child element.
This allows you to specify elements that don't have a corresponding wxWindow or
wxAccessible; for example, the sash of a splitter window.

For details on the semantics of functions and types, please refer to the
Microsoft Active Accessibility 1.2 documentation.

This class is compiled into wxWindows only if the wxUSE\_ACCESSIBILITY setup
symbol is set to 1.

\wxheading{Derived from}

\helpref{wxObject}{wxobject}

\wxheading{Include files}

<wx/access.h>

\wxheading{Data structures}

Functions return a wxAccStatus error code, which
may be one of the following:

{\small \begin{verbatim}
typedef enum
{
    wxACC\_FAIL,             // The function failed
    wxACC\_FALSE,            // The function returned false
    wxACC\_OK,               // The function completed successfully
    wxACC\_NOT\_IMPLEMENTED, // The function is not implemented
    wxACC\_NOT\_SUPPORTED    // The function is not supported
} wxAccStatus
\end{verbatim}}

Directions of navigation are represented by
the following:

{\small \begin{verbatim}
typedef enum
{
    wxNAVDIR\_DOWN,
    wxNAVDIR\_FIRSTCHILD,
    wxNAVDIR\_LASTCHILD,
    wxNAVDIR\_LEFT,
    wxNAVDIR\_NEXT,
    wxNAVDIR\_PREVIOUS,
    wxNAVDIR\_RIGHT,
    wxNAVDIR\_UP
} wxNavDir
\end{verbatim}}

The role of a user interface element is represented
by the following type:

{\small \begin{verbatim}
typedef enum {  
    wxROLE\_NONE,
    wxROLE\_SYSTEM\_ALERT,
    wxROLE\_SYSTEM\_ANIMATION,
    wxROLE\_SYSTEM\_APPLICATION,
    wxROLE\_SYSTEM\_BORDER,
    wxROLE\_SYSTEM\_BUTTONDROPDOWN,
    wxROLE\_SYSTEM\_BUTTONDROPDOWNGRID,
    wxROLE\_SYSTEM\_BUTTONMENU,
    wxROLE\_SYSTEM\_CARET,
    wxROLE\_SYSTEM\_CELL,
    wxROLE\_SYSTEM\_CHARACTER,
    wxROLE\_SYSTEM\_CHART,
    wxROLE\_SYSTEM\_CHECKBUTTON,
    wxROLE\_SYSTEM\_CLIENT,
    wxROLE\_SYSTEM\_CLOCK,
    wxROLE\_SYSTEM\_COLUMN,
    wxROLE\_SYSTEM\_COLUMNHEADER,
    wxROLE\_SYSTEM\_COMBOBOX,
    wxROLE\_SYSTEM\_CURSOR,
    wxROLE\_SYSTEM\_DIAGRAM,
    wxROLE\_SYSTEM\_DIAL,
    wxROLE\_SYSTEM\_DIALOG,
    wxROLE\_SYSTEM\_DOCUMENT,
    wxROLE\_SYSTEM\_DROPLIST,
    wxROLE\_SYSTEM\_EQUATION,
    wxROLE\_SYSTEM\_GRAPHIC,
    wxROLE\_SYSTEM\_GRIP,
    wxROLE\_SYSTEM\_GROUPING,
    wxROLE\_SYSTEM\_HELPBALLOON,
    wxROLE\_SYSTEM\_HOTKEYFIELD,
    wxROLE\_SYSTEM\_INDICATOR,
    wxROLE\_SYSTEM\_LINK,
    wxROLE\_SYSTEM\_LIST,
    wxROLE\_SYSTEM\_LISTITEM,
    wxROLE\_SYSTEM\_MENUBAR,
    wxROLE\_SYSTEM\_MENUITEM,
    wxROLE\_SYSTEM\_MENUPOPUP,
    wxROLE\_SYSTEM\_OUTLINE,
    wxROLE\_SYSTEM\_OUTLINEITEM,
    wxROLE\_SYSTEM\_PAGETAB,
    wxROLE\_SYSTEM\_PAGETABLIST,
    wxROLE\_SYSTEM\_PANE,
    wxROLE\_SYSTEM\_PROGRESSBAR,
    wxROLE\_SYSTEM\_PROPERTYPAGE,
    wxROLE\_SYSTEM\_PUSHBUTTON,
    wxROLE\_SYSTEM\_RADIOBUTTON,
    wxROLE\_SYSTEM\_ROW,
    wxROLE\_SYSTEM\_ROWHEADER,
    wxROLE\_SYSTEM\_SCROLLBAR,
    wxROLE\_SYSTEM\_SEPARATOR,
    wxROLE\_SYSTEM\_SLIDER,
    wxROLE\_SYSTEM\_SOUND,
    wxROLE\_SYSTEM\_SPINBUTTON,
    wxROLE\_SYSTEM\_STATICTEXT,
    wxROLE\_SYSTEM\_STATUSBAR,
    wxROLE\_SYSTEM\_TABLE,
    wxROLE\_SYSTEM\_TEXT,
    wxROLE\_SYSTEM\_TITLEBAR,
    wxROLE\_SYSTEM\_TOOLBAR,
    wxROLE\_SYSTEM\_TOOLTIP,
    wxROLE\_SYSTEM\_WHITESPACE,
    wxROLE\_SYSTEM\_WINDOW
} wxAccRole
\end{verbatim}}

Objects are represented by the following type:

{\small \begin{verbatim}
typedef enum {
    wxOBJID\_WINDOW =    0x00000000,
    wxOBJID\_SYSMENU =   0xFFFFFFFF,
    wxOBJID\_TITLEBAR =  0xFFFFFFFE,
    wxOBJID\_MENU =      0xFFFFFFFD,
    wxOBJID\_CLIENT =    0xFFFFFFFC,
    wxOBJID\_VSCROLL =   0xFFFFFFFB,
    wxOBJID\_HSCROLL =   0xFFFFFFFA,
    wxOBJID\_SIZEGRIP =  0xFFFFFFF9,
    wxOBJID\_CARET =     0xFFFFFFF8,
    wxOBJID\_CURSOR =    0xFFFFFFF7,
    wxOBJID\_ALERT =     0xFFFFFFF6,
    wxOBJID\_SOUND =     0xFFFFFFF5
} wxAccObject
\end{verbatim}}

Selection actions are identified by
this type:

{\small \begin{verbatim}
typedef enum
{
    wxACC\_SEL\_NONE            = 0,
    wxACC\_SEL\_TAKEFOCUS       = 1,
    wxACC\_SEL\_TAKESELECTION   = 2,
    wxACC\_SEL\_EXTENDSELECTION = 4,
    wxACC\_SEL\_ADDSELECTION    = 8,
    wxACC\_SEL\_REMOVESELECTION = 16
} wxAccSelectionFlags
\end{verbatim}}

States are represented by the following:

{\small \begin{verbatim}
#define wxACC\_STATE\_SYSTEM\_ALERT\_HIGH       0x00000001
#define wxACC\_STATE\_SYSTEM\_ALERT\_MEDIUM     0x00000002
#define wxACC\_STATE\_SYSTEM\_ALERT\_LOW        0x00000004
#define wxACC\_STATE\_SYSTEM\_ANIMATED         0x00000008
#define wxACC\_STATE\_SYSTEM\_BUSY             0x00000010
#define wxACC\_STATE\_SYSTEM\_CHECKED          0x00000020
#define wxACC\_STATE\_SYSTEM\_COLLAPSED        0x00000040
#define wxACC\_STATE\_SYSTEM\_DEFAULT          0x00000080
#define wxACC\_STATE\_SYSTEM\_EXPANDED         0x00000100
#define wxACC\_STATE\_SYSTEM\_EXTSELECTABLE    0x00000200
#define wxACC\_STATE\_SYSTEM\_FLOATING         0x00000400
#define wxACC\_STATE\_SYSTEM\_FOCUSABLE        0x00000800
#define wxACC\_STATE\_SYSTEM\_FOCUSED          0x00001000
#define wxACC\_STATE\_SYSTEM\_HOTTRACKED       0x00002000
#define wxACC\_STATE\_SYSTEM\_INVISIBLE        0x00004000
#define wxACC\_STATE\_SYSTEM\_MARQUEED         0x00008000
#define wxACC\_STATE\_SYSTEM\_MIXED            0x00010000
#define wxACC\_STATE\_SYSTEM\_MULTISELECTABLE  0x00020000
#define wxACC\_STATE\_SYSTEM\_OFFSCREEN        0x00040000
#define wxACC\_STATE\_SYSTEM\_PRESSED          0x00080000
#define wxACC\_STATE\_SYSTEM\_PROTECTED        0x00100000
#define wxACC\_STATE\_SYSTEM\_READONLY         0x00200000
#define wxACC\_STATE\_SYSTEM\_SELECTABLE       0x00400000
#define wxACC\_STATE\_SYSTEM\_SELECTED         0x00800000
#define wxACC\_STATE\_SYSTEM\_SELFVOICING      0x01000000
#define wxACC\_STATE\_SYSTEM\_UNAVAILABLE      0x02000000
\end{verbatim}}

Event identifiers that can be sent via \helpref{wxAccessible::NotifyEvent}{wxaccessiblenotifyevent} are
as follows:

{\small \begin{verbatim}
#define wxACC\_EVENT\_SYSTEM\_SOUND              0x0001
#define wxACC\_EVENT\_SYSTEM\_ALERT              0x0002
#define wxACC\_EVENT\_SYSTEM\_FOREGROUND         0x0003
#define wxACC\_EVENT\_SYSTEM\_MENUSTART          0x0004
#define wxACC\_EVENT\_SYSTEM\_MENUEND            0x0005
#define wxACC\_EVENT\_SYSTEM\_MENUPOPUPSTART     0x0006
#define wxACC\_EVENT\_SYSTEM\_MENUPOPUPEND       0x0007
#define wxACC\_EVENT\_SYSTEM\_CAPTURESTART       0x0008
#define wxACC\_EVENT\_SYSTEM\_CAPTUREEND         0x0009
#define wxACC\_EVENT\_SYSTEM\_MOVESIZESTART      0x000A
#define wxACC\_EVENT\_SYSTEM\_MOVESIZEEND        0x000B
#define wxACC\_EVENT\_SYSTEM\_CONTEXTHELPSTART   0x000C
#define wxACC\_EVENT\_SYSTEM\_CONTEXTHELPEND     0x000D
#define wxACC\_EVENT\_SYSTEM\_DRAGDROPSTART      0x000E
#define wxACC\_EVENT\_SYSTEM\_DRAGDROPEND        0x000F
#define wxACC\_EVENT\_SYSTEM\_DIALOGSTART        0x0010
#define wxACC\_EVENT\_SYSTEM\_DIALOGEND          0x0011
#define wxACC\_EVENT\_SYSTEM\_SCROLLINGSTART     0x0012
#define wxACC\_EVENT\_SYSTEM\_SCROLLINGEND       0x0013
#define wxACC\_EVENT\_SYSTEM\_SWITCHSTART        0x0014
#define wxACC\_EVENT\_SYSTEM\_SWITCHEND          0x0015
#define wxACC\_EVENT\_SYSTEM\_MINIMIZESTART      0x0016
#define wxACC\_EVENT\_SYSTEM\_MINIMIZEEND        0x0017
#define wxACC\_EVENT\_OBJECT\_CREATE                 0x8000
#define wxACC\_EVENT\_OBJECT\_DESTROY                0x8001
#define wxACC\_EVENT\_OBJECT\_SHOW                   0x8002
#define wxACC\_EVENT\_OBJECT\_HIDE                   0x8003
#define wxACC\_EVENT\_OBJECT\_REORDER                0x8004
#define wxACC\_EVENT\_OBJECT\_FOCUS                  0x8005
#define wxACC\_EVENT\_OBJECT\_SELECTION              0x8006
#define wxACC\_EVENT\_OBJECT\_SELECTIONADD           0x8007
#define wxACC\_EVENT\_OBJECT\_SELECTIONREMOVE        0x8008
#define wxACC\_EVENT\_OBJECT\_SELECTIONWITHIN        0x8009
#define wxACC\_EVENT\_OBJECT\_STATECHANGE            0x800A
#define wxACC\_EVENT\_OBJECT\_LOCATIONCHANGE         0x800B
#define wxACC\_EVENT\_OBJECT\_NAMECHANGE             0x800C
#define wxACC\_EVENT\_OBJECT\_DESCRIPTIONCHANGE      0x800D
#define wxACC\_EVENT\_OBJECT\_VALUECHANGE            0x800E
#define wxACC\_EVENT\_OBJECT\_PARENTCHANGE           0x800F
#define wxACC\_EVENT\_OBJECT\_HELPCHANGE             0x8010
#define wxACC\_EVENT\_OBJECT\_DEFACTIONCHANGE        0x8011
#define wxACC\_EVENT\_OBJECT\_ACCELERATORCHANGE      0x8012
\end{verbatim}}

\latexignore{\rtfignore{\wxheading{Members}}}

\membersection{wxAccessible::wxAccessible}\label{wxaccessiblewxaccessible}

\func{}{wxAccessible}{\param{wxWindow* }{win = NULL}}

Constructor, taking an optional window. The object can be associated with
a window later.

\membersection{wxAccessible::\destruct{wxAccessible}}\label{wxaccessibledtor}

\func{}{\destruct{wxAccessible}}{\void}

Destructor.

\membersection{wxAccessible::DoDefaultAction}\label{wxaccessibledodefaultaction}

\func{virtual wxAccStatus}{DoDefaultAction}{\param{int }{childId}}

Performs the default action for the object. {\it childId} is 0 (the action for this object)
or greater than 0 (the action for a child). Return wxACC\_NOT\_SUPPORTED if there
is no default action for this window (e.g. an edit control).

\membersection{wxAccessible::GetChild}\label{wxaccessiblegetchild}

\func{virtual wxAccStatus}{GetChild}{\param{int }{childId}, \param{wxAccessible** }{child}}

Gets the specified child (starting from 1). If {\it child} is NULL and the return value is wxACC\_OK,
this means that the child is a simple element and not an accessible object.

\membersection{wxAccessible::GetChildCount}\label{wxaccessiblegetchildcount}

\func{virtual wxAccStatus}{GetChildCount}{\param{int* }{childCount}}

Returns the number of children in {\it childCount}.

\membersection{wxAccessible::GetDefaultAction}\label{wxaccessiblegetdefaultaction}

\func{virtual wxAccStatus}{GetDefaultAction}{\param{int }{childId}, \param{wxString* }{actionName}}

Gets the default action for this object (0) or a child (greater than 0).
Return wxACC\_OK even if there is no action. {\it actionName} is the action, or the empty
string if there is no action. The retrieved string describes the action that is performed on an object,
not what the object does as a result. For example, a toolbar button that prints
a document has a default action of "Press" rather than "Prints the current document."

\membersection{wxAccessible::GetDescription}\label{wxaccessiblegetdescription}

\func{virtual wxAccStatus}{GetDescription}{\param{int }{childId}, \param{wxString* }{description}}

Returns the description for this object or a child.

\membersection{wxAccessible::GetFocus}\label{wxaccessiblegetfocus}

\func{virtual wxAccStatus}{GetFocus}{\param{int* }{childId}, \param{wxAccessible** }{child}}

Gets the window with the keyboard focus. If childId is 0 and child is NULL, no object in
this subhierarchy has the focus. If this object has the focus, child should be 'this'.

\membersection{wxAccessible::GetHelpText}\label{wxaccessiblegethelptext}

\func{virtual wxAccStatus}{GetHelpText}{\param{int }{childId}, \param{wxString* }{helpText}}

Returns help text for this object or a child, similar to tooltip text.

\membersection{wxAccessible::GetKeyboardShortcut}\label{wxaccessiblegetkeyboardshortcut}

\func{virtual wxAccStatus}{GetKeyboardShortcut}{\param{int }{childId}, \param{wxString* }{shortcut}}

Returns the keyboard shortcut for this object or child.
Return e.g. ALT+K.

\membersection{wxAccessible::GetLocation}\label{wxaccessiblegetlocation}

\func{virtual wxAccStatus}{GetLocation}{\param{wxRect\& }{rect}, \param{int }{elementId}}

Returns the rectangle for this object (id is 0) or a child element (id is greater than 0).
{\it rect} is in screen coordinates.

\membersection{wxAccessible::GetName}\label{wxaccessiblegetname}

\func{virtual wxAccStatus}{GetName}{\param{int }{childId}, \param{wxString* }{name}}

Gets the name of the specified object.

\membersection{wxAccessible::GetParent}\label{wxaccessiblegetparent}

\func{virtual wxAccStatus}{GetParent}{\param{wxAccessible** }{parent}}

Returns the parent of this object, or NULL.

\membersection{wxAccessible::GetRole}\label{wxaccessiblegetrole}

\func{virtual wxAccStatus}{GetRole}{\param{int }{childId}, \param{wxAccRole* }{role}}

Returns a role constant describing this object. See \helpref{wxAccessible}{wxaccessible} for a list
of these roles.

\membersection{wxAccessible::GetSelections}\label{wxaccessiblegetselections}

\func{virtual wxAccStatus}{GetSelections}{\param{wxVariant* }{selections}}

Gets a variant representing the selected children
of this object.

Acceptable values are:

\begin{itemize}
\item  a null variant (IsNull() returns TRUE)
\item  a list variant (GetType() == wxT("list"))
\item  an integer representing the selected child element,
or 0 if this object is selected (GetType() == wxT("long"))
\item a "void*" pointer to a wxAccessible child object
\end{itemize}

\membersection{wxAccessible::GetState}\label{wxaccessiblegetstate}

\func{virtual wxAccStatus}{GetState}{\param{int }{childId}, \param{long* }{state}}

Returns a state constant. See \helpref{wxAccessible}{wxaccessible} for a list
of these states.

\membersection{wxAccessible::GetValue}\label{wxaccessiblegetvalue}

\func{virtual wxAccStatus}{GetValue}{\param{int }{childId}, \param{wxString* }{strValue}}

Returns a localized string representing the value for the object
or child.

\membersection{wxAccessible::GetWindow}\label{wxaccessiblegetwindow}

\func{wxWindow*}{GetWindow}{\void}

Returns the window associated with this object.

\membersection{wxAccessible::HitTest}\label{wxaccessiblehittest}

\func{virtual wxAccStatus}{HitTest}{\param{const wxPoint\& }{pt}, \param{int* }{childId}, \param{wxAccessible** }{childObject}}

Returns a status value and object id to indicate whether the given point was on this or
a child object. Can return either a child object, or an integer
representing the child element, starting from 1.

{\it pt} is in screen coordinates.

\membersection{wxAccessible::Navigate}\label{wxaccessiblenavigate}

\func{virtual wxAccStatus}{Navigate}{\param{wxNavDir }{navDir}, \param{int }{fromId}, \param{int* }{toId}, \param{wxAccessible** }{toObject}}

Navigates from {\it fromId} to {\it toId}/{\it toObject}.

\membersection{wxAccessible::NotifyEvent}\label{wxaccessiblenotifyevent}

\func{virtual static void}{NotifyEvent}{\param{int}{ eventType}, \param{wxWindow* }{window}, \param{wxAccObjectt }{objectType}, \param{int }{objectType}}

Allows the application to send an event when something changes in an accessible object.

\membersection{wxAccessible::Select}\label{wxaccessibleselect}

\func{virtual wxAccStatus}{Select}{\param{int }{childId}, \param{wxAccSelectionFlags }{selectFlags}}

Selects the object or child. See \helpref{wxAccessible}{wxaccessible} for a list
of the selection actions.

\membersection{wxAccessible::SetWindow}\label{wxaccessiblesetwindow}

\func{void}{SetWindow}{\param{wxWindow* }{window}}

Sets the window associated with this object.

