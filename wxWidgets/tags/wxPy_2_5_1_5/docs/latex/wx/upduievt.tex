\section{\class{wxUpdateUIEvent}}\label{wxupdateuievent}

This class is used for pseudo-events which are called by wxWindows
to give an application the chance to update various user interface elements.

\wxheading{Derived from}

\helpref{wxCommandEvent}{wxcommandevent}\\
\helpref{wxEvent}{wxevent}\\
\helpref{wxObject}{wxobject}

\wxheading{Include files}

<wx/event.h>

\wxheading{Event table macros}

To process an update event, use these event handler macros to direct input to member
functions that take a wxUpdateUIEvent argument.

\twocolwidtha{7cm}
\begin{twocollist}\itemsep=0pt
\twocolitem{{\bf EVT\_UPDATE\_UI(id, func)}}{Process a wxEVT\_UPDATE\_UI event for the command with the given id.}
\twocolitem{{\bf EVT\_UPDATE\_UI\_RANGE(id1, id2, func)}}{Process a wxEVT\_UPDATE\_UI event for any command with id included in the given range.}
\end{twocollist}

\wxheading{Remarks}

Without update UI events, an application has to work hard to check/uncheck, enable/disable,
and set the text for elements such as menu items and toolbar buttons.
The code for doing this has to be mixed up with the code that is invoked when
an action is invoked for a menu item or button.

With update UI events, you define an event handler to look at the state of
the application and change UI elements accordingly. wxWindows will call your
member functions in idle time, so you don't have to worry where to call this code.
In addition to being a clearer and more declarative method, it also means you
don't have to worry whether you're updating a toolbar or menubar identifier.
The same handler can update a menu item and toolbar button, if the identifier is the same.

Instead of directly manipulating the menu or button, you call functions in the event
object, such as \helpref{wxUpdateUIEvent::Check}{wxupdateuieventcheck}. wxWindows
will determine whether such a call has been made, and which UI element to update.

These events will work for popup menus as well as menubars. Just before a menu is popped
up, \helpref{wxMenu::UpdateUI}{wxmenuupdateui} is called to process any UI events for
the window that owns the menu.

If you find that the overhead of UI update processing is affecting
your application, you can do one or both of the following:

\begin{enumerate}
\item Call \helpref{wxUpdateUIEvent::SetMode}{wxupdateuieventsetmode} with
a value of wxUPDATE\_UI\_PROCESS\_SPECIFIED, and set the extra style
wxWS\_EX\_PROCESS\_UPDATE\_EVENTS for every window that should receive update events.
No other windows will receive update events.
\item Call \helpref{wxUpdateUIEvent::SetUpdateInterval}{wxupdateuieventsetupdateinterval} with
a millisecond value to set the delay between updates. You may need
to call \helpref{wxWindow::UpdateWindowUI}{wxwindowupdatewindowui} at critical
points, for example when a dialog is about to be shown, in case the user
sees a slight delay before windows are updated.
\end{enumerate}

Note that although events are sent in idle time, defining a wxIdleEvent
handler for a window does not affect this because the events are sent from \helpref{wxWindow::OnInternalIdle}{wxwindowoninternalidle} 
which is {\bf always} called in idle time.

wxWindows tries to optimize update events on some platforms. On Windows
and GTK+, events for menubar items are only sent when the menu is about
to be shown, and not in idle time.

\wxheading{See also}

\helpref{Event handling overview}{eventhandlingoverview}

\latexignore{\rtfignore{\wxheading{Members}}}

\membersection{wxUpdateUIEvent::wxUpdateUIEvent}

\func{}{wxUpdateUIEvent}{\param{wxWindowID }{commandId = 0}}

Constructor.

\membersection{wxUpdateUIEvent::m\_checked}

\member{bool}{m\_checked}

true if the element should be checked, false otherwise.

\membersection{wxUpdateUIEvent::m\_enabled}

\member{bool}{m\_checked}

true if the element should be enabled, false otherwise.

\membersection{wxUpdateUIEvent::m\_setChecked}

\member{bool}{m\_setChecked}

true if the application has set the {\bf m\_checked} member.

\membersection{wxUpdateUIEvent::m\_setEnabled}

\member{bool}{m\_setEnabled}

true if the application has set the {\bf m\_enabled} member.

\membersection{wxUpdateUIEvent::m\_setText}

\member{bool}{m\_setText}

true if the application has set the {\bf m\_text} member.

\membersection{wxUpdateUIEvent::m\_text}

\member{wxString}{m\_text}

Holds the text with which the the application wishes to
update the UI element.

\membersection{wxUpdateUIEvent::CanUpdate}\label{wxupdateuieventcanupdate}

\func{static bool}{CanUpdate}{\param{wxWindow*}{ window}}

Returns {\tt true} if it is appropriate to update (send UI update events to)
this window.

This function looks at the mode used (see \helpref{wxUpdateUIEvent::SetMode}{wxupdateuieventsetmode}),
the wxWS\_EX\_PROCESS\_UPDATE\_EVENTS flag in {\it window},
the time update events were last sent in idle time, and
the update interval, to determine whether events should be sent to
this window now. By default this will always return {\tt true} because
the update mode is initially wxUPDATE\_UI\_PROCESS\_ALL and
the interval is set to 0; so update events will be sent as
often as possible. You can reduce the frequency that events
are sent by changing the mode and/or setting an update interval.

\wxheading{See also}

\helpref{wxUpdateUIEvent::ResetUpdateTime}{wxupdateuieventresetupdatetime}, 
\helpref{wxUpdateUIEvent::SetUpdateInterval}{wxupdateuieventsetupdateinterval}, 
\helpref{wxUpdateUIEvent::SetMode}{wxupdateuieventsetmode}

\membersection{wxUpdateUIEvent::Check}\label{wxupdateuieventcheck}

\func{void}{Check}{\param{bool}{ check}}

Check or uncheck the UI element.

\membersection{wxUpdateUIEvent::Enable}\label{wxupdateuieventenable}

\func{void}{Enable}{\param{bool}{ enable}}

Enable or disable the UI element.

\membersection{wxUpdateUIEvent::GetChecked}\label{wxupdateuieventgetchecked}

\constfunc{bool}{GetChecked}{\void}

Returns true if the UI element should be checked.

\membersection{wxUpdateUIEvent::GetEnabled}\label{wxupdateuieventgetenabled}

\constfunc{bool}{GetEnabled}{\void}

Returns true if the UI element should be enabled.

\membersection{wxUpdateUIEvent::GetSetChecked}\label{wxupdateuieventgetsetchecked}

\constfunc{bool}{GetSetChecked}{\void}

Returns true if the application has called {\bf SetChecked}. For wxWindows internal use only.

\membersection{wxUpdateUIEvent::GetSetEnabled}\label{wxupdateuieventgetsetenabled}

\constfunc{bool}{GetSetEnabled}{\void}

Returns true if the application has called {\bf SetEnabled}. For wxWindows internal use only.

\membersection{wxUpdateUIEvent::GetSetText}\label{wxupdateuieventgetsettext}

\constfunc{bool}{GetSetText}{\void}

Returns true if the application has called {\bf SetText}. For wxWindows internal use only.

\membersection{wxUpdateUIEvent::GetText}\label{wxupdateuieventgettext}

\constfunc{wxString}{GetText}{\void}

Returns the text that should be set for the UI element.

\membersection{wxUpdateUIEvent::GetMode}\label{wxupdateuieventgetmode}

\func{static wxUpdateUIMode}{GetMode}{\void}

Static function returning a value specifying how wxWindows
will send update events: to all windows, or only to those which specify that they
will process the events.

See \helpref{wxUpdateUIEvent::SetMode}{wxupdateuieventsetmode}.

\membersection{wxUpdateUIEvent::GetUpdateInterval}\label{wxupdateuieventgetupdateinterval}

\func{static long}{GetUpdateInterval}{\void}

Returns the current interval between updates in milliseconds.
-1 disables updates, 0 updates as frequently as possible.

See \helpref{wxUpdateUIEvent::SetUpdateInterval}{wxupdateuieventsetupdateinterval}.

\membersection{wxUpdateUIEvent::ResetUpdateTime}\label{wxupdateuieventresetupdatetime}

\func{static void}{ResetUpdateTime}{\void}

Used internally to reset the last-updated time to the
current time. It is assumed that update events are
normally sent in idle time, so this is called at the end of
idle processing.

\wxheading{See also}

\helpref{wxUpdateUIEvent::CanUpdate}{wxupdateuieventcanupdate}, 
\helpref{wxUpdateUIEvent::SetUpdateInterval}{wxupdateuieventsetupdateinterval}, 
\helpref{wxUpdateUIEvent::SetMode}{wxupdateuieventsetmode}

\membersection{wxUpdateUIEvent::SetMode}\label{wxupdateuieventsetmode}

\func{static void}{SetMode}{\param{wxIdleMode }{mode}}

Specify how wxWindows will send update events: to
all windows, or only to those which specify that they
will process the events.

{\it mode} may be one of the following values.
The default is wxUPDATE\_UI\_PROCESS\_ALL.

\begin{verbatim}
enum wxUpdateUIMode
{
        // Send UI update events to all windows
    wxUPDATE_UI_PROCESS_ALL,

        // Send UI update events to windows that have
        // the wxWS_EX_PROCESS_UI_UPDATES flag specified
    wxUPDATE_UI_PROCESS_SPECIFIED
};
\end{verbatim}

\membersection{wxUpdateUIEvent::SetText}\label{wxupdateuieventsettext}

\func{void}{SetText}{\param{const wxString\&}{ text}}

Sets the text for this UI element.

\membersection{wxUpdateUIEvent::SetUpdateInterval}\label{wxupdateuieventsetupdateinterval}

\func{static void}{SetUpdateInterval}{\param{long }{updateInterval}}

Sets the interval between updates in milliseconds.
Set to -1 to disable updates, or to 0 to update as frequently as possible.
The default is 0.

Use this to reduce the overhead of UI update events if your application
has a lot of windows. If you set the value to -1 or greater than 0,
you may also need to call \helpref{wxWindow::UpdateWindowUI}{wxwindowupdatewindowui} 
at appropriate points in your application, such as when a dialog
is about to be shown.

