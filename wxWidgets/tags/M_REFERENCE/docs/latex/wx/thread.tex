\section{\class{wxThread}}\label{wxthread}

A wxThread manages a system thread, code which executes as a mini-process within the application.

\wxheading{Derived from}

None.

\wxheading{See also}

\helpref{wxMutex}{wxmutex}, \helpref{wxCondition}{wxcondition}

\latexignore{\rtfignore{\wxheading{Members}}}

\membersection{wxThread::wxThread}\label{wxthreadconstr}

\func{}{wxThread}{\void}

Default constructor.

\membersection{wxThread::\destruct{wxThread}}

\func{}{\destruct{wxThread}}{\void}

Destroys the wxThread object.

\membersection{wxThread::Create}\label{wxthreadcreate}

\func{wxThreadError}{Create}{\void}

Creates a thread control.

\wxheading{Return value}

One of:

\twocolwidtha{7cm}
\begin{twocollist}\itemsep=0pt
\twocolitem{{\bf THREAD\_NO\_ERROR}}{There was no error.}
\twocolitem{{\bf THREAD\_NO\_RESOURCE}}{There were insufficient resources to create a new thread.}
\twocolitem{{\bf THREAD\_RUNNING}}{The thread is already running.}
\end{twocollist}

\membersection{wxThread::DeferDestroy}\label{wxthreaddeferdestroy}

\func{void}{DeferDestroy}{\param{bool}{ defer}}

If {\it defer} is TRUE, defers thread destruction. This function affects the
calling thread.

\membersection{wxThread::Destroy}\label{wxthreaddestroy}

\func{wxThreadError}{Destroy}{\void}

Destroys the thread immediately unless the application has specified deferral via \helpref{wxThread::DeferDestroy}{wxthreaddeferdestroy}.

\wxheading{Return value}

One of:

\twocolwidtha{7cm}
\begin{twocollist}\itemsep=0pt
\twocolitem{{\bf THREAD\_NO\_ERROR}}{There was no error.}
\twocolitem{{\bf THREAD\_NOT\_RUNNING}}{The thread is not running.}
\end{twocollist}

\membersection{wxThread::GetID}\label{wxthreadgetid}

\constfunc{unsigned long}{GetID}{\void}

Gets the thread identifier.

\membersection{wxThread::GetPriority}\label{wxthreadgetpriority}

\constfunc{int}{GetPriority}{\void}

Gets the priority of the thread, between zero and 100.

The following priorities are already defined:

\twocolwidtha{7cm}
\begin{twocollist}\itemsep=0pt
\twocolitem{{\bf WXTHREAD\_MIN\_PRIORITY}}{0}
\twocolitem{{\bf WXTHREAD\_DEFAULT\_PRIORITY}}{50}
\twocolitem{{\bf WXTHREAD\_MAX\_PRIORITY}}{100}
\end{twocollist}

\membersection{wxThread::IsAlive}\label{wxthreadisalive}

\constfunc{bool}{IsAlive}{\void}

Returns TRUE if the thread is alive.

\membersection{wxThread::IsMain}\label{wxthreadismain}

\constfunc{bool}{IsMain}{\void}

Returns TRUE if the thread is the main application thread.

\membersection{wxThread::Join}\label{wxthreadjoin}

\func{void*}{Join}{\void}

Waits for the termination of the thread. Returns a platform-specific exit code. TODO

\membersection{wxThread::OnExit}\label{wxthreadonexit}

\func{void}{OnExit}{\void}

Called when the thread exits. The default implementation calls \helpref{wxThread::Join}{wxthreadjoin}.

\membersection{wxThread::SetPriority}\label{wxthreadsetpriority}

\func{void}{SetPriority}{\param{int}{ priority}}

Sets the priority of the thread, between zero and 100. This must be set before the thread is created.

The following priorities are already defined:

\twocolwidtha{7cm}
\begin{twocollist}\itemsep=0pt
\twocolitem{{\bf WXTHREAD\_MIN\_PRIORITY}}{0}
\twocolitem{{\bf WXTHREAD\_DEFAULT\_PRIORITY}}{50}
\twocolitem{{\bf WXTHREAD\_MAX\_PRIORITY}}{100}
\end{twocollist}


%%% Local Variables: 
%%% mode: latex
%%% TeX-master: "referenc"
%%% End: 
