%%%%%%%%%%%%%%%%%%%%%%%%%%%%%%%%%%%%%%%%%%%%%%%%%%%%%%%%%%%%%%%%%%%%%%%%%%%%%%%
%% Name:        datespan.tex
%% Purpose:     wxDateSpan documentation
%% Author:      Vadim Zeitlin
%% Modified by:
%% Created:     04.04.00
%% RCS-ID:      $Id$
%% Copyright:   (c) Vadim Zeitlin
%% License:     wxWindows license
%%%%%%%%%%%%%%%%%%%%%%%%%%%%%%%%%%%%%%%%%%%%%%%%%%%%%%%%%%%%%%%%%%%%%%%%%%%%%%%

\section{\class{wxTimeSpan}}\label{wxtimespan}

wxTimeSpan class represents a time interval.

\wxheading{Derived from}

No base class

\wxheading{Include files}

<wx/datetime.h>

\wxheading{See also}

\helpref{Date classes overview}{wxdatetimeoverview},\rtfsp
\helpref{wxDateTime}{wxdatetime}

\latexignore{\rtfignore{\wxheading{Function groups}}}

\membersection{Static functions}

\membersection{Constructors}

wxTimeSpan()\\
\helpref{wxTimeSpan(hours, min, sec, msec)}{wxtimespan}

\membersection{Accessors}

\membersection{Operations}

\membersection{Tests}

\membersection{Formatting time spans}

\helpref{Format}{wxtimespanformat}

%%%%%%%%%%%%%%%%%%%%%%%%%%%%%%%%%%%%%%%%%%%%%%%%%%%%%%%%%%%%%%%%%%%%%%%%%%%%%%%
% Start of member function part                                               %
%%%%%%%%%%%%%%%%%%%%%%%%%%%%%%%%%%%%%%%%%%%%%%%%%%%%%%%%%%%%%%%%%%%%%%%%%%%%%%%

\helponly{\insertatlevel{2}{
    \wxheading{Members}
}}

\membersection{wxTimeSpan::Format}\label{wxtimespanformat}

\func{wxString}{Format}{\param{const wxChar * }{format = "\%H:\%M:\%S"}}

Returns the string containing the formatted representation of the time span.
The following format specifiers are allowed after \%:

\twocolwidtha{5cm}%
\begin{twocollist}\itemsep=0pt
\twocolitem{H}{number of {\bf H}ours}
\twocolitem{M}{number of {\bf M}inutes}
\twocolitem{S}{number of {\bf S}econds}
\twocolitem{l}{number of mi{\bf l}liseconds}
\twocolitem{D}{number of {\bf D}ays}
\twocolitem{E}{number of w{\bf E}eks}
\twocolitem{\%}{the percent character}
\end{twocollist}

Note that, for example, the number of hours in the description above is not
well defined: it can be either the total number of hours (for example, for a
time span of $50$ hours this would be $50$) or just the hour part of the time
span, which would be $2$ in this case as $50$ hours is equal to $2$ days and
$2$ hours.

wxTimeSpan resolves this ambiguity in the following way: if there had been,
indeed, the {\tt \%D} format specified preceding the {\tt \%H}, then it is
interpreted as $2$. Otherwise, it is $50$.

The same applies to all other format specifiers: if they follow a specifier of
larger unit, only the rest part is taken, otherwise the full value is used.

