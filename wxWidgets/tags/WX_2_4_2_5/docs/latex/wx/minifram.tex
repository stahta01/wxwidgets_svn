\section{\class{wxMiniFrame}}\label{wxminiframe}

A miniframe is a frame with a small title bar. It is suitable for floating toolbars that must not
take up too much screen area.

\wxheading{Derived from}

\helpref{wxFrame}{wxframe}\\
\helpref{wxWindow}{wxwindow}\\
\helpref{wxEvtHandler}{wxevthandler}\\
\helpref{wxObject}{wxobject}

\wxheading{Include files}

<wx/minifram.h>

\wxheading{Window styles}

\twocolwidtha{5cm}
\begin{twocollist}\itemsep=0pt
\twocolitem{\windowstyle{wxICONIZE}}{Display the frame iconized (minimized) (Windows only).}
\twocolitem{\windowstyle{wxCAPTION}}{Puts a caption on the frame.}
\twocolitem{\windowstyle{wxDEFAULT\_FRAME\_STYLE}}{Defined as {\bf wxMINIMIZE\_BOX \pipe wxMAXIMIZE\_BOX \pipe wxTHICK\_FRAME \pipe wxSYSTEM\_MENU \pipe wxCAPTION \pipe wxCLOSE\_BOX}.}
\twocolitem{\windowstyle{wxMINIMIZE}}{Identical to {\bf wxICONIZE}.}
\twocolitem{\windowstyle{wxMINIMIZE\_BOX}}{Displays a minimize box on the frame (Windows and Motif only).}
\twocolitem{\windowstyle{wxMAXIMIZE}}{Displays the frame maximized (Windows only).}
\twocolitem{\windowstyle{wxMAXIMIZE\_BOX}}{Displays a maximize box on the frame (Windows and Motif only).}
\twocolitem{\windowstyle{wxCLOSE\_BOX}}{Displays a close box on the frame.}
\twocolitem{\windowstyle{wxSTAY\_ON\_TOP}}{Stay on top of other windows (Windows only).}
\twocolitem{\windowstyle{wxSYSTEM\_MENU}}{Displays a system menu (Windows and Motif only).}
\twocolitem{\windowstyle{wxTHICK\_FRAME}}{Displays a thick frame around the window (Windows and Motif only).}
\twocolitem{\windowstyle{wxTINY\_CAPTION\_HORIZ}}{This style is obsolete and
not used any longer.}
\twocolitem{\windowstyle{wxTINY\_CAPTION\_VERT}}{This style is obsolete and not
used any longer.}
\twocolitem{\windowstyle{wxRESIZE\_BORDER}}{Displays a resizeable border around the window (Motif only;
for Windows, it is implicit in {\bf wxTHICK\_FRAME}).}
\end{twocollist}

See also \helpref{window styles overview}{windowstyles}. Note that all the window styles
above are ignored under GTK and the mini frame cannot be resized by the user.

\wxheading{Remarks}

This class has miniframe functionality under Windows and GTK, i.e. the presence
of mini frame will not be noted in the task bar and focus behaviour is different.
On other platforms, it behaves like a normal frame.

\wxheading{See also}

\helpref{wxMDIParentFrame}{wxmdiparentframe}, \helpref{wxMDIChildFrame}{wxmdichildframe},\rtfsp
\helpref{wxFrame}{wxframe}, \helpref{wxDialog}{wxdialog}

\latexignore{\rtfignore{\wxheading{Members}}}

\membersection{wxMiniFrame::wxMiniFrame}\label{wxminiframeconstr}

\func{}{wxMiniFrame}{\void}

Default constructor.

\func{}{wxMiniFrame}{\param{wxWindow* }{parent}, \param{wxWindowID }{id},\rtfsp
\param{const wxString\& }{title}, \param{const wxPoint\&}{ pos = wxDefaultPosition},\rtfsp
\param{const wxSize\&}{ size = wxDefaultSize}, \param{long}{ style = wxDEFAULT\_FRAME\_STYLE},\rtfsp
\param{const wxString\& }{name = ``frame"}}

Constructor, creating the window.

\wxheading{Parameters}

\docparam{parent}{The window parent. This may be NULL. If it is non-NULL, the frame will
always be displayed on top of the parent window on Windows.}

\docparam{id}{The window identifier. It may take a value of -1 to indicate a default value.}

\docparam{title}{The caption to be displayed on the frame's title bar.}

\docparam{pos}{The window position. A value of (-1, -1) indicates a default position, chosen by
either the windowing system or wxWidgets, depending on platform.}

\docparam{size}{The window size. A value of (-1, -1) indicates a default size, chosen by
either the windowing system or wxWidgets, depending on platform.}

\docparam{style}{The window style. See \helpref{wxMiniFrame}{wxminiframe}.}

\docparam{name}{The name of the window. This parameter is used to associate a name with the item,
allowing the application user to set Motif resource values for
individual windows.}

\wxheading{Remarks}

The frame behaves like a normal frame on non-Windows platforms.

\wxheading{See also}

\helpref{wxMiniFrame::Create}{wxminiframecreate}

\membersection{wxMiniFrame::\destruct{wxMiniFrame}}

\func{void}{\destruct{wxMiniFrame}}{\void}

Destructor. Destroys all child windows and menu bar if present.

\membersection{wxMiniFrame::Create}\label{wxminiframecreate}

\func{bool}{Create}{\param{wxWindow* }{parent}, \param{wxWindowID }{id},\rtfsp
\param{const wxString\& }{title}, \param{const wxPoint\&}{ pos = wxDefaultPosition},\rtfsp
\param{const wxSize\&}{ size = wxDefaultSize}, \param{long}{ style = wxDEFAULT\_FRAME\_STYLE},\rtfsp
\param{const wxString\& }{name = ``frame"}}

Used in two-step frame construction. See \helpref{wxMiniFrame::wxMiniFrame}{wxminiframeconstr}\rtfsp
for further details.



