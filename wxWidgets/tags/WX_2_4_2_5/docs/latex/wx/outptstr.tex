% -----------------------------------------------------------------------------
% wxOutputStream
% -----------------------------------------------------------------------------
\section{\class{wxOutputStream}}\label{wxoutputstream}

wxOutputStream is an abstract base class which may not be used directly.

\wxheading{Derived from}

\helpref{wxStreamBase}{wxstreambase}

\wxheading{Include files}

<wx/stream.h>

\latexignore{\rtfignore{\wxheading{Members}}}

% -----------
% ctor & dtor
% -----------

\membersection{wxOutputStream::wxOutputStream}

\func{}{wxOutputStream}{\void}

Creates a dummy wxOutputStream object.


\membersection{wxOutputStream::\destruct{wxOutputStream}}

\func{}{\destruct{wxOutputStream}}{\void}

Destructor.


\membersection{wxOutputStream::LastWrite}\label{wxoutputstreamlastwrite}

\constfunc{size\_t}{LastWrite}{\void}

Returns the number of bytes written during the last 
\helpref{Write()}{wxoutputstreamwrite}. It may return $0$ even if there is no
error on the stream if it is only temporarily impossible to write to it.


\membersection{wxOutputStream::PutC}

\func{void}{PutC}{\param{char}{ c}}

Puts the specified character in the output queue and increments the
stream position.


\membersection{wxOutputStream::SeekO}\label{wxoutputstreamseeko}

\func{off\_t}{SeekO}{\param{off\_t}{ pos}, \param{wxSeekMode}{ mode = wxFromStart}}

Changes the stream current position.

\wxheading{Parameters}

\docparam{pos}{Offset to seek to.}

\docparam{mode}{One of {\bf wxFromStart}, {\bf wxFromEnd}, {\bf wxFromCurrent}.}

\wxheading{Return value}

The new stream position or wxInvalidOffset on error.


\membersection{wxOutputStream::TellO}

\constfunc{off\_t}{TellO}{\void}

Returns the current stream position.


\membersection{wxOutputStream::Write}\label{wxoutputstreamwrite}

\func{wxOutputStream\&}{Write}{\param{const void *}{buffer}, \param{size\_t}{ size}}

Writes up to the specified amount of bytes using the data of {\it buffer}. Note
that not all data can always be written so you must check the number of bytes
really written to the stream using \helpref{LastWrite()}{wxoutputstreamlastwrite} 
when this function returns. In some cases (for example a write end of a pipe
which is currently full) it is even possible that there is no errors and zero
bytes have been written.

This function returns a reference on the current object, so the user can test
any states of the stream right away.

\func{wxOutputStream\&}{Write}{\param{wxInputStream\&}{ stream\_in}}

Reads data from the specified input stream and stores them 
in the current stream. The data is read until an error is raised
by one of the two streams.

