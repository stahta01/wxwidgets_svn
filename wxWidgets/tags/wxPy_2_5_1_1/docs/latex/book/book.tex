\documentstyle[a4,11pt,makeidx,verbatim,texhelp,fancyheadings,palatino]{report}
%
\newcommand{\indexit}[1]{#1\index{#1}}%
\newcommand{\pipe}[0]{$\|$\ }%
\definecolour{black}{0}{0}{0}%
\definecolour{cyan}{0}{255}{255}%
\definecolour{green}{0}{255}{0}%
\definecolour{magenta}{255}{0}{255}%
\definecolour{red}{255}{0}{0}%
\definecolour{blue}{0}{0}{200}%
\definecolour{yellow}{255}{255}{0}%
\definecolour{white}{255}{255}{255}%
%
\input psbox.tex
% Remove this for processing with dvi2ps instead of dvips
%\special{!/@scaleunit 1 def}
\parskip=10pt
\parindent=0pt
%
\title{Multiplatform application development with wxWindows}
\author{Julian Smart, Robert Roebling, Vadim Zeitlin, Robin Dunn, et al}
\date{August 13th 2000}
%
\makeindex
\begin{document}
\maketitle
\pagestyle{fancyplain}
\bibliographystyle{plain}
\setheader{{\it CONTENTS}}{}{}{}{}{{\it CONTENTS}}
\setfooter{\thepage}{}{}{}{}{\thepage}%
\pagenumbering{roman}
\tableofcontents

% Acknowledgements
\input chap\_acknowledge.tex
%
% Chapter 01: Introduction, advocacy, etc.
\input chap\_intro.tex
%
% Chapter 02: Installing wxWindows (and what tools to use)
\input chap\_install.tex
%
% Chapter 03: C++ and wxWindows. Summarises the sorts of constructs used/not used, plus wxString class, some conventions. Vadim suggests putting it in 1st chapter but I think it deserves a chapter of its own.
\input chap\_cpp.tex
%
% Chapter 04: Getting started: Hello World. Introduces app class, frames, menus, status bar, message box
\input chap\_getstart.tex
%
% Chapter 05: Basic event handling
\input chap\_basic\_events.tex
%
% Chapter 06: Frames and menubars. The components of a frame, menubars. 
\input chap\_frames.tex
%
% Chapter 07: Toolbars and status bars
\input chap\_toolbars.tex
%
% Chapter 08: Basic controls
\input chap\_basic\_controls.tex
%
% Chapter 09: Common dialogs
\input chap\_common\_dialogs.tex
%
% Chapter 10: Custom dialogs and resources (XML)
\input chap\_custom\_dialogs.tex
%
% Chapter 11: Drawing on device contexts
\input chap\_drawing.tex
%
% Chapter 12: Handling input (mouse, keyboard, joystick)
\input chap\_input.tex
%
% Chapter 14: Sizers
%
\input chap\_sizers.tex
%
% Chapter 15: Images and bitmaps
\input chap\_images.tex
%
% Chapter 16: Clipboard and drag and drop
\input chap\_clipboard\_dnd.tex
%
% Chapter 17: Advanced controls (list,tree,notebook,splitter,wxWizard,wxCalCtrl...)
\input chap\_advanced\_controls.tex
%
% Chapter 18: Document/view classes
\input chap\_docview.tex
%
% Chapter 19: Scrolling
\input chap\_scrolling.tex
%
% Chapter 20: MDI
\input chap\_mdi.tex
%
% Chapter 21: Printing
\input chap\_printing.tex
%
% Chapter 22: Providing help in your applications
\input chap\_help.tex
%
% Chapter 23: Strings and internationalization
\input chap\_strings.tex
%
\input chap\_data\_classes.tex
% Chapter 24: Collection and container classes
%
% Chapter 25: Memory management and debugging (including wxLog)
\input chap\_memory.tex
%
% Chapter 26: Run-time class information
\input chap\_runtime.tex
%
% Chapter 27: Advanced event handling (user-defined events, ...)
\input chap\_advanced\_events.tex
%
% Chapter 28: Communication classes, including wxSocket
\input chap\_comms.tex
%
% Chapter 29: Database classes
\input chap\_database.tex
%
% Chapter 30: File and stream classes
\input chap\_file\_stream.tex
%
% Chapter 31: Configuration classes
\input chap\_config.tex
%
% Chapter 32: Time, timers and idle processing
\input chap\_time.tex
%
% Chapter 33: Writing multithreading applications
\input chap\_multithreading.tex
%
% Chapter 34: Perfecting your UI (Adapting to system settings, accelerators, ...)
\input chap\_perfecting.tex
%
% Chapter 35: Platform-specific programming (metafiles, OLE automation, taskbar, ...)
\input chap\_platform.tex
%
% Chapter 36: Using wxHTML
\input chap\_wxhtml.tex
%
% Chapter 37: Using wxPython
\input chap\_wxpython.tex
%
% Chapter 38: wxBase?
\input chap\_wxbase.tex
%
% Appendix: Comparison with other toolkits: MFC, Qt etc.
\input chap\_comparison.tex
%
% Appendix: a compendium of external resources, libraries etc.
\input chap\_resources.tex

\bibliography{refs}
\addcontentsline{toc}{chapter}{Bibliography}
\setheader{{\it REFERENCES}}{}{}{}{}{{\it REFERENCES}}%
\setfooter{\thepage}{}{}{}{}{\thepage}%

\newpage
% Note: In RTF, the \printindex must come before the
% change of header/footer, since the \printindex inserts
% the RTF \sect command which divides one chapter from
% the next.
\rtfonly{\printindex
\addcontentsline{toc}{chapter}{Index}
\setheader{{\it INDEX}}{}{}{}{}{{\it INDEX}}%
\setfooter{\thepage}{}{}{}{}{\thepage}
}
% In Latex, it must be this way around (I think)
\latexonly{\addcontentsline{toc}{chapter}{Index}
\setheader{{\it INDEX}}{}{}{}{}{{\it INDEX}}%
\setfooter{\thepage}{}{}{}{}{\thepage}
\printindex
}

\end{document}
