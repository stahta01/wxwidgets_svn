% -----------------------------------------------------------------------------
% wxZlibInputStream
% -----------------------------------------------------------------------------
\section{\class{wxZlibInputStream}}\label{wxzlibinputstream}

This filter stream decompresses a stream that is in zlib or gzip format.
Note that reading the gzip format requires zlib version 1.2.0 greater.

The stream is not seekable, \helpref{SeekI()}{wxinputstreamseeki} returns
 {\it wxInvalidOffset}. Also \helpref{GetSize()}{wxstreambasegetsize} is
not supported, it always returns $0$.

\wxheading{Derived from}

\helpref{wxFilterInputStream}{wxfilterinputstream}

\wxheading{Include files}

<wx/zstream.h>

\wxheading{See also}

\helpref{wxInputStream}{wxinputstream}, 
 \helpref{wxZlibOutputStream}{wxzliboutputstream}.

\latexignore{\rtfignore{\wxheading{Members}}}

\membersection{wxZlibInputStream::wxZlibInputStream}

\func{}{wxZlibInputStream}{\param{wxInputStream\&}{ stream}, \param{int}{ flags = wxZLIB\_ZLIB | wxZLIB\_GZIP}}

The {\it flags} wxZLIB\_ZLIB and wxZLIB\_GZIP specify whether the input data
is in zlib or gzip format. If both are used, bitwise ored, then zlib will
autodetect the stream type, this is the default.
If {\it flags} is zero, then the data is assumed to be a raw deflate stream
without either zlib or gzip headers.

The following symbols can be use for the flags:

\begin{verbatim}
// Flags
enum {
    wxZLIB_NO_HEADER = 0,   // raw deflate stream, no header or checksum
    wxZLIB_ZLIB = 1,        // zlib header and checksum
    wxZLIB_GZIP = 2         // gzip header and checksum, requires zlib 1.2+
};
\end{verbatim}


% -----------------------------------------------------------------------------
% wxZlibOutputStream
% -----------------------------------------------------------------------------
\section{\class{wxZlibOutputStream}}\label{wxzliboutputstream}

This stream compresses all data written to it. The compressed output can be
in zlib or gzip format.
Note that writing the gzip format requires zlib version 1.2.0 greater.

The stream is not seekable, \helpref{SeekO()}{wxoutputstreamseeko} returns
 {\it wxInvalidOffset}.

\wxheading{Derived from}

\helpref{wxFilterOutputStream}{wxfilteroutputstream}

\wxheading{Include files}

<wx/zstream.h>

\wxheading{See also}

\helpref{wxOutputStream}{wxoutputstream},
 \helpref{wxZlibInputStream}{wxzlibinputstream}


\latexignore{\rtfignore{\wxheading{Members}}}

\membersection{wxZlibOutputStream::wxZlibOutputStream}

\func{}{wxZlibOutputStream}{\param{wxOutputStream\&}{ stream}, \param{int}{ level = -1}, \param{int}{ flags = wxZLIB\_ZLIB}}

Creates a new write-only compressed stream. {\it level} means level of 
compression. It is number between 0 and 9 (including these values) where
0 means no compression and 9 best but slowest compression. -1 is default
value (currently equivalent to 6).

The {\it flags} wxZLIB\_ZLIB and wxZLIB\_GZIP specify whether the output data
will be in zlib or gzip format. wxZLIB\_ZLIB is the default.
If {\it flags} is zero, then a raw deflate stream is output without either
zlib or gzip headers.

The following symbols can be use for the compression level and flags:

\begin{verbatim}
// Compression level
enum {
    wxZ_DEFAULT_COMPRESSION = -1,
    wxZ_NO_COMPRESSION = 0,
    wxZ_BEST_SPEED = 1,
    wxZ_BEST_COMPRESSION = 9
};

// Flags
enum {
    wxZLIB_NO_HEADER = 0,   // raw deflate stream, no header or checksum
    wxZLIB_ZLIB = 1,        // zlib header and checksum
    wxZLIB_GZIP = 2         // gzip header and checksum, requires zlib 1.2+
};
\end{verbatim}

