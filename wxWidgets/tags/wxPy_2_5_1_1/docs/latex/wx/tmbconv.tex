%%%%%%%%%%%%%%%%%%%%%%%%%%%%%%%%%%%%%%%%%%%%%%%%%%%%%%%%%%%%%%%%%%%%%%%%%%%%%
%% Name:        tmbconv.tex
%% Purpose:     Overview of the wxMBConv classes in wxWindows
%% Author:      Ove Kaaven
%% Modified by:
%% Created:     25.03.00
%% RCS-ID:      $Id$
%% Copyright:   (c) 2000 Ove Kaaven
%% Licence:     wxWindows license
%%%%%%%%%%%%%%%%%%%%%%%%%%%%%%%%%%%%%%%%%%%%%%%%%%%%%%%%%%%%%%%%%%%%%%%%%%%%%

\section{wxMBConv classes overview}\label{mbconvclasses}

Classes: \helpref{wxMBConv}{wxmbconv}, wxMBConvLibc, 
\helpref{wxMBConvUTF7}{wxmbconvutf7}, \helpref{wxMBConvUTF8}{wxmbconvutf8}, 
\helpref{wxCSConv}{wxcsconv},
\helpref{wxMBConvUTF16}{wxmbconvutf16}, \helpref{wxMBConvUTF32}{wxmbconvutf32}

The wxMBConv classes in wxWindows enables an Unicode-aware application to
easily convert between Unicode and the variety of 8-bit encoding systems still
in use.

\subsection{Background: The need for conversion}

As programs are becoming more and more globalized, and users exchange documents
across country boundaries as never before, applications increasingly need to
take into account all the different character sets in use around the world. It
is no longer enough to just depend on the default byte-sized character set that
computers have traditionally used.

A few years ago, a solution was proposed: the Unicode standard. Able to contain
the complete set of characters in use in one unified global coding system,
it would resolve the character set problems once and for all.

But it hasn't happened yet, and the migration towards Unicode has created new
challenges, resulting in "compatibility encodings" such as UTF-8. A large
number of systems out there still depends on the old 8-bit encodings, hampered
by the huge amounts of legacy code still widely deployed. Even sending
Unicode data from one Unicode-aware system to another may need encoding to an
8-bit multibyte encoding (UTF-7 or UTF-8 is typically used for this purpose), to
pass unhindered through any traditional transport channels.

\subsection{Background: The wxString class}

If you have compiled wxWindows in Unicode mode, the wxChar type will become
identical to wchar\_t rather than char, and a wxString stores wxChars. Hence,
all wxString manipulation in your application will then operate on Unicode
strings, and almost as easily as working with ordinary char strings (you
just need to remember to use the wxT() macro to encapsulate any string
literals).

But often, your environment doesn't want Unicode strings. You could be sending
data over a network, or processing a text file for some other application. You
need a way to quickly convert your easily-handled Unicode data to and from a
traditional 8-bit-encoding. And this is what the wxMBConv classes do.

\subsection{wxMBConv classes}

The base class for all these conversions is the wxMBConv class (which itself
implements standard libc locale conversion). Derived classes include
wxMBConvLibc, several different wxMBConvUTFxxx classes, and wxCSConv, which
implement different kinds of conversions. You can also derive your own class
for your own custom encoding and use it, should you need it. All you need to do
is override the MB2WC and WC2MB methods.

\subsection{wxMBConv objects}

Several of the wxWindows-provided wxMBConv classes have predefined instances
(wxConvLibc, wxConvFile, wxConvUTF7, wxConvUTF8, wxConvLocal). You can use
these predefined objects directly, or you can instantiate your own objects.

A variable, wxConvCurrent, points to the conversion object that the user
interface is supposed to use, in the case that the user interface is not
Unicode-based (like with GTK+ 1.2). By default, it points to wxConvLibc or
wxConvLocal, depending on which works best on the current platform.

\subsection{wxCSConv}

The wxCSConv class is special because when it is instantiated, you can tell it
which character set it should use, which makes it meaningful to keep many
instances of them around, each with a different character set (or you can
create a wxCSConv instance on the fly).

The predefined wxCSConv instance, wxConvLocal, is preset to use the
default user character set, but you should rarely need to use it directly,
it is better to go through wxConvCurrent.

\subsection{Converting strings}

Once you have chosen which object you want to use to convert your text,
here is how you would use them with wxString. These examples all assume
that you are using a Unicode build of wxWindows, although they will still
compile in a non-Unicode build (they just won't convert anything).

Example 1: Constructing a wxString from input in current encoding.

\begin{verbatim}
wxString str(input_data, *wxConvCurrent);
\end{verbatim}

Example 2: Input in UTF-8 encoding.

\begin{verbatim}
wxString str(input_data, wxConvUTF8);
\end{verbatim}

Example 3: Input in KOI8-R. Construction of wxCSConv instance on the fly.

\begin{verbatim}
wxString str(input_data, wxCSConv(wxT("koi8-r")));
\end{verbatim}

Example 4: Printing a wxString to stdout in UTF-8 encoding.

\begin{verbatim}
puts(str.mb_str(wxConvUTF8));
\end{verbatim}

Example 5: Printing a wxString to stdout in custom encoding.
Using preconstructed wxCSConv instance.

\begin{verbatim}
wxCSConv cust(user_encoding);
printf("Data: %s\n", (const char*) str.mb_str(cust));
\end{verbatim}

Note: Since mb\_str() returns a temporary wxCharBuffer to hold the result
of the conversion, you need to explicitly cast it to const char* if you use
it in a vararg context (like with printf).

\subsection{Converting buffers}

If you have specialized needs, or just don't want to use wxString, you
can also use the conversion methods of the conversion objects directly.
This can even be useful if you need to do conversion in a non-Unicode
build of wxWindows; converting a string from UTF-8 to the current
encoding should be possible by doing this:

\begin{verbatim}
wxString str(wxConvUTF8.cMB2WC(input_data), *wxConvCurrent);
\end{verbatim}

Here, cMB2WC of the UTF8 object returns a wxWCharBuffer containing a Unicode
string. The wxString constructor then converts it back to an 8-bit character
set using the passed conversion object, *wxConvCurrent. (In a Unicode build
of wxWindows, the constructor ignores the passed conversion object and
retains the Unicode data.)

This could also be done by first making a wxString of the original data:

\begin{verbatim}
wxString input_str(input_data);
wxString str(input_str.wc_str(wxConvUTF8), *wxConvCurrent);
\end{verbatim}

To print a wxChar buffer to a non-Unicode stdout:

\begin{verbatim}
printf("Data: %s\n", (const char*) wxConvCurrent->cWX2MB(unicode_data));
\end{verbatim}

If you need to do more complex processing on the converted data, you
may want to store the temporary buffer in a local variable:

\begin{verbatim}
const wxWX2MBbuf tmp_buf = wxConvCurrent->cWX2MB(unicode_data);
const char *tmp_str = (const char*) tmp_buf;
printf("Data: %s\n", tmp_str);
process_data(tmp_str);
\end{verbatim}

If a conversion had taken place in cWX2MB (i.e. in a Unicode build),
the buffer will be deallocated as soon as tmp\_buf goes out of scope.
(The macro wxWX2MBbuf reflects the correct return value of cWX2MB
(either char* or wxCharBuffer), except for the const.)

