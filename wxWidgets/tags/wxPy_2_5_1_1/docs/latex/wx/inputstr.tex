% -----------------------------------------------------------------------------
% wxInputStream
% -----------------------------------------------------------------------------
\section{\class{wxInputStream}}\label{wxinputstream}

wxInputStream is an abstract base class which may not be used directly.

\wxheading{Derived from}

\helpref{wxStreamBase}{wxstreambase}

\wxheading{Include files}

<wx/stream.h>

\latexignore{\rtfignore{\wxheading{Members}}}

% -----------
% ctor & dtor
% -----------
\membersection{wxInputStream::wxInputStream}

\func{}{wxInputStream}{\void}

Creates a dummy input stream.

\membersection{wxInputStream::\destruct{wxInputStream}}

\func{}{\destruct{wxInputStream}}{\void}

Destructor.

\membersection{wxInputStream::GetC}

\func{char}{GetC}{\void}

Returns the first character in the input queue and removes it.

\membersection{wxInputStream::Eof}\label{wxinputstreameof}

\constfunc{wxInputStream}{Eof}{\void}

Returns true if the end of stream has been reached.

\membersection{wxInputStream::LastRead}\label{wxinputstreamlastread}

\constfunc{size\_t}{LastRead}{\void}

Returns the last number of bytes read.

\membersection{wxInputStream::Peek}

\func{char}{Peek}{\void}

Returns the first character in the input queue without removing it.

\membersection{wxInputStream::Read}

\func{wxInputStream\&}{Read}{\param{void *}{buffer}, \param{size\_t}{ size}}

Reads the specified amount of bytes and stores the data in {\it buffer}.

\wxheading{Warning}

The buffer absolutely needs to have at least the specified size.

\wxheading{Return value}

This function returns a reference on the current object, so the user can test
any states of the stream right away.

\func{wxInputStream\&}{Read}{\param{wxOutputStream\&}{ stream\_out}}

Reads data from the input queue and stores it in the specified output stream.
The data is read until an error is raised by one of the two streams.

\wxheading{Return value}

This function returns a reference on the current object, so the user can test
any states of the stream right away.

\membersection{wxInputStream::SeekI}\label{wxinputstreamseeki}

\func{off\_t}{SeekI}{\param{off\_t}{ pos}, \param{wxSeekMode}{ mode = wxFromStart}}

Changes the stream current position.

\membersection{wxInputStream::TellI}

\constfunc{off\_t}{TellI}{\void}

Returns the current stream position.

\membersection{wxInputStream::Ungetch}\label{wxinputstreamungetch}

\func{size\_t}{Ungetch}{\param{const char*}{ buffer}, \param{size\_t}{ size}}

This function is only useful in {\it read} mode. It is the manager of the "Write-Back"
buffer. This buffer acts like a temporary buffer where data which has to be
read during the next read IO call are put. This is useful when you get a big
block of data which you didn't want to read: you can replace them at the top
of the input queue by this way.

Be very careful about this call in connection with calling SeekI() on the same
stream. Any call to SeekI() will invalidate any previous call to this method
(otherwise you could SeekI() to one position, "unread" a few bytes there, SeekI()
to another position and data would be either lost or corrupted).

\wxheading{Return value}

Returns the amount of bytes saved in the Write-Back buffer.

\func{bool}{Ungetch}{\param{char }{c}}

This function acts like the previous one except that it takes only one
character: it is sometimes shorter to use than the generic function.

