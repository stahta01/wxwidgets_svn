\section{wxHTML overview}\label{wxhtml}

This topic was written by Vaclav Slavik, the author of the wxHTML library.

The wxHTML library provides classes for parsing and displaying HTML.

It is not intended to be a high-end HTML browser. If you are looking for
something like that try \urlref{http://www.mozilla.org}{http://www.mozilla.org}.

wxHTML can be used as a generic rich text viewer - for example to display 
a nice About Box (like those of GNOME apps) or to display the result of
database searching. There is a \helpref{wxFileSystem}{wxfilesystem} 
class which allows you to use your own virtual file systems.

wxHtmlWindow supports tag handlers. This means that you can easily
extend wxHtml library with new, unsupported tags. Not only that,
you can even use your own application-specific tags!
See {\tt src/html/m\_*.cpp} files for details.

There is a generic wxHtmlParser class,
independent of wxHtmlWindow.

\input htmlstrt.tex
\input htmlprn.tex
\input htmlhlpf.tex
\input htmlfilt.tex
\input htmlcell.tex
\input htmlhand.tex
\input htmltags.tex

