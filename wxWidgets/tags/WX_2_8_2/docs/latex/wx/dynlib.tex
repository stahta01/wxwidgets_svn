%%%%%%%%%%%%%%%%%%%%%%%%%%%%%%%%%%%%%%%%%%%%%%%%%%%%%%%%%%%%%%%%%%%%%%%%%%%%%%%
%% Name:        dynlib.tex
%% Purpose:     wxDynamicLibrary and wxDynamicLibraryDetails documentation
%% Author:      Vadim Zeitlin
%% Modified by:
%% Created:     14.01.02 (extracted from dllload.tex)
%% RCS-ID:      $Id$
%% Copyright:   (c) Vadim Zeitlin
%% License:     wxWindows license
%%%%%%%%%%%%%%%%%%%%%%%%%%%%%%%%%%%%%%%%%%%%%%%%%%%%%%%%%%%%%%%%%%%%%%%%%%%%%%%

\section{\class{wxDynamicLibrary}}\label{wxdynamiclibrary}

wxDynamicLibrary is a class representing dynamically loadable library
(Windows DLL, shared library under Unix etc.). Just create an object of
this class to load a library and don't worry about unloading it -- it will be
done in the objects destructor automatically.

% deprecated now...
%
%\wxheading{See also}
%
%\helpref{wxDllLoader}{wxdllloader}

\wxheading{Derived from}

No base class.

\wxheading{Include files}

<wx/dynlib.h>

(only available if \texttt{wxUSE\_DYNLIB\_CLASS} is set to $1$)

\latexignore{\rtfignore{\wxheading{Members}}}

\membersection{wxDynamicLibrary::wxDynamicLibrary}\label{wxdynamiclibrarywxdynamiclibrary}

\func{}{wxDynamicLibrary}{\void}

\func{}{wxDynamicLibrary}{\param{const wxString\& }{name}, \param{int }{flags = wxDL\_DEFAULT}}

Constructor. Second form calls \helpref{Load}{wxdynamiclibraryload}.


\membersection{wxDynamicLibrary::CanonicalizeName}\label{wxdynamiclibrarycanonicalizename}

\func{static wxString}{CanonicalizeName}{\param{const wxString\& }{name}, \param{wxDynamicLibraryCategory}{ cat = wxDL\_LIBRARY}}

Returns the platform-specific full name for the library called \arg{name}. E.g.
it adds a {\tt ".dll"} extension under Windows and {\tt "lib"} prefix and 
{\tt ".so"}, {\tt ".sl"} or maybe {\tt ".dylib"} extension under Unix.

The possible values for \arg{cat} are:

\begin{twocollist}
    \twocolitem{wxDL\_LIBRARY}{normal library}
    \twocolitem{wxDL\_MODULE}{a loadable module or plugin}
\end{twocollist}

\wxheading{See also}

\helpref{CanonicalizePluginName}{wxdynamiclibrarycanonicalizepluginname}



\membersection{wxDynamicLibrary::CanonicalizePluginName}\label{wxdynamiclibrarycanonicalizepluginname}

\func{static wxString}{CanonicalizePluginName}{\param{const wxString\& }{name}, \param{wxPluginCategory}{ cat = wxDL\_PLUGIN\_GUI}}

This function does the same thing as 
\helpref{CanonicalizeName}{wxdynamiclibrarycanonicalizename} but for wxWidgets
plugins. The only difference is that compiler and version information are added
to the name to ensure that the plugin which is going to be loaded will be
compatible with the main program.

The possible values for \arg{cat} are:

\begin{twocollist}
    \twocolitem{wxDL\_PLUGIN\_GUI}{plugin which uses GUI classes (default)}
    \twocolitem{wxDL\_PLUGIN\_BASE}{plugin which only uses wxBase}
\end{twocollist}


\membersection{wxDynamicLibrary::Detach}\label{wxdynamiclibrarydetach}

\func{wxDllType}{Detach}{\void}

Detaches this object from its library handle, i.e. the object will not unload
the library any longer in its destructor but it is now the callers
responsibility to do this using \helpref{Unload}{wxdynamiclibraryunload}.


\membersection{wxDynamicLibrary::GetSymbol}\label{wxdynamiclibrarygetsymbol}

\constfunc{void *}{GetSymbol}{\param{const wxString\& }{name}}

Returns pointer to symbol {\it name} in the library or NULL if the library
contains no such symbol.

\wxheading{See also}

\helpref{wxDYNLIB\_FUNCTION}{wxdynlibfunction}


\membersection{wxDynamicLibrary::GetSymbolAorW}\label{wxdynamiclibrarygetsymbolaorw}

\constfunc{void *}{GetSymbolAorW}{\param{const wxString\& }{name}}

This function is available only under Windows as it is only useful when
dynamically loading symbols from standard Windows DLLs. Such functions have
either \texttt{'A'} (in ANSI build) or \texttt{'W'} (in Unicode, or wide
character build) suffix if they take string parameters. Using this function you
can use just the base name of the function and the correct suffix is appende
automatically depending on the current build. Otherwise, this method is
identical to \helpref{GetSymbol}{wxdynamiclibrarygetsymbol}.


\membersection{wxDynamicLibrary::GetProgramHandle}\label{wxdynamiclibrarygetprogramhandle}

\func{static wxDllType}{GetProgramHandle}{\void}

Return a valid handle for the main program itself or \texttt{NULL} if symbols
from the main program can't be loaded on this platform.


\membersection{wxDynamicLibrary::HasSymbol}\label{wxdynamiclibraryhassymbol}

\constfunc{bool}{HasSymbol}{\param{const wxString\& }{name}}

Returns \true if the symbol with the given \arg{name} is present in the dynamic
library, \false otherwise. Unlike \helpref{GetSymbol}{wxdynamiclibrarygetsymbol},
this function doesn't log an error message if the symbol is not found.

\newsince{2.5.4}


\membersection{wxDynamicLibrary::IsLoaded}\label{wxdynamiclibraryisloaded}

\constfunc{bool}{IsLoaded}{\void}

Returns \true if the library was successfully loaded, \false otherwise.


\membersection{wxDynamicLibrary::ListLoaded}\label{wxdynamiclibrarylistloaded}

\func{static wxDynamicLibraryDetailsArray}{ListLoaded}{\void}

This static method returns an \helpref{array}{wxarray} containing the details
of all modules loaded into the address space of the current project, the array
elements are object of \texttt{wxDynamicLibraryDetails} class. The array will
be empty if an error occurred.

This method is currently implemented only under Win32 and Linux and is useful
mostly for diagnostics purposes.


\membersection{wxDynamicLibrary::Load}\label{wxdynamiclibraryload}

\func{bool}{Load}{\param{const wxString\& }{name}, \param{int }{flags = wxDL\_DEFAULT}}

Loads DLL with the given \arg{name} into memory. The \arg{flags} argument can
be a combination of the following bits:

\begin{twocollist}
\twocolitem{wxDL\_LAZY}{equivalent of RTLD\_LAZY under Unix, ignored elsewhere}
\twocolitem{wxDL\_NOW}{equivalent of RTLD\_NOW under Unix, ignored elsewhere}
\twocolitem{wxDL\_GLOBAL}{equivalent of RTLD\_GLOBAL under Unix, ignored elsewhere}
\twocolitem{wxDL\_VERBATIM}{don't try to append the appropriate extension to
the library name (this is done by default).}
\twocolitem{wxDL\_DEFAULT}{default flags, same as wxDL\_NOW currently}
\end{twocollist}

Returns \true if the library was successfully loaded, \false otherwise.


\membersection{wxDynamicLibrary::Unload}\label{wxdynamiclibraryunload}

\func{void}{Unload}{\void}

\func{static void}{Unload}{\param{wxDllType }{handle}}

Unloads the library from memory. wxDynamicLibrary object automatically calls
this method from its destructor if it had been successfully loaded.

The second version is only used if you need to keep the library in memory
during a longer period of time than the scope of the wxDynamicLibrary object.
In this case you may call \helpref{Detach}{wxdynamiclibrarydetach} and store
the handle somewhere and call this static method later to unload it.

%%%%%%%%%%%%%%%%%%%%%%%%%%%%%%%%%%%%%%%%%%%%%%%%%%%%%%%%%%%%%%%%%%%%%%%%%%%%%%%

\section{\class{wxDynamicLibraryDetails}}\label{wxdynamiclibrarydetails}

This class is used for the objects returned by 
\helpref{wxDynamicLibrary::ListLoaded}{wxdynamiclibrarylistloaded} method and
contains the information about a single module loaded into the address space of
the current process. A module in this context may be either a dynamic library
or the main program itself.

\wxheading{Derived from}

No base class.

\wxheading{Include files}

<wx/dynlib.h>

(only available if \texttt{wxUSE\_DYNLIB\_CLASS} is set to $1$)

\latexignore{\rtfignore{\wxheading{Members}}}

\membersection{wxDynamicLibraryDetails::GetName}\label{wxdynamiclibrarygetname}

\constfunc{wxString}{GetName}{\void}

Returns the base name of this module, e.g. \texttt{kernel32.dll} or 
\texttt{libc-2.3.2.so}.


\membersection{wxDynamicLibraryDetails::GetPath}\label{wxdynamiclibrarygetpath}

\constfunc{wxString}{GetPath}{\void}

Returns the full path of this module if available, e.g. 
\texttt{c:$\backslash$windows$\backslash$system32$\backslash$kernel32.dll} or 
\texttt{/lib/libc-2.3.2.so}.


\membersection{wxDynamicLibraryDetails::GetAddress}\label{wxdynamiclibrarygetaddress}

\constfunc{bool}{GetAddress}{\param{void **}{addr}, \param{size\_t }{*len}}

Retrieves the load address and the size of this module.

\wxheading{Parameters}

\docparam{addr}{the pointer to the location to return load address in, may be
\texttt{NULL}}

\docparam{len}{pointer to the location to return the size of this module in
memory in, may be \texttt{NULL}}

\wxheading{Return value}

\true if the load address and module size were retrieved, \false if this
information is not available.


\membersection{wxDynamicLibraryDetails::GetVersion}\label{wxdynamiclibrarygetversion}

\constfunc{wxString}{GetVersion}{\void}

Returns the version of this module, e.g. \texttt{5.2.3790.0} or 
\texttt{2.3.2}. The returned string is empty if the version information is not
available.

