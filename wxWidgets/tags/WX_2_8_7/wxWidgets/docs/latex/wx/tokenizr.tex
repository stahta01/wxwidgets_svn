\section{\class{wxStringTokenizer}}\label{wxstringtokenizer}

wxStringTokenizer helps you to break a string up into a number of tokens. It
replaces the standard C function {\tt strtok()} and also extends it in a
number of ways.

To use this class, you should create a wxStringTokenizer object, give it the
string to tokenize and also the delimiters which separate tokens in the string
(by default, white space characters will be used).

Then \helpref{GetNextToken}{wxstringtokenizergetnexttoken} may be called
repeatedly until it \helpref{HasMoreTokens}{wxstringtokenizerhasmoretokens} 
returns \false.

For example:

\begin{verbatim}

wxStringTokenizer tkz(wxT("first:second:third:fourth"), wxT(":"));
while ( tkz.HasMoreTokens() )
{
    wxString token = tkz.GetNextToken();

    // process token here
}
\end{verbatim}

By default, wxStringTokenizer will behave in the same way as {\tt strtok()} if
the delimiters string only contains white space characters but, unlike the
standard function, it will return empty tokens if this is not the case. This
is helpful for parsing strictly formatted data where the number of fields is
fixed but some of them may be empty (i.e. {\tt TAB} or comma delimited text
files).

The behaviour is governed by the last 
\helpref{constructor}{wxstringtokenizerwxstringtokenizer}/\helpref{SetString}{wxstringtokenizersetstring} 
parameter {\tt mode} which may be one of the following:

\twocolwidtha{5cm}%
\begin{twocollist}\itemsep=0pt
\twocolitem{{\tt wxTOKEN\_DEFAULT}}{Default behaviour (as described above):
same as {\tt wxTOKEN\_STRTOK} if the delimiter string contains only
whitespaces, same as {\tt wxTOKEN\_RET\_EMPTY} otherwise}
\twocolitem{{\tt wxTOKEN\_RET\_EMPTY}}{In this mode, the empty tokens in the
middle of the string will be returned, i.e. {\tt "a::b:"} will be tokenized in
three tokens `a', `' and `b'. Notice that all trailing delimiters are ignored
in this mode, not just the last one, i.e. a string \texttt{"a::b::"} would
still result in the same set of tokens.}
\twocolitem{{\tt wxTOKEN\_RET\_EMPTY\_ALL}}{In this mode, empty trailing tokens
(including the one after the last delimiter character) will be returned as
well. The string \texttt{"a::b:"} will be tokenized in four tokens: the already
mentioned ones and another empty one as the last one and a string 
\texttt{"a::b::"} will have five tokens.}
\twocolitem{{\tt wxTOKEN\_RET\_DELIMS}}{In this mode, the delimiter character
after the end of the current token (there may be none if this is the last
token) is returned appended to the token. Otherwise, it is the same mode as 
\texttt{wxTOKEN\_RET\_EMPTY}. Notice that there is no mode like this one but
behaving like \texttt{wxTOKEN\_RET\_EMPTY\_ALL} instead of 
\texttt{wxTOKEN\_RET\_EMPTY}, use \texttt{wxTOKEN\_RET\_EMPTY\_ALL} and 
\helpref{GetLastDelimiter()}{wxstringtokenizergetlastdelimiter} to emulate it.}
\twocolitem{{\tt wxTOKEN\_STRTOK}}{In this mode the class behaves exactly like
the standard {\tt strtok()} function: the empty tokens are never returned.}
\end{twocollist}

\wxheading{Derived from}

\helpref{wxObject}{wxobject}

\wxheading{See also}

\helpref{wxStringTokenize}{wxstringtokenize}

\wxheading{Include files}

<wx/tokenzr.h>

\latexignore{\rtfignore{\wxheading{Members}}}


\membersection{wxStringTokenizer::wxStringTokenizer}\label{wxstringtokenizerwxstringtokenizer}

\func{}{wxStringTokenizer}{\void}

Default constructor. You must call 
\helpref{SetString}{wxstringtokenizersetstring} before calling any other
methods.

\func{}{wxStringTokenizer}{\param{const wxString\& }{str}, \param{const wxString\& }{delims = " $\backslash$t$\backslash$r$\backslash$n"}, \param{wxStringTokenizerMode }{mode = wxTOKEN\_DEFAULT}}

Constructor. Pass the string to tokenize, a string containing delimiters
and the mode specifying how the string should be tokenized.


\membersection{wxStringTokenizer::CountTokens}\label{wxstringtokenizercounttokens}

\constfunc{int}{CountTokens}{\void}

Returns the number of tokens remaining in the input string. The number of
tokens returned by this function is decremented each time 
\helpref{GetNextToken}{wxstringtokenizergetnexttoken} is called and when it
reaches $0$ \helpref{HasMoreTokens}{wxstringtokenizerhasmoretokens} returns
\false.


\membersection{wxStringTokenizer::HasMoreTokens}\label{wxstringtokenizerhasmoretokens}

\constfunc{bool}{HasMoreTokens}{\void}

Returns \true if the tokenizer has further tokens, \false if none are left.


\membersection{wxStringTokenizer::GetLastDelimiter}\label{wxstringtokenizergetlastdelimiter}

\func{wxChar}{GetLastDelimiter}{\void}

Returns the delimiter which ended scan for the last token returned by 
\helpref{GetNextToken()}{wxstringtokenizergetnexttoken} or \texttt{NUL} if
there had been no calls to this function yet or if it returned the trailing
empty token in \texttt{wxTOKEN\_RET\_EMPTY\_ALL} mode.

\newsince{2.7.0}


\membersection{wxStringTokenizer::GetNextToken}\label{wxstringtokenizergetnexttoken}

\constfunc{wxString}{GetNextToken}{\void}

Returns the next token or empty string if the end of string was reached.


\membersection{wxStringTokenizer::GetPosition}\label{wxstringtokenizergetposition}

\constfunc{size\_t}{GetPosition}{\void}

Returns the current position (i.e. one index after the last returned
token or 0 if GetNextToken() has never been called) in the original
string.


\membersection{wxStringTokenizer::GetString}\label{wxstringtokenizergetstring}

\constfunc{wxString}{GetString}{\void}

Returns the part of the starting string without all token already extracted.


\membersection{wxStringTokenizer::SetString}\label{wxstringtokenizersetstring}

\func{void}{SetString}{\param{const wxString\& }{to\_tokenize}, \param{const wxString\& }{delims = " $\backslash$t$\backslash$r$\backslash$n"}, \param{wxStringTokenizerMode }{mode = wxTOKEN\_DEFAULT}}

Initializes the tokenizer.

Pass the string to tokenize, a string containing delimiters,
and the mode specifying how the string should be tokenized.

