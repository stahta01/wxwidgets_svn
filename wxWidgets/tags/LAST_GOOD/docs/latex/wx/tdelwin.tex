\section{Window deletion overview}\label{windowdeletionoverview}

Classes: \helpref{wxCloseEvent}{wxcloseevent}, \helpref{wxWindow}{wxwindow}

Window deletion can be a confusing subject, so this overview is provided
to make it clear when and how you delete windows, or respond to user requests
to close windows.

\wxheading{What is the sequence of events in a window deletion?}

When the user clicks on the system close button or system close command,
in a frame or a dialog, wxWindows calls \helpref{wxWindow::Close}{wxwindowclose}.

This function then generates a \helpref{wxCloseEvent}{wxcloseevent} event which
can be handled by the application (by using an EVT\_CLOSE event table entry). It is the duty of the application to
define a suitable event handler, and decide whether or not to destroy the window.
If the application is for some reason forcing the application to close,
the window should always be destroyed, otherwise there is the option to
ignore the request, or maybe wait until the user has answered a question
before deciding whether it's safe to close.

The wxCloseEvent handler should only call \helpref{wxWindow::Destroy}{wxwindowdestroy} to
delete the window, and not use the {\bf delete} operator. This is because
for some window classes, wxWindows delays actual deletion of the window until all events have been processed,
since otherwise there is the danger that events will be sent to a non-existent window.

\wxheading{How can the application close a window itself?}

Your application can use the \helpref{wxWindow::Close}{wxwindowclose} event just as
the framework does. Pass a TRUE argument to this function to tell the event handler
that we definitely want to delete the frame.

If for some reason you don't wish to use the {\bf Close} function to delete a window, at least use
the {\bf Destroy} function so that wxWindows can decide when it's safe to delete the window.

\wxheading{What is the default behaviour?}

By default, the close event handlers for wxFrame and wxDialog
both call the old \helpref{wxWindow::OnClose}{wxwindowonclose} handler
for backward compatibility. So you can still use the old form if you wish.

In addition, the default close event handler for wxDialog simulates a Cancel command,
generating a wxID\_CANCEL event. Since the handler for this cancel event might
itself call {\bf Close}, there is a check for infinite looping.

Under Windows, wxDialog also defines a handler for \helpref{wxWindow::OnCharHook}{wxwindowoncharhook} that
generates a Cancel event if the Escape key has been pressed.

\wxheading{What should I do when the user calls up Exit from a menu?}

You can simply call \helpref{wxWindow::Close}{wxwindowclose} on the frame. This
will invoke your own close event handler which may destroy the frame.

You can do checking to see if your application can be safely exited at this point,
either from within your close event handler, or from within your exit menu command
handler. For example, you may wish to check that all files have been saved.
Give the user a chance to save and quit, to not save but quit anyway, or to cancel
the exit command altogether.

\wxheading{What should I do to upgrade my 1.xx OnClose to 2.0?}

In wxWindows 1.xx, the {\bf OnClose} function did not actually delete 'this', but signalled
to the calling function (either {\bf Close}, or the wxWindows framework) to delete
or not delete the window.

You can still use this function unchanged in 2.0, but it's worth upgrading to
the new method in case future versions of wxWindows does not support the old one.

To update your code, you should provide an event table entry in your frame or
dialog, using the EVT\_CLOSE macro. The event handler function might look like this:

{\small%
\begin{verbatim}
  void MyFrame::OnCloseWindow(wxCloseEvent& event)
  {
    // If the application forces the deletion,
    // obey without question.
    if (event.GetForce())
    {
      this->Destroy();
      return;
    }

    // Otherwise...
    if (MyDataHasBeenModified())
    {
      wxMessageDialog* dialog = new wxMessageDialog(this,
        "Save changed data?", "My app", wxYES_NO|wxCANCEL);

      int ans = dialog->ShowModal();
      dialog->Close(TRUE);

      switch (ans)
      {
        case wxID_YES:      // Save, then destroy, quitting app
          SaveMyData();
          this->Destroy();
          break;
        case wxID_NO:       // Don't save; just destroy, quitting app
          this->Destroy();
          break;
        case wxID_CANCEL:   // Do nothing - so don't quit app.
        default:
          break;
      }
    }
  }
\end{verbatim}
}%

\wxheading{How do I exit the application gracefully?}

A wxWindows application automatically exits when the top frame (returned
from \helpref{wxApp::OnInit}{wxapponinit}) is destroyed. This may be modified
in later versions to exit only when the {\it last} top-level frame is destroyed.

\wxheading{Do child windows get deleted automatically?}

Yes, child windows are deleted from within the parent destructor. This includes any children
that are themselves frames or dialogs, so you may wish to close these child frame or dialog windows
explicitly from within the parent close handler.

\wxheading{What about other kinds of window?}

So far we've been talking about `managed' windows, i.e. frames and dialogs. Windows
with parents, such as controls, don't have delayed destruction and don't usually have
close event handlers, though you can implement them if you wish. For consistency,
continue to use the \helpref{wxWindow::Destroy}{wxwindowdestroy} function instead
of the {\bf delete} operator when deleting these kinds of windows explicitly.

