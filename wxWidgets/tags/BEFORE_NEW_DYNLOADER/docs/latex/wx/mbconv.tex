%
% automatically generated by HelpGen from
% ../include/wx/strconv.h at 25/Mar/00 10:20:56
%

\section{\class{wxMBConv}}\label{wxmbconv}

This class is the base class of a hierarchy of classes capable of converting
text strings between multibyte (SBCS or DBCS) encodings and Unicode. It is itself
a wrapper around the standard libc mbstowcs() and wcstombs() routines, and has
one predefined instance, {\bf wxConvLibc}.

\wxheading{Derived from}

No base class

\wxheading{Include files}

<wx/strconv.h>

\wxheading{See also}

\helpref{wxCSConv}{wxcsconv}, 
\helpref{wxEncodingConverter}{wxencodingconverter}, 
\helpref{wxMBConv classes overview}{mbconvclasses}

\latexignore{\rtfignore{\wxheading{Members}}}


\membersection{wxMBConv::wxMBConv}\label{wxmbconvwxmbconv}

\func{}{wxMBConv}{\void}

Constructor.

\membersection{wxMBConv::MB2WC}\label{wxmbconvmb2wc}

\constfunc{virtual size\_t}{MB2WC}{\param{wchar\_t* }{buf}, \param{const char* }{psz}, \param{size\_t }{n}}

Converts from multibyte encoding to Unicode, using the libc routine mbstowcs()
(this is overridden by derived classes). Returns the size of the destination buffer.

\membersection{wxMBConv::WC2MB}\label{wxmbconvwc2mb}

\constfunc{virtual size\_t}{WC2MB}{\param{char* }{buf}, \param{const wchar\_t* }{psz}, \param{size\_t }{n}}

Converts from Unicode to multibyte encoding, using the libc routine wcstombs()
(this is overridden by derived classes). Returns the size of the destination buffer.

\membersection{wxMBConv::cMB2WC}\label{wxmbconvcmb2wc}

\constfunc{const wxWCharBuffer}{cMB2WC}{\param{const char* }{psz}}

Converts from multibyte encoding to Unicode by calling MB2WC,
allocating a temporary wxWCharBuffer to hold the result.

\membersection{wxMBConv::cWC2MB}\label{wxmbconvcwc2mb}

\constfunc{const wxCharBuffer}{cWC2MB}{\param{const wchar\_t* }{psz}}

Converts from Unicode to multibyte encoding by calling WC2MB,
allocating a temporary wxCharBuffer to hold the result.

\membersection{wxMBConv::cMB2WX}\label{wxmbconvcmb2wx}

\constfunc{const char*}{cMB2WX}{\param{const char* }{psz}}

\constfunc{const wxWCharBuffer}{cMB2WX}{\param{const char* }{psz}}

Converts from multibyte encoding to the current wxChar type
(which depends on whether wxUSE\_UNICODE is set to 1). If wxChar is char,
it returns the parameter unaltered. If wxChar is wchar\_t, it returns the
result in a wxWCharBuffer. The macro wxMB2WXbuf is defined as the correct
return type (without const).

\membersection{wxMBConv::cWX2MB}\label{wxmbconvcwx2mb}

\constfunc{const char*}{cWX2MB}{\param{const wxChar* }{psz}}

\constfunc{const wxCharBuffer}{cWX2MB}{\param{const wxChar* }{psz}}

Converts from the current wxChar type to multibyte encoding. If wxChar is char,
it returns the parameter unaltered. If wxChar is wchar\_t, it returns the
result in a wxCharBuffer. The macro wxWX2MBbuf is defined as the correct
return type (without const).

\membersection{wxMBConv::cWC2WX}\label{wxmbconvcwc2wx}

\constfunc{const wchar\_t*}{cWC2WX}{\param{const wchar\_t* }{psz}}

\constfunc{const wxCharBuffer}{cWC2WX}{\param{const wchar\_t* }{psz}}

Converts from Unicode to the current wxChar type. If wxChar is wchar\_t,
it returns the parameter unaltered. If wxChar is char, it returns the
result in a wxCharBuffer. The macro wxWC2WXbuf is defined as the correct
return type (without const).

\membersection{wxMBConv::cWX2WC}\label{wxmbconvcwx2wc}

\constfunc{const wchar\_t*}{cWX2WC}{\param{const wxChar* }{psz}}

\constfunc{const wxWCharBuffer}{cWX2WC}{\param{const wxChar* }{psz}}

Converts from the current wxChar type to Unicode. If wxChar is wchar\_t,
it returns the parameter unaltered. If wxChar is char, it returns the
result in a wxWCharBuffer. The macro wxWX2WCbuf is defined as the correct
return type (without const).

