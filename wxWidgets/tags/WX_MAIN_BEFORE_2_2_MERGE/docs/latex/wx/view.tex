\section{\class{wxView}}\label{wxview}

The view class can be used to model the viewing and editing component of
an application's file-based data. It is part of the document/view framework supported by wxWindows,
and cooperates with the \helpref{wxDocument}{wxdocument}, \helpref{wxDocTemplate}{wxdoctemplate} 
and \helpref{wxDocManager}{wxdocmanager} classes.

\wxheading{Derived from}

\helpref{wxEvtHandler}{wxevthandler}\\
\helpref{wxObject}{wxobject}

\wxheading{Include files}

<wx/docview.h>

\wxheading{See also}

\helpref{wxView overview}{wxviewoverview}, \helpref{wxDocument}{wxdocument}, \helpref{wxDocTemplate}{wxdoctemplate},\rtfsp
\helpref{wxDocManager}{wxdocmanager}

\latexignore{\rtfignore{\wxheading{Members}}}

\membersection{wxView::m\_viewDocument}

\member{wxDocument*}{m\_viewDocument}

The document associated with this view. There may be more than one view per
document, but there can never be more than one document for one view.

\membersection{wxView::m\_viewFrame}

\member{wxFrame*}{m\_viewFrame}

Frame associated with the view, if any.

\membersection{wxView::m\_viewTypeName}

\member{wxString}{m\_viewTypeName}

The view type name given to the wxDocTemplate constructor, copied to this
variable when the view is created. Not currently used by the framework.

\membersection{wxView::wxView}

\func{}{wxView}{\void}

Constructor. Define your own default constructor to initialize application-specific
data.

\membersection{wxView::\destruct{wxView}}

\func{}{\destruct{wxView}}{\void}

Destructor. Removes itself from the document's list of views.

\membersection{wxView::Activate}

\func{virtual void}{Activate}{\param{bool}{ activate}}

Call this from your view frame's OnActivate member to tell the framework which view is
currently active. If your windowing system doesn't call OnActivate, you may need to
call this function from OnMenuCommand or any place where you know the view must
be active, and the framework will need to get the current view.

The prepackaged view frame wxDocChildFrame calls wxView::Activate from its OnActivate member
and from its OnMenuCommand member.

This function calls wxView::OnActivateView.

\membersection{wxView::Close}

\func{virtual bool}{Close}{\param{bool}{ deleteWindow = TRUE}}

Closes the view by calling OnClose. If {\it deleteWindow} is TRUE, this function should
delete the window associated with the view.

\membersection{wxView::GetDocument}

\constfunc{wxDocument*}{GetDocument}{\void}

Gets a pointer to the document associated with the view.

\membersection{wxView::GetDocumentManager}

\constfunc{wxDocumentManager*}{GetDocumentManager}{\void}

Returns a pointer to the document manager instance associated with this view.

\membersection{wxView::GetFrame}

\func{wxFrame *}{GetFrame}{\void}

Gets the frame associated with the view (if any).

\membersection{wxView::GetViewName}

\constfunc{wxString}{GetViewName}{\void}

Gets the name associated with the view (passed to the wxDocTemplate constructor).
Not currently used by the framework.

\membersection{wxView::OnActivateView}

\func{virtual void}{OnActivateView}{\param{bool }{activate}, \param{wxView *}{activeView}, \param{wxView *}{deactiveView}}

Called when a view is activated by means of wxView::Activate. The default implementation does
nothing.

\membersection{wxView::OnChangeFilename}

\func{virtual void}{OnChangeFilename}{\void}

Called when the filename has changed. The default implementation constructs a
suitable title and sets the title of the view frame (if any).

\membersection{wxView::OnClose}

\func{virtual bool}{OnClose}{\param{bool}{ deleteWindow}}

Implements closing behaviour. The default implementation calls wxDocument::Close
to close the associated document. Does not delete the view. The application
may wish to do some cleaning up operations in this function, {\it if} a
call to wxDocument::Close succeeded. For example, if your application's
all share the same window, you need to disassociate the window from the view
and perhaps clear the window. If {\it deleteWindow} is TRUE, delete the
frame associated with the view.

\membersection{wxView::OnCreate}

\func{virtual bool}{OnCreate}{\param{wxDocument* }{doc}, \param{long}{ flags}}

Called just after view construction to give the view a chance to initialize
itself based on the passed document and flags (unused). By default, simply
returns TRUE. If the function returns FALSE, the view will be deleted.

The predefined document child frame, wxDocChildFrame, calls this function
automatically.

\membersection{wxView::OnCreatePrintout}

\func{virtual wxPrintout*}{OnCreatePrintout}{\void}

If the printing framework is enabled in the library, this function returns a
\rtfsp\helpref{wxPrintout}{wxprintout} object for the purposes of printing. It should create a new object
everytime it is called; the framework will delete objects it creates.

By default, this function returns an instance of wxDocPrintout, which prints
and previews one page by calling wxView::OnDraw.

Override to return an instance of a class other than wxDocPrintout.

\membersection{wxView::OnUpdate}

\func{virtual void}{OnUpdate}{\param{wxView* }{sender}, \param{wxObject* }{hint}}

Called when the view should be updated. {\it sender} is a pointer to the view
that sent the update request, or NULL if no single view requested the update (for instance,
when the document is opened). {\it hint} is as yet unused but may in future contain
application-specific information for making updating more efficient.

\membersection{wxView::SetDocument}

\func{void}{SetDocument}{\param{wxDocument* }{doc}}

Associates the given document with the view. Normally called by the
framework.

\membersection{wxView::SetFrame}

\func{void}{SetFrame}{\param{wxFrame* }{frame}}

Sets the frame associated with this view. The application should call this
if possible, to tell the view about the frame.

\membersection{wxView::SetViewName}

\func{void}{SetViewName}{\param{const wxString\& }{name}}

Sets the view type name. Should only be called by the framework.


