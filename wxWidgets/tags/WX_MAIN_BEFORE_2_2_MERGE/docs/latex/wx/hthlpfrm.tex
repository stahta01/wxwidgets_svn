%
% automatically generated by HelpGen from
% helpfrm.h at 24/Oct/99 18:03:10
%

\section{\class{wxHtmlHelpFrame}}\label{wxhtmlhelpframe}

This class is used by \helpref{wxHtmlHelpController}{wxhtmlhelpcontroller} 
to display help.
It is an internal class and should not be used directly - except for the case
when you're writing your own HTML help controller.

\wxheading{Derived from}

\helpref{wxFrame}{wxframe}

\wxheading{Include files}

<wx/html/helpfrm.h>

\latexignore{\rtfignore{\wxheading{Members}}}

\membersection{wxHtmlHelpFrame::wxHtmlHelpFrame}\label{wxhtmlhelpframewxhtmlhelpframe}

\func{}{wxHtmlHelpFrame}{\param{wxHtmlHelpData* }{data = NULL}}

\func{}{wxHtmlHelpFrame}{\param{wxWindow* }{parent}, \param{int }{wxWindowID}, \param{const wxString\& }{title = wxEmptyString}, \param{int }{style = wxHF\_DEFAULTSTYLE}, \param{wxHtmlHelpData* }{data = NULL}}

Constructor.

{\it style} is combination of these flags:

\begin{twocollist}\itemsep=0pt
\twocolitem{\windowstyle{wxHF\_TOOLBAR}}{Help frame has toolbar.}
\twocolitem{\windowstyle{wxHF\_CONTENTS}}{Help frame has contents panel.}
\twocolitem{\windowstyle{wxHF\_INDEX}}{Help frame has index panel.}
\twocolitem{\windowstyle{wxHF\_SEARCH}}{Help frame has search panel.}
\twocolitem{\windowstyle{wxHF\_BOOKMARKS}}{Help frame has bookmarks controls.}
\twocolitem{\windowstyle{wxHF\_OPENFILES}}{Allow user to open arbitrary HTML document.}
\twocolitem{\windowstyle{wxHF\_PRINT}}{Toolbar contains "print" button.}
\end{twocollist}

\membersection{wxHtmlHelpFrame::Create}\label{wxhtmlhelpframecreate}

\func{bool}{Create}{\param{wxWindow* }{parent}, \param{wxWindowID }{id}, \param{const wxString\& }{title = wxEmptyString}, \param{int }{style = wxHF\_DEFAULTSTYLE}}

Creates the frame.

{\it style} is combination of these flags:

\begin{twocollist}\itemsep=0pt
\twocolitem{\windowstyle{wxHF\_TOOLBAR}}{Help frame has toolbar.}
\twocolitem{\windowstyle{wxHF\_CONTENTS}}{Help frame has contents panel.}
\twocolitem{\windowstyle{wxHF\_INDEX}}{Help frame has index panel.}
\twocolitem{\windowstyle{wxHF\_SEARCH}}{Help frame has search panel.}
\end{twocollist}

\membersection{wxHtmlHelpFrame::CreateContents}\label{wxhtmlhelpframecreatecontents}

\func{void}{CreateContents}{\param{bool }{show\_progress = FALSE}}

Creates contents panel. (May take some time.)

\membersection{wxHtmlHelpFrame::CreateIndex}\label{wxhtmlhelpframecreateindex}

\func{void}{CreateIndex}{\param{bool }{show\_progress = FALSE}}

Creates index panel. (May take some time.)

\membersection{wxHtmlHelpFrame::CreateSearch}\label{wxhtmlhelpframecreatesearch}

\func{void}{CreateSearch}{\void}

Creates search panel.

\membersection{wxHtmlHelpFrame::Display}\label{wxhtmlhelpframedisplay}

\func{bool}{Display}{\param{const wxString\& }{x}}

\func{bool}{Display}{\param{const int }{id}}

Displays page x. If not found it will give the user the choice of
searching books.
Looking for the page runs in these steps:

\begin{enumerate}\itemsep=0pt
\item try to locate file named x (if x is for example "doc/howto.htm")
\item try to open starting page of book x
\item try to find x in contents (if x is for example "How To ...")
\item try to find x in index (if x is for example "How To ...")
\end{enumerate}

The second form takes numeric ID as the parameter.
(uses extension to MS format, <param name="ID" value=id>)

\pythonnote{The second form of this method is named DisplayId in
wxPython.}

\membersection{wxHtmlHelpFrame::DisplayContents}\label{wxhtmlhelpframedisplaycontents}

\func{bool}{DisplayContents}{\void}

Displays contents panel.

\membersection{wxHtmlHelpFrame::DisplayIndex}\label{wxhtmlhelpframedisplayindex}

\func{bool}{DisplayIndex}{\void}

Displays index panel.

\membersection{wxHtmlHelpFrame::GetData}\label{wxhtmlhelpframegetdata}

\func{wxHtmlHelpData*}{GetData}{\void}

Return wxHtmlHelpData object.

\membersection{wxHtmlHelpFrame::KeywordSearch}\label{wxhtmlhelpframekeywordsearch}

\func{bool}{KeywordSearch}{\param{const wxString\& }{keyword}}

Search for given keyword.

\membersection{wxHtmlHelpFrame::ReadCustomization}\label{wxhtmlhelpframereadcustomization}

\func{void}{ReadCustomization}{\param{wxConfigBase* }{cfg}, \param{const wxString\& }{path = wxEmptyString}}

Reads user's settings for this frame (see \helpref{wxHtmlHelpController::ReadCustomization}{wxhtmlhelpcontrollerreadcustomization})

\membersection{wxHtmlHelpFrame::RefreshLists}\label{wxhtmlhelpframerefreshlists}

\func{void}{RefreshLists}{\param{bool }{show\_progress = FALSE}}

Refresh all panels. This is necessary if a new book was added.

\membersection{wxHtmlHelpFrame::SetTitleFormat}\label{wxhtmlhelpframesettitleformat}

\func{void}{SetTitleFormat}{\param{const wxString\& }{format}}

Sets the frame's title format. {\it format} must contain exactly one "\%s"
(it will be replaced by the page title).

\membersection{wxHtmlHelpFrame::UseConfig}\label{wxhtmlhelpframeuseconfig}

\func{void}{UseConfig}{\param{wxConfigBase* }{config}, \param{const wxString\& }{rootpath = wxEmptyString}}

Add books to search choice panel.

\membersection{wxHtmlHelpFrame::WriteCustomization}\label{wxhtmlhelpframewritecustomization}

\func{void}{WriteCustomization}{\param{wxConfigBase* }{cfg}, \param{const wxString\& }{path = wxEmptyString}}

Saves user's settings for this frame (see \helpref{wxHtmlHelpController::WriteCustomization}{wxhtmlhelpcontrollerwritecustomization}).

\membersection{wxHtmlHelpFrame::AddToolbarButtons}\label{wxhtmlhelpframeaddtoolbarbuttons}

\func{virtual void}{AddToolbarButtons}{\param{wxToolBar *}{toolBar}, \param{int }{style}} 

You may override this virtual method to add more buttons into help frame's
toolbar. {\it toolBar} is a pointer to the toolbar and {\it style} is the style
flag as passed to Create method.

wxToolBar::Realize is called immediately after returning from this function.

See {\it samples/html/helpview} for an example.

