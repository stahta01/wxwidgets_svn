\documentstyle[a4,11pt,makeidx,verbatim,texhelp,fancyheadings,palatino]{report}
% JACS: doesn't make it through Tex2RTF, sorry. I will put it into texhelp.sty
% since Tex2RTF doesn't parse it.
% BTW, style MUST be report for it to work for Tex2RTF.
%KB:
%\addtolength{\textwidth}{1in}
%\addtolength{\oddsidemargin}{-0.5in}
%\addtolength{\topmargin}{-0.5in}
%\addtolength{\textheight}{1in}
%\sloppy
%end of my changes
\newcommand{\indexit}[1]{#1\index{#1}}%
\newcommand{\pipe}[0]{$\|$\ }%
\definecolour{black}{0}{0}{0}%
\definecolour{cyan}{0}{255}{255}%
\definecolour{green}{0}{255}{0}%
\definecolour{magenta}{255}{0}{255}%
\definecolour{red}{255}{0}{0}%
\definecolour{blue}{0}{0}{200}%
\definecolour{yellow}{255}{255}{0}%
\definecolour{white}{255}{255}{255}%
%
\input psbox.tex
% Remove this for processing with dvi2ps instead of dvips
%\special{!/@scaleunit 1 def}
\parskip=10pt
\parindent=0pt
\title{wxWindows 2.3: A portable C++ and Python GUI toolkit}
\winhelponly{\author{by Julian Smart et al
%\winhelponly{\\$$\image{1cm;0cm}{wxwin.wmf}$$}
}}
\winhelpignore{\author{Julian Smart, Robert Roebling, Vadim Zeitlin,
Robin Dunn, et al}
\date{September 6th 2001}
}
\makeindex
\begin{document}
\maketitle
\pagestyle{fancyplain}
\bibliographystyle{plain}
\setheader{{\it CONTENTS}}{}{}{}{}{{\it CONTENTS}}
\setfooter{\thepage}{}{}{}{}{\thepage}%
\pagenumbering{roman}
\tableofcontents

% A special table of contents for the WinHelp manual
\begin{comment}
\winhelponly{
\chapter{wxWindows class library reference}\label{winhelpcontents}

\center{
%\image{}{wxwin.wmf}
}%

\sethotspotcolour{off}%
\sethotspotunderline{on}%
\large{
\image{}{cpp.bmp} \helpref{Alphabetical class reference}{classref}

\image{}{shelves.bmp} \helpref{Classes by category}{classesbycat}

\image{}{book1.bmp} \helpref{Topic overviews}{overviews}

\image{}{hand1.bmp} \helpref{Guide to wxWindows}{wxwinchapters}
}
\sethotspotcolour{on}%
\sethotspotunderline{on}%

\chapter*{Overview of wxWindows}\label{wxwinchapters}

\helpref{Introduction}{introduction}\\
%\helpref{Resource guide}{resguide}\\
%\helpref{Comparison with other GUI models}{comparison}\\
%\helpref{Multi-platform development with wxWindows}{multiplat}\\
%\helpref{Tutorial}{tutorial}\\
\helpref{The wxWindows resource system}{resourceformats}\\
\helpref{Utilities}{utilities}\\
\helpref{Programming strategies}{strategies}\\
\helpref{Bugs and future directions}{bugs}\\
\helpref{References}{bibliography}
}
\end{comment}

\chapter{Copyright notice}
\setheader{{\it COPYRIGHT}}{}{}{}{}{{\it COPYRIGHT}}%
\setfooter{\thepage}{}{}{}{}{\thepage}%

\begin{center}
(c)  1999 Julian Smart, Robert Roebling, Vadim Zeitlin and other
members of the wxWindows team\\
Portions (c) 1996 Artificial Intelligence Applications Institute\\
\end{center}

Please also see the wxWindows license files (preamble.txt, lgpl.txt, gpl.txt, license.txt,
licendoc.txt) for conditions of software and documentation use.

\section*{wxWindows Library License, Version 3}

Copyright (C) 1998 Julian Smart, Robert Roebling, Vadim Zeitlin et al. 

Everyone is permitted to copy and distribute verbatim copies
of this license document, but changing it is not allowed. 

\begin{center}
WXWINDOWS LIBRARY LICENSE\\
TERMS AND CONDITIONS FOR COPYING, DISTRIBUTION AND MODIFICATION 
\end{center}

This library is free software; you can redistribute it and/or modify it 
under the terms of the GNU Library General Public License as published by 
the Free Software Foundation; either version 2 of the License, or (at 
your option) any later version. 

This library is distributed in the hope that it will be useful, but 
WITHOUT ANY WARRANTY; without even the implied warranty of 
MERCHANTABILITY or FITNESS FOR A PARTICULAR PURPOSE. See the GNU Library 
General Public License for more details. 

You should have received a copy of the GNU Library General Public License 
along with this software, usually in a file named COPYING.LIB. If not, 
write to the Free Software Foundation, Inc., 59 Temple Place, Suite 330,
Boston, MA~02111-1307 USA. 

EXCEPTION NOTICE 

1. As a special exception, the copyright holders of this library give 
permission for additional uses of the text contained in this release of 
the library as licensed under the wxWindows Library License, applying 
either version 3 of the License, or (at your option) any later version of 
the License as published by the copyright holders of version 3 of the 
License document. 

2. The exception is that you may create binary object code versions of any 
works using this library or based on this library, and use, copy, modify, 
link and distribute such binary object code files unrestricted under terms 
of your choice. 

3. If you copy code from files distributed under the terms of the GNU 
General Public License or the GNU Library General Public License into a 
copy of this library, as this license permits, the exception does not 
apply to the code that you add in this way. To avoid misleading anyone as 
to the status of such modified files, you must delete this exception 
notice from such code and/or adjust the licensing conditions notice
accordingly. 

4. If you write modifications of your own for this library, it is your 
choice whether to permit this exception to apply to your modifications. 
If you do not wish that, you must delete the exception notice from such 
code and/or adjust the licensing conditions notice accordingly. 

\section*{GNU Library General Public License, Version 2}

Copyright (C) 1991 Free Software Foundation, Inc.
675 Mass Ave, Cambridge, MA 02139, USA

Everyone is permitted to copy and distribute verbatim copies
of this license document, but changing it is not allowed.

[This is the first released version of the library GPL. It is
numbered 2 because it goes with version 2 of the ordinary GPL.]

\wxheading{Preamble}

The licenses for most software are designed to take away your
freedom to share and change it. By contrast, the GNU General Public
Licenses are intended to guarantee your freedom to share and change
free software -- to make sure the software is free for all its users.

This license, the Library General Public License, applies to some
specially designated Free Software Foundation software, and to any
other libraries whose authors decide to use it. You can use it for
your libraries, too.

When we speak of free software, we are referring to freedom, not
price. Our General Public Licenses are designed to make sure that you
have the freedom to distribute copies of free software (and charge for
this service if you wish), that you receive source code or can get it
if you want it, that you can change the software or use pieces of it
in new free programs; and that you know you can do these things.

To protect your rights, we need to make restrictions that forbid
anyone to deny you these rights or to ask you to surrender the rights.
These restrictions translate to certain responsibilities for you if
you distribute copies of the library, or if you modify it.

For example, if you distribute copies of the library, whether gratis
or for a fee, you must give the recipients all the rights that we gave
you. You must make sure that they, too, receive or can get the source
code. If you link a program with the library, you must provide
complete object files to the recipients so that they can relink them
with the library, after making changes to the library and recompiling
it. And you must show them these terms so they know their rights.

Our method of protecting your rights has two steps: (1) copyright
the library, and (2) offer you this license which gives you legal
permission to copy, distribute and/or modify the library.

Also, for each distributor's protection, we want to make certain
that everyone understands that there is no warranty for this free
library. If the library is modified by someone else and passed on, we
want its recipients to know that what they have is not the original
version, so that any problems introduced by others will not reflect on
the original authors' reputations.

Finally, any free program is threatened constantly by software
patents. We wish to avoid the danger that companies distributing free
software will individually obtain patent licenses, thus in effect
transforming the program into proprietary software. To prevent this,
we have made it clear that any patent must be licensed for everyone's
free use or not licensed at all.

Most GNU software, including some libraries, is covered by the ordinary
GNU General Public License, which was designed for utility programs. This
license, the GNU Library General Public License, applies to certain
designated libraries. This license is quite different from the ordinary
one; be sure to read it in full, and don't assume that anything in it is
the same as in the ordinary license.

The reason we have a separate public license for some libraries is that
they blur the distinction we usually make between modifying or adding to a
program and simply using it. Linking a program with a library, without
changing the library, is in some sense simply using the library, and is
analogous to running a utility program or application program. However, in
a textual and legal sense, the linked executable is a combined work, a
derivative of the original library, and the ordinary General Public License
treats it as such.

Because of this blurred distinction, using the ordinary General
Public License for libraries did not effectively promote software
sharing, because most developers did not use the libraries. We
concluded that weaker conditions might promote sharing better.

However, unrestricted linking of non-free programs would deprive the
users of those programs of all benefit from the free status of the
libraries themselves. This Library General Public License is intended to
permit developers of non-free programs to use free libraries, while
preserving your freedom as a user of such programs to change the free
libraries that are incorporated in them. (We have not seen how to achieve
this as regards changes in header files, but we have achieved it as regards
changes in the actual functions of the Library.) The hope is that this
will lead to faster development of free libraries.

The precise terms and conditions for copying, distribution and
modification follow. Pay close attention to the difference between a
"work based on the library" and a "work that uses the library". The
former contains code derived from the library, while the latter only
works together with the library.

Note that it is possible for a library to be covered by the ordinary
General Public License rather than by this special one.

\begin{center}
		GNU LIBRARY GENERAL PUBLIC LICENSE\\
 TERMS AND CONDITIONS FOR COPYING, DISTRIBUTION AND MODIFICATION
\end{center}

0. This License Agreement applies to any software library which
contains a notice placed by the copyright holder or other authorized
party saying it may be distributed under the terms of this Library
General Public License (also called "this License"). Each licensee is
addressed as "you".

A "library" means a collection of software functions and/or data
prepared so as to be conveniently linked with application programs
(which use some of those functions and data) to form executables.

The "Library", below, refers to any such software library or work
which has been distributed under these terms. A "work based on the
Library" means either the Library or any derivative work under
copyright law: that is to say, a work containing the Library or a
portion of it, either verbatim or with modifications and/or translated
straightforwardly into another language. (Hereinafter, translation is
included without limitation in the term "modification".)

"Source code" for a work means the preferred form of the work for
making modifications to it. For a library, complete source code means
all the source code for all modules it contains, plus any associated
interface definition files, plus the scripts used to control compilation
and installation of the library.

Activities other than copying, distribution and modification are not
covered by this License; they are outside its scope. The act of
running a program using the Library is not restricted, and output from
such a program is covered only if its contents constitute a work based
on the Library (independent of the use of the Library in a tool for
writing it). Whether that is true depends on what the Library does
and what the program that uses the Library does.

1. You may copy and distribute verbatim copies of the Library's
complete source code as you receive it, in any medium, provided that
you conspicuously and appropriately publish on each copy an
appropriate copyright notice and disclaimer of warranty; keep intact
all the notices that refer to this License and to the absence of any
warranty; and distribute a copy of this License along with the
Library.

You may charge a fee for the physical act of transferring a copy,
and you may at your option offer warranty protection in exchange for a
fee.

2. You may modify your copy or copies of the Library or any portion
of it, thus forming a work based on the Library, and copy and
distribute such modifications or work under the terms of Section 1
above, provided that you also meet all of these conditions:

\begin{indented}{1cm}
a) The modified work must itself be a software library.

b) You must cause the files modified to carry prominent notices
stating that you changed the files and the date of any change.

c) You must cause the whole of the work to be licensed at no
charge to all third parties under the terms of this License.

d) If a facility in the modified Library refers to a function or a
table of data to be supplied by an application program that uses
the facility, other than as an argument passed when the facility
is invoked, then you must make a good faith effort to ensure that,
in the event an application does not supply such function or
table, the facility still operates, and performs whatever part of
its purpose remains meaningful.

(For example, a function in a library to compute square roots has
a purpose that is entirely well-defined independent of the
application. Therefore, Subsection 2d requires that any
application-supplied function or table used by this function must
be optional: if the application does not supply it, the square
root function must still compute square roots.) 
\end{indented}

These requirements apply to the modified work as a whole. If
identifiable sections of that work are not derived from the Library,
and can be reasonably considered independent and separate works in
themselves, then this License, and its terms, do not apply to those
sections when you distribute them as separate works. But when you
distribute the same sections as part of a whole which is a work based
on the Library, the distribution of the whole must be on the terms of
this License, whose permissions for other licensees extend to the
entire whole, and thus to each and every part regardless of who wrote
it.

Thus, it is not the intent of this section to claim rights or contest
your rights to work written entirely by you; rather, the intent is to
exercise the right to control the distribution of derivative or
collective works based on the Library.

In addition, mere aggregation of another work not based on the Library
with the Library (or with a work based on the Library) on a volume of
a storage or distribution medium does not bring the other work under
the scope of this License.

3. You may opt to apply the terms of the ordinary GNU General Public
License instead of this License to a given copy of the Library. To do
this, you must alter all the notices that refer to this License, so
that they refer to the ordinary GNU General Public License, version 2,
instead of to this License. (If a newer version than version 2 of the
ordinary GNU General Public License has appeared, then you can specify
that version instead if you wish.) Do not make any other change in
these notices.

Once this change is made in a given copy, it is irreversible for
that copy, so the ordinary GNU General Public License applies to all
subsequent copies and derivative works made from that copy.

This option is useful when you wish to copy part of the code of
the Library into a program that is not a library.

4. You may copy and distribute the Library (or a portion or
derivative of it, under Section 2) in object code or executable form
under the terms of Sections 1 and 2 above provided that you accompany
it with the complete corresponding machine-readable source code, which
must be distributed under the terms of Sections 1 and 2 above on a
medium customarily used for software interchange.

If distribution of object code is made by offering access to copy
from a designated place, then offering equivalent access to copy the
source code from the same place satisfies the requirement to
distribute the source code, even though third parties are not
compelled to copy the source along with the object code.

5. A program that contains no derivative of any portion of the
Library, but is designed to work with the Library by being compiled or
linked with it, is called a "work that uses the Library". Such a
work, in isolation, is not a derivative work of the Library, and
therefore falls outside the scope of this License.

However, linking a "work that uses the Library" with the Library
creates an executable that is a derivative of the Library (because it
contains portions of the Library), rather than a "work that uses the
library". The executable is therefore covered by this License.
Section 6 states terms for distribution of such executables.

When a "work that uses the Library" uses material from a header file
that is part of the Library, the object code for the work may be a
derivative work of the Library even though the source code is not.
Whether this is true is especially significant if the work can be
linked without the Library, or if the work is itself a library. The
threshold for this to be true is not precisely defined by law.

If such an object file uses only numerical parameters, data
structure layouts and accessors, and small macros and small inline
functions (ten lines or less in length), then the use of the object
file is unrestricted, regardless of whether it is legally a derivative
work. (Executables containing this object code plus portions of the
Library will still fall under Section 6.)

Otherwise, if the work is a derivative of the Library, you may
distribute the object code for the work under the terms of Section 6.
Any executables containing that work also fall under Section 6,
whether or not they are linked directly with the Library itself.

6. As an exception to the Sections above, you may also compile or
link a "work that uses the Library" with the Library to produce a
work containing portions of the Library, and distribute that work
under terms of your choice, provided that the terms permit
modification of the work for the customer's own use and reverse
engineering for debugging such modifications.

You must give prominent notice with each copy of the work that the
Library is used in it and that the Library and its use are covered by
this License. You must supply a copy of this License. If the work
during execution displays copyright notices, you must include the
copyright notice for the Library among them, as well as a reference
directing the user to the copy of this License. Also, you must do one
of these things:

\begin{indented}{1cm}
a) Accompany the work with the complete corresponding
machine-readable source code for the Library including whatever
changes were used in the work (which must be distributed under
Sections 1 and 2 above); and, if the work is an executable linked
with the Library, with the complete machine-readable "work that
uses the Library", as object code and/or source code, so that the
user can modify the Library and then relink to produce a modified
executable containing the modified Library. (It is understood
that the user who changes the contents of definitions files in the
Library will not necessarily be able to recompile the application
to use the modified definitions.)

b) Accompany the work with a written offer, valid for at
least three years, to give the same user the materials
specified in Subsection 6a, above, for a charge no more
than the cost of performing this distribution.

c) If distribution of the work is made by offering access to copy
from a designated place, offer equivalent access to copy the above
specified materials from the same place.

d) Verify that the user has already received a copy of these
materials or that you have already sent this user a copy.
\end{indented}

For an executable, the required form of the "work that uses the
Library" must include any data and utility programs needed for
reproducing the executable from it. However, as a special exception,
the source code distributed need not include anything that is normally
distributed (in either source or binary form) with the major
components (compiler, kernel, and so on) of the operating system on
which the executable runs, unless that component itself accompanies
the executable.

It may happen that this requirement contradicts the license
restrictions of other proprietary libraries that do not normally
accompany the operating system. Such a contradiction means you cannot
use both them and the Library together in an executable that you
distribute.

7. You may place library facilities that are a work based on the
Library side-by-side in a single library together with other library
facilities not covered by this License, and distribute such a combined
library, provided that the separate distribution of the work based on
the Library and of the other library facilities is otherwise
permitted, and provided that you do these two things:

\begin{indented}{1cm}
a) Accompany the combined library with a copy of the same work
based on the Library, uncombined with any other library
facilities. This must be distributed under the terms of the
Sections above.

b) Give prominent notice with the combined library of the fact
that part of it is a work based on the Library, and explaining
where to find the accompanying uncombined form of the same work.
\end{indented}

8. You may not copy, modify, sublicense, link with, or distribute
the Library except as expressly provided under this License. Any
attempt otherwise to copy, modify, sublicense, link with, or
distribute the Library is void, and will automatically terminate your
rights under this License. However, parties who have received copies,
or rights, from you under this License will not have their licenses
terminated so long as such parties remain in full compliance.

9. You are not required to accept this License, since you have not
signed it. However, nothing else grants you permission to modify or
distribute the Library or its derivative works. These actions are
prohibited by law if you do not accept this License. Therefore, by
modifying or distributing the Library (or any work based on the
Library), you indicate your acceptance of this License to do so, and
all its terms and conditions for copying, distributing or modifying
the Library or works based on it.

10. Each time you redistribute the Library (or any work based on the
Library), the recipient automatically receives a license from the
original licensor to copy, distribute, link with or modify the Library
subject to these terms and conditions. You may not impose any further
restrictions on the recipients' exercise of the rights granted herein.
You are not responsible for enforcing compliance by third parties to
this License.

11. If, as a consequence of a court judgment or allegation of patent
infringement or for any other reason (not limited to patent issues),
conditions are imposed on you (whether by court order, agreement or
otherwise) that contradict the conditions of this License, they do not
excuse you from the conditions of this License. If you cannot
distribute so as to satisfy simultaneously your obligations under this
License and any other pertinent obligations, then as a consequence you
may not distribute the Library at all. For example, if a patent
license would not permit royalty-free redistribution of the Library by
all those who receive copies directly or indirectly through you, then
the only way you could satisfy both it and this License would be to
refrain entirely from distribution of the Library.

If any portion of this section is held invalid or unenforceable under any
particular circumstance, the balance of the section is intended to apply,
and the section as a whole is intended to apply in other circumstances.

It is not the purpose of this section to induce you to infringe any
patents or other property right claims or to contest validity of any
such claims; this section has the sole purpose of protecting the
integrity of the free software distribution system which is
implemented by public license practices. Many people have made
generous contributions to the wide range of software distributed
through that system in reliance on consistent application of that
system; it is up to the author/donor to decide if he or she is willing
to distribute software through any other system and a licensee cannot
impose that choice.

This section is intended to make thoroughly clear what is believed to
be a consequence of the rest of this License.

12. If the distribution and/or use of the Library is restricted in
certain countries either by patents or by copyrighted interfaces, the
original copyright holder who places the Library under this License may add
an explicit geographical distribution limitation excluding those countries,
so that distribution is permitted only in or among countries not thus
excluded. In such case, this License incorporates the limitation as if
written in the body of this License.

13. The Free Software Foundation may publish revised and/or new
versions of the Library General Public License from time to time.
Such new versions will be similar in spirit to the present version,
but may differ in detail to address new problems or concerns.

Each version is given a distinguishing version number. If the Library
specifies a version number of this License which applies to it and
"any later version", you have the option of following the terms and
conditions either of that version or of any later version published by
the Free Software Foundation. If the Library does not specify a
license version number, you may choose any version ever published by
the Free Software Foundation.

14. If you wish to incorporate parts of the Library into other free
programs whose distribution conditions are incompatible with these,
write to the author to ask for permission. For software which is
copyrighted by the Free Software Foundation, write to the Free
Software Foundation; we sometimes make exceptions for this. Our
decision will be guided by the two goals of preserving the free status
of all derivatives of our free software and of promoting the sharing
and reuse of software generally.

\begin{center}
NO WARRANTY
\end{center}

15. BECAUSE THE LIBRARY IS LICENSED FREE OF CHARGE, THERE IS NO
WARRANTY FOR THE LIBRARY, TO THE EXTENT PERMITTED BY APPLICABLE LAW.
EXCEPT WHEN OTHERWISE STATED IN WRITING THE COPYRIGHT HOLDERS AND/OR
OTHER PARTIES PROVIDE THE LIBRARY "AS IS" WITHOUT WARRANTY OF ANY
KIND, EITHER EXPRESSED OR IMPLIED, INCLUDING, BUT NOT LIMITED TO, THE
IMPLIED WARRANTIES OF MERCHANTABILITY AND FITNESS FOR A PARTICULAR
PURPOSE. THE ENTIRE RISK AS TO THE QUALITY AND PERFORMANCE OF THE
LIBRARY IS WITH YOU. SHOULD THE LIBRARY PROVE DEFECTIVE, YOU ASSUME
THE COST OF ALL NECESSARY SERVICING, REPAIR OR CORRECTION.

16. IN NO EVENT UNLESS REQUIRED BY APPLICABLE LAW OR AGREED TO IN
WRITING WILL ANY COPYRIGHT HOLDER, OR ANY OTHER PARTY WHO MAY MODIFY
AND/OR REDISTRIBUTE THE LIBRARY AS PERMITTED ABOVE, BE LIABLE TO YOU
FOR DAMAGES, INCLUDING ANY GENERAL, SPECIAL, INCIDENTAL OR
CONSEQUENTIAL DAMAGES ARISING OUT OF THE USE OR INABILITY TO USE THE
LIBRARY (INCLUDING BUT NOT LIMITED TO LOSS OF DATA OR DATA BEING
RENDERED INACCURATE OR LOSSES SUSTAINED BY YOU OR THIRD PARTIES OR A
FAILURE OF THE LIBRARY TO OPERATE WITH ANY OTHER SOFTWARE), EVEN IF
SUCH HOLDER OR OTHER PARTY HAS BEEN ADVISED OF THE POSSIBILITY OF SUCH
DAMAGES.


\begin{center}
END OF TERMS AND CONDITIONS
\end{center}

\wxheading{Appendix: How to Apply These Terms to Your New Libraries}

If you develop a new library, and you want it to be of the greatest
possible use to the public, we recommend making it free software that
everyone can redistribute and change. You can do so by permitting
redistribution under these terms (or, alternatively, under the terms of the
ordinary General Public License).

To apply these terms, attach the following notices to the library. It is
safest to attach them to the start of each source file to most effectively
convey the exclusion of warranty; and each file should have at least the
"copyright" line and a pointer to where the full notice is found.

\footnotesize{
\begin{verbatim}
<one line to give the library's name and a brief idea of what it does.>
Copyright (C) <year> <name of author>

This library is free software; you can redistribute it and/or
modify it under the terms of the GNU Library General Public
License as published by the Free Software Foundation; either
version 2 of the License, or (at your option) any later version.

This library is distributed in the hope that it will be useful,
but WITHOUT ANY WARRANTY; without even the implied warranty of
MERCHANTABILITY or FITNESS FOR A PARTICULAR PURPOSE. See the GNU
Library General Public License for more details.

You should have received a copy of the GNU Library General Public
License along with this library; if not, write to the Free
Software Foundation, Inc., 675 Mass Ave, Cambridge, MA 02139, USA.
\end{verbatim}
}

Also add information on how to contact you by electronic and paper mail.

You should also get your employer (if you work as a programmer) or your
school, if any, to sign a "copyright disclaimer" for the library, if
necessary. Here is a sample; alter the names:

\footnotesize{
\begin{verbatim}
Yoyodyne, Inc., hereby disclaims all copyright interest in the
library `Frob' (a library for tweaking knobs) written by James Random Hacker.

<signature of Ty Coon>, 1 April 1990
Ty Coon, President of Vice
\end{verbatim} 
}

That's all there is to it!

\chapter{Introduction}\label{introduction}
\pagenumbering{arabic}%
\setheader{{\it CHAPTER \thechapter}}{}{}{}{}{{\it CHAPTER \thechapter}}%
\setfooter{\thepage}{}{}{}{}{\thepage}%

\section{What is wxWindows?}

wxWindows is a C++ framework providing GUI (Graphical User
Interface) and other facilities on more than one platform.  Version 2.0 currently
supports MS Windows (16-bit, Windows 95 and Windows NT), Unix with GTK+, and Unix with Motif.
A Mac port is in an advanced state, an OS/2 port and a port to the MGL graphics library
have been started.

wxWindows was originally developed at the Artificial Intelligence
Applications Institute, University of Edinburgh, for internal use.
wxWindows has been released into the public domain in the hope
that others will also find it useful. Version 2.0 is written and
maintained by Julian Smart, Robert Roebling, Vadim Zeitlin and others.

This manual discusses wxWindows in the context of multi-platform
development.\helpignore{For more detail on the wxWindows version 2.0 API
(Application Programming Interface) please refer to the separate
wxWindows reference manual.}

Please note that in the following, ``MS Windows" often refers to all
platforms related to Microsoft Windows, including 16-bit and 32-bit
variants, unless otherwise stated. All trademarks are acknowledged.

\section{Why another cross-platform development tool?}

wxWindows was developed to provide a cheap and flexible way to maximize
investment in GUI application development.  While a number of commercial
class libraries already existed for cross-platform development,
none met all of the following criteria:

\begin{enumerate}\itemsep=0pt
\item low price;
\item source availability;
\item simplicity of programming;
\item support for a wide range of compilers.
\end{enumerate}

Since wxWindows was started, several other free or almost-free GUI frameworks have
emerged. However, none has the range of features, flexibility, documentation and the
well-established development team that wxWindows has.

As public domain software and a project open to everyone, wxWindows has
benefited from comments, ideas, bug fixes, enhancements and the sheer
enthusiasm of users, especially via the Internet. This gives wxWindows a
certain advantage over its commercial competitors (and over free libraries
without an independent development team), plus a robustness against
the transience of one individual or company. This openness and
availability of source code is especially important when the future of
thousands of lines of application code may depend upon the longevity of
the underlying class library.

Version 2.0 goes much further than previous versions in terms of generality and features,
allowing applications to be produced
that are often indistinguishable from those produced using single-platform
toolkits such as Motif and MFC.

The importance of using a platform-independent class library cannot be
overstated, since GUI application development is very time-consuming,
and sustained popularity of particular GUIs cannot be guaranteed.
Code can very quickly become obsolete if it addresses the wrong
platform or audience.  wxWindows helps to insulate the programmer from
these winds of change. Although wxWindows may not be suitable for
every application (such as an OLE-intensive program), it provides access to most of the functionality a
GUI program normally requires, plus some extras such as network programming
and PostScript output, and can of course be extended as needs dictate.  As a bonus, it provides
a cleaner programming interface than the native
APIs. Programmers may find it worthwhile to use wxWindows even if they
are developing on only one platform.

It is impossible to sum up the functionality of wxWindows in a few paragraphs, but
here are some of the benefits:

\begin{itemize}\itemsep=0pt
\item Low cost (free, in fact!)
\item You get the source.
\item Available on a variety of popular platforms.
\item Works with almost all popular C++ compilers.
\item Several example programs.
\item Over 900 pages of printable and on-line documentation.
\item Includes Tex2RTF, to allow you to produce your own documentation
in Windows Help, HTML and Word RTF formats.
\item Simple-to-use, object-oriented API.
\item Flexible event system.
\item Graphics calls include lines, rounded rectangles, splines, polylines, etc.
\item Constraint-based layout option.
\item Print/preview and document/view architectures.
\item Toolbar, notebook, tree control, advanced list control classes.
\item PostScript generation under Unix, normal MS Windows printing on the
PC.
\item MDI (Multiple Document Interface) support.
\item Can be used to create DLLs under Windows, dynamic libraries on Unix.
\item Common dialogs for file browsing, printing, colour selection, etc.
\item Under MS Windows, support for creating metafiles and copying
them to the clipboard.
\item An API for invoking help from applications.
\item Dialog Editor for building dialogs.
\item Network support via a family of socket and protocol classes.
\end{itemize}

\section{Changes from version 1.xx}\label{versionchanges}

These are a few of the major differences between versions 1.xx and 2.0.

Removals:

\begin{itemize}\itemsep=0pt
\item XView is no longer supported;
\item all controls (panel items) no longer have labels attached to them;
\item wxForm has been removed;
\item wxCanvasDC, wxPanelDC removed (replaced by wxClientDC, wxWindowDC, wxPaintDC which
can be used for any window);
\item wxMultiText, wxTextWindow, wxText removed and replaced by wxTextCtrl;
\item classes no longer divided into generic and platform-specific parts, for efficiency.
\end{itemize}

Additions and changes:

\begin{itemize}\itemsep=0pt
\item class hierarchy changed, and restrictions about subwindow nesting lifted;
\item header files reorganised to conform to normal C++ standards;
\item classes less dependent on each another, to reduce executable size;
\item wxString used instead of char* wherever possible;
\item the number of separate but mandatory utilities reduced;
\item the event system has been overhauled, with
virtual functions and callbacks being replaced with MFC-like event tables;
\item new controls, such as wxTreeCtrl, wxListCtrl, wxSpinButton;
\item less inconsistency about what events can be handled, so for example
mouse clicks or key presses on controls can now be intercepted;
\item the status bar is now a separate class, wxStatusBar, and is
implemented in generic wxWindows code;
\item some renaming of controls for greater consistency;
\item wxBitmap has the notion of bitmap handlers to allow for extension to new formats
without ifdefing;
\item new dialogs: wxPageSetupDialog, wxFileDialog, wxDirDialog,
wxMessageDialog, wxSingleChoiceDialog, wxTextEntryDialog;
\item GDI objects are reference-counted and are now passed to most functions
by reference, making memory management far easier;
\item wxSystemSettings class allows querying for various system-wide properties
such as dialog font, colours, user interface element sizes, and so on;
\item better platform look and feel conformance;
\item toolbar functionality now separated out into a family of classes with the
same API;
\item device contexts are no longer accessed using wxWindow::GetDC - they are created
temporarily with the window as an argument;
\item events from sliders and scrollbars can be handled more flexibly;
\item the handling of window close events has been changed in line with the new
event system;
\item the concept of {\it validator} has been added to allow much easier coding of
the relationship between controls and application data;
\item the documentation has been revised, with more cross-referencing.
\end{itemize}

Platform-specific changes:

\begin{itemize}\itemsep=0pt
\item The Windows header file (windows.h) is no longer included by wxWindows headers;
\item wx.dll supported under Visual C++;
\item the full range of Windows 95 window decorations are supported, such as modal frame
borders;
\item MDI classes brought out of wxFrame into separate classes, and made more flexible.
\end{itemize}

\section{wxWindows requirements}\label{requirements}

To make use of wxWindows, you currently need one or both of the
following setups.

(a) PC:

\begin{enumerate}\itemsep=0pt
\item A 486 or higher PC running MS Windows.
\item A Windows compiler: most are supported, but please see {\tt install.txt} for
details. Supported compilers include Microsoft Visual C++ 4.0 or higher, Borland C++, Cygwin,
Metrowerks CodeWarrior.
\item At least 60 MB of disk space.
\end{enumerate}

(b) Unix:

\begin{enumerate}\itemsep=0pt
\item Almost any C++ compiler, including GNU C++ (EGCS 1.1.1 or above).
\item Almost any Unix workstation, and one of: GTK+ 1.0, GTK+ 1.2, Motif 1.2 or higher, Lesstif.
\item At least 60 MB of disk space.
\end{enumerate}

\section{Availability and location of wxWindows}

wxWindows is currently available from the Artificial Intelligence
Applications Institute by anonymous FTP and World Wide Web:

\begin{verbatim}
  ftp://www.remstar.com/pub/wxwin
  http://www.wxwindows.org
\end{verbatim}

\section{Acknowledgments}

Thanks are due to AIAI for being willing to release the original version of
wxWindows into the public domain, and to our patient partners.

We would particularly like to thank the following for their contributions to wxWindows, and the many others who have been involved in
the project over the years. Apologies for any unintentional omissions from this list. 
 
Yiorgos Adamopoulos, Jamshid Afshar, Alejandro Aguilar-Sierra, AIAI, Patrick Albert, Karsten Ballueder, Michael Bedward, Kai Bendorf, Yura Bidus, Keith 
Gary Boyce, Chris Breeze, Pete Britton, Ian Brown, C. Buckley, Dmitri Chubraev, Robin Corbet, Cecil Coupe, Andrew Davison, Neil Dudman, Robin 
Dunn, Hermann Dunkel, Jos van Eijndhoven, Tom Felici, Thomas Fettig, Matthew Flatt, Pasquale Foggia, Josep Fortiana, Todd Fries, Dominic Gallagher, 
Wolfram Gloger, Norbert Grotz, Stefan Gunter, Bill Hale, Patrick Halke, Stefan Hammes, Guillaume Helle, Harco de Hilster, Cord Hockemeyer, Markus 
Holzem, Olaf Klein, Leif Jensen, Bart Jourquin, Guilhem Lavaux, Jan Lessner, Nicholas Liebmann, Torsten Liermann, Per Lindqvist, Thomas Runge, Tatu
M\"{a}nnist\"{o}, Scott Maxwell, Thomas Myers, Oliver Niedung, Hernan Otero, Ian Perrigo, Timothy Peters, Giordano Pezzoli, Harri Pasanen, Thomaso Paoletti, 
Garrett Potts, Marcel Rasche, Robert Roebling, Dino Scaringella, Jobst Schmalenbach, Arthur Seaton, Paul Shirley, Stein Somers, Petr Smilauer, Neil Smith, 
Kari Syst\"{a}, Arthur Tetzlaff-Deas, Jonathan Tonberg, Jyrki Tuomi, Janos Vegh, Andrea Venturoli, Vadim Zeitlin, Xiaokun Zhu, Edward Zimmermann.

`Graphplace', the basis for the wxGraphLayout library, is copyright Dr. Jos
T.J. van Eijndhoven of Eindhoven University of Technology. The code has
been used in wxGraphLayout with his permission.

We also acknowledge the author of XFIG, the excellent Unix drawing tool,
from the source of which we have borrowed some spline drawing code.
His copyright is included below.

{\it XFig2.1 is copyright (c) 1985 by Supoj Sutanthavibul. Permission to
use, copy, modify, distribute, and sell this software and its
documentation for any purpose is hereby granted without fee, provided
that the above copyright notice appear in all copies and that both that
copyright notice and this permission notice appear in supporting
documentation, and that the name of M.I.T. not be used in advertising or
publicity pertaining to distribution of the software without specific,
written prior permission.  M.I.T. makes no representations about the
suitability of this software for any purpose.  It is provided ``as is''
without express or implied warranty.}

\chapter{Multi-platform development with wxWindows}\label{multiplat}
\setheader{{\it CHAPTER \thechapter}}{}{}{}{}{{\it CHAPTER \thechapter}}%
\setfooter{\thepage}{}{}{}{}{\thepage}%

This chapter describes the practical details of using wxWindows. Please
see the file install.txt for up-to-date installation instructions, and
changes.txt for differences between versions.

\section{Include files}

The main include file is {\tt "wx/wx.h"}; this includes the most commonly
used modules of wxWindows.

To save on compilation time, include only those header files relevant to the
source file. If you are using precompiled headers, you should include
the following section before any other includes:

\begin{verbatim}
// For compilers that support precompilation, includes "wx.h".
#include <wx/wxprec.h>

#ifdef __BORLANDC__
#pragma hdrstop
#endif

#ifndef WX_PRECOMP
// Include your minimal set of headers here, or wx.h
#include <wx/wx.h>
#endif

... now your other include files ...
\end{verbatim}

The file {\tt "wx/wxprec.h"} includes {\tt "wx/wx.h"}. Although this incantation
may seem quirky, it is in fact the end result of a lot of experimentation,
and several Windows compilers to use precompilation (those tested are Microsoft Visual C++, Borland C++
and Watcom C++).

Borland precompilation is largely automatic. Visual C++ requires specification of {\tt "wx/wxprec.h"} as
the file to use for precompilation. Watcom C++ is automatic apart from the specification of
the .pch file. Watcom C++ is strange in requiring the precompiled header to be used only for
object files compiled in the same directory as that in which the precompiled header was created.
Therefore, the wxWindows Watcom C++ makefiles go through hoops deleting and recreating
a single precompiled header file for each module, thus preventing an accumulation of many
multi-megabyte .pch files.

\section{Libraries}

Please the wxGTK or wxMotif documentation for use of the Unix version of wxWindows.
Under Windows, use the library wx.lib for stand-alone Windows
applications, or wxdll.lib for creating DLLs.

\section{Configuration}

Options are configurable in the file
\rtfsp{\tt "wx/XXX/setup.h"} where XXX is the required platform (such as msw, motif, gtk, mac). Some settings are a matter
of taste, some help with platform-specific problems, and
others can be set to minimize the size of the library. Please see the setup.h file
and {\tt install.txt} files for details on configuration.

\section{Makefiles}

At the moment there is no attempt to make Unix makefiles and
PC makefiles compatible, i.e. one makefile is required for
each environment. wxGTK has its own configure system which can also
be used with wxMotif, although wxMotif has a simple makefile system of its own.

Sample makefiles for Unix (suffix .UNX), MS C++ (suffix .DOS and .NT), Borland
C++ (.BCC and .B32) and Symantec C++ (.SC) are included for the library, demos
and utilities.

The controlling makefile for wxWindows is in the platform-specific
directory, such as {\tt src/msw} or {\tt src/motif}.

Please see the platform-specific {\tt install.txt} file for further details.

\section{Windows-specific files}

wxWindows application compilation under MS Windows requires at least two
extra files, resource and module definition files.

\subsection{Resource file}\label{resources}

The least that must be defined in the Windows resource file (extension RC)
is the following statement:

\begin{verbatim}
rcinclude "wx/msw/wx.rc"
\end{verbatim}

which includes essential internal wxWindows definitions.  The resource script
may also contain references to icons, cursors, etc., for example:

\begin{verbatim}
wxicon icon wx.ico
\end{verbatim}

The icon can then be referenced by name when creating a frame icon. See
the MS Windows SDK documentation.

\normalbox{Note: include wx.rc {\it after} any ICON statements
so programs that search your executable for icons (such
as the Program Manager) find your application icon first.}

\subsection{Module definition file}

A module definition file (extension DEF) is required for 16-bit applications, and
looks like the following:

\begin{verbatim}
NAME         Hello
DESCRIPTION  'Hello'
EXETYPE      WINDOWS
STUB         'WINSTUB.EXE'
CODE         PRELOAD MOVEABLE DISCARDABLE
DATA         PRELOAD MOVEABLE MULTIPLE
HEAPSIZE     1024
STACKSIZE    8192
\end{verbatim}

The only lines which will usually have to be changed per application are
NAME and DESCRIPTION.

\section{Allocating and deleting wxWindows objects}

In general, classes derived from wxWindow must dynamically allocated
with {\it new} and deleted with {\it delete}. If you delete a window,
all of its children and descendants will be automatically deleted,
so you don't need to delete these descendants explicitly.

When deleting a frame or dialog, use {\bf Destroy} rather than {\bf delete} so
that the wxWindows delayed deletion can take effect. This waits until idle time
(when all messages have been processed) to actually delete the window, to avoid
problems associated with the GUI sending events to deleted windows.

Don't create a window on the stack, because this will interfere
with delayed deletion.

If you decide to allocate a C++ array of objects (such as wxBitmap) that may
be cleaned up by wxWindows, make sure you delete the array explicitly
before wxWindows has a chance to do so on exit, since calling {\it delete} on
array members will cause memory problems.

wxColour can be created statically: it is not automatically cleaned
up and is unlikely to be shared between other objects; it is lightweight
enough for copies to be made.

Beware of deleting objects such as a wxPen or wxBitmap if they are still in use.
Windows is particularly sensitive to this: so make sure you
make calls like wxDC::SetPen(wxNullPen) or wxDC::SelectObject(wxNullBitmap) before deleting
a drawing object that may be in use. Code that doesn't do this will probably work
fine on some platforms, and then fail under Windows.

\section{Architecture dependency}

A problem which sometimes arises from writing multi-platform programs is that
the basic C types are not defiend the same on all platforms. This holds true
for both the length in bits of the standard types (such as int and long) as 
well as their byte order, which might be little endian (typically
on Intel computers) or big endian (typically on some Unix workstations). wxWindows
defines types and macros that make it easy to write architecture independent
code. The types are:

wxInt32, wxInt16, wxInt8, wxUint32, wxUint16 = wxWord, wxUint8 = wxByte

where wxInt32 stands for a 32-bit signed integer type etc. You can also check
which architecture the program is compiled on using the wxBYTE\_ORDER define
which is either wxBIG\_ENDIAN or wxLITTLE\_ENDIAN (in the future maybe wxPDP\_ENDIAN
as well).

The macros handling bit-swapping with respect to the applications endianness
are described in the \helpref{Macros}{macros} section.

\section{Conditional compilation}

One of the purposes of wxWindows is to reduce the need for conditional
compilation in source code, which can be messy and confusing to follow.
However, sometimes it is necessary to incorporate platform-specific
features (such as metafile use under MS Windows). The symbols
listed in the file {\tt symbols.txt} may be used for this purpose,
along with any user-supplied ones.

\section{C++ issues}

The following documents some miscellaneous C++ issues.

\subsection{Templates}

wxWindows does not use templates since it is a notoriously unportable feature.

\subsection{RTTI}

wxWindows does not use run-time type information since wxWindows provides
its own run-time type information system, implemented using macros.

\subsection{Type of NULL}

Some compilers (e.g. the native IRIX cc) define NULL to be 0L so that
no conversion to pointers is allowed. Because of that, all these
occurences of NULL in the GTK port use an explicit conversion such 
as

{\small
\begin{verbatim}
  wxWindow *my_window = (wxWindow*) NULL;
\end{verbatim}
}

It is recommended to adhere to this in all code using wxWindows as
this make the code (a bit) more portable.

\subsection{Precompiled headers}

Some compilers, such as Borland C++ and Microsoft C++, support
precompiled headers. This can save a great deal of compiling time. The
recommended approach is to precompile {\tt "wx.h"}, using this
precompiled header for compiling both wxWindows itself and any
wxWindows applications. For Windows compilers, two dummy source files
are provided (one for normal applications and one for creating DLLs)
to allow initial creation of the precompiled header.

However, there are several downsides to using precompiled headers. One
is that to take advantage of the facility, you often need to include
more header files than would normally be the case. This means that
changing a header file will cause more recompilations (in the case of
wxWindows, everything needs to be recompiled since everything includes {\tt "wx.h"}!)

A related problem is that for compilers that don't have precompiled
headers, including a lot of header files slows down compilation
considerably. For this reason, you will find (in the common
X and Windows parts of the library) conditional
compilation that under Unix, includes a minimal set of headers;
and when using Visual C++, includes {\tt wx.h}. This should help provide
the optimal compilation for each compiler, although it is
biassed towards the precompiled headers facility available
in Microsoft C++.

\section{File handling}

When building an application which may be used under different
environments, one difficulty is coping with documents which may be
moved to different directories on other machines. Saving a file which
has pointers to full pathnames is going to be inherently unportable. One
approach is to store filenames on their own, with no directory
information.  The application searches through a number of locally
defined directories to find the file. To support this, the class {\bf
wxPathList} makes adding directories and searching for files easy, and
the global function {\bf wxFileNameFromPath} allows the application to
strip off the filename from the path if the filename must be stored.
This has undesirable ramifications for people who have documents of the
same name in different directories.

As regards the limitations of DOS 8+3 single-case filenames versus
unrestricted Unix filenames, the best solution is to use DOS filenames
for your application, and also for document filenames {\it if} the user
is likely to be switching platforms regularly. Obviously this latter
choice is up to the application user to decide.  Some programs (such as
YACC and LEX) generate filenames incompatible with DOS; the best
solution here is to have your Unix makefile rename the generated files
to something more compatible before transferring the source to DOS.
Transferring DOS files to Unix is no problem, of course, apart from EOL
conversion for which there should be a utility available (such as
dos2unix).

See also the File Functions section of the reference manual for
descriptions of miscellaneous file handling functions.

\begin{comment}
\chapter{Utilities supplied with wxWindows}\label{utilities}
\setheader{{\it CHAPTER \thechapter}}{}{}{}{}{{\it CHAPTER \thechapter}}%
\setfooter{\thepage}{}{}{}{}{\thepage}%

A number of `extras' are supplied with wxWindows, to complement
the GUI functionality in the main class library. These are found
below the utils directory and usually have their own source, library
and documentation directories. For other user-contributed packages,
see the directory ftp://www.remstar.com/pub/wxwin/contrib, which is
more easily accessed via the Contributions page on the Web site.

\section{wxHelp}\label{wxhelp}

wxHelp is a stand-alone program, written using wxWindows,
for displaying hypertext help. It is necessary since not all target
systems (notably X) supply an adequate
standard for on-line help. wxHelp is modelled on the MS Windows help
system, with contents, search and browse buttons, but does not reformat
text to suit the size of window, as WinHelp does, and its input files
are uncompressed ASCII with some embedded font commands and an .xlp
extension. Most wxWindows documentation (user manuals and class
references) is supplied in wxHelp format, and also in Windows Help
format. The wxWindows 2.0 project will presently use an HTML widget
in a new and improved wxHelp implementation, under X.

Note that an application can be programmed to use Windows Help under
MS Windows, and wxHelp under X. An alternative help viewer under X is
Mosaic, a World Wide Web viewer that uses HTML as its native hypertext
format. However, this is not currently integrated with wxWindows
applications.

wxHelp works in two modes---edit and end-user. In edit mode, an ASCII
file may be marked up with different fonts and colours, and divided into
sections. In end-user mode, no editing is possible, and the user browses
principally by clicking on highlighted blocks.

When an application invokes wxHelp, subsequent sections, blocks or
files may be viewed using the same instance of wxHelp since the two
programs are linked using wxWindows interprocess communication
facilities. When the application exits, that application's instance of
wxHelp may be made to exit also.  See the {\bf wxHelpControllerBase} entry in the
reference section for how an application controls wxHelp.

\section{Tex2RTF}\label{textortf}

Supplied with wxWindows is a utility called Tex2RTF for converting\rtfsp
\LaTeX\ manuals to the following formats:

\begin{description}
\item[wxHelp]
wxWindows help system format (XLP).
\item[Linear RTF]
Rich Text Format suitable for importing into a word processor.
\item[Windows Help RTF]
Rich Text Format suitable for compiling into a WinHelp HLP file with the
help compiler.
\item[HTML]
HTML is the native format for Mosaic, the main hypertext viewer for
the World Wide Web. Since it is freely available it is a good candidate
for being the wxWindows help system under X, as an alternative to wxHelp.
\end{description}

Tex2RTF is used for the wxWindows manuals and can be used independently
by authors wishing to create on-line and printed manuals from the same\rtfsp
\LaTeX\ source.  Please see the separate documentation for Tex2RTF.

\section{wxTreeLayout}

This is a simple class library for drawing trees in a reasonably pretty
fashion. It provides only minimal default drawing capabilities, since
the algorithm is meant to be used for implementing custom tree-based
tools.

Directed graphs may also be drawn using this library, if cycles are
removed before the nodes and arcs are passed to the algorithm.

Tree displays are used in many applications: directory browsers,
hypertext systems, class browsers, and decision trees are a few
possibilities.

See the separate manual and the directory utils/wxtree.

\section{wxGraphLayout}

The wxGraphLayout class is based on a tool called `graphplace' by Dr.
Jos T.J. van Eijndhoven of Eindhoven University of Technology. Given a
(possibly cyclic) directed graph, it does its best to lay out the nodes
in a sensible manner. There are many applications (such as diagramming)
where it is required to display a graph with no human intervention. Even
if manual repositioning is later required, this algorithm can make a good
first attempt.

See the separate manual and the directory utils/wxgraph. 

\section{Colours}\label{coloursampler}

A colour sampler for viewing colours and their names on each
platform.

%
\chapter{Tutorial}\label{tutorial}
\setheader{{\it CHAPTER \thechapter}}{}{}{}{}{{\it CHAPTER \thechapter}}%
\setfooter{\thepage}{}{}{}{}{\thepage}%

To be written.
\end{comment}

\chapter{Programming strategies}\label{strategies}
\setheader{{\it CHAPTER \thechapter}}{}{}{}{}{{\it CHAPTER \thechapter}}%
\setfooter{\thepage}{}{}{}{}{\thepage}%

This chapter is intended to list strategies that may be useful when
writing and debugging wxWindows programs. If you have any good tips,
please submit them for inclusion here.

\section{Strategies for reducing programming errors}

\subsection{Use ASSERT}

Although I haven't done this myself within wxWindows, it is good
practice to use ASSERT statements liberally, that check for conditions that
should or should not hold, and print out appropriate error messages.
These can be compiled out of a non-debugging version of wxWindows
and your application. Using ASSERT is an example of `defensive programming':
it can alert you to problems later on.

\subsection{Use wxString in preference to character arrays}

Using wxString can be much safer and more convenient than using char *.
Again, I haven't practised what I'm preaching, but I'm now trying to use
wxString wherever possible. You can reduce the possibility of memory
leaks substantially, and it's much more convenient to use the overloaded
operators than functions such as strcmp. wxString won't add a significant
overhead to your program; the overhead is compensated for by easier
manipulation (which means less code).

The same goes for other data types: use classes wherever possible.

\section{Strategies for portability}

\subsection{Use relative positioning or constraints}

Don't use absolute panel item positioning if you can avoid it. Different GUIs have
very differently sized panel items. Consider using the constraint system, although this
can be complex to program.

Alternatively, you could use alternative .wrc (wxWindows resource files) on different
platforms, with slightly different dimensions in each. Or space your panel items out
to avoid problems.

\subsection{Use wxWindows resource files}

Use .wrc (wxWindows resource files) where possible, because they can be easily changed
independently of source code. Bitmap resources can be set up to load different
kinds of bitmap depending on platform (see the section on resource files).

\section{Strategies for debugging}\label{debugstrategies}

\subsection{Positive thinking}

It's common to blow up the problem in one's imagination, so that it seems to threaten
weeks, months or even years of work. The problem you face may seem insurmountable:
but almost never is. Once you have been programming for some time, you will be able
to remember similar incidents that threw you into the depths of despair. But
remember, you always solved the problem, somehow!

Perseverance is often the key, even though a seemingly trivial problem
can take an apparently inordinate amount of time to solve. In the end,
you will probably wonder why you worried so much. That's not to say it
isn't painful at the time. Try not to worry -- there are many more important
things in life.

\subsection{Simplify the problem}

Reduce the code exhibiting the problem to the smallest program possible
that exhibits the problem. If it is not possible to reduce a large and
complex program to a very small program, then try to ensure your code
doesn't hide the problem (you may have attempted to minimize the problem
in some way: but now you want to expose it).

With luck, you can add a small amount of code that causes the program
to go from functioning to non-functioning state. This should give a clue
to the problem. In some cases though, such as memory leaks or wrong
deallocation, this can still give totally spurious results!

\subsection{Use a debugger}

This sounds like facetious advice, but it's surprising how often people
don't use a debugger. Often it's an overhead to install or learn how to
use a debugger, but it really is essential for anything but the most
trivial programs.

\subsection{Use logging functions}

There is a variety of logging functions that you can use in your program:
see \helpref{Logging functions}{logfunctions}.

Using tracing statements may be more convenient than using the debugger
in some circumstances (such as when your debugger doesn't support a lot
of debugging code, or you wish to print a bunch of variables).

\subsection{Use the wxWindows debugging facilities}

You can use wxDebugContext to check for
memory leaks and corrupt memory: in fact in debugging mode, wxWindows will
automatically check for memory leaks at the end of the program if wxWindows is suitably
configured. Depending on the operating system and compiler, more or less
specific information about the problem will be logged.

You should also use \helpref{debug macros}{debugmacros} as part of a `defensive programming' strategy,
scattering wxASSERTs liberally to test for problems in your code as early as possible. Forward thinking
will save a surprising amount of time in the long run.

See the \helpref{debugging overview}{debuggingoverview} for further information.

\subsection{Check Windows debug messages}

Under Windows, it's worth running your program with DBWIN running or
some other program that shows Windows-generated debug messages. It's
possible it'll show invalid handles being used. You may have fun seeing
what commercial programs cause these normally hidden errors! Microsoft
recommend using the debugging version of Windows, which shows up even
more problems. However, I doubt it's worth the hassle for most
applications. wxWindows is designed to minimize the possibility of such
errors, but they can still happen occasionally, slipping through unnoticed
because they are not severe enough to cause a crash.

\subsection{Genetic mutation}

If we had sophisticated genetic algorithm tools that could be applied
to programming, we could use them. Until then, a common -- if rather irrational --
technique is to just make arbitrary changes to the code until something
different happens. You may have an intuition why a change will make a difference;
otherwise, just try altering the order of code, comment lines out, anything
to get over an impasse. Obviously, this is usually a last resort.


\chapter{Alphabetical class reference}\label{classref}
\setheader{{\it CHAPTER \thechapter}}{}{}{}{}{{\it CHAPTER \thechapter}}%
\setfooter{\thepage}{}{}{}{}{\thepage}%

% NB: the files should be in alphabetic order of the classes documented in
%     them and *not* in alphabetic order of the file names!

\input accel.tex
\input accessible.tex
\input activevt.tex
\input activexcontainer.tex
\input activexevt.tex
\input app.tex
\input archive.tex
\input array.tex
\input arrstrng.tex
\input artprov.tex
\input autoobj.tex
\input bitmap.tex
\input bbutton.tex
\input bmpdatob.tex
\input bmphand.tex
\input boxsizer.tex
\input brush.tex
\input bufferdc.tex
\input strmbfrd.tex
\input busycurs.tex
\input busyinfo.tex
\input button.tex
\input calclevt.tex
\input calctrl.tex
\input caret.tex
\input checkbox.tex
\input checklst.tex
\input choice.tex
\input choicebk.tex
\input clasinfo.tex
\input ipcclint.tex
\input clientdc.tex
\input clientdat.tex
\input clipbrd.tex
\input closeevt.tex
\input cmdlpars.tex
\input colour.tex
\input colordlg.tex
\input combobox.tex
\input command.tex
\input cmdevent.tex
\input cmdproc.tex
\input conditn.tex
\input config.tex
\input ipcconn.tex
\input cntxtevt.tex
\input cshelp.tex
\input control.tex
\input ctrlsub.tex
\input countstr.tex
\input critsect.tex
\input crtslock.tex
\input csconv.tex
\input cursor.tex
\input custdobj.tex
\input dataform.tex
\input datistrm.tex
\input dataobj.tex
\input dobjcomp.tex
\input dobjsmpl.tex
\input datostrm.tex
\input dateevt.tex
\input datectrl.tex
\input datespan.tex
\input datetime.tex
\input db.tex
\input dc.tex
\input dcclipper.tex
\input ddeclint.tex
\input ddeconn.tex
\input ddeservr.tex
\input debugcxt.tex
\input debugrpt.tex
\input debugrptz.tex
\input debugrptpvw.tex
\input debugrptpvwstd.tex
\input debugrptup.tex
\input delgrend.tex
\input dialog.tex
\input dialevt.tex
\input dialup.tex
\input dir.tex
\input dirdlg.tex
\input dirtrav.tex
\input display.tex
\input dllload.tex
\input docchfrm.tex
\input docmanag.tex
\input docmdich.tex
\input docmdipr.tex
\input docprfrm.tex
\input doctempl.tex
\input document.tex
\input dragimag.tex
\input dropevt.tex
\input dropsrc.tex
\input droptrgt.tex
\input dynlib.tex
\input encconv.tex
\input eraseevt.tex
\input event.tex
\input evthand.tex
\input ffile.tex
\input ffilestr.tex
\input file.tex
\input fileconf.tex
\input fildatob.tex
\input filedlg.tex
\input fildrptg.tex
\input filehist.tex
\input fileistr.tex
\input filename.tex
\input fileostr.tex
\input filestrm.tex
\input filesys.tex
\input filesysh.tex
\input filetype.tex
\input fltinstr.tex
\input fltoutst.tex
\input fdrepdlg.tex
\input flexsizr.tex
\input focusevt.tex
\input font.tex
\input fontdlg.tex
\input fontenum.tex
\input fontlist.tex
\input fontmap.tex
\input frame.tex
\input fsfile.tex
\input ftp.tex
\input gauge.tex
\input gbposition.tex
\input gbsizeritem.tex
\input gbspan.tex
\input gdiobj.tex
\input dirctrl.tex
\input valgen.tex
\input glcanvas.tex
\input glcontext.tex
\input grid.tex
\input gridattr.tex
\input gridbagsizer.tex
\input gridedit.tex
\input gridevt.tex
\input gridrend.tex
\input gridtbl.tex
\input gridsizr.tex
\input hashmap.tex
\input hashset.tex
\input hash.tex
\input helpinst.tex
\input hprovcnt.tex
\input helpevt.tex
\input helpprov.tex
\input htcell.tex
\input htcolor.tex
\input htcontnr.tex
\input htdcrend.tex
\input hteasypr.tex
\input htfilter.tex
\input hthelpct.tex
\input hthlpdat.tex
\input hthlpfrm.tex
\input htlnkinf.tex
\input htmllbox.tex
\input htparser.tex
\input htprint.tex
\input httag.tex
\input httaghnd.tex
\input httagmod.tex
\input htwidget.tex
\input htwindow.tex
\input htwinprs.tex
\input htwinhnd.tex
\input http.tex
\input hvscroll.tex
\input icon.tex
\input iconbndl.tex
\input iconloc.tex
\input iconevt.tex
\input idleevt.tex
\input image.tex
\input imaglist.tex
\input ilayout.tex
\input indlgevt.tex
\input inputstr.tex
\input ipaddr.tex
\input ipvaddr.tex
\input joystick.tex
\input joyevent.tex
\input keyevent.tex
\input layalgor.tex
\input layout.tex
\input list.tex
\input listbook.tex
\input listbox.tex
\input listctrl.tex
\input listevt.tex
\input listitem.tex
\input listattr.tex
\input listview.tex
\input locale.tex
\input log.tex
\input longlong.tex
\input mask.tex
\input maxzevt.tex
\input mbconv.tex
\input mbcnvfil.tex
\input mbcnvut7.tex
\input mbcnvut8.tex
\input mbcnvutf.tex
\input mdi.tex
\input mediactrl.tex
\input mediaevt.tex
\input membuf.tex
\input memorydc.tex
\input fs_mem.tex
\input strmmem.tex
\input menu.tex
\input menuevt.tex
\input menuitem.tex
\input msgdlg.tex
\input metafile.tex
\input mimetype.tex
\input minifram.tex
\input mirrordc.tex
\input module.tex
\input mcaptevt.tex
\input mouseevt.tex
\input moveevt.tex
\input mltchdlg.tex
\input mutex.tex
\input mutexlck.tex
\input node.tex
\input notebook.tex
\input noteevt.tex
\input nbsizer.tex
\input notifevt.tex
\input object.tex
\input outptstr.tex
\input pagedlg.tex
\input paintdc.tex
\input paintevt.tex
\input palette.tex
\input panel.tex
\input passdlg.tex
\input pathlist.tex
\input pen.tex
\input point.tex
\input postscpt.tex
\input prevwin.tex
\input print.tex
\input process.tex
\input procevt.tex
\input progdlg.tex
\input propdlg.tex
\input protocol.tex
\input quantize.tex
\input qylayevt.tex
\input radiobox.tex
\input radiobut.tex
\input realpoin.tex
\input rect.tex
\input recguard.tex
\input regex.tex
\input region.tex
\input regkey.tex
\input renderer.tex
\input rendver.tex
\input sashevt.tex
\input sashlayw.tex
\input sashwin.tex
\input scpdarry.tex
\input scpdptr.tex
\input screendc.tex
\input scrolbar.tex
\input scrolwin.tex
\input scrolevt.tex
\input scrlwevt.tex
\input semaphor.tex
\input setcursorevt.tex
\input ipcservr.tex
\input hprovsmp.tex
\input sngchdlg.tex
\input snglinst.tex
\input size.tex
\input sizeevt.tex
\input sizer.tex
\input sizeritem.tex
\input slider.tex
\input sckaddr.tex
\input socket.tex
\input strmsock.tex
\input socksrv.tex
\input sound.tex
\input spinbutt.tex
\input spinctrl.tex
\input spinevt.tex
\input splash.tex
\input splitevt.tex
\input splitter.tex
\input splitpar.tex
\input stackframe.tex
\input stackwalker.tex
\input stdpaths.tex
\input statbmp.tex
\input statbox.tex
\input sbsizer.tex
\input statline.tex
\input stattext.tex
\input statusbr.tex
\input stdbtnsz.tex
\input stopwtch.tex
\input strmbase.tex
\input stream.tex
\input strtotxt.tex
\input wxstring.tex
\input strcldat.tex
\input sistream.tex
\input sostream.tex
\input tokenizr.tex
\input sysclevt.tex
\input sysopt.tex
\input settings.tex
\input taskbar.tex
\input tcpclint.tex
\input tcpconn.tex
\input tcpservr.tex
\input tempfile.tex
\input tempfilestrm.tex
\input text.tex
\input txtdatob.tex
\input txtdrptg.tex
\input textdlg.tex
\input textfile.tex
\input txtstrm.tex
\input valtext.tex
\input thread.tex
\input threadh.tex
\input timer.tex
\input timespan.tex
\input tipprov.tex
\input tipwin.tex
\input tglbtn.tex
\input toolbar.tex
\input toolbook.tex
\input tooltip.tex
\input tlw.tex
\input treebook.tex
\input treebookevent.tex
\input treectrl.tex
\input treeevt.tex
\input treedata.tex
\input upduievt.tex
\input uri.tex
\input url.tex
\input validatr.tex
\input variant.tex
\input view.tex
\input vlbox.tex
\input vscroll.tex
\input window.tex
\input wupdlock.tex
\input createevt.tex
\input windowdc.tex
\input destroyevt.tex
\input wnddisbl.tex
\input wizard.tex
\input wizevt.tex
\input wizpage.tex
\input xmldocument.tex
\input xmlnode.tex
\input xmlproperty.tex
\input xmlres.tex
\input xmlresh.tex
\input zipstrm.tex
\input strmzlib.tex

\chapter{Classes by category}\label{classesbycat}
\setheader{{\it CHAPTER \thechapter}}{}{}{}{}{{\it CHAPTER \thechapter}}%
\setfooter{\thepage}{}{}{}{}{\thepage}%

A classification of wxWindows classes by category.
\twocolwidtha{5cm}

{\large {\bf Managed windows}}

There are several types of window that are directly controlled by the
window manager (such as MS Windows, or the Motif Window Manager).
Frames may contain windows, and dialog boxes may directly contain controls.

\begin{twocollist}\itemsep=0pt
\twocolitem{\helpref{wxDialog}{wxdialog}}{Dialog box}
\twocolitem{\helpref{wxFrame}{wxframe}}{Normal frame}
\twocolitem{\helpref{wxMDIParentFrame}{wxmdiparentframe}}{MDI parent frame}
\twocolitem{\helpref{wxMDIChildFrame}{wxmdichildframe}}{MDI child frame}
\twocolitem{\helpref{wxMiniFrame}{wxminiframe}}{A frame with a small title bar}
\twocolitem{\helpref{wxTabbedDialog}{wxtabbeddialog}}{Tabbed dialog}
\end{twocollist}

See also {\bf Common dialogs}.

{\large {\bf Miscellaneous windows}}

The following are a variety of windows that are derived from wxWindow.

\begin{twocollist}\itemsep=0pt
\twocolitem{\helpref{wxGrid}{wxgrid}}{A grid (table) window}
\twocolitem{\helpref{wxPanel}{wxpanel}}{A window whose colour changes according to current user settings}
\twocolitem{\helpref{wxSashWindow}{wxsashwindow}}{Window with four optional sashes that can be dragged}
\twocolitem{\helpref{wxSashLayoutWindow}{wxsashlayoutwindow}}{Window that can be involved in an IDE-like layout arrangement}
\twocolitem{\helpref{wxScrolledWindow}{wxscrolledwindow}}{Window with automatically managed scrollbars}
\twocolitem{\helpref{wxSplitterWindow}{wxsplitterwindow}}{Window which can be split vertically or horizontally}
\twocolitem{\helpref{wxStatusBar}{wxstatusbar}}{Implements the status bar on a frame}
\twocolitem{\helpref{wxToolBar}{wxtoolbar}}{Toolbar class}
%\twocolitem{\helpref{wxTabbedPanel}{wxtabbedpanel}}{Tabbed panel (to be replaced with wxNotebook)}
\twocolitem{\helpref{wxNotebook}{wxnotebook}}{Notebook class}
\end{twocollist}

{\large {\bf Common dialogs}}

\overview{Overview}{commondialogsoverview}

Common dialogs are ready-made dialog classes which are frequently used
in an application.

\begin{twocollist}\itemsep=0pt
\twocolitem{\helpref{wxDialog}{wxdialog}}{Base class for common dialogs}
\twocolitem{\helpref{wxColourDialog}{wxcolourdialog}}{Colour chooser dialog}
\twocolitem{\helpref{wxDirDialog}{wxdirdialog}}{Directory selector dialog}
\twocolitem{\helpref{wxFileDialog}{wxfiledialog}}{File selector dialog}
\twocolitem{\helpref{wxMultipleChoiceDialog}{wxmultiplechoicedialog}}{Dialog to get one or more selections from a list}
\twocolitem{\helpref{wxSingleChoiceDialog}{wxsinglechoicedialog}}{Dialog to get a single selection from a list and return the string}
\twocolitem{\helpref{wxTextEntryDialog}{wxtextentrydialog}}{Dialog to get a single line of text from the user}
\twocolitem{\helpref{wxFontDialog}{wxfontdialog}}{Font chooser dialog}
\twocolitem{\helpref{wxPageSetupDialog}{wxpagesetupdialog}}{Standard page setup dialog}
\twocolitem{\helpref{wxPrintDialog}{wxprintdialog}}{Standard print dialog}
\twocolitem{\helpref{wxPageSetupDialog}{wxpagesetupdialog}}{Standard page setup dialog}
\twocolitem{\helpref{wxMessageDialog}{wxmessagedialog}}{Simple message box dialog}
\end{twocollist}

{\large {\bf Controls}}

Typically, these are small windows which provide interaction with the user. Controls
that are not static can have \helpref{validators}{wxvalidator} associated with them.

\begin{twocollist}\itemsep=0pt
\twocolitem{\helpref{wxControl}{wxcontrol}}{The base class for controls}
\twocolitem{\helpref{wxButton}{wxbutton}}{Push button control, displaying text}
\twocolitem{\helpref{wxBitmapButton}{wxbitmapbutton}}{Push button control, displaying a bitmap}
\twocolitem{\helpref{wxCheckBox}{wxcheckbox}}{Checkbox control}
\twocolitem{\helpref{wxCheckListBox}{wxchecklistbox}}{A listbox with a checkbox to the left of each item}
\twocolitem{\helpref{wxChoice}{wxchoice}}{Choice control (a combobox without the editable area)}
\twocolitem{\helpref{wxComboBox}{wxcombobox}}{A choice with an editable area}
\twocolitem{\helpref{wxGauge}{wxgauge}}{A control to represent a varying quantity, such as time remaining}
\twocolitem{\helpref{wxStaticBox}{wxstaticbox}}{A static, or group box for visually grouping related controls}
\twocolitem{\helpref{wxListBox}{wxlistbox}}{A list of strings for single or multiple selection}
\twocolitem{\helpref{wxListCtrl}{wxlistctrl}}{A control for displaying lists of strings and/or icons, plus a multicolumn report view}
\twocolitem{\helpref{wxTabCtrl}{wxtabctrl}}{Manages several tabs}
\twocolitem{\helpref{wxTextCtrl}{wxtextctrl}}{Single or multline text editing control}
\twocolitem{\helpref{wxTreeCtrl}{wxtreectrl}}{Tree (hierachy) control}
\twocolitem{\helpref{wxScrollBar}{wxscrollbar}}{Scrollbar control}
\twocolitem{\helpref{wxSpinButton}{wxspinbutton}}{A spin or `up-down' control}
\twocolitem{\helpref{wxStaticText}{wxstatictext}}{One or more lines of non-editable text}
\twocolitem{\helpref{wxStaticBitmap}{wxstaticbitmap}}{A control to display a bitmap}
\twocolitem{\helpref{wxRadioBox}{wxradiobox}}{A group of radio buttons}
\twocolitem{\helpref{wxRadioButton}{wxradiobutton}}{A round button to be used with others in a mutually exclusive way}
\twocolitem{\helpref{wxSlider}{wxslider}}{A slider that can be dragged by the user}
\end{twocollist}

{\large {\bf Menus}}

\begin{twocollist}\itemsep=0pt
\twocolitem{\helpref{wxMenu}{wxmenu}}{Displays a series of menu items for selection}
\twocolitem{\helpref{wxMenuBar}{wxmenubar}}{Contains a series of menus for use with a frame}
\twocolitem{\helpref{wxMenuItem}{wxmenuitem}}{Represents a single menu item}
\end{twocollist}

{\large {\bf Window layout}}

\overview{Overview}{constraintsoverview}

These are the classes relevant to automated window layout.

\begin{twocollist}\itemsep=0pt
\twocolitem{\helpref{wxIndividualLayoutConstraint}{wxindividuallayoutconstraint}}{Represents a single constraint dimension}
\twocolitem{\helpref{wxLayoutConstraints}{wxlayoutconstraints}}{Represents the constraints for a window class}
\end{twocollist}

{\large {\bf Device contexts}}

\overview{Overview}{dcoverview}

Device contexts are surfaces that may be drawn on, and provide an
abstraction that allows parameterisation of your drawing code
by passing different device contexts.

\begin{twocollist}\itemsep=0pt
\twocolitem{\helpref{wxClientDC}{wxclientdc}}{A device context to access the client area outside {\bf OnPaint} events}
\twocolitem{\helpref{wxPaintDC}{wxpaintdc}}{A device context to access the client area inside {\bf OnPaint} events}
\twocolitem{\helpref{wxWindowDC}{wxwindowdc}}{A device context to access the non-client area}
\twocolitem{\helpref{wxScreenDC}{wxscreendc}}{A device context to access the entire screen}
\twocolitem{\helpref{wxDC}{wxdc}}{The device context base class}
\twocolitem{\helpref{wxMemoryDC}{wxmemorydc}}{A device context for drawing into bitmaps}
\twocolitem{\helpref{wxMetafileDC}{wxmetafiledc}}{A device context for drawing into metafiles}
\twocolitem{\helpref{wxPostScriptDC}{wxpostscriptdc}}{A device context for drawing into PostScript files}
\twocolitem{\helpref{wxPrinterDC}{wxprinterdc}}{A device context for drawing to printers}
\end{twocollist}

{\large {\bf Graphics device interface}}

\overview{Bitmaps overview}{wxbitmapoverview}

These classes are related to drawing on device contexts and windows.

\begin{twocollist}\itemsep=0pt
\twocolitem{\helpref{wxColour}{wxcolour}}{Represents the red, blue and green elements of a colour}
\twocolitem{\helpref{wxBitmap}{wxbitmap}}{Represents a bitmap}
\twocolitem{\helpref{wxBrush}{wxbrush}}{Used for filling areas on a device context}
\twocolitem{\helpref{wxBrushList}{wxbrushlist}}{The list of previously-created brushes}
\twocolitem{\helpref{wxCursor}{wxcursor}}{A small, transparent bitmap representing the cursor}
\twocolitem{\helpref{wxFont}{wxfont}}{Represents fonts}
\twocolitem{\helpref{wxFontList}{wxfontlist}}{The list of previously-created fonts}
\twocolitem{\helpref{wxIcon}{wxicon}}{A small, transparent bitmap for assigning to frames and drawing on device contexts}
\twocolitem{\helpref{wxImage}{wximage}}{A platform-independent image class}
\twocolitem{\helpref{wxImageList}{wximagelist}}{A list of images, used with some controls}
\twocolitem{\helpref{wxMask}{wxmask}}{Represents a mask to be used with a bitmap for transparent drawing}
\twocolitem{\helpref{wxPen}{wxpen}}{Used for drawing lines on a device context}
\twocolitem{\helpref{wxPenList}{wxpenlist}}{The list of previously-created pens}
\twocolitem{\helpref{wxPalette}{wxpalette}}{Represents a table of indices into RGB values}
\twocolitem{\helpref{wxRegion}{wxregion}}{Represents a simple or complex region on a window or device context}
\end{twocollist}

{\large {\bf Events}}

\overview{Overview}{eventhandlingoverview}

An event object contains information about a specific event. Event handlers
(usually member functions) have a single, event argument.

\begin{twocollist}\itemsep=0pt
\twocolitem{\helpref{wxActivateEvent}{wxactivateevent}}{A window or application activation event}
\twocolitem{\helpref{wxCalculateLayoutEvent}{wxcalculatelayoutevent}}{Used to calculate window layout}
\twocolitem{\helpref{wxCloseEvent}{wxcloseevent}}{A close window or end session event}
\twocolitem{\helpref{wxCommandEvent}{wxcommandevent}}{An event from a variety of standard controls}
\twocolitem{\helpref{wxDropFilesEvent}{wxdropfilesevent}}{A drop files event}
\twocolitem{\helpref{wxEraseEvent}{wxeraseevent}}{An erase background event}
\twocolitem{\helpref{wxEvent}{wxevent}}{The event base class}
\twocolitem{\helpref{wxFocusEvent}{wxfocusevent}}{A window focus event}
\twocolitem{\helpref{wxKeyEvent}{wxkeyevent}}{A keypress event}
\twocolitem{\helpref{wxIdleEvent}{wxidleevent}}{An idle event}
\twocolitem{\helpref{wxInitDialogEvent}{wxinitdialogevent}}{A dialog initialisation event}
\twocolitem{\helpref{wxJoystickEvent}{wxjoystickevent}}{A joystick event}
\twocolitem{\helpref{wxListEvent}{wxlistevent}}{A list control event}
\twocolitem{\helpref{wxMenuEvent}{wxmenuevent}}{A menu event}
\twocolitem{\helpref{wxMouseEvent}{wxmouseevent}}{A mouse event}
\twocolitem{\helpref{wxMoveEvent}{wxmoveevent}}{A move event}
\twocolitem{\helpref{wxNotebookEvent}{wxnotebookevent}}{A notebook control event}
\twocolitem{\helpref{wxPaintEvent}{wxpaintevent}}{A paint event}
\twocolitem{\helpref{wxProcessEvent}{wxprocessevent}}{A process ending event}
\twocolitem{\helpref{wxQueryLayoutInfoEvent}{wxquerylayoutinfoevent}}{Used to query layout information}
\twocolitem{\helpref{wxSizeEvent}{wxsizeevent}}{A size event}
\twocolitem{\helpref{wxSocketEvent}{wxsocketevent}}{A socket event}
\twocolitem{\helpref{wxSysColourChangedEvent}{wxsyscolourchangedevent}}{A system colour change event}
\twocolitem{\helpref{wxTabEvent}{wxtabevent}}{A tab control event}
\twocolitem{\helpref{wxTreeEvent}{wxtreeevent}}{A tree control event}
\twocolitem{\helpref{wxUpdateUIEvent}{wxupdateuievent}}{A user interface update event}
\end{twocollist}

{\large {\bf Validators}}

\overview{Overview}{validatoroverview}

These are the window validators, used for filtering and validating
user input.

\begin{twocollist}\itemsep=0pt
\twocolitem{\helpref{wxValidator}{wxvalidator}}{Base validator class}
\twocolitem{\helpref{wxTextValidator}{wxtextvalidator}}{Text control validator class}
\twocolitem{\helpref{wxGenericValidator}{wxgenericvalidator}}{Generic control validator class}
\end{twocollist}

{\large {\bf Data structures}}

These are the data structure classes supported by wxWindows.

\begin{twocollist}\itemsep=0pt
\twocolitem{\helpref{wxExpr}{wxexpr}}{A class for flexible I/O}
\twocolitem{\helpref{wxExprDatabase}{wxexprdatabase}}{A class for flexible I/O}
\twocolitem{\helpref{wxDate}{wxdate}}{A class for date manipulation}
\twocolitem{\helpref{wxHashTable}{wxhashtable}}{A simple hash table implementation}
\twocolitem{\helpref{wxList}{wxlist}}{A simple linked list implementation}
\twocolitem{\helpref{wxNode}{wxnode}}{Represents a node in the wxList implementation}
\twocolitem{\helpref{wxObject}{wxobject}}{The root class for most wxWindows classes}
\twocolitem{\helpref{wxPathList}{wxpathlist}}{A class to help search multiple paths}
\twocolitem{\helpref{wxPoint}{wxpoint}}{Representation of a point}
\twocolitem{\helpref{wxRect}{wxrect}}{A class representing a rectangle}
\twocolitem{\helpref{wxRegion}{wxregion}}{A class representing a region}
\twocolitem{\helpref{wxString}{wxstring}}{A string class}
\twocolitem{\helpref{wxStringList}{wxstringlist}}{A class representing a list of strings}
\twocolitem{\helpref{wxRealPoint}{wxrealpoint}}{Representation of a point using floating point numbers}
\twocolitem{\helpref{wxSize}{wxsize}}{Representation of a size}
\twocolitem{\helpref{wxTime}{wxtime}}{A class for time manipulation}
\twocolitem{\helpref{wxVariant}{wxvariant}}{A class for storing arbitrary types
that may change at run-time}
\end{twocollist}

{\large {\bf Run-time class information system}}

\overview{Overview}{runtimeclassoverview}

wxWindows supports run-time manipulation of class information, and dynamic
creation of objects given class names.

\begin{twocollist}\itemsep=0pt
\twocolitem{\helpref{wxClassInfo}{wxclassinfo}}{Holds run-time class information}
\twocolitem{\helpref{wxObject}{wxobject}}{Root class for classes with run-time information}
\twocolitem{\helpref{Macros}{macros}}{Macros for manipulating run-time information}
\end{twocollist}

{\large {\bf Debugging features}}

\overview{Overview}{debuggingoverview}

wxWindows supports some aspects of debugging an application through
classes, functions and macros.

\begin{twocollist}\itemsep=0pt
\twocolitem{\helpref{wxDebugContext}{wxdebugcontext}}{Provides memory-checking facilities}
%\twocolitem{\helpref{wxDebugStreamBuf}{wxdebugstreambuf}}{A stream buffer writing to the debug stream}
\twocolitem{\helpref{wxLog}{wxlog}}{Logging facility}
\twocolitem{\helpref{Log functions}{logfunctions}}{Error and warning logging functions}
\twocolitem{\helpref{Debugging macros}{debugmacros}}{Debug macros for assertion and checking}
%\twocolitem{\helpref{wxTrace}{wxtrace}}{Tracing facility}
%\twocolitem{\helpref{wxTraceLevel}{wxtracelevel}}{Tracing facility with levels}
\twocolitem{\helpref{WXDEBUG\_NEW}{debugnew}}{Use this macro to give further debugging information}
%\twocolitem{\helpref{WXTRACE}{trace}}{Trace macro}
%\twocolitem{\helpref{WXTRACELEVEL}{tracelevel}}{Trace macro with levels}
\end{twocollist}

{\large {\bf Interprocess communication}}

\overview{Overview}{ipcoverview}

wxWindows provides a simple interprocess communications facilities
based on DDE.

\begin{twocollist}\itemsep=0pt
\twocolitem{\helpref{wxDDEClient}{wxddeclient}}{Represents a client}
\twocolitem{\helpref{wxDDEConnection}{wxddeconnection}}{Represents the connection between a client and a server}
\twocolitem{\helpref{wxDDEServer}{wxddeserver}}{Represents a server}
\twocolitem{\helpref{wxTCPClient}{wxtcpclient}}{Represents a client}
\twocolitem{\helpref{wxTCPConnection}{wxtcpconnection}}{Represents the connection between a client and a server}
\twocolitem{\helpref{wxTCPServer}{wxtcpserver}}{Represents a server}
\twocolitem{\helpref{wxSocketClient}{wxsocketclient}}{Represents a socket client}
\twocolitem{\helpref{wxSocketHandler}{wxsockethandler}}{Represents a socket handler}
\twocolitem{\helpref{wxSocketServer}{wxsocketserver}}{Represents a socket server}
\end{twocollist}

{\large {\bf Document/view framework}}

\overview{Overview}{docviewoverview}

wxWindows supports a document/view framework which provides
housekeeping for a document-centric application.

\begin{twocollist}\itemsep=0pt
\twocolitem{\helpref{wxDocument}{wxdocument}}{Represents a document}
\twocolitem{\helpref{wxView}{wxview}}{Represents a view}
\twocolitem{\helpref{wxDocTemplate}{wxdoctemplate}}{Manages the relationship between a document class and a veiw class}
\twocolitem{\helpref{wxDocManager}{wxdocmanager}}{Manages the documents and views in an application}
\twocolitem{\helpref{wxDocChildFrame}{wxdocchildframe}}{A child frame for showing a document view}
\twocolitem{\helpref{wxDocParentFrame}{wxdocparentframe}}{A parent frame to contain views}
%\twocolitem{\helpref{wxMDIDocChildFrame}{wxmdidocchildframe}}{An MDI child frame for showing a document view}
%\twocolitem{\helpref{wxMDIDocParentFrame}{wxmdidocparentframe}}{An MDI parent frame to contain views}
\end{twocollist}

{\large {\bf Printing framework}}

\overview{Overview}{printingoverview}

A printing and previewing framework is implemented to
make it relatively straighforward to provide document printing
facilities.

\begin{twocollist}\itemsep=0pt
\twocolitem{\helpref{wxPreviewFrame}{wxpreviewframe}}{Frame for displaying a print preview}
\twocolitem{\helpref{wxPreviewCanvas}{wxpreviewcanvas}}{Canvas for displaying a print preview}
\twocolitem{\helpref{wxPreviewControlBar}{wxpreviewcontrolbar}}{Standard control bar for a print preview}
\twocolitem{\helpref{wxPrintDialog}{wxprintdialog}}{Standard print dialog}
\twocolitem{\helpref{wxPageSetupDialog}{wxpagesetupdialog}}{Standard page setup dialog}
\twocolitem{\helpref{wxPrinter}{wxprinter}}{Class representing the printer}
\twocolitem{\helpref{wxPrinterDC}{wxprinterdc}}{Printer device context}
\twocolitem{\helpref{wxPrintout}{wxprintout}}{Class representing a particular printout}
\twocolitem{\helpref{wxPrintPreview}{wxprintpreview}}{Class representing a print preview}
\twocolitem{\helpref{wxPrintData}{wxprintdata}}{Represents information about the document being printed}
\twocolitem{\helpref{wxPrintDialogData}{wxprintdialogdata}}{Represents information about the print dialog}
\twocolitem{\helpref{wxPageSetupDialogData}{wxpagesetupdialogdata}}{Represents information about the page setup dialog}
\end{twocollist}

{\large {\bf Database classes}}

\overview{Database classes overview}{odbcoverview}

wxWindows provides two alternative sets of classes for accessing Microsoft's ODBC (Open Database Connectivity)
product. The new version by Remstar is documented in a separate manual.
The older classes are as follows:

\begin{twocollist}\itemsep=0pt
\twocolitem{\helpref{wxDatabase}{wxdatabase}}{Database class}
\twocolitem{\helpref{wxQueryCol}{wxquerycol}}{Class representing a column}
\twocolitem{\helpref{wxQueryField}{wxqueryfield}}{Class representing a field}
\twocolitem{\helpref{wxRecordSet}{wxrecordset}}{Class representing one or more record}
\end{twocollist}

{\large {\bf Drag and drop and clipboard classes}}

\overview{Drag and drop and clipboard overview}{wxdndoverview}

\begin{twocollist}\itemsep=0pt
\twocolitem{\helpref{wxDataObject}{wxdataobject}}{Data object class}
\twocolitem{\helpref{wxTextDataObject}{wxtextdataobject}}{Text data object class}
\twocolitem{\helpref{wxFileDataObject}{wxtextdataobject}}{File data object class}
\twocolitem{\helpref{wxBitmapDataObject}{wxbitmapdataobject}}{Bitmap data object class}
\twocolitem{\helpref{wxPrivateDataObject}{wxprivatedataobject}}{Private data object class}
\twocolitem{\helpref{wxClipboard}{wxclipboard}}{Clipboard class}
\twocolitem{\helpref{wxDropTarget}{wxdroptarget}}{Drop target class}
\twocolitem{\helpref{wxFileDropTarget}{wxfiledroptarget}}{File drop target class}
\twocolitem{\helpref{wxTextDropTarget}{wxtextdroptarget}}{Text drop target class}
\twocolitem{\helpref{wxDropSource}{wxdropsource}}{Drop source class}
\end{twocollist}

{\large {\bf File related classes}}

wxWindows has several small classes to work with disk files, see \helpref{file classes
overview}{wxfileoverview} for more details.

\begin{twocollist}\itemsep=0pt
\twocolitem{\helpref{wxFile}{wxfile}}{Low-level file input/output}
\twocolitem{\helpref{wxTempFile}{wxtempfile}}{Class to safely replace an existing file}
\twocolitem{\helpref{wxTextFile}{wxtextfile}}{Class for working with text files as with arrays of lines}
\end{twocollist}

{\large {\bf Stream classes}}

wxWindows has its own set of stream classes, as an alternative to often buggy standard stream
libraries, and to provide enhanced functionality.

\begin{twocollist}\itemsep=0pt
\twocolitem{\helpref{wxStreamBase}{wxstreambase}}{Stream base class}
\twocolitem{\helpref{wxStreamBuffer}{wxstreambuffer}}{Stream buffer class}
\twocolitem{\helpref{wxInputStream}{wxinputstream}}{Input stream class}
\twocolitem{\helpref{wxOutputStream}{wxoutputstream}}{Output stream class}
\twocolitem{\helpref{wxFilterInputStream}{wxfilterinputstream}}{Filtered input stream class}
\twocolitem{\helpref{wxFilterOutputStream}{wxfilteroutputstream}}{Filtered output stream class}
\twocolitem{\helpref{wxDataInputStream}{wxdatainputstream}}{Platform-independent data input stream class}
\twocolitem{\helpref{wxDataOutputStream}{wxdataoutputstream}}{Platform-independent data output stream class}
\twocolitem{\helpref{wxFileInputStream}{wxfileinputstream}}{File input stream class}
\twocolitem{\helpref{wxFileOutputStream}{wxfileoutputstream}}{File output stream class}
\twocolitem{\helpref{wxZlibInputStream}{wxzlibinputstream}}{Zlib (compression) input stream class}
\twocolitem{\helpref{wxZlibOutputStream}{wxzliboutputstream}}{Zlib (compression) output stream class}
\twocolitem{\helpref{wxSocketInputStream}{wxsocketinputstream}}{Socket input stream class}
\twocolitem{\helpref{wxSocketOutputStream}{wxsocketoutputstream}}{Socket output stream class}
\end{twocollist}

{\large {\bf Miscellaneous}}

\begin{twocollist}\itemsep=0pt
\twocolitem{\helpref{wxAcceleratorTable}{wxacceleratortable}}{Accelerator table}
\twocolitem{\helpref{wxApp}{wxapp}}{Application class}
\twocolitem{\helpref{wxAutomationObject}{wxautomationobject}}{OLE automation class}
\twocolitem{\helpref{wxConfig}{wxconfigbase}}{Classes for configuration reading/writing}
\twocolitem{\helpref{wxHelpController}{wxhelpcontroller}}{Family of classes for controlling help windows}
\twocolitem{\helpref{wxLayoutAlgorithm}{wxlayoutalgorithm}}{An alternative window layout facility}
\twocolitem{\helpref{wxProcess}{wxprocess}}{Process class}
\twocolitem{\helpref{wxTimer}{wxtimer}}{Timer class}
\twocolitem{\helpref{wxSystemSettings}{wxsystemsettings}}{System settings class}
\end{twocollist}


\chapter{Topic overviews}\label{overviews}
\setheader{{\it CHAPTER \thechapter}}{}{}{}{}{{\it CHAPTER \thechapter}}%
\setfooter{\thepage}{}{}{}{}{\thepage}%

This chapter contains a selection of topic overviews.


{\large {\bf Starting with wxWidgets}}

\helpref{Notes on using the reference}{referencenotes}\\
\helpref{Writing a wxWidgets application: a rough guide}{roughguide}\\
\helpref{wxWidgets Hello World sample}{helloworld}\\
\helpref{wxWidgets samples}{samples}\\
\helpref{Introduction to wxPython}{wxpython}

{\large {\bf Programming with wxWidgets}}

\helpref{Backward compatibility}{backwardcompatibility}\\
\helpref{Runtime class information (RTTI)}{runtimeclassoverview}\\
\helpref{Reference counting}{trefcount}\\
\helpref{Application class: wxApp}{wxappoverview}\\
\helpref{Unicode support in wxWidgets}{unicode}\\
\helpref{Conversion between Unicode and multibyte strings}{mbconvclasses}\\
\helpref{Internationalization}{internationalization}\\
\helpref{Writing non-English applications}{nonenglishoverview}\\
\helpref{Debugging overview}{debuggingoverview}\\
\helpref{Logging overview}{wxlogoverview}\\
\helpref{Event handling overview}{eventhandlingoverview}\\
\helpref{C++ exceptions overview}{exceptionsoverview}\\
\helpref{Window styles}{windowstyles}\\
\helpref{Window deletion overview}{windowdeletionoverview}\\
\helpref{Environment variables}{envvars}

{\large {\bf Overviews of non-GUI classes}}

\helpref{String class: wxString}{wxstringoverview}\\
\helpref{Buffer classes}{bufferclasses}\\
\helpref{Date and time classes}{wxdatetimeoverview}\\
\helpref{Container classes}{wxcontaineroverview}\\
\helpref{File classes and functions}{wxfileoverview}\\
\helpref{Stream classes}{wxstreamoverview}\\
\helpref{Multi-threaded applications}{wxthreadoverview}\\
\helpref{Working with program options: wxConfig}{wxconfigoverview}\\
\helpref{Virtual file system: wxFileSystem}{fs}\\
\helpref{Syntax of the built-in regular expression library}{wxresyn}\\
\helpref{Archive formats such as zip}{wxarc}\\
\helpref{Interprocess communication}{ipcoverview}\\
\helpref{ODBC Database classes}{odbcoverview}

{\large {\bf Drawing related classes}}

\helpref{Device contexts}{dcoverview}\\
\helpref{Bitmaps and icons}{wxbitmapoverview}\\
\helpref{Fonts}{wxfontoverview}\\
\helpref{Fonts encodings}{wxfontencodingoverview}\\
\helpref{Printing}{printingoverview}\\
\helpref{Printing under GTK+}{unixprinting}

{\large {\bf Overviews of GUI classes}}

\helpref{Laying out window elements with sizers}{sizeroverview}\\
\helpref{XML-based resource system}{xrcoverview}\\
\helpref{Window sizing}{windowsizingoverview}\\
\helpref{Scrolling}{scrollingoverview}\\
\helpref{Dialogs}{wxdialogoverview}\\
\helpref{Transferring and validating data}{validatoroverview}\\
\helpref{Data exchange: wxDataObject}{wxdataobjectoverview}\\
\helpref{Drag and drop}{wxdndoverview}\\
\helpref{Layout constraints}{constraintsoverview}

{\large {\bf Overviews of individual controls}}

\helpref{wxHTML}{wxhtml}\\
\helpref{wxRichTextCtrl}{wxrichtextctrloverview}\\
\helpref{wxAUI (advanced user interface)}{wxauioverview}\\
\helpref{Common dialogs}{commondialogsoverview}\\
\helpref{Toolbar}{wxtoolbaroverview}\\
\helpref{wxGrid}{gridoverview}\\
\helpref{wxTreeCtrl}{wxtreectrloverview}\\
\helpref{wxListCtrl}{wxlistctrloverview}\\
\helpref{wxSplitterWindow}{wxsplitterwindowoverview}\\
\helpref{wxImageList}{wximagelistoverview}\\
\helpref{wxBookCtrl}{wxbookctrloverview}\\
\helpref{wxTipProvider}{wxtipprovider}\\
\helpref{Document/view}{docviewoverview}


\section{wxHTML overview}\label{wxhtml}

This topic was written by Vaclav Slavik, the author of the wxHTML library.

The wxHTML library provides classes for parsing and displaying HTML.

It is not intended to be a high-end HTML browser. If you are looking for
something like that try \urlref{http://www.mozilla.org}{http://www.mozilla.org}.

wxHTML can be used as a generic rich text viewer - for example to display 
a nice About Box (like those of GNOME apps) or to display the result of
database searching. There is a \helpref{wxFileSystem}{wxfilesystem} 
class which allows you to use your own virtual file systems.

wxHtmlWindow supports tag handlers. This means that you can easily
extend wxHtml library with new, unsupported tags. Not only that,
you can even use your own application-specific tags!
See \verb$src/html/m_*.cpp$ files for details.

There is a generic wxHtmlParser class,
independent of wxHtmlWindow.

\input htmlstrt.tex
\input htmlprn.tex
\input htmlhlpf.tex
\input htmlfilt.tex
\input htmlcell.tex
\input htmlhand.tex
\input htmltags.tex


\chapter{Property sheet classes}\label{proplist}

\section{Introduction}\label{proplistintro}

The Property Sheet Classes help the programmer to specify complex dialogs and
their relationship with their associated data. By specifying data as a
wxPropertySheet containing wxProperty objects, the programmer can use
a range of available or custom wxPropertyView classes to allow the user to
edit this data. Classes derived from wxPropertyView act as mediators between the
wxPropertySheet and the actual window (and associated panel items).

For example, the wxPropertyListView is a kind of wxPropertyView which displays
data in a Visual Basic-style property list (see \helpref{the next section}{proplistappearance} for
screen shots). This is a listbox containing names and values, with
an edit control and other optional controls via which the user edits the selected
data item.

wxPropertyFormView is another kind of wxPropertyView which mediates between
the data and a panel or dialog box which has already been created. This makes it a contender for
the replacement of wxForm, since programmer-controlled layout is going to be much more
satisfactory. If automatic layout is desired, then wxPropertyListView could be used instead.

The main intention of this class library was to provide property {\it list} behaviour, but
it has been generalised as much as possible so that the concept of a property sheet and its viewers
can reduce programming effort in a range of user interface tasks.

For further details on the classes and how they are used, please see \helpref{Property classes overview}{proplistpropertyoverview}.

\subsection{The appearance and behaviour of a property list view}\label{proplistappearance}

The property list, as seen in an increasing number of development tools
such as Visual Basic and Delphi, is a convenient and compact method for
displaying and editing a number of items without the need for one
control per item, and without the need for designing a special form. The
controls are as follows:

\begin{itemize}\itemsep=0pt
\item A listbox showing the properties and their current values, which has double-click
properties dependent on the nature of the current property;
\item a text editing area at the top of the display, allowing the user to edit
the currently selected property if appropriate;
\item `confirm' and `cancel' buttons to confirm or cancel an edit (for the property, not the
whole sheet);
\item an optional list that appears when the user can make a choice from several known possible values;
\item a small Edit button to invoke `detailed editing' (perhaps showing or hiding the above value list, or
maybe invoking a common dialog);
\item optional OK/Close, Cancel and Help buttons for the whole dialog.
\end{itemize}

The concept of `detailed editing' versus quick editing gives the user a choice
of editing mode, so novice and expert behaviour can be catered for, or the user can just
use what he feels comfortable with.

Behaviour alters depending on the kind of property being edited. For example, a boolean value has
the following behaviour:

\begin{itemize}\itemsep=0pt
\item Double-clicking on the item toggles between TRUE and FALSE.
\item Showing the value list enables the user to select TRUE or FALSE.
\item The user may be able to type in the word TRUE or FALSE, or the edit control
may be read-only to disallow this since it is error-prone.
\end{itemize}

A list of strings may pop up a dialog for editing them, a simple string just allows text editing,
double-clicking a colour property may show a colour selector, double-clicking on a filename property may
show a file selector (in addition to being able to type in the name in the edit control), etc.

Note that the `type' of property, such as string or integer, does not
necessarily determine the behaviour of the property. The programmer has
to be able to specify different behaviours for the same type, depending
on the meaning of the property. For example, a colour and a filename may
both be strings, but their editing behaviour should be different. This
is why objects of type wxPropertyValidator need to be used, to define
behaviour for a given class of properties or even specific property
name.  Objects of class wxPropertyView contain a list of property
registries, which enable reuse of bunches of these validators in
different circumstances. Or a wxProperty can be explicitly set to use a
particular validator object. 

The following screen shot of the property classes test program shows the
user editing a string, which is constrained to be one of three possible
values.

\helponly{\image{}{prop1.bmp}}

The second picture shows the user having entered a integer that
was outside the range specified to the validator. Note that in this picture,
the value list is hidden because it is not used when editing an integer.

\helponly{\image{}{prop2.bmp}}

\section{Headers}\label{proplistfiles}

The property class library comprises the following files:

\begin{itemize}\itemsep=0pt
\item prop.h: base property class header
\item proplist.h: wxPropertyListView and associated classes
\item propform.h: wxPropertyListView and associated classes
\end{itemize}

\section{Topic overviews}\label{proplistoverviews}

This chapter contains a selection of topic overviews.

\subsection{Property classes overview}\label{proplistpropertyoverview}

The property classes help a programmer to express relationships between
data and physical windows, in particular:

\begin{itemize}\itemsep=0pt
\item the transfer of data to and from the physical controls;
\item the behaviour of various controls and custom windows for particular
types of data;
\item the validation of data, notifying the user when incorrect data is entered,
or even better, constraining the input so only valid data can be entered.
\end{itemize}

With a consistent framework, the programmer should be able to use existing
components and design new ones in a principled manner, to solve many data entry
requirements.

Each datum is represented in a \helpref{wxProperty}{wxproperty}, which has a name and a value.
Various C++ types are permitted in the value of a property, and the property can store a pointer
to the data instead of a copy of the data. A \helpref{wxPropertySheet}{wxpropertysheet} represents a number of these properties.

These two classes are independent from the way in which the data is visually manipulated. To
mediate between property sheets and windows, the abstract class \helpref{wxPropertyView}{wxpropertyview} is
available for programmers to derive new kinds of view. One kind of view that is available is the \helpref{wxPropertyListView}{wxpropertylistview},
which displays the data in a Visual Basic-style list, with a small number of controls for editing
the currently selected property. Another is \helpref{wxPropertyFormView}{wxpropertyformview} which
mediates between an existing dialog or panel and the property sheet.

The hard work of mediation is actually performed by validators, which are instances of classes
derived from \helpref{wxPropertyValidator}{wxpropertyvalidator}. A validator is associated with
a particular property and is responsible for
responding to user interface events, and displaying, updating and checking the property value.
Because a validator's behaviour depends largely on the kind of view being used, there has to be
a separate hierarchy of validators for each class of view. So for wxPropertyListView, there is
an abstract class \helpref{wxPropertyListValidator}{wxpropertylistvalidator} from which concrete
classes are derived, such as \helpref{wxRealListValidator}{wxreallistvalidator} and
\rtfsp\helpref{wxStringListValidator}{wxstringlistvalidator}.

A validator can be explicitly set for a property, so there is no doubt which validator
should be used to edit that property. However, it is also possible to define a registry
of validators, and have the validator chosen on the basis of the {\it role} of the property.
So a property with a ``filename" role would match the ``filename" validator, which pops
up a file selector when the user double clicks on the property.

You don't have to define your own frame or window classes: there are some predefined
that will work with the property list view. See \helpref{Window classes}{proplistwindowclasses} for
further details.

\subsubsection{Example 1: Property list view}

The following code fragment shows the essentials of creating a registry of
standard validators, a property sheet containing some properties, and
a property list view and dialog or frame. RegisterValidators will be
called on program start, and PropertySheetTest is called in response to a
menu command.

Note how some properties are created with an explicit reference to
a validator, and others are provided with a ``role'' which can be matched
against a validator in the registry.

The interface generated by this test program is shown in the section \helpref{Appearance and
behaviour of a property list view}{proplistappearance}.

\begin{verbatim}
void RegisterValidators(void)
{
  myListValidatorRegistry.RegisterValidator((wxString)"real", new wxRealListValidator);
  myListValidatorRegistry.RegisterValidator((wxString)"string", new wxStringListValidator);
  myListValidatorRegistry.RegisterValidator((wxString)"integer", new wxIntegerListValidator);
  myListValidatorRegistry.RegisterValidator((wxString)"bool", new wxBoolListValidator);
}

void PropertyListTest(Bool useDialog)
{
  wxPropertySheet *sheet = new wxPropertySheet;

  sheet->AddProperty(new wxProperty("fred", 1.0, "real"));
  sheet->AddProperty(new wxProperty("tough choice", (Bool)TRUE, "bool"));
  sheet->AddProperty(new wxProperty("ian", (long)45, "integer", new wxIntegerListValidator(-50, 50)));
  sheet->AddProperty(new wxProperty("bill", 25.0, "real", new wxRealListValidator(0.0, 100.0)));
  sheet->AddProperty(new wxProperty("julian", "one", "string"));
  sheet->AddProperty(new wxProperty("bitmap", "none", "string", new wxFilenameListValidator("Select a bitmap file", "*.bmp")));
  wxStringList *strings = new wxStringList("one", "two", "three", NULL);
  sheet->AddProperty(new wxProperty("constrained", "one", "string", new wxStringListValidator(strings)));

  wxPropertyListView *view =
    new wxPropertyListView(NULL,
     wxPROP_BUTTON_CHECK_CROSS|wxPROP_DYNAMIC_VALUE_FIELD|wxPROP_PULLDOWN);

  wxDialogBox *propDialog = NULL;
  wxPropertyListFrame *propFrame = NULL;
  if (useDialog)
  {
    propDialog = new wxPropertyListDialog(view, NULL, "Property Sheet Test", TRUE, -1, -1, 400, 500);
  }
  else
  {
    propFrame = new wxPropertyListFrame(view, NULL, "Property Sheet Test", -1, -1, 400, 500);
  }
  
  view->AddRegistry(&myListValidatorRegistry);

  if (useDialog)
  {
    view->ShowView(sheet, propDialog);
    propDialog->Centre(wxBOTH);
    propDialog->Show(TRUE);
  }
  else
  {
    propFrame->Initialize();
    view->ShowView(sheet, propFrame->GetPropertyPanel());
    propFrame->Centre(wxBOTH);
    propFrame->Show(TRUE);
  }
}
\end{verbatim}

\subsubsection{Example 2: Property form view}

This example is similar to Example 1, but uses a property form view to
edit a property sheet using a predefined dialog box.

\begin{verbatim}
void RegisterValidators(void)
{
  myFormValidatorRegistry.RegisterValidator((wxString)"real", new wxRealFormValidator);
  myFormValidatorRegistry.RegisterValidator((wxString)"string", new wxStringFormValidator);
  myFormValidatorRegistry.RegisterValidator((wxString)"integer", new wxIntegerFormValidator);
  myFormValidatorRegistry.RegisterValidator((wxString)"bool", new wxBoolFormValidator);
}

void PropertyFormTest(Bool useDialog)
{
  wxPropertySheet *sheet = new wxPropertySheet;

  sheet->AddProperty(new wxProperty("fred", 25.0, "real", new wxRealFormValidator(0.0, 100.0)));
  sheet->AddProperty(new wxProperty("tough choice", (Bool)TRUE, "bool"));
  sheet->AddProperty(new wxProperty("ian", (long)45, "integer", new wxIntegerFormValidator(-50, 50)));
  sheet->AddProperty(new wxProperty("julian", "one", "string"));
  wxStringList *strings = new wxStringList("one", "two", "three", NULL);
  sheet->AddProperty(new wxProperty("constrained", "one", "string", new wxStringFormValidator(strings)));

  wxPropertyFormView *view = new wxPropertyFormView(NULL);

  wxDialogBox *propDialog = NULL;
  wxPropertyFormFrame *propFrame = NULL;
  if (useDialog)
  {
    propDialog = new wxPropertyFormDialog(view, NULL, "Property Form Test", TRUE, -1, -1, 400, 300);
  }
  else
  {
    propFrame = new wxPropertyFormFrame(view, NULL, "Property Form Test", -1, -1, 400, 300);
    propFrame->Initialize();
  }
  
  wxPanel *panel = propDialog ? propDialog : propFrame->GetPropertyPanel();
  panel->SetLabelPosition(wxVERTICAL);
  
  // Add items to the panel
  
  (void) new wxButton(panel, (wxFunction)NULL, "OK", -1, -1, -1, -1, 0, "ok");
  (void) new wxButton(panel, (wxFunction)NULL, "Cancel", -1, -1, 80, -1, 0, "cancel");
  (void) new wxButton(panel, (wxFunction)NULL, "Update", -1, -1, 80, -1, 0, "update");
  (void) new wxButton(panel, (wxFunction)NULL, "Revert", -1, -1, -1, -1, 0, "revert");
  panel->NewLine();
  
  // The name of this text item matches the "fred" property
  (void) new wxText(panel, (wxFunction)NULL, "Fred", "", -1, -1, 90, -1, 0, "fred");
  (void) new wxCheckBox(panel, (wxFunction)NULL, "Yes or no", -1, -1, -1, -1, 0, "tough choice");
  (void) new wxSlider(panel, (wxFunction)NULL, "Sliding scale", 0, -50, 50, 100, -1, -1, wxHORIZONTAL, "ian");
  panel->NewLine();
  (void) new wxListBox(panel, (wxFunction)NULL, "Constrained", wxSINGLE, -1, -1, 100, 90, 0, NULL, 0, "constrained");

  view->AddRegistry(&myFormValidatorRegistry);

  if (useDialog)
  {
    view->ShowView(sheet, propDialog);
    view->AssociateNames();
    view->TransferToDialog();
    propDialog->Centre(wxBOTH);
    propDialog->Show(TRUE);
  }
  else
  {
    view->ShowView(sheet, propFrame->GetPropertyPanel());
    view->AssociateNames();
    view->TransferToDialog();
    propFrame->Centre(wxBOTH);
    propFrame->Show(TRUE);
  }
}
\end{verbatim}

\subsection{Validator classes overview}\label{proplistvalidatoroverview}

Classes: \helpref{Validator classes}{proplistvalidatorclasses}

The validator classes provide functionality for mediating between a wxProperty and
the actual display. There is a separate family of validator classes for each
class of view, since the differences in user interface for these views implies
that little common functionality can be shared amongst validators.



\subsubsection{wxPropertyValidator overview}\label{wxpropertyvalidatoroverview}

Class: \helpref{wxPropertyValidator}{wxpropertyvalidator}

This class is the root of all property validator classes. It contains a small
amount of common functionality, including functions to convert between
strings and C++ values.

A validator is notionally an object which sits between a property and its displayed
value, and checks that the value the user enters is correct, giving an error message
if the validation fails. In fact, the validator does more than that, and is akin to
a view class but at a finer level of detail. It is also responsible for
loading the dialog box control with the value from the property, putting it back
into the property, preparing special controls for editing the value, and
may even invoke special dialogs for editing the value in a convenient way.

In a property list dialog, there is quite a lot of scope for supplying custom dialogs,
such as file or colour selectors. For a form dialog, there is less scope because
there is no concept of `detailed editing' of a value: one control is associated with
one property, and there is no provision for invoking further dialogs. The reader
may like to work out how the form view could be extended to provide some of the
functionality of the property list!

Validator objects may be associated explictly with a wxProperty, or they may be
indirectly associated by virtue of a property `kind' that matches validators having
that kind. In the latter case, such validators are stored in a validator registry
which is passed to the view before the dialog is shown. If the validator takes
arguments, such as minimum and maximum values in the case of a wxIntegerListValidator,
then the validator must be associated explicitly with the property. The validator
will be deleted when the property is deleted.

\subsubsection{wxPropertyListValidator overview}\label{wxpropertylistvalidatoroverview}

Class: \helpref{wxPropertyListValidator}{wxpropertylistvalidator}

This class is the abstract base class for property list view validators.
The list view acts upon a user interface containing a list of properties,
a text item for direct property value editing, confirm/cancel buttons for the value,
a pulldown list for making a choice between values, and OK/Cancel/Help buttons
for the dialog (see \helpref{property list appearance}{proplistappearance}).

By overriding virtual functions, the programmer can create custom
behaviour for different kinds of property. Custom behaviour can use just the
available controls on the property list dialog, or the validator can
invoke custom editors with quite different controls, which pop up in
`detailed editing' mode.

See the detailed class documentation for the members you should override
to give your validator appropriate behaviour.

\subsubsection{wxPropertyFormValidator overview}\label{wxpropertyformvalidatoroverview}

This class is the abstract base class for property form view validators.
The form view acts upon an existing dialog box or panel, where either the
panel item names correspond to property names, or the programmer has explicitly
associated the panel item with the property.

By overriding virtual functions, the programmer determines how
values are displayed or retrieved, and the checking that the validator does.

See the detailed class documentation for the members you should override
to give your validator appropriate behaviour.

\subsection{View classes overview}\label{proplistviewoverview}

Classes: \helpref{View classes}{proplistviewclasses}

An instance of a view class relates a property sheet with an actual window.
Currently, there are two classes of view: wxPropertyListView and wxPropertyFormView.

\subsubsection{wxPropertyView overview}\label{wxpropertyviewoverview}

Class: \helpref{wxPropertyView}{wxpropertyview}

This is the abstract base class for property views.

\subsubsection{wxPropertyListView overview}\label{wxpropertylistviewoverview}

Class: \helpref{wxPropertyListView}{wxpropertylistview}

The property list view defines the relationship between a property sheet and
a property list dialog or panel. It manages user interface events such as
clicking on a property, pressing return in the text edit field, and clicking
on Confirm or Cancel. These events cause member functions of the
class to be called, and these in turn may call member functions of
the appropriate validator to be called, to prepare controls, check the property value,
invoke detailed editing, etc.

\subsubsection{wxPropertyFormView overview}\label{wxpropertyformviewoverview}

Class: \helpref{wxPropertyFormView}{wxpropertyformview}

The property form view manages the relationship between a property sheet
and an existing dialog or panel.

You must first create a panel or dialog box for the view to work on.
The panel should contain panel items with names that correspond to
properties in your property sheet; or you can explicitly set the
panel item for each property.

Apart from any custom panel items that you wish to control independently
of the property-editing items, wxPropertyFormView takes over the
processing of item events. It can also control normal dialog behaviour such
as OK, Cancel, so you should also create some standard buttons that the property view
can recognise. Just create the buttons with standard names and the view
will do the rest. The following button names are recognised:

\begin{itemize}\itemsep=0pt
\item {\bf ok}: indicates the OK button. Calls wxPropertyFormView::OnOk. By default,
checks and updates the form values, closes and deletes the frame or dialog, then deletes the view.
\item {\bf cancel}: indicates the Cancel button. Calls wxPropertyFormView::OnCancel. By default,
closes and deletes the frame or dialog, then deletes the view.
\item {\bf help}: indicates the Help button. Calls wxPropertyFormView::OnHelp. This needs
to be overridden by the application for anything interesting to happen.
\item {\bf revert}: indicates the Revert button. Calls wxPropertyFormView::OnRevert,
which by default transfers the wxProperty values to the panel items (in effect
undoing any unsaved changes in the items).
\item {\bf update}: indicates the Revert button. Calls wxPropertyFormView::OnUpdate, which
by defaults transfers the displayed values to the wxProperty objects.
\end{itemize}

\subsection{wxPropertySheet overview}\label{wxpropertysheetoverview}

Classes: \helpref{wxPropertySheet}{wxpropertysheet}, \helpref{wxProperty}{wxproperty}, \helpref{wxPropertyValue}{wxpropertyvalue}

A property sheet defines zero or more properties. This is a bit like an explicit representation of
a C++ object. wxProperty objects can have values which are pointers to C++ values, or they
can allocate their own storage for values.

Because the property sheet representation is explicit and can be manipulated by
a program, it is a convenient form to be used for a variety of
editing purposes. wxPropertyListView and wxPropertyFormView are two classes that
specify the relationship between a property sheet and a user interface. You could imagine
other uses for wxPropertySheet, for example to generate a form-like user interface without
the need for GUI programming. Or for storing the names and values of command-line switches, with the
option to subsequently edit these values using a wxPropertyListView.

A typical use for a property sheet is to represent values of an object
which are only implicit in the current representation of it. For
example, in Visual Basic and similar programming environments, you can
`edit a button', or rather, edit the button's properties.  One of the
properties you can edit is {\it width} - but there is no explicit
representation of width in a wxWindows button; instead, you call SetSize
and GetSize members. To translate this into a consisent,
property-oriented scheme, we could derive a new class
wxButtonWithProperties, which has two new functions: SetProperty and
GetProperty.  SetProperty accepts a property name and a value, and calls
an appropriate function for the property that is being passed.
GetProperty accepts a property name, returning a property value. So
instead of having to use the usual arbitrary set of C++ member functions
to set or access attributes of a window, programmer deals merely with
SetValue/GetValue, and property names and values.
We now have a single point at which we can modify or query an object by specifying
names and values at run-time. (The implementation of SetProperty and GetProperty
is probably quite messy and involves a large if-then-else statement to
test the property name and act accordingly.)

When the user invokes the property editor for a wxButtonWithProperties, the system
creates a wxPropertySheet with `imaginary' properties such as width, height, font size
and so on. For each property, wxButtonWithProperties::GetProperty is called, and the result is
passed to the corresponding wxProperty. The wxPropertySheet is passed to a wxPropertyListView
as described elsewhere, and the user edits away. When the user has finished editing, the system calls
wxButtonWithProperties::SetProperty to transfer the wxProperty value back into the button
by way of an appropriate call, wxWindow::SetSize in the case of width and height properties.

\section{Classes by category}\label{proplistclassesbycat}

A classification of property sheet classes by category.

\subsection{Data classes}

\begin{itemize}\itemsep=0pt
\item \helpref{wxProperty}{wxproperty}
\item \helpref{wxPropertyValue}{wxpropertyvalue}
\item \helpref{wxPropertySheet}{wxpropertysheet}
\end{itemize}

\subsection{Validator classes}\label{proplistvalidatorclasses}

Validators check that the values the user has entered for a property are
valid. They can also define specific ways of entering data, such as a
file selector for a filename, and they are responsible for transferring
values between the wxProperty and the physical display. 

Base classes:

\begin{itemize}\itemsep=0pt
\item \helpref{wxPropertyValidator}{wxproperty}
\item \helpref{wxPropertyListValidator}{wxpropertylistvalidator}
\item \helpref{wxPropertyFormValidator}{wxpropertyformvalidator}
\end{itemize}

List view validators:

\begin{itemize}\itemsep=0pt
\item \helpref{wxBoolListValidator}{wxboollistvalidator}
\item \helpref{wxFilenameListValidator}{wxfilenamelistvalidator}
\item \helpref{wxIntegerListValidator}{wxintegerlistvalidator}
\item \helpref{wxListOfStringsListValidator}{wxlistofstringslistvalidator}
\item \helpref{wxRealListValidator}{wxreallistvalidator}
\item \helpref{wxStringListValidator}{wxstringlistvalidator}
\end{itemize}

Form view validators:

\begin{itemize}\itemsep=0pt
\item \helpref{wxBoolFormValidator}{wxboolformvalidator}
\item \helpref{wxIntegerFormValidator}{wxintegerformvalidator}
\item \helpref{wxRealFormValidator}{wxrealformvalidator}
\item \helpref{wxStringFormValidator}{wxstringformvalidator}
\end{itemize}

\subsection{View classes}\label{proplistviewclasses}

View classes mediate between a property sheet and a physical window.

\begin{itemize}\itemsep=0pt
\item \helpref{wxPropertyView}{wxpropertyview}
\item \helpref{wxPropertyListView}{wxpropertylistview}
\item \helpref{wxPropertyFormView}{wxpropertyformview}
\end{itemize}

\subsection{Window classes}\label{proplistwindowclasses}

The class library defines some window classes that can be used as-is with a suitable
view class and property sheet.

\begin{itemize}\itemsep=0pt
\item \helpref{wxPropertyFormFrame}{wxpropertyformframe}
\item \helpref{wxPropertyFormDialog}{wxpropertyformdialog}
\item \helpref{wxPropertyFormPanel}{wxpropertyformpanel}
\item \helpref{wxPropertyListFrame}{wxpropertylistframe}
\item \helpref{wxPropertyListDialog}{wxpropertylistdialog}
\item \helpref{wxPropertyListPanel}{wxpropertylistpanel}
\end{itemize}

\subsection{Registry classes}

A validator registry is a list of validators that can be applied to properties in a property sheet.
There may be one or more registries per property view.

\begin{itemize}\itemsep=0pt
\item \helpref{wxPropertyValidatorRegistry}{wxpropertyvalidatorregistry}
\end{itemize}

\chapter{wxPython Notes}\label{wxPython}
\pagenumbering{arabic}%
\setheader{{\it CHAPTER \thechapter}}{}{}{}{}{{\it CHAPTER \thechapter}}%
\setfooter{\thepage}{}{}{}{}{\thepage}%

%----------------------------------------------------------------------
\section{What is wxPython?}\label{wxpwhat}

wxPython is a blending of the wxWindows GUI classes and the
\urlref{Python}{http://www.python.org/} programming language.

\wxheading{Python}

So what is Python?  Go to
\urlref{http://www.python.org}{http://www.python.org}
to learn more, but in a nutshell Python is an interpreted,
interactive, object-oriented programming language. It is often
compared to Tcl, Perl, Scheme or Java.

Python combines remarkable power with very clear syntax. It has
modules, classes, exceptions, very high level dynamic data types, and
dynamic typing. There are interfaces to many system calls and
libraries, and new built-in modules are easily written in C or
C++. Python is also usable as an extension language for applications
that need a programmable interface.

Python is copyrighted but freely usable and distributable, even for
commercial use.

\wxheading{wxPython}

wxPython is a Python package that can be imported at runtime that
includes a collection of Python modules and an extension module
(native code).  It provides a series of Python classes that mirror (or
shadow) many of the wxWindows GUI classes.  This extension module
attempts to mirror the class heiarchy of wxWindows as closely as
possble. This means that there is a wxFrame class in wxPython that
looks, smells, tastes and acts almost the same as the wxFrame class in
the C++ version.

wxPython is very versitile.  It can be used to create standalone GUI
applications, or in situations where Python is embedded in a C++
application as an internal scripting or macro language.

Currently wxPython is available for Win32 platforms and the GTK
toolkit (wxGTK) on most *nix/X-windows platforms.  The effort to
enable wxPython for wxMotif will begin shortly.  See \helpref{Building
Python}{wxpbuild} for details about getting wxPython working for you.


%----------------------------------------------------------------------
\section{Why use wxPython?}\label{wxpwhy}


So why would you want to use wxPython over just C++ and wxWindows?
Personally I prefer using Python for everything.  I only use C++ when
I absolutly have to eek more performance out of an algorithm, and even
then I ususally code it as an extension module and leave the majority
of the program in Python.

Another good thing to use wxPython for is quick prototyping of your
wxWindows apps.  With C++ you have to continuously go though the
edit-compile-link-run cycle, which can be quite time comsuming.  With
Python it is only an edit-run cycle.  You can easily build an
application in a few hours with Python that would normally take a few
days or longer with C++.  Converting a wxPython app to a C++/wxWindows app
should be a straight forward task.


%----------------------------------------------------------------------
\section{Other Python GUIs}\label{wxpother}

There are other GUI solutions out there for Python.

\wxheading{Tkinter}

Tkinter is the defacto standard GUI for Python.  It is available
on nearly every platform that Python and Tcl/TK are.  Why Tcl/Tk?
Well because Tkinter is just a wrapper around Tcl's GUI toolkit, Tk.
This has its upsides and its downsides...

The upside is that Tk is a pretty veristile toolkit.  It can be made
to do a lot of things in a lot of different environments.  It is fairly
easy to create new widgets and use them interchangably in your
programs.

The downside is Tcl.  When using Tkinter you actually have two
separate language interpreters running, the Python interpreter and the
Tcl interpreter for the GUI.  Since the guts of Tcl is mostly about
string processing, it is fairly slow as well.  (Not too bad on a fast
Pentium II, but you really notice the difference on slower machines.)

It wasn't until the lastest version of Tcl/Tk that native Look and
Feel's were possible on non-Motif platforms.  This is because Tk
usually implements it's own widgets (controls) even when there are
native controls available.

Tkinter is a pretty low-level toolkit.  You have to do a lot of work
(verbose program code) to do things that would be much simpler with a higher
level of abstraction.

\wxheading{PythonWin}

PythonWin is an add-on package for Python for the Win32 platform.  It
includes wrappers for MFC as well as much of the win32 API.  Because
of its foundation, it is very familiar for programmers who have
experience with MFC and the Win32 API.  It is obviously not compatible
with other platforms and toolkits.  PythonWin is organized as separate
packages and modules so you can use the pieces you need without having
to use the GUI portions.

\wxheading{Others}

There are quite a few other GUI modules available for Python, some in
active use, some that havn't been updated for ages.  Most are simple
wrappers around some C or C++ toolkit or another, and most are not
cross-platform compatible.  See \urlref{this
link}{http://www.python.org/download/Contributed.html\#Graphics}
for a listing of a few of them.


%----------------------------------------------------------------------
\section{Building wxPython}\label{wxpbuild}

I used SWIG (\urlref{http://www.swig.org}{http://www.swig.org}) to
create the source code for the extension module.  This enabled me to
only have to deal with a small amount of code and only have to bother
with the exceptional issues.  SWIG takes care of the rest and
generates all the repetative code for me.  You don't need SWIG to
build the extension module as all the generated C++ code is included
under the src directory.  If you try to build wxPython and get errors
because SWIG is missing, then simply touch the .cpp and .py files so
make won't attempt to build them from the .i files.

I added a few minor features to SWIG to control some of the code
generation.  If you want to play around with this the patches are in
wxPython/SWIG.patches and they should be applied to the 1.1p5 version
of SWIG.  These new patches are documented at
\urlref{this site}{http://starship.skyport.net/crew/robind/python/\#swig},
and they should also end up in the 1.2 version of SWIG.

wxPython is organized as a Python package.  This means that the
directory containing the results of the build process should be a
subdirectory of a directory on the \tt{PYTHONPATH}, (and preferably
should be named wxPython.)  You can control where the build process
will dump wxPython by setting the \tt{TARGETDIR} makefile variable.
The default is \tt{\$(WXWIN)/utils/wxPython}.  If you leave it here
then you should add \tt{\$(WXWIN)/utils} to your \tt{PYTHONPATH}.
However, you may prefer to use something that is already on your
\tt{PYTHONPATH}, such as the \tt{site-packages} directory on Unix
 systems.


\wxheading{Win32}

These instructions assume that you have Microsoft Visual C++ 5.0 or
6.0, that you have installed the command-line tools, and that the
appropriate environment variables are set for these tools.  You should
also have Python 1.5.1 installed, and wxWindows installed and built as
specified below.

\begin{enumerate}\itemsep=0pt
\item Build wxWindows with \tt{wxUSE_RESOURCE_LOADING_IN_MSW} set to 1 in
\tt{include/wx/msw/setup.h} so icons can be loaded dynamically.  While
there, make sure \tt{wxUSE_OWNER_DRAWN} is also set to 1.

\item Change into the \tt{\$(WXWIN)/utils/wxPython/src} directory.

\item Edit makefile.vc and specify where your python installation is at.
You may also want to fiddle with the \tt{TARGETDIR} variable as described
above.

\item Run \tt{nmake -f makefile.vc}

\item If it builds successfully, congratulations!  Move on to the next
step.  If not then you can try mailing the wxwin-developers list for
help.  Also, I will always have a pre-built win32 version of this extension module at
\urlref{http://alldunn.com/wxPython}{http://alldunn.com/wxPython}.

\item Change to the \tt{\$(WXWIN)/utils/wxPython/tests} directory.

\item Try executing the test programs.  Note that some of these print
diagnositc or test info to standard output, so they will require the
console version of python.  For example:

    \tt{python test1.py}

To run them without requiring a console, you can use the \tt{pythonw.exe}
version of Python either from the command line or from a shortcut.

\end{enumerate}


\wxheading{Unix}

These directions assume that you have already successfully built
wxWindows for GTK, and installed Python 1.5.1.  If you build Python
yourself, you will get everything installed that you need simply by
doing \bftt{make install}.  If you get Python from an RPM or other
pre-packaged source then there will probably be a separate package
with the development libraries, etc. that you will need to install.


\begin{enumerate}\itemsep=0pt
\item Change into the \tt{\$(WXWIN)/utils/wxPython/src} directory.

\item Edit \tt{Setup.in} and ensure that the flags, directories, and toolkit
options are correct, (hopefully this will be done by \tt{configure}
soon.)  See the above commentary about \tt{TARGETDIR}.  There are a
few sample Setup.in.[platform] files provided.

\item Run this command to generate a makefile:

    \tt{make -f Makefile.pre.in boot}

\item Once you have the \tt{Makefile}, run \bftt{make} to build and then
\bftt{make install} to install the wxPython extension module.

\item Change to the \tt{\$(WXWIN)/utils/wxPython/tests} directory.

\item Try executing the test programs.  For example:

    \tt{python test1.py}

\end{enumerate}



%----------------------------------------------------------------------
\section{Using wxPython}\label{wxpusing}

\wxheading{First things first...}

I'm not going to try and teach the Python language here.  You can do
that at the \urlref{Python Tutorial}{http://www.python.org/doc/tut/tut.html}.
I'm also going to assume that you know a bit about wxWindows already,
enough to notice the similarities in the classes used.

Take a look at the following wxPython program.  You can find a similar
program in the \tt{wxPython/tests} directory, named \tt{test7.py}.  If your
Python and wxPython are properly installed, you should be able to run
it by issuing this command:

\begin{indented}{1cm}
    \bftt{python test7.py}
\end{indented}

\hrule

\begin{verbatim}
001: ## import all of the wxPython GUI package
002: from wxPython.wx import *
003:
004: ## Create a new frame class, derived from the wxPython Frame.
005: class MyFrame(wxFrame):
006:
007:     def __init__(self, parent, id, title):
008:         # First, call the base class' __init__ method to create the frame
009:         wxFrame.__init__(self, parent, id, title,
010:                          wxPoint(100, 100), wxSize(160, 100))
011:
012:         # Associate some events with methods of this class
013:         EVT_SIZE(self, self.OnSize)
014:         EVT_MOVE(self, self.OnMove)
015:
016:         # Add a panel and some controls to display the size and position
017:         panel = wxPanel(self, -1)
018:         wxStaticText(panel, -1, "Size:",
019:                      wxDLG_PNT(panel, wxPoint(4, 4)),  wxDefaultSize)
020:         wxStaticText(panel, -1, "Pos:",
021:                      wxDLG_PNT(panel, wxPoint(4, 14)), wxDefaultSize)
022:         self.sizeCtrl = wxTextCtrl(panel, -1, "",
023:                                    wxDLG_PNT(panel, wxPoint(24, 4)),
024:                                    wxDLG_SZE(panel, wxSize(36, -1)),
025:                                    wxTE_READONLY)
026:         self.posCtrl = wxTextCtrl(panel, -1, "",
027:                                   wxDLG_PNT(panel, wxPoint(24, 14)),
028:                                   wxDLG_SZE(panel, wxSize(36, -1)),
029:                                   wxTE_READONLY)
030:
031:
032:     # This method is called automatically when the CLOSE event is
033:     # sent to this window
034:     def OnCloseWindow(self, event):
035:         # tell the window to kill itself
036:         self.Destroy()
037:
038:     # This method is called by the system when the window is resized,
039:     # because of the association above.
040:     def OnSize(self, event):
041:         size = event.GetSize()
042:         self.sizeCtrl.SetValue("%s, %s" % (size.width, size.height))
043:
044:         # tell the event system to continue looking for an event handler,
045:         # so the default handler will get called.
046:         event.Skip()
047:
048:     # This method is called by the system when the window is moved,
049:     # because of the association above.
050:     def OnMove(self, event):
051:         pos = event.GetPosition()
052:         self.posCtrl.SetValue("%s, %s" % (pos.x, pos.y))
053:
054:
055: # Every wxWindows application must have a class derived from wxApp
056: class MyApp(wxApp):
057:
058:     # wxWindows calls this method to initialize the application
059:     def OnInit(self):
060:
061:         # Create an instance of our customized Frame class
062:         frame = MyFrame(NULL, -1, "This is a test")
063:         frame.Show(true)
064:
065:         # Tell wxWindows that this is our main window
066:         self.SetTopWindow(frame)
067:
068:         # Return a success flag
069:         return true
070:
071:
072: app = MyApp(0)     # Create an instance of the application class
073: app.MainLoop()     # Tell it to start processing events
074:
\end{verbatim}
\hrule

\wxheading{Things to notice:}\begin{enumerate}\itemsep=0pt
\item At line 2 the wxPython classes, constants, and etc. are imported
into the current module's namespace.  If you prefer to reduce
namespace polution you can use "\tt{from wxPython import wx}" and
then access all the wxPython identifiers through the wx module, for
example, "\tt{wx.wxFrame}".

\item At line 13 the frame's sizing and moving events are connected to
methods of the class.  These helper functions are intended to be like
the event table macros that wxWindows employs.  But since static event
tables are impossible with wxPython, we use helpers that are named the
same to dynamically build the table.  The only real difference is
that the first arguemnt to the event helpers is always the window that
the event table entry should be added to.

\item Notice the use of \tt{wxDLG_PNT} and \tt{wxDLG_SZE} in lines 19
- 29 to convert from dialog units to pixels.  These helpers are unique
to wxPython since Python can't do method overloading like C++.

\item There is an \tt{OnCloseWindow} method at line 34 but no call to
EVT_CLOSE to attach the event to the method.  Does it really get
called?  The answer is, yes it does.  This is because many of the
\em{standard} events are attached to windows that have the associated
\em{standard} method names.  I have tried to follow the lead of the
C++ classes in this area to determine what is \em{standard} but since
that changes from time to time I can make no guarentees, nor will it
be fully documented.  When in doubt, use an EVT_*** function.

\item At lines 17 to 21 notice that there are no saved references to
the panel or the static text items that are created.  Those of you
who know Python might be wondering what happens when Python deletes
these objects when they go out of scope.  Do they disappear from the GUI?  They
don't.  Remember that in wxPython the Python objects are just shadows of the
coresponding C++ objects.  Once the C++ windows and controls are
attached to their parents, the parents manage them and delete them
when necessary.  For this reason, most wxPython objects do not need to
have a __del__ method that explicitly causes the C++ object to be
deleted.  If you ever have the need to forcibly delete a window, use
the Destroy() method as shown on line 36.

\item Just like wxWindows in C++, wxPython apps need to create a class
derived from \tt{wxApp} (line 56) that implements a method named
\tt{OnInit}, (line 59.) This method should create the application's
main window (line 62) and use \tt{wxApp.SetTopWindow()} (line 66) to
inform wxWindows about it.

\item And finally, at line 72 an instance of the application class is
created.  At this point wxPython finishes initializing itself, and calls
the \tt{OnInit} method to get things started.  (The zero parameter here is
a flag for functionality that isn't quite implemented yet.  Just
ignore it for now.)  The call to \tt{MainLoop} at line 73 starts the event
loop which continues until the application terminates or all the top
level windows are closed.

\end{enumerate}



%----------------------------------------------------------------------
\section{wxWindows classes implemented in wxPython}\label{wxpclasses}

The following classes are supported in wxPython.  Most provide nearly
full implementations of the public interfaces specified in the C++
documentation, others are less so.  They will all be brought as close
as possible to the C++ spec over time.

\begin{itemize}\itemsep=0pt
\item \helpref{wxAcceleratorEntry}{wxacceleratorentry}
\item \helpref{wxAcceleratorTable}{wxacceleratortable}
\item \helpref{wxActivateEvent}{wxactivateevent}
\item \helpref{wxBitmapButton}{wxbitmapbutton}
\item \helpref{wxBitmap}{wxbitmap}
\item \helpref{wxBrush}{wxbrush}
\item \helpref{wxButton}{wxbutton}
\item \helpref{wxCalculateLayoutEvent}{wxcalculatelayoutevent}
\item \helpref{wxCheckBox}{wxcheckbox}
\item \helpref{wxCheckListBox}{wxchecklistbox}
\item \helpref{wxChoice}{wxchoice}
\item \helpref{wxClientDC}{wxclientdc}
\item \helpref{wxCloseEvent}{wxcloseevent}
\item \helpref{wxColourData}{wxcolourdata}
\item \helpref{wxColourDialog}{wxcolourdialog}
\item \helpref{wxColour}{wxcolour}
\item \helpref{wxComboBox}{wxcombobox}
\item \helpref{wxCommandEvent}{wxcommandevent}
\item \helpref{wxConfig}{wxconfigbase}
\item \helpref{wxControl}{wxcontrol}
\item \helpref{wxCursor}{wxcursor}
\item \helpref{wxDC}{wxdc}
\item \helpref{wxDialog}{wxdialog}
\item \helpref{wxDirDialog}{wxdirdialog}
\item \helpref{wxDropFilesEvent}{wxdropfilesevent}
\item \helpref{wxEraseEvent}{wxeraseevent}
\item \helpref{wxEvent}{wxevent}
\item \helpref{wxEvtHandler}{wxevthandler}
\item \helpref{wxFileDialog}{wxfiledialog}
\item \helpref{wxFocusEvent}{wxfocusevent}
\item \helpref{wxFontData}{wxfontdata}
\item \helpref{wxFontDialog}{wxfontdialog}
\item \helpref{wxFont}{wxfont}
\item \helpref{wxFrame}{wxframe}
\item \helpref{wxGauge}{wxgauge}
\item \helpref{wxGridCell}{wxgridcell}
\item \helpref{wxGridEvent}{wxgridevent}
\item \helpref{wxGrid}{wxgrid}
\item \helpref{wxIconizeEvent}{wxiconizeevent}
\item \helpref{wxIcon}{wxicon}
\item \helpref{wxIdleEvent}{wxidleevent}
\item \helpref{wxImageList}{wximagelist}
\item \helpref{wxIndividualLayoutConstraint}{wxindividuallayoutconstraint}
\item \helpref{wxInitDialogEvent}{wxinitdialogevent}
\item \helpref{wxJoystickEvent}{wxjoystickevent}
\item \helpref{wxKeyEvent}{wxkeyevent}
\item \helpref{wxLayoutAlgorithm}{wxlayoutalgorithm}
\item \helpref{wxLayoutConstraints}{wxlayoutconstraints}
\item \helpref{wxListBox}{wxlistbox}
\item \helpref{wxListCtrl}{wxlistctrl}
\item \helpref{wxListEvent}{wxlistevent}
\item \helpref{wxListItem}{wxlistitem}
\item \helpref{wxMDIChildFrame}{wxmdichildframe}
\item \helpref{wxMDIClientWindow}{wxmdiclientwindow}
\item \helpref{wxMDIParentFrame}{wxmdiparentframe}
\item \helpref{wxMask}{wxmask}
\item \helpref{wxMaximizeEvent}{wxmaximizeevent}
\item \helpref{wxMemoryDC}{wxmemorydc}
\item \helpref{wxMenuBar}{wxmenubar}
\item \helpref{wxMenuEvent}{wxmenuevent}
\item \helpref{wxMenuItem}{wxmenuitem}
\item \helpref{wxMenu}{wxmenu}
\item \helpref{wxMessageDialog}{wxmessagedialog}
\item \helpref{wxMetaFileDC}{wxmetafiledc}
\item \helpref{wxMiniFrame}{wxminiframe}
\item \helpref{wxMouseEvent}{wxmouseevent}
\item \helpref{wxMoveEvent}{wxmoveevent}
\item \helpref{wxNotebookEvent}{wxnotebookevent}
\item \helpref{wxNotebook}{wxnotebook}
\item \helpref{wxPageSetupData}{wxpagesetupdata}
\item \helpref{wxPageSetupDialog}{wxpagesetupdialog}
\item \helpref{wxPaintDC}{wxpaintdc}
\item \helpref{wxPaintEvent}{wxpaintevent}
\item \helpref{wxPalette}{wxpalette}
\item \helpref{wxPanel}{wxpanel}
\item \helpref{wxPen}{wxpen}
\item \helpref{wxPoint}{wxpoint}
\item \helpref{wxPostScriptDC}{wxpostscriptdc}
\item \helpref{wxPrintData}{wxprintdata}
\item \helpref{wxPrintDialog}{wxprintdialog}
\item \helpref{wxPrinterDC}{wxprinterdc}
\item \helpref{wxQueryLayoutInfoEvent}{wxquerylayoutinfoevent}
\item \helpref{wxRadioBox}{wxradiobox}
\item \helpref{wxRadioButton}{wxradiobutton}
\item \helpref{wxRealPoint}{wxrealpoint}
\item \helpref{wxRect}{wxrect}
\item \helpref{wxRegionIterator}{wxregioniterator}
\item \helpref{wxRegion}{wxregion}
\item \helpref{wxSashEvent}{wxsashevent}
\item \helpref{wxSashLayoutWindow}{wxsashlayoutwindow}
\item \helpref{wxSashWindow}{wxsashwindow}
\item \helpref{wxScreenDC}{wxscreendc}
\item \helpref{wxScrollBar}{wxscrollbar}
\item \helpref{wxScrollEvent}{wxscrollevent}
\item \helpref{wxScrolledWindow}{wxscrolledwindow}
\item \helpref{wxShowEvent}{wxshowevent}
\item \helpref{wxSingleChoiceDialog}{wxsinglechoicedialog}
\item \helpref{wxSizeEvent}{wxsizeevent}
\item \helpref{wxSize}{wxsize}
\item \helpref{wxSlider}{wxslider}
\item \helpref{wxSpinButton}{wxspinbutton}
\item \helpref{wxSpinEvent}{wxspinevent}
\item \helpref{wxSplitterWindow}{wxsplitterwindow}
\item \helpref{wxStaticBitmap}{wxstaticbitmap}
\item \helpref{wxStaticBox}{wxstaticbox}
\item \helpref{wxStaticText}{wxstatictext}
\item \helpref{wxStatusBar}{wxstatusbar}
\item \helpref{wxSysColourChangedEvent}{wxsyscolourchangedevent}
\item \helpref{wxTaskBarIcon}{wxtaskbaricon}
\item \helpref{wxTextCtrl}{wxtextctrl}
\item \helpref{wxTextEntryDialog}{wxtextentrydialog}
\item \helpref{wxTimer}{wxtimer}
\item \helpref{wxToolBarTool}{wxtoolbartool}
\item \helpref{wxToolBar}{wxtoolbar}
\item \helpref{wxTreeCtrl}{wxtreectrl}
\item \helpref{wxTreeEvent}{wxtreeevent}
\item \helpref{wxTreeItemData}{wxtreeitemdata}
\item \helpref{wxTreeItemId}{wxtreeitemid}
\item \helpref{wxUpdateUIEvent}{wxupdateuievent}
\item \helpref{wxWindowDC}{wxwindowdc}
\item \helpref{wxWindow}{wxwindow}
\end{itemize}

%----------------------------------------------------------------------
\section{Where to go for help}\label{wxphelp}

Since wxPython is a blending of multiple technologies, help comes from
multiple sources.  See
\urlref{http://alldunn.com/wxPython}{http://alldunn.com/wxPython} for details on
various sources of help, but probably the best source is the
wxPython-users mail list.  You can view the archive or subscribe by
going to

\urlref{http://starship.python.net/mailman/listinfo/wxpython-users}{http://starship.python.net/mailman/listinfo/wxpython-users}

Or you can send mail directly to the list using this address:

wxpython-users@starship.python.net







\chapter{Porting from wxWindows 1.xx}\label{porting}

This addendum gives guidelines and tips for porting applications from
version 1.xx of wxWindows to version 2.0.

The first section offers tips for writing 1.xx applications in a way to
minimize porting time. The following sections detail the changes and
how you can modify your application to be 2.0-compliant.

You may be worrying that porting to 2.0 will be a lot of work,
particularly if you have only recently started using 1.xx. In fact,
the wxWindows 2.0 API has far more in common with 1.xx than it has differences.
The main challenges are using the new event system, doing without the default
panel item layout, and the lack of automatic labels in some controls.

Please don't be freaked out by the jump to 2.0! For one thing, 1.xx is still available
and will be supported by the user community for some time. And when you have
changed to 2.0, we hope that you will appreciate the benefits in terms
of greater flexibility, better user interface aesthetics, improved C++ conformance,
improved compilation speed, and many other enhancements. The revised architecture
of 2.0 will ensure that wxWindows can continue to evolve for the foreseeable
future.

{\it Please note that this document is a work in progress.}

\section{Preparing for version 2.0}\label{portingpreparing}

Even before compiling with version 2.0, there's also a lot you can do right now to make porting
relatively simple. Here are a few tips.

\begin{itemize}
\item {\bf Use constraints or .wxr resources} for layout, rather than the default layout scheme.
Constraints should be the same in 2.0, and resources will be translated.
\item {\bf Use separate wxMessage items} instead of labels for wxText, wxMultiText,
wxChoice, wxComboBox. These labels will disappear in 2.0. Use separate
wxMessages whether you're creating controls programmatically or using
the dialog editor. The future dialog editor will be able to translate
from old to new more accurately if labels are separated out.
\item {\bf Parameterise functions that use wxDC} or derivatives, i.e. make the wxDC
an argument to all functions that do drawing. Minimise the use of
wxWindow::GetDC and definitely don't store wxDCs long-term
because in 2.0, you can't use GetDC() and wxDCs are not persistent.
You will use wxClientDC, wxPaintDC stack objects instead. Minimising
the use of GetDC() will ensure that there are very few places you
have to change drawing code for 2.0.
\item {\bf Don't set GDI objects} (wxPen, wxBrush etc.) in windows or wxCanvasDCs before they're
needed (e.g. in constructors) - do so within your drawing routine instead. In
2.0, these settings will only take effect between the construction and destruction
of temporary wxClient/PaintDC objects.
\item {\bf Don't rely} on arguments to wxDC functions being floating point - they will
be 32-bit integers in 2.0.
\item {\bf Don't use the wxCanvas member functions} that duplicate wxDC functions, such as SetPen and DrawLine, since
they are going.
\item {\bf Using member callbacks} called from global callback functions will make the transition
easier - see the FAQ
for some notes on using member functions for callbacks. wxWindows 2.0 will banish global
callback functions (and OnMenuCommand), and nearly all event handling will be done by functions taking a single event argument.
So in future you will have code like:

{\small\begin{verbatim}
void MyFrame::OnOK(wxCommandEvent& event)
{
        ...
}
\end{verbatim}
}%

You may find that writing the extra code to call a member function isn't worth it at this stage,
but the option is there.
\item {\bf Use wxString wherever possible.} 2.0 replaces char * with wxString
in most cases, and if you use wxString to receive strings returned from
wxWindows functions (except when you need to save the pointer if deallocation is required), there should
be no conversion problems later on.
\item Be aware that under Windows, {\bf font sizes will change} to match standard Windows
font sizes (for example, a 12-point font will appear bigger than before). Write your application
to be flexible where fonts are concerned.
Don't rely on fonts being similarly-sized across platforms, as they were (by chance) between
Windows and X under wxWindows 1.66. Yes, this is not easy... but I think it is better to conform to the
standards of each platform, and currently the size difference makes it difficult to
conform to Windows UI standards. You may eventually wish to build in a global 'fudge-factor' to compensate
for size differences. The old font sizing will still be available via wx\_setup.h, so do not panic...
\item {\bf Consider dropping wxForm usage}:
wxPropertyFormView can be used in a wxForm-like way, except that you specify a pre-constructed panel
or dialog; or you can use a wxPropertyListView to show attributes in a scrolling list - you don't even need
to lay panel items out.

Because wxForm uses a number of features to be dropped in wxWindows 2.0, it cannot be
supported in the future, at least in its present state.
\item {\bf When creating a wxListBox}, put the wxLB\_SINGLE, wxLB\_MULTIPLE, wxLB\_EXTENDED styles in the window style parameter, and put
zero in the {\it multiple} parameter. The {\it multiple} parameter will be removed in 2.0.
\item {\bf For MDI applications}, don't reply on MDI being run-time-switchable in the way that the
MDI sample is. In wxWindows 2.0, MDI functionality is separated into distinct classes.
\end{itemize}

\section{The new event system}\label{portingeventsystem}

The way that events are handled has been radically changed in wxWindows 2.0. Please
read the topic `Event handling overview' in the wxWindows 2.0 manual for background
on this.

\subsection{Callbacks}

Instead of callbacks for panel items, menu command events, control commands and other events are directed to
the originating window, or an ancestor, or an event handler that has been plugged into the window
or its ancestor. Event handlers always have one argument, a derivative of wxEvent.

For menubar commands, the {\bf OnMenuCommand} member function will be replaced by a series of separate member functions,
each of which responds to a particular command. You need to add these (non-virtual) functions to your
frame class, add a DECLARE\_EVENT\_TABLE entry to the class, and then add an event table to
your implementation file, as a BEGIN\_EVENT\_TABLE and END\_EVENT\_TABLE block. The
individual event mapping macros will be of the form:

\begin{verbatim}
BEGIN_EVENT_TABLE(MyFrame, wxFrame)
    EVT_MENU(MYAPP_NEW, MyFrame::OnNew)
    EVT_MENU(wxID_EXIT, MyFrame::OnExit)
END_EVENT_TABLE()
\end{verbatim}

Control commands, such as button commands, can be routed to a derived button class,
the parent window, or even the frame. Here, you use a function of the form EVT\_BUTTON(id, func).
Similar macros exist for other control commands.

\subsection{Other events}

To intercept other events, you used to override virtual functions, such as OnSize. Now, while you can use
the OnSize name for such event handlers (or any other name of your choice), it has only a single argument
(wxSizeEvent) and must again be `mapped' using the EVT\_SIZE macro. The same goes for all other events,
including OnClose (although in fact you can still use the old, virtual form of OnClose for the time being).

\section{Class hierarchy}\label{portingclasshierarchy}

The class hierarchy has changed somewhat. wxToolBar and wxButtonBar
classes have been split into several classes, and are derived from wxControl (which was
called wxItem). wxPanel derives from wxWindow instead of from wxCanvas, which has
disappeared in favour of wxScrolledWindow (since all windows are now effectively canvases
which can be drawn into). The status bar has become a class in its own right, wxStatusBar.

There are new MDI classes so that wxFrame does not have to be overloaded with this
functionality.

There are new device context classes, with wxPanelDC and wxCanvasDC disappearing.
See \helpref{Device contexts and painting}{portingdc}.

\section{GDI objects}\label{portinggdiobjects}

These objects - instances of classes such as wxPen, wxBrush, wxBitmap (but not wxColour) -
are now implemented with reference-counting. This makes assignment a very cheap operation,
and also means that management of the resource is largely automatic. You now pass {\it references} to
objects to functions such as wxDC::SetPen, not pointers, so you will need to dereference your pointers.
The device context does not store a copy of the pen
itself, but takes a copy of it (via reference counting), and the object's data gets freed up
when the reference count goes to zero. The application does not have to worry so much about
who the object belongs to: it can pass the reference, then destroy the object without
leaving a dangling pointer inside the device context.

For the purposes of code migration, you can use the old style of object management - maintaining
pointers to GDI objects, and using the FindOrCreate... functions. However, it is preferable to
keep this explicit management to a minimum, instead creating objects on the fly as needed, on the stack,
unless this causes too much of an overhead in your application.

At a minimum, you will have to make sure that calls to SetPen, SetBrush etc. work. Also, where you pass NULL to these
functions, you will need to use an identifier such as wxNullPen or wxNullBrush.

\section{Dialogs and controls}\label{portingdialogscontrols}

\wxheading{Labels}

Most controls no longer have labels and values as they used to in 1.xx. Instead, labels
should be created separately using wxStaticText (the new name for wxMessage). This will
need some reworking of dialogs, unfortunately; programmatic dialog creation that doesn't
use constraints will be especially hard-hit. Perhaps take this opportunity to make more
use of dialog resources or constraints. Or consider using the wxPropertyListView class
which can do away with dialog layout issues altogether by presenting a list of editable
properties.

\wxheading{Constructors}

All window constructors have two main changes, apart from the label issue mentioned above.
Windows now have integer identifiers; and position and size are now passed as wxPoint and
wxSize objects. In addition, some windows have a wxValidator argument.

\wxheading{Show versus ShowModal}

If you have used or overridden the {\bf wxDialog::Show} function in the past, you may find
that modal dialogs no longer work as expected. This is because the function for modal showing
is now {\bf wxDialog:ShowModal}. This is part of a more fundamental change in which a
control may tell the dialog that it caused the dismissal of a dialog, by
calling {\bf wxDialog::EndModal} or {\bf wxWindow::SetReturnCode}. Using this
information, {\bf ShowModal} now returns the id of the control that caused dismissal,
giving greater feedback to the application than just TRUE or FALSE.

If you overrode or called {\bf wxDialog::Show}, use {\bf ShowModal} and test for a returned identifier,
commonly wxID\_OK or wxID\_CANCEL.

\wxheading{wxItem}

This is renamed wxControl.

\wxheading{wxText, wxMultiText and wxTextWindow}

These classes no longer exist and are replaced by the single class wxTextCtrl.
Multi-line text items are created using the wxTE\_MULTILINE style.

\wxheading{wxButton}

Bitmap buttons are now a separate class, instead of being part of wxBitmap.

\wxheading{wxMessage}

Bitmap messages are now a separate class, wxStaticBitmap, and wxMessage
is renamed wxStaticText.

\wxheading{wxGroupBox}

wxGroupBox is renamed wxStaticBox.

\wxheading{wxForm}

Note that wxForm is no longer supported in wxWindows 2.0. Consider using the wxPropertyFormView class
instead, which takes standard dialogs and panels and associates controls with property objects.
You may also find that the new validation method, combined with dialog resources, is easier
and more flexible than using wxForm.

\section{Device contexts and painting}\label{portingdc}

In wxWindows 2.0, device contexts are used for drawing into, as per 1.xx, but the way
they are accessed and constructed is a bit different.

You no longer use {\bf GetDC} to access device contexts for panels, dialogs and canvases.
Instead, you create a temporary device context, which means that any window or control can be drawn
into. The sort of device context you create depends on where your code is called from. If
painting within an {\bf OnPaint} handler, you create a wxPaintDC. If not within an {\bf OnPaint} handler,
you use a wxClientDC or wxWindowDC. You can still parameterise your drawing code so that it
doesn't have to worry about what sort of device context to create - it uses the DC it is passed
from other parts of the program.

You {\bf must } create a wxPaintDC if you define an OnPaint handler, even if you do not
actually use this device context, or painting will not work correctly under Windows.

If you used device context functions with wxPoint or wxIntPoint before, please note
that wxPoint now contains integer members, and there is a new class wxRealPoint. wxIntPoint
no longer exists.

wxMetaFile and wxMetaFileDC have been renamed to wxMetafile and wxMetafileDC.

\section{Miscellaneous}

\subsection{Strings}

wxString has replaced char* in the majority of cases. For passing strings into functions,
this should not normally require you to change your code if the syntax is otherwise the
same. This is because C++ will automatically convert a char* or const char* to a wxString by virtue
of appropriate wxString constructors.

However, when a wxString is returned from a function in wxWindows 2.0 where a char* was
returned in wxWindows 1.xx, your application will need to be changed. Usually you can
simplify your application's allocation and deallocation of memory for the returned string,
and simply assign the result to a wxString object. For example, replace this:

{\small\begin{verbatim}
  char* s = wxFunctionThatReturnsString();
  s = copystring(s); // Take a copy in case it is temporary
  .... // Do something with it
  delete[] s;
\end{verbatim}
}

with this:

{\small\begin{verbatim}
  wxString s = wxFunctionThatReturnsString();
  .... // Do something with it
\end{verbatim}
}

To indicate an empty return value or a problem, a function may return either the
empty string (``") or a null string. You can check for a null string with wxString::IsNull().

\subsection{Use of const}

The {\bf const} keyword is now used to denote constant functions that do not affect the
object, and for function arguments to denote that the object passed cannot be changed.

This should not affect your application except for where you are overriding virtual functions
which now have a different signature. If functions are not being called which were previously,
check whether there is a parameter mismatch (or function type mismatch) involving consts.

Try to use the {\bf const} keyword in your own code where possible.

\section{Backward compatibility}\label{portingcompat}

Some wxWindows 1.xx functionality has been left to ease the transition to 2.0. This functionality
(usually) only works if you compile with WXWIN\_COMPATIBILITY set to 1 in setup.h.

Mostly this defines old names to be the new names (e.g. wxRectangle is defined to be wxRect).

\section{Quick reference}\label{portingquickreference}

This section allows you to quickly find features that
need to be converted.

\subsection{Include files}

Use the form:

\begin{verbatim}
#include <wx/wx.h>
#include <wx/button.h>
\end{verbatim}

For precompiled header support, use this form:

\begin{verbatim}
// For compilers that support precompilation, includes "wx.h".
#include <wx/wxprec.h>

#ifdef __BORLANDC__
    #pragma hdrstop
#endif

// Any files you want to include if not precompiling by including
// the whole of <wx/wx.h>
#ifndef WX_PRECOMP
    #include <stdio.h>
    #include <wx/setup.h>
    #include <wx/bitmap.h>
    #include <wx/brush.h>
#endif

// Any files you want to include regardless of precompiled headers
#include <wx/toolbar.h>
\end{verbatim}

\subsection{IPC classes}

These are now separated out into wxDDEServer/Client/Connection (Windows only) and wxTCPServer/Client/Connection
(Windows and Unix). Take care to use wxString for your overridden function arguments, instead of char*, as per
the documentation.

\subsection{MDI style frames}

MDI is now implemented as a family of separate classes, so you can't switch to MDI just by
using a different frame style. Please see the documentation for the MDI frame classes, and the MDI
sample may be helpful too.

\subsection{OnActivate}

Replace the arguments with one wxActivateEvent\& argument, make sure the function isn't virtual,
and add an EVT\_ACTIVATE event table entry.

\subsection{OnChar}

This is now a non-virtual function, with the same wxKeyEvent\& argument as before.
Add an EVT\_CHAR macro to the event table
for your window, and the implementation of your function will need very few changes.

\subsection{OnClose}

The old virtual function OnClose is now obsolete.
Add an OnCloseWindow event handler using an EVT\_CLOSE event table entry. For details
about window destruction, see the Windows Deletion Overview in the manual. This is a subtle
topic so please read it very carefully. Basically, OnCloseWindow is now responsible for
destroying a window with Destroy(), but the default implementation (for example for wxDialog) may not
destroy the window, so to be sure, always provide this event handler so it is obvious what's going on.

\subsection{OnEvent}

This is now a non-virtual function, with the same wxMouseEvent\& argument as before. However
you may wish to rename it OnMouseEvent. Add an EVT\_MOUSE\_EVENTS macro to the event table
for your window, and the implementation of your function will need very few changes.
However, if you wish to intercept different events using different functions, you can
specify specific events in your event table, such as EVT\_LEFT\_DOWN.

Your OnEvent function is likely to have references to GetDC(), so make sure you create
a wxClientDC instead. See \helpref{Device contexts}{portingdc}.

If you are using a wxScrolledWindow (formerly wxCanvas), you should call
PrepareDC(dc) to set the correct translation for the current scroll position.

\subsection{OnMenuCommand}

You need to replace this virtual function with a series of non-virtual functions, one for
each case of your old switch statement. Each function takes a wxCommandEvent\& argument.
Create an event table for your frame
containing EVT\_MENU macros, and insert DECLARE\_EVENT\_TABLE() in your frame class, as
per the samples.

\subsection{OnPaint}

This is now a non-virtual function, with a wxPaintEvent\& argument.
Add an EVT\_PAINT macro to the event table
for your window.

Your function {\it must} create a wxPaintDC object, instead of using GetDC to
obtain the device context.

If you are using a wxScrolledWindow (formerly wxCanvas), you should call
PrepareDC(dc) to set the correct translation for the current scroll position.

\subsection{OnSize}

Replace the arguments with one wxSizeEvent\& argument, make it non-virtual, and add to your
event table using EVT\_SIZE.

\subsection{wxApp definition}

The definition of OnInit has changed. Return a bool value, not a wxFrame.

Also, do {\it not} declare a global application object. Instead, use the macros
DECLARE\_APP and IMPLEMENT\_APP as per the samples. Remove any occurrences of IMPLEMENT\_WXWIN\_MAIN:
this is subsumed in IMPLEMENT\_APP.

\subsection{wxButton}

For bitmap buttons, use wxBitmapButton.

\subsection{wxCanvas}

Change the name to wxScrolledWindow.

\subsection{wxDialogBox}

Change the name to wxDialog, and for modal dialogs, use ShowModal instead of Show.

\subsection{wxDialog::Show}

If you used {\bf Show} to show a modal dialog or to override the standard
modal dialog {\bf Show}, use {\bf ShowModal} instead.

\wxheading{See also}

\helpref{Dialogs and controls}{portingdialogscontrols}

\subsection{wxForm}

Sorry, this class is no longer available. Try using the wxPropertyListView or wxPropertyFormView class
instead, or use .wxr files and validators.

\subsection{wxPoint}

The old wxPoint is called wxRealPoint, and wxPoint now uses integers.

\subsection{wxRectangle}

This is now called wxRect.

\subsection{wxScrollBar}

The function names have changed for this class: please refer to the documentation for wxScrollBar. Instead
of setting properties individually, you will call SetScrollbar with several parameters.

\subsection{wxText, wxMultiText, wxTextWindow}

Change all these to wxTextCtrl. Add the window style wxTE\_MULTILINE if you
wish to have a multi-line text control.

\subsection{wxToolBar}

This name is an alias for the most popular form of toolbar for your platform. There is now a family
of toolbar classes, with for example wxToolBar95, wxToolBarMSW and wxToolBarSimple classes existing
under Windows 95.

Toolbar management is supported by frames, so calling wxFrame::CreateToolBar and adding tools is usually
enough, and the SDI or MDI frame will manage the positioning for you. The client area of the frame is the space
left over when the menu bar, toolbar and status bar have been taken into account.



\begin{comment}
\newpage

% Puts books in the bibliography without needing to cite them in the
% text
\nocite{helpbook}%
\nocite{wong93}%
\nocite{pree94}%
\nocite{gamma95}%
\nocite{smart95a}%
\nocite{smart95b}%

\end{comment}

\bibliography{refs}
\addcontentsline{toc}{chapter}{Bibliography}
\setheader{{\it REFERENCES}}{}{}{}{}{{\it REFERENCES}}%
\setfooter{\thepage}{}{}{}{}{\thepage}%

\newpage
% Note: In RTF, the \printindex must come before the
% change of header/footer, since the \printindex inserts
% the RTF \sect command which divides one chapter from
% the next.
\rtfonly{\printindex
\addcontentsline{toc}{chapter}{Index}
\setheader{{\it INDEX}}{}{}{}{}{{\it INDEX}}%
\setfooter{\thepage}{}{}{}{}{\thepage}
}
% In Latex, it must be this way around (I think)
\latexonly{\addcontentsline{toc}{chapter}{Index}
\setheader{{\it INDEX}}{}{}{}{}{{\it INDEX}}%
\setfooter{\thepage}{}{}{}{}{\thepage}
\printindex
}

\end{document}
