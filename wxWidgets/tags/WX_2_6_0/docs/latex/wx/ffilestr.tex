% -----------------------------------------------------------------------------
% wxFFileInputStream
% -----------------------------------------------------------------------------
\section{\class{wxFFileInputStream}}\label{wxffileinputstream}

This class represents data read in from a file. There are actually
two such groups of classes: this one is based on \helpref{wxFFile}{wxffile} 
whereas \helpref{wxFileInputStream}{wxfileinputstream} is based in
the \helpref{wxFile}{wxfile} class.

Note that \helpref{wxFile}{wxfile} and \helpref{wxFFile}{wxffile} differ
in one aspect, namely when to report that the end of the file has been
reached. This is documented in \helpref{wxFile::Eof}{wxfileeof} and 
\helpref{wxFFile::Eof}{wxffileeof} and the behaviour of the stream
classes reflects this difference, i.e. wxFileInputStream will report
wxSTREAM\_EOF after having read the last byte whereas wxFFileInputStream
will report wxSTREAM\_EOF after trying to read {\it past} the last byte.
Related to EOF behavior, note that \helpref{SeekI()}{wxinputstreamseeki} 
can seek beyond the end of the stream (file) and will thus not return 
{\it wxInvalidOffset} for that.

\wxheading{Derived from}

\helpref{wxInputStream}{wxinputstream}

\wxheading{Include files}

<wx/wfstream.h>

\wxheading{See also}

\helpref{wxBufferedInputStream}{wxbufferedinputstream}, \helpref{wxFFileOutputStream}{wxffileoutputstream}, \helpref{wxFileOutputStream}{wxfileoutputstream}

% ----------
% Members
% ----------
\latexignore{\rtfignore{\wxheading{Members}}}

\membersection{wxFFileInputStream::wxFFileInputStream}\label{wxffileinputstreamctor}

\func{}{wxFFileInputStream}{\param{const wxString\&}{ filename}, \param{const wxChar *}{ mode = "rb"}}

Opens the specified file using its {\it filename} name using the specified mode.

\func{}{wxFFileInputStream}{\param{wxFFile\&}{ file}}

Initializes a file stream in read-only mode using the file I/O object {\it file}.

\func{}{wxFFileInputStream}{\param{FILE *}{ fp}}

Initializes a file stream in read-only mode using the specified file pointer {\it fp}.

\membersection{wxFFileInputStream::\destruct{wxFFileInputStream}}\label{wxffileinputstreamdtor}

\func{}{\destruct{wxFFileInputStream}}{\void}

Destructor.

\membersection{wxFFileInputStream::Ok}\label{wxffileinputstreamok}

\constfunc{bool}{Ok}{\void}

Returns true if the stream is initialized and ready.

% -----------------------------------------------------------------------------
% wxFFileOutputStream
% -----------------------------------------------------------------------------
\section{\class{wxFFileOutputStream}}\label{wxffileoutputstream}

This class represents data written to a file. There are actually
two such groups of classes: this one is based on \helpref{wxFFile}{wxffile} 
whereas \helpref{wxFileInputStream}{wxffileinputstream} is based in
the \helpref{wxFile}{wxfile} class.

Note that \helpref{wxFile}{wxfile} and \helpref{wxFFile}{wxffile} differ
in one aspect, namely when to report that the end of the file has been
reached. This is documented in \helpref{wxFile::Eof}{wxfileeof} and 
\helpref{wxFFile::Eof}{wxffileeof} and the behaviour of the stream
classes reflects this difference, i.e. wxFileInputStream will report
wxSTREAM\_EOF after having read the last byte whereas wxFFileInputStream
will report wxSTREAM\_EOF after trying to read {\it past} the last byte.
Related to EOF behavior, note that \helpref{SeekO()}{wxoutputstreamseeko} 
can seek beyond the end of the stream (file) and will thus not return 
{\it wxInvalidOffset} for that.

\wxheading{Derived from}

\helpref{wxOutputStream}{wxoutputstream}

\wxheading{Include files}

<wx/wfstream.h>

\wxheading{See also}

\helpref{wxBufferedOutputStream}{wxbufferedoutputstream}, \helpref{wxFFileInputStream}{wxffileinputstream}, \helpref{wxFileInputStream}{wxfileinputstream}

% ----------
% Members
% ----------
\latexignore{\rtfignore{\wxheading{Members}}}

\membersection{wxFFileOutputStream::wxFFileOutputStream}\label{wxffileoutputstreamctor}

\func{}{wxFFileOutputStream}{\param{const wxString\&}{ filename}, \param{const wxChar *}{ mode="w+b"}}

Opens the file with the given {\it filename} name in the specified mode.

\func{}{wxFFileOutputStream}{\param{wxFFile\&}{ file}}

Initializes a file stream in write-only mode using the file I/O object {\it file}.

\func{}{wxFFileOutputStream}{\param{FILE *}{ fp}}

Initializes a file stream in write-only mode using the file descriptor {\it fp}.

\membersection{wxFFileOutputStream::\destruct{wxFFileOutputStream}}\label{wxffileoutputstreamdtor}

\func{}{\destruct{wxFFileOutputStream}}{\void}

Destructor.

\membersection{wxFFileOutputStream::Ok}\label{wxffileoutputstreamok}

\constfunc{bool}{Ok}{\void}

Returns true if the stream is initialized and ready.

% -----------------------------------------------------------------------------
% wxFFileStream
% -----------------------------------------------------------------------------
\section{\class{wxFFileStream}}\label{wxffilestream}

\wxheading{Derived from}

\helpref{wxFFileOutputStream}{wxffileoutputstream}, \helpref{wxFFileInputStream}{wxffileinputstream}

\wxheading{Include files}

<wx/wfstream.h>

\wxheading{See also}

\helpref{wxStreamBuffer}{wxstreambuffer}

\latexignore{\rtfignore{\wxheading{Members}}}

\membersection{wxFFileStream::wxFFileStream}\label{wxffilestreamctor}

\func{}{wxFFileStream}{\param{const wxString\&}{ iofileName}}

Initializes a new file stream in read-write mode using the specified 
{\it iofilename} name.


