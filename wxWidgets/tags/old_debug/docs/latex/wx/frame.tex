\section{\class{wxFrame}}\label{wxframe}

A frame is a window whose size and position can (usually) be changed by the user. It usually has
thick borders and a title bar, and can optionally contain a menu bar, toolbar and
status bar. A frame can contain any window that is not a frame or dialog.

\wxheading{Derived from}

\helpref{wxWindow}{wxwindow}\\
\helpref{wxEvtHandler}{wxevthandler}\\
\helpref{wxObject}{wxobject}

\wxheading{Window styles}

\twocolwidtha{5cm}
\begin{twocollist}\itemsep=0pt
\twocolitem{\windowstyle{wxICONIZE}}{Display the frame iconized (minimized) (Windows only).}
\twocolitem{\windowstyle{wxCAPTION}}{Puts a caption on the frame.}
\twocolitem{\windowstyle{wxDEFAULT\_FRAME\_STYLE}}{Defined as {\bf wxMINIMIZE\_BOX \pipe wxMAXIMIZE\_BOX \pipe wxTHICK\_FRAME \pipe wxSYSTEM\_MENU \pipe wxCAPTION}.}
\twocolitem{\windowstyle{wxMINIMIZE}}{Identical to {\bf wxICONIZE}.}
\twocolitem{\windowstyle{wxMINIMIZE\_BOX}}{Displays a minimize box on the frame (Windows and Motif only).}
\twocolitem{\windowstyle{wxMAXIMIZE}}{Displays the frame maximized (Windows only).}
\twocolitem{\windowstyle{wxMAXIMIZE\_BOX}}{Displays a maximize box on the frame (Windows and Motif only).}
\twocolitem{\windowstyle{wxSTAY\_ON\_TOP}}{Stay on top of other windows (Windows only).}
\twocolitem{\windowstyle{wxSYSTEM\_MENU}}{Displays a system menu (Windows and Motif only).}
\twocolitem{\windowstyle{wxTHICK\_FRAME}}{Displays a thick frame around the window (Windows and Motif only).}
\twocolitem{\windowstyle{wxRESIZE\_BORDER}}{Displays a resizeable border around the window (Motif only).}
\end{twocollist}

See also \helpref{window styles overview}{windowstyles}.

\wxheading{Remarks}

An application should normally define an \helpref{OnCloseWindow}{wxwindowonclosewindow} handler for the
frame to respond to system close events, for example so that related data and subwindows can be cleaned up.

\wxheading{See also}

\helpref{wxMDIParentFrame}{wxmdiparentframe}, \helpref{wxMDIChildFrame}{wxmdichildframe},\rtfsp
\helpref{wxMiniFrame}{wxminiframe}, \helpref{wxDialog}{wxdialog}

\latexignore{\rtfignore{\wxheading{Members}}}

\membersection{wxFrame::wxFrame}\label{wxframeconstr}

\func{}{wxFrame}{\void}

Default constructor.

\func{}{wxFrame}{\param{wxWindow* }{parent}, \param{wxWindowID }{id},\rtfsp
\param{const wxString\& }{title}, \param{const wxPoint\&}{ pos = wxDefaultPosition},\rtfsp
\param{const wxSize\&}{ size = wxDefaultSize}, \param{long}{ style = wxDEFAULT\_FRAME\_STYLE},\rtfsp
\param{const wxString\& }{name = ``frame"}}

Constructor, creating the window.

\wxheading{Parameters}

\docparam{parent}{The window parent. This may be NULL. If it is non-NULL, the frame will
always be displayed on top of the parent window on Windows.}

\docparam{id}{The window identifier. It may take a value of -1 to indicate a default value.}

\docparam{title}{The caption to be displayed on the frame's title bar.}

\docparam{pos}{The window position. A value of (-1, -1) indicates a default position, chosen by
either the windowing system or wxWindows, depending on platform.}

\docparam{size}{The window size. A value of (-1, -1) indicates a default size, chosen by
either the windowing system or wxWindows, depending on platform.}

\docparam{style}{The window style. See \helpref{wxFrame}{wxframe}.}

\docparam{name}{The name of the window. This parameter is used to associate a name with the item,
allowing the application user to set Motif resource values for
individual windows.}

\wxheading{Remarks}

For Motif, MWM (the Motif Window Manager) should be running for any window styles to work
(otherwise all styles take effect).

\wxheading{See also}

\helpref{wxFrame::Create}{wxframecreate}

\membersection{wxFrame::\destruct{wxFrame}}

\func{void}{\destruct{wxFrame}}{\void}

Destructor. Destroys all child windows and menu bar if present.

\membersection{wxFrame::Centre}\label{wxframecentre}

\func{void}{Centre}{\param{int}{ direction = wxBOTH}}

Centres the frame on the display.

\wxheading{Parameters}

\docparam{direction}{The parameter may be {\tt wxHORIZONTAL}, {\tt wxVERTICAL} or {\tt wxBOTH}.}

\membersection{wxFrame::Command}\label{wxframecommand}

\func{void}{Command}{\param{int }{id}}

Simulate a menu command.

\wxheading{Parameters}

\docparam{id}{The identifier for a menu item.}

\membersection{wxFrame::Create}\label{wxframecreate}

\func{bool}{Create}{\param{wxWindow* }{parent}, \param{wxWindowID }{id},\rtfsp
\param{const wxString\& }{title}, \param{const wxPoint\&}{ pos = wxDefaultPosition},\rtfsp
\param{const wxSize\&}{ size = wxDefaultSize}, \param{long}{ style = wxDEFAULT\_FRAME\_STYLE},\rtfsp
\param{const wxString\& }{name = ``frame"}}

Used in two-step frame construction. See \helpref{wxFrame::wxFrame}{wxframeconstr}\rtfsp
for further details.

\membersection{wxFrame::CreateStatusBar}\label{wxframecreatestatusbar}

\func{virtual bool}{CreateStatusBar}{\param{int}{ number = 1}}

Creates a status bar at the bottom of the frame.

\wxheading{Parameters}

\docparam{number}{The number of fields to create. Specify a
value greater than 1 to create a multi-field status bar.}

\wxheading{Return value}

TRUE if the status bar was created successfully.

\wxheading{Remarks}

The width of the status bar is the whole width of the frame (adjusted automatically when
resizing), and the height and text size are chosen by the host windowing system.

By default, the status bar is an instance of wxStatusBar. To use a different class,
override \helpref{wxFrame::OnCreateStatusBar}{wxframeoncreatestatusbar}.

Note that you can put controls and other windows on the status bar if you wish.

\wxheading{See also}

\helpref{wxFrame::SetStatusText}{wxframesetstatustext},\rtfsp
\helpref{wxFrame::OnCreateStatusBar}{wxframeoncreatestatusbar},\rtfsp
\helpref{wxFrame::GetStatusBar}{wxframegetstatusbar}

\membersection{wxFrame::GetMenuBar}\label{wxframegetmenubar}

\constfunc{wxMenuBar*}{GetMenuBar}{\void}

Returns a pointer to the menubar currently associated with the frame (if any).

\wxheading{See also}

\helpref{wxFrame::SetMenuBar}{wxframesetmenubar}, \helpref{wxMenuBar}{wxmenubar}, \helpref{wxMenu}{wxmenu}

\membersection{wxFrame::GetStatusBar}\label{wxframegetstatusbar}

\func{wxStatusBar*}{GetStatusBar}{\void}

Returns a pointer to the status bar currently associated with the frame (if any).

\wxheading{See also}

\helpref{wxFrame::CreateStatusBar}{wxframecreatestatusbar}, \helpref{wxStatusBar}{wxstatusbar}

\membersection{wxFrame::GetTitle}\label{wxframegettitle}

\func{wxString\&}{GetTitle}{\void}

Gets a temporary pointer to the frame title. See
\helpref{wxFrame::SetTitle}{wxframesettitle}.

\membersection{wxFrame::Iconize}\label{wxframeiconize}

\func{void}{Iconize}{\param{const bool}{ iconize}}

Iconizes or restores the frame.

\wxheading{Parameters}

\docparam{izonize}{If TRUE, iconizes the frame; if FALSE, shows and restores it.}

\wxheading{See also}

\helpref{wxFrame::IsIconized}{wxframeisiconized}, \helpref{wxFrame::Maximize}{wxframemaximize}.

\membersection{wxFrame::IsIconized}\label{wxframeisiconized}

\func{bool}{IsIconized}{\void}

Returns TRUE if the frame is iconized.

\membersection{wxFrame::LoadAccelerators}\label{wxframeloadaccelerators}

\func{void}{LoadAccelerators}{\param{const wxString\& }{table}}

Loads a keyboard accelerator table for this frame.

\wxheading{Parameters}

\docparam{table}{Accelerator table to load.}

\wxheading{Return value}

TRUE if the operation was successful, FALSE otherwise.

\wxheading{Remarks}

Accelerator tables map keystrokes onto control and menu identifiers, so the
programmer does not have to explicitly program this correspondence.

See the hello demo ({\tt hello.cpp} and {\tt hello.rc}) for
an example of accelerator usage. This is a fragment from {\tt hello.rc}:

\begin{verbatim}
#define HELLO_LOAD_FILE  111

menus_accel ACCELERATORS
{

"^L", HELLO_LOAD_FILE

}
\end{verbatim}

This function only works under Windows.

% huh? If you call LoadAccelerators, you need to override wxFrame::OnActivate to do nothing.

\membersection{wxFrame::Maximize}\label{wxframemaximize}

\func{void}{Maximize}{\param{const bool }{maximize}}

Maximizes or restores the frame.

\wxheading{Parameters}

\docparam{maximize}{If TRUE, maximizes the frame, otherwise it restores it}.

\wxheading{Remarks}

This function only works under Windows.

\wxheading{See also}

\helpref{wxFrame::Iconize}{wxframeiconize}

\membersection{wxFrame::OnActivate}

\func{void}{OnActivate}{\param{bool}{ active}}

Called when a window is activated or deactivated (MS Windows
only). If the window is being activated, {\it active} is TRUE, else it
is FALSE.

If you call wxFrame::LoadAccelerators, you need to override this function e.g.

\begin{verbatim}
   void OnActivate(bool) {};
\end{verbatim}

\membersection{wxFrame::OnCreateStatusBar}\label{wxframeoncreatestatusbar}

\func{virtual wxStatusBar*}{OnCreateStatusBar}{\param{int }{number}}

Virtual function called when a status bar is requested by \helpref{wxFrame::CreateStatusBar}{wxframecreatestatusbar}.

\wxheading{Parameters}

\docparam{number}{The number of fields to create.}

\wxheading{Return value}

A status bar object.

\wxheading{Remarks}

An application can override this function to return a different kind of status bar. The default
implementation returns an instance of \helpref{wxStatusBar}{wxstatusbar}.

\wxheading{See also}

\helpref{wxFrame::CreateStatusBar}{wxframecreatestatusbar}, \helpref{wxStatusBar}{wxstatusbar}.

\membersection{wxFrame::OnMenuCommand}\label{wxframeonmenucommand}

\func{void}{OnMenuCommand}{\param{wxCommandEvent\&}{ event}}

See \helpref{wxWindow::OnMenuCommand}{wxwindowonmenucommand}.

\membersection{wxFrame::OnMenuHighlight}\label{wxframeonmenuhighlight}

\func{void}{OnMenuHighlight}{\param{wxMenuEvent\&}{ event}}

See \helpref{wxWindow::OnMenuHighlight}{wxwindowonmenuhighlight}.

\membersection{wxFrame::OnSize}\label{wxframeonsize}

\func{void}{OnSize}{\param{wxSizeEvent\& }{event}}

See \helpref{wxWindow::OnSize}{wxwindowonsize}.

The default {\bf wxFrame::OnSize} implementation looks for a single subwindow,
and if one is found, resizes it to fit
inside the frame. Override this member if more complex behaviour
is required (for example, if there are several subwindows).

\membersection{wxFrame::SetIcon}\label{wxframeseticon}

\func{void}{SetIcon}{\param{const wxIcon\& }{icon}}

Sets the icon for this frame.

\wxheading{Parameters}

\docparam{icon}{The icon to associate with this frame.}

\wxheading{Remarks}

The frame takes a `copy' of {\it icon}, but since it uses reference
counting, the copy is very quick. It is safe to delete {\it icon} after
calling this function.

Under Windows, instead of using {\bf SetIcon}, you can add the
following lines to your MS Windows resource file:

\begin{verbatim}
wxSTD_MDIPARENTFRAME ICON icon1.ico
wxSTD_MDICHILDFRAME  ICON icon2.ico
wxSTD_FRAME          ICON icon3.ico
\end{verbatim}

where icon1.ico will be used for the MDI parent frame, icon2.ico
will be used for MDI child frames, and icon3.ico will be used for
non-MDI frames.

If these icons are not supplied, and {\bf SetIcon} is not called either,
then the following defaults apply if you have included wx.rc.

\begin{verbatim}
wxDEFAULT_FRAME               ICON std.ico
wxDEFAULT_MDIPARENTFRAME      ICON mdi.ico
wxDEFAULT_MDICHILDFRAME       ICON child.ico
\end{verbatim}

You can replace std.ico, mdi.ico and child.ico with your own defaults
for all your wxWindows application. Currently they show the same icon.

{\it Note:} a wxWindows application linked with subsystem equal to 4.0
(i.e. marked as a Windows 95 application) doesn't respond properly
to wxFrame::SetIcon. To work around this until a solution is found,
mark your program as a 3.5 application. This will also ensure
that Windows provides small icons for the application automatically.

See also \helpref{wxIcon}{wxicon}.

\membersection{wxFrame::SetMenuBar}\label{wxframesetmenubar}

\func{void}{SetMenuBar}{\param{wxMenuBar* }{menuBar}}

Tells the frame to show the given menu bar.

\wxheading{Parameters}

\docparam{menuBar}{The menu bar to associate with the frame.}

\wxheading{Remarks}

If the frame is destroyed, the
menu bar and its menus will be destroyed also, so do not delete the menu
bar explicitly (except by resetting the frame's menu bar to another
frame or NULL).

Under Windows, a call to \helpref{wxFrame::OnSize}{wxframeonsize} is generated, so be sure to initialize
data members properly before calling {\bf SetMenuBar}.

Note that it is not possible to call this function twice for the same frame object.

\wxheading{See also}

\helpref{wxFrame::GetMenuBar}{wxframegetmenubar}, \helpref{wxMenuBar}{wxmenubar}, \helpref{wxMenu}{wxmenu}.

\membersection{wxFrame::SetStatusText}\label{wxframesetstatustext}

\func{virtual void}{SetStatusText}{\param{const wxString\& }{ text}, \param{int}{ number = 0}}

Sets the status bar text and redraws the status bar.

\wxheading{Parameters}

\docparam{text}{The text for the status field.}

\docparam{number}{The status field (starting from zero).}

\wxheading{Remarks}

Use an empty string to clear the status bar.

\wxheading{See also}

\helpref{wxFrame::CreateStatusBar}{wxframecreatestatusbar}, \helpref{wxStatusBar}{wxstatusbar}

\membersection{wxFrame::SetStatusWidths}\label{wxframesetstatuswidths}

\func{virtual void}{SetStatusWidths}{\param{int}{ n}, \param{int *}{widths}}

Sets the widths of the fields in the status bar.

\wxheading{Parameters}

\wxheading{n}{The number of fields in the status bar. It must be the
same used in \helpref{CreateStatusBar}{wxframecreatestatusbar}.}

\docparam{widths}{Must contain an array of {\it n} integers, each of which is a status field width
in pixels. A value of -1 indicates that the field is variable width; at least one
field must be -1. You should delete this array after calling {\bf SetStatusWidths}.}

\wxheading{Remarks}

The widths of the variable fields are calculated from the total width of all fields,
minus the sum of widths of the non-variable fields, divided by the number of 
variable fields.

\membersection{wxFrame::SetTitle}\label{wxframesettitle}

\func{virtual void}{SetTitle}{\param{const wxString\& }{ title}}

Sets the frame title.

\wxheading{Parameters}

\docparam{title}{The frame title.}

\wxheading{See also}

\helpref{wxFrame::GetTitle}{wxframegettitle}

