\section{Bitmaps overview}\label{wxbitmapoverview}

Classes: \helpref{wxBitmap}{wxbitmap}, \helpref{wxBitmapHandler}{wxbitmaphandler}, \helpref{wxIcon}{wxicon}, \helpref{wxCursor}{wxcursor}.

The wxBitmap class encapsulates the concept of a platform-dependent bitmap,
either monochrome or colour. Platform-specific methods for creating a
wxBitmap object from an existing file are catered for, and
this is an occasion where conditional compilation will sometimes be
required.

A bitmap created dynamically or loaded from a file can be selected
into a memory device context (instance of \helpref{wxMemoryDC}{wxmemorydc}). This
enables the bitmap to be copied to a window or memory device context
using \helpref{wxDC::Blit}{wxdcblit}, or to be used as a drawing surface.  The {\bf
wxToolBarSimple} class is implemented using bitmaps, and the toolbar demo
shows one of the toolbar bitmaps being used for drawing a miniature
version of the graphic which appears on the main window.

See \helpref{wxMemoryDC}{wxmemorydc} for an example of drawing onto a bitmap.

The following shows the conditional compilation required to load a
bitmap in X and in Windows 3. The alternative is to use the string
version of the bitmap constructor, which loads a file under X and a
resource under Windows 3, but has the disadvantage of requiring the
X icon file to be available at run-time.

\begin{verbatim}
#ifdef wx_x
#include "aiai.xbm"
#endif
#ifdef wx_msw
  wxIcon *icon = new wxBitmap("aiai");
#endif
#ifdef wx_x
  wxIcon *icon = new wxBitmap(aiai_bits, aiai_width, aiai_height);
#endif
\end{verbatim}

\subsection{Loading bitmaps: further information}

There is provision for a number of bitmap
formats via the standard wxBitmap class. These facilities can
be enabled or disabled using settings in wx\_setup.h.

XPM colour pixmaps may be loaded and saved under Windows and X, with
some restrictions imposed by the lack of colourmap facility when
using XPM files. The user may elect to use XPM files as a cross-platform
stabdard, or translate between XPM and BMP files using a suitable
utility.

Also, under Windows, DIBs (device independent bitmaps with extension BMP)
may be dynamically loaded and saved. Under X, GIF and BMP files may be
loaded but not saved.

\subsection{Bitmap format handlers}

To provide extensibility, the functionality for loading and saving bitmap formats
is not implemented in the wxBitmap class, but in a number of handler classes,
derived from wxBitmapHandler. There is a static list of handlers which wxBitmap
examines when a file load/save operation is requested. Some handlers are provided as standard, but if you
have special requirements, you may wish to initialise the wxBitmap class with
some extra handlers which you write yourself or receive from a third party.

To add a handler object to wxBitmap, your application needs to include the header which implements it, and
then call the static function \helpref{wxBitmap::AddHandler}{wxbitmapaddhandler}. For example:

{\small
\begin{verbatim}
  #include "JPEGBitmapHandler.h"
  ...
  // Initialisation
  wxBitmap::AddHandler(new wxJPEGBitmapHandler);
  ...
\end{verbatim}
}

Assuming wxJPEGBitmapHandler has been written correctly, you should now be able to load and save JPEG files
using the usual wxBitmap API.

To see how bitmap handlers are implemented, please look at the files {\tt bitmap.h} and {\tt bitmap.cpp}.

\subsection{wxIcon overview}\label{wxiconoverview}

TODO.

