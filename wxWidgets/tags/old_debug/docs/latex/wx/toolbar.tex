\section{\class{wxToolBarBase}}\label{wxtoolbarbase}

{\bf wxToolBarBase} is the base class for a number of toolbar classes.  The most portable
one of these is the generic \helpref{wxToolBarSimple}{wxtoolbarsimple} class. {\bf wxToolBarBase} defines
automatic scrolling management functionality which is identical to \helpref{wxScrolledWindow}{wxscrolledwindow},
so please refer to this class also.

\wxheading{Derived from}

\helpref{wxControl}{wxcontrol}\\
\helpref{wxWindow}{wxwindow}\\
\helpref{wxEvtHandler}{wxevthandler}\\
\helpref{wxObject}{wxobject}

\wxheading{Remarks}

Because there is a variety of toolbar classes, you may wish to choose which class
is best for your application on each platform, and define {\bf wxToolBar} appropriately. For example:

\begin{verbatim}
#if WIN95
class wxToolBar: public wxToolBar95
#elif defined(wx_msw)
class wxToolBar: public wxToolBarMSW
#else
class wxToolBar: public wxToolBarSimple
#endif
{
};
\end{verbatim}

TODO: maybe change the confusing names: GetDefaultSize becomes GetToolBitmapSize, and
GetDefaultButtonSize becomes GetToolSize. Document SetRows for wxToolBar95, and make it
part of the base API?

\wxheading{Event handling}

Derive a new class from an existing toolbar class, and override appropriate virtual functions.

TODO: make consistent with other event handling; have wxToolBarEvent and appropriate macros.

\wxheading{See also}

\overview{Toolbar overview}{wxtoolbaroverview},\rtfsp
\helpref{wxToolBarSimple}{wxtoolbarsimple},\rtfsp
\helpref{wxToolBarMSW}{wxtoolbarmsw},\rtfsp
\helpref{wxToolBar95}{wxtoolbar95},\rtfsp
\helpref{wxScrolledWindow}{wxscrolledwindow}

\latexignore{\rtfignore{\wxheading{Members}}}

\membersection{wxToolBarBase::wxToolBarBase}\label{wxtoolbarbaseconstr}

\func{}{wxToolBarBase}{\void}

Default constructor.

%\wxheading{See also}
%
%\helpref{wxToolBarBase::Create}{wxtoolbarbasecreate}
%
\membersection{wxToolBarBase::\destruct{wxToolBarBase}}

\func{void}{\destruct{wxToolBarBase}}{\void}

Toolbar destructor.

\membersection{wxToolBarBase::AddSeparator}\label{wxtoolbarbaseaddseparator}

\func{void}{AddSeparator}{\void}

Adds a separator for spacing groups of tools.

\wxheading{See also}

\helpref{wxToolBarBase::AddTool}{wxtoolbarbaseaddtool}, \helpref{wxToolBarBase::SetToolSeparation}{wxtoolbarbasesettoolseparation}

\membersection{wxToolBarBase::AddTool}\label{wxtoolbarbaseaddtool}

\func{wxToolBarBaseTool*}{AddTool}{\param{int}{ toolIndex}, \param{const wxBitmap\&}{ bitmap1},\rtfsp
\param{const wxBitmap\&}{ bitmap2 = (wxBitmap *)NULL}, \param{const bool}{ isToggle = FALSE},\rtfsp
\param{const float}{ xPos = -1}, \param{const float}{ yPos = -1},\rtfsp
\param{wxObject *}{clientData = NULL}, \param{const wxString\& }{shortHelpString = ""}, \param{const wxString\& }{longHelpString = ""}}

Adds a tool to the toolbar.

\wxheading{Parameters}

\docparam{toolIndex}{An integer by which
the tool may be identified in subsequent operations.}

\docparam{isToggle}{Specifies whether the tool is a toggle or not: a toggle tool may be in
two states, whereas a non-toggle tool is just a button.}

\docparam{bitmap1}{The primary tool bitmap for toggle and button tools.}

\docparam{bitmap2}{The second bitmap specifies the on-state bitmap for a toggle
tool. If this is NULL, either an inverted version of the primary bitmap is
used for the on-state of a toggle tool (monochrome displays) or a black
border is drawn around the tool (colour displays). Note that to pass a NULL value,
you need to cast it to (wxBitmap *) so that C++ can construct an appropriate temporary
wxBitmap object.}

\docparam{xPos}{Specifies the x position of the tool if automatic layout is not suitable.}

\docparam{yPos}{Specifies the y position of the tool if automatic layout is not suitable.}

\docparam{clientData}{An optional pointer to client data which can be
retrieved later using \helpref{wxToolBarBase::GetToolClientData}{wxtoolbarbasegettoolclientdata}.}

\docparam{shortHelpString}{Used for displaying a tooltip for the tool in the
Windows 95 implementation of wxButtonBar. Pass the empty string if this is not required.}

\docparam{longHelpString}{Used to displayer longer help, such as status line help.
Pass the empty string if this is not required.}

\wxheading{See also}

\helpref{wxToolBarBase::AddSeparator}{wxtoolbarbaseaddseparator}

\membersection{wxToolBarBase::CreateTools}\label{wxtoolbarbasecreatetools}

\func{bool}{CreateTools}{\void}

Call this function after all tools have been added to the toolbar, to actually
create the tools.

\wxheading{Remarks}

Strictly speaking, this is required only for the Windows 95 version of wxButtonBar,
but for portability it should be called anyway.

\wxheading{See also}

\helpref{wxToolBarBase::AddTool}{wxtoolbarbaseaddtool}

\membersection{wxToolBarBase::DrawTool}\label{wxtoolbarbasedrawtool}

\func{void}{DrawTool}{\param{wxMemoryDC\& }{memDC}, \param{wxToolBarBaseTool* }{tool}}

Draws the specified tool onto the window using the given memory device context.

\wxheading{Parameters}

\docparam{memDC}{A memory DC to be used for drawing the tool.}

\docparam{tool}{Tool to be drawn.}

\wxheading{Remarks}

For internal use only.

\membersection{wxToolBarBase::EnableTool}\label{wxtoolbarbaseenabletool}

\func{void}{EnableTool}{\param{int }{toolIndex}, \param{const bool}{ enable}}

Enables or disables the tool.

\wxheading{Parameters}

\docparam{toolIndex}{Tool to enable or disable.}

\docparam{enable}{If TRUE, enables the tool, otherwise disables it.}

\wxheading{Remarks}

For \helpref{wxToolBarSimple}{wxtoolbarsimple}, does nothing. Some other implementations
will change the visible state of the tool to indicate that it is disabled.

\wxheading{See also}

\helpref{wxToolBarBase::GetToolEnabled}{wxtoolbarbasegettoolenabled},\rtfsp
%\helpref{wxToolBarBase::SetToolState}{wxtoolbarbasesettoolstate},\rtfsp
\helpref{wxToolBarBase::ToggleTool}{wxtoolbarbasetoggletool}

\membersection{wxToolBarBase::FindToolForPosition}\label{wxtoolbarbasefindtoolforposition}

\constfunc{wxToolBarBaseTool*}{FindToolForPosition}{\param{const float}{ x}, \param{const float}{ y}}

Finds a tool for the given mouse position.

\wxheading{Parameters}

\docparam{x}{X position.}

\docparam{y}{Y position.}

\wxheading{Return value}

A pointer to a tool if a tool is found, or NULL otherwise.

\wxheading{Remarks}

Used internally, and should not need to be used by the programmer.

\membersection{wxToolBarBase::GetDefaultButtonSize}\label{wxtoolbarbasegetdefaultbuttonsize}

\func{wxSize}{GetDefaultButtonSize}{\void}

Returns the size of a whole button, which is usually larger than a tool bitmap because
of added 3D effects.

\wxheading{See also}

\helpref{wxToolBarBase::SetDefaultSize}{wxtoolbarbasesetdefaultsize},\rtfsp
\helpref{wxToolBarBase::GetDefaultSize}{wxtoolbarbasegetdefaultsize}

\membersection{wxToolBarBase::GetDefaultSize}\label{wxtoolbarbasegetdefaultsize}

\func{wxSize}{GetDefaultSize}{\void}

Returns the size of bitmap that the toolbar expects to have. The default bitmap size is 16 by 15 pixels.

\wxheading{Remarks}

Note that this is the size of the bitmap you pass to \helpref{wxToolBarBase::AddTool}{wxtoolbarbaseaddtool},
and not the eventual size of the tool button.

\wxheading{See also}

\helpref{wxToolBarBase::SetDefaultSize}{wxtoolbarbasesetdefaultsize},\rtfsp
\helpref{wxToolBarBase::GetDefaultButtonSize}{wxtoolbarbasegetdefaultbuttonsize}

\membersection{wxToolBarBase::GetMargins}\label{wxtoolbarbasegetmargins}

\constfunc{wxSize}{GetMargins}{\void}

Returns the left/right and top/bottom margins, which are also used for inter-toolspacing.

\wxheading{See also}

\helpref{wxToolBarBase::SetMargins}{wxtoolbarbasesetmargins}

\membersection{wxToolBarBase::GetMaxSize}\label{wxtoolbarbasegetmaxsize}

\constfunc{void}{GetMaxSize}{\param{float*}{ w}, \param{float*}{ h}}

Gets the maximum size taken up by the tools after layout, including margins.
This can be used to size a frame around the toolbar window.

\wxheading{Parameters}

\docparam{w}{Receives the maximum horizontal size.}

\docparam{h}{Receives the maximum vertical size.}

\membersection{wxToolBarBase::GetToolClientData}\label{wxtoolbarbasegettoolclientdata}

\constfunc{wxObject*}{GetToolClientData}{\param{int }{toolIndex}}

Get any client data associated with the tool.

\wxheading{Parameters}

\docparam{toolIndex}{Index of the tool, as passed to \helpref{wxToolBarBase::AddTool}{wxtoolbarbaseaddtool}.}

\wxheading{Return value}

Client data, or NULL if there is none.

\membersection{wxToolBarBase::GetToolEnabled}\label{wxtoolbarbasegettoolenabled}

\constfunc{bool}{GetToolEnabled}{\param{int }{toolIndex}}

Called to determine whether a tool is enabled (responds to user input).

\wxheading{Parameters}

\docparam{toolIndex}{Index of the tool in question.}

\wxheading{Return value}

TRUE if the tool is enabled, FALSE otherwise.

%\wxheading{See also}
%
%\helpref{wxToolBarBase::SetToolEnabled}{wxtoolbarbasesettoolenabled}
%
\membersection{wxToolBarBase::GetToolLongHelp}\label{wxtoolbarbasegettoollonghelp}

\constfunc{wxString}{GetToolLongHelp}{\param{int }{toolIndex}}

Returns the long help for the given tool.

\wxheading{Parameters}

\docparam{toolIndex}{The tool in question.}

\wxheading{See also}

\helpref{wxToolBarBase::SetToolLongHelp}{wxtoolbarbasesettoollonghelp},\rtfsp
\helpref{wxToolBarBase::SetToolShortHelp}{wxtoolbarbasesettoolshorthelp}\rtfsp

\membersection{wxToolBarBase::GetToolPacking}\label{wxtoolbarbasegettoolpacking}

\constfunc{int}{GetToolPacking}{\void}

Returns the value used for packing tools.

\wxheading{See also}

\helpref{wxToolBarBase::SetToolPacking}{wxtoolbarbasesettoolpacking}

\membersection{wxToolBarBase::GetToolSeparation}\label{wxtoolbarbasegettoolseparation}

\constfunc{int}{GetToolSeparation}{\void}

Returns the default separator size.

\wxheading{See also}

\helpref{wxToolBarBase::SetToolSeparation}{wxtoolbarbasesettoolseparation}

\membersection{wxToolBarBase::GetToolShortHelp}\label{wxtoolbarbasegettoolshorthelp}

\constfunc{wxString}{GetToolShortHelp}{\param{int }{toolIndex}}

Returns the short help for the given tool.

Returns the long help for the given tool.

\wxheading{Parameters}

\docparam{toolIndex}{The tool in question.}

\wxheading{See also}

\helpref{wxToolBarBase::GetToolLongHelp}{wxtoolbarbasegettoollonghelp},\rtfsp
\helpref{wxToolBarBase::SetToolShortHelp}{wxtoolbarbasesettoolshorthelp}\rtfsp

\membersection{wxToolBarBase::GetToolState}\label{wxtoolbarbasegettoolstate}

\constfunc{bool}{GetToolState}{\param{int }{toolIndex}}

Gets the on/off state of a toggle tool.

\wxheading{Parameters}

\docparam{toolIndex}{The tool in question.}

\wxheading{Return value}

TRUE if the tool is toggled on, FALSE otherwise.

%\wxheading{See also}
%
%\helpref{wxToolBarBase::SetToolState}{wxtoolbarbasesettoolstate}
%
\membersection{wxToolBarBase::Layout}\label{wxtoolbarbaselayout}

\func{void}{Layout}{\void}

Called by the application after the tools have been added to
automatically lay the tools out on the window. If you have given
absolute positions when adding the tools, do not call this.

\membersection{wxToolBarBase::OnLeftClick}\label{wxtoolbarbaseonleftclick}

\func{bool}{OnLeftClick}{\param{int}{ toolIndex}, \param{bool}{ toggleDown}}

Called when the user clicks on a tool with the left mouse button. The
programmer should override this function to detect left tool clicks.

\wxheading{Parameters}

\docparam{toolIndex}{The identifier passed to \helpref{wxToolBarBase::AddTool}{wxtoolbarbaseaddtool}.}

\docparam{toggleDown}{TRUE if the tool is a toggle and the toggle is down, otherwise is FALSE.}

\wxheading{Return value}

If the tool is a toggle and this function returns FALSE, the toggle
toggle state (internal and visual) will not be changed. This provides a way of
specifying that toggle operations are not permitted in some circumstances.

\wxheading{See also}

\helpref{wxToolBarBase::OnMouseEnter}{wxtoolbarbaseonmouseenter},\rtfsp
\helpref{wxToolBarBase::OnRightClick}{wxtoolbarbaseonrightclick}

\membersection{wxToolBarBase::OnMouseEnter}\label{wxtoolbarbaseonmouseenter}

\func{void}{OnMouseEnter}{\param{int}{ toolIndex}}

This is called when the mouse cursor moves into a tool or out of
the toolbar.

\wxheading{Parameters}

\docparam{toolIndex}{Greater than -1 if the mouse cursor has moved into the tool,
or -1 if the mouse cursor has moved. The
programmer can override this to provide extra information about the tool,
such as a short description on the status line.}

\wxheading{Remarks}

With some derived toolbar classes, if the mouse moves quickly out of the toolbar, wxWindows may not be able to
detect it. Therefore this function may not always be called when expected.

\membersection{wxToolBarBase::OnRightClick}\label{wxtoolbarbaseonrightclick}

\func{void}{OnRightClick}{\param{int}{ toolIndex}, \param{float}{ x}, \param{float}{ y}}

Called when the user clicks on a tool with the right mouse button. The
programmer should override this function to detect right tool clicks.

\wxheading{Parameters}

\docparam{toolIndex}{The identifier passed to \helpref{wxToolBarBase::AddTool}{wxtoolbarbaseaddtool}.}

\docparam{x}{The x position of the mouse cursor.}

\docparam{y}{The y position of the mouse cursor.}

\wxheading{Remarks}

A typical use of this member might be to pop up a menu.

\wxheading{See also}

\helpref{wxToolBarBase::OnMouseEnter}{wxtoolbarbaseonmouseenter},\rtfsp
\helpref{wxToolBarBase::OnLeftClick}{wxtoolbarbaseonleftclick}

\membersection{wxToolBarBase::SetDefaultSize}\label{wxtoolbarbasesetdefaultsize}

\func{void}{SetDefaultSize}{\param{const wxSize\&}{ size}}

Sets the default size of each tool bitmap. The default bitmap size is 16 by 15 pixels.

\wxheading{Parameters}

\docparam{size}{The size of the bitmaps in the toolbar.}

\wxheading{Remarks}

This should be called to tell the toolbar what the tool bitmap size is. Call
it before you add tools.

Note that this is the size of the bitmap you pass to \helpref{wxToolBarBase::AddTool}{wxtoolbarbaseaddtool},
and not the eventual size of the tool button.

\wxheading{See also}

\helpref{wxToolBarBase::GetDefaultSize}{wxtoolbarbasegetdefaultsize},\rtfsp
\helpref{wxToolBarBase::GetDefaultButtonSize}{wxtoolbarbasegetdefaultbuttonsize}

\membersection{wxToolBarBase::SetMargins}\label{wxtoolbarbasesetmargins}

\func{void}{SetMargins}{\param{const wxSize\&}{ size}}

\func{void}{SetMargins}{\param{int}{ x}, \param{int}{ y}}

Set the values to be used as margins for the toolbar.

\wxheading{Parameters}

\docparam{size}{Margin size.}

\docparam{x}{Left margin, right margin and inter-tool separation value.}

\docparam{y}{Top margin, bottom margin and inter-tool separation value.}

\wxheading{Remarks}

This must be called before the tools are added if absolute positioning is to be used, and the
default (zero-size) margins are to be overridden.

\wxheading{See also}

\helpref{wxToolBarBase::GetMargins}{wxtoolbarbasegetmargins}, \helpref{wxSize}{wxsize}

\membersection{wxToolBarBase::SetToolLongHelp}\label{wxtoolbarbasesettoollonghelp}

\func{void}{SetToolLongHelp}{\param{int }{toolIndex}, \param{const wxString\& }{helpString}}

Sets the long help for the given tool.

\wxheading{Parameters}

\docparam{toolIndex}{The tool in question.}

\docparam{helpString}{A string for the long help.}

\wxheading{Remarks}

You might use the long help for displaying the tool purpose on the status line.

\wxheading{See also}

\helpref{wxToolBarBase::GetToolLongHelp}{wxtoolbarbasegettoollonghelp},\rtfsp
\helpref{wxToolBarBase::SetToolShortHelp}{wxtoolbarbasesettoolshorthelp},\rtfsp

\membersection{wxToolBarBase::SetToolPacking}\label{wxtoolbarbasesettoolpacking}

\func{void}{SetToolPacking}{\param{int}{ packing}}

Sets the value used for spacing tools. The default value is 1.

\wxheading{Parameters}

\docparam{packing}{The value for packing.}

\wxheading{Remarks}

The packing is used for spacing in the vertical direction if the toolbar is horizontal,
and for spacing in the horizontal direction if the toolbar is vertical.

\wxheading{See also}

\helpref{wxToolBarBase::GetToolPacking}{wxtoolbarbasegettoolpacking}

\membersection{wxToolBarBase::SetToolShortHelp}\label{wxtoolbarbasesettoolshorthelp}

\func{void}{SetToolShortHelp}{\param{int }{toolIndex}, \param{const wxString\& }{helpString}}

Sets the short help for the given tool.

\wxheading{Parameters}

\docparam{toolIndex}{The tool in question.}

\docparam{helpString}{The string for the short help.}

\wxheading{Remarks}

An application might use short help for identifying the tool purpose in a tooltip.

\wxheading{See also}

\helpref{wxToolBarBase::GetToolShortHelp}{wxtoolbarbasegettoolshorthelp}, \helpref{wxToolBarBase::SetToolLongHelp}{wxtoolbarbasesettoollonghelp}

\membersection{wxToolBarBase::SetToolSeparation}\label{wxtoolbarbasesettoolseparation}

\func{void}{SetToolSeparation}{\param{int}{ separation}}

Sets the default separator size. The default value is 5.

\wxheading{Parameters}

\docparam{separation}{The separator size.}

\wxheading{See also}

\helpref{wxToolBarBase::AddSeparator}{wxtoolbarbaseaddseparator}

\membersection{wxToolBarBase::ToggleTool}\label{wxtoolbarbasetoggletool}

\func{void}{ToggleTool}{\param{int }{toolIndex}, \param{const bool}{ toggle}}

Toggles a tool on or off.

\wxheading{Parameters}

\docparam{toolIndex}{Tool in question.}

\docparam{toggle}{If TRUE, toggles the tool on, otherwise toggles it off.}

\wxheading{Remarks}

Only applies to a tool that has been specified as a toggle tool.

\wxheading{See also}

\helpref{wxToolBarBase::GetToolState}{wxtoolbarbasegettoolstate}

\section{\class{wxToolBar95}}\label{wxtoolbar95}

This class should be used when a 3D-effect toolbar is required under Windows 95.
It uses the native toolbar control.

\wxheading{Derived from}

\helpref{wxToolBarBase}{wxtoolbarbase}\\
\helpref{wxControl}{wxcontrol}\\
\helpref{wxWindow}{wxwindow}\\
\helpref{wxEvtHandler}{wxevthandler}\\
\helpref{wxObject}{wxobject}

\wxheading{Window styles}

\twocolwidtha{5cm}
\begin{twocollist}\itemsep=0pt
\twocolitem{\windowstyle{wxTB\_FLAT}}{Gives the toolbar a flat look ('coolbar' or 'flatbar' style).}
\end{twocollist}

See also \helpref{window styles overview}{windowstyles}.

\wxheading{Remarks}

Note that this toolbar paints tools to reflect user-selected colours.
The toolbar orientation must always be {\bf wxVERTICAL}.

For member functions, see the documentation for \helpref{wxToolBarBase}{wxtoolbarbase}.

\wxheading{See also}

\overview{Toolbar overview}{wxtoolbaroverview},\rtfsp
\helpref{wxToolBarBase}{wxtoolbarbase},\rtfsp
\helpref{wxToolBarSimple}{wxtoolbarsimple},\rtfsp
\helpref{wxToolBarMSW}{wxtoolbarmsw}

\latexignore{\rtfignore{\wxheading{Members}}}

\membersection{wxToolBar95::wxToolBar95}\label{wxtoolbar95constr}

\func{}{wxToolBar95}{\param{wxWindow*}{ parent}, \param{wxWindowID }{id},\rtfsp
\param{const wxPoint\& }{pos = wxDefaultPosition}, \param{const wxSize\& }{size = wxDefaultSize},\rtfsp
\param{long }{style = 0}, \param{int }{orientation = wxVERTICAL},\rtfsp
\param{int }{nRowsOrColumns = 1}, \param{const wxString\& }{name = ``toolBar"}}

Constructs a toolbar.

\wxheading{Parameters}

\docparam{parent}{Parent window. Must not be NULL.}

\docparam{id}{Window identifier. A value of -1 indicates a default value.}

\docparam{pos}{Window position. If the position (-1, -1) is specified then a default position is chosen.}

\docparam{size}{Window size. If the default size (-1, -1) is specified then a default size is chosen.}

\docparam{orientation}{Specifies a wxVERTICAL or wxHORIZONTAL orientation for laying out
the toolbar.}

\docparam{nRowsOrColumns}{Specifies the number of rows or
columns, whose meaning depends on {\it orientation}.  If laid out
vertically, {\it nRowsOrColumns} specifies the number of rows to draw
before the next column is started; if horizontal, it refers to the
number of columns to draw before the next row is started.}

\docparam{style}{Window style. See \helpref{wxToolBar95}{wxtoolbar95}.}

\docparam{name}{Window name.}

\section{\class{wxToolBarMSW}}\label{wxtoolbarmsw}

This class should be used when a 3D-effect toolbar is required for Windows versions earlier
than Windows 95.

\wxheading{Derived from}

\helpref{wxToolBarBase}{wxtoolbarbase}\\
\helpref{wxControl}{wxcontrol}\\
\helpref{wxWindow}{wxwindow}\\
\helpref{wxEvtHandler}{wxevthandler}\\
\helpref{wxObject}{wxobject}

\wxheading{Window styles}

There are no specific styles for this class.

See also \helpref{window styles overview}{windowstyles}.

\wxheading{Remarks}

Note that this toolbar does not paint tools to reflect user-selected colours: grey shading is used.

For member functions, see the documentation for \helpref{wxToolBarBase}{wxtoolbarbase}.

\wxheading{See also}

\overview{Toolbar overview}{wxtoolbaroverview},\rtfsp
\helpref{wxToolBarBase}{wxtoolbarbase},\rtfsp
\helpref{wxToolBarSimple}{wxtoolbarsimple},\rtfsp
\helpref{wxToolBar95}{wxtoolbar95}

\latexignore{\rtfignore{\wxheading{Members}}}

\membersection{wxToolBarMSW::wxToolBarMSW}\label{wxtoolbarmswconstr}

\func{}{wxToolBarMSW}{\param{wxWindow*}{ parent}, \param{wxWindowID }{id},\rtfsp
\param{const wxPoint\& }{pos = wxDefaultPosition}, \param{const wxSize\& }{size = wxDefaultSize},\rtfsp
\param{long }{style = 0}, \param{int }{orientation = wxVERTICAL},\rtfsp
\param{int }{nRowsOrColumns = 1}, \param{const wxString\& }{name = ``toolBar"}}

Constructs a toolbar.

\wxheading{Parameters}

\docparam{parent}{Parent window. Must not be NULL.}

\docparam{id}{Window identifier. A value of -1 indicates a default value.}

\docparam{pos}{Window position. If the position (-1, -1) is specified then a default position is chosen.}

\docparam{size}{Window size. If the default size (-1, -1) is specified then a default size is chosen.}

\docparam{orientation}{Specifies a wxVERTICAL or wxHORIZONTAL orientation for laying out
the toolbar.}

\docparam{nRowsOrColumns}{Specifies the number of rows or
columns, whose meaning depends on {\it orientation}.  If laid out
vertically, {\it nRowsOrColumns} specifies the number of rows to draw
before the next column is started; if horizontal, it refers to the
number of columns to draw before the next row is started.}

\docparam{style}{Window style. See \helpref{wxToolBarMSW}{wxtoolbarmsw}.}

\docparam{name}{Window name.}


\section{\class{wxToolBarSimple}}\label{wxtoolbarsimple}

This is the generic toolbar class which has an identical appearance
on all platforms.

\wxheading{Derived from}

\helpref{wxToolBarBase}{wxtoolbarbase}\\
\helpref{wxControl}{wxcontrol}\\
\helpref{wxWindow}{wxwindow}\\
\helpref{wxEvtHandler}{wxevthandler}\\
\helpref{wxObject}{wxobject}

\wxheading{Window styles}

\twocolwidtha{5cm}
\begin{twocollist}\itemsep=0pt
\twocolitem{\windowstyle{wxTB\_3DBUTTONS}}{Gives the simple toolbar a mild 3D look to its buttons.}
\end{twocollist}

See also \helpref{window styles overview}{windowstyles}.

\wxheading{Remarks}

In this class, disabling a toolbar tool does not change its appearance.

For member functions, see the documentation for \helpref{wxToolBarBase}{wxtoolbarbase}.

\wxheading{See also}

\overview{Toolbar overview}{wxtoolbaroverview},\rtfsp
\helpref{wxToolBarBase}{wxtoolbarbase},\rtfsp
\helpref{wxToolBarSimple}{wxtoolbarsimple},\rtfsp
\helpref{wxToolBar95}{wxtoolbar95}

\latexignore{\rtfignore{\wxheading{Members}}}

\membersection{wxToolBarSimple::wxToolBarSimple}\label{wxtoolbarsimpleconstr}

\func{}{wxToolBarSimple}{\param{wxWindow*}{ parent}, \param{wxWindowID }{id},\rtfsp
\param{const wxPoint\& }{pos = wxDefaultPosition}, \param{const wxSize\& }{size = wxDefaultSize},\rtfsp
\param{long }{style = 0}, \param{int }{orientation = wxVERTICAL},\rtfsp
\param{int }{nRowsOrColumns = 1}, \param{const wxString\& }{name = ``toolBar"}}

Constructs a toolbar.

\wxheading{Parameters}

\docparam{parent}{Parent window. Must not be NULL.}

\docparam{id}{Window identifier. A value of -1 indicates a default value.}

\docparam{pos}{Window position. If the position (-1, -1) is specified then a default position is chosen.}

\docparam{size}{Window size. If the default size (-1, -1) is specified then a default size is chosen.}

\docparam{orientation}{Specifies a wxVERTICAL or wxHORIZONTAL orientation for laying out
the toolbar.}

\docparam{nRowsOrColumns}{Specifies the number of rows or
columns, whose meaning depends on {\it orientation}.  If laid out
vertically, {\it nRowsOrColumns} specifies the number of rows to draw
before the next column is started; if horizontal, it refers to the
number of columns to draw before the next row is started.}

\docparam{style}{Window style. See \helpref{wxToolBarSimple}{wxtoolbarsimple}.}

\docparam{name}{Window name.}



