\section{\class{wxMutex}}\label{wxmutex}

A mutex object is a synchronization object whose state is set to signaled when
it is not owned by any thread, and nonsignaled when it is owned. Its name comes
from its usefulness in coordinating mutually-exclusive access to a shared
resource. Only one thread at a time can own a mutex object.

For example, when several thread use the data stored in the linked list,
modifications to the list should be only allowed to one thread at a time
because during a new node addition the list integrity is temporarily broken
(this is also called {\it program invariant}).

\wxheading{Example}

{\small%
\begin{verbatim}
    // this variable has an "s_" prefix because it is static: seeing an "s_" in
    // a multithreaded program is in general a good sign that you should use a
    // mutex (or a critical section)
    static wxMutex *s_mutexProtectingTheGlobalData;

    // we store some numbers in this global array which is presumably used by
    // several threads simultaneously
    wxArrayInt s_data;

    void MyThread::AddNewNode(int num)
    {
        // ensure that no other thread accesses the list
        s_mutexProtectingTheGlobalList->Lock();

        s_data.Add(num);

        s_mutexProtectingTheGlobalList->Unlock();
    }

    // return TRUE the given number is greater than all array elements
    bool MyThread::IsGreater(int num)
    {
        // before using the list we must acquire the mutex
        wxMutexLocker lock(s_mutexProtectingTheGlobalData);

        size_t count = s_data.Count();
        for ( size_t n = 0; n < count; n++ )
        {
            if ( s_data[n] > num )
                return FALSE;
        }

        return TRUE;
    }
\end{verbatim}
}

Notice how wxMutexLocker was used in the second function to ensure that the
mutex is unlocked in any case: whether the function returns TRUE or FALSE
(because the destructor of the local object {\it lock} is always called). Using
this class instead of directly using wxMutex is, in general safer and is even
more so if yoor program uses C++ exceptions.

\wxheading{Derived from}

None.

\wxheading{Include files}

<wx/thread.h>

\wxheading{See also}

\helpref{wxThread}{wxthread}, \helpref{wxCondition}{wxcondition}, 
\helpref{wxMutexLocker}{wxmutexlocker}, \helpref{wxCriticalSection}{wxcriticalsection}

\latexignore{\rtfignore{\wxheading{Members}}}

\membersection{wxMutex::wxMutex}\label{wxmutexconstr}

\func{}{wxMutex}{\void}

Default constructor.

\membersection{wxMutex::\destruct{wxMutex}}

\func{}{\destruct{wxMutex}}{\void}

Destroys the wxMutex object.

\membersection{wxMutex::IsLocked}\label{wxmutexislocked}

\constfunc{bool}{IsLocked}{\void}

Returns TRUE if the mutex is locked, FALSE otherwise.

\membersection{wxMutex::Lock}\label{wxmutexlock}

\func{wxMutexError}{Lock}{\void}

Locks the mutex object.

\wxheading{Return value}

One of:

\twocolwidtha{7cm}
\begin{twocollist}\itemsep=0pt
\twocolitem{{\bf wxMUTEX\_NO\_ERROR}}{There was no error.}
\twocolitem{{\bf wxMUTEX\_DEAD\_LOCK}}{A deadlock situation was detected.}
\twocolitem{{\bf wxMUTEX\_BUSY}}{The mutex is already locked by another thread.}
\end{twocollist}

\membersection{wxMutex::TryLock}\label{wxmutextrylock}

\func{wxMutexError}{TryLock}{\void}

Tries to lock the mutex object. If it can't, returns immediately with an error.

\wxheading{Return value}

One of:

\twocolwidtha{7cm}
\begin{twocollist}\itemsep=0pt
\twocolitem{{\bf wxMUTEX\_NO\_ERROR}}{There was no error.}
\twocolitem{{\bf wxMUTEX\_DEAD\_LOCK}}{A deadlock situation was detected.}
\twocolitem{{\bf wxMUTEX\_BUSY}}{The mutex is already locked by another thread.}
\end{twocollist}

\membersection{wxMutex::Unlock}\label{wxmutexunlock}

\func{wxMutexError}{Unlock}{\void}

Unlocks the mutex object.

\wxheading{Return value}

One of:

\twocolwidtha{7cm}
\begin{twocollist}\itemsep=0pt
\twocolitem{{\bf wxMUTEX\_NO\_ERROR}}{There was no error.}
\twocolitem{{\bf wxMUTEX\_DEAD\_LOCK}}{A deadlock situation was detected.}
\twocolitem{{\bf wxMUTEX\_BUSY}}{The mutex is already locked by another thread.}
\twocolitem{{\bf wxMUTEX\_UNLOCKED}}{The calling thread tries to unlock an unlocked mutex.}
\end{twocollist}

