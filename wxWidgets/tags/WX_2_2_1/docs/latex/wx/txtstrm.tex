% ----------------------------------------------------------------------------
% wxTextInputStream
% ----------------------------------------------------------------------------
\section{\class{wxTextInputStream}}\label{wxtextinputstream}

This class provides functions that read text datas using an input stream.
So, you can read {\it text} floats, integers.

The wxTextInputStream correctly reads text files (or streams) in DOS, Macintosh
and Unix formats and reports a single newline char as a line ending.

Operator >> is overloaded and you can use this class like a standard C++ iostream.
Note, however, that the arguments are the fixed size types wxUint32, wxInt32 etc
and on a typical 32-bit computer, none of these match to the "long" type (wxInt32
is defined as int on 32-bit architectures) so that you cannot use long. To avoid
problems (here and elsewhere), make use of wxInt32, wxUint32 and similar types.

For example:

\begin{verbatim}
  wxFileInputStream input( "mytext.txt" );
  wxTextInputStream text( input );
  wxUint8 i1;
  float f2;
  wxString line;

  text >> i1;       // read a 8 bit integer.
  text >> i1 >> f2; // read a 8 bit integer followed by float.
  text >> line;     // read a text line
\end{verbatim}

\wxheading{Include files}

<wx/txtstrm.h>

\latexignore{\rtfignore{\wxheading{Members}}}

\membersection{wxTextInputStream::wxTextInputStream}\label{wxtextinputstreamconstr}

\func{}{wxTextInputStream}{\param{wxInputStream\&}{ stream}}

Constructs a text stream object from an input stream. Only read methods will
be available.

\wxheading{Parameters}

\docparam{stream}{The input stream.}

\membersection{wxTextInputStream::\destruct{wxTextInputStream}}

\func{}{\destruct{wxTextInputStream}}{\void}

Destroys the wxTextInputStream object.

\membersection{wxTextInputStream::Read8}

\func{wxUint8}{Read8}{\void}

Reads a single byte from the stream.

\membersection{wxTextInputStream::Read16}

\func{wxUint16}{Read16}{\void}

Reads a 16 bit integer from the stream.

\membersection{wxTextInputStream::Read32}

\func{wxUint16}{Read32}{\void}

Reads a 32 bit integer from the stream.

\membersection{wxTextInputStream::ReadDouble}

\func{double}{ReadDouble}{\void}

Reads a double (IEEE encoded) from the stream.

\membersection{wxTextInputStream::ReadLine}\label{wxtextinputstreamreadline}

\func{wxString}{wxTextInputStream::ReadLine}{\void}

Reads a line from the input stream and returns it (without the end of line
character).

\membersection{wxTextInputStream::ReadString}

\func{wxString}{wxTextInputStream::ReadString}{\void}

{\bf NB:} This method is deprecated, use \helpref{ReadLine}{wxtextinputstreamreadline} 
or \helpref{ReadWord}{wxtextinputstreamreadword} instead.

Same as \helpref{ReadLine}{wxtextinputstreamreadline}.

\membersection{wxTextInputStream::ReadWord}\label{wxtextinputstreamreadword}

\func{wxString}{wxTextInputStream::ReadWord}{\void}

Reads a word (a sequence of characters until the next separator) from the
input stream.

\wxheading{See also}

\helpref{SetStringSeparators}{wxtextinputstreamsetstringseparators}

\membersection{wxTextInputStream::SetStringSeparators}\label{wxtextinputstreamsetstringseparators}

\func{void}{SetStringSeparators}{\param{const wxString\& }{sep}}

Sets the characters which are used to define the word boundaries in 
\helpref{ReadWord}{wxtextinputstreamreadword}.

The default separators are the space and {\tt TAB} characters.

% ----------------------------------------------------------------------------
% wxTextOutputStream
% ----------------------------------------------------------------------------

\section{\class{wxTextOutputStream}}\label{wxtextoutputstream}

This class provides functions that write text datas using an output stream.
So, you can write {\it text} floats, integers.

You can also simulate the C++ cout class:

\begin{verbatim}
  wxFFileOutputStream output( stderr );
  wxTextOutputStream cout( output );

  cout << "This is a text line" << endl;
  cout << 1234;
  cout << 1.23456;
\end{verbatim}

The wxTextOutputStream writes text files (or streams) on DOS, Macintosh
and Unix in their native formats (concerning the line ending).

\latexignore{\rtfignore{\wxheading{Members}}}

\membersection{wxTextOutputStream::wxTextOutputStream}\label{wxtextoutputstreamconstr}

\func{}{wxTextOutputStream}{\param{wxOutputStream\&}{ stream}, \param{wxEOL}{ mode = wxEOL\_NATIVE}}

Constructs a text stream object from an output stream. Only write methods will
be available.

\wxheading{Parameters}

\docparam{stream}{The output stream.}

\docparam{mode}{The end-of-line mode. One of {\bf wxEOL\_NATIVE}, {\bf wxEOL\_DOS}, {\bf wxEOL\_MAC} and {\bf wxEOL\_UNIX}.}

\membersection{wxTextOutputStream::\destruct{wxTextOutputStream}}

\func{}{\destruct{wxTextOutputStream}}{\void}

Destroys the wxTextOutputStream object.

\membersection{wxTextOutputStream::GetMode}

\func{wxEOL}{wxTextOutputStream::GetMode}{\void}

Returns the end-of-line mode. One of {\bf wxEOL\_DOS}, {\bf wxEOL\_MAC} and {\bf wxEOL\_UNIX}.

\membersection{wxTextOutputStream::SetMode}

\func{void}{wxTextOutputStream::SetMode}{{\param wxEOL}{ mode = wxEOL\_NATIVE}}

Set the end-of-line mode. One of {\bf wxEOL\_NATIVE}, {\bf wxEOL\_DOS}, {\bf wxEOL\_MAC} and {\bf wxEOL\_UNIX}.

\membersection{wxTextOutputStream::Write8}

\func{void}{wxTextOutputStream::Write8}{{\param wxUint8 }{i8}}

Writes the single byte {\it i8} to the stream.

\membersection{wxTextOutputStream::Write16}

\func{void}{wxTextOutputStream::Write16}{{\param wxUint16 }{i16}}

Writes the 16 bit integer {\it i16} to the stream.

\membersection{wxTextOutputStream::Write32}

\func{void}{wxTextOutputStream::Write32}{{\param wxUint32 }{i32}}

Writes the 32 bit integer {\it i32} to the stream.

\membersection{wxTextOutputStream::WriteDouble}

\func{virtual void}{wxTextOutputStream::WriteDouble}{{\param double }{f}}

Writes the double {\it f} to the stream using the IEEE format.

\membersection{wxTextOutputStream::WriteString}

\func{virtual void}{wxTextOutputStream::WriteString}{{\param const wxString\& }{string}}

Writes {\it string} as a line. Depending on the end-of-line mode, it adds 
$\backslash$n, $\backslash$r or $\backslash$r$\backslash$n.

