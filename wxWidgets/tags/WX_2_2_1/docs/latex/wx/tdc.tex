\section{Device context overview}\label{dcoverview}

Classes: \helpref{wxDC}{wxdc}, \helpref{wxPostScriptDC}{wxpostscriptdc},\rtfsp
\rtfsp\helpref{wxMetafileDC}{wxmetafiledc}, \helpref{wxMemoryDC}{wxmemorydc}, \helpref{wxPrinterDC}{wxprinterdc},\rtfsp
\helpref{wxScreenDC}{wxscreendc}, \helpref{wxClientDC}{wxclientdc}, \helpref{wxPaintDC}{wxpaintdc},\rtfsp
\helpref{wxWindowDC}{wxwindowdc}.

A wxDC is a {\it device context} onto which graphics and text can be drawn.
The device context is intended to represent a number of output devices in a generic way,
with the same API being used throughout.

Some device contexts are created temporarily in order to draw on a window.
This is true of \helpref{wxScreenDC}{wxscreendc}, \helpref{wxClientDC}{wxclientdc}, \helpref{wxPaintDC}{wxpaintdc},
and \helpref{wxWindowDC}{wxwindowdc}. The following describes the differences between
these device contexts and when you should use them.

\begin{itemize}\itemsep=0pt
\item {\bf wxScreenDC.} Use this to paint on the screen, as opposed to an individual window.
\item {\bf wxClientDC.} Use this to paint on the client area of window (the part without
borders and other decorations), but do not use it from within an \helpref{wxWindow::OnPaint}{wxwindowonpaint} event.
\item {\bf wxPaintDC.} Use this to paint on the client area of a window, but {\it only} from
within an \helpref{wxWindow::OnPaint}{wxwindowonpaint} event.
\item {\bf wxWindowDC.} Use this to paint on the whole area of a window, including decorations.
This may not be available on non-Windows platforms.
\end{itemize}

To use a client, paint or window device context, create an object on the stack with
the window as argument, for example:

\begin{verbatim}
  void MyWindow::OnMyCmd(wxCommandEvent& event)
  {
    wxClientDC dc(window);
    DrawMyPicture(dc);
  }
\end{verbatim}

Try to write code so it is parameterised by wxDC - if you do this, the same piece of code may
write to a number of different devices, by passing a different device context. This doesn't
work for everything (for example not all device contexts support bitmap drawing) but
will work most of the time.

