\membersection{Help Files Format}\label{helpformat}

wxHTML library uses a reduced version of MS HTML Workshop format.

(See \helpref{wxHtmlHelpController}{wxhtmlhelpcontroller} for help controller description.)

A {\bf book} consists of three files : header file, contents file and index file.

\wxheading{Header file (.hhp)}

Header file must contain these lines (and may contain additional lines which are ignored) :

\begin{verbatim}
Contents file=@filename.hhc@
Index file=@filename.hhk@
Title=@title of your book@
Default topic=@default page to be displayed.htm@
\end{verbatim}

All filenames (including Default topic) are relative to the location of .hhp file.

For larger projects I recommend storing everything but .hhp file into one .zip archive. (E.g. contents file 
would then be reffered as myhelp.zip\#zip:contents.hhc)

\wxheading{Contents file (.hhc)}

Contents file has HTML syntax and it can be parsed by regular HTML parser. It contains exactly one list
(<ul>....</ul> statement):

\begin{verbatim}
<ul>

  <li> <object>
           <param name="Name" value="@topic name@">
	   <param name="ID" value=@numeric_id@>
	   <param name="Local" value="@filename.htm@">
       </object>
  <li> <object>
           <param name="Name" value="@topic name@">
	   <param name="ID" value=@numeric_id@>
	   <param name="Local" value="@filename.htm@">
       </object>
  ...    

</ul>
\end{verbatim}

You can modify value attributes of param tags. {\it topic name} is name of chapter/topic as is displayed in
contents, {\it filename.htm} is HTML page name (relative to .hhp file) and {\it numeric_id} is optional 
- it is used only when you use \helpref{wxHtmlHelpController::Display(int)}{wxhtmlhelpcontrollerdisplay}

Items in the list may be nested - one \<li\> statement may contain \<ul\> sub-statement:

\begin{verbatim}
<ul>

  <li> <object>
           <param name="Name" value="Top node">
	   <param name="Local" value="top.htm">
       </object>
       <ul>
         <li> <object>
                  <param name="Name" value="subnode in topnode">
	          <param name="Local" value="subnode1.htm">
              </object>
	  ...
       </ul>
       
  <li> <object>
           <param name="Name" value="Another Top">
	   <param name="Local" value="top2.htm">
       </object>
  ...    

</ul>
\end{verbatim}

\wxheading{Index file (.hhk)}

Index files have same format as contents file except that ID params are ignored and sublists are {\bf not} 
allowed.

