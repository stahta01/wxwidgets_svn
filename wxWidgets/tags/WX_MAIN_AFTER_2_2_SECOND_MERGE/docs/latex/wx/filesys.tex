%
% automatically generated by HelpGen from
% filesystem.tex at 21/Mar/99 23:00:52
%

\section{\class{wxFileSystem}}\label{wxfilesystem}

This class provides an interface for opening files on different
file systems. It can handle absolute and/or local filenames.
It uses a system of \helpref{handlers}{wxfilesystemhandler} to
provide access to user-defined virtual file systems.

\wxheading{Derived from}

\helpref{wxObject}{wxobject}

\wxheading{Include files}

<wx/filesys.h>

\wxheading{See Also}

\helpref{wxFileSystemHandler}{wxfilesystemhandler}, 
\helpref{wxFSFile}{wxfsfile}, 
\helpref{Overview}{fs}

\latexignore{\rtfignore{\wxheading{Members}}}

\membersection{wxFileSystem::wxFileSystem}\label{wxfilesystemwxfilesystem}

\func{}{wxFileSystem}{\void}

Constructor. 

\membersection{wxFileSystem::AddHandler}\label{wxfilesystemaddhandler}

\func{static void}{AddHandler}{\param{wxFileSystemHandler }{*handler}}

This static function adds new handler into the list of handlers.
The \helpref{handlers}{wxfilesystemhandler} provide access to virtual FS.

\wxheading{Note}

You can call:

\begin{verbatim}
wxFileSystem::AddHandler(new My_FS_Handler);
\end{verbatim}

This is because (a) AddHandler is a static method, and (b) the handlers
are deleted in wxFileSystem's destructor so that you don't have to
care about it.

\membersection{wxFileSystem::ChangePathTo}\label{wxfilesystemchangepathto}

\func{void}{ChangePathTo}{\param{const wxString\& }{location}, \param{bool }{is\_dir = FALSE}}

Sets the current location. {\it location} parameter passed to 
\helpref{OpenFile}{wxfilesystemopenfile} is relative to this path.

{\bf Caution! } Unless {\it is\_dir} is TRUE the {\it location} parameter
is not directory name but the name of the file in this directory!! All these
commands change path to "dir/subdir/" :

\begin{verbatim}
ChangePathTo("dir/subdir/xh.htm");
ChangePathTo("dir/subdir", TRUE);
ChangePathTo("dir/subdir/", TRUE);
\end{verbatim}

\wxheading{Parameters}

\docparam{location}{the new location. Its meaning depends on value of {\it is\_dir}}

\docparam{is\_dir}{if TRUE {\it location} is new directory. If FALSE (default)
{\it location} is {\bf file in} the new directory.}

\wxheading{Example}

\begin{verbatim}
f = fs -> OpenFile("hello.htm"); // opens file 'hello.htm'
fs -> ChangePathTo("subdir/folder", TRUE);
f = fs -> OpenFile("hello.htm"); // opens file 'subdir/folder/hello.htm' !!
\end{verbatim}

\membersection{wxFileSystem::GetPath}\label{wxfilesystemgetpath}

\func{wxString}{GetPath}{\void}

Returns actual path (set by \helpref{ChangePathTo}{wxfilesystemchangepathto}).


\membersection{wxFileSystem::FindFirst}\label{wxfilesystemfindfirst}

\func{wxString}{FindFirst}{\param{const wxString\& }{wildcard}, \param{int }{flags = 0}}

Works like \helpref{wxFindFirstFile}{wxfindfirstfile}. Returns name of the first
filename (withing filesystem's current path) that matches {\it wildcard}. {\it flags} may be one of
wxFILE (only files), wxDIR (only directories) or 0 (both).

\membersection{wxFileSystem::FindNext}\label{wxfilesystemfindnext}

\func{wxString}{FindNext}{\void}

Returns next filename that matches parameters passed to \helpref{FindFirst}{wxfilesystemfindfirst}.


\membersection{wxFileSystem::OpenFile}\label{wxfilesystemopenfile}

\func{wxFSFile*}{OpenFile}{\param{const wxString\& }{location}}

Opens file and returns pointer to \helpref{wxFSFile}{wxfsfile} object
or NULL if failed. It first tries to open the file in relative scope
(based on value passed to ChangePathTo() method) and then as an
absolute path.

