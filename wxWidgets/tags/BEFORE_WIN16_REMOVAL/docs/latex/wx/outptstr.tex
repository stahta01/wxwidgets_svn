% -----------------------------------------------------------------------------
% wxOutputStream
% -----------------------------------------------------------------------------
\section{\class{wxOutputStream}}\label{wxoutputstream}

wxOutputStream is an abstract base class which may not be used directly.

\wxheading{Derived from}

\helpref{wxStreamBase}{wxstreambase}

\wxheading{Include files}

<wx/stream.h>

\latexignore{\rtfignore{\wxheading{Members}}}

% -----------
% ctor & dtor
% -----------
\membersection{wxOutputStream::wxOutputStream}

\func{}{wxOutputStream}{\void}

Creates a dummy wxOutputStream object.

\membersection{wxOutputStream::\destruct{wxOutputStream}}

\func{}{\destruct{wxOutputStream}}{\void}

Destructor.

\membersection{wxOutputStream::LastWrite}

\constfunc{size\_t}{LastWrite}{\void}

Returns the number of bytes written during the last Write().

\membersection{wxOutputStream::PutC}

\func{void}{PutC}{\param{char}{ c}}

Puts the specified character in the output queue and increments the
stream position.

\membersection{wxOutputStream::SeekO}\label{wxoutputstreamseeko}

\func{off\_t}{SeekO}{\param{off\_t}{ pos}, \param{wxSeekMode}{ mode}}

Changes the stream current position.

\membersection{wxOutputStream::TellO}

\constfunc{off\_t}{TellO}{\void}

Returns the current stream position.

\membersection{wxOutputStream::Write}

\func{wxOutputStream\&}{Write}{\param{const void *}{buffer}, \param{size\_t}{ size}}

Writes the specified amount of bytes using the data of {\it buffer}. 
{\it WARNING!} The buffer absolutely needs to have at least the specified size.

This function returns a reference on the current object, so the user can test
any states of the stream right away.

\func{wxOutputStream\&}{Write}{\param{wxInputStream\&}{ stream\_in}}

Reads data from the specified input stream and stores them 
in the current stream. The data is read until an error is raised
by one of the two streams.

