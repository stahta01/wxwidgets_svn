%%%%%%%%%%%%%%%%%%%%%%%%%%%%%%%%%%%%%%%%%%%%%%%%%%%%%%%%%%%%%%%%%%%%%%%%%%%%%%%
%% Name:        wizard.tex
%% Purpose:     wxWizard class documentation
%% Author:      Vadim Zeitlin
%% Modified by: Robert Vazan (sizers)
%% Created:     02.04.00
%% RCS-ID:      $Id$
%% Copyright:   (c) Vadim Zeitlin
%% License:     wxWidgets license
%%%%%%%%%%%%%%%%%%%%%%%%%%%%%%%%%%%%%%%%%%%%%%%%%%%%%%%%%%%%%%%%%%%%%%%%%%%%%%%

\section{\class{wxWizard}}\label{wxwizard}

wxWizard is the central class for implementing `wizard-like' dialogs. These
dialogs are mostly familiar to Windows users and are nothing else but a
sequence of `pages' each of them displayed inside a dialog which has the
buttons to pass to the next (and previous) pages.

The wizards are typically used to decompose a complex dialog into several
simple steps and are mainly useful to the novice users, hence it is important
to keep them as simple as possible.

To show a wizard dialog, you must first create an object of wxWizard class
using either the non default constructor or a default one followed by call to 
\helpref{Create}{wxwizardcreate} function. Then you should add all pages you
want the wizard to show and call \helpref{RunWizard}{wxwizardrunwizard}.
Finally, don't forget to call {\tt wizard->Destroy()}.

\wxheading{Derived from}

\helpref{wxDialog}{wxdialog}\\
\helpref{wxPanel}{wxpanel}\\
\helpref{wxWindow}{wxwindow}\\
\helpref{wxEvtHandler}{wxevthandler}\\
\helpref{wxObject}{wxobject}

\wxheading{Include files}

<wx/wizard.h>

\wxheading{Event table macros}

To process input from a wizard dialog, use these event handler macros to
direct input to member functions that take a 
\helpref{wxWizardEvent}{wxwizardevent} argument. For some events, 
\helpref{Veto()}{wxnotifyeventveto} can be called to prevent the event from
happening.

\twocolwidtha{7cm}
\begin{twocollist}\itemsep=2pt
\twocolitem{{\bf EVT\_WIZARD\_PAGE\_CHANGED(id, func)}}{The page has been just
changed (this event can not be vetoed).}
\twocolitem{{\bf EVT\_WIZARD\_PAGE\_CHANGING(id, func)}}{The page is being
changed (this event can be vetoed).}
\twocolitem{{\bf EVT\_WIZARD\_CANCEL(id, func)}}{The user attempted to cancel
the wizard (this event may also be vetoed).}
\twocolitem{{\bf EVT\_WIZARD\_HELP(id, func)}}{The wizard help button was pressed.}
\twocolitem{{\bf EVT\_WIZARD\_FINISHED(id, func)}}{The wizard finished button was pressed.}
\end{twocollist}

\wxheading{Extended styles}

Use the \helpref{wxWindow::SetExtraStyle}{wxwindowsetextrastyle} function to set the following
style. You will need to use two-step construction (use the default constructor, call {\bf SetExtraStyle}, then call {\bf Create}).

\twocolwidtha{5cm}%
\begin{twocollist}\itemsep=0pt
\twocolitem{\windowstyle{wxWIZARD\_EX\_HELPBUTTON}}{Shows a Help button using wxID\_HELP.}
\end{twocollist}

See also \helpref{wxDialog}{wxdialog} for other extended styles.

\wxheading{See also}

\helpref{wxWizardEvent}{wxwizardevent}, \helpref{wxWizardPage}{wxwizardpage}, \helpref{wxWizard sample}{samplewizard}

\latexignore{\rtfignore{\wxheading{Members}}}

\membersection{wxWizard::wxWizard}\label{wxwizardctor}

\func{}{wxWizard}{\void}

Default constructor. Use this if you wish to derive from wxWizard and then call 
\helpref{Create}{wxwizardcreate}, for example if you wish to set an extra style
with \helpref{wxWindow::SetExtraStyle}{wxwindowsetextrastyle} between the two
calls.

\func{}{wxWizard}{\param{wxWindow* }{parent}, \param{int }{id = -1}, \param{const wxString\& }{title = wxEmptyString}, \param{const wxBitmap\& }{bitmap = wxNullBitmap}, \param{const wxPoint\& }{pos = wxDefaultPosition}, \param{long }{style = wxDEFAULT\_DIALOG\_STYLE}}

Constructor which really creates the wizard -- if you use this constructor, you
shouldn't call \helpref{Create}{wxwizardcreate}.

Notice that unlike almost all other wxWidgets classes, there is no {\it size} 
parameter in wxWizard constructor because the wizard will have a predefined
default size by default. If you want to change this, you should use the 
\helpref{GetPageAreaSizer}{wxwizardgetpageareasizer} function.

\wxheading{Parameters}

\docparam{parent}{The parent window, may be NULL.}

\docparam{id}{The id of the dialog, will usually be just $-1$.}

\docparam{title}{The title of the dialog.}

\docparam{bitmap}{The default bitmap used in the left side of the wizard. See
also \helpref{GetBitmap}{wxwizardpagegetbitmap}.}

\docparam{pos}{The position of the dialog, it will be centered on the screen
by default.}

\docparam{style}{Window style is passed to wxDialog.}


\membersection{wxWizard::Create}\label{wxwizardcreate}

\func{bool}{Create}{\param{wxWindow* }{parent}, \param{int }{id = -1}, \param{const wxString\& }{title = wxEmptyString}, \param{const wxBitmap\& }{bitmap = wxNullBitmap}, \param{const wxPoint\& }{pos = wxDefaultPosition}, \param{long }{style = wxDEFAULT\_DIALOG\_STYLE}}

Creates the wizard dialog. Must be called if the default constructor had been
used to create the object.

Notice that unlike almost all other wxWidgets classes, there is no {\it size} 
parameter in wxWizard constructor because the wizard will have a predefined
default size by default. If you want to change this, you should use the 
\helpref{GetPageAreaSizer}{wxwizardgetpageareasizer} function.

\wxheading{Parameters}

\docparam{parent}{The parent window, may be NULL.}

\docparam{id}{The id of the dialog, will usually be just $-1$.}

\docparam{title}{The title of the dialog.}

\docparam{bitmap}{The default bitmap used in the left side of the wizard. See
also \helpref{GetBitmap}{wxwizardpagegetbitmap}.}

\docparam{pos}{The position of the dialog, it will be centered on the screen
by default.}

\docparam{style}{Window style is passed to wxDialog.}


\membersection{wxWizard::FitToPage}\label{wxwizardfittopage}

\func{void}{FitToPage}{\param{const wxWizardPage* }{firstPage}}

This method is obsolete, use
\helpref{GetPageAreaSizer}{wxwizardgetpageareasizer} instead.

Sets the page size to be big enough for all the pages accessible via the
given {\it firstPage}, i.e. this page, its next page and so on.

This method may be called more than once and it will only change the page size
if the size required by the new page is bigger than the previously set one.
This is useful if the decision about which pages to show is taken during the
run-time as in this case, the wizard won't be able to get to all pages starting
from a single one and you should call {\it Fit} separately for the others.


\membersection{wxWizard::GetCurrentPage}\label{wxwizardgetcurrentpage}

\constfunc{wxWizardPage*}{GetCurrentPage}{\void}

Get the current page while the wizard is running. {\tt NULL} is returned if 
\helpref{RunWizard()}{wxwizardrunwizard} is not being executed now.


\membersection{wxWizard::GetPageAreaSizer}\label{wxwizardgetpageareasizer}

\constfunc{virtual wxSizer*}{GetPageAreaSizer}{\void}

Returns pointer to page area sizer. Wizard is laid out using sizers and
page area sizer is the place holder for the pages. All pages are resized before
being shown to match the wizard page area.

Page area sizer has minimal size that is maximum of several values. First,
all pages (or other objects) added to the sizer. Second, all pages reachable
by repeatedly applying 
\helpref{wxWizardPage::GetNext}{wxwizardpagegetnext} to
any page inserted into the sizer. Third,
minimal size specified using \helpref{SetPageSize}{wxwizardsetpagesize} and 
\helpref{FitToPage}{wxwizardfittopage}. Fourth, the total wizard height may
be increased to accomodate the bitmap height. Fifth and finally, wizards are
never smaller some built-in minimal size to avoid too small wizards.

Caller can use \helpref{wxSizer::SetMinSize}{wxsizersetminsize} to enlarge it
beyond minimal size. If {\tt wxRESIZE\_BORDER} was passed to constructor, user
can resize wizard and consequently page area (but not make it smaller than the
minimal size).

It is recommended to add first page to page area sizer. For simple wizards,
this will enlarge the wizard to fit biggest page. For non-linear wizards,
first page of every separate chain should be added. Caller-specified size
can be accomplished using \helpref{wxSizer::SetMinSize}{wxsizersetminsize}.

Adding pages to page area sizer affects default border width around page
area that can be altered with \helpref{SetBorder}{wxwizardsetborder}.


\membersection{wxWizard::GetPageSize}\label{wxwizardgetpagesize}

\constfunc{wxSize}{GetPageSize}{\void}

Returns the size available for the pages.


\membersection{wxWizard::HasNextPage}\label{wxwizardhasnextpage}

\func{virtual bool}{HasNextPage}{\param{wxWizardPage *}{page}}

Return {\tt true} if this page is not the last one in the wizard. The base
class version implements this by calling 
\helpref{page->GetNext}{wxwizardpagegetnext} but this could be undesirable if,
for example, the pages are created on demand only.

\wxheading{See also}

\helpref{HasPrevPage}{wxwizardhasprevpage}


\membersection{wxWizard::HasPrevPage}\label{wxwizardhasprevpage}

\func{virtual bool}{HasPrevPage}{\param{wxWizardPage *}{page}}

Return {\tt true} if this page is not the last one in the wizard. The base
class version implements this by calling 
\helpref{page->GetPrev}{wxwizardpagegetprev} but this could be undesirable if,
for example, the pages are created on demand only.

\wxheading{See also}

\helpref{HasNextPage}{wxwizardhasnextpage}


\membersection{wxWizard::RunWizard}\label{wxwizardrunwizard}

\func{bool}{RunWizard}{\param{wxWizardPage* }{firstPage}}

Executes the wizard starting from the given page, returns {\tt true} if it was
successfully finished or {\tt false} if user cancelled it. The {\it firstPage} 
can not be {\tt NULL}.


\membersection{wxWizard::SetPageSize}\label{wxwizardsetpagesize}

\func{void}{SetPageSize}{\param{const wxSize\& }{sizePage}}

This method is obsolete, use
\helpref{GetPageAreaSizer}{wxwizardgetpageareasizer} instead.

Sets the minimal size to be made available for the wizard pages. The wizard
will take into account the size of the bitmap (if any) itself. Also, the
wizard will never be smaller than the default size.

The recommended way to use this function is to layout all wizard pages using
the sizers (even though the wizard is not resizeable) and then use 
\helpref{wxSizer::CalcMin}{wxsizercalcmin} in a loop to calculate the maximum
of minimal sizes of the pages and pass it to SetPageSize().


\membersection{wxWizard::SetBorder}\label{wxwizardsetborder}

\func{void}{SetBorder}{\param{int }{border}}

Sets width of border around page area. Default is zero. For backward
compatibility, if there are no pages in
\helpref{GetPageAreaSizer}{wxwizardgetpageareasizer}, default is $5$ pixels.

If there is five point border around all controls in a page and border around
page area is left zero, five point white space along all dialog borders
will be added to control border to space page controls ten points from dialog
border and non-page controls.

