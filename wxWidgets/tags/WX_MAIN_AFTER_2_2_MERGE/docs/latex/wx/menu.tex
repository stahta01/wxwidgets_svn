\section{\class{wxMenu}}\label{wxmenu}

A menu is a popup (or pull down) list of items, one of which may be
selected before the menu goes away (clicking elsewhere dismisses the
menu).  Menus may be used to construct either menu bars or popup menus.

A menu item has an integer ID associated with it which can be used to
identify the selection, or to change the menu item in some way.

\wxheading{Derived from}

\helpref{wxEvtHandler}{wxevthandler}\\
\helpref{wxObject}{wxobject}

\wxheading{Include files}

<wx/menu.h>

\wxheading{Event handling}

If the menu is part of a menubar, then \helpref{wxMenuBar}{wxmenubar} event processing is used.

With a popup menu, there is a variety of ways to handle a menu selection event
(wxEVT\_COMMAND\_MENU\_SELECTED).

\begin{enumerate}\itemsep=0pt
\item Derive a new class from wxMenu and define event table entries using the EVT\_MENU macro.
\item Set a new event handler for wxMenu, using an object whose class has EVT\_MENU entries.
\item Provide EVT\_MENU handlers in the window which pops up the menu, or in an ancestor of
this window.
\item Define a callback of type wxFunction, which you pass to the wxMenu constructor.
The callback takes a reference to the menu, and a reference to a
\helpref{wxCommandEvent}{wxcommandevent}. This method is deprecated and should
not be used in the new code, it is provided for backwards compatibility only.
\end{enumerate}

\wxheading{See also}

\helpref{wxMenuBar}{wxmenubar}, \helpref{wxWindow::PopupMenu}{wxwindowpopupmenu},\rtfsp
\helpref{Event handling overview}{eventhandlingoverview}

\latexignore{\rtfignore{\wxheading{Members}}}

\membersection{wxMenu::wxMenu}\label{wxmenuconstr}

\func{}{wxMenu}{\param{const wxString\& }{title = ""}, \param{long}{ style = 0}}

Constructs a wxMenu object.

\wxheading{Parameters}

\docparam{title}{A title for the popup menu: the empty string denotes no title.}

\docparam{style}{If set to {\tt wxMENU\_TEAROFF}, the menu will be detachable.}

\func{}{wxMenu}{\param{long}{ style}}

Constructs a wxMenu object.

\wxheading{Parameters}

\docparam{style}{If set to {\tt wxMENU\_TEAROFF}, the menu will be detachable.}

\membersection{wxMenu::\destruct{wxMenu}}

\func{}{\destruct{wxMenu}}{\void}

Destructor, destroying the menu.

Note: under Motif, a popup menu must have a valid parent (the window
it was last popped up on) when being destroyed. Therefore, make sure
you delete or re-use the popup menu {\it before} destroying the
parent window. Re-use in this context means popping up the menu on
a different window from last time, which causes an implicit destruction
and recreation of internal data structures.

\membersection{wxMenu::Append}\label{wxmenuappend}

\func{void}{Append}{\param{int}{ id}, \param{const wxString\& }{ item}, \param{const wxString\& }{helpString = ""},\rtfsp
\param{const bool}{ checkable = FALSE}}

Adds a string item to the end of the menu.

\func{void}{Append}{\param{int}{ id}, \param{const wxString\& }{ item}, \param{wxMenu *}{subMenu},\rtfsp
\param{const wxString\& }{helpString = ""}}

Adds a pull-right submenu to the end of the menu.

\func{void}{Append}{\param{wxMenuItem*}{ menuItem}}

Adds a menu item object. This is the most generic variant of Append() method
because it may be used for both items (including separators) and submenus and
because you can also specify various extra properties of a menu item this way,
such as bitmaps and fonts.

\wxheading{Parameters}

\docparam{id}{The menu command identifier.}

\docparam{item}{The string to appear on the menu item.}

\docparam{menu}{Pull-right submenu.}

\docparam{checkable}{If TRUE, this item is checkable.}

\docparam{helpString}{An optional help string associated with the item.
By default, \helpref{wxFrame::OnMenuHighlight}{wxframeonmenuhighlight} displays
this string in the status line.}

\docparam{menuItem}{A menuitem object. It will be owned by the wxMenu object after this function
is called, so do not delete it yourself.}

\wxheading{Remarks}

This command can be used after the menu has been shown, as well as on initial
creation of a menu or menubar.

\wxheading{See also}

\helpref{wxMenu::AppendSeparator}{wxmenuappendseparator},\rtfsp
\helpref{wxMenu::Insert}{wxmenuinsert},\rtfsp
\helpref{wxMenu::SetLabel}{wxmenusetlabel}, \helpref{wxMenu::GetHelpString}{wxmenugethelpstring},\rtfsp
\helpref{wxMenu::SetHelpString}{wxmenusethelpstring}, \helpref{wxMenuItem}{wxmenuitem}

\pythonnote{In place of a single overloaded method name, wxPython
implements the following methods:\par
\indented{2cm}{\begin{twocollist}
\twocolitem{{\bf Append(id, string, helpStr="", checkable=FALSE)}}{}
\twocolitem{{\bf AppendMenu(id, string, aMenu, helpStr="")}}{}
\twocolitem{{\bf AppendItem(aMenuItem)}}{}
\end{twocollist}}
}

\membersection{wxMenu::AppendSeparator}\label{wxmenuappendseparator}

\func{void}{AppendSeparator}{\void}

Adds a separator to the end of the menu.

\wxheading{See also}

\helpref{wxMenu::Append}{wxmenuappend}

\membersection{wxMenu::Break}\label{wxmenubreak}

\func{void}{Break}{\void}

Inserts a break in a menu, causing the next appended item to appear in a new column.

\membersection{wxMenu::Check}\label{wxmenucheck}

\func{void}{Check}{\param{int}{ id}, \param{const bool}{ check}}

Checks or unchecks the menu item.

\wxheading{Parameters}

\docparam{id}{The menu item identifier.}

\docparam{check}{If TRUE, the item will be checked, otherwise it will be unchecked.}

\wxheading{See also}

\helpref{wxMenu::IsChecked}{wxmenuischecked}

\membersection{wxMenu::Delete}\label{wxmenudelete}

\func{void}{Delete}{\param{int }{id}}

\func{void}{Delete}{\param{wxMenuItem *}{item}}

Deletes the menu item from the menu. If the item is a submenu, it will
{\bf not} be deleted. Use \helpref{Destroy}{wxmenudestroy} if you want to
delete a submenu.

\wxheading{Parameters}

\docparam{id}{Id of the menu item to be deleted.}

\docparam{item}{Menu item to be deleted.}

\wxheading{See also}

\helpref{wxMenu::FindItem}{wxmenufinditem},\rtfsp
\helpref{wxMenu::Destroy}{wxmenudestroy},\rtfsp
\helpref{wxMenu::Remove}{wxmenuremove}

\membersection{wxMenu::Destroy}\label{wxmenudestroy}

\func{void}{Destroy}{\param{int }{id}}

\func{void}{Destroy}{\param{wxMenuItem *}{item}}

Deletes the menu item from the menu. If the item is a submenu, it will
be deleted. Use \helpref{Remove}{wxmenuremove} if you want to keep the submenu
(for example, to reuse it later).

\wxheading{Parameters}

\docparam{id}{Id of the menu item to be deleted.}

\docparam{item}{Menu item to be deleted.}

\wxheading{See also}

\helpref{wxMenu::FindItem}{wxmenufinditem},\rtfsp
\helpref{wxMenu::Deletes}{wxmenudelete},\rtfsp
\helpref{wxMenu::Remove}{wxmenuremove}

\membersection{wxMenu::Enable}\label{wxmenuenable}

\func{void}{Enable}{\param{int}{ id}, \param{const bool}{ enable}}

Enables or disables (greys out) a menu item.

\wxheading{Parameters}

\docparam{id}{The menu item identifier.}

\docparam{enable}{TRUE to enable the menu item, FALSE to disable it.}

\wxheading{See also}

\helpref{wxMenu::IsEnabled}{wxmenuisenabled}

\membersection{wxMenu::FindItem}\label{wxmenufinditem}

\constfunc{int}{FindItem}{\param{const wxString\& }{itemString}}

Finds the menu item id for a menu item string.

\constfunc{wxMenuItem *}{FindItem}{\param{int}{ id}, \param{wxMenu **}{menu = NULL}}

Finds the menu item object associated with the given menu item identifier and,
optionally, the (sub)menu it belongs to.

\wxheading{Parameters}

\docparam{itemString}{Menu item string to find.}

\docparam{id}{Menu item identifier.}

\docparam{menu}{If the pointer is not NULL, it will be filled with the items
parent menu (if the item was found)}

\wxheading{Return value}

First form: menu item identifier, or wxNOT\_FOUND if none is found.

Second form: returns the menu item object, or NULL if it is not found.

\wxheading{Remarks}

Any special menu codes are stripped out of source and target strings
before matching.

\pythonnote{The name of this method in wxPython is {\tt FindItemById} 
and it does not support the second parameter.}

\membersection{wxMenu::GetHelpString}\label{wxmenugethelpstring}

\constfunc{wxString}{GetHelpString}{\param{int}{ id}}

Returns the help string associated with a menu item.

\wxheading{Parameters}

\docparam{id}{The menu item identifier.}

\wxheading{Return value}

The help string, or the empty string if there is no help string or the
item was not found.

\wxheading{See also}

\helpref{wxMenu::SetHelpString}{wxmenusethelpstring}, \helpref{wxMenu::Append}{wxmenuappend}

\membersection{wxMenu::GetLabel}\label{wxmenugetlabel}

\constfunc{wxString}{GetLabel}{\param{int}{ id}}

Returns a menu item label.

\wxheading{Parameters}

\docparam{id}{The menu item identifier.}

\wxheading{Return value}

The item label, or the empty string if the item was not found.

\wxheading{See also}

\helpref{wxMenu::SetLabel}{wxmenusetlabel}

\membersection{wxMenu::GetMenuItemCount}\label{wxmenugetmenuitemcount}

\constfunc{size\_t}{GetMenuItemCount}{\void}

Returns the number of items in the menu.

\membersection{wxMenu::GetMenuItems}\label{wxmenugetmenuitems}

\constfunc{wxMenuItemList\&}{GetMenuItems}{\void}

Returns the list of items in the menu. wxMenuItemList is a pseudo-template
list class containing wxMenuItem pointers.

\membersection{wxMenu::GetTitle}\label{wxmenugettitle}

\constfunc{wxString}{GetTitle}{\void}

Returns the title of the menu.

\wxheading{Remarks}

This is relevant only to popup menus.

\wxheading{See also}

\helpref{wxMenu::SetTitle}{wxmenusettitle}

\membersection{wxMenu::Insert}\label{wxmenuinsert}

\func{bool}{Insert}{\param{size\_t }{pos}, \param{wxMenuItem *}{item}}

Inserts the given {\it item} before the position {\it pos}. Inserting the item
at the position \helpref{GetMenuItemCount}{wxmenugetmenuitemcount} is the same
as appending it.

\wxheading{See also}

\helpref{wxMenu::Append}{wxmenuappend}

\membersection{wxMenu::IsChecked}\label{wxmenuischecked}

\constfunc{bool}{IsChecked}{\param{int}{ id}}

Determines whether a menu item is checked.

\wxheading{Parameters}

\docparam{id}{The menu item identifier.}

\wxheading{Return value}

TRUE if the menu item is checked, FALSE otherwise.

\wxheading{See also}

\helpref{wxMenu::Check}{wxmenucheck}

\membersection{wxMenu::IsEnabled}\label{wxmenuisenabled}

\constfunc{bool}{IsEnabled}{\param{int}{ id}}

Determines whether a menu item is enabled.

\wxheading{Parameters}

\docparam{id}{The menu item identifier.}

\wxheading{Return value}

TRUE if the menu item is enabled, FALSE otherwise.

\wxheading{See also}

\helpref{wxMenu::Enable}{wxmenuenable}

\membersection{wxMenu::Remove}\label{wxmenuremove}

\func{wxMenuItem *}{Remove}{\param{int }{id}}

\func{wxMenuItem *}{Remove}{\param{wxMenuItem *}{item}}

Removes the menu item from the menu but doesn't delete the associated C++
object. This allows to reuse the same item later by adding it back to the menu
(especially useful with submenus).

\wxheading{Parameters}

\docparam{id}{The identifier of the menu item to remove.}

\docparam{item}{The menu item to remove.}

\wxheading{Return value}

The item which was detached from the menu.

\membersection{wxMenu::SetHelpString}\label{wxmenusethelpstring}

\func{void}{SetHelpString}{\param{int}{ id}, \param{const wxString\& }{helpString}}

Sets an item's help string.

\wxheading{Parameters}

\docparam{id}{The menu item identifier.}

\docparam{helpString}{The help string to set.}

\wxheading{See also}

\helpref{wxMenu::GetHelpString}{wxmenugethelpstring}

\membersection{wxMenu::SetLabel}\label{wxmenusetlabel}

\func{void}{SetLabel}{\param{int}{ id}, \param{const wxString\& }{label}}

Sets the label of a menu item.

\wxheading{Parameters}

\docparam{id}{The menu item identifier.}

\docparam{label}{The menu item label to set.}

\wxheading{See also}

\helpref{wxMenu::Append}{wxmenuappend}, \helpref{wxMenu::GetLabel}{wxmenugetlabel}

\membersection{wxMenu::SetTitle}\label{wxmenusettitle}

\func{void}{SetTitle}{\param{const wxString\& }{title}}

Sets the title of the menu.

\wxheading{Parameters}

\docparam{title}{The title to set.}

\wxheading{Remarks}

This is relevant only to popup menus.

\wxheading{See also}

\helpref{wxMenu::SetTitle}{wxmenusettitle}

\membersection{wxMenu::UpdateUI}\label{wxmenuupdateui}

\constfunc{void}{UpdateUI}{\param{wxEvtHandler*}{ source = NULL}}

Sends events to {\it source} (or owning window if NULL) to update the
menu UI. This is called just before the menu is popped up with \helpref{wxWindow::PopupMenu}{wxwindowpopupmenu}, but
the application may call it at other times if required.

\wxheading{See also}

\helpref{wxUpdateUIEvent}{wxupdateuievent}

\section{\class{wxMenuBar}}\label{wxmenubar}

A menu bar is a series of menus accessible from the top of a frame.

\wxheading{Derived from}

\helpref{wxEvtHandler}{wxevthandler}\\
\helpref{wxObject}{wxobject}

\wxheading{Include files}

<wx/menu.h>

\wxheading{Event handling}

To respond to a menu selection, provide a handler for EVT\_MENU, in the frame
that contains the menu bar. If you have a toolbar which uses the same identifiers
as your EVT\_MENU entries, events from the toolbar will also be processed by your
EVT\_MENU event handlers.

Note that menu commands (and UI update events for menus) are first sent to
the focus window within the frame. If no window within the frame has the focus,
then the events are sent directly to the frame. This allows command and UI update
handling to be processed by specific windows and controls, and not necessarily
by the application frame.

\wxheading{See also}

\helpref{wxMenu}{wxmenu}, \helpref{Event handling overview}{eventhandlingoverview}

\latexignore{\rtfignore{\wxheading{Members}}}

\membersection{wxMenuBar::wxMenuBar}\label{wxmenubarconstr}

\func{void}{wxMenuBar}{\param{long }{style = 0}}

Default constructor.

\func{void}{wxMenuBar}{\param{int}{ n}, \param{wxMenu*}{ menus[]}, \param{const wxString }{titles[]}}

Construct a menu bar from arrays of menus and titles.

\wxheading{Parameters}

\docparam{n}{The number of menus.}

\docparam{menus}{An array of menus. Do not use this array again - it now belongs to the
menu bar.}

\docparam{titles}{An array of title strings. Deallocate this array after creating the menu bar.}

\docparam{style}{If {\tt wxMB\_DOCKABLE} the menu bar can be detached (wxGTK only).}

\pythonnote{Only the default constructor is supported in wxPython.
Use wxMenuBar.Append instead.}

\membersection{wxMenuBar::\destruct{wxMenuBar}}

\func{void}{\destruct{wxMenuBar}}{\void}

Destructor, destroying the menu bar and removing it from the parent frame (if any).

\membersection{wxMenuBar::Append}\label{wxmenubarappend}

\func{bool}{Append}{\param{wxMenu *}{menu}, \param{const wxString\& }{title}}

Adds the item to the end of the menu bar.

\wxheading{Parameters}

\docparam{menu}{The menu to add. Do not deallocate this menu after calling {\bf Append}.}

\docparam{title}{The title of the menu.}

\wxheading{Return value}

TRUE on success, FALSE if an error occurred.

\wxheading{See also}

\helpref{wxMenuBar::Insert}{wxmenubarinsert}

\membersection{wxMenuBar::Check}\label{wxmenubarcheck}

\func{void}{Check}{\param{int}{ id}, \param{const bool}{ check}}

Checks or unchecks a menu item.

\wxheading{Parameters}

\docparam{id}{The menu item identifier.}

\docparam{check}{If TRUE, checks the menu item, otherwise the item is unchecked.}

\wxheading{Remarks}

Only use this when the menu bar has been associated
with a frame; otherwise, use the wxMenu equivalent call.

\membersection{wxMenuBar::Enable}\label{wxmenubarenable}

\func{void}{Enable}{\param{int}{ id}, \param{const bool}{ enable}}

Enables or disables (greys out) a menu item.

\wxheading{Parameters}

\docparam{id}{The menu item identifier.}

\docparam{enable}{TRUE to enable the item, FALSE to disable it.}

\wxheading{Remarks}

Only use this when the menu bar has been
associated with a frame; otherwise, use the wxMenu equivalent call.

\membersection{wxMenuBar::EnableTop}\label{wxmenubarenabletop}

\func{void}{EnableTop}{\param{int}{ pos}, \param{const bool}{ enable}}

Enables or disables a whole menu.

\wxheading{Parameters}

\docparam{pos}{The position of the menu, starting from zero.}

\docparam{enable}{TRUE to enable the menu, FALSE to disable it.}

\wxheading{Remarks}

Only use this when the menu bar has been
associated with a frame.

\membersection{wxMenuBar::FindMenu}\label{wxmenubarfindmenu}

\constfunc{int}{FindMenu}{\param{const wxString\& }{title}}

Returns the index of the menu with the given {\it title} or wxNOT\_FOUND if no
such menu exists in this menubar. The {\it title} parameter may specify either
the menu title (with accelerator characters, i.e. {\tt "\&File"}) or just the
menu label ({\tt "File"}) indifferently.

\membersection{wxMenuBar::FindMenuItem}\label{wxmenubarfindmenuitem}

\constfunc{int}{FindMenuItem}{\param{const wxString\& }{menuString}, \param{const wxString\& }{itemString}}

Finds the menu item id for a menu name/menu item string pair.

\wxheading{Parameters}

\docparam{menuString}{Menu title to find.}

\docparam{itemString}{Item to find.}

\wxheading{Return value}

The menu item identifier, or wxNOT\_FOUND if none was found.

\wxheading{Remarks}

Any special menu codes are stripped out of source and target strings
before matching.

\membersection{wxMenuBar::FindItem}\label{wxmenubarfinditem}

\constfunc{wxMenuItem *}{FindItem}{\param{int}{ id}, \param{wxMenu}{ **menu = NULL}}

Finds the menu item object associated with the given menu item identifier.

\wxheading{Parameters}

\docparam{id}{Menu item identifier.}

\docparam{menu}{If not NULL, menu will get set to the associated menu.}

\wxheading{Return value}

The found menu item object, or NULL if one was not found.

\membersection{wxMenuBar::GetHelpString}\label{wxmenubargethelpstring}

\constfunc{wxString}{GetHelpString}{\param{int}{ id}}

Gets the help string associated with the menu item identifer.

\wxheading{Parameters}

\docparam{id}{The menu item identifier.}

\wxheading{Return value}

The help string, or the empty string if there was no help string or the menu item
was not found.

\wxheading{See also}

\helpref{wxMenuBar::SetHelpString}{wxmenubarsethelpstring}

\membersection{wxMenuBar::GetLabel}\label{wxmenubargetlabel}

\constfunc{wxString}{GetLabel}{\param{int}{ id}}

Gets the label associated with a menu item.

\wxheading{Parameters}

\docparam{id}{The menu item identifier.}

\wxheading{Return value}

The menu item label, or the empty string if the item was not found.

\wxheading{Remarks}

Use only after the menubar has been associated with a frame.

\membersection{wxMenuBar::GetLabelTop}\label{wxmenubargetlabeltop}

\constfunc{wxString}{GetLabelTop}{\param{int}{ pos}}

Returns the label of a top-level menu.

\wxheading{Parameters}

\docparam{pos}{Position of the menu on the menu bar, starting from zero.}

\wxheading{Return value}

The menu label, or the empty string if the menu was not found.

\wxheading{Remarks}

Use only after the menubar has been associated with a frame.

\wxheading{See also}

\helpref{wxMenuBar::SetLabelTop}{wxmenubarsetlabeltop}

\membersection{wxMenuBar::GetMenu}\label{wxmenubargetmenu}

\constfunc{wxMenu*}{GetMenu}{\param{int}{ menuIndex}}

Returns the menu at {\it menuIndex} (zero-based).

\membersection{wxMenuBar::GetMenuCount}\label{wxmenubargetmenucount}

\constfunc{int}{GetMenuCount}{\void}

Returns the number of menus in this menubar.

\membersection{wxMenuBar::Insert}\label{wxmenubarinsert}

\func{bool}{Insert}{\param{size\_t }{pos}, \param{wxMenu *}{menu}, \param{const wxString\& }{title}}

Inserts the menu at the given position into the menu bar. Inserting menu at
position $0$ will insert it in the very beginning of it, inserting at position 
\helpref{GetMenuCount()}{wxmenubargetmenucount} is the same as calling 
\helpref{Append()}{wxmenubarappend}.

\wxheading{Parameters}

\docparam{pos}{The position of the new menu in the menu bar}

\docparam{menu}{The menu to add. wxMenuBar owns the menu and will free it.}

\docparam{title}{The title of the menu.}

\wxheading{Return value}

TRUE on success, FALSE if an error occurred.

\wxheading{See also}

\helpref{wxMenuBar::Append}{wxmenubarappend}

\membersection{wxMenuBar::IsChecked}\label{wxmenubarischecked}

\constfunc{bool}{IsChecked}{\param{int}{ id}}

Determines whether an item is checked.

\wxheading{Parameters}

\docparam{id}{The menu item identifier.}

\wxheading{Return value}

TRUE if the item was found and is checked, FALSE otherwise.

\membersection{wxMenuBar::IsEnabled}\label{wxmenubarisenabled}

\constfunc{bool}{IsEnabled}{\param{int}{ id}}

Determines whether an item is enabled.

\wxheading{Parameters}

\docparam{id}{The menu item identifier.}

\wxheading{Return value}

TRUE if the item was found and is enabled, FALSE otherwise.

\membersection{wxMenuBar::Refresh}\label{wxmenubarrefresh}

\func{void}{Refresh}{\void}

Redraw the menu bar

\membersection{wxMenuBar::Remove}\label{wxmenubarremove}

\func{wxMenu *}{Remove}{\param{size\_t }{pos}}

Removes the menu from the menu bar and returns the menu object - the caller is
reposnbile for deleting it. This function may be used together with 
\helpref{wxMenuBar::Insert}{wxmenubarinsert} to change the menubar
dynamically.

\wxheading{See also}

\helpref{wxMenuBar::Replace}{wxmenubarreplace}

\membersection{wxMenuBar::Replace}\label{wxmenubarreplace}

\func{wxMenu *}{Replace}{\param{size\_t }{pos}, \param{wxMenu *}{menu}, \param{const wxString\& }{title}}

Replaces the menu at the given position with another one.

\wxheading{Parameters}

\docparam{pos}{The position of the new menu in the menu bar}

\docparam{menu}{The menu to add.}

\docparam{title}{The title of the menu.}

\wxheading{Return value}

The menu which was previously at the position {\it pos}. The caller is
responsible for deleting it.

\wxheading{See also}

\helpref{wxMenuBar::Insert}{wxmenubarinsert},\rtfsp
\helpref{wxMenuBar::Remove}{wxmenubarremove}

\membersection{wxMenuBar::SetHelpString}\label{wxmenubarsethelpstring}

\func{void}{SetHelpString}{\param{int}{ id}, \param{const wxString\& }{helpString}}

Sets the help string associated with a menu item.

\wxheading{Parameters}

\docparam{id}{Menu item identifier.}

\docparam{helpString}{Help string to associate with the menu item.}

\wxheading{See also}

\helpref{wxMenuBar::GetHelpString}{wxmenubargethelpstring}

\membersection{wxMenuBar::SetLabel}\label{wxmenubarsetlabel}

\func{void}{SetLabel}{\param{int}{ id}, \param{const wxString\& }{label}}

Sets the label of a menu item.

\wxheading{Parameters}

\docparam{id}{Menu item identifier.}

\docparam{label}{Menu item label.}

\wxheading{Remarks}

Use only after the menubar has been associated with a frame.

\wxheading{See also}

\helpref{wxMenuBar::GetLabel}{wxmenubargetlabel}

\membersection{wxMenuBar::SetLabelTop}\label{wxmenubarsetlabeltop}

\func{void}{SetLabelTop}{\param{int}{ pos}, \param{const wxString\& }{label}}

Sets the label of a top-level menu.

\wxheading{Parameters}

\docparam{pos}{The position of a menu on the menu bar, starting from zero.}

\docparam{label}{The menu label.}

\wxheading{Remarks}

Use only after the menubar has been associated with a frame.

\wxheading{See also}

\helpref{wxMenuBar::GetLabelTop}{wxmenubargetlabeltop}

