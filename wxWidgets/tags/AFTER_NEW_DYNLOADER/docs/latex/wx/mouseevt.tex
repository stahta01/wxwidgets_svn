\section{\class{wxMouseEvent}}\label{wxmouseevent}

This event class contains information about mouse events.
See \helpref{wxWindow::OnMouseEvent}{wxwindowonmouseevent}.

{\bf NB: } Note that under Windows mouse enter and leave events are not natively supported
by the system but are generated by wxWindows itself. This has several
drawbacks: the LEAVE\_WINDOW event might be received some time after the mouse
left the window and the state variables for it may have changed during this
time.

{\bf NB: } Note the difference between methods like
\helpref{LeftDown}{wxmouseeventleftdown} and
\helpref{LeftIsDown}{wxmouseeventleftisdown}: the formet returns {\tt TRUE}
when the event corresponds to the left mouse button click while the latter
returns {\tt TRUE} if the left mouse button is currently being pressed. For
example, when the user is dragging the mouse you can use
\helpref{LeftIsDown}{wxmouseeventleftisdown} to test
whether the left mouse button is (still) depressed. Also, by convention, if
\helpref{LeftDown}{wxmouseeventleftdown} returns {\tt TRUE},
\helpref{LeftIsDown}{wxmouseeventleftisdown} will also return {\tt TRUE} in
wxWindows whatever the underlying GUI behaviour is (which is
platform-dependent). The same applies, of course, to other mouse buttons as
well.

\wxheading{Derived from}

\helpref{wxEvent}{wxevent}

\wxheading{Include files}

<wx/event.h>

\wxheading{Event table macros}

To process a mouse event, use these event handler macros to direct input to member
functions that take a wxMouseEvent argument.

\twocolwidtha{7cm}
\begin{twocollist}\itemsep=0pt
\twocolitem{{\bf EVT\_LEFT\_DOWN(func)}}{Process a wxEVT\_LEFT\_DOWN event.}
\twocolitem{{\bf EVT\_LEFT\_UP(func)}}{Process a wxEVT\_LEFT\_UP event.}
\twocolitem{{\bf EVT\_LEFT\_DCLICK(func)}}{Process a wxEVT\_LEFT\_DCLICK event.}
\twocolitem{{\bf EVT\_MIDDLE\_DOWN(func)}}{Process a wxEVT\_MIDDLE\_DOWN event.}
\twocolitem{{\bf EVT\_MIDDLE\_UP(func)}}{Process a wxEVT\_MIDDLE\_UP event.}
\twocolitem{{\bf EVT\_MIDDLE\_DCLICK(func)}}{Process a wxEVT\_MIDDLE\_DCLICK event.}
\twocolitem{{\bf EVT\_RIGHT\_DOWN(func)}}{Process a wxEVT\_RIGHT\_DOWN event.}
\twocolitem{{\bf EVT\_RIGHT\_UP(func)}}{Process a wxEVT\_RIGHT\_UP event.}
\twocolitem{{\bf EVT\_RIGHT\_DCLICK(func)}}{Process a wxEVT\_RIGHT\_DCLICK event.}
\twocolitem{{\bf EVT\_MOTION(func)}}{Process a wxEVT\_MOTION event.}
\twocolitem{{\bf EVT\_ENTER\_WINDOW(func)}}{Process a wxEVT\_ENTER\_WINDOW event.}
\twocolitem{{\bf EVT\_LEAVE\_WINDOW(func)}}{Process a wxEVT\_LEAVE\_WINDOW event.}
\twocolitem{{\bf EVT\_MOUSEWHEEL(func)}}{Process a wxEVT\_MOUSEWHEEL event.}
\twocolitem{{\bf EVT\_MOUSE\_EVENTS(func)}}{Process all mouse events.}
\end{twocollist}%

\latexignore{\rtfignore{\wxheading{Members}}}

\membersection{wxMouseEvent::m\_altDown}

\member{bool}{m\_altDown}

TRUE if the Alt key is pressed down.

\membersection{wxMouseEvent::m\_controlDown}

\member{bool}{m\_controlDown}

TRUE if control key is pressed down.

\membersection{wxMouseEvent::m\_leftDown}

\member{bool}{m\_leftDown}

TRUE if the left mouse button is currently pressed down.

\membersection{wxMouseEvent::m\_middleDown}

\member{bool}{m\_middleDown}

TRUE if the middle mouse button is currently pressed down.

\membersection{wxMouseEvent::m\_rightDown}

\member{bool}{m\_rightDown}

TRUE if the right mouse button is currently pressed down.

\membersection{wxMouseEvent::m\_metaDown}

\member{bool}{m\_metaDown}

TRUE if the Meta key is pressed down.

\membersection{wxMouseEvent::m\_shiftDown}

\member{bool}{m\_shiftDown}

TRUE if shift is pressed down.

\membersection{wxMouseEvent::m\_x}

\member{long}{m\_x}

X-coordinate of the event.

\membersection{wxMouseEvent::m\_y}

\member{long}{m\_y}

Y-coordinate of the event.

\membersection{wxMouseEvent::m\_wheelRotation}

\member{int}{m\_wheelRotation}

The distance the mouse wheel is rotated.

\membersection{wxMouseEvent::m\_wheelDelta}

\member{int}{m\_wheelDelta}

The wheel delta, normally 120.

\membersection{wxMouseEvent::m\_linesPerAction}

\member{int}{m\_linesPerAction}

The configured number of lines (or whatever) to be scrolled per wheel
action.


\membersection{wxMouseEvent::wxMouseEvent}

\func{}{wxMouseEvent}{\param{WXTYPE}{ mouseEventType = 0}, \param{int}{ id = 0}}

Constructor. Valid event types are:

\begin{itemize}
\itemsep=0pt
\item {\bf wxEVT\_ENTER\_WINDOW}
\item {\bf wxEVT\_LEAVE\_WINDOW}
\item {\bf wxEVT\_LEFT\_DOWN}
\item {\bf wxEVT\_LEFT\_UP}
\item {\bf wxEVT\_LEFT\_DCLICK}
\item {\bf wxEVT\_MIDDLE\_DOWN}
\item {\bf wxEVT\_MIDDLE\_UP}
\item {\bf wxEVT\_MIDDLE\_DCLICK}
\item {\bf wxEVT\_RIGHT\_DOWN}
\item {\bf wxEVT\_RIGHT\_UP}
\item {\bf wxEVT\_RIGHT\_DCLICK}
\item {\bf wxEVT\_MOTION}
\item {\bf wxEVT\_MOUSEWHEEL}
\end{itemize}

\membersection{wxMouseEvent::AltDown}

\func{bool}{AltDown}{\void}

Returns TRUE if the Alt key was down at the time of the event.

\membersection{wxMouseEvent::Button}

\func{bool}{Button}{\param{int}{ button}}

Returns TRUE if the identified mouse button is changing state. Valid
values of {\it button} are 1, 2 or 3 for left, middle and right
buttons respectively.

Not all mice have middle buttons so a portable application should avoid
this one.

\membersection{wxMouseEvent::ButtonDClick}\label{buttondclick}

\func{bool}{ButtonDClick}{\param{int}{ but = -1}}

If the argument is omitted, this returns TRUE if the event was a mouse
double click event. Otherwise the argument specifies which double click event
was generated (1, 2 or 3 for left, middle and right buttons respectively).

\membersection{wxMouseEvent::ButtonDown}

\func{bool}{ButtonDown}{\param{int}{ but = -1}}

If the argument is omitted, this returns TRUE if the event was a mouse
button down event. Otherwise the argument specifies which button-down event
was generated (1, 2 or 3 for left, middle and right buttons respectively).

\membersection{wxMouseEvent::ButtonUp}

\func{bool}{ButtonUp}{\param{int}{ but = -1}}

If the argument is omitted, this returns TRUE if the event was a mouse
button up event. Otherwise the argument specifies which button-up event
was generated (1, 2 or 3 for left, middle and right buttons respectively).

\membersection{wxMouseEvent::ControlDown}

\func{bool}{ControlDown}{\void}

Returns TRUE if the control key was down at the time of the event.

\membersection{wxMouseEvent::Dragging}

\func{bool}{Dragging}{\void}

Returns TRUE if this was a dragging event (motion while a button is depressed).

\membersection{wxMouseEvent::Entering}\label{wxmouseevententering}

\func{bool}{Entering}{\void}

Returns TRUE if the mouse was entering the window.

See also \helpref{wxMouseEvent::Leaving}{wxmouseeventleaving}.

\membersection{wxMouseEvent::GetPosition}\label{wxmouseeventgetposition}

\constfunc{wxPoint}{GetPosition}{\void}

\constfunc{void}{GetPosition}{\param{wxCoord*}{ x}, \param{wxCoord*}{ y}}

\constfunc{void}{GetPosition}{\param{long*}{ x}, \param{long*}{ y}}

Sets *x and *y to the position at which the event occurred.

Returns the physical mouse position in pixels.

\membersection{wxMouseEvent::GetLogicalPosition}\label{wxmouseeventgetlogicalposition}

\constfunc{wxPoint}{GetLogicalPosition}{\param{const wxDC\&}{ dc}}

Returns the logical mouse position in pixels (i.e. translated according to the
translation set for the DC, which usually indicates that the window has been scrolled).


\membersection{wxMouseEvent::GetLinesPerAction}\label{wxmouseeventgetlinesperaction}

\constfunc{int}{GetLinesPerAction}{\void}

Returns the configured number of lines (or whatever) to be scrolled per
wheel action.  Defaults to three.

\membersection{wxMouseEvent::GetWheelRotation}\label{wxmouseeventgetwheelrotation}

\constfunc{int}{GetWheelRotation}{\void}

Get wheel rotation, positive or negative indicates direction of
rotation.  Current devices all send an event when rotation is equal to
+/-WheelDelta, but this allows for finer resolution devices to be
created in the future.  Because of this you shouldn't assume that one
event is equal to 1 line or whatever, but you should be able to either
do partial line scrolling or wait until +/-WheelDelta rotation values
have been accumulated before scrolling.

\membersection{wxMouseEvent::GetWheelDelta}\label{wxmouseeventgetwheeldelta}

\constfunc{int}{GetWheelDelta}{\void}

Get wheel delta, normally 120.  This is the threshold for action to be
taken, and one such action (for example, scrolling one increment)
should occur for each delta.

\membersection{wxMouseEvent::GetX}\label{wxmouseeventgetx}

\constfunc{long}{GetX}{\void}

Returns X coordinate of the physical mouse event position.

\membersection{wxMouseEvent::GetY}\label{wxmouseeventgety}

\func{long}{GetY}{\void}

Returns Y coordinate of the physical mouse event position.

\membersection{wxMouseEvent::IsButton}

\constfunc{bool}{IsButton}{\void}

Returns TRUE if the event was a mouse button event (not necessarily a button down event -
that may be tested using {\it ButtonDown}).

\membersection{wxMouseEvent::Leaving}\label{wxmouseeventleaving}

\constfunc{bool}{Leaving}{\void}

Returns TRUE if the mouse was leaving the window.

See also \helpref{wxMouseEvent::Entering}{wxmouseevententering}.

\membersection{wxMouseEvent::LeftDClick}

\constfunc{bool}{LeftDClick}{\void}

Returns TRUE if the event was a left double click.

\membersection{wxMouseEvent::LeftDown}\label{wxmouseeventleftdown}

\constfunc{bool}{LeftDown}{\void}

Returns TRUE if the left mouse button changed to down.

\membersection{wxMouseEvent::LeftIsDown}\label{wxmouseeventleftisdown}

\constfunc{bool}{LeftIsDown}{\void}

Returns TRUE if the left mouse button is currently down, independent
of the current event type.

Please notice that it is {\bf not} the same as
\helpref{LeftDown}{wxmouseeventleftdown} which returns TRUE if the left mouse
button was just pressed. Rather, it describes the state of the mouse button
before the event happened.

This event is usually used in the mouse event handlers which process "move
mouse" messages to determine whether the user is (still) dragging the mouse.

\membersection{wxMouseEvent::LeftUp}

\constfunc{bool}{LeftUp}{\void}

Returns TRUE if the left mouse button changed to up.

\membersection{wxMouseEvent::MetaDown}

\constfunc{bool}{MetaDown}{\void}

Returns TRUE if the Meta key was down at the time of the event.

\membersection{wxMouseEvent::MiddleDClick}

\constfunc{bool}{MiddleDClick}{\void}

Returns TRUE if the event was a middle double click.

\membersection{wxMouseEvent::MiddleDown}

\constfunc{bool}{MiddleDown}{\void}

Returns TRUE if the middle mouse button changed to down.

\membersection{wxMouseEvent::MiddleIsDown}\label{wxmouseeventmiddleisdown}

\constfunc{bool}{MiddleIsDown}{\void}

Returns TRUE if the middle mouse button is currently down, independent
of the current event type.

\membersection{wxMouseEvent::MiddleUp}

\constfunc{bool}{MiddleUp}{\void}

Returns TRUE if the middle mouse button changed to up.

\membersection{wxMouseEvent::Moving}

\constfunc{bool}{Moving}{\void}

Returns TRUE if this was a motion event (no buttons depressed).

\membersection{wxMouseEvent::RightDClick}

\constfunc{bool}{RightDClick}{\void}

Returns TRUE if the event was a right double click.

\membersection{wxMouseEvent::RightDown}

\constfunc{bool}{RightDown}{\void}

Returns TRUE if the right mouse button changed to down.

\membersection{wxMouseEvent::RightIsDown}\label{wxmouseeventrightisdown}

\constfunc{bool}{RightIsDown}{\void}

Returns TRUE if the right mouse button is currently down, independent
of the current event type.

\membersection{wxMouseEvent::RightUp}

\constfunc{bool}{RightUp}{\void}

Returns TRUE if the right mouse button changed to up.

\membersection{wxMouseEvent::ShiftDown}

\constfunc{bool}{ShiftDown}{\void}

Returns TRUE if the shift key was down at the time of the event.

