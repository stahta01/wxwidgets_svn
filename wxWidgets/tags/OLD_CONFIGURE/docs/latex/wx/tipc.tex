\section{Interprocess communication overview}\label{ipcoverview}

Classes: \helpref{wxDDEServer}{wxddeserver}, \helpref{wxDDEConnection}{wxddeconnection},
\rtfsp\helpref{wxDDEClient}{wxddeclient}.

TODO: rewrite.

The following describes how wxWindows implements DDE.  The following
three classes are central.

\begin{enumerate}\itemsep=0pt
\item wxDDEClient. This represents the client application, and is used
only within a client program.
\item wxDDEServer. This represents the server application, and is used
only within a server program.
\item wxDDEConnection. This represents the connection from the current
client or server to the other application (server or client), and can be used
in both server and client programs. Most DDE
transactions operate on this object.
\end{enumerate}

Messages between applications are usually identified by three variables:
connection object, topic name and item name.  A data string is a fourth
element of some messages. To create a connection (a conversation in
Windows parlance), the client application sends the message
MakeConnection to the client object, with a string service name to
identify the server and a topic name to identify the topic for the
duration of the connection. Under UNIX, the service name must contain an
integer port identifier.

The server then responds and either vetos the connection or allows it.
If allowed, a connection object is created which persists until the
connection is closed.  The connection object is then used for subsequent
messages between client and server.

To create a working server, the programmer must:

\begin{enumerate}\itemsep=0pt
\item Derive a class from wxDDEServer.
\item Override the handler OnAcceptConnection for accepting or rejecting a connection,
on the basis of the topic argument. This member must create and return a connection
object if the connection is accepted.
\item Create an instance of your server object, and call Create to
activate it, giving it a service name.
\item Derive a class from wxDDEConnection.
\item Provide handlers for various messages that are sent to the server
side of a wxDDEConnection.
\end{enumerate}

To create a working client, the programmer must:

\begin{enumerate}\itemsep=0pt
\item Derive a class from wxDDEClient.
\item Override the handler OnMakeConnection to create and return
an appropriate connection object.
\item Create an instance of your client object.
\item Derive a class from wxDDEConnection.
\item Provide handlers for various messages that are sent to the client
side of a wxDDEConnection.
\item When appropriate, create a new connection by sending a MakeConnection
message to the client object, with arguments host name (processed in UNIX only),
service name, and topic name for this connection. The client object will call OnMakeConnection
to create a connection object of the desired type.
\item Use the wxDDEConnection member functions to send messages to the server.
\end{enumerate}

\subsection{Data transfer}

These are the ways that data can be transferred from one application to
another.

\begin{itemize}\itemsep=0pt
\item {\bf Execute:} the client calls the server with a data string representing
a command to be executed. This succeeds or fails, depending on the
server's willingness to answer. If the client wants to find the result
of the Execute command other than success or failure, it has to explicitly
call Request.
\item {\bf Request:} the client asks the server for a particular data string
associated with a given item string. If the server is unwilling to
reply, the return value is NULL. Otherwise, the return value is a string
(actually a pointer to the connection buffer, so it should not be
deallocated by the application).
\item {\bf Poke:} The client sends a data string associated with an item
string directly to the server. This succeeds or fails.
\item {\bf Advise:} The client asks to be advised of any change in data
associated with a particular item. If the server agrees, the server will
send an OnAdvise message to the client along with the item and data.
\end{itemize}

The default data type is wxCF\_TEXT (ASCII text), and the default data
size is the length of the null-terminated string. Windows-specific data
types could also be used on the PC.

\subsection{Examples}

See the sample programs {\it server}\/ and {\it client}\/ in the IPC
samples directory.  Run the server, then the client. This demonstrates
using the Execute, Request, and Poke commands from the client, together
with an Advise loop: selecting an item in the server list box causes
that item to be highlighted in the client list box.

See also the source for wxHelp, which is a DDE server, and the files
wx\_help.h and wx\_help.cc which implement the client interface to
wxHelp.

\subsection{More DDE details}

A wxDDEClient object represents the client part of a client-server DDE
(Dynamic Data Exchange) conversation (available in both
Windows and UNIX).

To create a client which can communicate with a suitable server,
you need to derive a class from wxDDEConnection and another from wxDDEClient.
The custom wxDDEConnection class will intercept communications in
a `conversation' with a server, and the custom wxDDEServer is required
so that a user-overriden \helpref{wxDDEClient::OnMakeConnection}{wxddeclientonmakeconnection} member can return
a wxDDEConnection of the required class, when a connection is made.

For example:

\begin{verbatim}
class MyConnection: public wxDDEConnection
{
 public:
  MyConnection(void)::wxDDEConnection(ipc_buffer, 3999) {}
  ~MyConnection(void) { }
  Bool OnAdvise(char *topic, char *item, char *data, int size, int format)
  { wxMessageBox(topic, data); }
};

class MyClient: public wxDDEClient
{
 public:
  MyClient(void) {}
  wxDDEConnection *OnMakeConnection(void) { return new MyConnection; }
};

\end{verbatim}

Here, {\bf MyConnection} will respond to \helpref{OnAdvise}{wxddeconnectiononadvise} messages sent
by the server.

When the client application starts, it must first call \helpref{wxIPCInitialize}{wxipcinitialize}\rtfsp
before creating an instance of the derived wxDDEClient. In the following, command line
arguments are used to pass the host name (the name of the machine the server is running
on) and the server name (identifying the server process). Calling \helpref{wxDDEClient::MakeConnection}{wxddeclientmakeconnection}\rtfsp
implicitly creates an instance of {\bf MyConnection} if the request for a
connection is accepted, and the client then requests an {\it Advise} loop
from the server, where the server calls the client when data has changed.

\begin{verbatim}
  wxIPCInitialize();

  char *server = "4242";
  char hostName[256];
  wxGetHostName(hostName, sizeof(hostName));

  char *host = hostName;

  if (argc > 1)
    server = argv[1];
  if (argc > 2)
    host = argv[2];

  // Create a new client
  MyClient *client = new MyClient;
  the_connection = (MyConnection *)client->MakeConnection(host, server, "IPC TEST");

  if (!the_connection)
  {
    wxMessageBox("Failed to make connection to server", "Client Demo Error");
    return NULL;
  }
  the_connection->StartAdvise("Item");
\end{verbatim}


