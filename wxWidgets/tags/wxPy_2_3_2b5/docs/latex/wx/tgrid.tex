\section{wxGrid classes overview}\label{gridoverview}

Classes: \helpref{wxGrid}{wxgrid}

\subsection{Introduction}
wxGrid and its related classes are used for displaying and editing tabular data.

\subsection{Getting started: a simple example}
For simple applications you need only refer to the wxGrid class in your
code.  This example shows how you might create a grid in a frame or
dialog constructor and illustrates some of the formatting functions. 

\begin{verbatim}

    // Create a wxGrid object
    
    grid = new wxGrid( this,
                       -1,
                       wxPoint( 0, 0 ),
                       wxSize( 400, 300 ) );

    // Then we call CreateGrid to set the dimensions of the grid
    // (100 rows and 10 columns in this example)
    grid->CreateGrid( 100, 10 );

    // We can set the sizes of individual rows and columns
    // in pixels
    grid->SetRowSize( 0, 60 );
    grid->SetColSize( 0, 120 );
    
    // And set grid cell contents as strings
    grid->SetCellValue( 0, 0, "wxGrid is good" );

    // We can specify that some cells are read-only
    grid->SetCellValue( 0, 3, "This is read-only" );
    grid->SetReadOnly( 0, 3 );

    // Colours can be specified for grid cell contents
    grid->SetCellValue(3, 3, "green on grey");
    grid->SetCellTextColour(3, 3, *wxGREEN);
    grid->SetCellBackgroundColour(3, 3, *wxLIGHT_GREY);

    // We can specify the some cells will store numeric 
    // values rather than strings. Here we set grid column 5 
    // to hold floating point values displayed with width of 6 
    // and precision of 2
    grid->SetColFormatFloat(5, 6, 2);
    grid->SetCellValue(0, 6, "3.1415");

\end{verbatim}

\subsection{A more complex example}
Yet to be written

\subsection{How the wxGrid classes relate to each other}
Yet to be written

\subsection{Keyboard and mouse actions}
Yet to be written

