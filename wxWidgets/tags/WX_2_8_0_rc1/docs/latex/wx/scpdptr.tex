\section{\class{wxScopedPtr}}\label{wxscopedptr}

This is a simple scoped smart pointer implementation that is similar to 
the \urlref{Boost}{http://www.boost.org/} smart pointers but rewritten to
use macros instead.

A smart pointer holds a pointer to an object. The memory used by the object is
deleted when the smart pointer goes out of scope. This class is different from
the \texttt{std::auto\_ptr<>} in so far as it doesn't provide copy constructor
nor assignment operator. This limits what you can do with it but is much less
surprizing than the ``destructive copy'' behaviour of the standard class.

\wxheading{Example}

Below is an example of using a wxWidgets scoped smart pointer and 
pointer array.

\begin{verbatim}
  class MyClass { /* ... */ };

  // declare a smart pointer to a MyClass called wxMyClassPtr
  wxDECLARE_SCOPED_PTR(MyClass, wxMyClassPtr)
  // declare a smart pointer to an array of chars
  wxDECLARE_SCOPED_ARRAY(char, wxCharArray)

  ...

  // define the first pointer class, must be complete
  wxDEFINE_SCOPED_PTR(MyClass, wxMyClassPtr)
  // define the second pointer class
  wxDEFINE_SCOPED_ARRAY(char, wxCharArray)

  // create an object with a new pointer to MyClass
  wxMyClassPtr theObj(new MyClass());
  // reset the pointer (deletes the previous one)
  theObj.reset(new MyClass());

  // access the pointer
  theObj->MyFunc();

  // create an object with a new array of chars
  wxCharArray theCharObj(new char[100]);

  // access the array
  theCharObj[0] = "!";
\end{verbatim}

\wxheading{Declaring new smart pointer types}

To declare the smart pointer class \texttt{CLASSNAME} containing pointes to a
(possibly incomplete) type \texttt{TYPE} you should use

\begin{verbatim}
    wxDECLARE_SCOPED_PTR( TYPE,     // type of the values
                                CLASSNAME ); // name of the class
\end{verbatim}

And later, when \texttt{TYPE} is fully defined, you must also use

\begin{verbatim}
    wxDEFINE_SCOPED_PTR( TYPE, CLASSNAME );
\end{verbatim}
to implement the scoped pointer class.

The first argument of these macro is the pointer type, the second is the name
of the new smart pointer class being created.  Below we will use wxScopedPtr to
represent the scoped pointer class, but the user may create the class with any
legal name.

Alternatively, if you don't have to separate the point of declaration and
definition of this class and if you accept the standard naming convention, that
is that the scoped pointer for the class \texttt{Foo} is called 
\texttt{FooPtr}, you can use a single macro which replaces two macros above:

\begin{verbatim}
    wxDEFINE_SCOPED_PTR_TYPE( TYPE );
\end{verbatim}

Once again, in this cass \texttt{CLASSNAME} will be \texttt{TYPEPtr}.

\wxheading{Include files}

<wx/ptr\_scpd.h>

\wxheading{See also}

\helpref{wxScopedArray}{wxscopedarray}\rtfsp

\latexignore{\rtfignore{\wxheading{Members}}}

\membersection{wxScopedPtr::wxScopedPtr}\label{wxscopedptrctor}

\func{}{explicit wxScopedPtr}{\param{type}{ * T = NULL}}

Creates the smart pointer with the given pointer or none if {\tt NULL}.  On
compilers that support it, this uses the explicit keyword.


\membersection{wxScopedPtr::\destruct{wxScopedPtr}}\label{wxscopedptrdtor}

\func{}{\destruct{wxScopedPtr}}{\void}

Destructor frees the pointer help by this object if it is not {\tt NULL}.


\membersection{wxScopedPtr::release}\label{wxscopedptrrelease}

\func{T *}{release}{\void}

Returns the currently hold pointer and resets the smart pointer object to 
{\tt NULL}. After a call to this function the caller is responsible for
deleting the pointer.


\membersection{wxScopedPtr::reset}\label{wxscopedptrreset}

\func{\void}{reset}{\param{T}{ p * = NULL}}

Deletes the currently held pointer and sets it to {\it p} or to NULL if no 
arguments are specified. This function does check to make sure that the
pointer you are assigning is not the same pointer that is already stored.


\membersection{wxScopedPtr::operator *}\label{wxscopedptrptr}

\func{const T\&}{operator *}{\void}

This operator works like the standard C++ pointer operator to return the object
being pointed to by the pointer.  If the pointer is NULL or invalid this will
crash.


\membersection{wxScopedPtr::operator -$>$}\label{wxscopedptrref}

\func{const T*}{operator -$>$}{\void} % TODO

This operator works like the standard C++ pointer operator to return the pointer
in the smart pointer or NULL if it is empty.


\membersection{wxScopedPtr::get}\label{wxscopedptrget}

\func{const T*}{get}{\void}

This operator gets the pointer stored in the smart pointer or returns NULL if
there is none.


\membersection{wxScopedPtr::swap}\label{wxscopedptrswap}

\func{\void}{swap}{\param{wxScopedPtr}{ \& other}}

Swap the pointer inside the smart pointer with {\it other}. The pointer being
swapped must be of the same type (hence the same class name).




%%%%%%% wxScopedTiedPtr %%%%%%%
\section{\class{wxScopedTiedPtr}}\label{wxscopedtiedptr}

This is a variation on the topic of \helpref{wxScopedPtr}{wxscopedptr}. This
class is also a smart pointer but in addition it ``ties'' the pointer value to
another variable. In other words, during the life time of this class the value
of that variable is set to be the same as the value of the pointer itself and
it is reset to its old value when the object is destroyed. This class is
especially useful when converting the existing code (which may already store
the pointers value in some variable) to the smart pointers.

\wxheading{Example}

\wxheading{Derives from}

\helpref{wxScopedPtr}{wxscopedptr}

\wxheading{Include files}

<wx/ptr\_scpd.h>

\latexignore{\rtfignore{\wxheading{Members}}}

\membersection{wxScopedTiedPtr::wxScopedTiedPtr}\label{wxscopedtiedptrctor}

\func{}{wxScopedTiedPtr}{\param{T **}{ppTie}, \param{T *}{ptr}}

Constructor creates a smart pointer initialized with \arg{ptr} and stores 
\arg{ptr} in the location specified by \arg{ppTie} which must not be 
{\tt NULL}.

\membersection{wxScopedTiedPtr::\destruct{wxScopedTiedPtr}}\label{wxscopedtiedptrdtor}

\func{}{\destruct{wxScopedTiedPtr}}{\void}

Destructor frees the pointer help by this object and restores the value stored
at the tied location (as specified in the \helpref{constructor}{wxscopedtiedptrctor})
to the old value.

Warning: this location may now contain an uninitialized value if it hadn't been
initialized previously, in particular don't count on it magically being 
{\tt NULL}!


