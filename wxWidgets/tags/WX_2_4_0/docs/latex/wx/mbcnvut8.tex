%
% automatically generated by HelpGen from
% ../include/wx/strconv.h at 25/Mar/00 10:20:56
%

\section{\class{wxMBConvUTF8}}\label{wxmbconvutf8}

This class converts between the UTF-8 encoding and Unicode.
It has one predefined instance, {\bf wxConvUTF8}.

\wxheading{Derived from}

\helpref{wxMBConv}{wxmbconv}

\wxheading{Include files}

<wx/strconv.h>

\wxheading{See also}

\helpref{wxMBConvUTF7}{wxmbconvutf7}, 
\helpref{wxMBConv classes overview}{mbconvclasses}

\wxheading{Remarks}

UTF-8 is a compatibility encoding used to encode Unicode text into anything that was
originally written for 8-bit strings, including (but not limited to) filenames, transfer
protocols, and database fields. Notable properties include:

\begin{itemize}
\item Variable-length encoding able to encode up to 31 bits per character
\item ASCII characters (character values under 128) are encoded as plain ASCII
(1 byte per character)
\item Null bytes do not occur in the encoding, except when there's an actual Unicode
null character
\item Preserves sort ordering for plain 8-bit comparison routines like strcmp()
\item High bit patterns disambiguates character boundaries, and makes it easy to
detect whether a string is encoded with UTF-8 or not
\end{itemize}

All of these properties make UTF-8 a very favorable solution in any situation
where full Unicode character support is desired while remaining compatible with
code written with only 8-bit extended-ASCII characters in mind.

\latexignore{\rtfignore{\wxheading{Members}}}


\membersection{wxMBConvUTF8::MB2WC}\label{wxmbconvutf8mb2wc}

\constfunc{size\_t}{MB2WC}{\param{wchar\_t* }{buf}, \param{const char* }{psz}, \param{size\_t }{n}}

Converts from UTF-8 encoding to Unicode. Returns the size of the destination buffer.

\membersection{wxMBConvUTF8::WC2MB}\label{wxmbconvutf8wc2mb}

\constfunc{size\_t}{WC2MB}{\param{char* }{buf}, \param{const wchar\_t* }{psz}, \param{size\_t }{n}}

Converts from Unicode to UTF-8 encoding. Returns the size of the destination buffer.

