\section{\class{wxSize}}\label{wxsize}

A {\bf wxSize} is a useful data structure for graphics operations.
It simply contains integer {\it width} and {\it height} members.

wxSize is used throughout wxWidgets as well as wxPoint which, although almost
equivalent to wxSize, has a different meaning: wxPoint represents a position
while wxSize - the size.

\pythonnote{wxPython defines aliases for the {\tt x} and {\tt y} members
named {\tt width} and {\tt height} since it makes much more sense for
sizes.
}

\wxheading{Derived from}

None

\wxheading{Include files}

<wx/gdicmn.h>

\wxheading{See also}

\helpref{wxPoint}{wxpoint}, \helpref{wxRealPoint}{wxrealpoint}

\latexignore{\rtfignore{\wxheading{Members}}}

\membersection{wxSize::wxSize}

\func{}{wxSize}{\void}

\func{}{wxSize}{\param{int}{ width}, \param{int}{ height}}

Creates a size object.

\membersection{wxSize::GetWidth}\label{wxsizegetwidth}

\constfunc{int}{GetWidth}{\void}

Gets the width member.

\membersection{wxSize::GetHeight}\label{wxsizegetheight}

\constfunc{int}{GetHeight}{\void}

Gets the height member.

\membersection{wxSize::Set}\label{wxsizeset}

\func{void}{Set}{\param{int}{ width}, \param{int}{ height}}

Sets the width and height members.

\membersection{wxSize::SetHeight}\label{wxsizesetheight}

\func{void}{SetHeight}{\param{int}{ height}}

Sets the height.

\membersection{wxSize::SetWidth}\label{wxsizesetwidth}

\func{void}{SetWidth}{\param{int}{ width}}

Sets the width.

\membersection{wxSize::operator $=$}

\func{void}{operator $=$}{\param{const wxSize\& }{sz}}

Assignment operator.


