%%%%%%%%%%%%%%%%%%%%%%%%%%%%%%%%%%%%%%%%%%%%%%%%%%%%%%%%%%%%%%%%%%%%%%%%%%%%%%%
%% Name:        membuf.tex
%% Purpose:     wxMemoryBuffer class documentation
%% Author:      Roger Gammans
%% Modified by:
%% Created:     08.06.2004
%% RCS-ID:      $Id$
%% Copyright:   (c) Roger Gammans
%% License:     wxWindows license
%%%%%%%%%%%%%%%%%%%%%%%%%%%%%%%%%%%%%%%%%%%%%%%%%%%%%%%%%%%%%%%%%%%%%%%%%%%%%%%

\section{\class{wxMemoryBuffer}}\label{wxmemorybuffer}

A {\bf wxMemoryBuffer} is a useful data structure for storing arbitrary sized blocks
of memory. wxMemoryBuffer guarantees deletion of the memory block when the object
is destroyed. 

\wxheading{Derived from}

None

\wxheading{Include files}

<wx/buffer.h>

\latexignore{\rtfignore{\wxheading{Members}}}

\membersection{wxMemoryBuffer::wxMemoryBuffer}\label{wxmemorybufferctor}

\func{}{wxMemoryBuffer}{\param{const wxMemoryBuffer\&}{ src}}

Copy constructor, refcounting is used for performance , but wxMemoryBuffer
is not a copy-on-write structure so changes made to one buffer effect
all copies made from it.

\func{}{wxMemoryBuffer}{\param{size\_t}{ size}}

Create a new buffer.

\docparam{size}{size of new buffer.}

\membersection{wxMemoryBuffer::GetData}\label{wxmemorybuffergetdata}

\func{void* }{GetData}{\void}

Return a pointer to the data in the buffer.

\membersection{wxMemoryBuffer::GetBufSize}\label{wxmemorybuffergetbufsize}

\func{size\_t}{GetBufSize}{\void}

Returns the size of the buffer.

\membersection{wxMemoryBuffer::GetDataLen}\label{wxmemorybuffergetdatalen}

\func{size\_t}{GetDataLen}{\void}

Returns the length of the valid data in the buffer.

\membersection{wxMemoryBuffer::SetBufSize}\label{wxmemorybuffersetbufsize}

\func{void}{SetBufSize}{\param{size\_t}{ size}}

Ensures the buffer has at least {\it size} bytes available.

\membersection{wxMemoryBuffer::SetDataLen}\label{wxmemorybuffersetdatalen}

\func{void}{SetDataLen}{\param{size\_t}{ size}}

Sets the length of the data stored in the buffer.  Mainly useful for truncating existing data.

\docparam{size}{New length of the valid data in the buffer. This is
distinct from the allocated size}

\membersection{wxMemoryBuffer::GetWriteBuf}\label{wxmemorybuffergetwritebuf}

\func{void *}{GetWriteBuf}{\param{size\_t}{ sizeNeeded}}

Ensure the buffer is big enough and return a pointer to the
buffer which can be used to directly write into the buffer
up to {\it sizeNeeded} bytes.

\membersection{wxMemoryBuffer::UngetWriteBuf}\label{wxmemorybufferungetwritebuf}

\func{void}{UngetWriteBuf}{\param{size\_t}{ sizeUsed}}

Update the buffer after completing a direct write, which
you must have used GetWriteBuf() to initialise.

\docparam{sizeUsed}{The amount of data written in to buffer
by the direct write}

\membersection{wxMemoryBuffer::GetAppendBuf}\label{wxmemorybuffergetappendbuf}

\func{void *}{GetAppendBuf}{\param{size\_t}{ sizeNeeded}}

Ensure that the buffer is big enough and return a pointer to the start
of the empty space in the buffer. This pointer can be used to directly 
write data into the buffer, this new data will be appended to
the existing data.

\docparam{sizeNeeded}{Amount of extra space required in the buffer for
the append operation}

\membersection{wxMemoryBuffer::UngetAppendBuf}\label{wxmemorybufferungetappendbuf}

\func{void}{UngetAppendBuf}{\param{size\_t}{ sizeUsed}}

Update the length after completing a direct append, which
you must have used GetAppendBuf() to initialise.

\docparam{sizeUsed}{This is the amount of new data that has been 
appended.}

\membersection{wxMemoryBuffer::AppendByte}\label{wxmemorybufferappendbyte}

\func{void}{AppendByte}{\param{char}{ data}}

Append a single byte to the buffer.

\docparam{data}{New byte to append to the buffer.}

\membersection{wxMemoryBuffer::AppendData}\label{wxmemorybufferappenddata}

\func{void}{AppendData}{\param{void*}{ data}, \param{size\_t}{ len}}

Single call to append a data block to the buffer.

\docparam{data}{Pointer to block to append to the buffer.}
\docparam{len}{Length of data to append.}
