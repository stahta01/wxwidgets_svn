% ----------------------------------------------------------------------------
% wxTextInputStream
% ----------------------------------------------------------------------------
\section{\class{wxTextInputStream}}\label{wxtextinputstream}

This class provides functions that read text datas using an input stream.
So, you can read {\it text} floats, integers.

The wxTextInputStream correctly reads text files (or streams) in DOS, Macintosh
and Unix formats and reports a single newline char as a line ending.

Operator >> is overloaded and you can use this class like a standard C++ iostream.
Note, however, that the arguments are the fixed size types wxUint32, wxInt32 etc
and on a typical 32-bit computer, none of these match to the "long" type (wxInt32
is defined as int on 32-bit architectures) so that you cannot use long. To avoid
problems (here and elsewhere), make use of wxInt32, wxUint32 and similar types.

If you're scanning through a file using wxTextInputStream, you should check for EOF {\bf before}
reading the next item (word / number), because otherwise the last item may get lost. 
You should however be prepared to receive an empty item (empty string / zero number) at the
end of file, especially on Windows systems. This is unavoidable because most (but not all) files end
with whitespace (i.e. usually a newline).

For example:

\begin{verbatim}
  wxFileInputStream input( "mytext.txt" );
  wxTextInputStream text( input );
  wxUint8 i1;
  float f2;
  wxString line;

  text >> i1;       // read a 8 bit integer.
  text >> i1 >> f2; // read a 8 bit integer followed by float.
  text >> line;     // read a text line
\end{verbatim}

\wxheading{Include files}

<wx/txtstrm.h>

\latexignore{\rtfignore{\wxheading{Members}}}


\membersection{wxTextInputStream::wxTextInputStream}\label{wxtextinputstreamctor}

\func{}{wxTextInputStream}{\param{wxInputStream\&}{ stream}, \param{const wxString\&}{ sep=" $\backslash$t"}, 
  \param{wxMBConv\&}{ conv = wxConvUTF8} }

Constructs a text stream associated to the given input stream.

\wxheading{Parameters}

\docparam{stream}{The underlying input stream.}

\docparam{sep}{The initial string separator characters.}

\docparam{conv}{{\it In Unicode build only:} The encoding converter used to convert the bytes in the
  underlying input stream to characters.}


\membersection{wxTextInputStream::\destruct{wxTextInputStream}}\label{wxtextinputstreamdtor}

\func{}{\destruct{wxTextInputStream}}{\void}

Destroys the wxTextInputStream object.


\membersection{wxTextInputStream::Read8}\label{wxtextinputstreamread8}

\func{wxUint8}{Read8}{\param{int}{ base = 10}}

Reads a single unsigned byte from the stream, given in base {\it base}.

The value of {\it base} must be comprised between $2$ and $36$, inclusive, or
be a special value $0$ which means that the usual rules of {\tt C} numbers are
applied: if the number starts with {\tt 0x} it is considered to be in base
$16$, if it starts with {\tt 0} - in base $8$ and in base $10$ otherwise. Note
that you may not want to specify the base $0$ if you are parsing the numbers
which may have leading zeroes as they can yield unexpected (to the user not
familiar with C) results.


\membersection{wxTextInputStream::Read8S}\label{wxtextinputstreamread8s}

\func{wxInt8}{Read8S}{\param{int}{ base = 10}}

Reads a single signed byte from the stream.

See \helpref{wxTextInputStream::Read8}{wxtextinputstreamread8} for the
description of the {\it base} parameter.


\membersection{wxTextInputStream::Read16}\label{wxtextinputstreamread16}

\func{wxUint16}{Read16}{\param{int}{ base = 10}}

Reads a unsigned 16 bit integer from the stream.

See \helpref{wxTextInputStream::Read8}{wxtextinputstreamread8} for the
description of the {\it base} parameter.


\membersection{wxTextInputStream::Read16S}\label{wxtextinputstreamread16s}

\func{wxInt16}{Read16S}{\param{int}{ base = 10}}

Reads a signed 16 bit integer from the stream.

See \helpref{wxTextInputStream::Read8}{wxtextinputstreamread8} for the
description of the {\it base} parameter.


\membersection{wxTextInputStream::Read32}\label{wxtextinputstreamread32}

\func{wxUint32}{Read32}{\param{int}{ base = 10}}

Reads a 32 bit unsigned integer from the stream.

See \helpref{wxTextInputStream::Read8}{wxtextinputstreamread8} for the
description of the {\it base} parameter.


\membersection{wxTextInputStream::Read32S}\label{wxtextinputstreamread32s}

\func{wxInt32}{Read32S}{\param{int}{ base = 10}}

Reads a 32 bit signed integer from the stream.

See \helpref{wxTextInputStream::Read8}{wxtextinputstreamread8} for the
description of the {\it base} parameter.


\membersection{wxTextInputStream::GetChar}\label{wxtextinputstreamgetchar}

\func{wxChar}{GetChar}{\void}

Reads a character, returns $0$ if there are no more characters in the stream.


\membersection{wxTextInputStream::ReadDouble}\label{wxtextinputstreamreaddouble}

\func{double}{ReadDouble}{\void}

Reads a double (IEEE encoded) from the stream.


\membersection{wxTextInputStream::ReadLine}\label{wxtextinputstreamreadline}

\func{wxString}{ReadLine}{\void}

Reads a line from the input stream and returns it (without the end of line
character).


\membersection{wxTextInputStream::ReadString}\label{wxtextinputstreamreadstring}

\func{wxString}{ReadString}{\void}

{\bf NB:} This method is deprecated, use \helpref{ReadLine}{wxtextinputstreamreadline} 
or \helpref{ReadWord}{wxtextinputstreamreadword} instead.

Same as \helpref{ReadLine}{wxtextinputstreamreadline}.


\membersection{wxTextInputStream::ReadWord}\label{wxtextinputstreamreadword}

\func{wxString}{ReadWord}{\void}

Reads a word (a sequence of characters until the next separator) from the
input stream.

\wxheading{See also}

\helpref{SetStringSeparators}{wxtextinputstreamsetstringseparators}


\membersection{wxTextInputStream::SetStringSeparators}\label{wxtextinputstreamsetstringseparators}

\func{void}{SetStringSeparators}{\param{const wxString\& }{sep}}

Sets the characters which are used to define the word boundaries in 
\helpref{ReadWord}{wxtextinputstreamreadword}.

The default separators are the space and {\tt TAB} characters.

% ----------------------------------------------------------------------------
% wxTextOutputStream
% ----------------------------------------------------------------------------

\section{\class{wxTextOutputStream}}\label{wxtextoutputstream}

This class provides functions that write text datas using an output stream.
So, you can write {\it text} floats, integers.

You can also simulate the C++ cout class:

\begin{verbatim}
  wxFFileOutputStream output( stderr );
  wxTextOutputStream cout( output );

  cout << "This is a text line" << endl;
  cout << 1234;
  cout << 1.23456;
\end{verbatim}

The wxTextOutputStream writes text files (or streams) on DOS, Macintosh
and Unix in their native formats (concerning the line ending).

\wxheading{Include files}

<wx/txtstrm.h>

\latexignore{\rtfignore{\wxheading{Members}}}


\membersection{wxTextOutputStream::wxTextOutputStream}\label{wxtextoutputstreamctor}

\func{}{wxTextOutputStream}{\param{wxOutputStream\&}{ stream}, \param{wxEOL}{ mode = wxEOL\_NATIVE}, \param{wxMBConv\&}{ conv = wxConvUTF8}}

Constructs a text stream object associated to the given output stream.

\wxheading{Parameters}

\docparam{stream}{The output stream.}

\docparam{mode}{The end-of-line mode. One of {\bf wxEOL\_NATIVE}, {\bf wxEOL\_DOS}, {\bf wxEOL\_MAC} and {\bf wxEOL\_UNIX}.}

\docparam{conv}{{\it In Unicode build only:} The object used to convert
Unicode text into ASCII characters written to the output stream.}


\membersection{wxTextOutputStream::\destruct{wxTextOutputStream}}\label{wxtextoutputstreamdtor}

\func{}{\destruct{wxTextOutputStream}}{\void}

Destroys the wxTextOutputStream object.


\membersection{wxTextOutputStream::GetMode}\label{wxtextoutputstreamgetmode}

\func{wxEOL}{GetMode}{\void}

Returns the end-of-line mode. One of {\bf wxEOL\_DOS}, {\bf wxEOL\_MAC} and {\bf wxEOL\_UNIX}.


\membersection{wxTextOutputStream::PutChar}\label{wxtextoutputstreamputchar}

\func{void}{PutChar}{{\param wxChar }{c}}

Writes a character to the stream.


\membersection{wxTextOutputStream::SetMode}\label{wxtextoutputstreamsetmode}

\func{void}{SetMode}{{\param wxEOL}{ mode = wxEOL\_NATIVE}}

Set the end-of-line mode. One of {\bf wxEOL\_NATIVE}, {\bf wxEOL\_DOS}, {\bf wxEOL\_MAC} and {\bf wxEOL\_UNIX}.


\membersection{wxTextOutputStream::Write8}\label{wxtextoutputstreamwrite8}

\func{void}{Write8}{{\param wxUint8 }{i8}}

Writes the single byte {\it i8} to the stream.


\membersection{wxTextOutputStream::Write16}\label{wxtextoutputstreamwrite16}

\func{void}{Write16}{{\param wxUint16 }{i16}}

Writes the 16 bit integer {\it i16} to the stream.


\membersection{wxTextOutputStream::Write32}\label{wxtextoutputstreamwrite32}

\func{void}{Write32}{{\param wxUint32 }{i32}}

Writes the 32 bit integer {\it i32} to the stream.


\membersection{wxTextOutputStream::WriteDouble}\label{wxtextoutputstreamwritedouble}

\func{virtual void}{WriteDouble}{{\param double }{f}}

Writes the double {\it f} to the stream using the IEEE format.


\membersection{wxTextOutputStream::WriteString}\label{wxtextoutputstreamwritestring}

\func{virtual void}{WriteString}{{\param const wxString\& }{string}}

Writes {\it string} as a line. Depending on the end-of-line mode the end of
line ('$\backslash$n') characters in the string are converted to the correct
line ending terminator.

