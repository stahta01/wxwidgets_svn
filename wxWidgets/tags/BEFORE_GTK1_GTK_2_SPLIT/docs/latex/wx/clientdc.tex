\section{\class{wxClientDC}}\label{wxclientdc}

A wxClientDC must be constructed if an application wishes to paint on the
client area of a window from outside an {\bf OnPaint} event.
This should normally be constructed as a temporary stack object; don't store
a wxClientDC object.

To draw on a window from within {\bf OnPaint}, construct a \helpref{wxPaintDC}{wxpaintdc} object.

To draw on the whole window including decorations, construct a \helpref{wxWindowDC}{wxwindowdc} object
(Windows only).

\wxheading{Derived from}

\helpref{wxWindowDC}{wxwindowdc}\\
\helpref{wxDC}{wxdc}

\wxheading{Include files}

<wx/dcclient.h>

\wxheading{See also}

\helpref{wxDC}{wxdc}, \helpref{wxMemoryDC}{wxmemorydc}, \helpref{wxPaintDC}{wxpaintdc},\rtfsp
\helpref{wxWindowDC}{wxwindowdc}, \helpref{wxScreenDC}{wxscreendc}

\latexignore{\rtfignore{\wxheading{Members}}}

\membersection{wxClientDC::wxClientDC}\label{wxclientdcctor}

\func{}{wxClientDC}{\param{wxWindow*}{ window}}

Constructor. Pass a pointer to the window on which you wish to paint.



