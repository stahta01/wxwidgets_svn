\section{\class{wxBitmapHandler}}\label{wxbitmaphandler}

\overview{Overview}{wxbitmapoverview}

This is the base class for implementing bitmap file loading/saving, and bitmap creation from data.
It is used within wxBitmap and is not normally seen by the application.

If you wish to extend the capabilities of wxBitmap, derive a class from wxBitmapHandler
and add the handler using \helpref{wxBitmap::AddHandler}{wxbitmapaddhandler} in your
application initialisation.

\wxheading{Derived from}

\helpref{wxObject}{wxobject}

\wxheading{Include files}

<wx/bitmap.h>

\wxheading{See also}

\helpref{wxBitmap}{wxbitmap}, \helpref{wxIcon}{wxicon}, \helpref{wxCursor}{wxcursor}

\latexignore{\rtfignore{\wxheading{Members}}}

\membersection{wxBitmapHandler::wxBitmapHandler}\label{wxbitmaphandlerconstr}

\func{}{wxBitmapHandler}{\void}

Default constructor. In your own default constructor, initialise the members
m\_name, m\_extension and m\_type.

\membersection{wxBitmapHandler::\destruct{wxBitmapHandler}}

\func{}{\destruct{wxBitmapHandler}}{\void}

Destroys the wxBitmapHandler object.

\membersection{wxBitmapHandler::Create}

\func{virtual bool}{Create}{\param{wxBitmap* }{bitmap}, \param{void*}{ data}, \param{int}{ type}, \param{int}{ width}, \param{int}{ height}, \param{int}{ depth = -1}}

Creates a bitmap from the given data, which can be of arbitrary type. The wxBitmap object {\it bitmap} is
manipulated by this function.

\wxheading{Parameters}

\docparam{bitmap}{The wxBitmap object.}

\docparam{width}{The width of the bitmap in pixels.}

\docparam{height}{The height of the bitmap in pixels.}

\docparam{depth}{The depth of the bitmap in pixels. If this is -1, the screen depth is used.}

\docparam{data}{Data whose type depends on the value of {\it type}.}

\docparam{type}{A bitmap type identifier - see \helpref{wxBitmapHandler::wxBitmapHandler}{wxbitmapconstr} for a list
of possible values.}

\wxheading{Return value}

true if the call succeeded, false otherwise (the default).

\membersection{wxBitmapHandler::GetName}

\constfunc{wxString}{GetName}{\void}

Gets the name of this handler.

\membersection{wxBitmapHandler::GetExtension}

\constfunc{wxString}{GetExtension}{\void}

Gets the file extension associated with this handler.

\membersection{wxBitmapHandler::GetType}

\constfunc{long}{GetType}{\void}

Gets the bitmap type associated with this handler.

\membersection{wxBitmapHandler::LoadFile}\label{wxbitmaphandlerloadfile}

\func{bool}{LoadFile}{\param{wxBitmap* }{bitmap}, \param{const wxString\&}{ name}, \param{long}{ type}}

Loads a bitmap from a file or resource, putting the resulting data into {\it bitmap}.

\wxheading{Parameters}

\docparam{bitmap}{The bitmap object which is to be affected by this operation.}

\docparam{name}{Either a filename or a Windows resource name.
The meaning of {\it name} is determined by the {\it type} parameter.}

\docparam{type}{See \helpref{wxBitmap::wxBitmap}{wxbitmapconstr} for values this can take.}

\wxheading{Return value}

true if the operation succeeded, false otherwise.

\wxheading{See also}

\helpref{wxBitmap::LoadFile}{wxbitmaploadfile}\\
\helpref{wxBitmap::SaveFile}{wxbitmapsavefile}\\
\helpref{wxBitmapHandler::SaveFile}{wxbitmaphandlersavefile}

\membersection{wxBitmapHandler::SaveFile}\label{wxbitmaphandlersavefile}

\func{bool}{SaveFile}{\param{wxBitmap* }{bitmap}, \param{const wxString\& }{name}, \param{int}{ type}, \param{wxPalette* }{palette = NULL}}

Saves a bitmap in the named file.

\wxheading{Parameters}

\docparam{bitmap}{The bitmap object which is to be affected by this operation.}

\docparam{name}{A filename. The meaning of {\it name} is determined by the {\it type} parameter.}

\docparam{type}{See \helpref{wxBitmap::wxBitmap}{wxbitmapconstr} for values this can take.}

\docparam{palette}{An optional palette used for saving the bitmap.}

\wxheading{Return value}

true if the operation succeeded, false otherwise.

\wxheading{See also}

\helpref{wxBitmap::LoadFile}{wxbitmaploadfile}\\
\helpref{wxBitmap::SaveFile}{wxbitmapsavefile}\\
\helpref{wxBitmapHandler::LoadFile}{wxbitmaphandlerloadfile}

\membersection{wxBitmapHandler::SetName}

\func{void}{SetName}{\param{const wxString\& }{name}}

Sets the handler name.

\wxheading{Parameters}

\docparam{name}{Handler name.}

\membersection{wxBitmapHandler::SetExtension}

\func{void}{SetExtension}{\param{const wxString\& }{extension}}

Sets the handler extension.

\wxheading{Parameters}

\docparam{extension}{Handler extension.}

\membersection{wxBitmapHandler::SetType}

\func{void}{SetType}{\param{long }{type}}

Sets the handler type.

\wxheading{Parameters}

\docparam{name}{Handler type.}



