% ----------------------------------------------------------------------------
% wxDataInputStream
% ----------------------------------------------------------------------------
\section{\class{wxDataInputStream}}\label{wxdatainputstream}

This class provides functions that read binary data types in a
portable way. Data can be read in either big-endian or litte-endian
format, little-endian being the default on all architectures.

If you want to read data from text files (or streams) use 
\helpref{wxTextInputStream}{wxtextinputstream} instead.

The >> operator is overloaded and you can use this class like a standard C++ iostream.
Note, however, that the arguments are the fixed size types wxUint32, wxInt32 etc
and on a typical 32-bit computer, none of these match to the "long" type (wxInt32
is defined as signed int on 32-bit architectures) so that you cannot use long. To avoid
problems (here and elsewhere), make use of the wxInt32, wxUint32, etc types.

For example:

\begin{verbatim}
  wxFileInputStream input( "mytext.dat" );
  wxDataInputStream store( input );
  wxUint8 i1;
  float f2;
  wxString line;

  store >> i1;       // read a 8 bit integer.
  store >> i1 >> f2; // read a 8 bit integer followed by float.
  store >> line;     // read a text line
\end{verbatim}

See also \helpref{wxDataOutputStream}{wxdataoutputstream}. 

\wxheading{Derived from}

None

\wxheading{Include files}

<wx/datstrm.h>

\latexignore{\rtfignore{\wxheading{Members}}}

\membersection{wxDataInputStream::wxDataInputStream}\label{wxdatainputstreamconstr}

\func{}{wxDataInputStream}{\param{wxInputStream\&}{ stream}}

Constructs a datastream object from an input stream. Only read methods will
be available.

\wxheading{Parameters}

\docparam{stream}{The input stream.}

\membersection{wxDataInputStream::\destruct{wxDataInputStream}}

\func{}{\destruct{wxDataInputStream}}{\void}

Destroys the wxDataInputStream object.

\membersection{wxDataInputStream::BigEndianOrdered}

\func{void}{BigEndianOrdered}{\param{bool}{ be\_order}}

If {\it be\_order} is TRUE, all data will be read in big-endian
order, such as written by programs on a big endian architecture 
(e.g. Sparc) or written by Java-Streams (which always use 
big-endian order).
  
\membersection{wxDataInputStream::Read8}

\func{wxUint8}{Read8}{\void}

Reads a single byte from the stream.

\membersection{wxDataInputStream::Read16}

\func{wxUint16}{Read16}{\void}

Reads a 16 bit integer from the stream.

\membersection{wxDataInputStream::Read32}

\func{wxUint32}{Read32}{\void}

Reads a 32 bit integer from the stream.

\membersection{wxDataInputStream::ReadDouble}

\func{double}{ReadDouble}{\void}

Reads a double (IEEE encoded) from the stream.

\membersection{wxDataInputStream::ReadString}

\func{wxString}{ReadString}{\void}

Reads a string from a stream. Actually, this function first reads a long integer
specifying the length of the string (without the last null character) and then
reads the string.

% ----------------------------------------------------------------------------
% wxDataOutputStream
% ----------------------------------------------------------------------------

\section{\class{wxDataOutputStream}}\label{wxdataoutputstream}

This class provides functions that write binary data types in a
portable way. Data can be written in either big-endian or litte-endian
format, little-endian being the default on all architectures.

If you want to write data to text files (or streams) use 
\helpref{wxTextOutputStream}{wxtextoutputstream} instead.

The << operator is overloaded and you can use this class like a standard 
C++ iostream. See \helpref{wxDataInputStream}{wxdatainputstream} for its 
usage and caveats.

See also \helpref{wxDataInputStream}{wxdatainputstream}. 

\wxheading{Derived from}

None

\latexignore{\rtfignore{\wxheading{Members}}}

\membersection{wxDataOutputStream::wxDataOutputStream}\label{wxdataoutputstreamconstr}

\func{}{wxDataOutputStream}{\param{wxOutputStream\&}{ stream}}

Constructs a datastream object from an output stream. Only write methods will
be available.

\wxheading{Parameters}

\docparam{stream}{The output stream.}

\membersection{wxDataOutputStream::\destruct{wxDataOutputStream}}

\func{}{\destruct{wxDataOutputStream}}{\void}

Destroys the wxDataOutputStream object.

\membersection{wxDataOutputStream::BigEndianOrdered}

\func{void}{BigEndianOrdered}{\param{bool}{ be\_order}}

If {\it be\_order} is TRUE, all data will be written in big-endian
order, e.g. for reading on a Sparc or from Java-Streams (which
always use big-endian order), otherwise data will be written in
little-endian order.
 
\membersection{wxDataOutputStream::Write8}

\func{void}{Write8}{{\param wxUint8 }{i8}}

Writes the single byte {\it i8} to the stream.

\membersection{wxDataOutputStream::Write16}

\func{void}{Write16}{{\param wxUint16 }{i16}}

Writes the 16 bit integer {\it i16} to the stream.

\membersection{wxDataOutputStream::Write32}

\func{void}{Write32}{{\param wxUint32 }{i32}}

Writes the 32 bit integer {\it i32} to the stream.

\membersection{wxDataOutputStream::WriteDouble}

\func{void}{WriteDouble}{{\param double }{f}}

Writes the double {\it f} to the stream using the IEEE format.

\membersection{wxDataOutputStream::WriteString}

\func{void}{WriteString}{{\param const wxString\& }{string}}

Writes {\it string} to the stream. Actually, this method writes the size of
the string before writing {\it string} itself.

