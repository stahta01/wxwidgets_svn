%%%%%%%%%%%%%%%%%%%%%%%%%%%%%%%%%%%%%%%%%%%%%%%%%%%%%%%%%%%%%%%%%%%%%%%%%%%%%%%
%% Name:        mediaevt.tex
%% Purpose:     wxMediaEvent docs
%% Author:      Ryan Norton <wxprojects@comcast.net>
%% Modified by:
%% Created:     11/7/2004
%% RCS-ID:      $Id$
%% Copyright:   (c) Ryan Norton
%% License:     wxWindows license
%%%%%%%%%%%%%%%%%%%%%%%%%%%%%%%%%%%%%%%%%%%%%%%%%%%%%%%%%%%%%%%%%%%%%%%%%%%%%%%

\section{\class{wxMediaEvent}}\label{wxmediaevent}

Event \helpref{wxMediaCtrl}{wxmediactrl} uses.

\wxheading{Derived from}

\helpref{wxNotifyEvent}{wxnotifyevent}

\wxheading{Include files}

<wx/mediactrl.h>

\wxheading{Event table macros}

\twocolwidtha{7cm}
\begin{twocollist}\itemsep=0pt
\twocolitem{{\bf EVT\_MEDIA\_LOADED(func)}}{Sent when a media has loaded enough data that it can start playing.}
\twocolitem{{\bf EVT\_MEDIA\_STOP(func)}}{
Triggerred right before the media stops.  You can Veto this event to prevent it from stopping, causing it to continue playing - even if it has reached that end of the media. }
\twocolitem{{\bf EVT\_MEDIA\_FINISHED(func)}}{Sent when a media has finished playing in a \helpref{wxMediaCtrl}{wxmediactrl}.  Note that if you connect to this event and don't skip it, it will override the looping behaviour of the media control.}
\end{twocollist}

\latexignore{\rtfignore{\wxheading{Members}}}
