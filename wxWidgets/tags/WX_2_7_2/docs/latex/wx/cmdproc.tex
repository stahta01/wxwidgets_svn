\section{\class{wxCommandProcessor}}\label{wxcommandprocessor}

wxCommandProcessor is a class that maintains a history of wxCommands,
with undo/redo functionality built-in. Derive a new class from this
if you want different behaviour.

\wxheading{Derived from}

\helpref{wxObject}{wxobject}

\wxheading{Include files}

<wx/cmdproc.h>

\wxheading{See also}

\helpref{wxCommandProcessor overview}{wxcommandprocessoroverview}, \helpref{wxCommand}{wxcommand}

\latexignore{\rtfignore{\wxheading{Members}}}


\membersection{wxCommandProcessor::wxCommandProcessor}\label{wxcommandprocessorctor}

\func{}{wxCommandProcessor}{\param{int}{ maxCommands = $-1$}}

Constructor.

{\it maxCommands} may be set to a positive integer to limit the number of
commands stored to it, otherwise (and by default) the list of commands can grow
arbitrarily.


\membersection{wxCommandProcessor::\destruct{wxCommandProcessor}}\label{wxcommandprocessordtor}

\func{}{\destruct{wxCommandProcessor}}{\void}

Destructor.


\membersection{wxCommandProcessor::CanUndo}\label{wxcommandprocessorcanundo}

\func{virtual bool}{CanUndo}{\void}

Returns true if the currently-active command can be undone, false otherwise.


\membersection{wxCommandProcessor::ClearCommands}\label{wxcommandprocessorclearcommands}

\func{virtual void}{ClearCommands}{\void}

Deletes all commands in the list and sets the current command pointer to \NULL.


\membersection{wxCommandProcessor::Redo}\label{wxcommandprocessorredo}

\func{virtual bool}{Redo}{\void}

Executes (redoes) the current command (the command that has just been undone if any).


\membersection{wxCommandProcessor::GetCommands}\label{wxcommandprocessorgetcommands}

\constfunc{wxList\&}{GetCommands}{\void}

Returns the list of commands.


\membersection{wxCommandProcessor::GetMaxCommands}\label{wxcommandprocessorgetmaxcommands}

\constfunc{int}{GetMaxCommands}{\void}

Returns the maximum number of commands that the command processor stores.


\membersection{wxCommandProcessor::GetEditMenu}\label{wxcommandprocessorgeteditmenu}

\constfunc{wxMenu*}{GetEditMenu}{\void}

Returns the edit menu associated with the command processor.


\membersection{wxCommandProcessor::GetRedoAccelerator}\label{wxcommandprocessorgetredoaccelerator}

\constfunc{const wxString\&}{GetRedoAccelerator}{\void}

Returns the string that will be appended to the Redo menu item.


\membersection{wxCommandProcessor::GetRedoMenuLabel}\label{wxcommandprocessorgetredomenulabel}

\constfunc{wxString}{GetRedoMenuLabel}{\void}

Returns the string that will be shown for the redo menu item.


\membersection{wxCommandProcessor::GetUndoAccelerator}\label{wxcommandprocessorgetundoaccelerator}

\constfunc{const wxString\&}{GetUndoAccelerator}{\void}

Returns the string that will be appended to the Undo menu item.


\membersection{wxCommandProcessor::GetUndoMenuLabel}\label{wxcommandprocessorgetundomenulabel}

\constfunc{wxString}{GetUndoMenuLabel}{\void}

Returns the string that will be shown for the undo menu item.


\membersection{wxCommandProcessor::Initialize}\label{wxcommandprocessorinitialize}

\func{virtual void}{Initialize}{\void}

Initializes the command processor, setting the current command to the
last in the list (if any), and updating the edit menu (if one has been
specified).


\membersection{wxCommandProcessor::IsDirty}\label{wxcommandprocessorisdirty}

\func{virtual bool}{IsDirty}{\void}

Returns a boolean value that indicates if changes have been made since
the last save operation. This only works if 
\helpref{wxCommandProcessor::MarkAsSaved}{wxcommandprocessormarkassaved}
is called whenever the project is saved.


\membersection{wxCommandProcessor::MarkAsSaved}\label{wxcommandprocessormarkassaved}

\func{virtual void}{MarkAsSaved}{\void}

You must call this method whenever the project is saved if you plan to use 
\helpref{wxCommandProcessor::IsDirty}{wxcommandprocessorisdirty}.


\membersection{wxCommandProcessor::SetEditMenu}\label{wxcommandprocessorseteditmenu}

\func{void}{SetEditMenu}{\param{wxMenu* }{menu}}

Tells the command processor to update the Undo and Redo items on this
menu as appropriate. Set this to NULL if the menu is about to be
destroyed and command operations may still be performed, or the command
processor may try to access an invalid pointer.


\membersection{wxCommandProcessor::SetMenuStrings}\label{wxcommandprocessorsetmenustrings}

\func{void}{SetMenuStrings}{\void}

Sets the menu labels according to the currently set menu and the current
command state.


\membersection{wxCommandProcessor::SetRedoAccelerator}\label{wxcommandprocessorsetredoaccelerator}

\func{void}{SetRedoAccelerator}{\param{const wxString\&}{accel}}

Sets the string that will be appended to the Redo menu item.


\membersection{wxCommandProcessor::SetUndoAccelerator}\label{wxcommandprocessorsetundoaccelerator}

\func{void}{SetUndoAccelerator}{\param{const wxString\&}{accel}}

Sets the string that will be appended to the Undo menu item.


\membersection{wxCommandProcessor::Submit}\label{wxcommandprocessorsubmit}

\func{virtual bool}{Submit}{\param{wxCommand *}{command}, \param{bool}{ storeIt = true}}

Submits a new command to the command processor. The command processor
calls wxCommand::Do to execute the command; if it succeeds, the command
is stored in the history list, and the associated edit menu (if any) updated
appropriately. If it fails, the command is deleted
immediately. Once Submit has been called, the passed command should not
be deleted directly by the application.

{\it storeIt} indicates whether the successful command should be stored
in the history list.


\membersection{wxCommandProcessor::Undo}\label{wxcommandprocessorundo}

\func{virtual bool}{Undo}{\void}

Undoes the command just executed.


