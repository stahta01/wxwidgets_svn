%%%%%%%%%%%%%%%%%%%%%%%%%%%%%%%%%%%%%%%%%%%%%%%%%%%%%%%%%%%%%%%%%%%%%%%%%%%%%%%
%% Name:        uri.tex
%% Purpose:     wxURI docs
%% Author:      Ryan Norton <wxprojects@comcast.net>
%% Modified by:
%% Created:     7/7/2004
%% RCS-ID:      $Id$
%% Copyright:   (c) Ryan Norton
%% License:     wxWindows license
%%%%%%%%%%%%%%%%%%%%%%%%%%%%%%%%%%%%%%%%%%%%%%%%%%%%%%%%%%%%%%%%%%%%%%%%%%%%%%%

\section{\class{wxURI}}\label{wxuri}

wxURI is used to extract information from
a URI (Uniform Resource Identifier).

For information about URIs, see 
\urlref{RFC 3986}{http://www.ietf.org/rfc/rfc3986.txt}.

In short, a URL \em{is} a URI.  In other
words, URL is a subset of a URI - all 
acceptable URLs are also acceptable URIs.

wxURI automatically escapes invalid characters in a string,
so there is no chance of wxURI "failing" on construction/creation.

wxURI supports copy construction and standard assignment
operators.  wxURI can also be inherited from to provide
furthur functionality.

\wxheading{Derived from}

\helpref{wxObject}{wxobject}

\wxheading{Include files}

<wx/uri.h>

\wxheading{See also}

\helpref{wxURL}{wxurl}

\latexignore{\rtfignore{\wxheading{Members}}}

\membersection{Obtaining individual components}\label{obtainingwxuricomponents}

To obtain individual components you can use 
one of the following methods

\helpref{GetScheme}{wxurigetscheme}\\
\helpref{GetUserInfo}{wxurigetuserinfo}\\
\helpref{GetServer}{wxurigetserver}\\
\helpref{GetPort}{wxurigetport}\\
\helpref{GetPath}{wxurigetpath}\\
\helpref{GetQuery}{wxurigetquery}\\
\helpref{GetFragment}{wxurigetfragment}

However, you should check HasXXX before
calling a get method, which determines whether or not the component referred
to by the method is defined according to RFC 2396.

Consider an undefined component equivalent to a 
NULL C string.\\
\\ 
\helpref{HasScheme}{wxurihasscheme}\\
\helpref{HasUserInfo}{wxurihasuserinfo}\\
\helpref{HasServer}{wxurihasserver}\\
\helpref{HasPort}{wxurihasserver}\\
\helpref{HasPath}{wxurihaspath}\\
\helpref{HasQuery}{wxurihasquery}\\
\helpref{HasFragment}{wxurihasfragment}

Example:
\begin{verbatim}
//protocol will hold the http protocol (i.e. "http")
wxString protocol;
wxURI myuri(wxT("http://mysite.com"));
if(myuri.HasScheme())
   protocol = myuri.GetScheme();
\end{verbatim}

\membersection{Deviations from the RFC}\label{deviationsfromrfc}

Note that on URIs with a "file" scheme wxURI does not
parse the userinfo, server, or port portion.  This is to keep 
compatability with wxFileSystem, the old wxURL, and older url specifications.

\membersection{wxURI::wxURI}\label{wxuriwxuri}

\func{}{wxURI}{\void}

Creates an empty URI.

\func{}{wxURI}{\param{const wxChar* }{uri}}

Constructor for quick creation.

\docparam{uri}{string to initialize with}

\func{}{wxURI}{\param{const wxURI\& }{uri}}

Copies this URI from another URI.

\docparam{uri}{URI (Uniform Resource Identifier) to initialize with}


\membersection{wxURI::BuildURI}\label{wxuribuilduri}

\constfunc{wxString}{BuildURI}{\void}

Builds the URI from its individual components and adds proper separators.

If the URI is not a reference or is not resolved, 
the URI that is returned from Get is the same one 
passed to Create.


\membersection{wxURI::BuildUnescapedURI}\label{wxuribuildunescapeduri}

\constfunc{wxString}{BuildUnescapedURI}{\void}

Builds the URI from its individual components, adds proper separators, and
returns escape sequences to normal characters.

Note that it is preferred to call this over Unescape(BuildURI()) since
\helpref{BuildUnescapedURI}{wxuribuildunescapeduri} performs some optimizations over the plain method.


\membersection{wxURI::Create}\label{wxuricreate}

\func{const wxChar*}{Create}{\param{const wxString&}{uri}}

Creates this URI from the string \arg{uri}.

Returns the position at which parsing stopped (there 
is no such thing as an "invalid" wxURI).

\docparam{uri}{string to initialize from}


\membersection{wxURI::GetFragment}\label{wxurigetfragment}

\constfunc{const wxString&}{GetFragment}{\void}

Obtains the fragment of this URI.

The fragment of a URI is the last value of the URI,
and is the value after a '#' character after the path 
of the URI.

\tt{http://mysite.com/mypath\#<fragment>}

\membersection{wxURI::GetHostType}\label{wxurigethosttype}

\constfunc{const HostType\&}{GetHostType}{\void}

Obtains the host type of this URI, which is of type
wxURI::HostType:

\twocolwidtha{7cm}
\begin{twocollist}\itemsep=0pt
\twocolitem{{\bf wxURI\_REGNAME}}{Server is a host name, or the Server component itself is undefined.}
\twocolitem{{\bf wxURI\_IPV4ADDRESS}}{Server is a IP version 4 address (XXX.XXX.XXX.XXX)}
\twocolitem{{\bf wxURI\_IPV6ADDRESS}}{Server is a IP version 6 address ((XXX:)XXX::(XXX)XXX:XXX}
\twocolitem{{\bf wxURI\_IPVFUTURE}}{Server is an IP address, but not versions 4 or 6}
\end{twocollist}


\membersection{wxURI::GetPassword}\label{wxurigetpassword}

\constfunc{const wxString&}{GetPassword}{\void}

Returns the password part of the userinfo component of
this URI.  Note that this is explicitly depreciated by
RFC 1396 and should generally be avoided if possible.

\tt{http://<user>:<password>@mysite.com/mypath}


\membersection{wxURI::GetPath}\label{wxurigetpath}

\constfunc{const wxString&}{GetPath}{\void}

Returns the (normalized) path of the URI.

The path component of a URI comes
directly after the scheme component
if followed by zero or one slashes ('/'),
or after the server/port component.

Absolute paths include the leading '/'
character.

\tt{http://mysite.com<path>}

\membersection{wxURI::GetPort}\label{wxurigetport}

\constfunc{const wxString&}{GetPort}{\void}

Returns a string representation of the URI's port.

The Port of a URI is a value after the server, and 
must come after a colon (:).

\tt{http://mysite.com:<port>}

Note that you can easily get the numeric value of the port
by using wxAtoi or wxString::Format.

\membersection{wxURI::GetQuery}\label{wxurigetquery}

\constfunc{const wxString&}{GetQuery}{\void}

Returns the Query component of the URI.

The query component is what is commonly passed to a 
cgi application, and must come after the path component,
and after a '?' character.

\tt{http://mysite.com/mypath?<query>}


\membersection{wxURI::GetScheme}\label{wxurigetscheme}

\constfunc{const wxString&}{GetScheme}{\void}

Returns the Scheme component of the URI.

The first part of the uri.

\tt{<scheme>://mysite.com}


\membersection{wxURI::GetServer}\label{wxurigetserver}

\constfunc{const wxString&}{GetServer}{\void}

Returns the Server component of the URI.

The server of the uri can be a server name or 
a type of ip address.  See
\helpref{GetHostType}{wxurigethosttype} for the
possible values for the host type of the 
server component.

\tt{http://<server>/mypath}


\membersection{wxURI::GetUser}\label{wxurigetuser}

\constfunc{const wxString&}{GetUser}{\void}

Returns the username part of the userinfo component of
this URI.  Note that this is explicitly depreciated by
RFC 1396 and should generally be avoided if possible.

\tt{http://<user>:<password>@mysite.com/mypath}


\membersection{wxURI::GetUserInfo}\label{wxurigetuserinfo}

\constfunc{const wxString&}{GetUserInfo}{\void}

Returns the UserInfo component of the URI.

The component of a URI before the server component
that is postfixed by a '@' character.

\tt{http://<userinfo>@mysite.com/mypath}


\membersection{wxURI::HasFragment}\label{wxurihasfragment}

\constfunc{bool}{HasFragment}{\void}

Returns \true if the Fragment component of the URI exists.


\membersection{wxURI::HasPath}\label{wxurihaspath}

\constfunc{bool}{HasPath}{\void}

Returns \true if the Path component of the URI exists.


\membersection{wxURI::HasPort}\label{wxurihasport}

\constfunc{bool}{HasPort}{\void}

Returns \true if the Port component of the URI exists.


\membersection{wxURI::HasQuery}\label{wxurihasquery}

\constfunc{bool}{HasQuery}{\void}

Returns \true if the Query component of the URI exists.


\membersection{wxURI::HasScheme}\label{wxurihasscheme}

\constfunc{bool}{HasScheme}{\void}

Returns \true if the Scheme component of the URI exists.


\membersection{wxURI::HasServer}\label{wxurihasserver}

\constfunc{bool}{HasServer}{\void}

Returns \true if the Server component of the URI exists.


\membersection{wxURI::HasUser}\label{wxurihasuserinfo}

\constfunc{bool}{HasUser}{\void}

Returns \true if the User component of the URI exists.


\membersection{wxURI::IsReference}\label{wxuriisreference}

\constfunc{bool}{IsReference}{\void}

Returns \true if a valid [absolute] URI, otherwise this URI
is a URI reference and not a full URI, and IsReference
returns \false.


\membersection{wxURI::operator ==}\label{wxurioperatorcompare}

\func{void}{operator ==}{\param{const wxURI\& }{uricomp}}

Compares this URI to another URI, and returns \true if 
this URI equals \arg{uricomp}, otherwise it returns \false.

\docparam{uricomp}{URI to compare to}


\membersection{wxURI::Resolve}\label{wxuriresolve}

\func{void}{Resolve}{\param{const wxURI\& }{base}, \param{int }{flags = \texttt{wxURI\_STRICT}}}

Inherits this URI from a base URI - components that do not
exist in this URI are copied from the base, and if this URI's
path is not an absolute path (prefixed by a '/'), then this URI's
path is merged with the base's path.

For instance, resolving "../mydir" from "http://mysite.com/john/doe" 
results in the scheme (http) and server (mysite.com) being copied into 
this URI, since they do not exist.  In addition, since the path
of this URI is not absolute (does not begin with '/'), the path
of the base's is merged with this URI's path, resulting in the URI
"http://mysite.com/john/mydir".

\docparam{base}{Base URI to inherit from.  Must be a full URI and not a reference}
\docparam{flags}{Currently either \texttt{wxURI\_STRICT} or $0$, in non-strict
mode some compatibility layers are enabled to allow loopholes from RFCs prior
to 2396}

\membersection{wxURI::Unescape}\label{wxuriunescape}

\func{wxString}{Unescape}{\param{const wxString\& }{uri}}

Translates all escape sequences (% hex hex) of \arg{uri} into
normal characters and returns the result.

This is the preferred over deprecated wxURL::ConvertFromURI.

If you want to unescape an entire wxURI, use \helpref{BuildUnescapedURI}{wxuribuildunescapeduri} instead,
as it performs some optimizations over this method.

\docparam{uri}{string with escaped characters to convert}


