\section{The wxWindows resource system}\label{resourceformats}

wxWindows has an optional {\it resource file} facility,
which allows separation of dialog, menu, bitmap and icon specifications
from the application code.

{\bf NOTE:} this format is now deprecated in favour of the XML-based \helpref{XRC resource system}{xrcoverview}.
However it is still available if wxUSE\_RESOURCES is enabled.

The format is similar in principle to the Windows resource file (whose ASCII form is
suffixed .RC and whose binary form is suffixed .RES). The wxWindows resource
file is currently ASCII-only, suffixed .WXR. Note that under Windows,
the .WXR file does not {\it replace} the native Windows resource file,
it merely supplements it. There is no existing native resource format in X
(except for the defaults file, which has limited expressive power).

For details of functions for manipulating resource files and loading
user interface elements, see \helpref{wxWindows resource functions}{resourcefuncs}.

You can use Dialog Editor to create resource files. Unfortunately neither
Dialog Editor nor the .WXR format currently cover all wxWindows controls;
some are missing, such as wxSpinCtrl, wxSpinButton, wxListCtrl, wxTreeCtrl and others.

Note that in later versions of wxWindows, this resource format will be replaced
by XML specifications that can also include sizers.

\subsection{The format of a .WXR file}

A wxWindows resource file may look a little odd at first. It is C++
compatible, comprising mostly of static string variable declarations with
wxExpr syntax within the string.

Here's a sample .WXR file:

\begin{verbatim}
/*
 * wxWindows Resource File
 *
 */

#include "noname.ids"

static char *my_resource = "bitmap(name = 'my_resource',\
  bitmap = ['myproject', wxBITMAP_TYPE_BMP_RESOURCE, 'WINDOWS'],\
  bitmap = ['myproject.xpm', wxBITMAP_TYPE_XPM, 'X']).";

static char *menuBar11 = "menu(name = 'menuBar11',\
  menu = \
  [\
    ['&File', 1, '', \
      ['&Open File', 2, 'Open a file'],\
      ['&Save File', 3, 'Save a file'],\
      [],\
      ['E&xit', 4, 'Exit program']\
    ],\
    ['&Help', 5, '', \
      ['&About', 6, 'About this program']\
    ]\
  ]).";

static char *project_resource = "icon(name = 'project_resource',\
  icon = ['project', wxBITMAP_TYPE_ICO_RESOURCE, 'WINDOWS'],\
  icon = ['project_data', wxBITMAP_TYPE_XBM, 'X']).";

static char *panel3 = "dialog(name = 'panel3',\
  style = '',\
  title = 'untitled',\
  button_font = [14, 'wxSWISS', 'wxNORMAL', 'wxBOLD', 0],\
  label_font = [10, 'wxSWISS', 'wxNORMAL', 'wxNORMAL', 0],\
  x = 0, y = 37, width = 292, height = 164,\
  control = [1000, wxButton, 'OK', '', 'button5', 23, 34, -1, -1, 'my_resource'],\
  control = [1001, wxStaticText, 'A Label', '', 'message7', 166, 61, -1, -1, 'my_resource'],\
  control = [1002, wxTextCtrl, 'Text', 'wxTE_MULTITEXT', 'text8', 24, 110, -1, -1]).";
\end{verbatim}

As you can see, C++-style comments are allowed, and apparently include files
are supported too: but this is a special case, where the included file
is a file of defines shared by the C++ application code and resource file
to relate identifiers (such as FILE\_OPEN) to integers.

Each {\it resource object} is of standard \helpref{wxExpr}{wxexpr} syntax, that is,
an object name such as {\bf dialog} or {\bf icon}, then an open
parenthesis, a list of comma-delimited attribute/value pairs, a closing
parenthesis, and a full stop. Backslashes are required to escape newlines,
for the benefit of C++ syntax. If double quotation marks are used to
delimit strings, they need to be escaped with backslash within a C++ string
(so it is easier to use single quotation marks instead).

\normalbox{{\it A note on string syntax:} A string that begins with
an alphabetic character, and contains only alphanumeric characters,
hyphens and underscores, need not be quoted at all. Single quotes and double
quotes may be used to delimit more complex strings. In fact, single-quoted
and no-quoted strings are actually called {\it words}, but are treated
as strings for the purpose of the resource system.}

A resource file like this is typically included in the application main file,
as if it were a normal C++ file. This eliminates the need for a separate
resource file to be distributed alongside the executable. However, the
resource file can be dynamically loaded if desired (useful for non-C++
languages such as Python).

Once included, the resources need to be `parsed' (interpreted), because
so far the data is just a number of static string variables. The function\rtfsp
{\bf ::wxResourceParseData} is called early on in initialization of the application
(usually in {\bf wxApp::OnInit}) with a variable as argument. This may need to be
called a number of times, one for each variable. However, more than one
resource `object' can be stored in one string variable at a time, so you can
get all your resources into one variable if you want to.

{\bf ::wxResourceParseData} parses the contents of the resource, ready for use
by functions such as {\bf ::wxResourceCreateBitmap} and {\bf wxPanel::LoadFromResource}.

If a wxWindows resource object (such as a bitmap resource) refers to a
C++ data structure, such as static XPM data, a further call ({\bf ::wxResourceRegisterBitmapData}) needs
to be made on initialization to tell
wxWindows about this data. The wxWindows resource object will refer to a
string identifier, such as `project\_data' in the example file above.
This identifier will be looked up in a table to get the C++ static data
to use for the bitmap or icon.

In the C++ fragment below, the WXR resource file is included,
and appropriate resource initialization is carried out in {\bf OnInit}.
Note that at this stage, no actual wxWindows dialogs, menus, bitmaps or
icons are created; their `templates' are merely being set up for later
use.

\begin{verbatim}
/*
 * File:    project.cpp
 * Purpose: main application module
 */

#include "wx/wx.h"
#include "project.h"

// Includes the dialog, menu etc. resources
#include "project.wxr"

// Includes XPM data
#include "project.xpm"

IMPLEMENT_APP(AppClass)

// Called to initialize the program
bool AppClass::OnInit()
{
  wxResourceRegisterBitmapData("project_data", project_bits, project_width, project_height);

  wxResourceParseData(menuBar11);
  wxResourceParseData(my_resource);
  wxResourceParseData(project_resource);
  wxResourceParseData(panel3);
  ...

  return TRUE;
}
\end{verbatim}

The following code shows a dialog:

\begin{verbatim}
  // project.wxr contains dialog1
  MyDialog *dialog = new MyDialog;
  if (dialog->LoadFromResource(this, "dialog1"))
  {
    wxTextCtrl *text = (wxTextCtrl *)wxFindWindowByName("text3", dialog);
    if (text)
      text->SetValue("wxWindows resource demo");
    dialog->ShowModal();
  }
  dialog->Destroy();
\end{verbatim}

Please see also the resource sample.

\subsection{Dialog resource format}

A dialog resource object may be used for either panels or dialog boxes, and
consists of the following attributes. In the following, a {\it font specification}\rtfsp
is a list consisting of point size, family, style, weight, underlined, optional facename.

\begin{twocollist}\itemsep=0pt
\twocolitemruled{Attribute}{Value}
\twocolitem{id}{The integer identifier of the resource.}
\twocolitem{name}{The name of the resource.}
\twocolitem{style}{Optional dialog box or panel window style.}
\twocolitem{title}{The title of the dialog box (unused if a panel).}.
\twocolitem{modal}{Whether modal: 1 if modal, 0 if modeless, absent if a panel resource.}
\twocolitem{use\_dialog\_units}{If 1, use dialog units (dependent on the dialog font size) for control sizes and positions.}
\twocolitem{use\_system\_defaults}{If 1, override colours and fonts to use system settings instead.}
\twocolitem{button\_font}{The font used for control buttons: a list comprising point size (integer),
family (string), font style (string), font weight (string) and underlining (0 or 1).}
\twocolitem{label\_font}{The font used for control labels: a list comprising point size (integer),
family (string), font style (string), font weight (string) and underlining (0 or 1). Now obsolete; use button\_font instead.}
\twocolitem{x}{The x position of the dialog or panel.}
\twocolitem{y}{The y position of the dialog or panel.}
\twocolitem{width}{The width of the dialog or panel.}
\twocolitem{height}{The height of the dialog or panel.}
\twocolitem{background\_colour}{The background colour of the dialog or panel.}
\twocolitem{label\_colour}{The default label colour for the children of the dialog or panel. Now obsolete; use button\_colour instead.}
\twocolitem{button\_colour}{The default button text colour for the children of the dialog or panel.}
\end{twocollist}

Then comes zero or more attributes named `control' for each control
(panel item) on the dialog or panel. The value is a list of further
elements. In the table below, the names in the first column correspond to
the first element of the value list, and the second column details the
remaining elements of the list. Note that titles for some controls are obsolete
(they don't have titles), but the syntax is retained for backward compatibility.

\begin{twocollist}\itemsep=0pt
\twocolitemruled{Control}{Values}
\twocolitem{wxButton}{id (integer), title (string), window style (string), name (string), x, y, width, height, button bitmap resource (optional string), button font spec}
\twocolitem{wxCheckBox}{id (integer), title (string), window style (string), name (string), x, y, width, height, default value (optional integer, 1 or 0), label font spec}
\twocolitem{wxChoice}{id (integer), title (string), window style (string), name (string), x, y, width, height, values (optional list of strings), label font spec, button font spec}
\twocolitem{wxComboBox}{id (integer), title (string), window style (string), name (string), x, y, width, height, default text value, values (optional list of strings), label font spec, button font spec}
\twocolitem{wxGauge}{id (integer), title (string), window style (string), name (string), x, y, width, height, value (optional integer), range (optional integer), label font spec, button font spec}
\twocolitem{wxStaticBox}{id (integer), title (string), window style (string), name (string), x, y, width, height, label font spec}
\twocolitem{wxListBox}{id (integer), title (string), window style (string), name (string), x, y, width, height, values (optional list of strings), multiple (optional string, wxSINGLE or wxMULTIPLE),
label font spec, button font spec}
\twocolitem{wxStaticText}{id (integer), title (string), window style (string), name (string), x, y, width, height, message bitmap resource (optional string), label font spec}
\twocolitem{wxRadioBox}{id (integer), title (string), window style (string), name (string), x, y, width, height, values (optional list of strings), number of rows or cols,
label font spec, button font spec}
\twocolitem{wxRadioButton}{id (integer), title (string), window style (string), name (string), x, y, width, height, default value (optional integer, 1 or 0), label font spec}
\twocolitem{wxScrollBar}{id (integer), title (string), window style (string), name (string), x, y, width, height, value (optional integer),
page length (optional integer), object length (optional integer), view length (optional integer)}
\twocolitem{wxSlider}{id (integer), title (string), window style (string), name (string), x, y, width, height, value (optional integer), minimum (optional integer), maximum (optional integer),
label font spec, button font spec}
\twocolitem{wxTextCtrl}{id (integer), title (string), window style (string), name (string), x, y, width, height, default value (optional string),
label font spec, button font spec}
\end{twocollist}

\subsection{Menubar resource format}

A menubar resource object consists of the following attributes.

\begin{twocollist}\itemsep=0pt
\twocolitemruled{Attribute}{Value}
\twocolitem{name}{The name of the menubar resource.}
\twocolitem{menu}{A list containing all the menus, as detailed below.}
\end{twocollist}

The value of the {\bf menu} attribute is a list of menu item specifications, where each menu
item specification is itself a list comprising:

\begin{itemize}\itemsep=0pt
\item title (a string)
\item menu item identifier (a string or non-zero integer, see below)
\item help string (optional)
\item 0 or 1 for the `checkable' parameter (optional)
\item optionally, further menu item specifications if this item is a pulldown menu.
\end{itemize}

If the menu item specification is the empty list ([]), this is interpreted as a menu separator.

If further (optional) information is associated with each menu item in a future release of wxWindows,
it will be placed after the help string and before the optional pulldown menu specifications.

Note that the menu item identifier must be an integer if the resource is being
included as C++ code and then parsed on initialisation. Unfortunately,\rtfsp
\#define substitution is not performed inside strings, and
therefore the program cannot know the mapping. However, if the .WXR file
is being loaded dynamically, wxWindows will attempt to replace string
identifiers with \#defined integers, because it is able to parse
the included \#defines.

\subsection{Bitmap resource format}

A bitmap resource object consists of a name attribute, and one or more {\bf bitmap} attributes.
There can be more than one of these to allow specification of bitmaps that are optimum for the
platform and display.

\begin{itemize}\itemsep=0pt
\item Bitmap name or filename.
\item Type of bitmap; for example, wxBITMAP\_TYPE\_BMP\_RESOURCE. See class reference under {\bf wxBitmap} for
a full list).
\item Platform this bitmap is valid for; one of WINDOWS, X, MAC and ANY.
\item Number of colours (optional).
\item X resolution (optional).
\item Y resolution (optional).
\end{itemize}

\subsection{Icon resource format}

An icon resource object consists of a name attribute, and one or more {\bf icon} attributes.
There can be more than one of these to allow specification of icons that are optimum for the
platform and display.

\begin{itemize}\itemsep=0pt
\item Icon name or filename.
\item Type of icon; for example, wxBITMAP\_TYPE\_ICO\_RESOURCE. See class reference under {\bf wxBitmap} for
a full list).
\item Platform this bitmap is valid for; one of WINDOWS, X, MAC and ANY.
\item Number of colours (optional).
\item X resolution (optional).
\item Y resolution (optional).
\end{itemize}

\subsection{Resource format design issues}

The .WXR file format is a recent addition and subject to change.
The use of an ASCII resource file format may seem rather inefficient, but this
choice has a number of advantages:

\begin{itemize}\itemsep=0pt
\item Since it is C++ compatible, it can be included into an application's source code,
eliminating the problems associated with distributing a separate resource file
with the executable. However, it can also be loaded dynamically from a file, which will be required
for non-C++ programs that use wxWindows.
\item No extra binary file format and separate converter need be maintained for the wxWindows project
(although others are welcome to add the equivalent of the Windows `rc' resource
parser and a binary format).
\item It would be difficult to append a binary resource component onto an executable
in a portable way.
\item The file format is essentially the \helpref{wxExpr}{wxexpr} object format, for which
a parser already exists, so parsing is easy. For those programs that use wxExpr
anyway, the size overhead of the parser is minimal.
\end{itemize}

The disadvantages of the approach include:

\begin{itemize}\itemsep=0pt
\item Parsing adds a small execution overhead to program initialization.
\item Under 16-bit Windows especially, global data is at a premium.
Using a .RC resource table for some wxWindows resource data may be a partial solution,
although .RC strings are limited to 255 characters.
\item Without a resource preprocessor, it is not possible to substitute integers
for identifiers (so menu identifiers have to be written as integers in the resource
object, in addition to providing \#defines for application code convenience).
\end{itemize}

\subsection{Compiling the resource system}

To enable the resource system, set {\bf wxUSE\_WX\_RESOURCES} to 1 in setup.h.

