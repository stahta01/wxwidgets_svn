\section{\class{wxMenuEvent}}\label{wxmenuevent}

This class is used for a variety of menu-related events. Note that
these do not include menu command events, which are
handled using \helpref{wxCommandEvent}{wxcommandevent} objects.

The handler \helpref{wxFrame::OnMenuHighlight}{wxframeonmenuhighlight} displays help
text in the first field of the status bar.

\wxheading{Derived from}

\helpref{wxEvent}{wxevent}\\
\helpref{wxObject}{wxobject}

\wxheading{Include files}

<wx/event.h>

\wxheading{Event table macros}

To process a menu event, use these event handler macros to direct input to member
functions that take a wxMenuEvent argument.

\twocolwidtha{7cm}
\begin{twocollist}\itemsep=0pt
\twocolitem{{\bf EVT\_MENU\_OPEN(func)}}{A menu is about to be opened.}
\twocolitem{{\bf EVT\_MENU\_CLOSE(func)}}{A menu has been just closed.}
\twocolitem{{\bf EVT\_MENU\_HIGHLIGHT(id, func)}}{The menu item with the
specified id has been highlighted: used to show help prompts in the status bar
by \helpref{wxFrame}{wxframe}}
\twocolitem{{\bf EVT\_MENU\_HIGHLIGHT\_ALL(func)}}{A menu item has been
highlighted, i.e. the currently selected menu item has changed.}
\end{twocollist}%

\wxheading{See also}

\helpref{Command events}{wxcommandevent},\\
\helpref{Event handling overview}{eventhandlingoverview}

\latexignore{\rtfignore{\wxheading{Members}}}

\membersection{wxMenuEvent::wxMenuEvent}

\func{}{wxMenuEvent}{\param{WXTYPE }{id = 0}, \param{int }{id = 0}, \param{wxDC* }{dc = NULL}}

Constructor.

\membersection{wxMenuEvent::m\_menuId}

\member{int}{m\_menuId}

The relevant menu identifier.

\membersection{wxMenuEvent::GetMenuId}\label{wxmenueventgetmenuid}

\constfunc{int}{GetMenuId}{\void}

Returns the menu identifier associated with the event. This method should be
only used with the {\tt HIGHLIGHT} events.

\membersection{wxMenuEvent::IsPopup}\label{wxmenueventispopup}

\constfunc{bool}{IsPopup}{\void}

Returns {\tt TRUE} if the menu which is being opened or closed is a popup menu, 
{\tt FALSE} if it is a normal one.

This method should be only used with the {\tt OPEN} and {\tt CLOSE} events.


