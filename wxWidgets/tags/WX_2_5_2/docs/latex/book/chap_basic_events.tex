\chapter{Basic event handling}\label{chapbasicevents}
\pagenumbering{arabic}%
\setheader{{\it CHAPTER \thechapter: BASIC EVENT HANDLING}}{}{}{}{}{{\it CHAPTER \thechapter: BASIC EVENT HANDLING}}%
\setfooter{\thepage}{}{}{}{}{\thepage}%

\section{Introduction}

In most cases, wxWindows uses the concept of {\it event tables} to catch user input.

An event table is placed in an implementation file to tell wxWindows how to map
events to member functions. These member functions are not virtual functions, but
they are all similar in form: they take a single wxEvent-derived argument, and have a void return
type.

Here's an example of an event table.

\begin{verbatim}
BEGIN_EVENT_TABLE(MyFrame, wxFrame)
  EVT_MENU    (wxID_EXIT, MyFrame::OnExit)
  EVT_MENU    (DO_TEST,   MyFrame::DoTest)
  EVT_SIZE    (           MyFrame::OnSize)
  EVT_BUTTON  (BUTTON1,   MyFrame::OnButton1)
END_EVENT_TABLE()
\end{verbatim}

The first two entries map menu commands to two different member functions. The EVT\_SIZE macro
doesn't need a window identifier, since normally you are only interested in the
current window's size events. (In fact you could intercept a particular window's size event
by using EVT\_CUSTOM(wxEVT\_SIZE, id, func).)

The EVT\_BUTTON macro demonstrates that the originating event does not have to come from
the window class implementing the event table - if the event source is a button within a panel within a frame, this will still
work, because event tables are searched up through the hierarchy of windows. In this
case, the button's event table will be searched, then the parent panel's, then the frame's.

As mentioned before, the member functions that handle events do not have to be virtual.
Indeed, the member functions should not be virtual as the event handler ignores that
the functions are virtual, i.e. overriding a virtual member function in a derived class
will not have any effect.
These member functions take an event argument, and the class of event differs according
to the type of event and the class of the originating window. For size
events, \wxhelpref{wxSizeEvent}{wxsizeevent} is used. For menu commands and most control
commands (such as button presses), \wxhelpref{wxCommandEvent}{wxcommandevent} is used.
When controls get more complicated, then specific event classes are used, such
as \wxhelpref{wxTreeEvent}{wxtreeevent} for events from \wxhelpref{wxTreeCtrl}{wxtreectrl} windows.

As well as the event table in the implementation file, there must be a DECLARE\_EVENT\_TABLE
macro in the class definition. For example:

{\small%
\begin{verbatim}
class MyFrame: public wxFrame {

  DECLARE_DYNAMIC_CLASS(MyFrame)

public:
  ...
  void OnExit(wxCommandEvent& event);
  void OnSize(wxSizeEvent& event);
protected:
  int       m_count;
  ...
  DECLARE_EVENT_TABLE()
};
\end{verbatim}
}%

\section{How events are processed}\label{eventprocessing}

When an event is received from the windowing system, wxWindows calls \wxhelpref{wxEvtHandler::ProcessEvent}{wxevthandlerprocessevent} on
the first event handler object belonging to the window generating the event.

It may be noted that wxWindows' event processing system implements something
very close to virtual methods in normal C++, i.e. it is possible to alter
the behaviour of a class by overriding its event handling functions. In
many cases this works even for changing the behaviour of native controls.
For example it is possible to filter out a number of key events sent by the
system to a native text control by overriding wxTextCtrl and defining a
handler for key events using EVT\_KEY\_DOWN. This would indeed prevent
any key events from being sent to the native control - which might not be
what is desired. In this case the event handler function has to call Skip()
so as to indicate that the search for the event handler should continue.

To summarize, instead of explicitly calling the base class version as you
would have done with C++ virtual functions (i.e. {\it wxTextCtrl::OnChar()}),
you should instead call \wxhelpref{wxEvent::Skip}{wxeventskip}.

In practice, this would look like the following if the derived text control only
accepts 'a' to 'z' and 'A' to 'Z':

{\small%
\begin{verbatim}
void MyTextCtrl::OnChar(wxKeyEvent& event)
{
    if ( isalpha( event.KeyCode() ) )
    {
       // key code is within legal range. we call event.Skip() so the
       // event can be processed either in the base wxWindows class
       // or the native control.

       event.Skip();
    }
    else
    {
       // illegal key hit. we don't call event.Skip() so the
       // event is not processed anywhere else.

       wxBell();
    }
}
\end{verbatim}
}%


The normal order of event table searching by ProcessEvent is as follows:

\begin{enumerate}\itemsep=0pt
\item If the object is disabled (via a call to \wxhelpref{wxEvtHandler::SetEvtHandlerEnabled}{wxevthandlersetevthandlerenabled})
the function skips to step (6).
\item If the object is a wxWindow, {\bf ProcessEvent} is recursively called on the window's\rtfsp
\wxhelpref{wxValidator}{wxvalidator}. If this returns TRUE, the function exits.
\item {\bf SearchEventTable} is called for this event handler. If this fails, the base
class table is tried, and so on until no more tables exist or an appropriate function was found,
in which case the function exits.
\item The search is applied down the entire chain of event handlers (usually the chain has a length
of one). If this succeeds, the function exits.
\item If the object is a wxWindow and the event is a wxCommandEvent, {\bf ProcessEvent} is
recursively applied to the parent window's event handler. If this returns TRUE, the function exits.
\item Finally, {\bf ProcessEvent} is called on the wxApp object.
\end{enumerate}

{\bf Pay close attention to Step 5.}  People often overlook or get
confused by this powerful feature of the wxWindows event processing
system.  To put it a different way, events derived either directly or
indirectly from wxCommandEvent will travel up the containment
hierarchy from child to parent until an event handler is found that
doesn't call event.Skip().  Events not derived from wxCommandEvent are
sent only to the window they occurred in and then stop.

Typically events that deal with a window as a window (size, motion,
paint, mouse, keyboard, etc.) are sent only to the window.  Events
that have a higher level of meaning and/or are generated by the window
itself, (button click, menu select, tree expand, etc.) are command
events and are sent up to the parent to see if it is interested in the
event.

Note that your application may wish to override ProcessEvent to redirect processing of
events. This is done in the document/view framework, for example, to allow event handlers
to be defined in the document or view. To test for command events (which will probably
be the only events you wish to redirect), you may use wxEvent::IsCommandEvent for
efficiency, instead of using the slower run-time type system.

As mentioned above, only command events are recursively applied to the parents event
handler. As this quite often causes confusion for users, here is a list of system
events which will {\it not} get sent to the parent's event handler:

\begin{twocollist}\itemsep=0pt
\twocolitem{\wxhelpref{wxEvent}{wxevent}}{The event base class}
\twocolitem{\wxhelpref{wxActivateEvent}{wxactivateevent}}{A window or application activation event}
\twocolitem{\wxhelpref{wxCloseEvent}{wxcloseevent}}{A close window or end session event}
\twocolitem{\wxhelpref{wxEraseEvent}{wxeraseevent}}{An erase background event}
\twocolitem{\wxhelpref{wxFocusEvent}{wxfocusevent}}{A window focus event}
\twocolitem{\wxhelpref{wxKeyEvent}{wxkeyevent}}{A keypress event}
\twocolitem{\wxhelpref{wxIdleEvent}{wxidleevent}}{An idle event}
\twocolitem{\wxhelpref{wxInitDialogEvent}{wxinitdialogevent}}{A dialog initialisation event}
\twocolitem{\wxhelpref{wxJoystickEvent}{wxjoystickevent}}{A joystick event}
\twocolitem{\wxhelpref{wxMenuEvent}{wxmenuevent}}{A menu event}
\twocolitem{\wxhelpref{wxMouseEvent}{wxmouseevent}}{A mouse event}
\twocolitem{\wxhelpref{wxMoveEvent}{wxmoveevent}}{A move event}
\twocolitem{\wxhelpref{wxPaintEvent}{wxpaintevent}}{A paint event}
\twocolitem{\wxhelpref{wxQueryLayoutInfoEvent}{wxquerylayoutinfoevent}}{Used to query layout information}
\twocolitem{\wxhelpref{wxSizeEvent}{wxsizeevent}}{A size event}
\twocolitem{\wxhelpref{wxScrollWinEvent}{wxscrollwinevent}}{A scroll event sent by a scrolled window (not a scroll bar)}
\twocolitem{\wxhelpref{wxSysColourChangedEvent}{wxsyscolourchangedevent}}{A system colour change event}
\twocolitem{\wxhelpref{wxUpdateUIEvent}{wxupdateuievent}}{A user interface update event}
\end{twocollist}

In some cases, it might be desired by the programmer to get a certain number
of system events in a parent window, for example all key events sent to, but not
used by, the native controls in a dialog. In this case, a special event handler
will have to be written that will override ProcessEvent() in order to pass
all events (or any selection of them) to the parent window.

% VZ: it doesn't work like this, but just in case we ever reenable this
%     behaviour, I leave it here
%
% \section{Redirection of command events to the window with the focus}
% 
% The usual upward search through the window hierarchy for command event
% handlers does not always meet an application's requirements. Say you have two
% wxTextCtrl windows in a frame, plus a toolbar with Cut, Copy and Paste
% buttons. To avoid the need to define event handlers in the frame
% and redirect them explicitly to the window with the focus, command events
% are sent to the window with the focus first, for
% menu and toolbar command and UI update events only. This means that
% each window can handle its own commands and UI updates independently. In
% fact wxTextCtrl can handle Cut, Copy, Paste, Undo and Redo commands and UI update
% requests, so no extra coding is required to support them in your menus and
% toolbars.

\section{Pluggable event handlers}

In fact, you don't have to derive a new class from a window class
if you don't want to. You can derive a new class from wxEvtHandler instead,
defining the appropriate event table, and then call
\rtfsp\wxhelpref{wxWindow::SetEventHandler}{wxwindowseteventhandler} (or, preferably,
\rtfsp\wxhelpref{wxWindow::PushEventHandler}{wxwindowpusheventhandler}) to make this
event handler the object that responds to events. This way, you can avoid
a lot of class derivation, and use the same event handler object to
handle events from instances of different classes. If you ever have to call a window's event handler
manually, use the GetEventHandler function to retrieve the window's event handler and use that
to call the member function. By default, GetEventHandler returns a pointer to the window itself
unless an application has redirected event handling using SetEventHandler or PushEventHandler.

One use of PushEventHandler is to temporarily or permanently change the
behaviour of the GUI. For example, you might want to invoke a dialog editor
in your application that changes aspects of dialog boxes. You can
grab all the input for an existing dialog box, and edit it `in situ',
before restoring its behaviour to normal. So even if the application
has derived new classes to customize behaviour, your utility can indulge
in a spot of body-snatching. It could be a useful technique for on-line
tutorials, too, where you take a user through a serious of steps and
don't want them to diverge from the lesson. Here, you can examine the events
coming from buttons and windows, and if acceptable, pass them through to
the original event handler. Use PushEventHandler/PopEventHandler
to form a chain of event handlers, where each handler processes a different
range of events independently from the other handlers.

\section{Window identifiers}\label{windowids}

\index{identifiers}\index{wxID}Window identifiers are integers, and are used to uniquely determine window identity in the
event system (though you can use it for other purposes). In fact, identifiers do not need
to be unique across your entire application just so long as they are unique within a particular context you're interested
in, such as a frame and its children. You may use the wxID\_OK identifier, for example, on
any number of dialogs so long as you don't have several within the same dialog.

If you pass -1 to a window constructor, an identifier will be generated for you, but beware:
if things don't respond in the way they should, it could be because of an id conflict. It is safer
to supply window ids at all times. Automatic generation of identifiers starts at 1 so may well conflict
with your own identifiers.

The following standard identifiers are supplied. You can use wxID\_HIGHEST to determine the
number above which it is safe to define your own identifiers. Or, you can use identifiers below
wxID\_LOWEST.

\begin{verbatim}
#define wxID_LOWEST             4999

#define wxID_OPEN               5000
#define wxID_CLOSE              5001
#define wxID_NEW                5002
#define wxID_SAVE               5003
#define wxID_SAVEAS             5004
#define wxID_REVERT             5005
#define wxID_EXIT               5006
#define wxID_UNDO               5007
#define wxID_REDO               5008
#define wxID_HELP               5009
#define wxID_PRINT              5010
#define wxID_PRINT_SETUP        5011
#define wxID_PREVIEW            5012
#define wxID_ABOUT              5013
#define wxID_HELP_CONTENTS      5014
#define wxID_HELP_COMMANDS      5015
#define wxID_HELP_PROCEDURES    5016
#define wxID_HELP_CONTEXT       5017

#define wxID_CUT                5030
#define wxID_COPY               5031
#define wxID_PASTE              5032
#define wxID_CLEAR              5033
#define wxID_FIND               5034
#define wxID_DUPLICATE          5035
#define wxID_SELECTALL          5036

#define wxID_FILE1              5050
#define wxID_FILE2              5051
#define wxID_FILE3              5052
#define wxID_FILE4              5053
#define wxID_FILE5              5054
#define wxID_FILE6              5055
#define wxID_FILE7              5056
#define wxID_FILE8              5057
#define wxID_FILE9              5058

#define wxID_OK                 5100
#define wxID_CANCEL             5101
#define wxID_APPLY              5102
#define wxID_YES                5103
#define wxID_NO                 5104
#define wxID_STATIC             5105

#define wxID_HIGHEST            5999
\end{verbatim}

\section{Event macros summary}\label{eventmacros}

\wxheading{Generic event table macros}

\twocolwidtha{8cm}%
\begin{twocollist}\itemsep=0pt
\twocolitem{\windowstyle{EVT\_CUSTOM(event, id, func)}}{Allows you to add a custom event table
entry by specifying the event identifier (such as wxEVT\_SIZE), the window identifier,
and a member function to call.}
\twocolitem{\windowstyle{EVT\_CUSTOM\_RANGE(event, id1, id2, func)}}{The same as EVT\_CUSTOM,
but responds to a range of window identifiers.}
\twocolitem{\windowstyle{EVT\_COMMAND(id, event, func)}}{The same as EVT\_CUSTOM, but
expects a member function with a wxCommandEvent argument.}
\twocolitem{\windowstyle{EVT\_COMMAND\_RANGE(id1, id2, event, func)}}{The same as EVT\_CUSTOM\_RANGE, but
expects a member function with a wxCommandEvent argument.}
\end{twocollist}

\wxheading{Macros listed by event class}

The documentation for specific event macros is organised by event class. Please refer
to these sections for details.

\twocolwidtha{8cm}%
\begin{twocollist}\itemsep=0pt
\twocolitem{\wxhelpref{wxActivateEvent}{wxactivateevent}}{The EVT\_ACTIVATE and EVT\_ACTIVATE\_APP macros intercept
activation and deactivation events.}
\twocolitem{\wxhelpref{wxCommandEvent}{wxcommandevent}}{A range of commonly-used control events.}
\twocolitem{\wxhelpref{wxCloseEvent}{wxcloseevent}}{The EVT\_CLOSE macro handles window closure
called via \wxhelpref{wxWindow::Close}{wxwindowclose}.}
\twocolitem{\wxhelpref{wxDropFilesEvent}{wxdropfilesevent}}{The EVT\_DROP\_FILES macros handles
file drop events.}
\twocolitem{\wxhelpref{wxEraseEvent}{wxeraseevent}}{The EVT\_ERASE\_BACKGROUND macro is used to handle window erase requests.}
\twocolitem{\wxhelpref{wxFocusEvent}{wxfocusevent}}{The EVT\_SET\_FOCUS and EVT\_KILL\_FOCUS macros are used to handle keyboard focus events.}
\twocolitem{\wxhelpref{wxKeyEvent}{wxkeyevent}}{EVT\_CHAR and EVT\_CHAR\_HOOK macros handle keyboard
input for any window.}
\twocolitem{\wxhelpref{wxIdleEvent}{wxidleevent}}{The EVT\_IDLE macro handle application idle events
(to process background tasks, for example).}
\twocolitem{\wxhelpref{wxInitDialogEvent}{wxinitdialogevent}}{The EVT\_INIT\_DIALOG macro is used
to handle dialog initialisation.}
\twocolitem{\wxhelpref{wxListEvent}{wxlistevent}}{These macros handle \wxhelpref{wxListCtrl}{wxlistctrl} events.}
\twocolitem{\wxhelpref{wxMenuEvent}{wxmenuevent}}{These macros handle special menu events (not menu commands).}
\twocolitem{\wxhelpref{wxMouseEvent}{wxmouseevent}}{Mouse event macros can handle either individual
mouse events or all mouse events.}
\twocolitem{\wxhelpref{wxMoveEvent}{wxmoveevent}}{The EVT\_MOVE macro is used to handle a window move.}
\twocolitem{\wxhelpref{wxPaintEvent}{wxpaintevent}}{The EVT\_PAINT macro is used to handle window paint requests.}
\twocolitem{\wxhelpref{wxScrollEvent}{wxscrollevent}}{These macros are used to handle scroll events from 
\wxhelpref{wxScrollBar}{wxscrollbar}, \wxhelpref{wxSlider}{wxslider},and \wxhelpref{wxSpinButton}{wxspinbutton}.}
\twocolitem{\wxhelpref{wxSizeEvent}{wxsizeevent}}{The EVT\_SIZE macro is used to handle a window resize.}
\twocolitem{\wxhelpref{wxSplitterEvent}{wxsplitterevent}}{The EVT\_SPLITTER\_SASH\_POS\_CHANGED, EVT\_SPLITTER\_UNSPLIT
and EVT\_SPLITTER\_DCLICK macros are used to handle the various splitter window events.}
\twocolitem{\wxhelpref{wxSysColourChangedEvent}{wxsyscolourchangedevent}}{The EVT\_SYS\_COLOUR\_CHANGED macro is used to handle
events informing the application that the user has changed the system colours (Windows only).}
\twocolitem{\wxhelpref{wxTreeEvent}{wxtreeevent}}{These macros handle \wxhelpref{wxTreeCtrl}{wxtreectrl} events.}
\twocolitem{\wxhelpref{wxUpdateUIEvent}{wxupdateuievent}}{The EVT\_UPDATE\_UI macro is used to handle user interface
update pseudo-events, which are generated to give the application the chance to update the visual state of menus,
toolbars and controls.}
\end{twocollist}

