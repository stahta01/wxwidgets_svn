\section{wxValidator overview}\label{validatoroverview}

Classes: \helpref{wxValidator}{wxvalidator}, \helpref{wxTextValidator}{wxtextvalidator}, 
\helpref{wxGenericValidator}{wxgenericvalidator}

The aim of the validator concept is to make dialogs very much easier to write.
A validator is an object that can be plugged into a control (such as a wxTextCtrl), and
mediates between C++ data and the control, transferring the data in either direction
and validating it. It also is able to intercept events generated
by the control, providing filtering behaviour without the need to derive a new control class.

You can use a stock validator, such as \helpref{wxTextValidator}{wxtextvalidator} (which does text
control data transfer, validation and filtering) and 
\helpref{wxGenericValidator}{wxgenericvalidator} (which does data transfer for a range of controls);
or you can write your own.

\wxheading{Example}

Here is an example of wxTextValidator usage.

\begin{verbatim}
  wxTextCtrl *txt1 = new wxTextCtrl(this, -1, wxT(""),
    wxPoint(10, 10), wxSize(100, 80), 0,
    wxTextValidator(wxFILTER_ALPHA, &g_data.m_string));
\end{verbatim}

In this example, the text validator object provides the following functionality:

\begin{enumerate}\itemsep=0pt
\item It transfers the value of g\_data.m\_string (a wxString variable) to the wxTextCtrl when
the dialog is initialised.
\item It transfers the wxTextCtrl data back to this variable when the dialog is dismissed.
\item It filters input characters so that only alphabetic characters are allowed.
\end{enumerate}

The validation and filtering of input is accomplished in two ways. When a character is input,
wxTextValidator checks the character against the allowed filter flag (wxFILTER\_ALPHA in this case). If
the character is inappropriate, it is vetoed (does not appear) and a warning beep sounds.
The second type of validation is performed when the dialog is about to be dismissed, so if
the default string contained invalid characters already, a dialog box is shown giving the
error, and the dialog is not dismissed.

\wxheading{Anatomy of a validator}

A programmer creating a new validator class should provide the following functionality.

A validator constructor is responsible for allowing the programmer to specify the kind
of validation required, and perhaps a pointer to a C++ variable that is used for storing the
data for the control. If such a variable address is not supplied by the user, then
the validator should store the data internally.

The \helpref{wxValidator::Validate}{wxvalidatorvalidate} member function should return
true if the data in the control (not the C++ variable) is valid. It should also show
an appropriate message if data was not valid.

The \helpref{wxValidator::TransferToWindow}{wxvalidatortransfertowindow} member function should
transfer the data from the validator or associated C++ variable to the control.

The \helpref{wxValidator::TransferFromWindow}{wxvalidatortransferfromwindow} member function should
transfer the data from the control to the validator or associated C++ variable.

There should be a copy constructor, and a \helpref{wxValidator::Clone}{wxvalidatorclone} function
which returns a copy of the validator object. This is important because validators
are passed by reference to window constructors, and must therefore be cloned internally.

You can optionally define event handlers for the validator, to implement filtering. These handlers
will capture events before the control itself does.

For an example implementation, see the valtext.h and valtext.cpp files in the wxWidgets library.

\wxheading{How validators interact with dialogs}

For validators to work correctly, validator functions must be called at the right times during
dialog initialisation and dismissal.

When a \helpref{wxDialog::Show}{wxdialogshow} is called (for a modeless dialog)
or \helpref{wxDialog::ShowModal}{wxdialogshowmodal} is called (for a modal dialog),
the function \helpref{wxWindow::InitDialog}{wxwindowinitdialog} is automatically called.
This in turn sends an initialisation event to the dialog. The default handler for
the wxEVT\_INIT\_DIALOG event is defined in the wxWindow class to simply call
the function \helpref{wxWindow::TransferDataToWindow}{wxwindowtransferdatatowindow}. This
function finds all the validators in the window's children and calls the TransferToWindow
function for each. Thus, data is transferred from C++ variables to the dialog
just as the dialog is being shown.

\normalbox{If you are using a window or panel instead of a dialog, you will need to
call \helpref{wxWindow::InitDialog}{wxwindowinitdialog} explicitly before showing the
window.}

When the user clicks on a button, for example the OK button, the application should
first call \helpref{wxWindow::Validate}{wxwindowvalidate}, which returns false if
any of the child window validators failed to validate the window data. The button handler
should return immediately if validation failed. Secondly, the application should
call \helpref{wxWindow::TransferDataFromWindow}{wxwindowtransferdatafromwindow} and
return if this failed. It is then safe to end the dialog by calling EndModal (if modal)
or Show (if modeless).

In fact, wxDialog contains a default command event handler for the wxID\_OK button. It goes like
this:

\begin{verbatim}
void wxDialog::OnOK(wxCommandEvent& event)
{
    if ( Validate() && TransferDataFromWindow() )
    {
        if ( IsModal() )
            EndModal(wxID_OK);
        else
        {
            SetReturnCode(wxID_OK);
            this->Show(false);
        }
    }
}
\end{verbatim}

So if using validators and a normal OK button, you may not even need to write any
code for handling dialog dismissal.

If you load your dialog from a resource file, you will need to iterate through the controls
setting validators, since validators can't be specified in a dialog resource.

