% automatically generated by HelpGen from tempfile.tex at 10/Jan/99 19:24:44
\section{\class{wxTempFile}}\label{wxtempfile}

wxTempFile provides a relatively safe way to replace the contents of the
existing file. The name is explained by the fact that it may be also used as
just a temporary file if you don't replace the old file contents.

Usually, when a program replaces the contents of some file it first opens it for
writing, thus losing all of the old data and then starts recreating it. This
approach is not very safe because during the regeneration of the file bad things
may happen: the program may find that there is an internal error preventing it
from completing file generation, the user may interrupt it (especially if file
generation takes long time) and, finally, any other external interrupts (power
supply failure or a disk error) will leave you without either the original file
or the new one.

wxTempFile addresses this problem by creating a temporary file which is meant to
replace the original file - but only after it is fully written. So, if the user
interrupts the program during the file generation, the old file won't be lost.
Also, if the program discovers itself that it doesn't want to replace the old
file there is no problem - in fact, wxTempFile will {\bf not} replace the old
file by default, you should explicitly call \helpref{Commit}{wxtempfilecommit} 
to do it. Calling \helpref{Discard}{wxtempfilediscard} explicitly discards any
modifications: it closes and deletes the temporary file and leaves the original
file unchanged. If you don't call neither of Commit() and Discard(), the
destructor will call Discard() automatically.

To summarize: if you want to replace another file, create an instance of
wxTempFile passing the name of the file to be replaced to the constructor (you
may also use default constructor and pass the file name to 
\helpref{Open}{wxtempfileopen}). Then you can \helpref{write}{wxtempfilewrite} 
to wxTempFile using \helpref{wxFile}{wxfile}-like functions and later call
Commit() to replace the old file (and close this one) or call Discard() to cancel
the modifications.

\wxheading{Derived from}

No base class

\wxheading{Include files}

<wx/file.h>

\wxheading{See also:}

\helpref{wxFile}{wxfile}

\latexignore{\rtfignore{\wxheading{Members}}}

\membersection{wxTempFile::wxTempFile}\label{wxtempfilewxtempfilector}
\func{}{wxTempFile}{\void}

Default constructor - \helpref{Open}{wxtempfileopen} must be used to open the
file.

\membersection{wxTempFile::wxTempFile}\label{wxtempfilewxtempfile}
\func{}{wxTempFile}{\param{const wxString\& }{strName}}

Associates wxTempFile with the file to be replaced and opens it. You should use
\helpref{IsOpened}{wxtempfileisopened} to verify if the constructor succeeded.

\membersection{wxTempFile::Open}\label{wxtempfileopen}
\func{bool}{Open}{\param{const wxString\& }{strName}}

Open the temporary file (strName is the name of file to be replaced), returns
TRUE on success, FALSE if an error occured.

\membersection{wxTempFile::IsOpened}\label{wxtempfileisopened}
\constfunc{bool}{IsOpened}{\void}

Returns TRUE if the file was successfully opened.

\membersection{wxTempFile::Write}\label{wxtempfilewrite}
\func{bool}{Write}{\param{const void }{*p}, \param{size\_t }{n}}

Write to the file, return TRUE on success, FALSE on failure.

\membersection{wxTempFile::Write}\label{wxtempfilewrites}
\func{bool}{Write}{\param{const wxString\& }{str}}

Write to the file, return TRUE on success, FALSE on failure.

\membersection{wxTempFile::Commit}\label{wxtempfilecommit}
\func{bool}{Commit}{\void}

Validate changes: deletes the old file of name m\_strName and renames the new
file to the old name. Returns TRUE if both actions succeeded. If FALSE is
returned it may unfortunately mean two quite different things: either that
either the old file couldn't be deleted or that the new file couldn't be renamed
to the old name.

\membersection{wxTempFile::Discard}\label{wxtempfilediscard}
\func{void}{Discard}{\void}

Discard changes: the old file contents is not changed, temporary file is
deleted.

\membersection{wxTempFile::\destruct{wxTempFile}}\label{wxtempfiledtor}
\func{}{\destruct{wxTempFile}}{\void}

Destructor calls \helpref{Discard()}{wxtempfilediscard} if temporary file
is still opened.

