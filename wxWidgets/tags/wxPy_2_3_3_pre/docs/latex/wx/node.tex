\section{\class{wxNodeBase}}\label{wxnode}

A node structure used in linked lists (see \helpref{wxList}{wxlist}) and
derived classes. You should never use wxNodeBase class directly because it
works with untyped (void *) data and this is unsafe. Use wxNode-derived classes
which are defined by WX\_DECLARE\_LIST and WX\_DEFINE\_LIST macros instead as
described in \helpref{wxList}{wxlist} documentation (see example there). wxNode
is defined for compatibility as wxNodeBase containing "wxObject *" pointer, but
usage of this class is deprecated.

\wxheading{Derived from}

None.

\wxheading{Include files}

<wx/list.h>

\wxheading{See also}

\helpref{wxList}{wxlist}, \helpref{wxHashTable}{wxhashtable}

\latexignore{\rtfignore{\wxheading{Members}}}

\membersection{wxNodeBase::GetData}

\func{void *}{Data}{\void}

Retrieves the client data pointer associated with the node.

\membersection{wxNodeBase::GetNext}

\func{wxNodeBase *}{Next}{\void}

Retrieves the next node (NULL if at end of list).

\membersection{wxNodeBase::GetPrevious}

\func{wxNodeBase *}{GetPrevious}{\void}

Retrieves the previous node (NULL if at start of list).

\membersection{wxNodeBase::SetData}

\func{void}{SetData}{\param{void *}{data}}

Sets the data associated with the node (usually the pointer will have been
set when the node was created).

\membersection{wxNodeBase::IndexOf}

\func{int}{IndexOf}{\void}

Returns the zero-based index of this node within the list. The return value
will be NOT\_FOUND if the node has not been added to a list yet.

