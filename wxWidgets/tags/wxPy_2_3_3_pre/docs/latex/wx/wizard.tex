%%%%%%%%%%%%%%%%%%%%%%%%%%%%%%%%%%%%%%%%%%%%%%%%%%%%%%%%%%%%%%%%%%%%%%%%%%%%%%%
%% Name:        wizard.tex
%% Purpose:     wxWizard class documentation
%% Author:      Vadim Zeitlin
%% Modified by:
%% Created:     02.04.00
%% RCS-ID:      $Id$
%% Copyright:   (c) Vadim Zeitlin
%% License:     wxWindows license
%%%%%%%%%%%%%%%%%%%%%%%%%%%%%%%%%%%%%%%%%%%%%%%%%%%%%%%%%%%%%%%%%%%%%%%%%%%%%%%

\section{\class{wxWizard}}\label{wxwizard}

wxWizard is the central class for implementing `wizard-like' dialogs. These
dialogs are mostly familiar to Windows users and are nothing else but a
sequence of `pages' each of them displayed inside a dialog which has the
buttons to pass to the next (and previous) pages.

The wizards are typically used to decompose a complex dialog into several
simple steps and are mainly useful to the novice users, hence it is important
to keep them as simple as possible.

To show a wizard dialog, you must first create an object of wxWizard class
using either the non default constructor or a default one followed by call to 
\helpref{Create}{wxwizardcreate} function. Then you should add all pages you
want the wizard to show and call \helpref{RunWizard}{wxwizardrunwizard}.
Finally, don't forget to call {\tt wizard->Destroy()}.

\wxheading{Derived from}

\helpref{wxDialog}{wxdialog}\\
\helpref{wxPanel}{wxpanel}\\
\helpref{wxWindow}{wxwindow}\\
\helpref{wxEvtHandler}{wxevthandler}\\
\helpref{wxObject}{wxobject}

\wxheading{Include files}

<wx/wizard.h>

\wxheading{Event table macros}

To process input from a wizard dialog, use these event handler macros to
direct input to member functions that take a 
\helpref{wxWizardEvent}{wxwizardevent} argument. For some events, 
\helpref{Veto()}{wxnotifyeventveto} can be called to prevent the event from
happening.

\twocolwidtha{7cm}
\begin{twocollist}\itemsep=2pt
\twocolitem{{\bf EVT\_WIZARD\_PAGE\_CHANGED(id, func)}}{The page has been just
changed (this event can not be vetoed).}
\twocolitem{{\bf EVT\_WIZARD\_PAGE\_CHANGING(id, func)}}{The page is being
changed (this event can be vetoed).}
\twocolitem{{\bf EVT\_WIZARD\_CANCEL(id, func)}}{The user attempted to cancel
the wizard (this event may also be vetoed).}
\twocolitem{{\bf EVT\_WIZARD\_HELP(id, func)}}{The wizard help button was pressed.}
\end{twocollist}%

\wxheading{Extended styles}

Use the \helpref{wxWindow::SetExtraStyle}{wxwindowsetextrastyle} function to set the following
style. You will need to use two-step construction (use the default constructor, call {\bf SetExtraStyle}, then call {\bf Create}).

\twocolwidtha{5cm}%
\begin{twocollist}\itemsep=0pt
\twocolitem{\windowstyle{wxWIZARD\_EX\_HELPBUTTON}}{Shows a Help button using wxID\_HELP.}
\end{twocollist}

See also \helpref{wxDialog}{wxdialog} for other extended styles.

\wxheading{See also}

\helpref{wxWizardEvent}{wxwizardevent}, \helpref{wxWizardPage}{wxwizardpage}, \helpref{wxWizard sample}{samplewizard}

\latexignore{\rtfignore{\wxheading{Members}}}

\membersection{wxWizard::wxWizard}\label{wxwizardctor}

\func{}{wxWizard}{\void}

Default constructor. Use this if you wish to derive from wxWizard and then call 
\helpref{Create}{wxwizardcreate}, for example if you wish to set an extra style
with \helpref{wxWindow::SetExtraStyle}{wxwindowsetextrastyle} between the two
calls.

\func{}{wxWizard}{\param{wxWindow* }{parent}, \param{int }{id = -1}, \param{const wxString\& }{title = wxEmptyString}, \param{const wxBitmap\& }{bitmap = wxNullBitmap}, \param{const wxPoint\& }{pos = wxDefaultPosition}}

Constructor which really creates the wizard -- if you use this constructor, you
shouldn't call \helpref{Create}{wxwizardcreate}.

Notice that unlike almost all other wxWindows classes, there is no {\it size} 
parameter in wxWizard constructor because the wizard will have a predefined
default size by default. If you want to change this, you should use the 
\helpref{SetPageSize}{wxwizardsetpagesize} function.

\wxheading{Parameters}

\docparam{parent}{The parent window, may be NULL.}

\docparam{id}{The id of the dialog, will usually be just $-1$.}

\docparam{title}{The title of the dialog.}

\docparam{bitmap}{The default bitmap used in the left side of the wizard. See
also \helpref{GetBitmap}{wxwizardpagegetbitmap}.}

\docparam{pos}{The position of the dialog, it will be centered on the screen
by default.}

\membersection{wxWizard::Create}\label{wxwizardcreate}

\func{bool}{Create}{\param{wxWindow* }{parent}, \param{int }{id = -1}, \param{const wxString\& }{title = wxEmptyString}, \param{const wxBitmap\& }{bitmap = wxNullBitmap}, \param{const wxPoint\& }{pos = wxDefaultPosition}}

Creates the wizard dialog. Must be called if the default constructor had been
used to create the object.

Notice that unlike almost all other wxWindows classes, there is no {\it size} 
parameter in wxWizard constructor because the wizard will have a predefined
default size by default. If you want to change this, you should use the 
\helpref{SetPageSize}{wxwizardsetpagesize} function.

\wxheading{Parameters}

\docparam{parent}{The parent window, may be NULL.}

\docparam{id}{The id of the dialog, will usually be just $-1$.}

\docparam{title}{The title of the dialog.}

\docparam{bitmap}{The default bitmap used in the left side of the wizard. See
also \helpref{GetBitmap}{wxwizardpagegetbitmap}.}

\docparam{pos}{The position of the dialog, it will be centered on the screen
by default.}

\membersection{wxWizard::Fit}\label{wxwizardfit}

\func{void}{Fit}{\param{const wxWizardPage* }{firstPage}}

Sets the page size to be big enough for all the pages accessible via the
given {\it firstPage}, i.e. this page, its next page and so on.

This method may be called more than once and it will only change the page size
if the size required by the new page is bigger than the previously set one.
This is useful if the decision about which pages to show is taken during the
run-time as in this case, the wizard won't be able to get to all pages starting
from a single one and you should call {\it Fit} separately for the others.

\membersection{wxWizard::GetCurrentPage}\label{wxwizardgetcurrentpage}

\constfunc{wxWizardPage*}{GetCurrentPage}{\void}

Get the current page while the wizard is running. {\tt NULL} is returned if 
\helpref{RunWizard()}{wxwizardrunwizard} is not being executed now.

\membersection{wxWizard::GetPageSize}\label{wxwizardgetpagesize}

\constfunc{wxSize}{GetPageSize}{\void}

Returns the size available for the pages.

\membersection{wxWizard::RunWizard}\label{wxwizardrunwizard}

\func{bool}{RunWizard}{\param{wxWizardPage* }{firstPage}}

Executes the wizard starting from the given page, returns {\tt TRUE} if it was
successfully finished or {\tt FALSE} if user cancelled it. The {\it firstPage} 
can not be {\tt NULL}.

\membersection{wxWizard::SetPageSize}\label{wxwizardsetpagesize}

\func{void}{SetPageSize}{\param{const wxSize\& }{sizePage}}

Sets the minimal size to be made available for the wizard pages. The wizard
will take into account the size of the bitmap (if any) itself. Also, the
wizard will never be smaller than the default size.

The recommended way to use this function is to layout all wizard pages using
the sizers (even though the wizard is not resizeable) and then use 
\helpref{wxSizer::CalcMin}{wxsizercalcmin} in a loop to calculate the maximum
of minimal sizes of the pages and pass it to SetPageSize().

