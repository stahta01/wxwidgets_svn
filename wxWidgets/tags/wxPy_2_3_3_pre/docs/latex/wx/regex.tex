%%%%%%%%%%%%%%%%%%%%%%%%%%%%%%%%%%%%%%%%%%%%%%%%%%%%%%%%%%%%%%%%%%%%%%%%%%%%%%%
%% Name:        regex.tex
%% Purpose:     wxRegEx documentation
%% Author:      Vadim Zeitlin
%% Modified by:
%% Created:     14.07.01
%% RCS-ID:      $Id$
%% Copyright:   (c) 2001 Vadim Zeitlin
%% License:     wxWindows license
%%%%%%%%%%%%%%%%%%%%%%%%%%%%%%%%%%%%%%%%%%%%%%%%%%%%%%%%%%%%%%%%%%%%%%%%%%%%%%%

\section{\class{wxRegEx}}\label{wxregex}

wxRegEx represents a regular expression. The regular expressions syntax
supported is the POSIX one. Both basic and extended regular expressions are
supported but, unlike POSIX C API, the extended ones are used by default.

This class provides support for regular expressions matching and also
replacement. It is built on top of either the system library (if it has support
for POSIX regular expressions - which is the case of the most modern Unices) or
uses the built in Henry Spencer's library. In the latter case you need to abide
by the terms of its copyright:

\begin{verbatim}
Copyright 1992, 1993, 1994, 1997 Henry Spencer.  All rights reserved.
This software is not subject to any license of the American Telephone
and Telegraph Company or of the Regents of the University of California.

Permission is granted to anyone to use this software for any purpose on
any computer system, and to alter it and redistribute it, subject
to the following restrictions:

1. The author is not responsible for the consequences of use of this
   software, no matter how awful, even if they arise from flaws in it.

2. The origin of this software must not be misrepresented, either by
   explicit claim or by omission.  Since few users ever read sources,
   credits must appear in the documentation.

3. Altered versions must be plainly marked as such, and must not be
   misrepresented as being the original software.  Since few users
   ever read sources, credits must appear in the documentation.

4. This notice may not be removed or altered.
\end{verbatim}

\wxheading{Derived from}

No base class

\wxheading{Data structures}

Flags for regex compilation to be used with \helpref{Compile()}{wxregexcompile}:

\begin{verbatim}
enum
{
    // use extended regex syntax (default)
    wxRE_EXTENDED = 0,

    // use basic RE syntax
    wxRE_BASIC    = 2,

    // ignore case in match
    wxRE_ICASE    = 4,

    // only check match, don't set back references
    wxRE_NOSUB    = 8,

    // if not set, treat '\n' as an ordinary character, otherwise it is
    // special: it is not matched by '.' and '^' and '$' always match
    // after/before it regardless of the setting of wxRE_NOT[BE]OL
    wxRE_NEWLINE  = 16,

    // default flags
    wxRE_DEFAULT  = wxRE_EXTENDED
}
\end{verbatim}

Flags for regex matching to be used with \helpref{Matches()}{wxregexmatches}.

These flags are mainly useful when doing several matches in a long string
to prevent erroneous matches for {\tt '\textasciicircum'} and {\tt '\$'}:

\begin{verbatim}
enum
{
    // '^' doesn't match at the start of line
    wxRE_NOTBOL = 32,

    // '$' doesn't match at the end of line
    wxRE_NOTEOL = 64
}
\end{verbatim}

\wxheading{Examples}

A bad example of processing some text containing email addresses (the example
is bad because the real email addresses can have more complicated form than
{\tt user@host.net}):

\begin{verbatim}
wxString text;
...
wxRegEx reEmail = "([^@]+)@([[:alnum:].-_].)+([[:alnum:]]+)";
if ( reEmail.Matches(text) )
{
    wxString text = reEmail.GetMatch(email);
    wxString username = reEmail.GetMatch(email, 1);
    if ( reEmail.GetMatch(email, 3) == "com" ) // .com TLD?
    {
        ...
    }
}

// or we could do this to hide the email address
size_t count = reEmail.ReplaceAll(text, "HIDDEN@\\2\\3");
printf("text now contains %u hidden addresses", count);
\end{verbatim}

\latexignore{\rtfignore{\wxheading{Members}}}

\membersection{wxRegEx::wxRegEx}\label{wxregexwxregex}

\func{}{wxRegEx}{\void}

Default ctor: use \helpref{Compile()}{wxregexcompile} later.

\func{}{wxRegEx}{\param{const wxString\& }{expr}, \param{int }{flags = wxRE\_DEFAULT}}

Create and compile the regular expression, use 
\helpref{IsValid}{wxregexisvalid} to test for compilation errors.

\membersection{wxRegEx::\destruct{wxRegEx}}\label{wxregexdtor}

\func{}{\destruct{wxRegEx}}{\void}

dtor not virtual, don't derive from this class

\membersection{wxRegEx::Compile}\label{wxregexcompile}

\func{bool}{Compile}{\param{const wxString\& }{pattern}, \param{int }{flags = wxRE\_DEFAULT}}

Compile the string into regular expression, return {\tt TRUE} if ok or {\tt FALSE} 
if string has a syntax error.

\membersection{wxRegEx::IsValid}\label{wxregexisvalid}

\constfunc{bool}{IsValid}{\void}

Return {\tt TRUE} if this is a valid compiled regular expression, {\tt FALSE} 
otherwise.

\membersection{wxRegEx::GetMatch}\label{wxregexgetmatch}

\constfunc{bool}{GetMatch}{\param{size\_t* }{start}, \param{size\_t* }{len}, \param{size\_t }{index = 0}}

Get the start index and the length of the match of the expression
(if {\it index} is $0$) or a bracketed subexpression ({\it index} different
from $0$).

May only be called after successful call to \helpref{Matches()}{wxregexmatches} 
and only if {\tt wxRE\_NOSUB} was {\bf not} used in 
\helpref{Compile()}{wxregexcompile}.

Returns {\tt FALSE} if no match or if an error occured.

\constfunc{wxString}{GetMatch}{\param{const wxString\& }{text}, \param{size\_t }{index = 0}}

Returns the part of string corresponding to the match where {\it index} is
interpreted as above. Empty string is returned if match failed

May only be called after successful call to \helpref{Matches()}{wxregexmatches} 
and only if {\tt wxRE\_NOSUB} was {\bf not} used in 
\helpref{Compile()}{wxregexcompile}.

\membersection{wxRegEx::Matches}\label{wxregexmatches}

\constfunc{bool}{Matches}{\param{const wxChar* }{text}, \param{int }{flags = 0}}

Matches the precompiled regular expression against the string {\it text},
returns {\tt TRUE} if matches and {\tt FALSE} otherwise.

Flags may be combination of {\tt wxRE\_NOTBOL} and {\tt wxRE\_NOTEOL}.

May only be called after successful call to \helpref{Compile()}{wxregexcompile}.

\membersection{wxRegEx::Replace}\label{wxregexreplace}

\constfunc{int}{Replace}{\param{wxString* }{text}, \param{const wxString\& }{replacement}, \param{size\_t }{maxMatches = 0}}

Replaces the current regular expression in the string pointed to by
{\it text}, with the text in {\it replacement} and return number of matches
replaced (maybe $0$ if none found) or $-1$ on error.

The replacement text may contain back references {\tt $\backslash$number} which will be
replaced with the value of the corresponding subexpression in the
pattern match. {\tt $\backslash$0} corresponds to the entire match and {\tt \&} is a
synonym for it. Backslash may be used to quote itself or {\tt \&} character.

{\it maxMatches} may be used to limit the number of replacements made, setting
it to $1$, for example, will only replace first occurrence (if any) of the
pattern in the text while default value of $0$ means replace all.

\membersection{wxRegEx::ReplaceAll}\label{wxregexreplaceall}

\constfunc{int}{ReplaceAll}{\param{wxString* }{text}, \param{const wxString\& }{replacement}}

Replace all occurrences: this is actually a synonym for 
\helpref{Replace()}{wxregexreplace}.

\wxheading{See also}

\helpref{ReplaceFirst}{wxregexreplacefirst}

\membersection{wxRegEx::ReplaceFirst}\label{wxregexreplacefirst}

\constfunc{int}{ReplaceFirst}{\param{wxString* }{text}, \param{const wxString\& }{replacement}}

Replace the first occurrence.

\wxheading{See also}

\helpref{Replace}{wxregexreplace}

