\section{\class{wxDropFilesEvent}}\label{wxdropfilesevent}

This class is used for drop files events, that is, when files have been dropped
onto the window. This functionality is currently only available under Windows.
The window must have previously been enabled for dropping by calling 
\helpref{wxWindow::DragAcceptFiles}{wxwindowdragacceptfiles}.

Important note: this is a separate implementation to the more general
drag and drop implementation documented \helpref{here}{wxdndoverview}. It uses the
older, Windows message-based approach of dropping files.

\wxheading{Derived from}

\helpref{wxEvent}{wxevent}\\
\helpref{wxObject}{wxobject}

\wxheading{Include files}

<wx/event.h>

\wxheading{Event table macros}

To process a drop files event, use these event handler macros to direct input to a member
function that takes a wxDropFilesEvent argument.

\twocolwidtha{7cm}
\begin{twocollist}\itemsep=0pt
\twocolitem{{\bf EVT\_DROP\_FILES(func)}}{Process a wxEVT\_DROP\_FILES event.}
\end{twocollist}%

\wxheading{See also}

%\helpref{wxWindow::OnDropFiles}{wxwindowondropfiles},
\helpref{Event handling overview}{eventhandlingoverview}

\latexignore{\rtfignore{\wxheading{Members}}}

\membersection{wxDropFilesEvent::wxDropFilesEvent}

\func{}{wxDropFilesEvent}{\param{WXTYPE }{id = 0}, \param{int }{noFiles = 0},\rtfsp
\param{wxString* }{files = NULL}}

Constructor.

\membersection{wxDropFilesEvent::m\_files}

\member{wxString*}{m\_files}

An array of filenames.

\membersection{wxDropFilesEvent::m\_noFiles}

\member{int}{m\_noFiles}

The number of files dropped.

\membersection{wxDropFilesEvent::m\_pos}

\member{wxPoint}{m\_pos}

The point at which the drop took place.

\membersection{wxDropFilesEvent::GetFiles}\label{wxdropfileseventgetfiles}

\constfunc{wxString*}{GetFiles}{\void}

Returns an array of filenames.

\membersection{wxDropFilesEvent::GetNumberOfFiles}\label{wxdropfileseventgetnumberoffiles}

\constfunc{int}{GetNumberOfFiles}{\void}

Returns the number of files dropped.

\membersection{wxDropFilesEvent::GetPosition}\label{wxdropfileseventgetposition}

\constfunc{wxPoint}{GetPosition}{\void}

Returns the position at which the files were dropped.

Returns an array of filenames.


