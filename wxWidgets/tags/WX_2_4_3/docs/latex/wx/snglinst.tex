%%%%%%%%%%%%%%%%%%%%%%%%%%%%%%%%%%%%%%%%%%%%%%%%%%%%%%%%%%%%%%%%%%%%%%%%%%%%%%%
%% Name:        snglinst.tex
%% Purpose:     wxSingleInstanceChecker documentation
%% Author:      Vadim Zeitlin
%% Modified by:
%% Created:     08.06.01
%% RCS-ID:      $Id$
%% Copyright:   (c) 2001 Vadim Zeitlin
%% License:     wxWidgets license
%%%%%%%%%%%%%%%%%%%%%%%%%%%%%%%%%%%%%%%%%%%%%%%%%%%%%%%%%%%%%%%%%%%%%%%%%%%%%%%

\section{\class{wxSingleInstanceChecker}}\label{wxsingleinstancechecker}

wxSingleInstanceChecker class allows to check that only a single instance of a
program is running. To do it, you should create an object of this class. As
long as this object is alive, calls to 
\helpref{IsAnotherRunning()}{wxsingleinstancecheckerisanotherrunning} from
other processes will return {\tt TRUE}.

As the object should have the life span as big as possible, it makes sense to
create it either as a global or in \helpref{wxApp::OnInit}{wxapponinit}. For
example:

\begin{verbatim}
bool MyApp::OnInit()
{
    const wxString name = wxString::Format("MyApp-%s", wxGetUserId().c_str());
    m_checker = new wxSingleInstanceChecker(name);
    if ( m_checker->IsAnotherRunning() )
    {
        wxLogError(_("Another program instance is already running, aborting."));

        return FALSE;
    }

    ... more initializations ...

    return TRUE;
}

int MyApp::OnExit()
{
    delete m_checker;

    return 0;
}
\end{verbatim}

Note using \helpref{wxGetUserId()}{wxgetuserid} to construct the name: this
allows different user to run the application concurrently which is usually the
intended goal. If you don't use the user name in the wxSingleInstanceChecker
name, only one user would be able to run the application at a time.

This class is implemented for Win32 and Unix platforms (supporting {\tt fcntl()}
system call, but almost all of modern Unix systems do) only.

\wxheading{Derived from}

No base class

\wxheading{Include files}

<wx/snglinst.h>

\latexignore{\rtfignore{\wxheading{Members}}}

\membersection{wxSingleInstanceChecker::wxSingleInstanceChecker}\label{wxsingleinstancecheckerctor}

\func{}{wxSingleInstanceChecker}{\void}

Default ctor, use \helpref{Create()}{wxsingleinstancecheckercreate} after it.

\membersection{wxSingleInstanceChecker::wxSingleInstanceChecker}\label{wxsingleinstancecheckerwxsingleinstancechecker}

\func{}{wxSingleInstanceChecker}{\param{const wxString\& }{name}, \param{const wxString\& }{path = wxEmptyString}}

Like \helpref{Create()}{wxsingleinstancecheckercreate} but without
error checking.

\membersection{wxSingleInstanceChecker::Create}\label{wxsingleinstancecheckercreate}

\func{bool}{Create}{\param{const wxString\& }{name}, \param{const wxString\& }{path = wxEmptyString}}

Initialize the object if it had been created using the default constructor.
Note that you can't call Create() more than once, so calling it if the 
\helpref{non default ctor}{wxsingleinstancecheckerwxsingleinstancechecker} 
had been used is an error.

\wxheading{Parameters}

\docparam{name}{must be given and be as unique as possible. It is used as the
mutex name under Win32 and the lock file name under Unix. 
\helpref{GetAppName()}{wxappgetappname} and \helpref{wxGetUserId()}{wxgetuserid} 
are commonly used to construct this parameter.}

\docparam{path}{is optional and is ignored under Win32 and used as the directory to
create the lock file in under Unix (default is 
\helpref{wxGetHomeDir()}{wxgethomedir})}

\wxheading{Return value}

Returns {\tt FALSE} if initialization failed, it doesn't mean that another
instance is running - use 
\helpref{IsAnotherRunning()}{wxsingleinstancecheckerisanotherrunning} to check
for it.

\membersection{wxSingleInstanceChecker::IsAnotherRunning}\label{wxsingleinstancecheckerisanotherrunning}

\constfunc{bool}{IsAnotherRunning}{\void}

Returns {\tt TRUE} if another copy of this program is already running, {\tt
FALSE} otherwise.

\membersection{wxSingleInstanceChecker::\destruct{wxSingleInstanceChecker}}\label{wxsingleinstancecheckerdtor}

\func{}{\destruct{wxSingleInstanceChecker}}{\void}

Destructor frees the associated resources.

Note that it is not virtual, this class is not meant to be used polymorphically

