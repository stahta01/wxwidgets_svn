\section{\class{wxTextFile}}\label{wxtextfile}

The wxTextFile is a simple class which allows to work with text files on line by
line basis. It also understands the differences in line termination characters
under different platforms and will not do anything bad to files with "non
native" line termination sequences - in fact, it can be also used to modify the
text files and change the line termination characters from one type (say DOS) to
another (say Unix).

One word of warning: the class is not at all optimized for big files and so it
will load the file entirely into memory when opened. Of course, you should not
work in this way with large files (as an estimation, anything over 1 Megabyte is
surely too big for this class). On the other hand, it is not a serious
limitation for the small files like configuration files or programs sources
which are well handled by wxTextFile.

The typical things you may do with wxTextFile in order are:

\begin{itemize}\itemsep=0pt
\item Create and open it: this is done with \helpref{Open}{wxtextfileopen} 
function which opens the file (name may be specified either as Open argument or
in the constructor), reads its contents in memory and closes it. If all of these
operations are successful, Open() will return TRUE and FALSE on error.
\item Work with the lines in the file: this may be done either with "direct
access" functions like \helpref{GetLineCount}{wxtextfilegetlinecount} and 
\helpref{GetLine}{wxtextfilegetline} ({\it operator[]} does exactly the same
but looks more like array addressing) or with "sequential access" functions
which include \helpref{GetFirstLine}{wxtextfilegetfirstline}/
\helpref{GetNextLine}{wxtextfilegetnextline} and also 
\helpref{GetLastLine}{wxtextfilegetlastline}/\helpref{GetPrevLine}{wxtextfilegetprevline}.
For the sequential access functions the current line number is maintained: it is
returned by \helpref{GetCurrentLine}{wxtextfilegetcurrentline} and may be
changed with \helpref{GoToLine}{wxtextfilegotoline}.
\item Add/remove lines to the file: \helpref{AddLine}{wxtextfileaddline} and 
\helpref{InsertLine}{wxtextfileinsertline} add new lines while 
\helpref{RemoveLine}{wxtextfileremoveline} deletes the existing ones.
\item Save your changes: notice that the changes you make to the file will {\bf
not} be saved automatically; calling \helpref{Close}{wxtextfileclose} or doing
nothing discards them! To save the changes you must explicitly call 
\helpref{Write}{wxtextfilewrite} - here, you may also change the line
termination type if you wish.
\end{itemize}

\wxheading{Derived from}

No base class

\wxheading{Include files}

<wx/textfile.h>

\wxheading{Data structures}

The following constants identify the line termination type:
\begin{verbatim}
enum wxTextFileType
{
    wxTextFileType_None,  // incomplete (the last line of the file only)
    wxTextFileType_Unix,  // line is terminated with 'LF' = 0xA = 10 = '\n'
    wxTextFileType_Dos,   //                         'CR' 'LF'
    wxTextFileType_Mac    //                         'CR' = 0xD = 13 = '\r'
};
\end{verbatim}

\wxheading{See also}

\helpref{wxFile}{wxfile}

\latexignore{\rtfignore{\wxheading{Members}}}

\membersection{wxTextFile::wxTextFile}\label{wxtextfilectordef}

\constfunc{}{wxTextFile}{\void}

Default constructor, use Open(string) to initialize the object.

\membersection{wxTextFile::wxTextFile}\label{wxtextfilector}

\constfunc{}{wxTextFile}{\param{const wxString\& }{strFile}}

Constructor does not load the file into memory, use Open() to do it. 

\membersection{wxTextFile::Exists}\label{wxtextfileexists}

\constfunc{bool}{Exists}{\void}

Return TRUE if file exists - the name of the file should have been specified
in the constructor before calling Exists().

\membersection{wxTextFile::Open}\label{wxtextfileopen}

\constfunc{bool}{Open}{\void}

Open() opens the file with the name which was given in the \helpref{constructor}{wxtextfilector} 
and also loads file in memory on success.

\membersection{wxTextFile::Open}\label{wxtextfileopenname}

\constfunc{bool}{Open}{\param{const wxString\& }{strFile}}

Same as \helpref{Open()}{wxtextfileopen} but allows to specify the file name
(must be used if the default constructor was used to create the object).

\membersection{wxTextFile::Close}\label{wxtextfileclose}

\constfunc{bool}{Close}{\void}

Closes the file and frees memory, {\bf losing all changes}. Use \helpref{Write()}{wxtextfilewrite} 
if you want to save them.

\membersection{wxTextFile::IsOpened}\label{wxtextfileisopened}

\constfunc{bool}{IsOpened}{\void}

Returns TRUE if the file is currently opened.

\membersection{wxTextFile::GetLineCount}\label{wxtextfilegetlinecount}

\constfunc{size\_t}{GetLineCount}{\void}

Get the number of lines in the file.

\membersection{wxTextFile::GetLine}\label{wxtextfilegetline}

\constfunc{wxString\&}{GetLine}{\param{size\_t }{n}}

Retrieves the line number {\it n} from the file. The returned line may be
modified but you shouldn't add line terminator at the end - this will be done
by wxTextFile.

\membersection{wxTextFile::operator[]}\label{wxtextfileoperatorarray}

\constfunc{wxString\&}{operator[]}{\param{size\_t }{n}}

The same as \helpref{GetLine}{wxtextfilegetline}.

\membersection{wxTextFile::GetCurrentLine}\label{wxtextfilegetcurrentline}

\constfunc{size\_t}{GetCurrentLine}{\void}

Returns the current line: it has meaning only when you're using
GetFirstLine()/GetNextLine() functions, it doesn't get updated when
you're using "direct access" functions like GetLine(). GetFirstLine() and
GetLastLine() also change the value of the current line, as well as
GoToLine().

\membersection{wxTextFile::GoToLine}\label{wxtextfilegotoline}

\constfunc{void}{GoToLine}{\param{size\_t }{n}}

Changes the value returned by \helpref{GetCurrentLine}{wxtextfilegetcurrentline} 
and used by \helpref{GetFirstLine()}{wxtextfilegetfirstline}/\helpref{GetNextLine()}{wxtextfilegetnextline}.

\membersection{wxTextFile::Eof}\label{wxtextfileeof}

\constfunc{bool}{Eof}{\void}

Returns TRUE if the current line is the last one.

\membersection{wxTextFile::GetFirstLine}\label{wxtextfilegetfirstline}

\constfunc{wxString\&}{GetFirstLine}{\void}

This method together with \helpref{GetNextLine()}{wxtextfilegetnextline} 
allows more "iterator-like" traversal of the list of lines, i.e. you may
write something like:

\begin{verbatim}
for ( str = GetFirstLine(); !Eof(); str = GetNextLine() )
{
    // do something with the current line in str
}
\end{verbatim}

\membersection{wxTextFile::GetNextLine}\label{wxtextfilegetnextline}

\func{wxString\&}{GetNextLine}{\void}

Gets the next line (see \helpref{GetFirstLine}{wxtextfilegetfirstline} for 
the example).

\membersection{wxTextFile::GetPrevLine}\label{wxtextfilegetprevline}

\func{wxString\&}{GetPrevLine}{\void}

Gets the previous line in the file.

\membersection{wxTextFile::GetLastLine}\label{wxtextfilegetlastline}

\func{wxString\&}{GetLastLine}{\void}

Gets the last line of the file.

\membersection{wxTextFile::GetLineType}\label{wxtextfilegetlinetype}

\constfunc{wxTextFileType}{GetLineType}{\param{size\_t }{n}}

Get the type of the line (see also \helpref{GetEOL}{wxtextfilegeteol})

\membersection{wxTextFile::GuessType}\label{wxtextfileguesstype}

\constfunc{wxTextFileType}{GuessType}{\void}

Guess the type of file (which is supposed to be opened). If sufficiently
many lines of the file are in DOS/Unix/Mac format, the corresponding value will
be returned. If the detection mechanism fails wxTextFileType\_None is returned.

\membersection{wxTextFile::GetName}\label{wxtextfilegetname}

\constfunc{const char*}{GetName}{\void}

Get the name of the file.

\membersection{wxTextFile::AddLine}\label{wxtextfileaddline}

\constfunc{void}{AddLine}{\param{const wxString\& }{str}, \param{wxTextFileType }{type = typeDefault}}

Adds a line to the end of file.

\membersection{wxTextFile::InsertLine}\label{wxtextfileinsertline}

\constfunc{void}{InsertLine}{\param{const wxString\& }{str}, \param{size\_t }{n}, \param{wxTextFileType }{type = typeDefault}}

Insert a line before the line number {\it n}.

\membersection{wxTextFile::RemoveLine}\label{wxtextfileremoveline}

\constfunc{void}{RemoveLine}{\param{size\_t }{n}}

Delete line number {\it n} from the file.

\membersection{wxTextFile::Write}\label{wxtextfilewrite}

\constfunc{bool}{Write}{\param{wxTextFileType }{typeNew = wxTextFileType\_None}}

Change the file on disk. The {\it typeNew} parameter allows you to change the
file format (default argument means "don't change type") and may be used to
convert, for example, DOS files to Unix.

Returns TRUE if operation succeeded, FALSE if it failed.

\membersection{wxTextFile::GetEOL}\label{wxtextfilegeteol}

\constfunc{static const char*}{GetEOL}{\param{wxTextFileType }{type = typeDefault}}

Get the line termination string corresponding to given constant. {\it typeDefault} is
the value defined during the compilation and corresponds to the native format of the
platform, i.e. it will be wxTextFileType\_Dos under Windows, wxTextFileType\_Unix under
Unix and wxTextFileType\_Mac under Mac.

\membersection{wxTextFile::\destruct{wxTextFile}}\label{wxtextfiledtor}

\constfunc{}{\destruct{wxTextFile}}{\void}

Destructor does nothing.

