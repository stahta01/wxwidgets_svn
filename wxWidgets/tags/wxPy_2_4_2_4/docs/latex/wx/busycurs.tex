\section{\class{wxBusyCursor}}\label{wxbusycursor}

This class makes it easy to tell your user that the program is temporarily busy.
Just create a wxBusyCursor object on the stack, and within the current scope,
the hourglass will be shown.

For example:

\begin{verbatim}
  wxBusyCursor wait;

  for (int i = 0; i < 100000; i++)
    DoACalculation();
\end{verbatim}

It works by calling \helpref{wxBeginBusyCursor}{wxbeginbusycursor} in the constructor,
and \helpref{wxEndBusyCursor}{wxendbusycursor} in the destructor.

\wxheading{Derived from}

None

\wxheading{Include files}

<wx/utils.h>

\wxheading{See also}

\helpref{wxBeginBusyCursor}{wxbeginbusycursor},\rtfsp
\helpref{wxEndBusyCursor}{wxendbusycursor},\rtfsp
\helpref{wxWindowDisabler}{wxwindowdisabler}

\latexignore{\rtfignore{\wxheading{Members}}}

\membersection{wxBusyCursor::wxBusyCursor}

\func{}{wxBusyCursor}{\param{wxCursor*}{ cursor = wxHOURGLASS\_CURSOR}}

Constructs a busy cursor object, calling \helpref{wxBeginBusyCursor}{wxbeginbusycursor}.

\membersection{wxBusyCursor::\destruct{wxBusyCursor}}

\func{}{\destruct{wxBusyCursor}}{\void}

Destroys the busy cursor object, calling \helpref{wxEndBusyCursor}{wxendbusycursor}.

