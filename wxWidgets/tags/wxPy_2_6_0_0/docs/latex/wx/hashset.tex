\section{\class{wxHashSet}}\label{wxhashset}

This is a simple, type-safe, and reasonably efficient hash set class,
whose interface is a subset of the interface of STL containers. In
particular, the interface is modeled after std::set, and the various,
non standard, std::hash\_map.

\wxheading{Example}

\begin{verbatim}
    class MyClass { /* ... */ };

    // same, with MyClass* keys (only uses pointer equality!)
    WX_DECLARE_HASH_SET( MyClass*, wxPointerHash, wxPointerEqual, MySet1 );
    // same, with int keys
    WX_DECLARE_HASH_SET( int, wxIntegerHash, wxIntegerEqual, MySet2 );
    // declare a hash set with string keys
    WX_DECLARE_HASH_SET( wxString, wxStringHash, wxStringEqual, MySet3 );

    MySet1 h1;
    MySet2 h1;
    MySet3 h3;

    // store and retrieve values
    h1.insert( new MyClass( 1 ) );

    h3.insert( "foo" );
    h3.insert( "bar" );
    h3.insert( "baz" );

    int size = h3.size(); // now is three
    bool has_foo = h3.find( "foo" ) != h3.end();

    h3.insert( "bar" ); // still has size three

    // iterate over all the elements in the class
    MySet3::iterator it;
    for( it = h3.begin(); it != h3.end(); ++it )
    {
        wxString key = *it;
        // do something useful with key
    }
\end{verbatim}

\wxheading{Declaring new hash set types}

\begin{verbatim}
    WX_DECLARE_HASH_SET( KEY_T,      // type of the keys
                         HASH_T,     // hasher
                         KEY_EQ_T,   // key equality predicate
                         CLASSNAME); // name of the class
\end{verbatim}

The HASH\_T and KEY\_EQ\_T are the types
used for the hashing function and key comparison. wxWidgets provides
three predefined hashing functions: {\tt wxIntegerHash}
for integer types ( {\tt int}, {\tt long}, {\tt short},
and their unsigned counterparts ), {\tt wxStringHash} for strings
( {\tt wxString}, {\tt wxChar*}, {\tt char*} ), and
{\tt wxPointerHash} for any kind of pointer.
Similarly three equality predicates:
{\tt wxIntegerEqual}, {\tt wxStringEqual}, {\tt wxPointerEqual} are provided.

Using this you could declare a hash set using {\tt int} values like this:

\begin{verbatim}
    WX_DECLARE_HASH_SET( int,
                         wxIntegerHash,
                         wxIntegerEqual,
                         MySet );

    // using an user-defined class for keys
    class MyKey { /* ... */ };

    // hashing function
    class MyKeyHash
    {
    public:
        MyKeyHash() { }

        unsigned long operator()( const MyKey& k ) const
            { /* compute the hash */ }

        MyKeyHash& operator=(const MyKeyHash&) { return *this; }
    };

    // comparison operator
    class MyKeyEqual
    {
    public:
        MyKeyEqual() { }
        bool operator()( const MyKey& a, const MyKey& b ) const
            { /* compare for equality */ }

        MyKeyEqual& operator=(const MyKeyEqual&) { return *this; }
    };

    WX_DECLARE_HASH_SET( MyKey,      // type of the keys
                         MyKeyHash,  // hasher
                         MyKeyEqual, // key equality predicate
                         CLASSNAME); // name of the class
\end{verbatim}

\latexignore{\rtfignore{\wxheading{Types}}}

In the documentation below you should replace wxHashSet with the name
you used in the class declaration.

\begin{twocollist}
\twocolitem{wxHashSet::key\_type}{Type of the hash keys}
\twocolitem{wxHashSet::mapped\_type}{Type of hash keys}
\twocolitem{wxHashSet::value\_type}{Type of hash keys}
\twocolitem{wxHashSet::iterator}{Used to enumerate all the elements in a hash
set; it is similar to a {\tt value\_type*}}
\twocolitem{wxHashSet::const\_iterator}{Used to enumerate all the elements
in a constant hash set; it is similar to a {\tt const value\_type*}}
\twocolitem{wxHashSet::size\_type}{Used for sizes}
\twocolitem{wxHashSet::Insert\_Result}{The return value for
\helpref{insert()}{wxhashsetinsert}}
\end{twocollist}

\wxheading{Iterators}

An iterator is similar to a pointer, and so you can use the usual pointer
operations: {\tt ++it} ( and {\tt it++} ) to move to the next element,
{\tt *it} to access the element pointed to, {\tt *it}
to access the value of the element pointed to.
Hash sets provide forward only iterators, this
means that you can't use {\tt --it}, {\tt it + 3}, {\tt it1 - it2}.

\wxheading{Include files}

<wx/hashset.h>

\latexignore{\rtfignore{\wxheading{Members}}}

\membersection{wxHashSet::wxHashSet}\label{wxhashsetctor}

\func{}{wxHashSet}{\param{size\_type}{ size = 10}}

The size parameter is just a hint, the table will resize automatically
to preserve performance.

\func{}{wxHashSet}{\param{const wxHashSet\&}{ set}}

Copy constructor.

\membersection{wxHashSet::begin}\label{wxhashsetbegin}

\constfunc{const\_iterator}{begin}{}

\func{iterator}{begin}{}

Returns an iterator pointing at the first element of the hash set.
Please remember that hash sets do not guarantee ordering.

\membersection{wxHashSet::clear}\label{wxhashsetclear}

\func{void}{clear}{}

Removes all elements from the hash set.

\membersection{wxHashSet::count}\label{wxhashsetcount}

\constfunc{size\_type}{count}{\param{const key\_type\&}{ key}}

Counts the number of elements with the given key present in the set.
This function returns only 0 or 1.

\membersection{wxHashSet::empty}\label{wxhashsetempty}

\constfunc{bool}{empty}{}

Returns true if the hash set does not contain any elements, false otherwise.

\membersection{wxHashSet::end}\label{wxhashsetend}

\constfunc{const\_iterator}{end}{}

\func{iterator}{end}{}

Returns an iterator pointing at the one-after-the-last element of the hash set.
Please remember that hash sets do not guarantee ordering.

\membersection{wxHashSet::erase}\label{wxhashseterase}

\func{size\_type}{erase}{\param{const key\_type\&}{ key}}

Erases the element with the given key, and returns the number of elements
erased (either 0 or 1).

\func{void}{erase}{\param{iterator}{ it}}

\func{void}{erase}{\param{const\_iterator}{ it}}

Erases the element pointed to by the iterator. After the deletion
the iterator is no longer valid and must not be used.

\membersection{wxHashSet::find}\label{wxhashsetfind}

\func{iterator}{find}{\param{const key\_type\&}{ key}}

\constfunc{const\_iterator}{find}{\param{const key\_type\&}{ key}}

If an element with the given key is present, the functions returns
an iterator pointing at that element, otherwise an invalid iterator
is returned (i.e. hashset.find( non\_existent\_key ) == hashset.end()).

\membersection{wxHashSet::insert}\label{wxhashsetinsert}

\func{Insert\_Result}{insert}{\param{const value\_type\&}{ v}}

Inserts the given value in the hash set. The return value is
equivalent to a \texttt{std::pair<wxHashMap::iterator, bool>};
the iterator points to the inserted element, the boolean value
is \texttt{true} if \texttt{v} was actually inserted.

\membersection{wxHashSet::size}\label{wxhashsetsize}

\constfunc{size\_type}{size}{}

Returns the number of elements in the set.

