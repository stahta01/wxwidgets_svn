\section{\class{wxObject}}\label{wxobject}

This is the root class of all wxWindows classes.
It declares a virtual destructor which ensures that
destructors get called for all derived class objects where necessary.

wxObject is the hub of a dynamic object creation
scheme, enabling a program to create instances of a class only knowing
its string class name, and to query the class hierarchy.

The class contains optional debugging versions
of {\bf new} and {\bf delete}, which can help trace memory allocation
and deallocation problems.

wxObject can be used to implement reference counted objects, such as
wxPen, wxBitmap and others.

\wxheading{See also}

\helpref{wxClassInfo}{wxclassinfo}, \helpref{Debugging overview}{debuggingoverview},\rtfsp
\helpref{wxObjectRefData}{wxobjectrefdata}

\latexignore{\rtfignore{\wxheading{Members}}}

\membersection{wxObject::wxObject}\label{wxobjectconstr}

\func{}{wxObject}{\void}

Default constructor.

\membersection{wxObject::\destruct{wxObject}}

\func{}{wxObject}{\void}

Destructor. Performs dereferencing, for those objects
that use reference counting.

\membersection{wxObject::m\_refData}\label{wxobjectmrefdata}

\member{wxObjectRefData* }{m\_refData}

Pointer to an object which is the object's reference-counted data.

\wxheading{See also}

\helpref{wxObject::Ref}{wxobjectref}, \helpref{wxObject::UnRef}{wxobjectunref},\rtfsp
\helpref{wxObject::SetRefData}{wxobjectsetrefdata},\rtfsp
\helpref{wxObject::GetRefData}{wxobjectgetrefdata},\rtfsp
\helpref{wxObjectRefData}{wxobjectrefdata}

\membersection{wxObject::CopyObject}\label{wxobjectcopyobject}

\func{virtual void}{CopyObject}{\param{wxObject\&}{ object\_dest}}

Create a copy of this object in object\_dest. Must perform a full copy of self
so that object\_dest will be valid after this object is deleted

\membersection{wxObject::Dump}\label{wxobjectdump}

\func{void}{Dump}{\param{ostream\&}{ stream}}

A virtual function that should be redefined by derived classes to allow dumping of
memory states.

\wxheading{Parameters}

\docparam{stream}{Stream on which to output dump information.}

\wxheading{Remarks}

Currently wxWindows does not define Dump for derived classes, but
programmers may wish to use it for their own applications. Be sure to
call the Dump member of the class's base class to allow all information to be dumped.

The implementation of this function just writes the class name of the object
in debug build (\_\_WXDEBUG\_\_ defined), otherwise it does nothing.

\membersection{wxObject::GetClassInfo}\label{wxobjectgetclassinfo}

\func{wxClassInfo *}{GetClassInfo}{\void}

This virtual function is redefined for every class that requires run-time
type information, when using DECLARE\_CLASS macros.

\membersection{wxObject::GetRefData}\label{wxobjectgetrefdata}

\constfunc{wxObjectRefData*}{GetRefData}{\void}

Returns the {\bf m\_refData} pointer.

\wxheading{See also}

\helpref{wxObject::Ref}{wxobjectref}, \helpref{wxObject::UnRef}{wxobjectunref}, \helpref{wxObject::m\_refData}{wxobjectmrefdata},\rtfsp
\helpref{wxObject::SetRefData}{wxobjectsetrefdata},\rtfsp
\helpref{wxObjectRefData}{wxobjectrefdata}

\membersection{wxObject::IsKindOf}\label{wxobjectiskindof}

\func{bool}{IsKindOf}{\param{wxClassInfo *}{info}}

Determines whether this class is a subclass of (or the same class as)
the given class.

\wxheading{Parameters}

\docparam{info}{A pointer to a class information object, which may be obtained
by using the CLASSINFO macro.}

\wxheading{Return value}

TRUE if the class represented by {\it info} is the same class as
this one or is derived from it.

\wxheading{Example}

\begin{verbatim}
  bool tmp = obj->IsKindOf(CLASSINFO(wxFrame));
\end{verbatim}

\membersection{wxObject::Ref}\label{wxobjectref}

\func{void}{Ref}{\param{const wxObject\& }{clone}}

Makes this object refer to the data in {\it clone}.

\wxheading{Parameters}

\docparam{clone}{The object to `clone'.}

\wxheading{Remarks}

First this function calls \helpref{wxObject::UnRef}{wxobjectunref} on itself
to decrement (and perhaps free) the data it is currently referring to.

It then sets its own m\_refData to point to that of {\it clone}, and increments the reference count
inside the data.

\wxheading{See also}

\helpref{wxObject::UnRef}{wxobjectunref}, \helpref{wxObject::m\_refData}{wxobjectmrefdata},\rtfsp
\helpref{wxObject::SetRefData}{wxobjectsetrefdata}, \helpref{wxObject::GetRefData}{wxobjectgetrefdata},\rtfsp
\helpref{wxObjectRefData}{wxobjectrefdata}

\membersection{wxObject::SetRefData}\label{wxobjectsetrefdata}

\func{void}{SetRefData}{\param{wxObjectRefData*}{ data}}

Sets the {\bf m\_refData} pointer.

\wxheading{See also}

\helpref{wxObject::Ref}{wxobjectref}, \helpref{wxObject::UnRef}{wxobjectunref}, \helpref{wxObject::m\_refData}{wxobjectmrefdata},\rtfsp
\helpref{wxObject::GetRefData}{wxobjectgetrefdata},\rtfsp
\helpref{wxObjectRefData}{wxobjectrefdata}

\membersection{wxObject::UnRef}\label{wxobjectunref}

\func{void}{UnRef}{\void}

Decrements the reference count in the associated data, and if it is zero, deletes the data.
The {\bf m\_refData} member is set to NULL.

\wxheading{See also}

\helpref{wxObject::Ref}{wxobjectref}, \helpref{wxObject::m\_refData}{wxobjectmrefdata},\rtfsp
\helpref{wxObject::SetRefData}{wxobjectsetrefdata}, \helpref{wxObject::GetRefData}{wxobjectgetrefdata},\rtfsp
\helpref{wxObjectRefData}{wxobjectrefdata}

\membersection{wxObject::operator new}\label{wxobjectnew}

\func{void *}{new}{\param{size\_t }{size}, \param{const wxString\& }{filename = NULL}, \param{int}{ lineNum = 0}}

The {\it new} operator is defined for debugging versions of the library only, when
the identifier \_\_WXDEBUG\_\_ is defined. It takes over memory allocation, allowing
wxDebugContext operations.

\membersection{wxObject::operator delete}\label{wxobjectdelete}

\func{void}{delete}{\param{void }{buf}}

The {\it delete} operator is defined for debugging versions of the library only, when
the identifier \_\_WXDEBUG\_\_ is defined. It takes over memory deallocation, allowing
wxDebugContext operations.

\section{\class{wxObjectRefData}}\label{wxobjectrefdata}

This class is used to store reference-counted data. Derive classes from this to
store your own data. When retrieving information from a {\bf wxObject}'s reference data,
you will need to cast to your own derived class.

\wxheading{Friends}

\helpref{wxObject}{wxobject}

\wxheading{See also}

\helpref{wxObject}{wxobject}

\latexignore{\rtfignore{\wxheading{Members}}}

\membersection{wxObjectRefData::m\_count}

\member{int}{m\_count}

Reference count. When this goes to zero during a \helpref{wxObject::UnRef}{wxobjectunref}, an object
can delete the {\bf wxObjectRefData} object.

\membersection{wxObjectRefData::wxObjectRefData}\label{wxobjectrefdataconstr}

\func{}{wxObjectRefData}{\void}

Default constructor. Initialises the {\bf m\_count} member to 1.

\membersection{wxObjectRefData::\destruct{wxObjectRefData}}

\func{}{wxObjectRefData}{\void}

Destructor.


