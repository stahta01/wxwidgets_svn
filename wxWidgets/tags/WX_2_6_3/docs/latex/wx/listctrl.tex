%%%%%%%%%%%%%%%%%%%%%%%%%%%%%%%%%%%%%%%%%%%%%%%%%%%%%%%%%%%%%%%%%%%%%%%%%%%%%
%% Name:        listctrl.tex
%% Purpose:     wxListCtrl docs
%% Author:
%% Modified by:
%% Created:
%% RCS-ID:      $Id$
%% Copyright:   (c) wxWidgets
%% License:     wxWindows license
%%%%%%%%%%%%%%%%%%%%%%%%%%%%%%%%%%%%%%%%%%%%%%%%%%%%%%%%%%%%%%%%%%%%%%%%%%%%%

\section{\class{wxListCtrl}}\label{wxlistctrl}

A list control presents lists in a number of formats: list view, report view,
icon view and small icon view. In any case, elements are numbered from zero.
For all these modes, the items are stored in the control and must be added to
it using \helpref{InsertItem}{wxlistctrlinsertitem} method.

A special case of report view quite different from the other modes of the list
control is a virtual control in which the items data (including text, images
and attributes) is managed by the main program and is requested by the control
itself only when needed which allows to have controls with millions of items
without consuming much memory. To use virtual list control you must use
\helpref{SetItemCount}{wxlistctrlsetitemcount} first and overload at least
\helpref{OnGetItemText}{wxlistctrlongetitemtext} (and optionally
\helpref{OnGetItemImage}{wxlistctrlongetitemimage} and
\helpref{OnGetItemAttr}{wxlistctrlongetitemattr}) to return the information
about the items when the control requests it. Virtual list control can be used
as a normal one except that no operations which can take time proportional to
the number of items in the control happen -- this is required to allow having a
practically infinite number of items. For example, in a multiple selection
virtual list control, the selections won't be sent when many items are selected
at once because this could mean iterating over all the items.

Using many of wxListCtrl features is shown in the
\helpref{corresponding sample}{samplelistctrl}.

To intercept events from a list control, use the event table macros described
in \helpref{wxListEvent}{wxlistevent}.

\wxheading{Derived from}

\helpref{wxControl}{wxcontrol}\\
\helpref{wxWindow}{wxwindow}\\
\helpref{wxEvtHandler}{wxevthandler}\\
\helpref{wxObject}{wxobject}

\wxheading{Include files}

<wx/listctrl.h>

\wxheading{Window styles}

\twocolwidtha{7cm}
\begin{twocollist}\itemsep=0pt
\twocolitem{\windowstyle{wxLC\_LIST}}{Multicolumn list view, with optional small icons.
Columns are computed automatically, i.e. you don't set columns as in wxLC\_REPORT. In other words,
the list wraps, unlike a wxListBox.}
\twocolitem{\windowstyle{wxLC\_REPORT}}{Single or multicolumn report view, with optional header.}
\twocolitem{\windowstyle{wxLC\_VIRTUAL}}{The application provides items text on demand. May only be used with wxLC\_REPORT.}
\twocolitem{\windowstyle{wxLC\_ICON}}{Large icon view, with optional labels.}
\twocolitem{\windowstyle{wxLC\_SMALL\_ICON}}{Small icon view, with optional labels.}
\twocolitem{\windowstyle{wxLC\_ALIGN\_TOP}}{Icons align to the top. Win32 default, Win32 only. }
\twocolitem{\windowstyle{wxLC\_ALIGN\_LEFT}}{Icons align to the left. }
\twocolitem{\windowstyle{wxLC\_AUTOARRANGE}}{Icons arrange themselves. Win32 only. }
\twocolitem{\windowstyle{wxLC\_EDIT\_LABELS}}{Labels are editable: the application will be notified when editing starts.}
\twocolitem{\windowstyle{wxLC\_NO\_HEADER}}{No header in report mode. }
\twocolitem{\windowstyle{wxLC\_SINGLE\_SEL}}{Single selection (default is multiple).}
\twocolitem{\windowstyle{wxLC\_SORT\_ASCENDING}}{Sort in ascending order (must still supply a comparison callback in SortItems.}
\twocolitem{\windowstyle{wxLC\_SORT\_DESCENDING}}{Sort in descending order (must still supply a comparison callback in SortItems.}
\twocolitem{\windowstyle{wxLC\_HRULES}}{Draws light horizontal rules between rows in report mode.}
\twocolitem{\windowstyle{wxLC\_VRULES}}{Draws light vertical rules between columns in report mode.}
\end{twocollist}

See also \helpref{window styles overview}{windowstyles}.

\wxheading{Event handling}

To process input from a list control, use these event handler macros to direct input to member
functions that take a \helpref{wxListEvent}{wxlistevent} argument.

\twocolwidtha{7cm}
\begin{twocollist}\itemsep=0pt
\twocolitem{{\bf EVT\_LIST\_BEGIN\_DRAG(id, func)}}{Begin dragging with the left mouse button.}
\twocolitem{{\bf EVT\_LIST\_BEGIN\_RDRAG(id, func)}}{Begin dragging with the right mouse button.}
\twocolitem{{\bf EVT\_LIST\_BEGIN\_LABEL\_EDIT(id, func)}}{Begin editing a label. This can be prevented by calling \helpref{Veto()}{wxnotifyeventveto}.}
\twocolitem{{\bf EVT\_LIST\_END\_LABEL\_EDIT(id, func)}}{Finish editing a label. This can be prevented by calling \helpref{Veto()}{wxnotifyeventveto}.}
\twocolitem{{\bf EVT\_LIST\_DELETE\_ITEM(id, func)}}{Delete an item.}
\twocolitem{{\bf EVT\_LIST\_DELETE\_ALL\_ITEMS(id, func)}}{Delete all items.}
%\twocolitem{{\bf EVT\_LIST\_GET\_INFO(id, func)}}{Request information from the application, usually the item text.}
%\twocolitem{{\bf EVT\_LIST\_SET\_INFO(id, func)}}{Information is being supplied (not implemented).}
\twocolitem{{\bf EVT\_LIST\_ITEM\_SELECTED(id, func)}}{The item has been selected.}
\twocolitem{{\bf EVT\_LIST\_ITEM\_DESELECTED(id, func)}}{The item has been deselected.}
\twocolitem{{\bf EVT\_LIST\_ITEM\_ACTIVATED(id, func)}}{The item has been activated (ENTER or double click).}
\twocolitem{{\bf EVT\_LIST\_ITEM\_FOCUSED(id, func)}}{The currently focused item has changed.}
\twocolitem{{\bf EVT\_LIST\_ITEM\_MIDDLE\_CLICK(id, func)}}{The middle mouse button has been clicked on an item.}
\twocolitem{{\bf EVT\_LIST\_ITEM\_RIGHT\_CLICK(id, func)}}{The right mouse button has been clicked on an item.}
\twocolitem{{\bf EVT\_LIST\_KEY\_DOWN(id, func)}}{A key has been pressed.}
\twocolitem{{\bf EVT\_LIST\_INSERT\_ITEM(id, func)}}{An item has been inserted.}
\twocolitem{{\bf EVT\_LIST\_COL\_CLICK(id, func)}}{A column ({\bf m\_col}) has been left-clicked.}
\twocolitem{{\bf EVT\_LIST\_COL\_RIGHT\_CLICK(id, func)}}{A column ({\bf m\_col}) has been right-clicked.}
\twocolitem{{\bf EVT\_LIST\_COL\_BEGIN\_DRAG(id, func)}}{The user started resizing a column - can be vetoed.}
\twocolitem{{\bf EVT\_LIST\_COL\_DRAGGING(id, func)}}{The divider between columns is being dragged.}
\twocolitem{{\bf EVT\_LIST\_COL\_END\_DRAG(id, func)}}{A column has been resized by the user.}
\twocolitem{{\bf EVT\_LIST\_CACHE\_HINT(id, func)}}{Prepare cache for a virtual list control}
\end{twocollist}%

\wxheading{See also}

\helpref{wxListCtrl overview}{wxlistctrloverview}, \helpref{wxListView}{wxlistview}, \helpref{wxListBox}{wxlistbox},\rtfsp
\helpref{wxTreeCtrl}{wxtreectrl}, \helpref{wxImageList}{wximagelist}, \helpref{wxListEvent}{wxlistevent},
\helpref{wxListItem}{wxlistitem}

\latexignore{\rtfignore{\wxheading{Members}}}


\membersection{wxListCtrl::wxListCtrl}\label{wxlistctrlctor}

\func{}{wxListCtrl}{\void}

Default constructor.

\func{}{wxListCtrl}{\param{wxWindow*}{ parent}, \param{wxWindowID}{ id},\rtfsp
\param{const wxPoint\&}{ pos = wxDefaultPosition}, \param{const wxSize\&}{ size = wxDefaultSize},\rtfsp
\param{long}{ style = wxLC\_ICON}, \param{const wxValidator\& }{validator = wxDefaultValidator}, \param{const wxString\& }{name = wxListCtrlNameStr}}

Constructor, creating and showing a list control.

\wxheading{Parameters}

\docparam{parent}{Parent window. Must not be NULL.}

\docparam{id}{Window identifier. A value of -1 indicates a default value.}

\docparam{pos}{Window position.}

\docparam{size}{Window size. If the default size (-1, -1) is specified then the window is sized
appropriately.}

\docparam{style}{Window style. See \helpref{wxListCtrl}{wxlistctrl}.}

\docparam{validator}{Window validator.}

\docparam{name}{Window name.}

\wxheading{See also}

\helpref{wxListCtrl::Create}{wxlistctrlcreate}, \helpref{wxValidator}{wxvalidator}


\membersection{wxListCtrl::\destruct{wxListCtrl}}\label{wxlistctrldtor}

\func{void}{\destruct{wxListCtrl}}{\void}

Destructor, destroying the list control.


\membersection{wxListCtrl::Arrange}\label{wxlistctrlarrange}

\func{bool}{Arrange}{\param{int }{flag = wxLIST\_ALIGN\_DEFAULT}}

Arranges the items in icon or small icon view. This only has effect on Win32. {\it flag} is one of:

\twocolwidtha{5cm}
\begin{twocollist}\itemsep=0pt
\twocolitem{wxLIST\_ALIGN\_DEFAULT}{Default alignment.}
\twocolitem{wxLIST\_ALIGN\_LEFT}{Align to the left side of the control.}
\twocolitem{wxLIST\_ALIGN\_TOP}{Align to the top side of the control.}
\twocolitem{wxLIST\_ALIGN\_SNAP\_TO\_GRID}{Snap to grid.}
\end{twocollist}


\membersection{wxListCtrl::AssignImageList}\label{wxlistctrlassignimagelist}

\func{void}{AssignImageList}{\param{wxImageList*}{ imageList}, \param{int }{which}}

Sets the image list associated with the control and
takes ownership of it (i.e. the control will, unlike when using
SetImageList, delete the list when destroyed). {\it which} is one of
wxIMAGE\_LIST\_NORMAL, wxIMAGE\_LIST\_SMALL, wxIMAGE\_LIST\_STATE (the last is unimplemented).

\wxheading{See also}

\helpref{wxListCtrl::SetImageList}{wxlistctrlsetimagelist}


\membersection{wxListCtrl::ClearAll}\label{wxlistctrlclearall}

\func{void}{ClearAll}{}

Deletes all items and all columns.


\membersection{wxListCtrl::Create}\label{wxlistctrlcreate}

\func{bool}{Create}{\param{wxWindow*}{ parent}, \param{wxWindowID}{ id},\rtfsp
\param{const wxPoint\&}{ pos = wxDefaultPosition}, \param{const wxSize\&}{ size = wxDefaultSize},\rtfsp
\param{long}{ style = wxLC\_ICON}, \param{const wxValidator\& }{validator = wxDefaultValidator}, \param{const wxString\& }{name = wxListCtrlNameStr}}

Creates the list control. See \helpref{wxListCtrl::wxListCtrl}{wxlistctrlctor} for further details.


\membersection{wxListCtrl::DeleteAllItems}\label{wxlistctrldeleteallitems}

\func{bool}{DeleteAllItems}{}

Deletes all the items in the list control.

{\bf NB:} This function does {\it not} send the
{\tt wxEVT\_COMMAND\_LIST\_DELETE\_ITEM} event because deleting many items
from the control would be too slow then (unlike \helpref{DeleteItem}{wxlistctrldeleteitem}).


\membersection{wxListCtrl::DeleteColumn}\label{wxlistctrldeletecolumn}

\func{bool}{DeleteColumn}{\param{int }{col}}

Deletes a column.


\membersection{wxListCtrl::DeleteItem}\label{wxlistctrldeleteitem}

\func{bool}{DeleteItem}{\param{long }{item}}

Deletes the specified item. This function sends the
{\tt wxEVT\_COMMAND\_LIST\_DELETE\_ITEM} event for the item being deleted.

See also: \helpref{DeleteAllItems}{wxlistctrldeleteallitems}


\membersection{wxListCtrl::EditLabel}\label{wxlistctrledit}

\func{void}{EditLabel}{\param{long }{item}}

Starts editing the label of the given item. This function generates a
EVT\_LIST\_BEGIN\_LABEL\_EDIT event which can be vetoed so that no
text control will appear for in-place editing.

If the user changed the label (i.e. s/he does not press ESC or leave
the text control without changes, a EVT\_LIST\_END\_LABEL\_EDIT event
will be sent which can be vetoed as well.


\membersection{wxListCtrl::EnsureVisible}\label{wxlistctrlensurevisible}

\func{bool}{EnsureVisible}{\param{long }{item}}

Ensures this item is visible.


\membersection{wxListCtrl::FindItem}\label{wxlistctrlfinditem}

\func{long}{FindItem}{\param{long }{start}, \param{const wxString\& }{str}, \param{const bool }{partial = false}}

Find an item whose label matches this string, starting from {\it start} or
the beginning if {\it start} is -1.

\func{long}{FindItem}{\param{long }{start}, \param{long }{data}}

Find an item whose data matches this data, starting from {\it start} or
the beginning if 'start' is -1.

\func{long}{FindItem}{\param{long }{start}, \param{const wxPoint\& }{pt}, \param{int }{direction}}

Find an item nearest this position in the specified direction, starting from
{\it start} or the beginning if {\it start} is -1.

\pythonnote{In place of a single overloaded method name, wxPython
implements the following methods:\par
\indented{2cm}{\begin{twocollist}
\twocolitem{{\bf FindItem(start, str, partial=false)}}{}
\twocolitem{{\bf FindItemData(start, data)}}{}
\twocolitem{{\bf FindItemAtPos(start, point, direction)}}{}
\end{twocollist}}
}

\perlnote{In wxPerl there are three methods instead of a single overloaded
method:\par
\indented{2cm}{\begin{twocollist}
\twocolitem{{\bf FindItem( start, str, partial = false ) }}{}
\twocolitem{{\bf FindItemData( start, data ) }}{}
\twocolitem{{\bf FindItemAtPos( start, point, direction )}}{}
\end{twocollist}
}}


\membersection{wxListCtrl::GetColumn}\label{wxlistctrlgetcolumn}

\constfunc{bool}{GetColumn}{\param{int }{col}, \param{wxListItem\& }{item}}

Gets information about this column. See \helpref{wxListCtrl::SetItem}{wxlistctrlsetitem} for more
information.

\perlnote{In wxPerl this method takes only the {\bf col} parameter and
  returns a Wx::ListItem ( or undef ).}


\membersection{wxListCtrl::GetColumnCount}\label{wxlistctrlgetcolumncount}

\constfunc{int}{GetColumnCount}{\void}

Returns the number of columns.


\membersection{wxListCtrl::GetColumnWidth}\label{wxlistctrlgetcolumnwidth}

\constfunc{int}{GetColumnWidth}{\param{int }{col}}

Gets the column width (report view only).


\membersection{wxListCtrl::GetCountPerPage}\label{wxlistctrlgetcountperpage}

\constfunc{int}{GetCountPerPage}{\void}

Gets the number of items that can fit vertically in the
visible area of the list control (list or report view)
or the total number of items in the list control (icon
or small icon view).


\membersection{wxListCtrl::GetEditControl}\label{wxlistctrlgeteditcontrol}

\constfunc{wxTextCtrl *}{GetEditControl}{\void}

Returns the edit control being currently used to edit a label. Returns {\tt NULL}
if no label is being edited.

{\bf NB:} It is currently only implemented for wxMSW.


\membersection{wxListCtrl::GetImageList}\label{wxlistctrlgetimagelist}

\constfunc{wxImageList*}{GetImageList}{\param{int }{which}}

Returns the specified image list. {\it which} may be one of:

\twocolwidtha{5cm}
\begin{twocollist}\itemsep=0pt
\twocolitem{\windowstyle{wxIMAGE\_LIST\_NORMAL}}{The normal (large icon) image list.}
\twocolitem{\windowstyle{wxIMAGE\_LIST\_SMALL}}{The small icon image list.}
\twocolitem{\windowstyle{wxIMAGE\_LIST\_STATE}}{The user-defined state image list (unimplemented).}
\end{twocollist}


\membersection{wxListCtrl::GetItem}\label{wxlistctrlgetitem}

\constfunc{bool}{GetItem}{\param{wxListItem\& }{info}}

Gets information about the item. See \helpref{wxListCtrl::SetItem}{wxlistctrlsetitem} for more
information.

You must call {\it info.SetId()} to the ID of item you're interested in
before calling this method.

\pythonnote{The wxPython version of this method takes an integer parameter
for the item ID, an optional integer for the column number, and
returns the wxListItem object.}

\perlnote{In wxPerl this method takes as parameter the {\bf ID} of the item
and ( optionally ) the column, and returns a Wx::ListItem object.}


\membersection{wxListCtrl::GetItemBackgroundColour}\label{wxlistctrlgetitembackgroundcolour}

\constfunc{wxColour}{GetItemBackgroundColour}{\param{long }{item}}

Returns the colour for this item. If the item has no specific colour, returns
an invalid colour (and not the default background control of the control
itself).

\wxheading{See also}

\helpref{GetItemTextColour}{wxlistctrlgetitemtextcolour}


\membersection{wxListCtrl::GetItemCount}\label{wxlistctrlgetitemcount}

\constfunc{int}{GetItemCount}{\void}

Returns the number of items in the list control.


\membersection{wxListCtrl::GetItemData}\label{wxlistctrlgetitemdata}

\constfunc{long}{GetItemData}{\param{long }{item}}

Gets the application-defined data associated with this item.


\membersection{wxListCtrl::GetItemFont}\label{wxlistctrlgetitemfont}

\constfunc{wxFont}{GetItemFont}{\param{long }{item}}

Returns the item's font.


\membersection{wxListCtrl::GetItemPosition}\label{wxlistctrlgetitemposition}

\constfunc{bool}{GetItemPosition}{\param{long }{item}, \param{wxPoint\& }{pos}}

Returns the position of the item, in icon or small icon view.

\pythonnote{The wxPython version of this method accepts only the item
ID and returns the wxPoint.}

\perlnote{In wxPerl this method takes only the {\bf item} parameter and
returns a Wx::Point ( or undef ).}


\membersection{wxListCtrl::GetItemRect}\label{wxlistctrlgetitemrect}

\constfunc{bool}{GetItemRect}{\param{long }{item}, \param{wxRect\& }{rect}, \param{int }{code = wxLIST\_RECT\_BOUNDS}}

Returns the rectangle representing the item's size and position, in physical
coordinates.

{\it code} is one of wxLIST\_RECT\_BOUNDS, wxLIST\_RECT\_ICON, wxLIST\_RECT\_LABEL.

\pythonnote{The wxPython version of this method accepts only the item
ID and code and returns the wxRect.}

\perlnote{In wxPerl this method takes only the {\bf item} parameter and
returns a Wx::Rect ( or undef ).}



\membersection{wxListCtrl::GetItemSpacing}\label{wxlistctrlgetitemspacing}

\constfunc{wxSize}{GetItemSpacing}{\void}

Retrieves the spacing between icons in pixels: horizontal spacing is returned
as \texttt{x} component of the \helpref{wxSize}{wxsize} object and the vertical
spacing as its \texttt{y} component.



\membersection{wxListCtrl::GetItemState}\label{wxlistctrlgetitemstate}

\constfunc{int}{GetItemState}{\param{long }{item}, \param{long }{stateMask}}

Gets the item state. For a list of state flags, see \helpref{wxListCtrl::SetItem}{wxlistctrlsetitem}.

The {\bf stateMask} indicates which state flags are of interest.


\membersection{wxListCtrl::GetItemText}\label{wxlistctrlgetitemtext}

\constfunc{wxString}{GetItemText}{\param{long }{item}}

Gets the item text for this item.


\membersection{wxListCtrl::GetItemTextColour}\label{wxlistctrlgetitemtextcolour}

\constfunc{wxColour}{GetItemTextColour}{\param{long }{item}}

Returns the colour for this item. If the item has no specific colour, returns
an invalid colour (and not the default foreground control of the control itself
as this wouldn't allow distinguishing between items having the same colour as
the current control foreground and items with default colour which, hence, have
always the same colour as the control).


\membersection{wxListCtrl::GetNextItem}\label{wxlistctrlgetnextitem}

\constfunc{long}{GetNextItem}{\param{long }{item}, \param{int }{geometry = wxLIST\_NEXT\_ALL}, \param{int }{state = wxLIST\_STATE\_DONTCARE}}

Searches for an item with the given geometry or state, starting from
{\it item} but excluding the {\it item} itself. If {\it item} is -1,
the first item that matches the specified flags will be returned.

Returns the first item with given state following {\it item} or -1 if
no such item found.

This function may be used to find all selected items in the control like this:

\begin{verbatim}
    long item = -1;
    for ( ;; )
    {
        item = listctrl->GetNextItem(item,
                                     wxLIST_NEXT_ALL,
                                     wxLIST_STATE_SELECTED);
        if ( item == -1 )
            break;

        // this item is selected - do whatever is needed with it
        wxLogMessage("Item %ld is selected."), item);
    }
\end{verbatim}

{\it geometry} can be one of:

\twocolwidtha{5cm}
\begin{twocollist}\itemsep=0pt
\twocolitem{wxLIST\_NEXT\_ABOVE}{Searches for an item above the specified item.}
\twocolitem{wxLIST\_NEXT\_ALL}{Searches for subsequent item by index.}
\twocolitem{wxLIST\_NEXT\_BELOW}{Searches for an item below the specified item.}
\twocolitem{wxLIST\_NEXT\_LEFT}{Searches for an item to the left of the specified item.}
\twocolitem{wxLIST\_NEXT\_RIGHT}{Searches for an item to the right of the specified item.}
\end{twocollist}

{\bf NB:} this parameter is only supported by wxMSW currently and ignored on
other platforms.

{\it state} can be a bitlist of the following:

\twocolwidtha{5cm}
\begin{twocollist}\itemsep=0pt
\twocolitem{wxLIST\_STATE\_DONTCARE}{Don't care what the state is.}
\twocolitem{wxLIST\_STATE\_DROPHILITED}{The item indicates it is a drop target.}
\twocolitem{wxLIST\_STATE\_FOCUSED}{The item has the focus.}
\twocolitem{wxLIST\_STATE\_SELECTED}{The item is selected.}
\twocolitem{wxLIST\_STATE\_CUT}{The item is selected as part of a cut and paste operation.}
\end{twocollist}


\membersection{wxListCtrl::GetSelectedItemCount}\label{wxlistctrlgetselecteditemcount}

\constfunc{int}{GetSelectedItemCount}{\void}

Returns the number of selected items in the list control.


\membersection{wxListCtrl::GetTextColour}\label{wxlistctrlgettextcolour}

\constfunc{wxColour}{GetTextColour}{\void}

Gets the text colour of the list control.


\membersection{wxListCtrl::GetTopItem}\label{wxlistctrlgettopitem}

\constfunc{long}{GetTopItem}{\void}

Gets the index of the topmost visible item when in
list or report view.



\membersection{wxListCtrl::GetViewRect}\label{wxlistctrlgetviewrect}

\constfunc{wxRect}{GetViewRect}{\void}

Returns the rectangle taken by all items in the control. In other words, if the
controls client size were equal to the size of this rectangle, no scrollbars
would be needed and no free space would be left.

Note that this function only works in the icon and small icon views, not in
list or report views (this is a limitation of the native Win32 control).



\membersection{wxListCtrl::HitTest}\label{wxlistctrlhittest}

\func{long}{HitTest}{\param{const wxPoint\& }{point}, \param{int\& }{flags}}

Determines which item (if any) is at the specified point,
giving details in {\it flags}. Returns index of the item or {\tt wxNOT\_FOUND}
if no item is at the specified point.
{\it flags} will be a combination of the following flags:

\twocolwidtha{5cm}
\begin{twocollist}\itemsep=0pt
\twocolitem{wxLIST\_HITTEST\_ABOVE}{Above the client area.}
\twocolitem{wxLIST\_HITTEST\_BELOW}{Below the client area.}
\twocolitem{wxLIST\_HITTEST\_NOWHERE}{In the client area but below the last item.}
\twocolitem{wxLIST\_HITTEST\_ONITEMICON}{On the bitmap associated with an item.}
\twocolitem{wxLIST\_HITTEST\_ONITEMLABEL}{On the label (string) associated with an item.}
\twocolitem{wxLIST\_HITTEST\_ONITEMRIGHT}{In the area to the right of an item.}
\twocolitem{wxLIST\_HITTEST\_ONITEMSTATEICON}{On the state icon for a tree view item that is in a user-defined state.}
\twocolitem{wxLIST\_HITTEST\_TOLEFT}{To the right of the client area.}
\twocolitem{wxLIST\_HITTEST\_TORIGHT}{To the left of the client area.}
\twocolitem{wxLIST\_HITTEST\_ONITEM}{Combination of wxLIST\_HITTEST\_ONITEMICON, wxLIST\_HITTEST\_ONITEMLABEL,
wxLIST\_HITTEST\_ONITEMSTATEICON.}
\end{twocollist}

\pythonnote{A tuple of values is returned in the wxPython version of
this method. The first value is the item id and the second is the
flags value mentioned above.}

\perlnote{In wxPerl this method only takes the {\bf point} parameter
  and returns a 2-element list {\tt ( item, flags )}.}


\membersection{wxListCtrl::InsertColumn}\label{wxlistctrlinsertcolumn}

\func{long}{InsertColumn}{\param{long }{col}, \param{wxListItem\& }{info}}

\func{long}{InsertColumn}{\param{long }{col}, \param{const wxString\& }{heading}, \param{int }{format = wxLIST\_FORMAT\_LEFT},\rtfsp
\param{int }{width = -1}}

For report view mode (only), inserts a column. For more details, see \helpref{wxListCtrl::SetItem}{wxlistctrlsetitem}.

\pythonnote{In place of a single overloaded method name, wxPython
implements the following methods:\par
\indented{2cm}{\begin{twocollist}
\twocolitem{{\bf InsertColumn(col, heading, format=wxLIST\_FORMAT\_LEFT,
width=-1)}}{Creates a column using a header string only.}
\twocolitem{{\bf InsertColumnItem(col, item)}}{Creates a column using a
wxListItem.}
\end{twocollist}}
}


\membersection{wxListCtrl::InsertItem}\label{wxlistctrlinsertitem}

\func{long}{InsertItem}{\param{wxListItem\& }{info}}

Inserts an item, returning the index of the new item if successful,
-1 otherwise.

\func{long}{InsertItem}{\param{long }{index}, \param{const wxString\& }{label}}

Inserts a string item.

\func{long}{InsertItem}{\param{long }{index}, \param{int }{imageIndex}}

Inserts an image item.

\func{long}{InsertItem}{\param{long }{index}, \param{const wxString\& }{label}, \param{int }{imageIndex}}

Insert an image/string item.

\wxheading{Parameters}

\docparam{info}{wxListItem object}

\docparam{index}{Index of the new item, supplied by the application}

\docparam{label}{String label}

\docparam{imageIndex}{index into the image list associated with this control and view style}

\pythonnote{In place of a single overloaded method name, wxPython
implements the following methods:\par
\indented{2cm}{\begin{twocollist}\itemsep=0pt
\twocolitem{{\bf InsertItem(item)}}{Inserts an item using a wxListItem.}
\twocolitem{{\bf InsertStringItem(index, label)}}{Inserts a string item.}
\twocolitem{{\bf InsertImageItem(index, imageIndex)}}{Inserts an image item.}
\twocolitem{{\bf InsertImageStringItem(index, label, imageIndex)}}{Insert an image/string item.}
\end{twocollist}}
}

\perlnote{In wxPerl there are four methods instead of a single overloaded
method:\par
\indented{2cm}{\begin{twocollist}
\twocolitem{{\bf InsertItem( item )}}{Inserts a Wx::ListItem}
\twocolitem{{\bf InsertStringItem( index, label )}}{Inserts a string item}
\twocolitem{{\bf InsertImageItem( index, imageIndex )}}{Inserts an image item}
\twocolitem{{\bf InsertImageStringItem( index, label, imageIndex )}}{Inserts
 an item with a string and an image}
\end{twocollist}
}}


\membersection{wxListCtrl::OnGetItemAttr}\label{wxlistctrlongetitemattr}

\constfunc{virtual wxListItemAttr *}{OnGetItemAttr}{\param{long }{item}}

This function may be overloaded in the derived class for a control with
{\tt wxLC\_VIRTUAL} style. It should return the attribute for the
for the specified {\tt item} or {\tt NULL} to use the default appearance
parameters.

wxListCtrl will not delete the pointer or keep a reference of it. You can 
return the same wxListItemAttr pointer for every OnGetItemAttr call.

The base class version always returns {\tt NULL}.

\wxheading{See also}

\helpref{OnGetItemImage}{wxlistctrlongetitemimage},\\
\helpref{OnGetItemText}{wxlistctrlongetitemtext}


\membersection{wxListCtrl::OnGetItemImage}\label{wxlistctrlongetitemimage}

\func{virtual int}{OnGetItemImage}{\param{long }{item}}

This function must be overloaded in the derived class for a control with
{\tt wxLC\_VIRTUAL} style having an \helpref{image list}{wxlistctrlsetimagelist}
(if the control doesn't have an image list, it is not necessary to overload
 it). It should return the index of the items image in the controls image list
or $-1$ for no image.
In a control with {\tt wxLC\_REPORT} style, OnGetItemImage only gets called for
the first column of each line.

The base class version always returns $-1$.

\wxheading{See also}

\helpref{OnGetItemText}{wxlistctrlongetitemtext},\\
\helpref{OnGetItemAttr}{wxlistctrlongetitemattr}


\membersection{wxListCtrl::OnGetItemText}\label{wxlistctrlongetitemtext}

\constfunc{virtual wxString}{OnGetItemText}{\param{long }{item}, \param{long }{column}}

This function {\bf must} be overloaded in the derived class for a control with
{\tt wxLC\_VIRTUAL} style. It should return the string containing the text of
the given {\it column} for the specified {\tt item}.

\wxheading{See also}

\helpref{SetItemCount}{wxlistctrlsetitemcount},\\
\helpref{OnGetItemImage}{wxlistctrlongetitemimage},\\
\helpref{OnGetItemAttr}{wxlistctrlongetitemattr}


\membersection{wxListCtrl::RefreshItem}\label{wxlistctrlrefreshitem}

\func{void}{RefreshItem}{\param{long }{item}}

Redraws the given {\it item}. This is only useful for the virtual list controls
as without calling this function the displayed value of the item doesn't change
even when the underlying data does change.

\wxheading{See also}

\helpref{RefreshItems}{wxlistctrlrefreshitems}



\membersection{wxListCtrl::RefreshItems}\label{wxlistctrlrefreshitems}

\func{void}{RefreshItems}{\param{long }{itemFrom}, \param{long }{itemTo}}

Redraws the items between {\it itemFrom} and {\it itemTo}. The starting item
must be less than or equal to the ending one.

Just as \helpref{RefreshItem}{wxlistctrlrefreshitem} this is only useful for
virtual list controls.



\membersection{wxListCtrl::ScrollList}\label{wxlistctrlscrolllist}

\func{bool}{ScrollList}{\param{int }{dx}, \param{int }{dy}}

Scrolls the list control. If in icon, small icon or report view mode,
{\it dx} specifies the number of pixels to scroll. If in list view mode,
{\it dx} specifies the number of columns to scroll. {\it dy} always specifies
the number of pixels to scroll vertically.

{\bf NB:} This method is currently only implemented in the Windows version.


\membersection{wxListCtrl::SetBackgroundColour}\label{wxlistctrlsetbackgroundcolour}

\func{void}{SetBackgroundColour}{\param{const wxColour\& }{col}}

Sets the background colour (GetBackgroundColour already implicit in
wxWindow class).


\membersection{wxListCtrl::SetColumn}\label{wxlistctrlsetcolumn}

\func{bool}{SetColumn}{\param{int }{col}, \param{wxListItem\& }{item}}

Sets information about this column. See \helpref{wxListCtrl::SetItem}{wxlistctrlsetitem} for more
information.


\membersection{wxListCtrl::SetColumnWidth}\label{wxlistctrlsetcolumnwidth}

\func{bool}{SetColumnWidth}{\param{int }{col}, \param{int }{width}}

Sets the column width.

{\it width} can be a width in pixels or wxLIST\_AUTOSIZE (-1) or wxLIST\_AUTOSIZE\_USEHEADER (-2).
wxLIST\_AUTOSIZE will resize the column to the length of its longest item. wxLIST\_AUTOSIZE\_USEHEADER
will resize the column to the length of the header (Win32) or 80 pixels (other platforms).

In small or normal icon view, {\it col} must be -1, and the column width is set for all columns.


\membersection{wxListCtrl::SetImageList}\label{wxlistctrlsetimagelist}

\func{void}{SetImageList}{\param{wxImageList*}{ imageList}, \param{int }{which}}

Sets the image list associated with the control. {\it which} is one of
wxIMAGE\_LIST\_NORMAL, wxIMAGE\_LIST\_SMALL, wxIMAGE\_LIST\_STATE (the last is unimplemented).

This method does not take ownership of the image list, you have to
delete it yourself.

\wxheading{See also}

\helpref{wxListCtrl::AssignImageList}{wxlistctrlassignimagelist}



\membersection{wxListCtrl::SetItem}\label{wxlistctrlsetitem}

\func{bool}{SetItem}{\param{wxListItem\& }{info}}

\func{long}{SetItem}{\param{long }{index}, \param{int }{col}, \param{const }{wxString\& label}, \param{int }{imageId = -1}}

Sets information about the item.

wxListItem is a class with the following members:

\twocolwidtha{5cm}
\begin{twocollist}\itemsep=0pt
\twocolitem{long m\_mask}{Indicates which fields are valid. See the list of valid mask flags below.}
\twocolitem{long m\_itemId}{The zero-based item position.}
\twocolitem{int m\_col}{Zero-based column, if in report mode.}
\twocolitem{long m\_state}{The state of the item. See the list of valid state flags below.}
\twocolitem{long m\_stateMask}{A mask indicating which state flags are valid. See the list of valid state flags below.}
\twocolitem{wxString m\_text}{The label/header text.}
\twocolitem{int m\_image}{The zero-based index into an image list.}
\twocolitem{long m\_data}{Application-defined data.}
\twocolitem{int m\_format}{For columns only: the format. Can be wxLIST\_FORMAT\_LEFT, wxLIST\_FORMAT\_RIGHT or
wxLIST\_FORMAT\_CENTRE.}
\twocolitem{int m\_width}{For columns only: the column width.}
\end{twocollist}

The {\bf m\_mask} member contains a bitlist specifying which of the other fields are valid. The flags are:

\twocolwidtha{5cm}
\begin{twocollist}\itemsep=0pt
\twocolitem{wxLIST\_MASK\_STATE}{The {\bf m\_state} field is valid.}
\twocolitem{wxLIST\_MASK\_TEXT}{The {\bf m\_text} field is valid.}
\twocolitem{wxLIST\_MASK\_IMAGE}{The {\bf m\_image} field is valid.}
\twocolitem{wxLIST\_MASK\_DATA}{The {\bf m\_data} field is valid.}
\twocolitem{wxLIST\_MASK\_WIDTH}{The {\bf m\_width} field is valid.}
\twocolitem{wxLIST\_MASK\_FORMAT}{The {\bf m\_format} field is valid.}
\end{twocollist}

The {\bf m\_stateMask} and {\bf m\_state} members take flags from the following:

\twocolwidtha{5cm}
\begin{twocollist}\itemsep=0pt
\twocolitem{wxLIST\_STATE\_DONTCARE}{Don't care what the state is. Win32 only. }
\twocolitem{wxLIST\_STATE\_DROPHILITED}{The item is highlighted to receive a drop event. Win32 only. }
\twocolitem{wxLIST\_STATE\_FOCUSED}{The item has the focus.}
\twocolitem{wxLIST\_STATE\_SELECTED}{The item is selected.}
\twocolitem{wxLIST\_STATE\_CUT}{The item is in the cut state. Win32 only. }
\end{twocollist}

The wxListItem object can also contain item-specific colour and font
information: for this you need to call one of SetTextColour(),
SetBackgroundColour() or SetFont() functions on it passing it the colour/font
to use. If the colour/font is not specified, the default list control
colour/font is used.

\func{long}{SetItem}{\param{long }{index}, \param{int }{col}, \param{const wxString\& }{label}, \param{int }{imageId = -1}}

Sets a string field at a particular column.

\pythonnote{In place of a single overloaded method name, wxPython
implements the following methods:\par
\indented{2cm}{\begin{twocollist}
\twocolitem{{\bf SetItem(item)}}{Sets information about the given wxListItem.}
\twocolitem{{\bf SetStringItem(index, col, label, imageId)}}{Sets a
string or image at a given location.}
\end{twocollist}}
}

\membersection{wxListCtrl::SetItemBackgroundColour}\label{wxlistctrlsetitembackgroundcolour}

\func{void}{SetItemBackgroundColour}{\param{long }{item}, \param{const wxColour\& }{col}}

Sets the background colour for this item. This function only works in report view.

The colour can be retrieved using
\helpref{GetItemBackgroundColour}{wxlistctrlgetitembackgroundcolour}.



\membersection{wxListCtrl::SetItemCount}\label{wxlistctrlsetitemcount}

\func{void}{SetItemCount}{\param{long }{count}}

This method can only be used with virtual list controls. It is used to indicate
to the control the number of items it contains. After calling it, the main
program should be ready to handle calls to various item callbacks (such as
\helpref{OnGetItemText}{wxlistctrlongetitemtext}) for all items in the range
from $0$ to {\it count}.


\membersection{wxListCtrl::SetItemData}\label{wxlistctrlsetitemdata}

\func{bool}{SetItemData}{\param{long }{item}, \param{long }{data}}

Associates application-defined data with this item.


\membersection{wxListCtrl::SetItemFont}\label{wxlistctrlsetitemfont}

\func{void}{SetItemFont}{\param{long }{item}, \param{const wxFont\& }{font}}

Sets the item's font.


\membersection{wxListCtrl::SetItemImage}\label{wxlistctrlsetitemimage}

\func{bool}{SetItemImage}{\param{long }{item}, \param{int }{image}}

Sets the image associated with the item. The image is an index into the
image list associated with the list control. In report view, this only sets
the image for the first column.

\func{bool}{SetItemImage}{\param{long }{item}, \param{int }{image}, \param{int }{selImage}}

Sets the unselected and selected images associated with the item. The images are indices into the
image list associated with the list control. This form is deprecated: {\it selImage} is not
used.


\membersection{wxListCtrl::SetItemColumnImage}\label{wxlistctrlsetitemcolumnimage}

\func{bool}{SetItemImage}{\param{long }{item}, \param{long }{column}\param{int }{image}}

Sets the image associated with the item. In report view, you can specify the column.
The image is an index into the image list associated with the list control.


\membersection{wxListCtrl::SetItemPosition}\label{wxlistctrlsetitemposition}

\func{bool}{SetItemPosition}{\param{long }{item}, \param{const wxPoint\& }{pos}}

Sets the position of the item, in icon or small icon view. Windows only.


\membersection{wxListCtrl::SetItemState}\label{wxlistctrlsetitemstate}

\func{bool}{SetItemState}{\param{long }{item}, \param{long }{state}, \param{long }{stateMask}}

Sets the item state. For a list of state flags, see \helpref{wxListCtrl::SetItem}{wxlistctrlsetitem}.

The {\bf stateMask} indicates which state flags are valid.


\membersection{wxListCtrl::SetItemText}\label{wxlistctrlsetitemtext}

\func{void}{SetItemText}{\param{long }{item}, \param{const wxString\& }{text}}

Sets the item text for this item.


\membersection{wxListCtrl::SetItemTextColour}\label{wxlistctrlsetitemtextcolour}

\func{void}{SetItemTextColour}{\param{long }{item}, \param{const wxColour\& }{col}}

Sets the colour for this item. This function only works in report view.

The colour can be retrieved using
\helpref{GetItemTextColour}{wxlistctrlgetitemtextcolour}.


\membersection{wxListCtrl::SetSingleStyle}\label{wxlistctrlsetsinglestyle}

\func{void}{SetSingleStyle}{\param{long }{style}, \param{const bool }{add = true}}

Adds or removes a single window style.


\membersection{wxListCtrl::SetTextColour}\label{wxlistctrlsettextcolour}

\func{void}{SetTextColour}{\param{const wxColour\& }{col}}

Sets the text colour of the list control.


\membersection{wxListCtrl::SetWindowStyleFlag}\label{wxlistctrlsetwindowstyleflag}

\func{void}{SetWindowStyleFlag}{\param{long }{style}}

Sets the whole window style, deleting all items.

\membersection{wxListCtrl::SortItems}\label{wxlistctrlsortitems}

\func{bool}{SortItems}{\param{wxListCtrlCompare }{fnSortCallBack}, \param{long }{data}}

Call this function to sort the items in the list control. Sorting is done
using the specified {\it fnSortCallBack} function. This function must have the
following prototype:

\begin{verbatim}
int wxCALLBACK wxListCompareFunction(long item1, long item2, long sortData)
\end{verbatim}

It is called each time when the two items must be compared and should return 0
if the items are equal, negative value if the first item is less than the
second one and positive value if the first one is greater than the second one
(the same convention as used by {\tt qsort(3)}).

\wxheading{Parameters}

\docparam{item1}{client data associated with the first item ({\bf NOT} the index).}

\docparam{item2}{client data associated with the second item ({\bf NOT} the index).}

\docparam{data}{the value passed to SortItems() itself.}

Notice that the control may only be sorted on client data associated with the
items, so you {\bf must} use \helpref{SetItemData}{wxlistctrlsetitemdata} if
you want to be able to sort the items in the control.

Please see the \helpref{listctrl sample}{samplelistctrl} for an example of
using this function.

\pythonnote{wxPython uses the sortData parameter to pass the Python
function to call, so it is not available for programmer use.  Call
SortItems with a reference to a callable object that expects two
parameters.}

\perlnote{In wxPerl the comparison function must take just two parameters;
however, you may use a closure to achieve an effect similar to the
SortItems third parameter.}

