\section{\class{wxRect}}\label{wxrect}

A class for manipulating rectangles.

\wxheading{Derived from}

None

\wxheading{Include files}

<wx/gdicmn.h>

\wxheading{See also}

\helpref{wxPoint}{wxpoint}, \helpref{wxSize}{wxsize}

\latexignore{\rtfignore{\wxheading{Members}}}


\membersection{wxRect::wxRect}\label{wxrectctor}

\func{}{wxRect}{\void}

Default constructor.

\func{}{wxRect}{\param{int}{ x}, \param{int}{ y}, \param{int}{ width}, \param{int}{ height}}

Creates a wxRect object from x, y, width and height values.

\func{}{wxRect}{\param{const wxPoint\&}{ topLeft}, \param{const wxPoint\&}{ bottomRight}}

Creates a wxRect object from top-left and bottom-right points.

\func{}{wxRect}{\param{const wxPoint\&}{ pos}, \param{const wxSize\&}{ size}}

Creates a wxRect object from position and size values.

\func{}{wxRect}{\param{const wxSize\&}{ size}}

Creates a wxRect object from size values at the origin.


\membersection{wxRect::x}\label{wxrectx}

\member{int}{x}

x coordinate of the top-level corner of the rectangle.


\membersection{wxRect::y}\label{wxrecty}

\member{int}{y}

y coordinate of the top-level corner of the rectangle.


\membersection{wxRect::width}\label{wxrectwidth}

\member{int}{width}

Width member.


\membersection{wxRect::height}\label{wxrectheight}

\member{int}{height}

Height member.


\membersection{wxRect::Deflate}\label{wxrectdeflate}

\func{void}{Deflate}{\param{wxCoord }{dx}, \param{wxCoord }{dy}}

\func{void}{Deflate}{\param{wxCoord }{diff}}

\constfunc{wxRect}{Deflate}{\param{wxCoord }{dx}, \param{wxCoord }{dy}}

Decrease the rectangle size.

This method is the opposite from \helpref{Inflate}{wxrectinflate}:
Deflate(a, b) is equivalent to Inflate(-a, -b).
Please refer to \helpref{Inflate}{wxrectinflate} for full description.

\wxheading{See also}

\helpref{Inflate}{wxrectinflate}


\membersection{wxRect::GetBottom}\label{wxrectgetbottom}

\constfunc{int}{GetBottom}{\void}

Gets the bottom point of the rectangle.


\membersection{wxRect::GetHeight}\label{wxrectgetheight}

\constfunc{int}{GetHeight}{\void}

Gets the height member.


\membersection{wxRect::GetLeft}\label{wxrectgetleft}

\constfunc{int}{GetLeft}{\void}

Gets the left point of the rectangle (the same as \helpref{wxRect::GetX}{wxrectgetx}).


\membersection{wxRect::GetPosition}\label{wxrectgetposition}

\constfunc{wxPoint}{GetPosition}{\void}

Gets the position.


\membersection{wxRect::GetTopLeft}\label{wxrectgettopleft}

\constfunc{wxPoint}{GetTopLeft}{\void}

Gets the topleft position of the rectangle. (Same as GetPosition).


\membersection{wxRect::GetBottomRight}\label{wxrectgetbottomright}

\constfunc{wxPoint}{GetBottomRight}{\void}

Gets the bottom right position. Returns the bottom right point inside the rectangle.


\membersection{wxRect::GetRight}\label{wxrectgetright}

\constfunc{int}{GetRight}{\void}

Gets the right point of the rectangle.


\membersection{wxRect::GetSize}\label{wxrectgetsize}

\constfunc{wxSize}{GetSize}{\void}

Gets the size.

\wxheading{See also}

\helpref{wxRect::SetSize}{wxrectsetsize}


\membersection{wxRect::GetTop}\label{wxrectgettop}

\constfunc{int}{GetTop}{\void}

Gets the top point of the rectangle (the same as \helpref{wxRect::GetY}{wxrectgety}).


\membersection{wxRect::GetWidth}\label{wxrectgetwidth}

\constfunc{int}{GetWidth}{\void}

Gets the width member.


\membersection{wxRect::GetX}\label{wxrectgetx}

\constfunc{int}{GetX}{\void}

Gets the x member.


\membersection{wxRect::GetY}\label{wxrectgety}

\constfunc{int}{GetY}{\void}

Gets the y member.


\membersection{wxRect::Inflate}\label{wxrectinflate}

\func{void}{Inflate}{\param{wxCoord }{dx}, \param{wxCoord }{dy}}

\func{void}{Inflate}{\param{wxCoord }{diff}}

\constfunc{wxRect}{Inflate}{\param{wxCoord }{dx}, \param{wxCoord }{dy}}

Increases the size of the rectangle.

The second form uses the same {\it diff} for both {\it dx} and {\it dy}.

The first two versions modify the rectangle in place, the last one returns a
new rectangle leaving this one unchanged.

The left border is moved farther left and the right border is moved farther
right by {\it dx}. The upper border is moved farther up and the bottom border
is moved farther down by {\it dy}. (Note the the width and height of the
rectangle thus change by 2*{\it dx} and 2*{\it dy}, respectively.) If one or
both of {\it dx} and {\it dy} are negative, the opposite happens: the rectangle
size decreases in the respective direction.

Inflating and deflating behaves ``naturally''. Defined more precisely, that
means:
\begin{enumerate}
    \item ``Real'' inflates (that is, {\it dx} and/or {\it dy} >= 0) are not
        constrained. Thus inflating a rectangle can cause its upper left corner
        to move into the negative numbers. (the versions prior to 2.5.4 forced
        the top left coordinate to not fall below (0, 0), which implied a
        forced move of the rectangle.)

    \item Deflates are clamped to not reduce the width or height of the
        rectangle below zero. In such cases, the top-left corner is nonetheless
        handled properly. For example, a rectangle at (10, 10) with size (20,
        40) that is inflated by (-15, -15) will become located at (20, 25) at
        size (0, 10). Finally, observe that the width and height are treated
        independently. In the above example, the width is reduced by 20,
        whereas the height is reduced by the full 30 (rather than also stopping
        at 20, when the width reached zero).
\end{enumerate}

\wxheading{See also}

\helpref{Deflate}{wxrectdeflate}


\membersection{wxRect::Inside}\label{wxrectinside}

\constfunc{bool}{Inside}{\param{int }{x}, \param{int }{y}}

\constfunc{bool}{Inside}{\param{const wxPoint\& }{pt}}

Returns {\tt true} if the given point is inside the rectangle (or on its
boundary) and {\tt false} otherwise.


\membersection{wxRect::Intersects}\label{wxrectintersects}

\constfunc{bool}{Intersects}{\param{const wxRect\& }{rect}}

Returns {\tt true} if this rectangle has a non empty intersection with the
rectangle {\it rect} and {\tt false} otherwise.


\membersection{wxRect::IsEmpty}\label{wxrectisempty}

\constfunc{bool}{IsEmpty}{}

Returns {\tt true} if this rectangle has a width or height less than or equal to 
0 and {\tt false} otherwise.


\membersection{wxRect::Offset}\label{wxrectoffset}

\func{void}{Offset}{\param{wxCoord }{dx}, \param{wxCoord }{dy}}

\func{void}{Offset}{\param{const wxPoint\& }{pt}}

Moves the rectangle by the specified offset. If {\it dx} is positive, the
rectangle is moved to the right, if {\it dy} is positive, it is moved to the
bottom, otherwise it is moved to the left or top respectively.


\membersection{wxRect::SetHeight}\label{wxrectsetheight}

\func{void}{SetHeight}{\param{int}{ height}}

Sets the height.


\membersection{wxRect::SetSize}\label{wxrectsetsize}

\func{void}{SetSize}{\param{const wxSize\&}{ s}}

Sets the size.

\wxheading{See also}

\helpref{wxRect::GetSize}{wxrectgetsize}


\membersection{wxRect::SetWidth}\label{wxrectsetwidth}

\func{void}{SetWidth}{\param{int}{ width}}

Sets the width.


\membersection{wxRect::SetX}\label{wxrectsetx}

\func{void}{SetX}{\param{int}{ x}}

Sets the x position.


\membersection{wxRect::SetY}\label{wxrectsety}

\func{void}{SetY}{\param{int}{ y}}

Sets the y position.


\membersection{wxRect::Union}\label{wxrectunion}

\constfunc{wxRect}{Union}{\param{const wxRect\&}{ rect}}

\func{wxRect\&}{Union}{\param{const wxRect\&}{ rect}}

Modifies the rectangle to contain the bounding box of this rectangle and the
one passed in as parameter. The const version returns the new rectangle, the
other one modifies this rectangle in place.


\membersection{wxRect::operator $=$}\label{wxrectassign}

\func{void}{operator $=$}{\param{const wxRect\& }{rect}}

Assignment operator.


\membersection{wxRect::operator $==$}\label{wxrectequal}

\func{bool}{operator $==$}{\param{const wxRect\& }{rect}}

Equality operator.


\membersection{wxRect::operator $!=$}\label{wxrectnotequal}

\func{bool}{operator $!=$}{\param{const wxRect\& }{rect}}

Inequality operator.

