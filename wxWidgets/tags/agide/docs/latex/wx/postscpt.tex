\section{\class{wxPostScriptDC}}\label{wxpostscriptdc}

This defines the wxWidgets Encapsulated PostScript device context,
which can write PostScript files on any platform. See \helpref{wxDC}{wxdc} for
descriptions of the member functions.

\wxheading{Derived from}

\helpref{wxDC}{wxdc}\\
\helpref{wxObject}{wxobject}

\wxheading{Include files}

<wx/dcps.h>

\latexignore{\rtfignore{\wxheading{Members}}}

\membersection{wxPostScriptDC::wxPostScriptDC}\label{wxpostscriptdcctor}

\func{}{wxPostScriptDC}{\param{const wxPrintData\&}{ printData}}

Constructs a PostScript printer device context from a \helpref{wxPrintData}{wxprintdata} object.

\func{}{wxPostScriptDC}{\param{const wxString\& }{output}, \param{bool }{interactive = true},\\
  \param{wxWindow *}{parent}}

Constructor. {\it output} is an optional file for printing to, and if
\rtfsp{\it interactive} is true a dialog box will be displayed for adjusting
various parameters. {\it parent} is the parent of the printer dialog box.

Use the {\it Ok} member to test whether the constructor was successful
in creating a usable device context.

See \helpref{Printer settings}{printersettings} for functions to set and
get PostScript printing settings.

This constructor and the global printer settings are now deprecated;
use the wxPrintData constructor instead.


\membersection{wxPostScriptDC::SetResolution}\label{wxpostscriptdcsetresolution}

\func{static void}{SetResolution}{\param{int}{ ppi}}

Set resolution (in pixels per inch) that will be used in PostScript
output. Default is 720ppi.

\membersection{wxPostScriptDC::GetResolution}\label{wxpostscriptdcgetresolution}

\func{static int}{GetResolution}{\void}

Return resolution used in PostScript output. See 
\helpref{SetResolution}{wxpostscriptdcsetresolution}.

