\section{Bitmaps and icons overview}\label{wxbitmapoverview}

Classes: \helpref{wxBitmap}{wxbitmap}, \helpref{wxBitmapHandler}{wxbitmaphandler}, \helpref{wxIcon}{wxicon}, \helpref{wxCursor}{wxcursor}.

The wxBitmap class encapsulates the concept of a platform-dependent bitmap,
either monochrome or colour. Platform-specific methods for creating a
wxBitmap object from an existing file are catered for, and
this is an occasion where conditional compilation will sometimes be
required.

A bitmap created dynamically or loaded from a file can be selected
into a memory device context (instance of \helpref{wxMemoryDC}{wxmemorydc}). This
enables the bitmap to be copied to a window or memory device context
using \helpref{wxDC::Blit}{wxdcblit}, or to be used as a drawing surface.  The {\bf
wxToolBarSimple} class is implemented using bitmaps, and the toolbar demo
shows one of the toolbar bitmaps being used for drawing a miniature
version of the graphic which appears on the main window.

See \helpref{wxMemoryDC}{wxmemorydc} for an example of drawing onto a bitmap.

All wxWindows platforms support XPMs for small bitmaps and icons.
You may include the XPM inline as below, since it's C code, or you
can load it at run-time.

\begin{verbatim}
#include "mondrian.xpm"
\end{verbatim}

Sometimes you wish to use a .ico resource on Windows, and XPMs on
other platforms (for example to take advantage of Windows' support for multiple icon resolutions).
A macro, \helpref{wxICON}{wxiconmacro}, is available which creates an icon using an XPM
on the appropriate platform, or an icon resource on Windows.

\begin{verbatim}
wxIcon icon(wxICON(mondrian));

// Equivalent to:

#if defined(__WXGTK__) || defined(__WXMOTIF__)
wxIcon icon(mondrian_xpm);
#endif

#if defined(__WXMSW__)
wxIcon icon("mondrian");
#endif
\end{verbatim}

There is also a corresponding \helpref{wxBITMAP}{wxbitmapmacro} macro which allows
to create the bitmaps in much the same way as \helpref{wxICON}{wxiconmacro} creates
icons. It assumes that bitmaps live in resources under Windows or OS2 and XPM
files under all other platforms (for XPMs, the corresponding file must be
included before this macro is used, of course, and the name of the bitmap
should be the same as the resource name under Windows with {\tt \_xpm}
suffix). For example:

\begin{verbatim}
// an easy and portable way to create a bitmap
wxBitmap bmp(wxBITMAP(bmpname));

// which is roughly equivalent to the following
#if defined(__WXMSW__) || defined(__WXPM__)
    wxBitmap bmp("bmpname", wxBITMAP_TYPE_RESOURCE);
#else // Unix
    wxBitmap bmp(bmpname_xpm, wxBITMAP_TYPE_XPM);
#endif
\end{verbatim}

You should always use wxICON and wxBITMAP macros because they work for any
platform (unlike the code above which doesn't deal with wxMac, wxX11, ...) and
are more short and clear than versions with {\tt \#ifdef}s. Even better,
use the same XPMs on all platforms.

\subsection{Supported bitmap file formats}\label{supportedbitmapformats}

The following lists the formats handled on different platforms. Note
that missing or partially-implemented formats are automatically supplemented
by the \helpref{wxImage}{wximage} to load the data, and then converting
it to wxBitmap form. Note that using wxImage is the preferred way to
load images in wxWindows, with the exception of resources (XPM-files or
native Windows resources). Writing an image format handler for wxImage
is also far easier than writing one for wxBitmap, because wxImage has
exactly one format on all platforms whereas wxBitmap can store pixel data
very differently, depending on colour depths and platform.

\wxheading{wxBitmap}

Under Windows, wxBitmap may load the following formats:

\begin{itemize}\itemsep=0pt
\item Windows bitmap resource (wxBITMAP\_TYPE\_BMP\_RESOURCE)
\item Windows bitmap file (wxBITMAP\_TYPE\_BMP)
\item XPM data and file (wxBITMAP\_TYPE\_XPM)
\item All formats that are supported by the \helpref{wxImage}{wximage} class.
\end{itemize}

Under wxGTK, wxBitmap may load the following formats:

\begin{itemize}\itemsep=0pt
\item XPM data and file (wxBITMAP\_TYPE\_XPM)
\item All formats that are supported by the \helpref{wxImage}{wximage} class.
\end{itemize}

Under wxMotif and wxX11, wxBitmap may load the following formats:

\begin{itemize}\itemsep=0pt
\item XBM data and file (wxBITMAP\_TYPE\_XBM)
\item XPM data and file (wxBITMAP\_TYPE\_XPM)
\item All formats that are supported by the \helpref{wxImage}{wximage} class.
\end{itemize}

\wxheading{wxIcon}

Under Windows, wxIcon may load the following formats:

\begin{itemize}\itemsep=0pt
\item Windows icon resource (wxBITMAP\_TYPE\_ICO\_RESOURCE)
\item Windows icon file (wxBITMAP\_TYPE\_ICO)
\item XPM data and file (wxBITMAP\_TYPE\_XPM)
\end{itemize}

Under wxGTK, wxIcon may load the following formats:

\begin{itemize}\itemsep=0pt
\item XPM data and file (wxBITMAP\_TYPE\_XPM)
\item All formats that are supported by the \helpref{wxImage}{wximage} class.
\end{itemize}

Under wxMotif and wxX11, wxIcon may load the following formats:

\begin{itemize}\itemsep=0pt
\item XBM data and file (wxBITMAP\_TYPE\_XBM)
\item XPM data and file (wxBITMAP\_TYPE\_XPM)
\item All formats that are supported by the \helpref{wxImage}{wximage} class.
\end{itemize}

\wxheading{wxCursor}

Under Windows, wxCursor may load the following formats:

\begin{itemize}\itemsep=0pt
\item Windows cursor resource (wxBITMAP\_TYPE\_CUR\_RESOURCE)
\item Windows cursor file (wxBITMAP\_TYPE\_CUR)
\item Windows icon file (wxBITMAP\_TYPE\_ICO)
\item Windows bitmap file (wxBITMAP\_TYPE\_BMP)
\end{itemize}

Under wxGTK, wxCursor may load the following formats (in additional
to stock cursors):

\begin{itemize}\itemsep=0pt
\item None (stock cursors only).
\end{itemize}

Under wxMotif and wxX11, wxCursor may load the following formats:

\begin{itemize}\itemsep=0pt
\item XBM data and file (wxBITMAP\_TYPE\_XBM)
\end{itemize}

\subsection{Bitmap format handlers}\label{bitmaphandlers}

To provide extensibility, the functionality for loading and saving bitmap formats
is not implemented in the wxBitmap class, but in a number of handler classes,
derived from wxBitmapHandler. There is a static list of handlers which wxBitmap
examines when a file load/save operation is requested. Some handlers are provided as standard, but if you
have special requirements, you may wish to initialise the wxBitmap class with
some extra handlers which you write yourself or receive from a third party.

To add a handler object to wxBitmap, your application needs to include the header which implements it, and
then call the static function \helpref{wxBitmap::AddHandler}{wxbitmapaddhandler}.

{\bf Note:} bitmap handlers are not implemented on all platforms, and new ones rarely need
to be implemented since wxImage can be used for loading most formats, as noted earlier.

