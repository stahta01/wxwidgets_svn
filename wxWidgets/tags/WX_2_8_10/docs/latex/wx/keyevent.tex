\section{\class{wxKeyEvent}}\label{wxkeyevent}

This event class contains information about keypress (character) events.

Notice that there are three different kinds of keyboard events in wxWidgets:
key down and up events and char events. The difference between the first two
is clear - the first corresponds to a key press and the second to a key
release - otherwise they are identical. Just note that if the key is
maintained in a pressed state you will typically get a lot of (automatically
generated) down events but only one up so it is wrong to assume that there is
one up event corresponding to each down one.

Both key events provide untranslated key codes while the char event carries
the translated one. The untranslated code for alphanumeric keys is always
an upper case value. For the other keys it is one of {\tt WXK\_XXX} values
from the \helpref{keycodes table}{keycodes}. The translated key is, in
general, the character the user expects to appear as the result of the key
combination when typing the text into a text entry zone, for example.

A few examples to clarify this (all assume that {\sc Caps Lock} is unpressed
and the standard US keyboard): when the {\tt 'A'} key is pressed, the key down
event key code is equal to {\tt ASCII A} $== 65$. But the char event key code
is {\tt ASCII a} $== 97$. On the other hand, if you press both {\sc Shift} and
{\tt 'A'} keys simultaneously , the key code in key down event will still be
just {\tt 'A'} while the char event key code parameter will now be {\tt 'A'}
as well.

Although in this simple case it is clear that the correct key code could be
found in the key down event handler by checking the value returned by
\helpref{ShiftDown()}{wxkeyeventshiftdown}, in general you should use
{\tt EVT\_CHAR} for this as for non-alphanumeric keys the translation is
keyboard-layout dependent and can only be done properly by the system itself.

Another kind of translation is done when the control key is pressed: for
example, for {\sc Ctrl-A} key press the key down event still carries the
same key code {\tt 'a'} as usual but the char event will have key code of
$1$, the ASCII value of this key combination.

You may discover how the other keys on your system behave interactively by
running the \helpref{text}{sampletext} wxWidgets sample and pressing some keys
in any of the text controls shown in it.

{\bf Note:} If a key down ({\tt EVT\_KEY\_DOWN}) event is caught and
the event handler does not call {\tt event.Skip()} then the corresponding
char event ({\tt EVT\_CHAR}) will not happen.  This is by design and
enables the programs that handle both types of events to be a bit
simpler.

{\bf Note for Windows programmers:} The key and char events in wxWidgets are
similar to but slightly different from Windows {\tt WM\_KEYDOWN} and
{\tt WM\_CHAR} events. In particular, Alt-x combination will generate a char
event in wxWidgets (unless it is used as an accelerator).

{\bf Tip:} be sure to call {\tt event.Skip()} for events that you don't process in
key event function, otherwise menu shortcuts may cease to work under Windows.

\wxheading{Derived from}

\helpref{wxEvent}{wxevent}

\wxheading{Include files}

<wx/event.h>

\wxheading{Event table macros}

To process a key event, use these event handler macros to direct input to member
functions that take a wxKeyEvent argument.

\twocolwidtha{7cm}
\begin{twocollist}\itemsep=0pt
\twocolitem{{\bf EVT\_KEY\_DOWN(func)}}{Process a wxEVT\_KEY\_DOWN event (any key has been pressed).}
\twocolitem{{\bf EVT\_KEY\_UP(func)}}{Process a wxEVT\_KEY\_UP event (any key has been released).}
\twocolitem{{\bf EVT\_CHAR(func)}}{Process a wxEVT\_CHAR event.}
%\twocolitem{{\bf EVT\_CHAR\_HOOK(func)}}{Process a wxEVT\_CHAR\_HOOK event.}
\end{twocollist}%


\latexignore{\rtfignore{\wxheading{Members}}}


\membersection{wxKeyEvent::m\_altDown}\label{wxkeyeventmaltdown}

\member{bool}{m\_altDown}

\textbf{Deprecated: } Please use \helpref{GetModifiers}{wxkeyeventgetmodifiers}
instead!

true if the Alt key is pressed down.


\membersection{wxKeyEvent::m\_controlDown}\label{wxkeyeventmcontroldown}

\member{bool}{m\_controlDown}

\textbf{Deprecated: } Please use \helpref{GetModifiers}{wxkeyeventgetmodifiers}
instead!

true if control is pressed down.


\membersection{wxKeyEvent::m\_keyCode}\label{wxkeyeventmkeycode}

\member{long}{m\_keyCode}

\textbf{Deprecated: } Please use \helpref{GetKeyCode}{wxkeyeventgetkeycode}
instead!

Virtual keycode. See \helpref{Keycodes}{keycodes} for a list of identifiers.


\membersection{wxKeyEvent::m\_metaDown}\label{wxkeyeventmmetadown}

\member{bool}{m\_metaDown}

\textbf{Deprecated: } Please use \helpref{GetModifiers}{wxkeyeventgetmodifiers}
instead!

true if the Meta key is pressed down.


\membersection{wxKeyEvent::m\_shiftDown}\label{wxkeyeventmshiftdown}

\member{bool}{m\_shiftDown}

\textbf{Deprecated: } Please use \helpref{GetModifiers}{wxkeyeventgetmodifiers}
instead!

true if shift is pressed down.


\membersection{wxKeyEvent::m\_x}\label{wxkeyeventmx}

\member{int}{m\_x}

\textbf{Deprecated: } Please use \helpref{GetX}{wxkeyeventgetx} instead!

X position of the event.


\membersection{wxKeyEvent::m\_y}\label{wxkeyeventmy}

\member{int}{m\_y}

\textbf{Deprecated: } Please use \helpref{GetY}{wxkeyeventgety} instead!

Y position of the event.


\membersection{wxKeyEvent::wxKeyEvent}\label{wxkeyeventctor}

\func{}{wxKeyEvent}{\param{WXTYPE}{ keyEventType}}

Constructor. Currently, the only valid event types are wxEVT\_CHAR and wxEVT\_CHAR\_HOOK.


\membersection{wxKeyEvent::AltDown}\label{wxkeyeventaltdown}

\constfunc{bool}{AltDown}{\void}

Returns true if the Alt key was down at the time of the key event.

Notice that \helpref{GetModifiers}{wxkeyeventgetmodifiers} is easier to use
correctly than this function so you should consider using it in new code.


\membersection{wxKeyEvent::CmdDown}\label{wxkeyeventcmddown}

\constfunc{bool}{CmdDown}{\void}

\textsc{Cmd} is a pseudo key which is the same as Control for PC and Unix
platforms but the special \textsc{Apple} (a.k.a as \textsc{Command}) key under
Macs: it makes often sense to use it instead of, say, ControlDown() because Cmd
key is used for the same thing under Mac as Ctrl elsewhere (but Ctrl still
exists, just not used for this purpose under Mac). So for non-Mac platforms
this is the same as \helpref{ControlDown()}{wxkeyeventcontroldown} and under
Mac this is the same as \helpref{MetaDown()}{wxkeyeventmetadown}.


\membersection{wxKeyEvent::ControlDown}\label{wxkeyeventcontroldown}

\constfunc{bool}{ControlDown}{\void}

Returns true if the control key was down at the time of the key event.

Notice that \helpref{GetModifiers}{wxkeyeventgetmodifiers} is easier to use
correctly than this function so you should consider using it in new code.


\membersection{wxKeyEvent::GetKeyCode}\label{wxkeyeventgetkeycode}

\constfunc{int}{GetKeyCode}{\void}

Returns the virtual key code. ASCII events return normal ASCII values,
while non-ASCII events return values such as {\bf WXK\_LEFT} for the
left cursor key. See \helpref{Keycodes}{keycodes} for a full list of
the virtual key codes.

Note that in Unicode build, the returned value is meaningful only if the
user entered a character that can be represented in current locale's default
charset. You can obtain the corresponding Unicode character using
\helpref{GetUnicodeKey}{wxkeyeventgetunicodekey}.


\membersection{wxKeyEvent::GetModifiers}\label{wxkeyeventgetmodifiers}

\constfunc{int}{GetModifiers}{\void}

Return the bitmask of modifier keys which were pressed when this event
happened. See \helpref{key modifier constants}{keymodifiers} for the full list
of modifiers.

Notice that this function is easier to use correctly than, for example, 
\helpref{ControlDown}{wxkeyeventcontroldown} because when using the latter you
also have to remember to test that none of the other modifiers is pressed:

\begin{verbatim}
    if ( ControlDown() && !AltDown() && !ShiftDown() && !MetaDown() )
        ... handle Ctrl-XXX ...
\end{verbatim}

and forgetting to do it can result in serious program bugs (e.g. program not
working with European keyboard layout where \textsc{AltGr} key which is seen by
the program as combination of \textsc{Ctrl} and \textsc{Alt} is used). On the
other hand, you can simply write

\begin{verbatim}
    if ( GetModifiers() == wxMOD_CONTROL )
        ... handle Ctrl-XXX ...
\end{verbatim}

with this function.


\membersection{wxKeyEvent::GetPosition}\label{wxkeyeventgetposition}

\constfunc{wxPoint}{GetPosition}{\void}

\constfunc{void}{GetPosition}{\param{long *}{x}, \param{long *}{y}}

Obtains the position (in client coordinates) at which the key was pressed.


\membersection{wxKeyEvent::GetRawKeyCode}\label{wxkeyeventgetrawkeycode}

\constfunc{wxUint32}{GetRawKeyCode}{\void}

Returns the raw key code for this event. This is a platform-dependent scan code
which should only be used in advanced applications.

{\bf NB:} Currently the raw key codes are not supported by all ports, use
{\tt\#ifdef wxHAS\_RAW\_KEY\_CODES} to determine if this feature is available.


\membersection{wxKeyEvent::GetRawKeyFlags}\label{wxkeyeventgetrawkeyflags}

\constfunc{wxUint32}{GetRawKeyFlags}{\void}

Returns the low level key flags for this event. The flags are
platform-dependent and should only be used in advanced applications.

{\bf NB:} Currently the raw key flags are not supported by all ports, use
{\tt \#ifdef wxHAS\_RAW\_KEY\_CODES} to determine if this feature is available.


\membersection{wxKeyEvent::GetUnicodeKey}\label{wxkeyeventgetunicodekey}

\constfunc{wxChar}{GetUnicodeKey}{\void}

Returns the Unicode character corresponding to this key event.

This function is only available in Unicode build, i.e. when
\texttt{wxUSE\_UNICODE} is $1$.


\membersection{wxKeyEvent::GetX}\label{wxkeyeventgetx}

\constfunc{long}{GetX}{\void}

Returns the X position (in client coordinates) of the event.


\membersection{wxKeyEvent::GetY}\label{wxkeyeventgety}

\constfunc{long}{GetY}{\void}

Returns the Y (in client coordinates) position of the event.


\membersection{wxKeyEvent::HasModifiers}\label{wxkeyeventhasmodifiers}

\constfunc{bool}{HasModifiers}{\void}

Returns true if either {\sc Ctrl} or {\sc Alt} keys was down
at the time of the key event. Note that this function does not take into
account neither {\sc Shift} nor {\sc Meta} key states (the reason for ignoring
the latter is that it is common for {\sc NumLock} key to be configured as
{\sc Meta} under X but the key presses even while {\sc NumLock} is on should
be still processed normally).


\membersection{wxKeyEvent::MetaDown}\label{wxkeyeventmetadown}

\constfunc{bool}{MetaDown}{\void}

Returns true if the Meta key was down at the time of the key event.

Notice that \helpref{GetModifiers}{wxkeyeventgetmodifiers} is easier to use
correctly than this function so you should consider using it in new code.


\membersection{wxKeyEvent::ShiftDown}\label{wxkeyeventshiftdown}

\constfunc{bool}{ShiftDown}{\void}

Returns true if the shift key was down at the time of the key event.

Notice that \helpref{GetModifiers}{wxkeyeventgetmodifiers} is easier to use
correctly than this function so you should consider using it in new code.

