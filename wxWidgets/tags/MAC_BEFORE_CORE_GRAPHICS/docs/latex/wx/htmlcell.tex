\subsection{Cells and Containers}\label{cells}

This article describes mechanism used by 
\helpref{wxHtmlWinParser}{wxhtmlwinparser} and 
\helpref{wxHtmlWindow}{wxhtmlwindow} to parse and display HTML documents.

\wxheading{Cells}

You can divide any text (or HTML) into small fragments. Let's call these
fragments {\bf cells}. Cell is for example one word, horizontal line, image
or any other part of document. Each cell has width and height (except special
"magic" cells with zero dimensions - e.g. colour changers or font changers).

See \helpref{wxHtmlCell}{wxhtmlcell}.

\wxheading{Containers}

Container is kind of cell that may contain sub-cells. Its size depends
on number and sizes of its sub-cells (and also depends on width of window). 

See \helpref{wxHtmlContainerCell}{wxhtmlcontainercell}, 
\helpref{wxHtmlCell::Layout}{wxhtmlcelllayout}.

\begin{comment}
% Bitmap is corrupt!
This image shows you cells and containers:

\helponly{\image{}{contbox.bmp}}
\end{comment}
\wxheading{Using Containers in Tag Handler}

\helpref{wxHtmlWinParser}{wxhtmlwinparser} provides a user-friendly way
of managing containers. It is based on the idea of opening and closing containers.

Use \helpref{OpenContainer}{wxhtmlwinparseropencontainer} to open new
a container {\it within an already opened container}. This new container is a 
{\it sub-container} of the old one. (If you want to create a new container with
the same depth level you can call {\tt CloseContainer(); OpenContainer();}.)

Use \helpref{CloseContainer}{wxhtmlwinparserclosecontainer} to close the 
container. This doesn't create a new container with same depth level but
it returns "control" to the parent container.

\begin{comment}
% Bitmap corrupt!
See explanation:

\helponly{\image{}{cont.bmp}}
\end{comment}
It is clear there must be same number of calls to 
OpenContainer as to CloseContainer...

\wxheading{Example}

This code creates a new paragraph (container at same depth level)
with "Hello, world!":

\begin{verbatim}
m_WParser -> CloseContainer();
c = m_WParser -> OpenContainer();

m_WParser -> AddWord("Hello, ");
m_WParser -> AddWord("world!");

m_WParser -> CloseContainer();
m_WParser -> OpenContainer();
\end{verbatim}

\begin{comment}
% Bitmap corrupt!
and here is image of the situation:

\helponly{\image{}{hello.bmp}}
\end{comment}

You can see that there was opened container before running the code. We closed
it, created our own container, then closed our container and opened
new container. The result was that we had {\it same depth level} after
executing. This is general rule that should be followed by tag handlers:
leave depth level of containers unmodified (in other words, number of
OpenContainer and CloseContainer calls should be same within \helpref{HandleTag}{wxhtmltaghandlerhandletag}'s body).

