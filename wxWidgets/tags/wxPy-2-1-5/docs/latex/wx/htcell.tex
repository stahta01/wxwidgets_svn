%
% automatically generated by HelpGen from
% htmlcell.tex at 21/Mar/99 22:45:23
%

\section{\class{wxHtmlCell}}\label{wxhtmlcell}

Internal data structure. It represents fragments of parsed HTML
page, so-called {\bf cell} - a word, picture, table, horizontal line and so on.
It is used by \helpref{wxHtmlWindow}{wxhtmlwindow} and 
\helpref{wxHtmlWinParser}{wxhtmlwinparser} to represent HTML page in memory.

You can divide cells into two groups : {\it visible} cells with non-zero width and
height and {\it helper} cells (usually with zero width and height) that
perform special actions such as color or font change.

\wxheading{Derived from}

wxObject

\wxheading{See Also}

\helpref{Cells Overview}{cells},
\helpref{wxHtmlContainerCell}{wxhtmlcontainercell}

\latexignore{\rtfignore{\wxheading{Members}}}

\membersection{wxHtmlCell::wxHtmlCell}\label{wxhtmlcellwxhtmlcell}

\func{}{wxHtmlCell}{\void}

Constructor.

\membersection{wxHtmlCell::SetParent}\label{wxhtmlcellsetparent}

\func{void}{SetParent}{\param{wxHtmlContainerCell }{*p}}

Sets parent container of this cell. This is called from
\helpref{wxHtmlContainerCell::InsertCell}{wxhtmlcontainercellinsertcell}.

\membersection{wxHtmlCell::GetParent}\label{wxhtmlcellgetparent}

\constfunc{wxHtmlContainerCell*}{GetParent}{\void}

Returns pointer to parent container.

\membersection{wxHtmlCell::GetPosX}\label{wxhtmlcellgetposx}

\constfunc{int}{GetPosX}{\void}

Returns X position within parent (the value is relative to parent's
upper left corner). The returned value is meaningful only if
parent's \helpref{Layout}{wxhtmlcelllayout} was called before!

\membersection{wxHtmlCell::GetPosY}\label{wxhtmlcellgetposy}

\constfunc{int}{GetPosY}{\void}

Returns Y position within parent (the value is relative to parent's
upper left corner). The returned value is meaningful only if
parent's \helpref{Layout}{wxhtmlcelllayout} was called before!

\membersection{wxHtmlCell::GetWidth}\label{wxhtmlcellgetwidth}

\constfunc{int}{GetWidth}{\void}

Returns width of the cell (m_Width member).

\membersection{wxHtmlCell::GetHeight}\label{wxhtmlcellgetheight}

\constfunc{int}{GetHeight}{\void}

Returns height of the cell (m_Height member).

\membersection{wxHtmlCell::GetDescent}\label{wxhtmlcellgetdescent}

\constfunc{int}{GetDescent}{\void}

Returns descent value of the cell (m_Descent member). See explanation:

\image{}{descent.bmp}

\membersection{wxHtmlCell::GetLink}\label{wxhtmlcellgetlink}

\constfunc{virtual wxString}{GetLink}{\param{int }{x = 0}, \param{int }{y = 0}}

Returns hypertext link if associated with this cell or empty string otherwise.
(Note : this makes sense only for visible tags).

\wxheading{Parameters}

\docparam{x,y}{Coordinates of position where the user pressed mouse button.
These coordinates are used e.g. by COLORMAP. Values are relative to the
upper left corner of THIS cell (i.e. from 0 to m_Width or m_Height)}

\membersection{wxHtmlCell::GetNext}\label{wxhtmlcellgetnext}

\constfunc{wxHtmlCell*}{GetNext}{\void}

Returns pointer to the next cell in list (see htmlcell.h if you're
interested in details).

\membersection{wxHtmlCell::SetPos}\label{wxhtmlcellsetpos}

\func{void}{SetPos}{\param{int }{x}, \param{int }{y}}

Sets cell's position within parent container.

\membersection{wxHtmlCell::SetLink}\label{wxhtmlcellsetlink}

\func{void}{SetLink}{\param{const wxString\& }{link}}

Sets the hypertext link asocciated with this cell. (Default value
is wxEmptyString (no link))

\membersection{wxHtmlCell::SetNext}\label{wxhtmlcellsetnext}

\func{void}{SetNext}{\param{wxHtmlCell }{*cell}}

Sets the next cell in the list. This shouldn't be called by user - it is
to be used only by \helpref{wxHtmlContainerCell::InsertCell}{wxhtmlcontainercellinsertcell}

\membersection{wxHtmlCell::Layout}\label{wxhtmlcelllayout}

\func{virtual void}{Layout}{\param{int }{w}}

This method performs 2 actions:

\begin{enumerate}
\item adjusts cell's width according to the fact that maximal possible width is {\it w}.
(this has sense when working with horizontal lines, tables etc.)
\item prepares layout (=fill-in m\_PosX, m\_PosY (and sometimes m\_Height) members)
based on actual width {\it w}
\end{enumerate}

It must be called before displaying cells structure because
m\_PosX and m\_PosY are undefined (or invalid)
before calling Layout.

\membersection{wxHtmlCell::Draw}\label{wxhtmlcelldraw}

\func{virtual void}{Draw}{\param{wxDC\& }{dc}, \param{int }{x}, \param{int }{y}, \param{int }{view\_y1}, \param{int }{view\_y2}}

Renders the cell.

\wxheading{Parameters}

\docparam{dc}{Device context to which the cell is to be drawn}

\docparam{x,y}{Coordinates of parent's upper left corner (origin). You must
add this to m\_PosX,m\_PosY when passing coordinates to dc's methods
Example : {\tt dc -> DrawText("hello", x + m\_PosX, y + m\_PosY)}}

\docparam{view_y1}{y-coord of the first line visible in window. This is
used to optimize rendering speed}

\docparam{view_y2}{y-coord of the last line visible in window. This is
used to optimize rendering speed}

\membersection{wxHtmlCell::DrawInvisible}\label{wxhtmlcelldrawinvisible}

\func{virtual void}{DrawInvisible}{\param{wxDC\& }{dc}, \param{int }{x}, \param{int }{y}}

This method is called instead of \helpref{Draw}{wxhtmlcelldraw} when the
cell is certainly out of the screen (and thus invisible). This is not
nonsense - some tags (like \helpref{wxHtmlColourCell}{wxhtmlcolourcell}
or font setter) must be drawn even if they are invisible!

\wxheading{Parameters}

\docparam{dc}{Device context to which the cell is to be drawn}

\docparam{x,y}{Coordinates of parent's upper left corner. You must
add this to m\_PosX,m\_PosY when passing coordinates to dc's methods
Example : {\tt dc -> DrawText("hello", x + m\_PosX, y + m\_PosY)}}

\membersection{wxHtmlCell::Find}\label{wxhtmlcellfind}

\func{virtual const wxHtmlCell*}{Find}{\param{int }{condition}, \param{const void* }{param}}

Returns pointer to itself if this cell matches condition (or if any of the cells
following in the list matches), NULL otherwise.
(In other words if you call top-level container's Find it will
return pointer to the first cell that matches the condition)

It is recommended way how to obtain pointer to particular cell or
to cell of some type (e.g. wxHtmlAnchorCell reacts on
HTML_COND_ISANCHOR condition)

\wxheading{Parameters}

\docparam{condition}{Unique integer identifier of condition}

\docparam{param}{Optional parameters}

\wxheading{Defined conditions}

\begin{twocollist}
\twocolitem{{\bf HTML_COND_ISANCHOR}}{Finds particular anchor.
{\it param} is pointer to wxString with name of the anchor.}
\twocolitem{{\bf HTML_COND_USER}}{User-defined conditions start
from this number}
\end{twocollist}

\membersection{wxHtmlCell::OnMouseClick}\label{wxhtmlcellonmouseclick}

\func{virtual void}{OnMouseClick}{\param{wxWindow* }{parent}, \param{int }{x}, \param{int }{y}, \param{bool }{left}, \param{bool }{middle}, \param{bool }{right}}

This function is simple event handler. Each time user clicks mouse button over a cell
within \helpref{wxHtmlWindow}{wxhtmlwindow} this method of that cell is called. Default behavior is
that it calls \helpref{wxHtmlWindow::LoadPage}{wxhtmlwindowloadpage}.

\wxheading{Note}

If you need more "advanced" behaviour (for example you'd like to catch mouse movement events or
key events or whatsoever) you should use wxHtmlBinderCell instead.

\wxheading{Parameters}

\docparam{parent}{parent window (always wxHtmlWindow!)}

\docparam{x, y}{coordinates of mouse click (this is relative to cell's origin}

\docparam{left, middle, right}{boolean flags for mouse buttons. TRUE if the left/middle/right
button is pressed, FALSE otherwise}

