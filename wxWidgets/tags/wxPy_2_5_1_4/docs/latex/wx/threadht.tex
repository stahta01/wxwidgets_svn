\section{\class{wxThreadHelperThread}}\label{wxThreadHelperThread}

The wxThreadHelperThread class is used internally by the
\helpref{wxThreadHelper}{wxthreadhelper} mix-in class.  This class simply
calls \helpref{wxThreadHelper::Entry}{wxthreadhelperentry} in its owner class
when the thread runs.

\wxheading{Derived from}

\helpref{wxThread}{wxthread}

\wxheading{Include files}

<wx/thread.h>

\wxheading{See also}

\helpref{wxThread}{wxthread}, \helpref{wxThreadHelper}{wxthreadhelper}

\latexignore{\rtfignore{\wxheading{Members}}}

\membersection{wxThreadHelperThread::wxThreadHelperThread}\label{wxthreadhelperthreadctor}

\func{}{wxThreadHelperThread}{\void}

This constructor simply initializes member variables.

\membersection{wxThreadHelperThread::m\_owner}

\member{wxThreadHelperThread\& }{m\_owner}

the \helpref{wxThreadHelper}{wxthreadhelper} object which holds the code to
run inside the thread.

\membersection{wxThreadHelperThread::Entry}\label{wxthreadhelperthreadentry}

\func{virtual ExitCode}{Entry}{\void}

This is the entry point of the thread.  This function eventually calls
\helpref{wxThreadHelper::Entry}{wxthreadhelperentry}.  The actual worker
thread code should be implemented in
\helpref{wxThreadHelper::Entry}{wxthreadhelperentry}, not here, so all
shared data and synchronization objects can be shared easily between the
main thread and the worker thread.

The returned value is the thread exit code which is the value returned by
\helpref{Wait()}{wxthreadwait}.

This function is called by wxWindows itself and should never be called
directly.

\membersection{wxThreadHelperThread::CallEntry}\label{wxthreadhelperthreadcallentry}

\func{virtual ExitCode}{CallEntry}{\void}

This is a convenience method that actually calls
\helpref{wxThreadHelper::Entry}{wxthreadhelperentry}.  This function
eventually calls \helpref{wxThreadHelper::Entry}{wxthreadhelperentry}.
The actual worker thread code should be implemented in
\helpref{wxThreadHelper::Entry}{wxthreadhelperentry}, not here, so all
shared data and synchronization objects can be shared easily between the
main thread and the worker thread.

It must be declared after \helpref{wxThreadHelper}{wxthreadhelper} so it
can access \helpref{wxThreadHelper::Entry}{wxthreadhelperentry} and avoid
circular dependencies.  Thus, it uses the inline keyword to allow its
definition outside of the class definition.  To avoid any conflicts
between the virtual and inline keywords, it is a non-virtual method.

The returned value is the thread exit code which is the value returned by
\helpref{Wait()}{wxthreadwait}.

This function is called by wxWindows itself and should never be called
directly.

