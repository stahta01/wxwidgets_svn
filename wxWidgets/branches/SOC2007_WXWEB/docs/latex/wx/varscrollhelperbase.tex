%%%%%%%%%%%%%%%%%%%%%%%%%%%%%%%%%%%%%%%%%%%%%%%%%%%%%%%%%%%%%%%%%%%%%%%%%%%%%
%% Name:        varscrollhelperbase.tex
%% Purpose:     wxVarScrollHelperBase Documentation
%% Author:      Bryan Petty
%% Modified by: 
%% Created:     2007-04-04
%% RCS-ID:      $Id$
%% Copyright:   (c) 2007 wxWidgets Team
%% License:     wxWindows Licence
%%%%%%%%%%%%%%%%%%%%%%%%%%%%%%%%%%%%%%%%%%%%%%%%%%%%%%%%%%%%%%%%%%%%%%%%%%%%%

\section{\class{wxVarScrollHelperBase}}\label{wxvarscrollhelperbase}

This class provides all common base functionality for scroll calculations
shared among all variable scrolled window implementations as well as
automatic scrollbar functionality, saved scroll positions, controlling
target windows to be scrolled, as well as defining all required virtual
functions that need to be implemented for any orientation specific work.

Documentation of this class is provided specifically for referencing use
of the functions provided by this class for use with the variable scrolled
windows that derive from here. You will likely want to derive your window
from one of the already implemented variable scrolled windows rather than
from wxVarScrollHelperBase directly.

\wxheading{Include files}

<wx/vscroll.h>

\wxheading{See also}

\helpref{wxHScrolledWindow}{wxhscrolledwindow},
\rtfsp\helpref{wxHVScrolledWindow}{wxhvscrolledwindow},
\rtfsp\helpref{wxVScrolledWindow}{wxvscrolledwindow}

\latexignore{\rtfignore{\wxheading{Members}}}


\membersection{wxVarScrollHelperBase::wxVarScrollHelperBase}\label{wxvarscrollhelperbasewxvarscrollhelperbase}

\func{}{wxVarScrollHelperBase}{\param{wxWindow* }{winToScroll}}

Constructor taking the target window to be scrolled by this helper class.
This will attach scroll event handlers to the target window to catch and
handle scroll events appropriately.


\membersection{wxVarScrollHelperBase::\destruct{wxVarScrollHelperBase}}\label{wxvarscrollhelperbasedtor}

\func{virtual }{\destruct{wxVarScrollHelperBase}}{\void}

Virtual destructor for detaching scroll event handlers attached with this
helper class.


\membersection{wxVarScrollHelperBase::CalcScrolledPosition}\label{wxvarscrollhelperbasecalcscrolledposition}

\constfunc{int}{CalcScrolledPosition}{\param{int }{coord}}

Translates the logical coordinate given to the current device coordinate.
For example, if the window is scrolled 10 units and each scroll unit
represents 10 device units (which may not be the case since this class allows
for variable scroll unit sizes), a call to this function with a coordinate of
15 will return -85.

\wxheading{See also}

\helpref{CalcUnscrolledPosition()}{wxvarscrollhelperbasecalcunscrolledposition}


\membersection{wxVarScrollHelperBase::CalcUnscrolledPosition}\label{wxvarscrollhelperbasecalcunscrolledposition}

\constfunc{int}{CalcUnscrolledPosition}{\param{int }{coord}}

Translates the device coordinate given to the corresponding logical
coordinate. For example, if the window is scrolled 10 units and each scroll
unit represents 10 device units (which may not be the case since this class
allows for variable scroll unit sizes), a call to this function with a
coordinate of 15 will return 115.

\wxheading{See also}

\helpref{CalcScrolledPosition()}{wxvarscrollhelperbasecalcscrolledposition}


\membersection{wxVarScrollHelperBase::EnablePhysicalScrolling}\label{wxvarscrollhelperbaseenablephysicalscrolling}

\func{void}{EnablePhysicalScrolling}{\param{bool }{scrolling = true}}

With physical scrolling on (when this is {\tt true}), the device origin is
changed properly when a \rtfsp\helpref{wxPaintDC}{wxpaintdc} is prepared,
children are actually moved and laid out properly, and the contents of the
window (pixels) are actually moved. When this is {\tt false}, you are
responsible for repainting any invalidated areas of the window yourself to
account for the new scroll position.


\membersection{wxVarScrollHelperBase::EstimateTotalSize}\label{wxvarscrollhelperbaseestimatetotalsize}

\constfunc{virtual wxCoord}{EstimateTotalSize}{\void}

When the number of scroll units change, we try to estimate the total size of
all units when the full window size is needed (i.e. to calculate the scrollbar
thumb size). This is a rather expensive operation in terms of unit access, so
if the user code may estimate the average size better or faster than we do, it
should override this function to implement its own logic. This function should
return the best guess for the total virtual window size.

Note that although returning a totally wrong value would still work, it risks
resulting in very strange scrollbar behaviour so this function should really
try to make the best guess possible.


\membersection{wxVarScrollHelperBase::GetNonOrientationTargetSize}\label{wxvarscrollhelperbasegetnonorientationtargetsize}

\constfunc{virtual int}{GetNonOrientationTargetSize}{\void}

This function needs to be overridden in the in the derived class to return the
window size with respect to the opposing orientation. If this is a vertical
scrolled window, it should return the height.

\wxheading{See also}

\helpref{GetOrientationTargetSize()}{wxvarscrollhelperbasegetorientationtargetsize}


\membersection{wxVarScrollHelperBase::GetOrientation}\label{wxvarscrollhelperbasegetorientation}

\constfunc{virtual wxOrientation}{GetOrientation}{\void}

This function need to be overridden to return the orientation that this helper
is working with, either {\tt wxHORIZONTAL} or {\tt wxVERTICAL}.


\membersection{wxVarScrollHelperBase::GetOrientationTargetSize}\label{wxvarscrollhelperbasegetorientationtargetsize}

\constfunc{virtual int}{GetOrientationTargetSize}{\void}

This function needs to be overridden in the in the derived class to return the
window size with respect to the orientation this helper is working with. If
this is a vertical scrolled window, it should return the width.

\wxheading{See also}

\helpref{GetNonOrientationTargetSize()}{wxvarscrollhelperbasegetnonorientationtargetsize}


\membersection{wxVarScrollHelperBase::GetTargetWindow}\label{wxvarscrollhelperbasegettargetwindow}

\constfunc{wxWindow*}{GetTargetWindow}{\void}

This function will return the target window this helper class is currently
scrolling.

\wxheading{See also}

\helpref{SetTargetWindow()}{wxvarscrollhelperbasesettargetwindow}


\membersection{wxVarScrollHelperBase::GetVisibleBegin}\label{wxvarscrollhelperbasegetvisiblebegin}

\constfunc{size\_t}{GetVisibleBegin}{\void}

Returns the index of the first visible unit based on the scroll position.


\membersection{wxVarScrollHelperBase::GetVisibleEnd}\label{wxvarscrollhelperbasegetvisibleend}

\constfunc{size\_t}{GetVisibleEnd}{\void}

Returns the index of the last visible unit based on the scroll position. This
includes the last unit even if it is only partially visible.


\membersection{wxVarScrollHelperBase::VirtualHitTest}\label{wxvarscrollhelperbasevirtualhittest}

\constfunc{int}{VirtualHitTest}{\param{wxCoord }{coord}}

Returns the virtual scroll unit under the device unit given accounting for
scroll position or {\tt wxNOT\_FOUND} if none (i.e. if it is below the last
item).


\membersection{wxVarScrollHelperBase::IsVisible}\label{wxvarscrollhelperbaseisvisible}

\constfunc{bool}{IsVisible}{\param{size\_t }{unit}}

Returns {\tt true} if the given scroll unit is currently visible (even if only
partially visible) or {\tt false} otherwise.


\membersection{wxVarScrollHelperBase::OnGetUnitSize}\label{wxvarscrollhelperbaseongetunitsize}

\constfunc{virtual wxCoord}{OnGetUnitSize}{\param{size\_t }{unit}}

This function must be overridden in the derived class, and should return the
size of the given unit in pixels.


\membersection{wxVarScrollHelperBase::OnGetUnitsSizeHint}\label{wxvarscrollhelperbaseongetunitssizehint}

\constfunc{virtual void}{OnGetUnitsSizeHint}{\param{size\_t }{unitMin}, \param{size\_t }{unitMax}}

This function doesn't have to be overridden but it may be useful to do so if
calculating the units' sizes is a relatively expensive operation as it gives
your code a chance to calculate several of them at once and cache the result
if necessary.

{\tt OnGetUnitsSizeHint()} is normally called just before
\helpref{OnGetUnitSize()}{wxvarscrollhelperbaseongetunitsize} but you
shouldn't rely on the latter being called for all units in the interval
specified here. It is also possible that OnGetUnitSize() will be called for
units outside of this interval, so this is really just a hint, not a promise.

Finally, note that unitMin is inclusive, while unitMax is exclusive.


\membersection{wxVarScrollHelperBase::RefreshAll}\label{wxvarscrollhelperbaserefreshall}

\func{virtual void}{RefreshAll}{\void}

Recalculate all parameters and repaint all units.


\membersection{wxVarScrollHelperBase::SetTargetWindow}\label{wxvarscrollhelperbasesettargetwindow}

\func{void}{SetTargetWindow}{\param{wxWindow* }{target}}

Normally the window will scroll itself, but in some rare occasions you might
want it to scroll (part of) another window (e.g. a child of it in order to
scroll only a portion the area between the scrollbars like a spreadsheet where
only the cell area will move).

\wxheading{See also}

\helpref{GetTargetWindow()}{wxvarscrollhelperbasegettargetwindow}


\membersection{wxVarScrollHelperBase::UpdateScrollbar}\label{wxvarscrollhelperbaseupdatescrollbar}

\func{virtual void}{UpdateScrollbar}{\void}

Update the thumb size shown by the scrollbar.

