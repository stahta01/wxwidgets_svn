\section{\class{wxWindowDestroyEvent}}\label{wxwindowdestroyevent}

% Note: the original description is still here commented out in case those
%       semantics are restored to this event.

%This event is sent just before the actual window associated with a wxWindow object
%is destroyed. Since it is derived from wxCommandEvent, the event propagates up
%the window hierarchy.

This event is sent from the wxWindow destructor wxWindow::~wxWindow() when a
window is destroyed.

When a class derived from wxWindow is destroyed its destructor will have
already run by the time this event is sent. Therefore this event will not
usually be received at all.

To receive this event \helpref{wxEvtHandler::Connect}{wxevthandlerconnect}
must be used (using an event table macro will not work). Since it is
received after the destructor has run, an object should not handle its
own wxWindowDestroyEvent, but it can be used to get notification of the
destruction of another window.

\wxheading{Derived from}

\helpref{wxCommandEvent}{wxevent}\\
\helpref{wxEvent}{wxevent}\\
\helpref{wxObject}{wxobject}

\wxheading{Include files}

<wx/event.h>

\wxheading{Event table macros}

%To process a window destruction event, use this event handler macro to direct input to a member
%function that takes a wxWindowDestroyEvent argument.
%
%\twocolwidtha{7cm}
%\begin{twocollist}\itemsep=0pt
%\twocolitem{{\bf EVT\_WINDOW\_DESTROY(func)}}{Process a wxEVT\_DESTROY event.}
%\end{twocollist}%

It is not possible to receive this event using an event table macro.

\wxheading{See also}

\helpref{Event handling overview}{eventhandlingoverview},\rtfsp
\helpref{wxWindowCreateEvent}{wxwindowcreateevent}

\latexignore{\rtfignore{\wxheading{Members}}}

\membersection{wxWindowDestroyEvent::wxWindowDestroyEvent}\label{wxwindowdestroyeventctor}

\func{}{wxWindowDestroyEvent}{\param{wxWindow* }{win = NULL}}

Constructor.

