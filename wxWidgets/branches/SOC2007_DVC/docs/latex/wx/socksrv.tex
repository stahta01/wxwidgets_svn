%%%%%%%%%%%%%%%%%%%%%%%%%%%%%%%%%%%%%%%%%%%%%%%%%%%%%%%%%%%%%%%%%%%%%%%%%%%%%%%
%% Name:        socket.tex
%% Purpose:     wxSocket docs
%% Author:      Guillermo Rodriguez Garcia <guille@iies.es>
%% Modified by:
%% Created:     14.01.02 (extracted from socket.tex)
%% RCS-ID:      $Id$
%% Copyright:   (c) wxWidgets team
%% License:     wxWindows license
%%%%%%%%%%%%%%%%%%%%%%%%%%%%%%%%%%%%%%%%%%%%%%%%%%%%%%%%%%%%%%%%%%%%%%%%%%%%%%%

% ---------------------------------------------------------------------------
% CLASS: wxSocketServer
% ---------------------------------------------------------------------------
\section{\class{wxSocketServer}}\label{wxsocketserver}

\wxheading{Derived from}

\helpref{wxSocketBase}{wxsocketbase}

\wxheading{Include files}

<wx/socket.h>

\wxheading{Library}

\helpref{wxNet}{librarieslist}

% ---------------------------------------------------------------------------
% Members
% ---------------------------------------------------------------------------
\latexignore{\rtfignore{\wxheading{Members}}}

%
% wxSocketServer
%
\membersection{wxSocketServer::wxSocketServer}\label{wxsocketserverctor}

\func{}{wxSocketServer}{\param{const wxSockAddress\&}{ address}, \param{wxSocketFlags}{ flags = wxSOCKET\_NONE}}

Constructs a new server and tries to bind to the specified {\it address}.
Before trying to accept new connections, test whether it succeeded with 
\helpref{wxSocketBase:IsOk}{wxsocketbaseisok}.

\wxheading{Parameters}

\docparam{address}{Specifies the local address for the server (e.g. port number).}

\docparam{flags}{Socket flags (See \helpref{wxSocketBase::SetFlags}{wxsocketbasesetflags})}

%
% ~wxSocketServer
%
\membersection{wxSocketServer::\destruct{wxSocketServer}}\label{wxsocketserverdtor}

\func{}{\destruct{wxSocketServer}}{\void}

Destructor (it doesn't close the accepted connections).

%
% Accept
%
\membersection{wxSocketServer::Accept}\label{wxsocketserveraccept}

\func{wxSocketBase *}{Accept}{\param{bool}{ wait = true}}

Accepts an incoming connection request, and creates a new 
\helpref{wxSocketBase}{wxsocketbase} object which represents
the server-side of the connection.

If {\it wait} is true and there are no pending connections to be
accepted, it will wait for the next incoming connection to
arrive. {\bf Warning:} This will block the GUI.

If {\it wait} is false, it will try to accept a pending connection
if there is one, but it will always return immediately without blocking
the GUI. If you want to use Accept in this way, you can either check for
incoming connections with \helpref{WaitForAccept}{wxsocketserverwaitforaccept} 
or catch {\bf wxSOCKET\_CONNECTION} events, then call Accept once you know
that there is an incoming connection waiting to be accepted.

\wxheading{Return value}

Returns an opened socket connection, or NULL if an error occurred or
if the {\it wait} parameter was false and there were no pending
connections.

\wxheading{See also}

\helpref{wxSocketServer::WaitForAccept}{wxsocketserverwaitforaccept}, 
\helpref{wxSocketBase::SetNotify}{wxsocketbasesetnotify}, 
\helpref{wxSocketBase::Notify}{wxsocketbasenotify}, 
\helpref{wxSocketServer::AcceptWith}{wxsocketserveracceptwith}

%
% AcceptWith
%
\membersection{wxSocketServer::AcceptWith}\label{wxsocketserveracceptwith}

\func{bool}{AcceptWith}{\param{wxSocketBase\&}{ socket}, \param{bool}{ wait = true}}

Accept an incoming connection using the specified socket object.

\wxheading{Parameters}

\docparam{socket}{Socket to be initialized}

\wxheading{Return value}

Returns true on success, or false if an error occurred or if the
{\it wait} parameter was false and there were no pending
connections.

\helpref{wxSocketServer::WaitForAccept}{wxsocketserverwaitforaccept}, 
\helpref{wxSocketBase::SetNotify}{wxsocketbasesetnotify}, 
\helpref{wxSocketBase::Notify}{wxsocketbasenotify}, 
\helpref{wxSocketServer::Accept}{wxsocketserveraccept}

%
% WaitForAccept
%
\membersection{wxSocketServer::WaitForAccept}\label{wxsocketserverwaitforaccept}

\func{bool}{WaitForAccept}{\param{long}{ seconds = -1}, \param{long}{ millisecond = 0}}

This function waits for an incoming connection. Use it if you want to call 
\helpref{Accept}{wxsocketserveraccept} or \helpref{AcceptWith}{wxsocketserveracceptwith} 
with {\it wait} set to false, to detect when an incoming connection is waiting
to be accepted.

\wxheading{Parameters}

\docparam{seconds}{Number of seconds to wait.
If -1, it will wait for the default timeout,
as set with \helpref{SetTimeout}{wxsocketbasesettimeout}.}

\docparam{millisecond}{Number of milliseconds to wait.}

\wxheading{Return value}

Returns true if an incoming connection arrived, false if the timeout elapsed.

\wxheading{See also}

\helpref{wxSocketServer::Accept}{wxsocketserveraccept}, 
\helpref{wxSocketServer::AcceptWith}{wxsocketserveracceptwith},
\helpref{wxSocketBase::InterruptWait}{wxsocketbaseinterruptwait}


