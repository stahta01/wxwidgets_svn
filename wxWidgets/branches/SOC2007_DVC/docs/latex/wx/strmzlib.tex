% -----------------------------------------------------------------------------
% wxZlibInputStream
% -----------------------------------------------------------------------------
\section{\class{wxZlibInputStream}}\label{wxzlibinputstream}

This filter stream decompresses a stream that is in zlib or gzip format.
Note that reading the gzip format requires zlib version 1.2.1 or greater,
(the builtin version does support gzip format).

The stream is not seekable, \helpref{SeekI()}{wxinputstreamseeki} returns
 {\it wxInvalidOffset}. Also \helpref{GetSize()}{wxstreambasegetsize} is
not supported, it always returns $0$.

\wxheading{Derived from}

\helpref{wxFilterInputStream}{wxfilterinputstream}

\wxheading{Include files}

<wx/zstream.h>

\wxheading{Library}

\helpref{wxBase}{librarieslist}

\wxheading{See also}

\helpref{wxInputStream}{wxinputstream}, 
 \helpref{wxZlibOutputStream}{wxzliboutputstream}.

\latexignore{\rtfignore{\wxheading{Members}}}

\membersection{wxZlibInputStream::wxZlibInputStream}\label{wxzlibinputstreamwxzlibinputstream}

\func{}{wxZlibInputStream}{\param{wxInputStream\&}{ stream}, \param{int}{ flags = wxZLIB\_AUTO}}

\func{}{wxZlibInputStream}{\param{wxInputStream*}{ stream}, \param{int}{ flags = wxZLIB\_AUTO}}

If the parent stream is passed as a pointer then the new filter stream
takes ownership of it. If it is passed by reference then it does not.

The {\it flags} wxZLIB\_ZLIB and wxZLIB\_GZIP specify whether the input data
is in zlib or gzip format. If wxZLIB\_AUTO is used, then zlib will
autodetect the stream type, this is the default.

If {\it flags} is wxZLIB\_NO\_HEADER, then the data is assumed to be a raw
deflate stream without either zlib or gzip headers. This is a lower level
mode, which is not usually used directly. It can be used to read a raw
deflate stream embedded in a higher level protocol.

%if WXWIN_COMPATIBILITY_2_4
This version is not by default compatible with the output produced by
the version of {\it wxZlibOutputStream} in wxWidgets 2.4.x. However,
there is a compatibility mode, which is switched on by passing
wxZLIB\_24COMPATIBLE for flags. Note that in when operating in compatibility
mode error checking is very much reduced.
%endif

The following symbols can be use for the flags:

\begin{verbatim}
// Flags
enum {
#if WXWIN_COMPATIBILITY_2_4
    wxZLIB_24COMPATIBLE = 4, // read v2.4.x data without error
#endif
    wxZLIB_NO_HEADER = 0,    // raw deflate stream, no header or checksum
    wxZLIB_ZLIB = 1,         // zlib header and checksum
    wxZLIB_GZIP = 2,         // gzip header and checksum, requires zlib 1.2.1+
    wxZLIB_AUTO = 3          // autodetect header zlib or gzip
};
\end{verbatim}

\membersection{wxZlibInputStream::CanHandleGZip}\label{wxzlibinputstreamcanhandlegzip}

\func{static bool}{CanHandleGZip}{\void}

Returns true if zlib library in use can handle gzip compressed data.

% -----------------------------------------------------------------------------
% wxZlibOutputStream
% -----------------------------------------------------------------------------
\section{\class{wxZlibOutputStream}}\label{wxzliboutputstream}

This stream compresses all data written to it. The compressed output can be
in zlib or gzip format.
Note that writing the gzip format requires zlib version 1.2.1 or greater
(the builtin version does support gzip format).

The stream is not seekable, \helpref{SeekO()}{wxoutputstreamseeko} returns
 {\it wxInvalidOffset}.

\wxheading{Derived from}

\helpref{wxFilterOutputStream}{wxfilteroutputstream}

\wxheading{Include files}

<wx/zstream.h>

\wxheading{Library}

\helpref{wxBase}{librarieslist}

\wxheading{See also}

\helpref{wxOutputStream}{wxoutputstream},
 \helpref{wxZlibInputStream}{wxzlibinputstream}


\latexignore{\rtfignore{\wxheading{Members}}}

\membersection{wxZlibOutputStream::wxZlibOutputStream}\label{wxzliboutputstreamwxzliboutputstream}

\func{}{wxZlibOutputStream}{\param{wxOutputStream\&}{ stream}, \param{int}{ level = -1}, \param{int}{ flags = wxZLIB\_ZLIB}}

\func{}{wxZlibOutputStream}{\param{wxOutputStream*}{ stream}, \param{int}{ level = -1}, \param{int}{ flags = wxZLIB\_ZLIB}}

Creates a new write-only compressed stream. {\it level} means level of 
compression. It is number between 0 and 9 (including these values) where
0 means no compression and 9 best but slowest compression. -1 is default
value (currently equivalent to 6).

If the parent stream is passed as a pointer then the new filter stream
takes ownership of it. If it is passed by reference then it does not.

The {\it flags} wxZLIB\_ZLIB and wxZLIB\_GZIP specify whether the output data
will be in zlib or gzip format. wxZLIB\_ZLIB is the default.

If {\it flags} is wxZLIB\_NO\_HEADER, then a raw deflate stream is output
without either zlib or gzip headers. This is a lower level
mode, which is not usually used directly. It can be used to embed a raw
deflate stream in a higher level protocol.

The following symbols can be use for the compression level and flags:

\begin{verbatim}
// Compression level
enum {
    wxZ_DEFAULT_COMPRESSION = -1,
    wxZ_NO_COMPRESSION = 0,
    wxZ_BEST_SPEED = 1,
    wxZ_BEST_COMPRESSION = 9
};

// Flags
enum {
    wxZLIB_NO_HEADER = 0,   // raw deflate stream, no header or checksum
    wxZLIB_ZLIB = 1,        // zlib header and checksum
    wxZLIB_GZIP = 2         // gzip header and checksum, requires zlib 1.2.1+
};
\end{verbatim}

\membersection{wxZlibOutputStream::CanHandleGZip}\label{wxoutputstreamcanhandlegzip}

\func{static bool}{CanHandleGZip}{\void}

Returns true if zlib library in use can handle gzip compressed data.

