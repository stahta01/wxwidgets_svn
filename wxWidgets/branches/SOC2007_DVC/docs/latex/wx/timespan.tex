%%%%%%%%%%%%%%%%%%%%%%%%%%%%%%%%%%%%%%%%%%%%%%%%%%%%%%%%%%%%%%%%%%%%%%%%%%%%%%
%% Name:        datespan.tex
%% Purpose:     wxDateSpan documentation
%% Author:      Vadim Zeitlin
%% Modified by:
%% Created:     04.04.00
%% RCS-ID:      $Id$
%% Copyright:   (c) Vadim Zeitlin
%% License:     wxWindows license
%%%%%%%%%%%%%%%%%%%%%%%%%%%%%%%%%%%%%%%%%%%%%%%%%%%%%%%%%%%%%%%%%%%%%%%%%%%%%%%

\section{\class{wxTimeSpan}}\label{wxtimespan}

wxTimeSpan class represents a time interval.

\wxheading{Derived from}

No base class

\wxheading{Include files}

<wx/datetime.h>

\wxheading{See also}

\helpref{Date classes overview}{wxdatetimeoverview},\rtfsp
\helpref{wxDateTime}{wxdatetime}

\latexignore{\rtfignore{\wxheading{Function groups}}}


\membersection{Static functions}\label{timespanstaticfunctions}

\helpref{Milliseconds}{wxtimespanmilliseconds}\\
\helpref{Millisecond}{wxtimespanmillisecond}\\
\helpref{Seconds}{wxtimespanseconds}\\
\helpref{Second}{wxtimespansecond}\\
\helpref{Minutes}{wxtimespanminutes}\\
\helpref{Minute}{wxtimespanminute}\\
\helpref{Hours}{wxtimespanhours}\\
\helpref{Hour}{wxtimespanhour}\\
\helpref{Days}{wxtimespandays}\\
\helpref{Day}{wxtimespanday}\\
\helpref{Weeks}{wxtimespanweeks}\\
\helpref{Week}{wxtimespanweek}


\membersection{Constructors}\label{timespanconstructors}

\helpref{wxTimeSpan}{wxtimespanctor}


\membersection{Accessors}\label{timespanaccessors}

\helpref{GetSeconds}{wxtimespangetseconds}\\
\helpref{GetMinutes}{wxtimespangetminutes}\\
\helpref{GetHours}{wxtimespangethours}\\
\helpref{GetDays}{wxtimespangetdays}\\
\helpref{GetWeeks}{wxtimespangetweeks}\\
\helpref{GetValue}{wxtimespangetvalue}


\membersection{Operations}\label{timespanoperations}

\helpref{Add}{wxtimespanadd}\\
\helpref{Subtract}{wxtimespansubtract}\\
\helpref{Multiply}{wxtimespanmultiply}\\
\helpref{Negate}{wxtimespannegate}\\
\helpref{Neg}{wxtimespanneg}\\
\helpref{Abs}{wxtimespanabs}


\membersection{Tests}\label{timespantests}

\helpref{IsNull}{wxtimespanisnull}\\
\helpref{IsPositive}{wxtimespanispositive}\\
\helpref{IsNegative}{wxtimespanisnegative}\\
\helpref{IsEqualTo}{wxtimespanisequalto}\\
\helpref{IsLongerThan}{wxtimespanislongerthan}\\
\helpref{IsShorterThan}{wxtimespanisshorterthan}


\membersection{Formatting time spans}\label{timespanformatting}

\helpref{Format}{wxtimespanformat}

%%%%%%%%%%%%%%%%%%%%%%%%%%%%%%%%%%%%%%%%%%%%%%%%%%%%%%%%%%%%%%%%%%%%%%%%%%%%%%%
% Start of member function part                                               %
%%%%%%%%%%%%%%%%%%%%%%%%%%%%%%%%%%%%%%%%%%%%%%%%%%%%%%%%%%%%%%%%%%%%%%%%%%%%%%%

\helponly{\insertatlevel{2}{
    \wxheading{Members}
}}


\membersection{wxTimeSpan::Abs}\label{wxtimespanabs}

\constfunc{wxTimeSpan}{Abs}{\void}

Returns the absolute value of the timespan: does not modify the
object.


\membersection{wxTimeSpan::Add}\label{wxtimespanadd}

\constfunc{wxTimeSpan}{Add}{\param{const wxTimeSpan\& }{diff}}

\func{wxTimeSpan\&}{Add}{\param{const wxTimeSpan\& }{diff}}

\func{wxTimeSpan\&}{operator$+=$}{\param{const wxTimeSpan\&}{diff}}

Returns the sum of two timespans.


\membersection{wxTimeSpan::Days}\label{wxtimespandays}

\func{static wxTimespan}{Days}{\param{long }{days}}

Returns the timespan for the given number of days.


\membersection{wxTimeSpan::Day}\label{wxtimespanday}

\func{static wxTimespan}{Day}{\void}

Returns the timespan for one day.


\membersection{wxTimeSpan::Format}\label{wxtimespanformat}

\func{wxString}{Format}{\param{const wxChar * }{format = wxDefaultTimeSpanFormat}}

Returns the string containing the formatted representation of the time span.
The following format specifiers are allowed after \%:

\twocolwidtha{5cm}%
\begin{twocollist}\itemsep=0pt
\twocolitem{H}{number of {\bf H}ours}
\twocolitem{M}{number of {\bf M}inutes}
\twocolitem{S}{number of {\bf S}econds}
\twocolitem{l}{number of mi{\bf l}liseconds}
\twocolitem{D}{number of {\bf D}ays}
\twocolitem{E}{number of w{\bf E}eks}
\twocolitem{\%}{the percent character}
\end{twocollist}

Note that, for example, the number of hours in the description above is not
well defined: it can be either the total number of hours (for example, for a
time span of $50$ hours this would be $50$) or just the hour part of the time
span, which would be $2$ in this case as $50$ hours is equal to $2$ days and
$2$ hours.

wxTimeSpan resolves this ambiguity in the following way: if there had been,
indeed, the {\tt \%D} format specified preceding the {\tt \%H}, then it is
interpreted as $2$. Otherwise, it is $50$.

The same applies to all other format specifiers: if they follow a specifier of
larger unit, only the rest part is taken, otherwise the full value is used.


\membersection{wxTimeSpan::GetDays}\label{wxtimespangetdays}

\constfunc{int}{GetDays}{\void}

Returns the difference in number of days.


\membersection{wxTimeSpan::GetHours}\label{wxtimespangethours}

\constfunc{int}{GetHours}{\void}

Returns the difference in number of hours.


\membersection{wxTimeSpan::GetMilliseconds}\label{wxtimespangetmilliseconds}

\constfunc{wxLongLong}{GetMilliseconds}{\void}

Returns the difference in number of milliseconds.


\membersection{wxTimeSpan::GetMinutes}\label{wxtimespangetminutes}

\constfunc{int}{GetMinutes}{\void}

Returns the difference in number of minutes.


\membersection{wxTimeSpan::GetSeconds}\label{wxtimespangetseconds}

\constfunc{wxLongLong}{GetSeconds}{\void}

Returns the difference in number of seconds.


\membersection{wxTimeSpan::GetValue}\label{wxtimespangetvalue}

\constfunc{wxLongLong}{GetValue}{\void}

Returns the internal representation of timespan.


\membersection{wxTimeSpan::GetWeeks}\label{wxtimespangetweeks}

\constfunc{int}{GetWeeks}{\void}

Returns the difference in number of weeks.


\membersection{wxTimeSpan::Hours}\label{wxtimespanhours}

\func{static wxTimespan}{Hours}{\param{long }{hours}}

Returns the timespan for the given number of hours.


\membersection{wxTimeSpan::Hour}\label{wxtimespanhour}

\func{static wxTimespan}{Hour}{\void}

Returns the timespan for one hour.


\membersection{wxTimeSpan::IsEqualTo}\label{wxtimespanisequalto}

\constfunc{bool}{IsEqualTo}{\param{const wxTimeSpan\& }{ts}}

Returns {\tt true} if two timespans are equal.


\membersection{wxTimeSpan::IsLongerThan}\label{wxtimespanislongerthan}

\constfunc{bool}{IsLongerThan}{\param{const wxTimeSpan\& }{ts}}

Compares two timespans: works with the absolute values, i.e. -2
hours is longer than 1 hour. Also, it will return {\tt false} if
the timespans are equal in absolute value.


\membersection{wxTimeSpan::IsNegative}\label{wxtimespanisnegative}

\constfunc{bool}{IsNegative}{\void}

Returns {\tt true} if the timespan is negative.


\membersection{wxTimeSpan::IsNull}\label{wxtimespanisnull}

\constfunc{bool}{IsNull}{\void}

Returns {\tt true} if the timespan is empty.


\membersection{wxTimeSpan::IsPositive}\label{wxtimespanispositive}

\constfunc{bool}{IsPositive}{\void}

Returns {\tt true} if the timespan is positive.


\membersection{wxTimeSpan::IsShorterThan}\label{wxtimespanisshorterthan}

\constfunc{bool}{IsShorterThan}{\param{const wxTimeSpan\& }{ts}}

Compares two timespans: works with the absolute values, i.e. 1
hour is shorter than -2 hours. Also, it will return {\tt false} if
the timespans are equal in absolute value.


\membersection{wxTimeSpan::Minutes}\label{wxtimespanminutes}

\func{static wxTimespan}{Minutes}{\param{long }{min}}

Returns the timespan for the given number of minutes.


\membersection{wxTimeSpan::Minute}\label{wxtimespanminute}

\func{static wxTimespan}{Minute}{\void}

Returns the timespan for one minute.


\membersection{wxTimeSpan::Multiply}\label{wxtimespanmultiply}

\constfunc{wxTimeSpan}{Multiply}{\param{int }{n}}

\func{wxTimeSpan\&}{Multiply}{\param{int }{n}}

\func{wxTimeSpan\&}{operator$*=$}{\param{int }{n}}

Multiplies timespan by a scalar.


\membersection{wxTimeSpan::Negate}\label{wxtimespannegate}

\constfunc{wxTimeSpan}{Negate}{\void}

Returns timespan with inverted sign.


\membersection{wxTimeSpan::Neg}\label{wxtimespanneg}

\func{wxTimeSpan\&}{Neg}{\void}

\func{wxTimeSpan\&}{operator$-$}{\void}

Negate the value of the timespan.


\membersection{wxTimeSpan::Milliseconds}\label{wxtimespanmilliseconds}

\func{static wxTimespan}{Milliseconds}{\param{long }{ms}}

Returns the timespan for the given number of milliseconds.


\membersection{wxTimeSpan::Millisecond}\label{wxtimespanmillisecond}

\func{static wxTimespan}{Millisecond}{\void}

Returns the timespan for one millisecond.


\membersection{wxTimeSpan::Seconds}\label{wxtimespanseconds}

\func{static wxTimespan}{Seconds}{\param{long }{sec}}

Returns the timespan for the given number of seconds.


\membersection{wxTimeSpan::Second}\label{wxtimespansecond}

\func{static wxTimespan}{Second}{\void}

Returns the timespan for one second.


\membersection{wxTimeSpan::Subtract}\label{wxtimespansubtract}

\constfunc{wxTimeSpan}{Subtract}{\param{const wxTimeSpan\&}{diff}}

\func{wxTimeSpan\&}{Subtract}{\param{const wxTimeSpan\& }{diff}}

\func{wxTimeSpan\&}{operator$-=$}{\param{const wxTimeSpan\&}{diff}}

Returns the difference of two timespans.


\membersection{wxTimeSpan::Weeks}\label{wxtimespanweeks}

\func{static wxTimespan}{Weeks}{\param{long }{weeks}}

Returns the timespan for the given number of weeks.


\membersection{wxTimeSpan::Week}\label{wxtimespanweek}

\func{static wxTimespan}{Week}{\void}

Returns the timespan for one week.


\membersection{wxTimeSpan::wxTimeSpan}\label{wxtimespanctor}

\func{}{wxTimeSpan}{\void}

Default constructor, constructs a zero timespan.

\func{}{wxTimeSpan}{\param{long }{hours}, \param{long }{min}, \param{long }{sec}, \param{long }{msec}}

Constructs timespan from separate values for each component, with the date
set to 0. Hours are not restricted to 0..24 range, neither are
minutes, seconds or milliseconds.

