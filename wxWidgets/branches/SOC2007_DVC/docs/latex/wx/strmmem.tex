% -----------------------------------------------------------------------------
% wxMemoryInputStream
% -----------------------------------------------------------------------------
\section{\class{wxMemoryInputStream}}\label{wxmeminputstream}

\wxheading{Derived from}

\helpref{wxInputStream}{wxinputstream}

\wxheading{Include files}

<wx/mstream.h>

\wxheading{Library}

\helpref{wxBase}{librarieslist}

\wxheading{See also}

\helpref{wxStreamBuffer}{wxstreambuffer}, \helpref{wxMemoryOutputStream}{wxmemoutputstream}

% ----------
% Members
% ----------
\latexignore{\rtfignore{\wxheading{Members}}}

\membersection{wxMemoryInputStream::wxMemoryInputStream}\label{wxmemoryinputstreamctor}

\func{}{wxMemoryInputStream}{\param{const char *}{ data}, \param{size\_t}{ len}}

Initializes a new read-only memory stream which will use the specified buffer
{\it data} of length {\it len}. The stream does not take ownership of the 
buffer, i.e. the buffer will not be deleted in its destructor.

\func{}{wxMemoryInputStream}{\param{const wxMemoryOutputStream&}{ stream}}

Creates a new read-only memory stream, initializing it with the
data from the given output stream \arg{stream}.

\func{}{wxMemoryInputStream}{\param{wxInputStream\&}{ stream}, \param{wxFileOffset}{ len = wxInvalidOffset}}

Creates a new read-only memory stream, initializing it with the
data from the given input stream \arg{stream}.

The \arg{len} argument specifies the amount of data to read from
the \arg{stream}. Setting it to {\it wxInvalidOffset} means that
the \arg{stream} is to be read entirely (i.e. till the EOF is reached).


\membersection{wxMemoryInputStream::\destruct{wxMemoryInputStream}}\label{wxmemoryinputstreamdtor}

\func{}{\destruct{wxMemoryInputStream}}{\void}

Destructor.

\membersection{wxMemoryInputStream::GetInputStreamBuffer}\label{wxmemoryinputstreamgetistrmbuf}

\constfunc{wxStreamBuffer *}{GetInputStreamBuffer}{\void}

Returns the pointer to the stream object used as an internal buffer
for that stream.

% -----------------------------------------------------------------------------
% wxMemoryOutputStream
% -----------------------------------------------------------------------------
\section{\class{wxMemoryOutputStream}}\label{wxmemoutputstream}

\wxheading{Derived from}

\helpref{wxOutputStream}{wxoutputstream}

\wxheading{Include files}

<wx/mstream.h>

\wxheading{Library}

\helpref{wxBase}{librarieslist}

\wxheading{See also}

\helpref{wxStreamBuffer}{wxstreambuffer}

% ----------
% Members
% ----------
\latexignore{\rtfignore{\wxheading{Members}}}

\membersection{wxMemoryOutputStream::wxMemoryOutputStream}\label{wxmemoryoutputstreamctor}

\func{}{wxMemoryOutputStream}{\param{char *}{ data = NULL}, \param{size\_t}{ length = 0}}

If {\it data} is NULL, then it will initialize a new empty buffer which will
grow if required.

\wxheading{Warning}

If the buffer is created, it will be destroyed at the destruction of the
stream.

\membersection{wxMemoryOutputStream::\destruct{wxMemoryOutputStream}}\label{wxmemoryoutputstreamdtor}

\func{}{\destruct{wxMemoryOutputStream}}{\void}

Destructor.

\membersection{wxMemoryOutputStream::CopyTo}\label{wxmemoryoutputstreamcopyto}

\constfunc{size\_t}{CopyTo}{\param{char *}{buffer}, \param{size\_t }{len}}

CopyTo allowed you to transfer data from the internal buffer of
wxMemoryOutputStream to an external buffer. {\it len} specifies the size of
the buffer.

\wxheading{Returned value}

CopyTo returns the number of bytes copied to the buffer. Generally it is either
len or the size of the stream buffer.

\membersection{wxMemoryOutputStream::GetOutputStreamBuffer}\label{wxmemoryoutputstreamgetostrmbuf}

\constfunc{wxStreamBuffer *}{GetOutputStreamBuffer}{\void}

Returns the pointer to the stream object used as an internal buffer
for that stream.

