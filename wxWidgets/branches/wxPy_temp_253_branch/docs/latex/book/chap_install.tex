\chapter{Installing wxWindows}\label{chapinstall}
\pagenumbering{arabic}%
\setheader{{\it CHAPTER \thechapter: INSTALLING wxWINDOWS}}{}{}{}{}{{\it CHAPTER \thechapter: INSTALLING wxWINDOWS}}%
\setfooter{\thepage}{}{}{}{}{\thepage}%

CONTENTS: Installing wxWindows (and what tools to use).

Installing wxWindows isn't too hard. Each platform has a different method, so we'll look
at each major platform in turn.

\section{Unix: GTK+ and Motif}\label{installunix}

\subsection{The simplest case}

If you are compile wxWindows on Linux for the first time and don't like to read 
install instructions, just do this in the base directory:

\begin{verbatim}
  ./configure --with-gtk
  make
  su <type root password>
  make install
  ldconfig
  exit
\end{verbatim}

This is using the GTK+ port. If using the Motif port, type --with-motif instead of --with-gtk.

Afterwards you can continue with:

\begin{verbatim}
  make
  su <type root password>
  make install
  ldconfig
  exit
\end{verbatim}

If you want to remove wxWindows on Unix you can do this:

\begin{verbatim}
  su <type root password>
  make uninstall
  ldconfig
  exit
\end{verbatim}

\subsection{The expert case}

If you want to do some more serious cross-platform programming with wxWindows, 
such as for GTK and Motif, you can now build two complete libraries and use 
them concurrently. For this end, you have to create a directory for each build 
of wxWindows - you may also want to create different versions of wxWindows
and test them concurrently. Most typically, this would be a version configured 
with --enable-debug\_flag and one without. Note, that only one build can currently 
be installed, so you'd have to use a local version of the library for that purpose.
For building three versions (one for GTK+, one for Motif and a debug GTK+ version) you'd do this:

\begin{verbatim}
  md buildmotif
  cd buildmotif
 ../configure --with-motif
  make
  cd ..

  md buildgtk
  cd buildgtk
  ../configure --with-gtk
  make
  cd ..

  md buildgtkd
  cd buildgtkd
  ../configure --with-gtk --enable-debug_flag
  make
  cd ..
\end{verbatim}

\subsection{The simplest errors}

\begin{itemize}\itemsep=0pt
\item Configure reports, that you don't have GTK 1.2 installed although you are 
very sure you have. Well, you have installed it, but you also have another 
version of the GTK installed, which you may need to remove including other 
versions of glib (and its headers). Also, look for the PATH variable and check 
if it includes the path to the correct gtk-config! The check your LDPATH if it 
points to the correct library. There is no way to compile wxGTK if configure 
doesn't pass this test as all this test does is compile and link a GTK program.
\item You get errors during compilation: The reason is that you probably have a
broken compiler.  GCC 2.8 and earlier versions and egcs are likely to cause
problems due to incomplete support for C++ and optimisation bugs.  Best to use
GCC 2.95 or later.
\item You get immediate segfault when starting any sample or application: This is
either due to having compiled the library with different flags or options than
your program - typically you might have the \_\_WXDEBUG\_\_ option set for the
library but not for your program - or due to using a compiler with optimisation
bugs.
\end{itemize}

\subsection{The simplest program}

Now create your super-application myfoo.app and compile anywhere with:

\begin{verbatim}
  g++ myfoo.cpp `wx-config --libs --cxxflags` -o myfoo
\end{verbatim}

\wxheading{General}

The Unix variants of wxWindows use GNU configure. If you have problems with your 
make use GNU make instead.

If you have general problems with installation, visit Robert Roebling's homepage at 

\begin{verbatim}
  http://wesley.informatik.uni-freiburg.de/~wxxt
\end{verbatim}
  
for the latest information. If you still don't have any success, please send a bug 
report to one of the mailing lists.

\wxheading{Libraries needed}

wxWindows/GTK requires the GTK+ library to be installed on your system. It has to 
be a stable version, preferably version 1.2.3.

You can get the newest version of the GTK+ from the GTK homepage at:

\begin{verbatim}
  http://www.gtk.org
\end{verbatim}
  
wxWindows/Gtk requires a thread library and X libraries known to work with threads. 
This is the case on all commercial Unix-Variants and all Linux-Versions that are 
based on glibc 2 except RedHat 5.0 which is broken in many aspects. As of writing 
this, these Linux distributions have correct glibc 2 support:

\begin{itemize}\itemsep=0pt
\item RedHat 5.1
\item Debian 2.0 and 3.0
\item Stampede
\item DLD 6.0
\item SuSE 6.0
\end{itemize}
 
You can disable thread support by running 

\begin{verbatim}
./configure --disable-threads
make
su <type root password>
make install
ldconfig
exit
\end{verbatim}
  
\subsection{Building wxGTK on OS/2}

Please send comments and question about the OS/2 installation
to Andrea Venturoli <a.ventu@flashnet.it> and patches to
the wxWindows mailing list.

You'll need OS/2 Warp (4.00FP#6), X-Free86/2 (3.3.3 or newer), 
GTK+ (1.2.5 or newer), emx (0.9d fix 1), flex (2.5.4), yacc (1.8), 
korn shell (5.2.13), Autoconf (2.13),  GNU file utilities (3.6), 
GNU text utilities (1.3), GNU shell utilites (1.12), m4 (1.4), 
sed (2.05), grep (2.0), Awk (3.0.3), GNU Make (3.76.1).

Open an OS/2 prompt and switch to the directory above.
First set some global environment variables we need:

\begin{verbatim}
  SET CXXFLAGS=-Zmtd -D__ST_MT_ERRNO__
  SET CFLAGS=-Zmtd -D__ST_MT_ERRNO__
  SET OSTYPE=OS2X              
  SET COMSPEC=sh
  \end{verbatim}

Notice you can choose whatever you want, if you don't like OS2X.

Now, run autoconf in the main directory and in the samples, demos
and utils subdirectory. This will generate the OS/2 specific
versions of the configure scripts. Now run

\begin{verbatim}
    configure --with-gtk
\end{verbatim}

as described above.

If you have pthreads library installed, but have a gtk version
which does not yet support threading, you need to explicitly
disable threading by using the option --disable-threads.

Note that configure assumes your flex will generate files named
"lexyy.c", not "lex.yy.c". If you have a version which does
generate "lex.yy.c", you need to manually change the generated
makefile.

\subsection{Building wxGTK on SGI}

Using the SGI native compilers, it is recommended that you
also set CFLAGS and CXXFLAGS before running configure. These 
should be set to:

\begin{verbatim}
  CFLAGS="-mips3 -n32" 
  CXXFLAGS="-mips3 -n32"
\end{verbatim}

This is essential if you want to use the resultant binaries 
on any other machine than the one it was compiled on. If you 
have a 64-bit machine (Octane) you should also do this to ensure 
you don't accidently build the libraries as 64bit (which is 
untested).

The SGI native compiler support has only been tested on Irix 6.5.

\subsection{Create your configuration}

Usage:

\begin{verbatim}
	./configure options
\end{verbatim}

If you want to use system's C and C++ compiler,
set environment variables CC and CCC as

\begin{verbatim}
  setenv CC cc
  setenv CCC CC
  ./configure options
\end{verbatim}

to see all the options please use:

\begin{verbatim}
  ./configure --help
\end{verbatim}

The basic philosophy is that if you want to use different
configurations, like a debug and a release version, 
or use the same source tree on different systems,
you have only to change the environment variable OSTYPE.
(Sadly this variable is not set by default on some systems
in some shells - on SGI's for example). So you will have to 
set it there. This variable HAS to be set before starting 
configure, so that it knows which system it tries to 
configure for.

Configure (and sometimes make) will complain if the system variable OSTYPE has 
not been defined.

\subsubsection{General options}

Given below are the commands to change the default behaviour,
i.e. if it says "--disable-threads" it means that threads
are enabled by default.

Normally, you won't have to choose a toolkit, because when
you download wxGTK, it will default to --with-gtk etc. But
if you use all of our CVS repository you have to choose a 
toolkit. You must do this by running configure with either of:

\begin{verbatim}
    --without-gtk            Don't use the GIMP ToolKit (GTK)
	
	--with-motif             Use either Motif or Lesstif
	                         Configure will look for both. 
\end{verbatim}

The following options handle the kind of library you want to build.

\begin{verbatim}
	--disable-threads       Compile without thread support.

	--disable-shared        Do not create shared libraries.

	--enable-static         Create static libraries.

	--disable-optimise	    Do not optimise the code. Can
	                        sometimes be useful for debugging
                            and is required on some architectures
                            such as Sun with gcc 2.8.X which
                            and otherwise produce segvs.

	--enable-profile        Add profiling info to the object 
				            files. Currently broken, I think.
				
	--enable-no_rtti        Enable compilation without creation of
	                        C++ RTTI information in object files. 
                            This will speed-up compilation and reduce 
                            binary size.
				
	--enable-no_exceptions  Enable compilation without creation of
	                        C++ exception information in object files. 
                            This will speed-up compilation and reduce 
                            binary size. Also fewer crashes during the
                            actual compilation...
				
	--enable-no_deps        Enable compilation without creation of
	                        dependency information.
				
        --enable-permissive     Enable compilation without checking for strict
                                ANSI conformance.  Useful to prevent the build
                                dying with errors as soon as you compile with
                                Solaris' ANSI-defying headers.
				
	--enable-mem_tracing    Add built-in memory tracing.
				
	--enable-dmalloc        Use the dmalloc memory debugger.
	                        Read more at www.letters.com/dmalloc/
				
	--enable-debug_info	    Add debug info to object files and
	                        executables for use with debuggers
				            such as gdb (or its many frontends).

	--enable-debug_flag	    Define __DEBUG__ and __WXDEBUG__ when
	                        compiling. This enable wxWindows' very
                            useful internal debugging tricks (such
                            as automatically reporting illegal calls)
                            to work. Note that program and library
                            must be compiled with the same debug 
                            options.
\end{verbatim}

\subsubsection{Feature Options}

When producing an executable that is linked statically with wxGTK
you'll be surprised at its immense size. This can sometimes be
drastically reduced by removing features from wxWindows that 
are not used in your program. The most relevant such features
are

\begin{verbatim}
	--with-odbc             Enables ODBC code. This is disabled
                            by default because iODBC is under the
                            L-GPL license.
	
	--without-libpng	    Disables PNG image format code.
	
	--without-libjpeg	    Disables JPEG image format code.
	
	--without-libtiff	    Disables TIFF image format code.
    
	--disable-pnm		    Disables PNM image format code.
	
	--disable-gif		    Disables GIF image format code.
	
	--disable-pcx		    Disables PCX image format code.
	
    --disable-resources     Disables the use of *.wxr type
	                        resources.
		
	--disable-threads       Disables threads. Will also
	                        disable sockets.

	--disable-sockets       Disables sockets.

	--disable-dnd           Disables Drag'n'Drop.
	
	--disable-clipboard     Disables Clipboard.
	
	--disable-serial        Disables object instance serialisation.
	
	--disable-streams       Disables the wxStream classes.
	
	--disable-file          Disables the wxFile class.
	
	--disable-textfile      Disables the wxTextFile class.
	
	--disable-intl          Disables the internationalisation.
	
	--disable-validators    Disables validators.
	
	--disable-accel         Disables accel.
\end{verbatim}
	
Apart from disabling certain features you can very often "strip"
the program of its debugging information resulting in a significant
reduction in size.

\subsubsection{Compiling}

The following must be done in the base directory (e.g. ~/wxGTK
or ~/wxWin or whatever)

Now the makefiles are created (by configure) and you can compile 
the library by typing:

\begin{verbatim}
	make
\end{verbatim}

make yourself some coffee, as it will take some time. On an old
386SX possibly two weeks. During compilation, you'll get a few 
warning messages depending in your compiler.

If you want to be more selective, you can change into a specific
directory and type "make" there.

Then you may install the library and it's header files under
/usr/local/include/wx and /usr/local/lib respectively. You
have to log in as root (i.e. run "su" and enter the root
password) and type

\begin{verbatim}
        make install	
\end{verbatim}

You can remove any traces of wxWindows by typing

\begin{verbatim}
        make uninstall
\end{verbatim}
	
If you want to save disk space by removing unnecessary
object-files:

\begin{verbatim}
	    make clean
\end{verbatim}

in the various directories will do the work for you.

\subsubsection{Creating a new Project}

1\ket The first way uses the installed libraries and header files
automatically using wx-config

\begin{verbatim}
g++ myfoo.cpp `wx-config --cxxflags --libs` -o myfoo
\end{verbatim}

Using this way, a make file for the minimal sample would look
like this

\begin{verbatim}
CXX = g++

minimal: minimal.o
    $(CXX) -o minimal minimal.o `wx-config --libs` 

minimal.o: minimal.cpp mondrian.xpm
    $(CXX) `wx-config --cxxflags` -c minimal.cpp -o minimal.o

clean: 
	rm -f *.o minimal
\end{verbatim}

This is certain to become the standard way unless we decide
to stick to tmake.

2\ket The other way creates a project within the source code 
directories of wxWindows. For this endeavour, you'll need
GNU autoconf version 2.14 and add an entry to your Makefile.in
to the bottom of the configure.in script and run autoconf
and configure before you can type make.

\section{Windows}\label{installwindows}


\section{Mac}\label{installmac}

We don't have information about Mac installation at this time.

