\section{\class{wxRegion}}\label{wxregion}

A wxRegion represents a simple or complex region on a device context or window. It uses
reference counting, so copying and assignment operations are fast.

\wxheading{Derived from}

\helpref{wxGDIObject}{wxgdiobject}\\
\helpref{wxObject}{wxobject}

\wxheading{Include files}

<wx/region.h>

\wxheading{See also}

\helpref{wxRegionIterator}{wxregioniterator}

\latexignore{\rtfignore{\wxheading{Members}}}

\membersection{wxRegion::wxRegion}\label{wxregionconstr}

\func{}{wxRegion}{\void}

Default constructor.

\func{}{wxRegion}{\param{wxCoord}{ x}, \param{wxCoord}{ y}, \param{wxCoord}{ width}, \param{wxCoord}{ height}}

Constructs a rectangular region with the given position and size.

\func{}{wxRegion}{\param{const wxPoint\&}{ topLeft}, \param{const wxPoint\&}{ bottomRight}}

Constructs a rectangular region from the top left point and the bottom right point.

\func{}{wxRegion}{\param{const wxRect\&}{ rect}}

Constructs a rectangular region a wxRect object.

\func{}{wxRegion}{\param{const wxRegion\&}{ region}}

Constructs a region by copying another region.

\func{}{wxRegion}{\param{size\_t}{ n}, \param{const wxPoint }{*points}, \param{int }{fillStyle = wxWINDING\_RULE}}

Constructs a region corresponding to the polygon made of {\it n} points in the
provided array. {\it fillStyle} parameter may have values
{\tt wxWINDING\_RULE} or {\tt wxODDEVEN\_RULE}.

{\bf NB:} This constructor is only implemented for Win32 and GTK+ wxWindows ports.

\func{}{wxRegion}{\param{const wxBitmap&}{ bmp},
                  \param{const wxColour&}{ transColour = wxNullColour},
		  \param{int}{ tolerance = 0}}

Constructs a region using the non-transparent pixels of a bitmap.  See
\helpref{Union}{wxregionunion} for more details.


\membersection{wxRegion::\destruct{wxRegion}}

\func{}{\destruct{wxRegion}}{\void}

Destructor.

\membersection{wxRegion::Clear}\label{wxregionclear}

\func{void}{Clear}{\void}

Clears the current region.

\membersection{wxRegion::Contains}\label{wxregioncontains}

\constfunc{wxRegionContain}{Contains}{\param{long\& }{x}, \param{long\& }{y}}

Returns a value indicating whether the given point is contained within the region.

\constfunc{wxRegionContain}{Contains}{\param{const wxPoint\&}{ pt}}

Returns a value indicating whether the given point is contained within the region.

\constfunc{wxRegionContain}{Contains}{\param{long\& }{x}, \param{long\& }{y}, \param{long\& }{width}, \param{long\& }{height}}

Returns a value indicating whether the given rectangle is contained within the region.

\constfunc{wxRegionContain}{Contains}{\param{const wxRect\& }{rect}}

Returns a value indicating whether the given rectangle is contained within the region.

\wxheading{Return value}

The return value is one of wxOutRegion, wxPartRegion and wxInRegion.

On Windows, only wxOutRegion and wxInRegion are returned; a value wxInRegion then indicates that
all or some part of the region is contained in this region.

\membersection{wxRegion::ConvertToBitmap}\label{wxregionconverttobitmap}

\constfunc{wxBitmap}{ConvertToBitmap}{}

Convert the region to a black and white bitmap with the black pixels
being inside the region.

\membersection{wxRegion::GetBox}\label{wxregiongetbox}

\constfunc{void}{GetBox}{\param{long\& }{x}, \param{long\& }{y}, \param{long\& }{width}, \param{long\& }{height}}

Returns the outer bounds of the region.

\constfunc{wxRect}{GetBox}{\void}

Returns the outer bounds of the region.

\membersection{wxRegion::Intersect}\label{wxregionintersect}

\func{bool}{Intersect}{\param{wxCoord}{ x}, \param{wxCoord}{ y}, \param{wxCoord}{ width}, \param{wxCoord}{ height}}

Finds the intersection of this region and another, rectangular region, specified using position and size.

\func{bool}{Intersect}{\param{const wxRect\&}{ rect}}

Finds the intersection of this region and another, rectangular region.

\func{bool}{Intersect}{\param{const wxRegion\&}{ region}}

Finds the intersection of this region and another region.

\wxheading{Return value}

{\tt TRUE} if successful, {\tt FALSE} otherwise.

\wxheading{Remarks}

Creates the intersection of the two regions, that is, the parts which are in both regions. The result
is stored in this region.

\membersection{wxRegion::IsEmpty}\label{wxregionisempty}

\constfunc{bool}{IsEmpty}{\void}

Returns {\tt TRUE} if the region is empty, {\tt FALSE} otherwise.

\membersection{wxRegion::Subtract}\label{wxregionsubstract}

\func{bool}{Subtract}{\param{const wxRect\&}{ rect}}

Subtracts a rectangular region from this region.

\func{bool}{Subtract}{\param{const wxRegion\&}{ region}}

Subtracts a region from this region.

\wxheading{Return value}

{\tt TRUE} if successful, {\tt FALSE} otherwise.

\wxheading{Remarks}

This operation combines the parts of 'this' region that are not part of the second region.
The result is stored in this region.

\membersection{wxRegion::Offset}\label{wxregionoffset}

\func{bool}{Offset}{\param{wxCoord}{ x}, \param{wxCoord}{ y}}

Moves the region by the specified offsets in horizontal and vertical
directions.

\wxheading{Return value}

{\tt TRUE} if successful, {\tt FALSE} otherwise (the region is unchanged then).

\membersection{wxRegion::Union}\label{wxregionunion}

\func{bool}{Union}{\param{wxCoord}{ x}, \param{wxCoord}{ y}, \param{wxCoord}{ width}, \param{wxCoord}{ height}}

Finds the union of this region and another, rectangular region, specified using position and size.

\func{bool}{Union}{\param{const wxRect\&}{ rect}}

Finds the union of this region and another, rectangular region.

\func{bool}{Union}{\param{const wxRegion\&}{ region}}

Finds the union of this region and another region.

\func{bool}{Union}{\param{const wxBitmap&}{ bmp},
                  \param{const wxColour&}{ transColour = wxNullColour},
		  \param{int}{ tolerance = 0}}

Finds the union of this region and the the non-transparent pixels of a
bitmap.  If the bitmap has a mask then it will be used, otherwise the
colour to be treated as transparent may be specified, along with an
optional colour tolerance value.

\wxheading{Return value}

{\tt TRUE} if successful, {\tt FALSE} otherwise.

\wxheading{Remarks}

This operation creates a region that combines all of this region and the second region.
The result is stored in this region.

\membersection{wxRegion::Xor}\label{wxregionxor}

\func{bool}{Xor}{\param{wxCoord}{ x}, \param{wxCoord}{ y}, \param{wxCoord}{ width}, \param{wxCoord}{ height}}

Finds the Xor of this region and another, rectangular region, specified using position and size.

\func{bool}{Xor}{\param{const wxRect\&}{ rect}}

Finds the Xor of this region and another, rectangular region.

\func{bool}{Xor}{\param{const wxRegion\&}{ region}}

Finds the Xor of this region and another region.

\wxheading{Return value}

{\tt TRUE} if successful, {\tt FALSE} otherwise.

\wxheading{Remarks}

This operation creates a region that combines all of this region and the second region, except
for any overlapping areas. The result is stored in this region.

\membersection{wxRegion::operator $=$}\label{wxregionassign}

\func{void}{operator $=$}{\param{const wxRegion\&}{ region}}

Copies {\it region} by reference counting.

\section{\class{wxRegionIterator}}\label{wxregioniterator}

This class is used to iterate through the rectangles in a region,
typically when examining the damaged regions of a window within an OnPaint call.

To use it, construct an iterator object on the stack and loop through the
regions, testing the object and incrementing the iterator at the end of the loop.

See \helpref{wxPaintEvent}{wxpaintevent} for an example of use.

\wxheading{Derived from}

\helpref{wxObject}{wxobject}

\wxheading{Include files}

<wx/region.h>

\wxheading{See also}

\helpref{wxPaintEvent}{wxpaintevent}

\latexignore{\rtfignore{\wxheading{Members}}}

\membersection{wxRegionIterator::wxRegionIterator}

\func{}{wxRegionIterator}{\void}

Default constructor.

\func{}{wxRegionIterator}{\param{const wxRegion\&}{ region}}

Creates an iterator object given a region.

\membersection{wxRegionIterator::GetX}

\constfunc{wxCoord}{GetX}{\void}

Returns the x value for the current region.

\membersection{wxRegionIterator::GetY}

\constfunc{wxCoord}{GetY}{\void}

Returns the y value for the current region.

\membersection{wxRegionIterator::GetW}

\constfunc{wxCoord}{GetW}{\void}

An alias for GetWidth.

\membersection{wxRegionIterator::GetWidth}

\constfunc{wxCoord}{GetWidth}{\void}

Returns the width value for the current region.

\membersection{wxRegionIterator::GetH}

\constfunc{wxCoord}{GetH}{\void}

An alias for GetHeight.

\membersection{wxRegionIterator::GetHeight}

\constfunc{wxCoord}{GetWidth}{\void}

Returns the width value for the current region.

\membersection{wxRegionIterator::GetRect}

\constfunc{wxRect}{GetRect}{\void}

Returns the current rectangle.

\membersection{wxRegionIterator::HaveRects}

\constfunc{bool}{HaveRects}{\void}

Returns {\tt TRUE} if there are still some rectangles; otherwise returns {\tt FALSE}.

\membersection{wxRegionIterator::Reset}

\func{void}{Reset}{\void}

Resets the iterator to the beginning of the rectangles.

\func{void}{Reset}{\param{const wxRegion\&}{ region}}

Resets the iterator to the given region.

\membersection{wxRegionIterator::operator $++$}

\func{void}{operator $++$}{\void}

Increment operator. Increments the iterator to the next region.

\pythonnote{A wxPython alias for this operator is called {\tt Next}.}

\membersection{wxRegionIterator::operator bool}

\constfunc{}{operator bool}{\void}

Returns {\tt TRUE} if there are still some rectangles; otherwise returns {\tt FALSE}.

You can use this to test the iterator object as if it were of type bool.

