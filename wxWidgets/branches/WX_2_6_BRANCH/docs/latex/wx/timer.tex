\section{\class{wxTimer}}\label{wxtimer}

The wxTimer class allows you to execute code at specified intervals. Its
precision is platform-dependent, but in general will not be better than 1ms nor
worse than 1s.

There are three different ways to use this class:

\begin{enumerate}
\item You may derive a new class from wxTimer and override the 
\helpref{Notify}{wxtimernotify} member to perform the required action.
\item Or you may redirect the notifications to any 
\helpref{wxEvtHandler}{wxevthandler} derived object by using the non default
constructor or \helpref{SetOwner}{wxtimersetowner}. Then use the {\tt EVT\_TIMER} 
macro to connect it to the event handler which will receive 
\helpref{wxTimerEvent}{wxtimerevent} notifications.
\item Or you may use a derived class and the {\tt EVT\_TIMER} 
macro to connect it to an event handler defined in the derived class.
If the default constructor is used, the timer object will be its
own owner object, since it is derived from wxEvtHandler.
\end{enumerate}

In any case, you must start the timer with \helpref{Start}{wxtimerstart} 
after constructing it before it actually starts sending notifications. It can
be stopped later with \helpref{Stop}{wxtimerstop}.

{\bf Note:} A timer can only be used from the main thread.

\wxheading{Derived from}

\helpref{wxEvtHandler}{wxevthandler}

\wxheading{Include files}

<wx/timer.h>

\wxheading{See also}

\helpref{::wxStartTimer}{wxstarttimer}, \helpref{::wxGetElapsedTime}{wxgetelapsedtime}, \helpref{wxStopWatch}{wxstopwatch}

\latexignore{\rtfignore{\wxheading{Members}}}

\membersection{wxTimer::wxTimer}\label{wxtimerwxtimer}

\func{}{wxTimer}{\void}

Default constructor. If you use it to construct the object and don't call 
\helpref{SetOwner}{wxtimersetowner} later, you must override 
\helpref{Notify}{wxtimernotify} method to process the notifications.

\func{}{wxTimer}{\param{wxEvtHandler *}{owner}, \param{int }{id = -1}}

Creates a timer and associates it with {\it owner}. Please see 
\helpref{SetOwner}{wxtimersetowner} for the description of parameters.

\membersection{wxTimer::\destruct{wxTimer}}\label{wxtimerdtor}

\func{}{\destruct{wxTimer}}{\void}

Destructor. Stops the timer if it is running.

\membersection{wxTimer::GetInterval}\label{wxtimergetinterval}

\constfunc{int}{GetInterval}{\void}

Returns the current interval for the timer (in milliseconds).

\membersection{wxTimer::IsOneShot}\label{wxtimerisoneshot}

\constfunc{bool}{IsOneShot}{\void}

Returns {\tt true} if the timer is one shot, i.e.\ if it will stop after firing the
first notification automatically.

\membersection{wxTimer::IsRunning}\label{wxtimerisrunning}

\constfunc{bool}{IsRunning}{\void}

Returns {\tt true} if the timer is running, {\tt false} if it is stopped.

\membersection{wxTimer::Notify}\label{wxtimernotify}

\func{void}{Notify}{\void}

This member should be overridden by the user if the default constructor was
used and \helpref{SetOwner}{wxtimersetowner} wasn't called.

Perform whatever action which is to be taken periodically here.

\membersection{wxTimer::SetOwner}\label{wxtimersetowner}

\func{void}{SetOwner}{\param{wxEvtHandler *}{owner}, \param{int }{id = -1}}

Associates the timer with the given {\it owner}\/ object. When the timer is
running, the owner will receive \helpref{timer events}{wxtimerevent} with
id equal to {\it id}\/ specified here.

\membersection{wxTimer::Start}\label{wxtimerstart}

\func{bool}{Start}{\param{int }{milliseconds = -1}, \param{bool }{oneShot = {\tt false}}}

(Re)starts the timer. If {\it milliseconds}\/ parameter is -1 (value by default),
the previous value is used. Returns {\tt false} if the timer could not be started,
{\tt true} otherwise (in MS Windows timers are a limited resource).

If {\it oneShot}\/ is {\tt false} (the default), the \helpref{Notify}{wxtimernotify} 
function will be called repeatedly until the timer is stopped. If {\tt true},
it will be called only once and the timer will stop automatically. To make your
code more readable you may also use the following symbolic constants:

\twocolwidtha{5cm}
\begin{twocollist}\itemsep=0pt
\twocolitem{wxTIMER\_CONTINUOUS}{Start a normal, continuously running, timer}
\twocolitem{wxTIMER\_ONE\_SHOT}{Start a one shot timer}
\end{twocollist}

If the timer was already running, it will be stopped by this method before
restarting it.

\membersection{wxTimer::Stop}\label{wxtimerstop}

\func{void}{Stop}{\void}

Stops the timer.

\section{\class{wxTimerEvent}}\label{wxtimerevent}

wxTimerEvent object is passed to the event handler of timer events.

For example:

\begin{verbatim}
class MyFrame : public wxFrame
{
public:
    ...
    void OnTimer(wxTimerEvent& event);

private:
    wxTimer m_timer;
};

BEGIN_EVENT_TABLE(MyFrame, wxFrame)
    EVT_TIMER(TIMER_ID, MyFrame::OnTimer)
END_EVENT_TABLE()

MyFrame::MyFrame()
       : m_timer(this, TIMER_ID)
{
    m_timer.Start(1000);    // 1 second interval
}

void MyFrame::OnTimer(wxTimerEvent& event)
{
    // do whatever you want to do every second here
}

\end{verbatim}

\wxheading{Include files}

<wx/timer.h>

\wxheading{See also}

\helpref{wxTimer}{wxtimer}

\latexignore{\rtfignore{\wxheading{Members}}}

\membersection{wxTimerEvent::GetInterval}\label{wxtimereventgetinterval}

\constfunc{int}{GetInterval}{\void}

Returns the interval of the timer which generated this event.

