\section{\class{wxAcceleratorEntry}}\label{wxacceleratorentry}

An object used by an application wishing to create an \helpref{accelerator table}{wxacceleratortable}.

\wxheading{Derived from}

None

\wxheading{Include files}

<wx/accel.h>

\wxheading{See also}

\helpref{wxAcceleratorTable}{wxacceleratortable}, \helpref{wxWindow::SetAcceleratorTable}{wxwindowsetacceleratortable}

\latexignore{\rtfignore{\wxheading{Members}}}

\membersection{wxAcceleratorEntry::wxAcceleratorEntry}\label{wxacceleratorentryctor}

\func{}{wxAcceleratorEntry}{\void}

Default constructor.

\func{}{wxAcceleratorEntry}{\param{int}{ flags}, \param{int}{ keyCode}, \param{int}{ cmd}}

Constructor.

\wxheading{Parameters}

\docparam{flags}{One of wxACCEL\_ALT, wxACCEL\_SHIFT, wxACCEL\_CTRL and wxACCEL\_NORMAL. Indicates
which modifier key is held down.}

\docparam{keyCode}{The keycode to be detected. See \helpref{Keycodes}{keycodes} for a full list of keycodes.}

\docparam{cmd}{The menu or control command identifier.}

\membersection{wxAcceleratorEntry::GetCommand}\label{wxacceleratorentrygetcommand}

\constfunc{int}{GetCommand}{\void}

Returns the command identifier for the accelerator table entry.

\membersection{wxAcceleratorEntry::GetFlags}\label{wxacceleratorentrygetflags}

\constfunc{int}{GetFlags}{\void}

Returns the flags for the accelerator table entry.

\membersection{wxAcceleratorEntry::GetKeyCode}\label{wxacceleratorentrygetkeycode}

\constfunc{int}{GetKeyCode}{\void}

Returns the keycode for the accelerator table entry.

\membersection{wxAcceleratorEntry::Set}\label{wxacceleratorentryset}

\func{void}{Set}{\param{int}{ flags}, \param{int}{ keyCode}, \param{int}{ cmd}}

Sets the accelerator entry parameters.

\wxheading{Parameters}

\docparam{flags}{One of wxACCEL\_ALT, wxACCEL\_SHIFT, wxACCEL\_CTRL and wxACCEL\_NORMAL. Indicates
which modifier key is held down.}

\docparam{keyCode}{The keycode to be detected. See \helpref{Keycodes}{keycodes} for a full list of keycodes.}

\docparam{cmd}{The menu or control command identifier.}

\section{\class{wxAcceleratorTable}}\label{wxacceleratortable}

An accelerator table allows the application to specify a table of keyboard shortcuts for
menus or other commands. On Windows, menu or button commands are supported; on GTK,
only menu commands are supported.

The object {\bf wxNullAcceleratorTable} is defined to be a table with no data, and is the
initial accelerator table for a window.

\wxheading{Derived from}

\helpref{wxObject}{wxobject}

\wxheading{Include files}

<wx/accel.h>

\wxheading{Example}

{\small%
\begin{verbatim}
  wxAcceleratorEntry entries[4];
  entries[0].Set(wxACCEL_CTRL,  (int) 'N',     ID_NEW_WINDOW);
  entries[1].Set(wxACCEL_CTRL,  (int) 'X',     wxID_EXIT);
  entries[2].Set(wxACCEL_SHIFT, (int) 'A',     ID_ABOUT);
  entries[3].Set(wxACCEL_NORMAL,  WXK_DELETE,    wxID_CUT);
  wxAcceleratorTable accel(4, entries);
  frame->SetAcceleratorTable(accel);
\end{verbatim}
}%

\wxheading{Remarks}

An accelerator takes precedence over normal processing and can be a convenient way to program some event handling.
For example, you can use an accelerator table to enable a dialog with a multi-line text control to
accept CTRL-Enter as meaning `OK' (but not in GTK+ at present).

\wxheading{See also}

\helpref{wxAcceleratorEntry}{wxacceleratorentry}, \helpref{wxWindow::SetAcceleratorTable}{wxwindowsetacceleratortable}

\latexignore{\rtfignore{\wxheading{Members}}}

\membersection{wxAcceleratorTable::wxAcceleratorTable}\label{wxacceleratortablector}

\func{}{wxAcceleratorTable}{\void}

Default constructor.

\func{}{wxAcceleratorTable}{\param{const wxAcceleratorTable\& }{bitmap}}

Copy constructor.

\func{}{wxAcceleratorTable}{\param{int}{ n}, \param{wxAcceleratorEntry}{ entries[]}}

Creates from an array of \helpref{wxAcceleratorEntry}{wxacceleratorentry} objects.

\func{}{wxAcceleratorTable}{\param{const wxString\&}{ resource}}

Loads the accelerator table from a Windows resource (Windows only).

\wxheading{Parameters}

\docparam{n}{Number of accelerator entries.}

\docparam{entries}{The array of entries.}

\docparam{resource}{Name of a Windows accelerator.}

\pythonnote{The wxPython constructor accepts a list of
wxAcceleratorEntry objects, or 3-tuples consisting of flags, keyCode,
and cmd values like you would construct wxAcceleratorEntry objects with.}

\perlnote{The wxPerl constructor accepts a list of either
  Wx::AcceleratorEntry objects or references to 3-element arrays
  ( flags, keyCode, cmd ), like the parameters of Wx::AcceleratorEntry::new.}

\membersection{wxAcceleratorTable::\destruct{wxAcceleratorTable}}\label{wxacceleratortabledtor}

\func{}{\destruct{wxAcceleratorTable}}{\void}

Destroys the wxAcceleratorTable object.

\membersection{wxAcceleratorTable::Ok}\label{wxacceleratortableok}

\constfunc{bool}{Ok}{\void}

Returns true if the accelerator table is valid.

\membersection{wxAcceleratorTable::operator $=$}\label{wxacceleratortableassign}

\func{wxAcceleratorTable\& }{operator $=$}{\param{const wxAcceleratorTable\& }{accel}}

Assignment operator. This operator does not copy any data, but instead
passes a pointer to the data in {\it accel} and increments a reference
counter. It is a fast operation.

\wxheading{Parameters}

\docparam{accel}{Accelerator table to assign.}

\wxheading{Return value}

Returns reference to this object.

\membersection{wxAcceleratorTable::operator $==$}\label{wxacceleratortableequal}

\func{bool}{operator $==$}{\param{const wxAcceleratorTable\& }{accel}}

Equality operator. This operator tests whether the internal data pointers are
equal (a fast test).

\wxheading{Parameters}

\docparam{accel}{Accelerator table to compare with}

\wxheading{Return value}

Returns true if the accelerator tables were effectively equal, false otherwise.

\membersection{wxAcceleratorTable::operator $!=$}\label{wxacceleratortablenotequal}

\func{bool}{operator $!=$}{\param{const wxAcceleratorTable\& }{accel}}

Inequality operator. This operator tests whether the internal data pointers are
unequal (a fast test).

\wxheading{Parameters}

\docparam{accel}{Accelerator table to compare with}

\wxheading{Return value}

Returns true if the accelerator tables were unequal, false otherwise.


