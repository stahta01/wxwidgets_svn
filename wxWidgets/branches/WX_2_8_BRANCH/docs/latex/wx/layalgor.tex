\section{\class{wxLayoutAlgorithm}}\label{wxlayoutalgorithm}

wxLayoutAlgorithm implements layout of subwindows in MDI or SDI frames.
It sends a wxCalculateLayoutEvent event
to children of the frame, asking them for information about
their size. For MDI parent frames, the algorithm allocates
the remaining space to the MDI client window (which contains the MDI child frames).
For SDI (normal) frames, a 'main' window is specified as taking up the
remaining space.

Because the event system is used, this technique can be applied to any windows,
which are not necessarily 'aware' of the layout classes (no virtual functions
in wxWindow refer to wxLayoutAlgorithm or its events). However, you
may wish to use \helpref{wxSashLayoutWindow}{wxsashlayoutwindow} for your subwindows
since this class provides handlers for the required events, and accessors
to specify the desired size of the window. The sash behaviour in the base class
can be used, optionally, to make the windows user-resizable.

wxLayoutAlgorithm is typically used in IDE (integrated development environment) applications,
where there are several resizable windows in addition to the MDI client window, or
other primary editing window. Resizable windows might include toolbars, a project
window, and a window for displaying error and warning messages.

When a window receives an OnCalculateLayout event, it should call SetRect in
the given event object, to be the old supplied rectangle minus whatever space the
window takes up. It should also set its own size accordingly.
wxSashLayoutWindow::OnCalculateLayout generates an OnQueryLayoutInfo event
which it sends to itself to determine the orientation, alignment and size of the window,
which it gets from internal member variables set by the application.

The algorithm works by starting off with a rectangle equal to the whole frame client area.
It iterates through the frame children, generating OnCalculateLayout events which subtract
the window size and return the remaining rectangle for the next window to process. It
is assumed (by wxSashLayoutWindow::OnCalculateLayout) that a window stretches the full dimension
of the frame client, according to the orientation it specifies. For example, a horizontal window
will stretch the full width of the remaining portion of the frame client area.
In the other orientation, the window will be fixed to whatever size was specified by
OnQueryLayoutInfo. An alignment setting will make the window 'stick' to the left, top, right or
bottom of the remaining client area. This scheme implies that order of window creation is important.
Say you wish to have an extra toolbar at the top of the frame, a project window to the left of
the MDI client window, and an output window above the status bar. You should therefore create
the windows in this order: toolbar, output window, project window. This ensures that the toolbar and
output window take up space at the top and bottom, and then the remaining height in-between is used for
the project window.

wxLayoutAlgorithm is quite independent of the way in which
OnCalculateLayout chooses to interpret a window's size and alignment. Therefore you
could implement a different window class with a new OnCalculateLayout event handler,
that has a more sophisticated way of laying out the windows. It might allow
specification of whether stretching occurs in the specified orientation, for example,
rather than always assuming stretching. (This could, and probably should, be added to the existing
implementation).

{\it Note:} wxLayoutAlgorithm has nothing to do with wxLayoutConstraints. It is an alternative
way of specifying layouts for which the normal constraint system is unsuitable.

\wxheading{Derived from}

\helpref{wxObject}{wxobject}

\wxheading{Include files}

<wx/laywin.h>

\wxheading{Event handling}

The algorithm object does not respond to events, but itself generates the
following events in order to calculate window sizes.

\twocolwidtha{7cm}%
\begin{twocollist}\itemsep=0pt
\twocolitem{{\bf EVT\_QUERY\_LAYOUT\_INFO(func)}}{Process a wxEVT\_QUERY\_LAYOUT\_INFO event,
to get size, orientation and alignment from a window. See \helpref{wxQueryLayoutInfoEvent}{wxquerylayoutinfoevent}.}
\twocolitem{{\bf EVT\_CALCULATE\_LAYOUT(func)}}{Process a wxEVT\_CALCULATE\_LAYOUT event,
which asks the window to take a 'bite' out of a rectangle provided by the algorithm.
See \helpref{wxCalculateLayoutEvent}{wxcalculatelayoutevent}.}
\end{twocollist}

\wxheading{Data types}

{\small
\begin{verbatim}
enum wxLayoutOrientation {
    wxLAYOUT_HORIZONTAL,
    wxLAYOUT_VERTICAL
};

enum wxLayoutAlignment {
    wxLAYOUT_NONE,
    wxLAYOUT_TOP,
    wxLAYOUT_LEFT,
    wxLAYOUT_RIGHT,
    wxLAYOUT_BOTTOM,
};
\end{verbatim}
}

\wxheading{See also}

\helpref{wxSashEvent}{wxsashevent}, \helpref{wxSashLayoutWindow}{wxsashlayoutwindow}, \helpref{Event handling overview}{eventhandlingoverview}

\helpref{wxCalculateLayoutEvent}{wxcalculatelayoutevent},\rtfsp
\helpref{wxQueryLayoutInfoEvent}{wxquerylayoutinfoevent},\rtfsp
\helpref{wxSashLayoutWindow}{wxsashlayoutwindow},\rtfsp
\helpref{wxSashWindow}{wxsashwindow}

\latexignore{\rtfignore{\wxheading{Members}}}

\membersection{wxLayoutAlgorithm::wxLayoutAlgorithm}\label{wxlayoutalgorithmctor}

\func{}{wxLayoutAlgorithm}{\void}

Default constructor.

\membersection{wxLayoutAlgorithm::\destruct{wxLayoutAlgorithm}}\label{wxlayoutalgorithmdtor}

\func{}{\destruct{wxLayoutAlgorithm}}{\void}

Destructor.

\membersection{wxLayoutAlgorithm::LayoutFrame}\label{wxlayoutalgorithmlayoutframe}

\constfunc{bool}{LayoutFrame}{\param{wxFrame* }{frame}, \param{wxWindow*}{ mainWindow = NULL}}

Lays out the children of a normal frame. {\it mainWindow} is set to occupy the remaining space.

This function simply calls \helpref{wxLayoutAlgorithm::LayoutWindow}{wxlayoutalgorithmlayoutwindow}.

\membersection{wxLayoutAlgorithm::LayoutMDIFrame}\label{wxlayoutalgorithmlayoutmdiframe}

\constfunc{bool}{LayoutMDIFrame}{\param{wxMDIParentFrame* }{frame}, \param{wxRect*}{ rect = NULL}}

Lays out the children of an MDI parent frame. If {\it rect} is non-NULL, the
given rectangle will be used as a starting point instead of the frame's client area.

The MDI client window is set to occupy the remaining space.

\membersection{wxLayoutAlgorithm::LayoutWindow}\label{wxlayoutalgorithmlayoutwindow}

\constfunc{bool}{LayoutWindow}{\param{wxWindow* }{parent}, \param{wxWindow*}{ mainWindow = NULL}}

Lays out the children of a normal frame or other window.

{\it mainWindow} is set to occupy the remaining space. If this is not specified, then
the last window that responds to a calculate layout event in query mode will get the remaining space
(that is, a non-query OnCalculateLayout event will not be sent to this window and the window will be set
to the remaining size).

