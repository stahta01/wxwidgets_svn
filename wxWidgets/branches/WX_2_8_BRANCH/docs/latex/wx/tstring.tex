\section{wxString overview}\label{wxstringoverview}

Classes: \helpref{wxString}{wxstring}, \helpref{wxArrayString}{wxarraystring}, \helpref{wxStringTokenizer}{wxstringtokenizer}

\subsection{Introduction}\label{introductiontowxstring}

wxString is a class which represents a character string of arbitrary length (limited by 
{\it MAX\_INT} which is usually 2147483647 on 32 bit machines) and containing
arbitrary characters. The ASCII NUL character is allowed, but be aware that
in the current string implementation some methods might not work correctly
in this case.

wxString works with both ASCII (traditional, 7 or 8 bit, characters) as well as
Unicode (wide characters) strings.

This class has all the standard operations you can expect to find in a string class:
dynamic memory management (string extends to accommodate new characters),
construction from other strings, C strings and characters, assignment operators,
access to individual characters, string concatenation and comparison, substring
extraction, case conversion, trimming and padding (with spaces), searching and
replacing and both C-like \helpref{Printf()}{wxstringprintf} and stream-like
insertion functions as well as much more - see \helpref{wxString}{wxstring} 
for a list of all functions.

\subsection{Comparison of wxString to other string classes}\label{otherstringclasses}

The advantages of using a special string class instead of working directly with
C strings are so obvious that there is a huge number of such classes available.
The most important advantage is the need to always
remember to allocate/free memory for C strings; working with fixed size buffers almost
inevitably leads to buffer overflows. At last, C++ has a standard string class
(std::string). So why the need for wxString?

There are several advantages:

\begin{enumerate}\itemsep=0pt
\item {\bf Efficiency} This class was made to be as efficient as possible: both
in terms of size (each wxString objects takes exactly the same space as a {\it
char *} pointer, sing \helpref{reference counting}{wxstringrefcount}) and speed.
It also provides performance \helpref{statistics gathering code}{wxstringtuning} 
which may be enabled to fine tune the memory allocation strategy for your
particular application - and the gain might be quite big.
\item {\bf Compatibility} This class tries to combine almost full compatibility
with the old wxWidgets 1.xx wxString class, some reminiscence to MFC CString
class and 90\% of the functionality of std::string class.
\item {\bf Rich set of functions} Some of the functions present in wxString are
very useful but don't exist in most of other string classes: for example, 
\helpref{AfterFirst}{wxstringafterfirst}, 
\helpref{BeforeLast}{wxstringbeforelast}, \helpref{operator<<}{wxstringoperatorout} 
or \helpref{Printf}{wxstringprintf}. Of course, all the standard string
operations are supported as well.
\item {\bf Unicode} wxString is Unicode friendly: it allows to easily convert
to and from ANSI and Unicode strings in any build mode (see the 
\helpref{Unicode overview}{unicode} for more details) and maps to either
{\tt string} or {\tt wstring} transparently depending on the current mode.
\item {\bf Used by wxWidgets} And, of course, this class is used everywhere
inside wxWidgets so there is no performance loss which would result from
conversions of objects of any other string class (including std::string) to
wxString internally by wxWidgets.
\end{enumerate}

However, there are several problems as well. The most important one is probably
that there are often several functions to do exactly the same thing: for
example, to get the length of the string either one of 
length(), \helpref{Len()}{wxstringlen} or 
\helpref{Length()}{wxstringlength} may be used. The first function, as almost
all the other functions in lowercase, is std::string compatible. The second one
is "native" wxString version and the last one is wxWidgets 1.xx way. So the
question is: which one is better to use? And the answer is that:

{\bf The usage of std::string compatible functions is strongly advised!} It will
both make your code more familiar to other C++ programmers (who are supposed to
have knowledge of std::string but not of wxString), let you reuse the same code
in both wxWidgets and other programs (by just typedefing wxString as std::string
when used outside wxWidgets) and by staying compatible with future versions of
wxWidgets which will probably start using std::string sooner or later too.

In the situations where there is no corresponding std::string function, please
try to use the new wxString methods and not the old wxWidgets 1.xx variants
which are deprecated and may disappear in future versions.

\subsection{Some advice about using wxString}\label{wxstringadvices}

Probably the main trap with using this class is the implicit conversion operator to 
{\it const char *}. It is advised that you use \helpref{c\_str()}{wxstringcstr}
instead to clearly indicate when the conversion is done. Specifically, the
danger of this implicit conversion may be seen in the following code fragment:

\begin{verbatim}
// this function converts the input string to uppercase, output it to the screen
// and returns the result
const char *SayHELLO(const wxString& input)
{
    wxString output = input.Upper();

    printf("Hello, %s!\n", output);

    return output;
}
\end{verbatim}

There are two nasty bugs in these three lines. First of them is in the call to the 
{\it printf()} function. Although the implicit conversion to C strings is applied
automatically by the compiler in the case of

\begin{verbatim}
    puts(output);
\end{verbatim}

because the argument of {\it puts()} is known to be of the type {\it const char *},
this is {\bf not} done for {\it printf()} which is a function with variable
number of arguments (and whose arguments are of unknown types). So this call may
do anything at all (including displaying the correct string on screen), although
the most likely result is a program crash. The solution is to use 
\helpref{c\_str()}{wxstringcstr}: just replace this line with

\begin{verbatim}
    printf("Hello, %s!\n", output.c_str());
\end{verbatim}

The second bug is that returning {\it output} doesn't work. The implicit cast is
used again, so the code compiles, but as it returns a pointer to a buffer
belonging to a local variable which is deleted as soon as the function exits,
its contents is totally arbitrary. The solution to this problem is also easy:
just make the function return wxString instead of a C string.

This leads us to the following general advice: all functions taking string
arguments should take {\it const wxString\&} (this makes assignment to the
strings inside the function faster because of 
\helpref{reference counting}{wxstringrefcount}) and all functions returning
strings should return {\it wxString} - this makes it safe to return local
variables.

\subsection{Other string related functions and classes}\label{relatedtostring}

As most programs use character strings, the standard C library provides quite
a few functions to work with them. Unfortunately, some of them have rather
counter-intuitive behaviour (like strncpy() which doesn't always terminate the
resulting string with a NULL) and are in general not very safe (passing NULL
to them will probably lead to program crash). Moreover, some very useful
functions are not standard at all. This is why in addition to all wxString
functions, there are also a few global string functions which try to correct
these problems: \helpref{wxIsEmpty()}{wxisempty} verifies whether the string
is empty (returning {\tt true} for {\tt NULL} pointers), 
\helpref{wxStrlen()}{wxstrlen} also handles NULLs correctly and returns 0 for
them and \helpref{wxStricmp()}{wxstricmp} is just a platform-independent
version of case-insensitive string comparison function known either as
stricmp() or strcasecmp() on different platforms.

The {\tt <wx/string.h>} header also defines \helpref{wxSnprintf}{wxsnprintf} 
and \helpref{wxVsnprintf}{wxvsnprintf} functions which should be used instead
of the inherently dangerous standard {\tt sprintf()} and which use {\tt
snprintf()} instead which does buffer size checks whenever possible. Of
course, you may also use \helpref{wxString::Printf}{wxstringprintf} which is
also safe.

There is another class which might be useful when working with wxString: 
\helpref{wxStringTokenizer}{wxstringtokenizer}. It is helpful when a string must
be broken into tokens and replaces the standard C library {\it
strtok()} function.

And the very last string-related class is \helpref{wxArrayString}{wxarraystring}: it
is just a version of the "template" dynamic array class which is specialized to work
with strings. Please note that this class is specially optimized (using its
knowledge of the internal structure of wxString) for storing strings and so it is
vastly better from a performance point of view than a wxObjectArray of wxStrings.

\subsection{Reference counting and why you shouldn't care about it}\label{wxstringrefcount}

All considerations for wxObject-derived \helpref{reference counted}{trefcount} objects
are valid also for wxString, even if it does not derive from wxObject.

Probably the unique case when you might want to think about reference
counting is when a string character is taken from a string which is not a
constant (or a constant reference). In this case, due to C++ rules, the
"read-only" {\it operator[]} (which is the same as 
\helpref{GetChar()}{wxstringgetchar}) cannot be chosen and the "read/write" 
{\it operator[]} (the same as 
\helpref{GetWritableChar()}{wxstringgetwritablechar}) is used instead. As the
call to this operator may modify the string, its data is unshared (COW is done)
and so if the string was really shared there is some performance loss (both in
terms of speed and memory consumption). In the rare cases when this may be
important, you might prefer using \helpref{GetChar()}{wxstringgetchar} instead
of the array subscript operator for this reasons. Please note that 
\helpref{at()}{wxstringat} method has the same problem as the subscript operator in
this situation and so using it is not really better. Also note that if all
string arguments to your functions are passed as {\it const wxString\&} (see the
section \helpref{Some advice}{wxstringadvices}) this situation will almost
never arise because for constant references the correct operator is called automatically.

\subsection{Tuning wxString for your application}\label{wxstringtuning}

\normalbox{{\bf Note:} this section is strictly about performance issues and is
absolutely not necessary to read for using wxString class. Please skip it unless
you feel familiar with profilers and relative tools. If you do read it, please
also read the preceding section about 
\helpref{reference counting}{wxstringrefcount}.}

For the performance reasons wxString doesn't allocate exactly the amount of
memory needed for each string. Instead, it adds a small amount of space to each
allocated block which allows it to not reallocate memory (a relatively
expensive operation) too often as when, for example, a string is constructed by
subsequently adding one character at a time to it, as for example in:

\begin{verbatim}
// delete all vowels from the string
wxString DeleteAllVowels(const wxString& original)
{
    wxString result;

    size_t len = original.length();
    for ( size_t n = 0; n < len; n++ )
    {
        if ( strchr("aeuio", tolower(original[n])) == NULL )
            result += original[n];
    }

    return result;
}
\end{verbatim}

This is quite a common situation and not allocating extra memory at all would
lead to very bad performance in this case because there would be as many memory
(re)allocations as there are consonants in the original string. Allocating too
much extra memory would help to improve the speed in this situation, but due to
a great number of wxString objects typically used in a program would also
increase the memory consumption too much.

The very best solution in precisely this case would be to use 
\helpref{Alloc()}{wxstringalloc} function to preallocate, for example, len bytes
from the beginning - this will lead to exactly one memory allocation being
performed (because the result is at most as long as the original string).

However, using Alloc() is tedious and so wxString tries to do its best. The
default algorithm assumes that memory allocation is done in granularity of at
least 16 bytes (which is the case on almost all of wide-spread platforms) and so
nothing is lost if the amount of memory to allocate is rounded up to the next
multiple of 16. Like this, no memory is lost and 15 iterations from 16 in the
example above won't allocate memory but use the already allocated pool.

The default approach is quite conservative. Allocating more memory may bring
important performance benefits for programs using (relatively) few very long
strings. The amount of memory allocated is configured by the setting of {\it
EXTRA\_ALLOC} in the file string.cpp during compilation (be sure to understand
why its default value is what it is before modifying it!). You may try setting
it to greater amount (say twice nLen) or to 0 (to see performance degradation
which will follow) and analyse the impact of it on your program. If you do it,
you will probably find it helpful to also define WXSTRING\_STATISTICS symbol
which tells the wxString class to collect performance statistics and to show
them on stderr on program termination. This will show you the average length of
strings your program manipulates, their average initial length and also the
percent of times when memory wasn't reallocated when string concatenation was
done but the already preallocated memory was used (this value should be about
98\% for the default allocation policy, if it is less than 90\% you should
really consider fine tuning wxString for your application).

It goes without saying that a profiler should be used to measure the precise
difference the change to EXTRA\_ALLOC makes to your program.

