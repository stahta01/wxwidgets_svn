%%%%%%%%%%%%%%%%%%%%%%%%%%%%%%%%%%%%%%%%%%%%%%%%%%%%%%%%%%%%%%%%%%%%%%%%%%%%%%%
%% Name:        xmlnode.tex
%% Purpose:     wxXmlDocument documentation
%% Author:      Francesco Montorsi
%% Created:     2006-04-18
%% RCS-ID:      $Id$
%% Copyright:   (c) 2006 Francesco Montorsi
%% License:     wxWindows license
%%%%%%%%%%%%%%%%%%%%%%%%%%%%%%%%%%%%%%%%%%%%%%%%%%%%%%%%%%%%%%%%%%%%%%%%%%%%%%%

\section{\class{wxXmlDocument}}\label{wxxmldocument}

This class holds XML data/document as parsed by XML parser in the root node.

wxXmlDocument internally uses the expat library which comes with wxWidgets to parse the given stream.

A simple example of using XML classes is:

\begin{verbatim}
wxXmlDocument doc;
if (!doc.Load(wxT("myfile.xml")))
    return false;

// start processing the XML file
if (doc.GetRoot()->GetName() != wxT("myroot-node"))
    return false;

wxXmlNode *child = doc.GetRoot()->GetChildren();
while (child) {

    if (child->GetName() == wxT("tag1")) {

        // process text enclosed by <tag1></tag1>
        wxString content = child->GetNodeContent();

        ...


        // process properties of <tag1>
        wxString propvalue1 = 
            child->GetPropVal(wxT("prop1"), 
                              wxT("default-value"));
        wxString propvalue2 = 
            child->GetPropVal(wxT("prop2"), 
                              wxT("default-value"));

        ...

    } else if (child->GetName() == wxT("tag2")) {

        // process tag2 ...
    }

    child = child->GetNext();
}
\end{verbatim}

{\bf Note:} if you want to preserve the original formatting of the loaded file including whitespaces
and indentation, you need to turn off whitespace-only textnode removal and automatic indentation:

\begin{verbatim}
wxXmlDocument doc;
doc.Load(wxT("myfile.xml"), wxT("UTF-8"), wxXMLDOC_KEEP_WHITESPACE_NODES);

// myfile2.xml will be indentic to myfile.xml saving it this way:
doc.Save(wxT("myfile2.xml"), wxXML_NO_INDENTATION);
\end{verbatim}

Using default parameters, you will get a reformatted document which in general is different from
the original loaded content:

\begin{verbatim}
wxXmlDocument doc;
doc.Load(wxT("myfile.xml"));
doc.Save(wxT("myfile2.xml"));  // myfile2.xml != myfile.xml
\end{verbatim}


\wxheading{Derived from}

\helpref{wxObject}{wxobject}

\wxheading{Include files}

<wx/xml/xml.h>

\wxheading{See also}

\helpref{wxXmlNode}{wxxmlnode}, \helpref{wxXmlProperty}{wxxmlproperty}


\latexignore{\rtfignore{\wxheading{Members}}}



\membersection{wxXmlDocument::wxXmlDocument}\label{wxxmldocumentwxxmldocument}

\func{}{wxXmlDocument}{\void}


\func{}{wxXmlDocument}{\param{const wxString\& }{filename}, \param{const wxString\& }{encoding = wxT("UTF-8")}, \param{int }{flags = wxXMLDOC\_NONE}}

Loads the given {\it filename} using the given encoding. See \helpref{Load}{wxxmldocumentload}.

\func{}{wxXmlDocument}{\param{wxInputStream\& }{stream}, \param{const wxString\& }{encoding = wxT("UTF-8")}, \param{int }{flags = wxXMLDOC\_NONE}}

Loads the XML document from given stream using the given encoding. See \helpref{Load}{wxxmldocumentload}.

\func{}{wxXmlDocument}{\param{const wxXmlDocument\& }{doc}}

Copy constructor. Deep copies all the XML tree of the given document.


\membersection{wxXmlDocument::\destruct{wxXmlDocument}}\label{wxxmldocumentdtor}

\func{}{\destruct{wxXmlDocument}}{\void}

Virtual destructor. Frees the document root node.



\membersection{wxXmlDocument::DetachRoot}\label{wxxmldocumentdetachroot}

\func{wxXmlNode*}{DetachRoot}{\void}

Detaches the document root node and returns it. The document root node will be set to \NULL
and thus \helpref{IsOk}{wxxmldocumentisok} will return \false after calling this function.

Note that the caller is reponsible for deleting the returned node in order to avoid memory leaks.



\membersection{wxXmlDocument::GetEncoding}\label{wxxmldocumentgetencoding}

\constfunc{wxString}{GetEncoding}{\void}

Returns encoding of in-memory representation of the document
(same as passed to \helpref{Load}{wxxmldocumentload} or constructor, defaults to UTF-8).

NB: this is meaningless in Unicode build where data are stored as {\tt wchar\_t*}.



\membersection{wxXmlDocument::GetFileEncoding}\label{wxxmldocumentgetfileencoding}

\constfunc{wxString}{GetFileEncoding}{\void}

Returns encoding of document (may be empty).

Note: this is the encoding original file was saved in, {\bf not} the
encoding of in-memory representation!



\membersection{wxXmlDocument::GetRoot}\label{wxxmldocumentgetroot}

\constfunc{wxXmlNode*}{GetRoot}{\void}

Returns the root node of the document.



\membersection{wxXmlDocument::GetVersion}\label{wxxmldocumentgetversion}

\constfunc{wxString}{GetVersion}{\void}

Returns the version of document.
This is the value in the {\tt <?xml version="1.0"?>} header of the XML document.
If the version property was not explicitely given in the header, this function
returns an empty string.



\membersection{wxXmlDocument::IsOk}\label{wxxmldocumentisok}

\constfunc{bool}{IsOk}{\void}

Returns \true if the document has been loaded successfully.



\membersection{wxXmlDocument::Load}\label{wxxmldocumentload}

\func{bool}{Load}{\param{const wxString\& }{filename}, \param{const wxString\& }{encoding = wxT("UTF-8")}, \param{int }{flags = wxXMLDOC\_NONE}}

Parses {\it filename} as an xml document and loads its data.

If {\tt flags} does not contain {\tt wxXMLDOC\_KEEP\_WHITESPACE\_NODES}, then, while loading, all nodes of
type {\tt wxXML\_TEXT\_NODE} (see \helpref{wxXmlNode}{wxxmlnode}) are automatically skipped if they
contain whitespaces only.
The removal of these nodes makes the load process slightly faster and requires less memory however
makes impossible to recreate exactly the loaded text with a \helpref{Save}{wxxmldocumentsave} call later.
Read the initial description of this class for more info.

Returns \true on success, \false otherwise.

\func{bool}{Load}{\param{wxInputStream\& }{stream}, \param{const wxString\& }{encoding = wxT("UTF-8")}, \param{int }{flags = wxXMLDOC\_NONE}}

Like above but takes the data from given input stream.


\membersection{wxXmlDocument::Save}\label{wxxmldocumentsave}

\constfunc{bool}{Save}{\param{const wxString\& }{filename}, \param{int }{indentstep = 1}}

Saves XML tree creating a file named with given string.

If {\tt indentstep} is greater than or equal to zero, then, while saving, an automatic indentation
is added with steps composed by {\tt indentstep} spaces.
If {\tt indentstep} is {\tt wxXML\_NO\_INDENTATION}, then, automatic indentation is turned off.

\constfunc{bool}{Save}{\param{wxOutputStream\& }{stream}, \param{int }{indentstep = 1}}

Saves XML tree in the given output stream. See other overload for a description of {\tt indentstep}.


\membersection{wxXmlDocument::SetEncoding}\label{wxxmldocumentsetencoding}

\func{void}{SetEncoding}{\param{const wxString\& }{enc}}

Sets the enconding of the document.


\membersection{wxXmlDocument::SetFileEncoding}\label{wxxmldocumentsetfileencoding}

\func{void}{SetFileEncoding}{\param{const wxString\& }{encoding}}

Sets the enconding of the file which will be used to save the document.


\membersection{wxXmlDocument::SetRoot}\label{wxxmldocumentsetroot}

\func{void}{SetRoot}{\param{wxXmlNode* }{node}}

Sets the root node of this document. Deletes previous root node.
Use \helpref{DetachRoot}{wxxmldocumentdetachroot} and then 
\helpref{SetRoot}{wxxmldocumentsetroot} if you want
to replace the root node without deleting the old document tree.


\membersection{wxXmlDocument::SetVersion}\label{wxxmldocumentsetversion}

\func{void}{SetVersion}{\param{const wxString\& }{version}}

Sets the version of the XML file which will be used to save the document.


\membersection{wxXmlDocument::operator=}\label{wxxmldocumentoperatorassign}

\func{wxXmlDocument\& operator}{operator=}{\param{const wxXmlDocument\& }{doc}}

Deep copies the given document.

