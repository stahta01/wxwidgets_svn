%%%%%%%%%%%%%%%%%%%%%%%%%%%%%%%%%%%%%%%%%%%%%%%%%%%%%%%%%%%%%%%%%%%%%%%%%%%%%
%% Name:        varvscrollhelper.tex
%% Purpose:     wxVarVScrollHelper Documentation
%% Author:      Bryan Petty
%% Modified by: 
%% Created:     2007-04-04
%% RCS-ID:      $Id$
%% Copyright:   (c) 2007 wxWidgets Team
%% License:     wxWindows Licence
%%%%%%%%%%%%%%%%%%%%%%%%%%%%%%%%%%%%%%%%%%%%%%%%%%%%%%%%%%%%%%%%%%%%%%%%%%%%%

\section{\class{wxVarVScrollHelper}}\label{wxvarvscrollhelper}

This class provides functions wrapping the 
\helpref{wxVarScrollHelperBase}{wxvarscrollhelperbase} class, targeted for
vertical-specific scrolling using \helpref{wxVScrolledWindow}{wxvscrolledwindow}.

Like wxVarScrollHelperBase, this class is mostly only useful to those classes
built into wxWidgets deriving from here, and this documentation is mostly
only provided for referencing those functions provided. You will likely want
to derive your window from wxVScrolledWindow rather than from here directly.

\wxheading{Derived from}

\helpref{wxVarScrollHelperBase}{wxvarscrollhelperbase}

\wxheading{Include files}

<wx/vscroll.h>

\wxheading{See also}

\helpref{wxHScrolledWindow}{wxhscrolledwindow},
\rtfsp\helpref{wxHVScrolledWindow}{wxhvscrolledwindow},
\rtfsp\helpref{wxVScrolledWindow}{wxvscrolledwindow}

\latexignore{\rtfignore{\wxheading{Members}}}


\membersection{wxVarVScrollHelper::wxVarVScrollHelper}\label{wxvarvscrollhelperwxvarvscrollhelper}

\func{}{wxVarVScrollHelper}{\param{wxWindow* }{winToScroll}}

Constructor taking the target window to be scrolled by this helper class.
This will attach scroll event handlers to the target window to catch and
handle scroll events appropriately.


\membersection{wxVarVScrollHelper::EstimateTotalHeight}\label{wxvarvscrollhelperestimatetotalheight}

\constfunc{virtual wxCoord}{EstimateTotalHeight}{\void}

This class forwards calls from
\helpref{wxVarScrollHelperBase::EstimateTotalSize}{wxvarscrollhelperbaseestimatetotalsize}
to this function so derived classes can override either just the height or
the width estimation, or just estimate both differently if desired in any
\helpref{wxHVScrolledWindow}{wxhvscrolledwindow} derived class.

Please note that this function will not be called if {\tt EstimateTotalSize()}
is overridden in your derived class.


\membersection{wxVarVScrollHelper::GetRowCount}\label{wxvarvscrollhelpergetrowcount}

\constfunc{size\_t}{GetRowCount}{\void}

Returns the number of rows the target window contains.

\wxheading{See also}

\helpref{SetRowCount()}{wxvarvscrollhelpersetrowcount}


\membersection{wxVarVScrollHelper::GetVisibleRowsBegin}\label{wxvarvscrollhelpergetvisiblerowsbegin}

\constfunc{size\_t}{GetVisibleRowsBegin}{\void}

Returns the index of the first visible row based on the scroll position.


\membersection{wxVarVScrollHelper::GetVisibleRowsEnd}\label{wxvarvscrollhelpergetvisiblerowsend}

\constfunc{size\_t}{GetVisibleRowsEnd}{\void}

Returns the index of the last visible row based on the scroll position. This
includes the last row even if it is only partially visible.


\membersection{wxVarVScrollHelper::IsRowVisible}\label{wxvarvscrollhelperisrowvisible}

\constfunc{bool}{IsRowVisible}{\param{size\_t }{row}}

Returns {\tt true} if the given row is currently visible (even if only
partially visible) or {\tt false} otherwise.


\membersection{wxVarVScrollHelper::OnGetRowHeight}\label{wxvarvscrollhelperongetrowheight}

\constfunc{virtual wxCoord}{OnGetRowHeight}{\param{size\_t }{row}}

This function must be overridden in the derived class, and should return the
height of the given row in pixels.


\membersection{wxVarVScrollHelper::OnGetRowsHeightHint}\label{wxvarvscrollhelperongetrowsheighthint}

\constfunc{virtual void}{OnGetRowsHeightHint}{\param{size\_t }{rowMin}, \param{size\_t }{rowMax}}

This function doesn't have to be overridden but it may be useful to do so if
calculating the rows' sizes is a relatively expensive operation as it gives
your code a chance to calculate several of them at once and cache the result
if necessary.

{\tt OnGetRowsHeightHint()} is normally called just before
\helpref{OnGetRowHeight()}{wxvarvscrollhelperongetrowheight} but you
shouldn't rely on the latter being called for all rows in the interval
specified here. It is also possible that OnGetRowHeight() will be called for
units outside of this interval, so this is really just a hint, not a promise.

Finally, note that rowMin is inclusive, while rowMax is exclusive.


\membersection{wxVarVScrollHelper::RefreshRow}\label{wxvarvscrollhelperrefreshrow}

\func{virtual void}{RefreshRow}{\param{size\_t }{row}}

Triggers a refresh for just the given row's area of the window if it's visible.


\membersection{wxVarVScrollHelper::RefreshRows}\label{wxvarvscrollhelperrefreshrows}

\func{virtual void}{RefreshRows}{\param{size\_t }{from}, \param{size\_t }{to}}

Triggers a refresh for the area between the specified range of rows given
(inclusively).


\membersection{wxVarVScrollHelper::ScrollRowPages}\label{wxvarvscrollhelperscrollrowpages}

\func{virtual bool}{ScrollRowPages}{\param{int }{pages}}

Scroll by the specified number of pages which may be positive (to scroll down)
or negative (to scroll up).


\membersection{wxVarVScrollHelper::ScrollRows}\label{wxvarvscrollhelperscrollrows}

\func{virtual bool}{ScrollRows}{\param{int }{rows}}

Scroll by the specified number of rows which may be positive (to scroll down)
or negative (to scroll up).

Returns {\tt true} if the window was scrolled, {\tt false} otherwise (for
example, if we're trying to scroll down but we are already showing the last
row).


\membersection{wxVarVScrollHelper::ScrollToRow}\label{wxvarvscrollhelperscrolltorow}

\func{bool}{ScrollToRow}{\param{size\_t }{row}}

Scroll to the specified row. It will become the first visible row in the window.

Returns {\tt true} if we scrolled the window, {\tt false} if nothing was done.


\membersection{wxVarVScrollHelper::SetRowCount}\label{wxvarvscrollhelpersetrowcount}

\func{void}{SetRowCount}{\param{size\_t }{rowCount}}

Set the number of rows the window contains. The derived class must provide the
heights for all rows with indices up to the one given here in it's
\helpref{OnGetRowHeight()}{wxvarvscrollhelperongetrowheight} implementation.

\wxheading{See also}

\helpref{GetRowCount()}{wxvarvscrollhelpergetrowcount}

