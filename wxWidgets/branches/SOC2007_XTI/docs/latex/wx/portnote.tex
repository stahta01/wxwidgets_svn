\chapter{Platform details}\label{platformdetails}
\setheader{{\it CHAPTER \thechapter}}{}{}{}{}{{\it CHAPTER \thechapter}}%
\setfooter{\thepage}{}{}{}{}{\thepage}%

wxWidgets defines a common API across platforms, but uses the native graphical
user interface (GUI) on each platform, so your program will take on the native
look and feel that users are familiar with. Unfortunately native toolkits and
hardware do not always support the functionality that the wxWidgets API
requires. This chapter collects notes about differences among supported platforms and ports.

\input wxgtk.tex
\input wxmsw.tex
\input wxmac.tex
\input wxpalm.tex
\input wxos2.tex
\input wxmgl.tex
\input wxx11.tex

\subsection{Documentation for the native toolkits}\label{nativedocs}

It's sometimes useful to interface directly with the underlying toolkit
used by wxWidgets to e.g. use toolkit-specific features.
In such case (or when you want to e.g. write a port-specific patch) it can be
necessary to use the underlying toolkit API directly:

\begin{description}\itemsep=0pt
\item[{\bf wxMSW}]
wxMSW port uses win32 API: \urlref{MSDN docs}{http://msdn2.microsoft.com/en-us/library/ms649779.aspx}

\item[{\bf wxGTK}]
wxGTK port uses GTK+: \urlref{GTK+ 2.x docs}{http://developer.gnome.org/doc/API/2.0/gtk/index.html}

\end{description}