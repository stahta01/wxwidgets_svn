\section{\class{wxEventBlocker}}\label{wxeventblocker}

This class is a special event handler which allows to discard
any event (or a set of event types) directed to a specific window.

Example:

\begin{verbatim}

  {
    // block all events directed to this window while
    // we do the 1000 FuncWhichSendsEvents() calls
    wxEventBlocker blocker(this);

    for ( int i = 0; i < 1000; i++ )
       FuncWhichSendsEvents(i);

  } // ~wxEventBlocker called, old event handler is restored

  // the event generated by this call will be processed
  FuncWhichSendsEvents(0)
\end{verbatim}


\wxheading{Derived from}

\helpref{wxEvtHandler}{wxevthandler}\\
\helpref{wxObject}{wxobject}

\wxheading{Include files}

<wx/event.h>

\wxheading{See also}

\overview{Event handling overview}{eventhandlingoverview},
\helpref{wxEvtHandler}{wxevthandler}


\latexignore{\rtfignore{\wxheading{Members}}}

\membersection{wxEventBlocker::wxEventBlocker}\label{wxeventblockerctor}

\func{}{wxEventBlocker}{\param{wxWindow* }{win}, \param{wxEventType}{type = wxEVT\_ANY}}

Constructs the blocker for the given window and for the given event type.
If \arg{type} is \texttt{wxEVT\_ANY}, then all events for that window are
blocked. You can call \helpref{Block}{wxeventblockerblock} after creation to
add other event types to the list of events to block.

Note that the \arg{win} window \textbf{must} remain alive until the
wxEventBlocker object destruction.


\membersection{wxEventBlocker::\destruct{wxEventBlocker}}\label{wxeventblockerdtor}

\func{}{\destruct{wxEventBlocker}}{\void}

Destructor. The blocker will remove itself from the chain of event handlers for
the window provided in the constructor, thus restoring normal processing of
events.


\membersection{wxEventBlocker::Block}\label{wxeventblockerblock}

\func{void}{Block}{\param{wxEventType }{eventType}}

Adds to the list of event types which should be blocked the given \arg{eventType}.

