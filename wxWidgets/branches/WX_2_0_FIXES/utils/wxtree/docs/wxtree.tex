\documentstyle[a4,makeidx,verbatim,texhelp,fancyhea,mysober,mytitle]{report}%
\newcommand{\indexit}[1]{#1\index{#1}}%
\newcommand{\pipe}[0]{$\|$\ }%
\definecolour{black}{0}{0}{0}%
\definecolour{cyan}{0}{255}{255}%
\definecolour{green}{0}{255}{0}%
\definecolour{magenta}{255}{0}{255}%
\definecolour{red}{255}{0}{0}%
\definecolour{blue}{0}{0}{200}%
\definecolour{yellow}{255}{255}{0}%
\definecolour{white}{255}{255}{255}%
\input psbox.tex
\parskip=10pt%
\title{Manual for wxTreeLayout 2.0: a tree layout library for wxWindows}
\author{Julian Smart\\Anthemion Software}
\date{July 1998}%
\makeindex%
\begin{document}%
\maketitle

\pagestyle{fancyplain}
\bibliographystyle{plain}
\pagenumbering{roman}
\setheader{{\it CONTENTS}}{}{}{}{}{{\it CONTENTS}}
\setfooter{\thepage}{}{}{}{}{\thepage}
\tableofcontents%

\chapter{Introduction}
\pagenumbering{arabic}%
\setheader{{\it CHAPTER \thechapter}}{}{}{}{}{{\it CHAPTER \thechapter}}%
\setfooter{\thepage}{}{}{}{}{\thepage}

This manual describes a tree-drawing class library for wxWindows. It
provides layout of simple trees with one root node, drawn left-to-right,
with user-defined spacing between nodes.

wxTreeLayout is an abstract class that must be subclassed. The programmer
defines various member functions which will access whatever data structures
are appropriate for the application, and wxTreeLayout uses these when laying
out the tree.

wxStoredTree is a class derived from wxTreeLayout that may be used directly to
draw trees on a canvas. It supplies storage for the nodes, and draws
to a device context.

\helponly{Below is the example tree generated by the program test.cc.

\begin{figure}
$$\image{11cm;0cm}{treetst.ps}$$
\caption{Example tree}\label{exampletree}
\end{figure}
}

\chapter{Implementation}
\setheader{{\it CHAPTER \thechapter}}{}{}{}{}{{\it CHAPTER \thechapter}}%
\setfooter{\thepage}{}{}{}{}{\thepage}

The algorithm is due to Gabriel Robins \cite{robins87}, a linear-time
algorithm originally implemented in LISP for AI applications.

The original algorithm has been modified so that both X and Y planes
are calculated simultaneously, increasing efficiency slightly. The basic
code is only a page or so long.

\input classes.tex
%
\bibliography{tree}

\addcontentsline{toc}{chapter}{Index}
\setheader{{\it INDEX}}{}{}{}{}{{\it INDEX}}%
\setfooter{\thepage}{}{}{}{}{\thepage}
\printindex
\end{document}
