\section{\class{wxDataObject}}\label{wxdataobject}

A wxDataObject represents data that can be copied to or from the clipboard, or
dragged and dropped.

There are several predefined data object classes, such as \helpref{wxFileDataObject}{wxfiledataobject}, 
\helpref{wxTextDataObject}{wxtextdataobject}, and \helpref{wxBitmapDataObject}{wxbitmapdataobject} which
can be used without change or can be altered (by deriving a new class from them) in order to deliver
data and data size on-demand. There is no need to ever use wxDataObject itself or derive directly from it.

You may also derive your own data object classes from \helpref{wxPrivateDataObject}{wxprivatedataobject} 
for user-defined types. The format of user-defined data is given as mime-type string literal, 
such as "application/word" or "image/png". These strings are used as they are under Unix (so 
far only GTK) to identify a format and are translated into their Windows equivalent under 
Win32 (using the OLE IDataObject for data exchange to and from the clipboard and for Drag'n'Drop).
Note that the format string translation under Windows is not yet finnished.

As mentioned above, data may be placed into the \helpref{wxClipboard}{wxclipboard} 
or a \helpref{wxDropSource}{wxdropsource} instance either directly or on-demand. 
As long as only one format is offerred, putting data directly into the clipboard may 
be sufficient. But imagine that you paste a large piece of text to the clipboard and 
offer it in "text/plain", "text/rtf", "text/html", "application/word" and your own 
format for internal use - here offering data on-demand is required to minimize memory 
consumption. This would generally get implemented using a central object that
contains clipboard information in the format with the maximum of information. Note
that neither the GTK data transfer mechanisms for the clipboard and Drag'n'Drop
nor the OLE data transfer copies any data until another application actually
requests the data. This is in contrast to the "feel" offered to the user of a
program who would normally think that the data resides in the clipboard after
having pressed "Copy" - in reality it is only declared to be available.

Let's assume that you have written an HTML editor and want it to paste contents
in the formats "text/plain" and "text/html" to the clipboard. For offering
data on-demand in "text/plain" you would derive your class from \helpref{wxTextDataObject}{wxtextdataobject} 
and for offering data on-demand in "text/html" you would derive your own class from
\helpref{wxPrivateDataObject}{wxprivatedataobject} and set its ID string
identifying the format to "text/html" using \helpref{wxPrivateDataObject::SetId}{wxprivatedataobjectsetid}.
In your two derived classed you'd then have a pointer or reference to the central
data container and you'd override the methods returning the size of the
available data and the WriteData() methods in both classes.

\wxheading{Derived from}

\helpref{wxObject}{wxobject}

\wxheading{Include files}

<wx/dataobj.h>

\wxheading{See also}

\helpref{wxFileDataObject}{wxfiledataobject}, 
\helpref{wxTextDataObject}{wxtextdataobject}, 
\helpref{wxBitmapDataObject}{wxbitmapdataobject}, 
\helpref{wxPrivateDataObject}{wxprivatedataobject}, 
\helpref{Drag and drop overview}{wxdndoverview}, \helpref{wxDropTarget}{wxdroptarget}, 
\helpref{wxDropSource}{wxdropsource}, 
\helpref{wxTextDropTarget}{wxtextdroptarget}, \helpref{wxFileDropTarget}{wxfiledroptarget}

\latexignore{\rtfignore{\wxheading{Members}}}

\membersection{wxDataObject::wxDataObject}\label{wxdataobjectwxdataobject}

\func{}{wxDataObject}{\void}

Constructor.

\membersection{wxDataObject::\destruct{wxDataObject}}\label{wxdataobjectdtor}

\func{}{\destruct{wxDataObject}}{\void}

Destructor.

\membersection{wxDataObject::WriteData}\label{wxdataobjectwritedata}

\constfunc{virtual void}{WriteData}{\param{void}{*dest} }

Write the data owned by this class to {\it dest}. This method is a pure
virtual function and must be overridden.

\membersection{wxDataObject::GetSize}\label{wxdataobjectgetdatasize}

\constfunc{virtual size\_t}{GetSize}{\void}

Returns the data size. This method is a pure
virtual function and must be overridden.


