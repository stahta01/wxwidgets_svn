\section{\class{wxSlider}}\label{wxslider}

A slider is a control with a handle which can be pulled
back and forth to change the value.

In Windows versions below Windows 95, a scrollbar is used to simulate the slider. In Windows 95,
the track bar control is used.

Slider events are handled in the same way as a scrollbar.

\wxheading{Derived from}

\helpref{wxControl}{wxcontrol}\\
\helpref{wxWindow}{wxwindow}\\
\helpref{wxEvtHandler}{wxevthandler}\\
\helpref{wxObject}{wxobject}

\wxheading{Include files}

<wx/slider.h>

\wxheading{Window styles}

\twocolwidtha{5cm}
\begin{twocollist}\itemsep=0pt
\twocolitem{\windowstyle{wxSL\_HORIZONTAL}}{Displays the slider horizontally.}
\twocolitem{\windowstyle{wxSL\_VERTICAL}}{Displays the slider vertically.}
\twocolitem{\windowstyle{wxSL\_AUTOTICKS}}{Displays tick marks.}
\twocolitem{\windowstyle{wxSL\_LABELS}}{Displays minimum, maximum and value labels.}
\twocolitem{\windowstyle{wxSL\_LEFT}}{Displays ticks on the left, if a vertical slider.}
\twocolitem{\windowstyle{wxSL\_RIGHT}}{Displays ticks on the right, if a vertical slider.}
\twocolitem{\windowstyle{wxSL\_TOP}}{Displays ticks on the top, if a horizontal slider.}
\twocolitem{\windowstyle{wxSL\_SELRANGE}}{Allows the user to select a range on the slider. Windows 95 only.}
\end{twocollist}

See also \helpref{window styles overview}{windowstyles}.

\wxheading{Event handling}

To process input from a slider, use one of these event handler macros to direct input to member
functions that take a \helpref{wxScrollEvent}{wxscrollevent} argument:

\twocolwidtha{7cm}
\begin{twocollist}
\twocolitem{{\bf EVT\_COMMAND\_SCROLL(id, func)}}{Catch all scroll commands.}
\twocolitem{{\bf EVT\_COMMAND\_TOP(id, func)}}{Catch a command to put the scroll thumb at the maximum position.}
\twocolitem{{\bf EVT\_COMMAND\_BOTTOM(id, func)}}{Catch a command to put the scroll thumb at the maximum position.}
\twocolitem{{\bf EVT\_COMMAND\_LINEUP(id, func)}}{Catch a line up command.}
\twocolitem{{\bf EVT\_COMMAND\_LINEDOWN(id, func)}}{Catch a line down command.}
\twocolitem{{\bf EVT\_COMMAND\_PAGEUP(id, func)}}{Catch a page up command.}
\twocolitem{{\bf EVT\_COMMAND\_PAGEDOWN(id, func)}}{Catch a page down command.}
\twocolitem{{\bf EVT\_COMMAND\_THUMBTRACK(id, func)}}{Catch a thumbtrack command (continuous movement of the scroll thumb).}
\twocolitem{{\bf EVT\_SLIDER(id, func)}}{Process a wxEVT\_COMMAND\_SLIDER\_UPDATED event,
when the slider is moved. Though provided for backward compatibility, this is obsolete.}
\end{twocollist}%

\wxheading{See also}

\helpref{Event handling overview}{eventhandlingoverview}, \helpref{wxScrollBar}{wxscrollbar}

\latexignore{\rtfignore{\wxheading{Members}}}

\membersection{wxSlider::wxSlider}\label{wxsliderconstr}

\func{}{wxSlider}{\void}

Default slider.

\func{}{wxSlider}{\param{wxWindow* }{parent}, \param{wxWindowID }{id}, \param{int }{value },\rtfsp
\param{int}{ minValue}, \param{int}{ maxValue},\rtfsp
\param{const wxPoint\& }{point = wxDefaultPosition}, \param{const wxSize\& }{size = wxDefaultSize},\rtfsp
\param{long}{ style = wxSL\_HORIZONTAL},\rtfsp
\param{const wxValidator\& }{validator = wxDefaultValidator},\rtfsp
\param{const wxString\& }{name = ``slider"}}

Constructor, creating and showing a slider.

\wxheading{Parameters}

\docparam{parent}{Parent window. Must not be NULL.}

\docparam{id}{Window identifier. A value of -1 indicates a default value.}

\docparam{value}{Initial position for the slider.}

\docparam{minValue}{Minimum slider position.}

\docparam{maxValue}{Maximum slider position.}

\docparam{size}{Window size. If the default size (-1, -1) is specified then a default size is chosen.}

\docparam{style}{Window style. See \helpref{wxSlider}{wxslider}.}

\docparam{validator}{Window validator.}

\docparam{name}{Window name.}

\wxheading{See also}

\helpref{wxSlider::Create}{wxslidercreate}, \helpref{wxValidator}{wxvalidator}

\membersection{wxSlider::\destruct{wxSlider}}

\func{void}{\destruct{wxSlider}}{\void}

Destructor, destroying the slider.

\membersection{wxSlider::ClearSel}\label{wxsliderclearsel}

\func{void}{ClearSel}{\void}

Clears the selection, for a slider with the {\bf wxSL\_SELRANGE} style.

\wxheading{Remarks}

Windows 95 only.

\membersection{wxSlider::ClearTicks}\label{wxsliderclearticks}

\func{void}{ClearTicks}{\void}

Clears the ticks.

\wxheading{Remarks}

Windows 95 only.

\membersection{wxSlider::Create}\label{wxslidercreate}

\func{bool}{Create}{\param{wxWindow* }{parent}, \param{wxWindowID }{id}, \param{int }{value },\rtfsp
\param{int}{ minValue}, \param{int}{ maxValue},\rtfsp
\param{const wxPoint\& }{point = wxDefaultPosition}, \param{const wxSize\& }{size = wxDefaultSize},\rtfsp
\param{long}{ style = wxSL\_HORIZONTAL},\rtfsp
\param{const wxValidator\& }{validator = wxDefaultValidator},\rtfsp
\param{const wxString\& }{name = ``slider"}}

Used for two-step slider construction. See \helpref{wxSlider::wxSlider}{wxsliderconstr}\rtfsp
for further details.

\membersection{wxSlider::GetLineSize}\label{wxslidergetlinesize}

\constfunc{int}{GetLineSize}{\void}

Returns the line size.

\wxheading{See also}

\helpref{wxSlider::SetLineSize}{wxslidersetlinesize}

\membersection{wxSlider::GetMax}\label{wxslidergetmax}

\constfunc{int}{GetMax}{\void}

Gets the maximum slider value.

\wxheading{See also}

\helpref{wxSlider::GetMin}{wxslidergetmin}, \helpref{wxSlider::SetRange}{wxslidersetrange}

\membersection{wxSlider::GetMin}\label{wxslidergetmin}

\constfunc{int}{GetMin}{\void}

Gets the minimum slider value.

\wxheading{See also}

\helpref{wxSlider::GetMin}{wxslidergetmin}, \helpref{wxSlider::SetRange}{wxslidersetrange}

\membersection{wxSlider::GetPageSize}\label{wxslidergetpagesize}

\constfunc{int}{GetPageSize}{\void}

Returns the page size.

\wxheading{See also}

\helpref{wxSlider::SetPageSize}{wxslidersetpagesize}

\membersection{wxSlider::GetSelEnd}\label{wxslidergetselend}

\constfunc{int}{GetSelEnd}{\void}

Returns the selection end point.

\wxheading{Remarks}

Windows 95 only.

\wxheading{See also}

\helpref{wxSlider::GetSelStart}{wxslidergetselstart}, \helpref{wxSlider::SetSelection}{wxslidersetselection}

\membersection{wxSlider::GetSelStart}\label{wxslidergetselstart}

\constfunc{int}{GetSelStart}{\void}

Returns the selection start point.

\wxheading{Remarks}

Windows 95 only.

\wxheading{See also}

\helpref{wxSlider::GetSelEnd}{wxslidergetselend}, \helpref{wxSlider::SetSelection}{wxslidersetselection}

\membersection{wxSlider::GetThumbLength}\label{wxslidergetthumblength}

\constfunc{int}{GetThumbLength}{\void}

Returns the thumb length.

\wxheading{Remarks}

Windows 95 only.

\wxheading{See also}

\helpref{wxSlider::SetThumbLength}{wxslidersetthumblength}

\membersection{wxSlider::GetTickFreq}\label{wxslidergettickfreq}

\constfunc{int}{GetTickFreq}{\void}

Returns the tick frequency.

\wxheading{Remarks}

Windows 95 only.

\wxheading{See also}

\helpref{wxSlider::SetTickFreq}{wxslidersettickfreq}

\membersection{wxSlider::GetValue}\label{wxslidergetvalue}

\constfunc{int}{GetValue}{\void}

Gets the current slider value.

\wxheading{See also}

\helpref{wxSlider::GetMin}{wxslidergetmin}, \helpref{wxSlider::GetMax}{wxslidergetmax},\rtfsp
\helpref{wxSlider::SetValue}{wxslidersetvalue}

\membersection{wxSlider::SetRange}\label{wxslidersetrange}

\func{void}{SetRange}{\param{int}{ minValue}, \param{int}{ maxValue}}

Sets the minimum and maximum slider values.

\wxheading{See also}

\helpref{wxSlider::GetMin}{wxslidergetmin}, \helpref{wxSlider::GetMax}{wxslidergetmax}

\membersection{wxSlider::SetTickFreq}\label{wxslidersettickfreq}

\func{void}{SetTickFreq}{\param{int }{n}, \param{int }{pos}}

Sets the tick mark frequency and position.

\wxheading{Parameters}

\docparam{n}{Frequency. For example, if the frequency is set to two, a tick mark is displayed for
every other increment in the slider's range.}

\docparam{pos}{Position. Must be greater than zero. TODO: what is this for?}

\wxheading{Remarks}

Windows 95 only.

\wxheading{See also}

\helpref{wxSlider::GetTickFreq}{wxslidergettickfreq}

\membersection{wxSlider::SetLineSize}\label{wxslidersetlinesize}

\func{void}{SetLineSize}{\param{int }{lineSize}}

Sets the line size for the slider.

\wxheading{Parameters}

\docparam{lineSize}{The number of steps the slider moves when the user moves it up or down a line.}

\wxheading{See also}

\helpref{wxSlider::GetLineSize}{wxslidergetlinesize}

\membersection{wxSlider::SetPageSize}\label{wxslidersetpagesize}

\func{void}{SetPageSize}{\param{int }{pageSize}}

Sets the page size for the slider.

\wxheading{Parameters}

\docparam{pageSize}{The number of steps the slider moves when the user pages up or down.}

\wxheading{See also}

\helpref{wxSlider::GetPageSize}{wxslidergetpagesize}

\membersection{wxSlider::SetSelection}\label{wxslidersetselection}

\func{void}{SetSelection}{\param{int }{startPos}, \param{int }{endPos}}

Sets the selection.

\wxheading{Parameters}

\docparam{startPos}{The selection start position.}

\docparam{endPos}{The selection end position.}

\wxheading{Remarks}

Windows 95 only.

\wxheading{See also}

\helpref{wxSlider::GetSelStart}{wxslidergetselstart}, \helpref{wxSlider::GetSelEnd}{wxslidergetselend}

\membersection{wxSlider::SetThumbLength}\label{wxslidersetthumblength}

\func{void}{SetThumbLength}{\param{int }{len}}

Sets the slider thumb length.

\wxheading{Parameters}

\docparam{len}{The thumb length.}

\wxheading{Remarks}

Windows 95 only.

\wxheading{See also}

\helpref{wxSlider::GetThumbLength}{wxslidergetthumblength}

\membersection{wxSlider::SetTick}\label{wxslidersettick}

\func{void}{SetTick}{\param{int}{ tickPos}}

Sets a tick position.

\wxheading{Parameters}

\docparam{tickPos}{The tick position.}

\wxheading{Remarks}

Windows 95 only.

\wxheading{See also}

\helpref{wxSlider::SetTickFreq}{wxslidersettickfreq}

\membersection{wxSlider::SetValue}\label{wxslidersetvalue}

\func{void}{SetValue}{\param{int}{ value}}

Sets the slider position.

\wxheading{Parameters}

\docparam{value}{The slider position.}

\wxheading{See also}

\helpref{wxSlider::GetValue}{wxslidergetvalue}

