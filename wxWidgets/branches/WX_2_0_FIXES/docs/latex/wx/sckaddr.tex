% ----------------------------------------------------------------------------
% CLASS: wxSockAddress
% ----------------------------------------------------------------------------
\section{\class{wxSockAddress}}\label{wxsockaddress}

You are unlikely to need to use this class: only wxSocketBase uses it.

\wxheading{Derived from}

\helpref{wxObject}{wxobject}

\wxheading{Include files}

<wx/socket.h>

\wxheading{See also}

\helpref{wxSocketBase}{wxsocketbase}
%\helpref{wxIPV4address}{wxipv4address}\\
%\helpref{wxIPV6address}{wxipv6address}\\
%\helpref{wxunixaddress}{wxunixaddress}

% ----------------------------------------------------------------------------
% Members
% ----------------------------------------------------------------------------

\latexignore{\rtfignore{\wxheading{Members}}}

%
% ctor/dtor
%

\membersection{wxSockAddress::wxSockAddress}

\func{}{wxSockAddress}{\void}

Default constructor.

\membersection{wxSockAddress::\destruct{wxSockAddress}}

\func{}{\destruct{wxSockAddress}}{\void}

Default destructor.

%
% Clear
%
\membersection{wxSockAddress::Clear}
\func{void}{Clear}{\void}

Delete all informations about the address.

%
% Build
%
\membersection{wxSockAddress::Build}

\func{void}{Build}{\param{struct sockaddr *\&}{ addr}, \param{size\_t\&}{ len}}

Build a coded socket address.

%
% Disassemble
%
\membersection{wxSockAddress::Disassemble}

\func{void}{Disassemble}{\param{struct sockaddr *}{addr}, \param{size\_t}{ len}}

Decode a socket address. {\bf Actually, you don't have to use this
function: only wxSocketBase use it.}

%
% SockAddrLen
%
\membersection{wxSockAddress::SockAddrLen}

\func{int}{SockAddrLen}{\void};

Returns the length of the socket address.

