%
% automatically generated by HelpGen from
% htmprint.h at 17/Oct/99 12:48:02
%


\section{\class{wxHtmlDCRenderer}}\label{wxhtmldcrenderer}

This class can render HTML document into specified area of DC. You can use it
in your own printing code, although use of \helpref{wxHtmlEasyPrinting}{wxhtmleasyprinting}
or \helpref{wxHtmlPrintout}{wxhtmlprintout} is strongly recommended.


\wxheading{Derived from}

\helpref{wxObject}{wxobject}



\latexignore{\rtfignore{\wxheading{Members}}}


\membersection{wxHtmlDCRenderer::wxHtmlDCRenderer}\label{wxhtmldcrendererwxhtmldcrenderer}

\func{}{wxHtmlDCRenderer}{\void}

Constructor.


\membersection{wxHtmlDCRenderer::SetDC}\label{wxhtmldcrenderersetdc}

\func{void}{SetDC}{\param{wxDC* }{dc}, \param{int }{maxwidth}}

Assign DC instance to the renderer.

\wxheading{Parameters}

\docparam{maxwidth}{width of the area (on this DC) that is equivalent to screen's width, 
in pixels (you should set it to page width minus margins). 

{\bf Note:} In current implementation
screen width is always 800 pixels : it gives best results and ensures (almost) same printed outputs
across platforms and differently configured desktops.}

Also see \helpref{SetSize}{wxhtmldcrenderersetsize}



\membersection{wxHtmlDCRenderer::SetSize}\label{wxhtmldcrenderersetsize}

\func{void}{SetSize}{\param{int }{width}, \param{int }{height}}

Set size of output rectangle, in pixels. Note that you {\bf can't} change
width of the rectangle between calls to \helpref{Render}{wxhtmldcrendererrender}! 
(You can freely change height.)
If you set width equal to maxwidth then HTML is rendered as if it were displayed in fullscreen.
If you set width = 1/2 maxwidth the it is rendered as if it covered half the screen
and so on.


\membersection{wxHtmlDCRenderer::SetHtmlText}\label{wxhtmldcrenderersethtmltext}

\func{void}{SetHtmlText}{\param{const wxString\& }{html}, \param{const wxString\& }{basepath = wxEmptyString}, \param{bool }{isdir = TRUE}}

Assign text to the renderer. \helpref{Render}{wxhtmldcrendererrender} then draws 
the text onto DC.


\wxheading{Parameters}

\docparam{html}{HTML text. (NOT file!)}

\docparam{basepath}{base directory (html string would be stored there if it was in
file). It is used to determine path for loading images, for example.}

\docparam{isdir}{FALSE if basepath is filename, TRUE if it is directory name
(see \helpref{wxFileSystem}{wxfilesystem} for detailed explanation)}


\membersection{wxHtmlDCRenderer::Render}\label{wxhtmldcrendererrender}

\func{int}{Render}{\param{int }{x}, \param{int }{y}, \param{int }{from = 0}, \param{int }{dont\_render = FALSE}}

Renders HTML text to the DC.

\wxheading{Parameters}


\docparam{x,y}{ position of upper-left corner of printing rectangle (see \helpref{SetSize}{wxhtmldcrenderersetsize})}


\docparam{from}{y-coordinate of the very first visible cell}

\docparam{dont\_render}{if TRUE then this method only returns y coordinate of the next page
and does not output anything}

Returned value is y coordinate of first cell than didn't fit onto page.
Use this value as {\it from} in next call to Render in order to print multipages
document.


\wxheading{Caution!}

Following 3 methods {\bf must} always be called before any call to Render (preferably
in this order):

\begin{itemize}

\item \helpref{SetDC}{wxhtmldcrenderersetdc}
\item \helpref{SetSize}{wxhtmldcrenderersetsize}
\item \helpref{SetHtmlText}{wxhtmldcrenderersethtmltext}

\end{itemize}

{\bf Render() changes DC's user scale and does NOT restore it!!}



\membersection{wxHtmlDCRenderer::GetTotalHeight}\label{wxhtmldcrenderergettotalheight}

\func{int}{GetTotalHeight}{\void}

Returns height of the HTML text. This is important if area height (see \helpref{SetSize}{wxhtmldcrenderersetsize})
is smaller that total height and thus the page cannot fit into it. In that case you're supposed to
call \helpref{Render}{wxhtmldcrendererrender} as long as it's return value is smaller than GetTotalHeight's.




