%
% automatically generated by HelpGen from
% htmprint.h at 17/Oct/99 12:48:02
%

\section{\class{wxHtmlDCRenderer}}\label{wxhtmldcrenderer}

This class can render HTML document into a specified area of a DC. You can use it
in your own printing code, although use of \helpref{wxHtmlEasyPrinting}{wxhtmleasyprinting} 
or \helpref{wxHtmlPrintout}{wxhtmlprintout} is strongly recommended.

\wxheading{Derived from}

\helpref{wxObject}{wxobject}

\wxheading{Include files}

<wx/html/htmprint.h>

\latexignore{\rtfignore{\wxheading{Members}}}

\membersection{wxHtmlDCRenderer::wxHtmlDCRenderer}\label{wxhtmldcrendererwxhtmldcrenderer}

\func{}{wxHtmlDCRenderer}{\void}

Constructor.

\membersection{wxHtmlDCRenderer::SetDC}\label{wxhtmldcrenderersetdc}

\func{void}{SetDC}{\param{wxDC* }{dc}, \param{double }{pixel\_scale = 1.0}}

Assign DC instance to the renderer.

{\it pixel\_scale} can be used when rendering to high-resolution DCs (e.g. printer) to adjust size of pixel metrics.
(Many dimensions in HTML are given in pixels -- e.g. image sizes. 300x300 image would be only one
inch wide on typical printer. With pixel\_scale = 3.0 it would be 3 inches.)

\wxheading{Parameters}

\docparam{maxwidth}{width of the area (on this DC) that is equivalent to the screen's width, 
in pixels (you should set it to page width minus margins). 

{\bf Note:} In the current implementation
the screen width is always 800 pixels: it gives best results and ensures (almost) same printed outputs
across platforms and differently configured desktops.}

\membersection{wxHtmlDCRenderer::SetFonts}\label{wxhtmldcrenderersetfonts}

\func{void}{SetFonts}{\param{wxString }{normal\_face}, \param{wxString }{fixed\_face}, \param{const int }{*sizes = NULL}}

Sets fonts. See \helpref{wxHtmlWindow::SetFonts}{wxhtmlwindowsetfonts} for
detailed description.

See also \helpref{SetSize}{wxhtmldcrenderersetsize}.

\membersection{wxHtmlDCRenderer::SetSize}\label{wxhtmldcrenderersetsize}

\func{void}{SetSize}{\param{int }{width}, \param{int }{height}}

Set size of output rectangle, in pixels. Note that you {\bf can't} change
width of the rectangle between calls to \helpref{Render}{wxhtmldcrendererrender}!
(You can freely change height.)

\membersection{wxHtmlDCRenderer::SetHtmlText}\label{wxhtmldcrenderersethtmltext}

\func{void}{SetHtmlText}{\param{const wxString\& }{html}, \param{const wxString\& }{basepath = wxEmptyString}, \param{bool }{isdir = true}}

Assign text to the renderer. \helpref{Render}{wxhtmldcrendererrender} then draws 
the text onto DC.

\wxheading{Parameters}

\docparam{html}{HTML text. This is {\it not} a filename.}

\docparam{basepath}{base directory (html string would be stored there if it was in
file). It is used to determine path for loading images, for example.}

\docparam{isdir}{false if basepath is filename, true if it is directory name
(see \helpref{wxFileSystem}{wxfilesystem} for detailed explanation)}

\membersection{wxHtmlDCRenderer::Render}\label{wxhtmldcrendererrender}

\func{int}{Render}{\param{int }{x}, \param{int }{y}, \param{int }{from = 0}, \param{int }{dont\_render = false}}

Renders HTML text to the DC.

\wxheading{Parameters}

\docparam{x,y}{ position of upper-left corner of printing rectangle (see \helpref{SetSize}{wxhtmldcrenderersetsize})}

\docparam{from}{y-coordinate of the very first visible cell}

\docparam{dont\_render}{if true then this method only returns y coordinate of the next page
and does not output anything}

Returned value is y coordinate of first cell than didn't fit onto page.
Use this value as {\it from} in next call to Render in order to print multipages
document.

\wxheading{Caution!}

The Following three methods {\bf must} always be called before any call to Render (preferably
in this order):

\begin{itemize}\itemsep=0pt
\item \helpref{SetDC}{wxhtmldcrenderersetdc}
\item \helpref{SetSize}{wxhtmldcrenderersetsize}
\item \helpref{SetHtmlText}{wxhtmldcrenderersethtmltext}
\end{itemize}

{\bf Render() changes the DC's user scale and does NOT restore it.}

\membersection{wxHtmlDCRenderer::GetTotalHeight}\label{wxhtmldcrenderergettotalheight}

\func{int}{GetTotalHeight}{\void}

Returns the height of the HTML text. This is important if area height (see \helpref{SetSize}{wxhtmldcrenderersetsize})
is smaller that total height and thus the page cannot fit into it. In that case you're supposed to
call \helpref{Render}{wxhtmldcrendererrender} as long as its return value is smaller than GetTotalHeight's.

