\section{\class{wxSocketBase}}\label{wxsocketbase}

\wxheading{Derived from}

\helpref{wxEvtHandler}{wxevthandler}

\wxheading{Include files}

<wx/socket.h>

\wxheading{wxSocket errors}

\twocolwidtha{7cm}
\begin{twocollist}\itemsep=0pt
\twocolitem{{\bf wxSOCKET\_NOERROR}}{No error happened.}
\twocolitem{{\bf wxSOCKET\_INVOP}}{Invalid operation.}
\twocolitem{{\bf wxSOCKET\_IOERR}}{Input/Output error.}
\twocolitem{{\bf wxSOCKET\_INVADDR}}{Invalid address passed to wxSocket.}
\twocolitem{{\bf wxSOCKET\_INVSOCK}}{Invalid socket (uninitialized).}
\twocolitem{{\bf wxSOCKET\_NOHOST}}{No corresponding host.}
\twocolitem{{\bf wxSOCKET\_INVPORT}}{Invalid port.}
\twocolitem{{\bf wxSOCKET\_WOULDBLOCK}}{The socket is non-blocking and the operation would block.}
\twocolitem{{\bf wxSOCKET\_TIMEDOUT}}{The timeout for this operation expired.}
\twocolitem{{\bf wxSOCKET\_MEMERR}}{Memory exhausted.}
\end{twocollist}%

\wxheading{wxSocket events}

\twocolwidtha{7cm}
\begin{twocollist}\itemsep=0pt
\twocolitem{{\bf wxSOCKET\_INPUT}}{Some data has arrived to the socket.}
\twocolitem{{\bf wxSOCKET\_OUTPUT}}{The socket is ready to be written to.}
\twocolitem{{\bf wxSOCKET\_CONNECTION}}{Incoming connection arrival (server), or connection establishment (client).}
\twocolitem{{\bf wxSOCKET\_LOST}}{The connection has been closed.}
\twocolitem{{\bf wxSOCKET\_MAX\_EVENT}}{This should never happen but the compiler may complain about it.}
\end{twocollist}%

A brief note on how to use these events:

The {\bf wxSOCKET\_INPUT} event will be issued when the incoming queue
was empty and new data arrives, but NOT if new data arrives when there
was data waiting in the incoming queue.

The {\bf wxSOCKET\_OUTPUT} event is issued when a socket is first connected
with Connect or accepted with Accept, and then, only after an output operation
fails because the output buffer was full, and buffer space becomes available
again.

The {\bf wxSOCKET\_CONNECTION} event is issued when a connection request
completes (client) or when a new connection arrives at the pending
connections queue (server).

The {\bf wxSOCKET\_LOST} event is issued when a close indication is
received for the socket. This means that the connection broke down or
that it was closed by the peer.

% ---------------------------------------------------------------------------
% Event handling
% ---------------------------------------------------------------------------
\wxheading{Event handling}

To process events from a socket, use the following event handler macro to direct
input to member functions that take a \helpref{wxSocketEvent}{wxsocketevent} argument.

\twocolwidtha{7cm}%
\begin{twocollist}\itemsep=0pt
\twocolitem{{\bf EVT\_SOCKET(id, func)}}{A socket event occured.}
\end{twocollist}%

% ---------------------------------------------------------------------------
% See also ...
% ---------------------------------------------------------------------------
\wxheading{See also}

\helpref{wxSocketEvent}{wxsocketevent}, 
\helpref{wxSocketClient}{wxsocketclient}, 
\helpref{wxSocketServer}{wxsocketserver}

% ---------------------------------------------------------------------------
% Members
% ---------------------------------------------------------------------------
\latexignore{\rtfignore{\wxheading{Members}}}

\membersection{wxSocketBase::wxSocketBase}

\func{}{wxSocketBase}{\void}

Default constructor. Don't use it; use \helpref{wxSocketClient}{wxsocketclient} 
or \helpref{wxSocketServer}{wxsocketserver}.

\membersection{wxSocketBase::\destruct{wxSocketBase}}

\func{}{\destruct{wxSocketBase}}{\void}

Destroys the wxSocketBase object.

% ---------------------------------------------------------------------------
% State functions
% ---------------------------------------------------------------------------

%
% SetFlags
%

\membersection{wxSocketBase::SetFlags}\label{wxsocketbasesetflags}

\func{void}{SetFlags}{\param{wxSocketBase::wxSockFlags}{ flags}}

\twocolwidtha{7cm}
\begin{twocollist}\itemsep=0pt
\twocolitem{{\bf wxSOCKET\_NONE}}{Normal functionnality.}
\twocolitem{{\bf wxSOCKET\_NOWAIT}}{Get the available data in the input queue and return immediately.}
\twocolitem{{\bf wxSOCKET\_WAITALL}}{Wait for all required data unless an error occurs.}
\twocolitem{{\bf wxSOCKET\_BLOCK}}{Block the GUI (do not wxYield) while reading/writing data.}
\end{twocollist}

A brief overview on how to use these flags follows.

If no flag is specified (this is the same as {\bf wxSOCKET\_NONE}),
IO calls will return after some data has been read or written, even
when the transfer might not be complete. This is the same as issuing
exactly one blocking low-level call to recv() or send(). Note that
blocking here refers to when the function returns, not to whether
the GUI blocks during this time.

If {\bf wxSOCKET\_NOWAIT} is specified, IO calls will return immediately.
Read operations will retrieve only available data. Write operations will
write as much data as possible, depending on how much space is available
in the output buffer. This is the same as issuing exactly one nonblocking
low-level call to recv() or send(). Note that nonblocking here refers to
when the function returns, not to whether the GUI blocks during this time.

If {\bf wxSOCKET\_WAITALL} is specified, IO calls won't return until ALL
the data has been read or written (or until an error occurs), blocking if
necessary, and issuing several low level calls if necessary. This is the
same as having a loop which makes as many blocking low-level calls to
recv() or send() as needed so as to transfer all the data. Note that
"blocking" here refers to when the function returns, not to whether
the GUI blocks during this time.

The {\bf wxSOCKET\_BLOCK} flag controls whether the GUI blocks during
IO operations. If this flag is not used, then the application must take
extra care to avoid unwanted reentrance. 

So:

{\bf wxSOCKET\_NONE} will try to read SOME data, no matter how much.

{\bf wxSOCKET\_NOWAIT} will always return immediately, even if it cannot
read or write ANY data.

{\bf wxSOCKET\_WAITALL} will only return when it has read or written ALL
the data.

{\bf wxSOCKET\_BLOCK} has nothing to do with the previous flags and
it controls whether the GUI blocks.

%
% SetNotify
%
\membersection{wxSocketBase::SetNotify}\label{wxsocketbasesetnotify}

\func{void}{SetNotify}{\param{wxSocketEventFlags}{ flags}}

SetNotify specifies which socket events are to be sent to the event handler.
The {\it flags} parameter is a combination of flags ORed toghether. The
following flags can be used:

\twocolwidtha{7cm}
\begin{twocollist}\itemsep=0pt
\twocolitem{{\bf wxSOCKET\_INPUT\_FLAG}}{to receive wxSOCKET\_INPUT}
\twocolitem{{\bf wxSOCKET\_OUTPUT\_FLAG}}{to receive wxSOCKET\_OUTPUT}
\twocolitem{{\bf wxSOCKET\_CONNECTION\_FLAG}}{to receive wxSOCKET\_CONNECTION}
\twocolitem{{\bf wxSOCKET\_LOST\_FLAG}}{to receive wxSOCKET\_LOST}
\end{twocollist}%

For example:

\begin{verbatim}
  sock.SetNotify(wxSOCKET_INPUT_FLAG | wxSOCKET_LOST_FLAG);
\end{verbatim}

In this example, the user will be notified about incoming socket data and
whenever the connection is closed.

For more information on socket events see \helpref{wxSocket events}{wxsocketbase}.

%
% SetTimeout
%
\membersection{wxSocketBase::SetTimeout}\label{wxsocketbasesettimeout}

\func{void}{SetTimeout}{\param{int }{seconds}}

This function sets the default socket timeout in seconds. This
timeout applies to IO calls and also to Wait functions if you
don't specify a wait interval. If you never use SetTimeout, the
default timeout will be 10 minutes.

%
% Notify
%
\membersection{wxSocketBase::Notify}\label{wxsocketbasenotify}

\func{void}{Notify}{\param{bool}{ notify}}

Notify will enable (notify is TRUE) or disable (notify is FALSE) the propagation
of socket events.

% 
% Ok
%

\membersection{wxSocketBase::Ok}\label{wxsocketbaseok}

\constfunc{bool}{Ok}{\void}

Returns TRUE if the socket is initialized and ready and FALSE in other
cases.

\membersection{wxSocketBase::Error}\label{wxsocketbaseerror}

\constfunc{bool}{Error}{\void}

Returns TRUE if an error occured in the last IO operation.

The following operations update the Error() status:
Read, Write, ReadMsg, WriteMsg, Peek, Unread, Discard.

\membersection{wxSocketBase::IsConnected}\label{wxsocketbaseconnected}

\constfunc{bool}{IsConnected}{\void}

Returns TRUE if the socket is connected.

\membersection{wxSocketBase::IsData}\label{wxsocketbaseisdata}

\constfunc{bool}{IsData}{\void}

Returns TRUE if there is data available to be read. 

\membersection{wxSocketBase::IsDisconnected}\label{wxsocketbasedisconnected}

\constfunc{bool}{IsDisconnected}{\void}

Returns TRUE if the socket is disconnected.

\membersection{wxSocketBase::IsNoWait}\label{wxsocketbasenowait}

\constfunc{bool}{IsNoWait}{\void}

Returns TRUE if the socket mustn't wait.

\membersection{wxSocketBase::LastCount}\label{wxsocketbaselastcount}

\constfunc{wxUint32}{LastCount}{\void}

Returns the number of bytes read or written by the last IO call.

The following operations update the LastCount() value:
Read, Write, ReadMsg, WriteMsg, Peek, Unread, Discard.

\membersection{wxSocketBase::LastError}\label{wxsocketbaselasterror}

\constfunc{wxSocketError}{LastError}{\void}

Returns the last wxSocket error. See \helpref{wxSocket errors}{wxsocketbase}.

Please note that this function merely returns the last error code,
but it should not be used to determine if an error has occured (this
is because successful operations do not change tha LastError value).
Use Error, instead of LastError, to determine if the last IO call
failed. If Error returns TRUE, use LastError to discover the
cause of the error.

% ---------------------------------------------------------------------------
% IO calls
% ---------------------------------------------------------------------------
%
% Peek
%
\membersection{wxSocketBase::Peek}\label{wxsocketbasepeek}

\func{wxSocketBase\&}{Peek}{\param{char *}{ buffer}, \param{wxUint32}{ nbytes}}

This function peeks a buffer of {\it nbytes} bytes from the socket. Peeking a buffer
doesn't delete it from the system socket in-queue.

Use LastCount to verify the number of bytes actually peeked.

Use Error to determine if the operation succeeded.

\wxheading{Parameters}

\docparam{buffer}{Buffer where to put peeked data.}

\docparam{nbytes}{Number of bytes.}

\wxheading{Return value}

Returns a reference to the current object.

\wxheading{Remark/Warning}

The exact behaviour of wxSocketBase::Peek() depends on the combination
of flags being used. For a detailed explanation, see \helpref{wxSocketBase::SetFlags}{wxsocketbasesetflags}

\wxheading{See also}

\helpref{wxSocketBase::Error}{wxsocketbaseerror}, 
\helpref{wxSocketBase::LastError}{wxsocketbaselasterror}, 
\helpref{wxSocketBase::LastCount}{wxsocketbaselastcount}, 
\helpref{wxSocketBase::SetFlags}{wxsocketbasesetflags}

%
% Read
%
\membersection{wxSocketBase::Read}\label{wxsocketbaseread}

\func{wxSocketBase\&}{Read}{\param{char *}{ buffer}, \param{wxUint32}{ nbytes}}

This function reads a buffer of {\it nbytes} bytes from the socket.

Use LastCount to verify the number of bytes actually read.

Use Error to determine if the operation succeeded.

\wxheading{Parameters}

\docparam{buffer}{Buffer where to put read data.}

\docparam{nbytes}{Number of bytes.}

\wxheading{Return value}

Returns a reference to the current object.

\wxheading{Remark/Warning}

The exact behaviour of wxSocketBase::Read() depends on the combination
of flags being used. For a detailed explanation, see \helpref{wxSocketBase::SetFlags}{wxsocketbasesetflags}.

\wxheading{See also}

\helpref{wxSocketBase::Error}{wxsocketbaseerror}, 
\helpref{wxSocketBase::LastError}{wxsocketbaselasterror}, 
\helpref{wxSocketBase::LastCount}{wxsocketbaselastcount}, 
\helpref{wxSocketBase::SetFlags}{wxsocketbasesetflags}

%
% Write
%
\membersection{wxSocketBase::Write}\label{wxsocketbasewrite}

\func{wxSocketBase\&}{Write}{\param{const char *}{ buffer}, \param{wxUint32}{ nbytes}}

This function writes a buffer of {\it nbytes} bytes to the socket.

Use LastCount to verify the number of bytes actually written.

Use Error to determine if the operation succeeded.

\wxheading{Parameters}

\docparam{buffer}{Buffer with the data to be sent.}

\docparam{nbytes}{Number of bytes.}

\wxheading{Return value}

Returns a reference to the current object.

\wxheading{Remark/Warning}

The exact behaviour of wxSocketBase::Write() depends on the combination
of flags being used. For a detailed explanation, see \helpref{wxSocketBase::SetFlags}{wxsocketbasesetflags}.

\wxheading{See also}

\helpref{wxSocketBase::Error}{wxsocketbaseerror}, 
\helpref{wxSocketBase::LastError}{wxsocketbaselasterror}, 
\helpref{wxSocketBase::LastCount}{wxsocketbaselastcount}, 
\helpref{wxSocketBase::SetFlags}{wxsocketbasesetflags}

%
% WriteMsg
%
\membersection{wxSocketBase::WriteMsg}\label{wxsocketbasewritemsg}

\func{wxSocketBase\&}{WriteMsg}{\param{const char *}{ buffer}, \param{wxUint32}{ nbytes}}

This function writes a buffer of {\it nbytes} bytes from the socket, but it
writes a short header before so that ReadMsg can alloc the right size for
the buffer. So, a buffer sent with WriteMsg {\bf must} be read with ReadMsg.
This function always waits for the entire buffer to be sent, unless an
error occurs.

Use LastCount to verify the number of bytes actually written.

Use Error to determine if the operation succeeded.

\wxheading{Parameters}

\docparam{buffer}{Buffer with the data to be sent.}

\docparam{nbytes}{Number of bytes.}

\wxheading{Return value}

Returns a reference to the current object.

\wxheading{Remark/Warning}

wxSocketBase::WriteMsg() will behave as if the wxSOCKET\_WAITALL flag was always set
and it will always ignore the wxSOCKET\_NOWAIT flag. The exact behaviour of WriteMsg
depends on the wxSOCKET\_BLOCK flag. For a detailed explanation, see \helpref{wxSocketBase::SetFlags}{wxsocketbasesetflags}.

\wxheading{See also}

\helpref{wxSocketBase::Error}{wxsocketbaseerror}, 
\helpref{wxSocketBase::LastError}{wxsocketbaselasterror}, 
\helpref{wxSocketBase::LastCount}{wxsocketbaselastcount}, 
\helpref{wxSocketBase::SetFlags}{wxsocketbasesetflags}, 
\helpref{wxSocketBase::ReadMsg}{wxsocketbasereadmsg}

%
% ReadMsg
%
\membersection{wxSocketBase::ReadMsg}\label{wxsocketbasereadmsg}

\func{wxSocketBase\&}{ReadMsg}{\param{char *}{ buffer}, \param{wxUint32}{ nbytes}}

This function reads a buffer sent by WriteMsg on a socket. If the buffer passed
to the function isn't big enough, the remaining bytes will be discarded. This
function always waits for the buffer to be entirely filled, unless an error occurs.

Use LastCount to verify the number of bytes actually read.

Use Error to determine if the operation succeeded.

\wxheading{Parameters}

\docparam{buffer}{Buffer where to put read data.}

\docparam{nbytes}{Number of bytes allocated for the buffer.}

\wxheading{Return value}

Returns a reference to the current object.

\wxheading{Remark/Warning}

wxSocketBase::ReadMsg() will behave as if the wxSOCKET\_WAITALL flag was always set
and it will always ignore the wxSOCKET\_NOWAIT flag. The exact behaviour of ReadMsg
depends on the wxSOCKET\_BLOCK flag. For a detailed explanation, see \helpref{wxSocketBase::SetFlags}{wxsocketbasesetflags}.

\wxheading{See also}

\helpref{wxSocketBase::Error}{wxsocketbaseerror}, 
\helpref{wxSocketBase::LastError}{wxsocketbaselasterror}, 
\helpref{wxSocketBase::LastCount}{wxsocketbaselastcount}, 
\helpref{wxSocketBase::SetFlags}{wxsocketbasesetflags}, 
\helpref{wxSocketBase::WriteMsg}{wxsocketbasewritemsg}

%
% Unread
%
\membersection{wxSocketBase::Unread}\label{wxsocketbaseunread}

\func{wxSocketBase\&}{Unread}{\param{const char *}{ buffer}, \param{wxUint32}{ nbytes}}

This function unreads a buffer. That is, the data in the buffer is put back
in the incoming queue. This function is not affected by wxSocket flags.

If you use LastCount, it will always return {\it nbytes}.

If you use Error, it will always return FALSE.

\wxheading{Parameters}

\docparam{buffer}{Buffer to be unread.}

\docparam{nbytes}{Number of bytes.}

\wxheading{Return value}

Returns a reference to the current object.

\wxheading{See also}

\helpref{wxSocketBase::Error}{wxsocketbaseerror}, 
\helpref{wxSocketBase::LastCount}{wxsocketbaselastcount}, 
\helpref{wxSocketBase::LastError}{wxsocketbaselasterror}

%
% Discard
%
\membersection{wxSocketBase::Discard}\label{wxsocketbasediscard}

\func{wxSocketBase\&}{Discard}{\void}

This function simply deletes all bytes in the incoming queue. This function
doesn't wait. That is, it will behave as if the wxSOCKET\_NOWAIT flag was set. The
wxSOCKET\_BLOCK and wxSOCKET\_WAITALL flags have no effect on this function.

Use LastCount to see the number of bytes discarded.

If you use Error, it will always return FALSE.

% ---------------------------------------------------------------------------
% Wait functions
% ---------------------------------------------------------------------------
\membersection{wxSocketBase::Wait}\label{wxsocketbasewait}

\func{bool}{Wait}{\param{long}{ seconds = -1}, \param{long}{ millisecond = 0}}

This function waits until one of the following conditions is true: there
is data available for reading; the output buffer is empty (you can send
new data); the connection has been lost; an incoming connection arrived
(only for servers); a connection request has completed (only for clients).

\wxheading{Parameters}

\docparam{seconds}{Number of seconds to wait. If -1, it will wait for the default timeout set with SetTimeout.}

\docparam{millisecond}{Number of milliseconds to wait.}

\wxheading{Return value}

Returns TRUE if an event occured, FALSE if the timeout was reached.

\wxheading{See also}

\helpref{wxSocketBase::WaitForRead}{wxsocketbasewaitforread}, 
\helpref{wxSocketBase::WaitForWrite}{wxsocketbasewaitforwrite}, 
\helpref{wxSocketBase::WaitForLost}{wxsocketbasewaitforlost}

%
% WaitForRead
%
\membersection{wxSocketBase::WaitForRead}\label{wxsocketbasewaitforread}

\func{bool}{WaitForRead}{\param{long}{ seconds = -1}, \param{long}{ millisecond = 0}}

This function waits until there is data available to be read.

\wxheading{Parameters}

\docparam{seconds}{Number of seconds to wait. If -1, it will wait for the default timeout set with SetTimeout.}

\docparam{millisecond}{Number of milliseconds to wait.}

\wxheading{Return value}

Returns TRUE if there is data to be read, FALSE if the timeout was reached.

\wxheading{See also}

\helpref{wxSocketBase::Wait}{wxsocketbasewait}, 
\helpref{wxSocketBase::WaitForWrite}{wxsocketbasewaitforwrite}, 
\helpref{wxSocketBase::WaitForLost}{wxsocketbasewaitforlost}

%
% WaitForWrite
%
\membersection{wxSocketBase::WaitForWrite}\label{wxsocketbasewaitforwrite}

\func{bool}{WaitForWrite}{\param{long}{ seconds = -1}, \param{long}{ millisecond = 0}}

This function waits until you can write to the socket.

\wxheading{Parameters}

\docparam{seconds}{Number of seconds to wait. If -1, it will wait for the default timeout set with SetTimeout.}

\docparam{millisecond}{Number of milliseconds to wait.}

\wxheading{Return value}

Returns TRUE if you can write to the socket, FALSE if the timeout was reached.

\wxheading{See also}

\helpref{wxSocketBase::Wait}{wxsocketbasewait}, 
\helpref{wxSocketBase::WaitForRead}{wxsocketbasewaitforread}, 
\helpref{wxSocketBase::WaitForLost}{wxsocketbasewaitforlost}

%
% WaitForLost
%
\membersection{wxSocketBase::WaitForLost}\label{wxsocketbasewaitforlost}

\func{bool}{Wait}{\param{long}{ seconds = -1}, \param{long}{ millisecond = 0}}

This function waits until the connection is lost. This may happen if the
peer closes the connection or if the connection breaks.

\wxheading{Parameters}

\docparam{seconds}{Number of seconds to wait. If -1, it will wait for the default timeout set with SetTimeout.}

\docparam{millisecond}{Number of milliseconds to wait.}

\wxheading{Return value}

Returns TRUE if the connection was lost, FALSE if the timeout was reached.

\wxheading{See also}

\helpref{wxSocketBase::WaitForRead}{wxsocketbasewaitforread}, 
\helpref{wxSocketBase::WaitForWrite}{wxsocketbasewaitforwrite}, 
\helpref{wxSocketBase::WaitForLost}{wxsocketbasewaitforlost}

% ---------------------------------------------------------------------------
% Socket state
% ---------------------------------------------------------------------------

%
% RestoreState
%
\membersection{wxSocketBase::RestoreState}\label{wxsocketbaserestorestate}

\func{void}{RestoreState}{\void}

This function restores the previous state of the socket, as saved
with SaveState.

Calls to SaveState / RestoreState can be nested.

\wxheading{See also}

\helpref{wxSocketBase::SaveState}{wxsocketbasesavestate}

%
% SaveState
%
\membersection{wxSocketBase::SaveState}\label{wxsocketbasesavestate}

\func{void}{SaveState}{\void}

This function saves the current state of the socket object in a stack:
actually it saves all flags (those set with SetFlags, SetNotify, Notify)
and the state of the asynchronous callbacks (Callback, CallbackData).

Calls to SaveState / RestoreState can be nested.

\wxheading{See also}

\helpref{wxSocketBase::RestoreState}{wxsocketbaserestorestate}

%
% GetLocal
%
\membersection{wxSocketBase::GetLocal}{wxsocketbasegetlocal}

\constfunc{bool}{GetLocal}{\param{wxSockAddress\& }{addr\_man}}

This function returns the local address field of the socket. The local
address field contains the complete local address of the socket (local
address, local port, ...).

\wxheading{Return value}

It returns TRUE if no errors happened, FALSE otherwise.

%
% GetPeer
%
\membersection{wxSocketBase::GetPeer}{wxsocketbasegetlocal}

\constfunc{bool}{GetPeer}{\param{wxSockAddress\& }{addr\_man}}

This function returns the peer address field of the socket. The peer 
address field contains the complete peer host address of the socket
(address, port, ...).

\wxheading{Return value}

It returns TRUE if no errors happened, FALSE otherwise.

% ---------------------------------------------------------------------------
% Socket callbacks
% ---------------------------------------------------------------------------
\membersection{wxSocketBase::SetEventHandler}\label{wxsocketbaseseteventhandler}

\func{void}{SetEventHandler}{\param{wxEvtHandler\&}{ evt\_hdlr}, \param{int}{ id = -1}}

Sets an event handler to be called when a socket event occurs. The handler
will be called for those events for which notification is enabled with
SetNotify and Notify.

You can also specify a C callback to be called when an event occurs. See
Callback and CallbackData.

\wxheading{Parameters}

\docparam{evt\_hdlr}{Specifies the event handler you want to use.}

\docparam{id}{The id of socket event.}

\wxheading{See also}

\helpref{wxSocketBase::SetNotify}{wxsocketbasesetnotify}, 
\helpref{wxSocketBase::Notify}{wxsocketbasenotify}, 
\helpref{wxSocketEvent}{wxsocketevent}, 
\helpref{wxEvtHandler}{wxevthandler}, 
\helpref{wxSocketBase::Callback}{wxsocketbasecallback}, 
\helpref{wxSocketBase::CallbackData}{wxsocketbasecallbackdata}

\membersection{wxSocketBase::Callback}\label{wxsocketbasecallback}

\func{wxSocketBase::wxSockCbk}{Callback}{\param{wxSocketBase::wxSockCbk}{ callback}}

You can setup a C callback to be called when an event occurs. The callback
will be called only for those events for which notification has been enabled
with Notify and SetNotify. The prototype of the callback must be as follows:

\begin{verbatim}
void SocketCallback(wxSocketBase& sock,wxSocketNotify evt,char *cdata);
\end{verbatim}

The first parameter is a reference to the socket object in which the event
occured. The second parameter tells you which event occured. (See \helpref{wxSocket events}{wxsocketbase}).
The third parameter is the user data you specified using \helpref{CallbackData}{wxsocketbasecallbackdata}.

\wxheading{Return value}

A pointer to the previous callback.

\wxheading{See also}

\helpref{wxSocketBase::CallbackData}{wxsocketbasecallbackdata}, 
\helpref{wxSocketBase::SetNotify}{wxsocketbasesetnotify}, 
\helpref{wxSocketBase::Notify}{wxsocketbasenotify}

\membersection{wxSocketBase::CallbackData}\label{wxsocketbasecallbackdata}

\func{char *}{CallbackData}{\param{char *}{cdata}}

This function sets the the user data which will be passed to a \helpref{C callback}{wxsocketbasecallback}.

\wxheading{Return value}

A pointer to the previous user data.

\helpref{wxSocketBase::Callback}{wxsocketbasecallback}, 
\helpref{wxSocketBase::SetNotify}{wxsocketbasesetnotify}, 
\helpref{wxSocketBase::Notify}{wxsocketbasenotify}

% ---------------------------------------------------------------------------
% CLASS wxSocketClient
% ---------------------------------------------------------------------------
\section{\class{wxSocketClient}}\label{wxsocketclient}

\wxheading{Derived from}

\helpref{wxSocketBase}{wxsocketbase}

\wxheading{Include files}

<wx/socket.h>

% ---------------------------------------------------------------------------
% Members
% ---------------------------------------------------------------------------
%
% wxSocketClient
%
\membersection{wxSocketClient::wxSocketClient}

\func{}{wxSocketClient}{\param{wxSockFlags}{ flags = wxSocketBase::NONE}}

Constructs a new wxSocketClient.

\wxheading{Parameters}

\docparam{flags}{Socket flags (See \helpref{wxSocketBase::SetFlags}{wxsocketbasesetflags})}

%
% ~wxSocketClient
%
\membersection{wxSocketClient::\destruct{wxSocketClient}}

\func{}{\destruct{wxSocketClient}}{\void}

Destroys a wxSocketClient object.

%
% Connect
%
\membersection{wxSocketClient::Connect}\label{wxsocketclientconnect}

\func{bool}{Connect}{\param{wxSockAddress\&}{ address}, \param{bool}{ wait = TRUE}}

Connects to a server using the specified address.

If {\it wait} is TRUE, Connect will wait until the connection completes and
the socket is ready to send or receive data, or until an event occurs.

{\bf Warning !} This will block the GUI. 

If {\it wait} is FALSE, Connect will try to establish the connection and
return immediately, without blocking the GUI. When used this way, even if
Connect returns FALSE, the connection request can be completed later.
To detect this, use WaitConnection, or watch "connection" events (for
succesful establishment) and "lost" events (for connection failure).

\wxheading{Parameters}

\docparam{address}{Address of the server.}

\docparam{wait}{If true, waits for the connection to be ready.}

\wxheading{Return value}

Returns TRUE if the connection is established and no error occurs.

If {\it wait} was TRUE, and Connect returns FALSE, an error occured
and the connection failed.

If {\it wait} was FALSE, and Connect returns FALSE, you should still
be prepared to handle the completion of this connection request, either
with WaitOnConnect or by watching "connection" and "lost" events.

\wxheading{See also}

\helpref{wxSocketClient::WaitOnConnect}{wxsocketclientwaitonconnect}, 
\helpref{wxSocketBase::SetNotify}{wxsocketbasesetnotify}, 
\helpref{wxSocketBase::Notify}{wxsocketbasenotify}

%
% WaitOnConnect
%
\membersection{wxSocketClient::WaitOnConnect}\label{wxsocketclientwaitonconnect}

\func{bool}{WaitOnConnect}{\param{long}{ seconds = -1}, \param{long}{ milliseconds = 0}}

Wait until the connection is succesfully established or until it fails.
Use this function after a call to Connect with {\it wait} set to FALSE. 

\wxheading{Parameters}

\docparam{seconds}{Number of seconds to wait. If -1, it will wait for the default timeout set with SetTimeout.}

\docparam{millisecond}{Number of milliseconds to wait.}

\wxheading{Return value}

If the connection is succesfully established, it returns TRUE.

If the timeout expires, or if the connection fails, it returns FALSE.

\wxheading{See also}

\helpref{wxSocketClient::Connect}{wxsocketclientconnect}

% ---------------------------------------------------------------------------
% CLASS: wxSocketEvent
% ---------------------------------------------------------------------------
\section{\class{wxSocketEvent}}\label{wxsocketevent}

This event class contains information about socket events.

\wxheading{Derived from}

\helpref{wxEvent}{wxevent}

\wxheading{Include files}

<wx/socket.h>

\wxheading{Event table macros}

To process a socket event, use these event handler macros to direct input to member
functions that take a wxSocketEvent argument.

\twocolwidtha{7cm}
\begin{twocollist}\itemsep=0pt
\twocolitem{{\bf EVT\_SOCKET(id, func)}}{Process a socket event, supplying the member function.}
\end{twocollist}%

\wxheading{See also}

\helpref{wxSocketBase}{wxsocketbase}, 
\helpref{wxSocketClient}{wxsocketclient}, 
\helpref{wxSocketServer}{wxsocketserver}

\latexignore{\rtfignore{\wxheading{Members}}}

\membersection{wxSocketEvent::wxSocketEvent}

\func{}{wxSocketEvent}{\param{int}{ id = 0}}

Constructor.

\membersection{wxSocketEvent::SocketEvent}\label{wxsocketeventsocketevent}

\constfunc{wxSocketNotify}{SocketEvent}{\void}

Returns the socket event type.

% ---------------------------------------------------------------------------
% CLASS: wxSocketServer
% ---------------------------------------------------------------------------
\section{\class{wxSocketServer}}\label{wxsocketserver}

\wxheading{Derived from}

\helpref{wxSocketBase}{wxsocketbase}

\wxheading{Include files}

<wx/socket.h>

% ---------------------------------------------------------------------------
% Members
% ---------------------------------------------------------------------------
\latexignore{\rtfignore{\wxheading{Members}}}

%
% wxSocketServer
%
\membersection{wxSocketServer::wxSocketServer}\label{wxsocketserverconstr}

\func{}{wxSocketServer}{\param{wxSockAddress\&}{ address}, \param{wxSockFlags}{ flags = wxSocketBase::NONE}}

Constructs a new wxSocketServer.

\wxheading{Parameters}

\docparam{address}{Specifies the local address for the server (e.g. port number).}

\docparam{flags}{Socket flags (See \helpref{wxSocketBase::SetFlags}{wxsocketbasesetflags})}

%
% ~wxSocketServer
%
\membersection{wxSocketServer::\destruct{wxSocketServer}}

\func{}{\destruct{wxSocketServer}}{\void}

Destroys a wxSocketServer object (it doesn't close the accepted connections).

%
% Accept
%
\membersection{wxSocketServer::Accept}\label{wxsocketserveraccept}

\func{wxSocketBase *}{Accept}{\param{bool}{ wait = TRUE}}

Creates a new object wxSocketBase and accepts an incoming connection.

If {\it wait} is TRUE and there are no pending connections to be
accepted, it will wait for the next incoming connection to arrive.
{\bf Warning !} This will block the GUI.

If {\it wait} is FALSE, it will try to accept a pending connection
if there is one, but it will always return immediately without
blocking the GUI. If you want to use Accept in this way, you can
either check for incoming connections with WaitForAccept or watch
"connection" events, then call Accept once you know that there is
an incoming connection waiting to be accepted.

\wxheading{Return value}

Returns an opened socket connection, or NULL if an error occured or
if the {\it wait} parameter was FALSE and there were no pending
connections.

\wxheading{See also}

\helpref{wxSocketServer::WaitForAccept}{wxsocketserverwaitforaccept}, 
\helpref{wxSocketBase::SetNotify}{wxsocketbasesetnotify}, 
\helpref{wxSocketBase::Notify}{wxsocketbasenotify}, 
\helpref{wxSocketServer::AcceptWith}{wxsocketserveracceptwith}

%
% AcceptWith
%
\membersection{wxSocketServer::AcceptWith}\label{wxsocketserveracceptwith}

\func{bool}{AcceptWith}{\param{wxSocketBase\&}{ socket}, \param{bool}{ wait = TRUE}}

Accept an incoming connection using the specified socket object.
This is useful when someone wants to inherit wxSocketBase.

\wxheading{Parameters}

\docparam{socket}{Socket to be initialized}

\wxheading{Return value}

Returns TRUE on success, or FALSE if an error occured or if the
{\it wait} parameter was FALSE and there were no pending
connections.

\helpref{wxSocketServer::WaitForAccept}{wxsocketserverwaitforaccept}, 
\helpref{wxSocketBase::SetNotify}{wxsocketbasesetnotify}, 
\helpref{wxSocketBase::Notify}{wxsocketbasenotify}, 
\helpref{wxSocketServer::Accept}{wxsocketserveraccept} for a detailed explanation

%
% WaitForAccept
%
\membersection{wxSocketServer::WaitForAccept}\label{wxsocketserverwaitforaccept}

\func{bool}{WaitForAccept}{\param{long}{ seconds = -1}, \param{long}{ millisecond = 0}}

This function waits for an incoming connection. Use it if you want to call
Accept or AcceptWith with {\it wait} set to FALSE, to detect when an incoming
connection is waiting to be accepted.

\wxheading{Parameters}

\docparam{seconds}{Number of seconds to wait. If -1, it will wait for the default timeout set with SetTimeout.}

\docparam{millisecond}{Number of milliseconds to wait.}

\wxheading{Return value}

Returns TRUE if an incoming connection arrived, FALSE if the timeout expired.

\wxheading{See also}

\helpref{wxSocketServer::Accept}{wxsocketserveraccept}, 
\helpref{wxSocketServer::AcceptWith}{wxsocketserveracceptwith}

