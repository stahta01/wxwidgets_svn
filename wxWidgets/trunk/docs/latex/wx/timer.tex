\section{\class{wxTimer}}\label{wxtimer}

The wxTimer object is an abstraction of MS Windows and X toolkit timers. To
use it, derive a new class and override the {\bf Notify} member to
perform the required action. Start with {\bf Start}, stop with {\bf
Stop}, it's as simple as that.

\wxheading{Derived from}

\helpref{wxObject}{wxobject}

\wxheading{See also}

\helpref{::wxStartTimer}{wxstarttimer}, \helpref{::wxGetElapsedTime}{wxgetelapsedtime}

\latexignore{\rtfignore{\wxheading{Members}}}

\membersection{wxTimer::wxTimer}

\func{}{wxTimer}{\void}

Constructor.

\membersection{wxTimer::\destruct{wxTimer}}

\func{}{\destruct{wxTimer}}{\void}

Destructor. Stops the timer if activated.

\membersection{wxTimer::Interval}

\func{int}{Interval}{\void}

Returns the current interval for the timer.

\membersection{wxTimer::Notify}

\func{void}{Notify}{\void}

This member should be overridden by the user. It is called on timeout.

\membersection{wxTimer::Start}

\func{bool}{Start}{\param{int}{ milliseconds = -1}, \param{bool}{ oneShot=FALSE}}

(Re)starts the timer. If {\it milliseconds}\/ is absent or -1, the
previous value is used. Returns FALSE if the timer could not be started,
TRUE otherwise (in MS Windows timers are a limited resource).

If {\it oneShot} is FALSE (the default), the Notify function will be repeatedly
called. If TRUE, Notify will be called only once.

\membersection{wxTimer::Stop}

\func{void}{Stop}{\void}

Stops the timer.


