% -----------------------------------------------------------------------------
% wxFilterInputStream
% -----------------------------------------------------------------------------
\section{\class{wxFilterInputStream}}\label{wxfilterinputstream}

A filter stream has the capability of a normal stream but it can be placed on top
of another stream. So, for example, it can uncompress or decrypt the data which are read
from another stream and pass it to the requester.

\wxheading{Derived from}

\helpref{wxInputStream}{wxinputstream}\\
\helpref{wxStreamBase}{wxstreambase}

\wxheading{Include files}

<wx/stream.h>

\wxheading{Library}

\helpref{wxBase}{librarieslist}

\wxheading{Note}

The interface of this class is the same as that of wxInputStream. Only a constructor
differs and it is documented below.

\wxheading{See also}

\helpref{wxFilterClassFactory}{wxfilterclassfactory}\\
\helpref{wxFilterOutputStream}{wxfilteroutputstream}

\latexignore{\rtfignore{\wxheading{Members}}}

% -----------
% ctor & dtor
% -----------
\membersection{wxFilterInputStream::wxFilterInputStream}\label{wxfilterinputstreamctor}

\func{}{wxFilterInputStream}{\param{wxInputStream\&}{ stream}}

\func{}{wxFilterInputStream}{\param{wxInputStream*}{ stream}}

Initializes a "filter" stream.

If the parent stream is passed as a pointer then the new filter stream
takes ownership of it. If it is passed by reference then it does not.

