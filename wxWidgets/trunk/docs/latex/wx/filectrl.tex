%%%%%%%%%%%%%%%%%%%%%%%%%%%%%%%%%%%%%%%%%%%%%%%%%%%%%%%%%%%%%%%%%%%%%%%%%%%%%%%
%% Name:        filectrl.tex
%% Purpose:     wxFilerCtrl documentation
%% Author:      Diaa M. Sami
%% Created:     2007-07-25
%% RCS-ID:      $Id: $
%% Copyright:   (c) 2007 Diaa M. Sami
%% License:     wxWindows license
%%%%%%%%%%%%%%%%%%%%%%%%%%%%%%%%%%%%%%%%%%%%%%%%%%%%%%%%%%%%%%%%%%%%%%%%%%%%%%%

\section{\class{wxFileCtrl}}\label{wxfilectrl}

This control allows the user to select a file. two implemetations exist, one
for Gtk and another generic one for anything other than Gtk.
It is only available if \texttt{wxUSE\_FILECTRL} is set to $1$.

\wxheading{Derived from}

\helpref{wxWindow}{wxwindow}\\
\helpref{wxEvtHandler}{wxevthandler}\\
\helpref{wxObject}{wxobject}

\wxheading{Include files}

<wx/filectrl.h>

\wxheading{Window styles}

\twocolwidtha{5cm}%
\begin{twocollist}\itemsep=0pt
\twocolitem{\windowstyle{wxFC\_DEFAULT\_STYLE}}{The default style: wxFC\_OPEN}
\twocolitem{\windowstyle{wxFC\_OPEN}}{Creates an file control suitable for opening files.
Cannot be combined with wxFC\_SAVE.}
\twocolitem{\windowstyle{wxFC\_SAVE}}{Creates an file control suitable for saving files. Cannot be combined with wxFC\_OPEN.}
\twocolitem{\windowstyle{wxFC\_MULTIPLE}}{For open control only, Allows selecting multiple files. Cannot be combined with wxFC\_SAVE}
\twocolitem{\windowstyle{wxFC\_NOSHOWHIDDEN}}{Hides the ``Show Hidden Files" checkbox (Generic only)}
\end{twocollist}

\wxheading{Event handling}

To process a file control event, use these event handler macros to direct
input to member functions that take a \helpref{wxFileCtrlEvent}{wxfilectrlevent}
argument.

\twocolwidtha{7cm}%
\begin{twocollist}\itemsep=0pt
\twocolitem{{\bf EVT\_FILECTRL\_FILEACTIVATED(id, func)}}{The user activated a file(by double-clicking or pressing Enter)}
\twocolitem{{\bf EVT\_FILECTRL\_SELECTIONCHANGED(id, func)}}{The user changed the current selection(by selecting or deselecting a file)}
\twocolitem{{\bf EVT\_FILECTRL\_FOLDERCHANGED(id, func)}}{The current folder of the file ctrl has been changed}
\end{twocollist}

\wxheading{See also}

\helpref{wxGenericDirCtrl}{wxgenericdirctrl}

\latexignore{\rtfignore{\wxheading{Members}}}

\membersection{wxFileCtrl::wxFileCtrl}\label{wxfilectrlctor}

\func{}{wxFileCtrl}{\void}

Default constructor.

\func{}{wxFileCtrl}{\param{wxWindow *}{parent},\rtfsp
\param{wxWindowID}{ id},\rtfsp
\param{const wxString\& }{defaultDirectory = wxEmptyString},\rtfsp
\param{const wxString\& }{defaultFilename = wxEmptyString},\rtfsp
\param{const wxPoint\& }{wildCard = wxFileSelectorDefaultWildcardStr},\rtfsp
\param{long}{ style = wxFC\_DEFAULT\_STYLE},\rtfsp
\param{const wxPoint\& }{pos = wxDefaultPosition},
\param{const wxSize\& }{size = wxDefaultSize},
\param{const wxString\& }{name = ``filectrl"}}

\wxheading{Parameters}

\docparam{parent}{Parent window, must not be non-\texttt{NULL}.}

\docparam{id}{The identifier for the control.}

\docparam{defaultDirectory}{The initial directory shown in the control. Must be
a valid path to a directory or the empty string.
In case it is the empty string, the current working directory is used.}

\docparam{defaultFilename}{The default filename, or the empty string.}

\docparam{wildcard}{A wildcard specifying which files can be selected,
such as ``*.*" or ``BMP files (*.bmp)|*.bmp|GIF files (*.gif)|*.gif".}

\docparam{style}{The window style, see {\tt wxFC\_*} flags.}

\docparam{pos}{Initial position.}

\docparam{size}{Initial size.}

\docparam{name}{Control name.}

\wxheading{Return value}

\true if the control was successfully created or \false if creation failed.

\membersection{wxFileCtrl::Create}\label{wxfilectrlcreate}

\func{bool}{Create}{\param{wxWindow *}{parent},\rtfsp
\param{wxWindowID}{ id},\rtfsp
\param{const wxString\& }{defaultDirectory = wxEmptyString},\rtfsp
\param{const wxString\& }{defaultFilename = wxEmptyString},\rtfsp
\param{const wxPoint\& }{wildCard = wxFileSelectorDefaultWildcardStr},\rtfsp
\param{long}{ style = wxFC\_DEFAULT\_STYLE},\rtfsp
\param{const wxPoint\& }{pos = wxDefaultPosition},
\param{const wxSize\& }{size = wxDefaultSize},
\param{const wxString\& }{name = ``filectrl"}}

Create function for two-step construction. See \helpref{wxFileCtrl::wxFileCtrl}{wxfilectrlctor} for details.

\membersection{wxFileDialog::GetFilename}\label{wxfilectrlgetfilename}

\constfunc{wxString}{GetFilename}{\void}

Returns the currently selected filename.
For the controls having the {\tt wxFC\_MULTIPLE} style, use \helpref{GetFilenames}{wxfilectrlgetfilenames}
instead

\membersection{wxFileCtrl::GetDirectory}\label{wxfilectrlgetdirectory}

\constfunc{wxString}{GetDirectory}{\void}

Returns the current directory of the file ctrl(the directory shown in the file ctrl).

\membersection{wxFileCtrl::GetWildcard}\label{wxfilectrlgetwildcard}

\constfunc{wxString}{GetWildcard}{\void}

Returns the current wildcard.

\membersection{wxFileCtrl::GetPath}\label{wxfilectrlgetpath}

\constfunc{wxString}{GetPath}{\void}

Returns the full path (directory and filename) of the currently selected file.
For the controls having the {\tt wxFC\_MULTIPLE} style, use \helpref{GetPaths}{wxfilectrlgetpaths}
instead

\membersection{wxFileCtrl::GetPaths}\label{wxfilectrlgetpaths}

\constfunc{void}{GetPaths}{\param{wxArrayString\& }{paths}}

Fills the array {\it paths} with the full paths of the files chosen. This
function should be used with the controls having the {\tt wxFC\_MULTIPLE} style,
use \helpref{GetPath}{wxfilectrlgetpath} otherwise.

\wxheading{Remarks}

{\it paths} is emptied first.

\membersection{wxFileCtrl::GetFilenames}\label{wxfilectrlgetfilenames}

\constfunc{void}{GetFilenames}{\param{wxArrayString\& }{filenames}}

Fills the array {\it filenames} with the filenames only of selected items. This
function should only be used with the controls having the {\tt wxFC\_MULTIPLE} style,
use \helpref{GetFilename}{wxfilectrlgetfilename} for the others.

\wxheading{Remarks}

{\it filenames} is emptied first.

\membersection{wxFileCtrl::GetFilterIndex}\label{wxfilectrlgetfilterindex}

\constfunc{int}{GetFilterIndex}{\void}

Returns the zero-based index of the currently selected filter.

\membersection{wxFileCtrl::ShowHidden}\label{wxfilectrlshowhidden}

\func{void}{ShowHidden}{\param{const bool }{show}}

Sets whether hidden files and folders are shown or not.

\membersection{wxFileCtrl::SetWildcard}\label{wxfilectrlsetwildcard}

\func{void}{SetWildcard}{\param{const wxString\& }{wildCard}}

Sets the wildcard, which can contain multiple file types, for example:

``BMP files (*.bmp)|*.bmp|GIF files (*.gif)|*.gif"

\membersection{wxFileCtrl::SetFilterIndex}\label{wxfilectrlsetfilterindex}

\func{void}{SetFilterIndex}{\param{int }{filterIndex}}

Sets the current filter index, starting from zero.

\membersection{wxFileCtrl::SetDirectory}\label{wxfilectrlsetdirectory}

\func{bool}{SetDirectory}{\param{const wxString\& }{directory}}

Sets(changes) the current directory displayed in the control.

\wxheading{Return value}

Returns \true on success, \false otherwise.

\membersection{wxFileCtrl::SetFilename}\label{wxfilectrlsetfilename}

\func{bool}{SetFilename}{\param{const wxString\& }{filename}}

Selects a certain file.

\wxheading{Return value}

Returns \true on success, \false otherwise

\membersection{wxFileCtrl::SetPath}\label{wxfilectrlsetpath}

\func{bool}{SetPath}{\param{const wxString\& }{path}}

Selects a certain file using its path (the combined directory and filename).
Equivalent to \helpref{SetPath}{wxfilectrlsetpath} then \helpref{SetFilename}{wxfilectrlsetfilename}.

\wxheading{Return value}

Returns \true on success, \false otherwise.
