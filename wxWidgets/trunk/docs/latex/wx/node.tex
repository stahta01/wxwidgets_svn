\section{\class{wxNode}}\label{wxnode}

wxNodeBase is the node structure used in linked lists (see 
\helpref{wxList}{wxlist}) and derived classes. You should never use wxNodeBase
class directly, however, because it works with untyped ({\tt void *}) data and
this is unsafe. Use wxNodeBase-derived classes which are automatically defined
by WX\_DECLARE\_LIST and WX\_DEFINE\_LIST macros instead as described in
\helpref{wxList}{wxlist} documentation (see example there). Also note that
although there is a class called wxNode, it is defined for backwards
compatibility only and usage of this class is strongly deprecated.

In the documentation below, the type {\tt T} should be thought of as a
``template'' parameter: this is the type of data stored in the linked list or,
in other words, the first argument of WX\_DECLARE\_LIST macro. Also, wxNode is
written as wxNode<T> even though it isn't really a template class -- but it
helps to think of it as if it were.

\wxheading{Derived from}

None.

\wxheading{Include files}

<wx/list.h>

\wxheading{See also}

\helpref{wxList}{wxlist}, \helpref{wxHashTable}{wxhashtable}

\latexignore{\rtfignore{\wxheading{Members}}}

\membersection{wxNode<T>::GetData}\label{wxnodegetdata}

\constfunc{T *}{GetData}{\void}

Retrieves the client data pointer associated with the node.

\membersection{wxNode<T>::GetNext}\label{wxnodegetnext}

\constfunc{wxNode<T> *}{GetNext}{\void}

Retrieves the next node or NULL if this node is the last one.

\membersection{wxNode<T>::GetPrevious}\label{wxnodegetprevious}

\func{wxNode<T> *}{GetPrevious}{\void}

Retrieves the previous node or NULL if this node is the first one in the list.

\membersection{wxNode<T>::SetData}\label{wxnodesetdata}

\func{void}{SetData}{\param{T *}{data}}

Sets the data associated with the node (usually the pointer will have been
set when the node was created).

\membersection{wxNode<T>::IndexOf}\label{wxnodeindexof}

\func{int}{IndexOf}{\void}

Returns the zero-based index of this node within the list. The return value
will be {\tt wxNOT\_FOUND} if the node has not been added to a list yet.

