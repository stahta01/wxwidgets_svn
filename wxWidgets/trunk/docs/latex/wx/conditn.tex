\section{\class{wxCondition}}\label{wxcondition}

TODO

\wxheading{Derived from}

None.

\wxheading{See also}

\helpref{wxThread}{wxthread}, \helpref{wxMutex}{wxmutex}

\latexignore{\rtfignore{\wxheading{Members}}}

\membersection{wxCondition::wxCondition}\label{wxconditionconstr}

\func{}{wxCondition}{\void}

Default constructor.

\membersection{wxCondition::\destruct{wxCondition}}

\func{}{\destruct{wxCondition}}{\void}

Destroys the wxCondition object.

\membersection{wxCondition::Broadcast}\label{wxconditionbroadcast}

\func{void}{Broadcast}{\void}

Broadcasts to all waiting objects.

\membersection{wxCondition::Signal}\label{wxconditionsignal}

\func{void}{Signal}{\void}

Signals the object.

\membersection{wxCondition::Wait}\label{wxconditionwait}

\func{void}{Wait}{\param{wxMutex\&}{ mutex}}

Waits indefinitely.

\func{bool}{Wait}{\param{wxMutex\&}{ mutex}, \param{unsigned long}{ sec}, \param{unsigned long}{ nsec}}

Waits until a signal is raised or the timeout has elapsed.

\wxheading{Parameters}

\docparam{mutex}{wxMutex object.}

\docparam{sec}{Timeout in seconds}

\docparam{nsec}{Timeout nanoseconds component (added to {\it sec}).}

\wxheading{Return value}

The second form returns if the signal was raised, or FALSE if there was a timeout.


