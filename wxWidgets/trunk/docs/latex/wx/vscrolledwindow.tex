%%%%%%%%%%%%%%%%%%%%%%%%%%%%%%%%%%%%%%%%%%%%%%%%%%%%%%%%%%%%%%%%%%%%%%%%%%%%%
%% Name:        vscrolledwindow.tex (moved from vscroll.tex)
%% Purpose:     wxVScrolledWindow Documentation
%% Author:      Vadim Zeitlin, Bryan Petty
%% Modified by: 
%% Created:     2003-05-30, modified 2007-04-04
%% RCS-ID:      $Id$
%% Copyright:   (c) 2003, 2007 wxWidgets Team
%% License:     wxWindows Licence
%%%%%%%%%%%%%%%%%%%%%%%%%%%%%%%%%%%%%%%%%%%%%%%%%%%%%%%%%%%%%%%%%%%%%%%%%%%%%

\section{\class{wxVScrolledWindow}}\label{wxvscrolledwindow}

In the name of this class, "V" may stand for "variable" because it can be
used for scrolling rows of variable heights; "virtual", because it is not
necessary to know the heights of all rows in advance -- only those which
are shown on the screen need to be measured; or even "vertical", because
this class only supports scrolling vertically.

In any case, this is a generalization of the
\helpref{wxScrolledWindow}{wxscrolledwindow} class which can be only used when
all rows have the same heights. It lacks some other wxScrolledWindow features
however, notably it can't scroll only a rectangle of the window and not its
entire client area.

To use this class, you need to derive from it and implement the
\helpref{OnGetRowHeight()}{wxvarvscrollhelperongetrowheight} pure virtual
method. You also must call \helpref{SetRowCount}{wxvarvscrollhelpersetrowcount}
to let the base class know how many rows it should display, but from that
moment on the scrolling is handled entirely by wxVScrolledWindow. You only
need to draw the visible part of contents in your {\tt OnPaint()} method as
usual. You should use \helpref{GetVisibleRowsBegin()}{wxvarvscrollhelpergetvisiblerowsbegin}
and \helpref{GetVisibleRowsEnd()}{wxvarvscrollhelpergetvisiblerowsend} to
select the lines to display. Note that the device context origin is not shifted
so the first visible row always appears at the point $(0, 0)$ in physical as
well as logical coordinates.

\wxheading{Derived from}

\helpref{wxPanel}{wxpanel}\\
\helpref{wxWindow}{wxwindow}\\
\helpref{wxEvtHandler}{wxevthandler}\\
\helpref{wxObject}{wxobject}

\helpref{wxVarVScrollHelper}{wxvarvscrollhelper}\\
\helpref{wxVarScrollHelperBase}{wxvarscrollhelperbase}

\wxheading{Include files}

<wx/vscroll.h>

\wxheading{See also}

\helpref{wxHScrolledWindow}{wxhscrolledwindow},
\rtfsp\helpref{wxHVScrolledWindow}{wxhvscrolledwindow}

\latexignore{\rtfignore{\wxheading{Members}}}


%%%%%%%%%%%%%%%%%%%%%%%%%%%%%%%%%%%%%%%%%%%%%%%%%%%%%%%%%%%%%%%%%%%%%%%%%%%%%%
%% wxVarVScrollLegacyAdapter Compatibility Functions
%% This section will need to be removed when 2.8 compatibility is removed.
%%%%%%%%%%%%%%%%%%%%%%%%%%%%%%%%%%%%%%%%%%%%%%%%%%%%%%%%%%%%%%%%%%%%%%%%%%%%%%

\membersection{wxWidgets 2.6 and 2.8 Compatibility Functions}\label{wxvscrolledwindowbackwardcompatibility}

The following functions provide backwards compatibility for applications
originally built using wxVScrolledWindow in 2.6 or 2.8. Originally,
wxVScrolledWindow referred to scrolling "lines". We now use "units" in
wxVarScrollHelperBase to avoid implying any orientation (since the functions
are used for both horizontal and vertical scrolling in derived classes). And
in the new wxVScrolledWindow and wxHScrolledWindow classes, we refer to them
as "rows" and "columns", respectively. This is to help clear some confusion
in not only those classes, but also in wxHVScrolledWindow where functions
are inherited from both.

You are encouraged to update any existing code using these function to use
the new replacements mentioned below, and avoid using these functions for
any new code as they are deprecated.

\constfunc{size\_t}{GetFirstVisibleLine}{\void}

Deprecated for \helpref{GetVisibleRowsBegin()}{wxvarvscrollhelpergetvisiblerowsbegin}.

\constfunc{size\_t}{GetLastVisibleLine}{\void}

Deprecated for \helpref{GetVisibleRowsEnd()}{wxvarvscrollhelpergetvisiblerowsend}.

This function originally had a slight design flaw in that it was possible to
return \texttt{(size\_t)-1} (ie: a large positive number) if the scroll
position was 0 and the first line wasn't completely visible.

\constfunc{size\_t}{GetLineCount}{\void}

Deprecated for \helpref{GetRowCount()}{wxvarvscrollhelpergetrowcount}.

\constfunc{virtual wxCoord}{OnGetLineHeight}{\param{size\_t }{line}}

Deprecated for \helpref{OnGetRowHeight()}{wxvarvscrollhelperongetrowheight}.

\constfunc{virtual void}{OnGetLinesHint}{\param{size\_t }{lineMin}, \param{size\_t }{lineMax}}

Deprecated for \helpref{OnGetRowsHeightHint()}{wxvarvscrollhelperongetrowsheighthint}.

\func{virtual void}{RefreshLine}{\param{size\_t }{line}}

Deprecated for \helpref{RefreshRow()}{wxvarvscrollhelperrefreshrow}.

\func{virtual void}{RefreshLines}{\param{size\_t }{from}, \param{size\_t }{to}}

Deprecated for \helpref{RefreshRows()}{wxvarvscrollhelperrefreshrows}.

\func{virtual bool}{ScrollLines}{\param{int }{lines}}

Deprecated for \helpref{ScrollRows()}{wxvarvscrollhelperscrollrows}.

\func{virtual bool}{ScrollPages}{\param{int }{pages}}

Deprecated for \helpref{ScrollRowPages()}{wxvarvscrollhelperscrollrowpages}.

\func{bool}{ScrollToLine}{\param{size\_t }{line}}

Deprecated for \helpref{ScrollToRow()}{wxvarvscrollhelperscrolltorow}.

\func{void}{SetLineCount}{\param{size\_t }{count}}

Deprecated for \helpref{SetRowCount()}{wxvarvscrollhelpersetrowcount}.

%%%%%%%%%%%%%%%%%%%%%%% END COMPATIBILITY FUNCTIONS %%%%%%%%%%%%%%%%%%%%%%%%%%
%%%%%%%%%%%%%%%%%%%%%%%%%%%%%%%%%%%%%%%%%%%%%%%%%%%%%%%%%%%%%%%%%%%%%%%%%%%%%%


\membersection{wxVScrolledWindow::wxVScrolledWindow}\label{wxvscrolledwindowwxvscrolledwindow}

\func{}{wxVScrolledWindow}{\void}

Default constructor, you must call \helpref{Create()}{wxvscrolledwindowcreate}
later.

\func{}{wxVScrolledWindow}{\param{wxWindow* }{parent}, \param{wxWindowID }{id = wxID\_ANY}, \param{const wxPoint\& }{pos = wxDefaultPosition}, \param{const wxSize\& }{size = wxDefaultSize}, \param{long }{style = 0}, \param{const wxString\& }{name = wxPanelNameStr}}

This is the normal constructor, no need to call {\tt Create()} after using this one.

Note that {\tt wxVSCROLL} is always automatically added to our style, there is
no need to specify it explicitly.

\wxheading{Parameters}

\docparam{parent}{The parent window, must not be {\tt NULL}}

\docparam{id}{The identifier of this window, {\tt wxID\_ANY} by default}

\docparam{pos}{The initial window position}

\docparam{size}{The initial window size}

\docparam{style}{The window style. There are no special style bits defined for
this class.}

\docparam{name}{The name for this window; usually not used}


\membersection{wxVScrolledWindow::Create}\label{wxvscrolledwindowcreate}

\func{bool}{Create}{\param{wxWindow* }{parent}, \param{wxWindowID }{id = wxID\_ANY}, \param{const wxPoint\& }{pos = wxDefaultPosition}, \param{const wxSize\& }{size = wxDefaultSize}, \param{long }{style = 0}, \param{const wxString\& }{name = wxPanelNameStr}}

Same as the \helpref{non-default constuctor}{wxvscrolledwindowwxvscrolledwindow}
but returns status code: {\tt true} if ok, {\tt false} if the window couldn't
be created.

Just as with the constructor above, the {\tt wxVSCROLL} style is always used,
there is no need to specify it explicitly.

