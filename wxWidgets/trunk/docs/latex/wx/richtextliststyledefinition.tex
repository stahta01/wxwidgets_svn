\section{\class{wxRichTextListStyleDefinition}}\label{wxrichtextliststyledefinition}

This class represents a list style definition, usually added to a \helpref{wxRichTextStyleSheet}{wxrichtextstylesheet}.

The class inherits paragraph attributes from wxRichTextStyleParagraphDefinition, and adds 10 further attribute objects, one for each level of a list.
When applying a list style to a paragraph, the list style's base and appropriate level attributes are merged with the
paragraph's existing attributes.

You can apply a list style to one or more paragraphs using \helpref{wxRichTextCtrl::SetListStyle}{wxrichtextctrlsetliststyle}. You
can also use the functions \helpref{wxRichTextCtrl::NumberList}{wxrichtextctrlnumberlist}, \helpref{wxRichTextCtrl::PromoteList}{wxrichtextctrlpromotelist} and 
\helpref{wxRichTextCtrl::ClearListStyle}{wxrichtextctrlclearliststyle}. As usual, there are wxRichTextBuffer versions of these functions
so that you can apply them directly to a buffer without requiring a control.

\wxheading{Derived from}

\helpref{wxRichTextParagraphStyleDefinition}{wxrichtextparagraphstyledefinition}

\wxheading{Include files}

<wx/richtext/richtextstyles.h>

\wxheading{Data structures}

\latexignore{\rtfignore{\wxheading{Members}}}

\membersection{wxRichTextListStyleDefinition::wxRichTextListStyleDefinition}\label{wxrichtextliststyledefinitionwxrichtextliststyledefinition}

\func{}{wxRichTextListStyleDefinition}{\param{const wxString\& }{name = wxEmptyString}}

Constructor.

\membersection{wxRichTextListStyleDefinition::\destruct{wxRichTextListStyleDefinition}}\label{wxrichtextliststyledefinitiondtor}

\func{}{\destruct{wxRichTextListStyleDefinition}}{\void}

Destructor.

\membersection{wxRichTextListStyleDefinition::CombineWithParagraphStyle}\label{wxrichtextliststyledefinitioncombinewithparagraphstyle}

\func{wxRichTextAttr}{CombineWithParagraphStyle}{\param{int }{indent}, \param{const wxRichTextAttr\&}{ paraStyle}}

This function combines the given paragraph style with the list style's base attributes and level style matching the given indent, returning the combined attributes.

\membersection{wxRichTextListStyleDefinition::FindLevelForIndent}\label{wxrichtextliststyledefinitionfindlevelforindent}

\constfunc{int}{FindLevelForIndent}{\param{int }{indent}}

This function finds the level (from 0 to 9) whose indentation attribute mostly closely matches {\it indent} (expressed in tenths of a millimetre).

\membersection{wxRichTextListStyleDefinition::GetCombinedStyle}\label{wxrichtextliststyledefinitioncombinewithparagraphstyle}

\constfunc{wxRichTextAttr}{GetCombinedStyle}{\param{int }{indent}}

This function combines the list style's base attributes and the level style matching the given indent, returning the combined attributes.

\membersection{wxRichTextListStyleDefinition::GetLevelAttributes}\label{wxrichtextliststyledefinitiongetlevelattributes}

\constfunc{const wxRichTextAttr*}{GetLevelAttributes}{\param{int }{level}}

Returns the style for the given level. {\it level} is a number between 0 and 9.

\membersection{wxRichTextListStyleDefinition::GetLevelCount}\label{wxrichtextliststyledefinitiongetlevelcount}

\constfunc{int}{GetLevelCount}{\void}

Returns the number of levels. This is hard-wired to 10.

Returns the style for the given level. {\it level} is a number between 0 and 9.

\membersection{wxRichTextListStyleDefinition::IsNumbered}\label{wxrichtextliststyledefinitionisnumbered}

\constfunc{int}{IsNumbered}{\param{int}{ level}}

Returns \true if the given level has numbered list attributes.

\membersection{wxRichTextListStyleDefinition::SetLevelAttributes}\label{wxrichtextliststyledefinitionsetlevelattributes}

\func{void}{SetLevelAttributes}{\param{int }{level}, \param{const wxRichTextAttr\&}{ attr}}

\func{void}{SetLevelAttributes}{\param{int }{level}, \param{int}{ leftIndent}, \param{int}{ leftSubIndent}, \param{int}{ bulletStyle}, \param{const wxString\&}{ bulletSymbol = wxEmptyString}}

Sets the style for the given level. {\it level} is a number between 0 and 9.

The first and most flexible form uses a wxRichTextAttr object, while the second form is for convenient setting of the most commonly-used attributes.

