\documentstyle[a4,11pt,makeidx,verbatim,texhelp,fancyheadings,palatino]{report}
% JACS: doesn't make it through Tex2RTF, sorry. I'll put it into texhelp.sty
% since Tex2RTF doesn't parse it.
% BTW, style MUST be report for it to work for Tex2RTF.
%KB:
%\addtolength{\textwidth}{1in}
%\addtolength{\oddsidemargin}{-0.5in}
%\addtolength{\topmargin}{-0.5in}
%\addtolength{\textheight}{1in}
%\sloppy
%end of my changes
\newcommand{\indexit}[1]{#1\index{#1}}%
\newcommand{\pipe}[0]{$\|$\ }%
\definecolour{black}{0}{0}{0}%
\definecolour{cyan}{0}{255}{255}%
\definecolour{green}{0}{255}{0}%
\definecolour{magenta}{255}{0}{255}%
\definecolour{red}{255}{0}{0}%
\definecolour{blue}{0}{0}{200}%
\definecolour{yellow}{255}{255}{0}%
\definecolour{white}{255}{255}{255}%
%
\input psbox.tex
% Remove this for processing with dvi2ps instead of dvips
%\special{!/@scaleunit 1 def}
\parskip=10pt
\parindent=0pt
\title{wxWindows 2.2 (beta): A portable C++ and Python GUI toolkit}
\winhelponly{\author{by Julian Smart et al
%\winhelponly{\\$$\image{1cm;0cm}{wxwin.wmf}$$}
}}
\winhelpignore{\author{Julian Smart, Robert Roebling, Vadim Zeitlin,
Robin Dunn, et al}
\date{November 20th 1999}
}
\makeindex
\begin{document}
\maketitle
\pagestyle{fancyplain}
\bibliographystyle{plain}
\setheader{{\it CONTENTS}}{}{}{}{}{{\it CONTENTS}}
\setfooter{\thepage}{}{}{}{}{\thepage}%
\pagenumbering{roman}
\tableofcontents

% A special table of contents for the WinHelp manual
\begin{comment}
\winhelponly{
\chapter*{wxWindows class library reference}\label{winhelpcontents}

\center{
%\image{}{wxwin.wmf}
}%

\sethotspotcolour{off}%
\sethotspotunderline{on}%
\large{
\image{}{cpp.bmp} \helpref{Alphabetical class reference}{classref}

\image{}{shelves.bmp} \helpref{Classes by category}{classesbycat}

\image{}{book1.bmp} \helpref{Topic overviews}{overviews}

\image{}{hand1.bmp} \helpref{Guide to wxWindows}{wxwinchapters}
}
\sethotspotcolour{on}%
\sethotspotunderline{on}%

\chapter*{Overview of wxWindows}\label{wxwinchapters}

\helpref{Introduction}{introduction}\\
%\helpref{Resource guide}{resguide}\\
%\helpref{Comparison with other GUI models}{comparison}\\
%\helpref{Multi-platform development with wxWindows}{multiplat}\\
%\helpref{Tutorial}{tutorial}\\
\helpref{The wxWindows resource system}{resourceformats}\\
\helpref{Utilities}{utilities}\\
\helpref{Programming strategies}{strategies}\\
\helpref{Bugs and future directions}{bugs}\\
\helpref{References}{bibliography}
}
\end{comment}

\chapter*{Copyright notice}
\setheader{{\it COPYRIGHT}}{}{}{}{}{{\it COPYRIGHT}}%
\setfooter{\thepage}{}{}{}{}{\thepage}%

\begin{center}
(c)  1999 Julian Smart, Robert Roebling, Vadim Zeitlin and other
members of the wxWindows team\\
Portions (c) 1996 Artificial Intelligence Applications Institute\\
\end{center}

Please also see the wxWindows licence files (preamble.txt, lgpl.txt, gpl.txt, licence.txt,
licendoc.txt) for conditions of software and documentation use.

\begin{verbatim}
wxWindows Library License, Version 3
====================================
     
  Copyright (C) 1998 Julian Smart, Robert Roebling, Vadim Zeitlin et al. 

  Everyone is permitted to copy and distribute verbatim copies
  of this license document, but changing it is not allowed. 
 
                          WXWINDOWS LIBRARY LICENSE 
     TERMS AND CONDITIONS FOR COPYING, DISTRIBUTION AND MODIFICATION 
 
  This library is free software; you can redistribute it and/or modify it 
  under the terms of the GNU Library General Public License as published by 
  the Free Software Foundation; either version 2 of the License, or (at 
  your option) any later version. 
   
  This library is distributed in the hope that it will be useful, but 
  WITHOUT ANY WARRANTY; without even the implied warranty of MERCHAN- 
  TABILITY or FITNESS FOR A PARTICULAR PURPOSE.  See the GNU Library 
  General Public License for more details. 
 
  You should have received a copy of the GNU Library General Public License 
  along with this software, usually in a file named COPYING.LIB.  If not,   
  write to the Free Software Foundation, Inc., 59 Temple Place, Suite 330,  
  Boston, MA 02111-1307 USA. 
 
  EXCEPTION NOTICE 
 
  1. As a special exception, the copyright holders of this library give 
  permission for additional uses of the text contained in this release of 
  the library as licensed under the wxWindows Library License, applying 
  either version 3 of the License, or (at your option) any later version of 
  the License as published by the copyright holders of version 3 of the 
  License document. 
 
  2. The exception is that you may create binary object code versions of any 
  works using this library or based on this library, and use, copy, modify, 
  link and distribute such binary object code files unrestricted under terms 
  of your choice. 
 
  3. If you copy code from files distributed under the terms of the GNU 
  General Public License or the GNU Library General Public License into a 
  copy of this library, as this license permits, the exception does not   
  apply to the code that you add in this way.  To avoid misleading anyone as 
  to the status of such modified files, you must delete this exception 
  notice from such code and/or adjust the licensing conditions notice  
  accordingly. 
 
  4. If you write modifications of your own for this library, it is your 
  choice whether to permit this exception to apply to your modifications. 
  If you do not wish that, you must delete the exception notice from such 
  code and/or adjust the licensing conditions notice accordingly. 


GNU Library General Public License, Version 2
=============================================

 Copyright (C) 1991 Free Software Foundation, Inc.
                    675 Mass Ave, Cambridge, MA 02139, USA
 Everyone is permitted to copy and distribute verbatim copies
 of this license document, but changing it is not allowed.

[This is the first released version of the library GPL.  It is
 numbered 2 because it goes with version 2 of the ordinary GPL.]

			    Preamble

  The licenses for most software are designed to take away your
freedom to share and change it.  By contrast, the GNU General Public
Licenses are intended to guarantee your freedom to share and change
free software--to make sure the software is free for all its users.

  This license, the Library General Public License, applies to some
specially designated Free Software Foundation software, and to any
other libraries whose authors decide to use it.  You can use it for
your libraries, too.

  When we speak of free software, we are referring to freedom, not
price.  Our General Public Licenses are designed to make sure that you
have the freedom to distribute copies of free software (and charge for
this service if you wish), that you receive source code or can get it
if you want it, that you can change the software or use pieces of it
in new free programs; and that you know you can do these things.

  To protect your rights, we need to make restrictions that forbid
anyone to deny you these rights or to ask you to surrender the rights.
These restrictions translate to certain responsibilities for you if
you distribute copies of the library, or if you modify it.

  For example, if you distribute copies of the library, whether gratis
or for a fee, you must give the recipients all the rights that we gave
you.  You must make sure that they, too, receive or can get the source
code.  If you link a program with the library, you must provide
complete object files to the recipients so that they can relink them
with the library, after making changes to the library and recompiling
it.  And you must show them these terms so they know their rights.

  Our method of protecting your rights has two steps: (1) copyright
the library, and (2) offer you this license which gives you legal
permission to copy, distribute and/or modify the library.

  Also, for each distributor's protection, we want to make certain
that everyone understands that there is no warranty for this free
library.  If the library is modified by someone else and passed on, we
want its recipients to know that what they have is not the original
version, so that any problems introduced by others will not reflect on
the original authors' reputations.

  Finally, any free program is threatened constantly by software
patents.  We wish to avoid the danger that companies distributing free
software will individually obtain patent licenses, thus in effect
transforming the program into proprietary software.  To prevent this,
we have made it clear that any patent must be licensed for everyone's
free use or not licensed at all.

  Most GNU software, including some libraries, is covered by the ordinary
GNU General Public License, which was designed for utility programs.  This
license, the GNU Library General Public License, applies to certain
designated libraries.  This license is quite different from the ordinary
one; be sure to read it in full, and don't assume that anything in it is
the same as in the ordinary license.

  The reason we have a separate public license for some libraries is that
they blur the distinction we usually make between modifying or adding to a
program and simply using it.  Linking a program with a library, without
changing the library, is in some sense simply using the library, and is
analogous to running a utility program or application program.  However, in
a textual and legal sense, the linked executable is a combined work, a
derivative of the original library, and the ordinary General Public License
treats it as such.

  Because of this blurred distinction, using the ordinary General
Public License for libraries did not effectively promote software
sharing, because most developers did not use the libraries.  We
concluded that weaker conditions might promote sharing better.

  However, unrestricted linking of non-free programs would deprive the
users of those programs of all benefit from the free status of the
libraries themselves.  This Library General Public License is intended to
permit developers of non-free programs to use free libraries, while
preserving your freedom as a user of such programs to change the free
libraries that are incorporated in them.  (We have not seen how to achieve
this as regards changes in header files, but we have achieved it as regards
changes in the actual functions of the Library.)  The hope is that this
will lead to faster development of free libraries.

  The precise terms and conditions for copying, distribution and
modification follow.  Pay close attention to the difference between a
"work based on the library" and a "work that uses the library".  The
former contains code derived from the library, while the latter only
works together with the library.

  Note that it is possible for a library to be covered by the ordinary
General Public License rather than by this special one.

		  GNU LIBRARY GENERAL PUBLIC LICENSE
   TERMS AND CONDITIONS FOR COPYING, DISTRIBUTION AND MODIFICATION

  0. This License Agreement applies to any software library which
contains a notice placed by the copyright holder or other authorized
party saying it may be distributed under the terms of this Library
General Public License (also called "this License").  Each licensee is
addressed as "you".

  A "library" means a collection of software functions and/or data
prepared so as to be conveniently linked with application programs
(which use some of those functions and data) to form executables.

  The "Library", below, refers to any such software library or work
which has been distributed under these terms.  A "work based on the
Library" means either the Library or any derivative work under
copyright law: that is to say, a work containing the Library or a
portion of it, either verbatim or with modifications and/or translated
straightforwardly into another language.  (Hereinafter, translation is
included without limitation in the term "modification".)

  "Source code" for a work means the preferred form of the work for
making modifications to it.  For a library, complete source code means
all the source code for all modules it contains, plus any associated
interface definition files, plus the scripts used to control compilation
and installation of the library.

  Activities other than copying, distribution and modification are not
covered by this License; they are outside its scope.  The act of
running a program using the Library is not restricted, and output from
such a program is covered only if its contents constitute a work based
on the Library (independent of the use of the Library in a tool for
writing it).  Whether that is true depends on what the Library does
and what the program that uses the Library does.
  
  1. You may copy and distribute verbatim copies of the Library's
complete source code as you receive it, in any medium, provided that
you conspicuously and appropriately publish on each copy an
appropriate copyright notice and disclaimer of warranty; keep intact
all the notices that refer to this License and to the absence of any
warranty; and distribute a copy of this License along with the
Library.

  You may charge a fee for the physical act of transferring a copy,
and you may at your option offer warranty protection in exchange for a
fee.

  2. You may modify your copy or copies of the Library or any portion
of it, thus forming a work based on the Library, and copy and
distribute such modifications or work under the terms of Section 1
above, provided that you also meet all of these conditions:

    a) The modified work must itself be a software library.

    b) You must cause the files modified to carry prominent notices
    stating that you changed the files and the date of any change.

    c) You must cause the whole of the work to be licensed at no
    charge to all third parties under the terms of this License.

    d) If a facility in the modified Library refers to a function or a
    table of data to be supplied by an application program that uses
    the facility, other than as an argument passed when the facility
    is invoked, then you must make a good faith effort to ensure that,
    in the event an application does not supply such function or
    table, the facility still operates, and performs whatever part of
    its purpose remains meaningful.

    (For example, a function in a library to compute square roots has
    a purpose that is entirely well-defined independent of the
    application.  Therefore, Subsection 2d requires that any
    application-supplied function or table used by this function must
    be optional: if the application does not supply it, the square
    root function must still compute square roots.)

These requirements apply to the modified work as a whole.  If
identifiable sections of that work are not derived from the Library,
and can be reasonably considered independent and separate works in
themselves, then this License, and its terms, do not apply to those
sections when you distribute them as separate works.  But when you
distribute the same sections as part of a whole which is a work based
on the Library, the distribution of the whole must be on the terms of
this License, whose permissions for other licensees extend to the
entire whole, and thus to each and every part regardless of who wrote
it.

Thus, it is not the intent of this section to claim rights or contest
your rights to work written entirely by you; rather, the intent is to
exercise the right to control the distribution of derivative or
collective works based on the Library.

In addition, mere aggregation of another work not based on the Library
with the Library (or with a work based on the Library) on a volume of
a storage or distribution medium does not bring the other work under
the scope of this License.

  3. You may opt to apply the terms of the ordinary GNU General Public
License instead of this License to a given copy of the Library.  To do
this, you must alter all the notices that refer to this License, so
that they refer to the ordinary GNU General Public License, version 2,
instead of to this License.  (If a newer version than version 2 of the
ordinary GNU General Public License has appeared, then you can specify
that version instead if you wish.)  Do not make any other change in
these notices.

  Once this change is made in a given copy, it is irreversible for
that copy, so the ordinary GNU General Public License applies to all
subsequent copies and derivative works made from that copy.

  This option is useful when you wish to copy part of the code of
the Library into a program that is not a library.

  4. You may copy and distribute the Library (or a portion or
derivative of it, under Section 2) in object code or executable form
under the terms of Sections 1 and 2 above provided that you accompany
it with the complete corresponding machine-readable source code, which
must be distributed under the terms of Sections 1 and 2 above on a
medium customarily used for software interchange.

  If distribution of object code is made by offering access to copy
from a designated place, then offering equivalent access to copy the
source code from the same place satisfies the requirement to
distribute the source code, even though third parties are not
compelled to copy the source along with the object code.

  5. A program that contains no derivative of any portion of the
Library, but is designed to work with the Library by being compiled or
linked with it, is called a "work that uses the Library".  Such a
work, in isolation, is not a derivative work of the Library, and
therefore falls outside the scope of this License.

  However, linking a "work that uses the Library" with the Library
creates an executable that is a derivative of the Library (because it
contains portions of the Library), rather than a "work that uses the
library".  The executable is therefore covered by this License.
Section 6 states terms for distribution of such executables.

  When a "work that uses the Library" uses material from a header file
that is part of the Library, the object code for the work may be a
derivative work of the Library even though the source code is not.
Whether this is true is especially significant if the work can be
linked without the Library, or if the work is itself a library.  The
threshold for this to be true is not precisely defined by law.

  If such an object file uses only numerical parameters, data
structure layouts and accessors, and small macros and small inline
functions (ten lines or less in length), then the use of the object
file is unrestricted, regardless of whether it is legally a derivative
work.  (Executables containing this object code plus portions of the
Library will still fall under Section 6.)

  Otherwise, if the work is a derivative of the Library, you may
distribute the object code for the work under the terms of Section 6.
Any executables containing that work also fall under Section 6,
whether or not they are linked directly with the Library itself.

  6. As an exception to the Sections above, you may also compile or
link a "work that uses the Library" with the Library to produce a
work containing portions of the Library, and distribute that work
under terms of your choice, provided that the terms permit
modification of the work for the customer's own use and reverse
engineering for debugging such modifications.

  You must give prominent notice with each copy of the work that the
Library is used in it and that the Library and its use are covered by
this License.  You must supply a copy of this License.  If the work
during execution displays copyright notices, you must include the
copyright notice for the Library among them, as well as a reference
directing the user to the copy of this License.  Also, you must do one
of these things:

    a) Accompany the work with the complete corresponding
    machine-readable source code for the Library including whatever
    changes were used in the work (which must be distributed under
    Sections 1 and 2 above); and, if the work is an executable linked
    with the Library, with the complete machine-readable "work that
    uses the Library", as object code and/or source code, so that the
    user can modify the Library and then relink to produce a modified
    executable containing the modified Library.  (It is understood
    that the user who changes the contents of definitions files in the
    Library will not necessarily be able to recompile the application
    to use the modified definitions.)

    b) Accompany the work with a written offer, valid for at
    least three years, to give the same user the materials
    specified in Subsection 6a, above, for a charge no more
    than the cost of performing this distribution.

    c) If distribution of the work is made by offering access to copy
    from a designated place, offer equivalent access to copy the above
    specified materials from the same place.

    d) Verify that the user has already received a copy of these
    materials or that you have already sent this user a copy.

  For an executable, the required form of the "work that uses the
Library" must include any data and utility programs needed for
reproducing the executable from it.  However, as a special exception,
the source code distributed need not include anything that is normally
distributed (in either source or binary form) with the major
components (compiler, kernel, and so on) of the operating system on
which the executable runs, unless that component itself accompanies
the executable.

  It may happen that this requirement contradicts the license
restrictions of other proprietary libraries that do not normally
accompany the operating system.  Such a contradiction means you cannot
use both them and the Library together in an executable that you
distribute.

  7. You may place library facilities that are a work based on the
Library side-by-side in a single library together with other library
facilities not covered by this License, and distribute such a combined
library, provided that the separate distribution of the work based on
the Library and of the other library facilities is otherwise
permitted, and provided that you do these two things:

    a) Accompany the combined library with a copy of the same work
    based on the Library, uncombined with any other library
    facilities.  This must be distributed under the terms of the
    Sections above.

    b) Give prominent notice with the combined library of the fact
    that part of it is a work based on the Library, and explaining
    where to find the accompanying uncombined form of the same work.

  8. You may not copy, modify, sublicense, link with, or distribute
the Library except as expressly provided under this License.  Any
attempt otherwise to copy, modify, sublicense, link with, or
distribute the Library is void, and will automatically terminate your
rights under this License.  However, parties who have received copies,
or rights, from you under this License will not have their licenses
terminated so long as such parties remain in full compliance.

  9. You are not required to accept this License, since you have not
signed it.  However, nothing else grants you permission to modify or
distribute the Library or its derivative works.  These actions are
prohibited by law if you do not accept this License.  Therefore, by
modifying or distributing the Library (or any work based on the
Library), you indicate your acceptance of this License to do so, and
all its terms and conditions for copying, distributing or modifying
the Library or works based on it.

  10. Each time you redistribute the Library (or any work based on the
Library), the recipient automatically receives a license from the
original licensor to copy, distribute, link with or modify the Library
subject to these terms and conditions.  You may not impose any further
restrictions on the recipients' exercise of the rights granted herein.
You are not responsible for enforcing compliance by third parties to
this License.

  11. If, as a consequence of a court judgment or allegation of patent
infringement or for any other reason (not limited to patent issues),
conditions are imposed on you (whether by court order, agreement or
otherwise) that contradict the conditions of this License, they do not
excuse you from the conditions of this License.  If you cannot
distribute so as to satisfy simultaneously your obligations under this
License and any other pertinent obligations, then as a consequence you
may not distribute the Library at all.  For example, if a patent
license would not permit royalty-free redistribution of the Library by
all those who receive copies directly or indirectly through you, then
the only way you could satisfy both it and this License would be to
refrain entirely from distribution of the Library.

If any portion of this section is held invalid or unenforceable under any
particular circumstance, the balance of the section is intended to apply,
and the section as a whole is intended to apply in other circumstances.

It is not the purpose of this section to induce you to infringe any
patents or other property right claims or to contest validity of any
such claims; this section has the sole purpose of protecting the
integrity of the free software distribution system which is
implemented by public license practices.  Many people have made
generous contributions to the wide range of software distributed
through that system in reliance on consistent application of that
system; it is up to the author/donor to decide if he or she is willing
to distribute software through any other system and a licensee cannot
impose that choice.

This section is intended to make thoroughly clear what is believed to
be a consequence of the rest of this License.

  12. If the distribution and/or use of the Library is restricted in
certain countries either by patents or by copyrighted interfaces, the
original copyright holder who places the Library under this License may add
an explicit geographical distribution limitation excluding those countries,
so that distribution is permitted only in or among countries not thus
excluded.  In such case, this License incorporates the limitation as if
written in the body of this License.

  13. The Free Software Foundation may publish revised and/or new
versions of the Library General Public License from time to time.
Such new versions will be similar in spirit to the present version,
but may differ in detail to address new problems or concerns.

Each version is given a distinguishing version number.  If the Library
specifies a version number of this License which applies to it and
"any later version", you have the option of following the terms and
conditions either of that version or of any later version published by
the Free Software Foundation.  If the Library does not specify a
license version number, you may choose any version ever published by
the Free Software Foundation.

  14. If you wish to incorporate parts of the Library into other free
programs whose distribution conditions are incompatible with these,
write to the author to ask for permission.  For software which is
copyrighted by the Free Software Foundation, write to the Free
Software Foundation; we sometimes make exceptions for this.  Our
decision will be guided by the two goals of preserving the free status
of all derivatives of our free software and of promoting the sharing
and reuse of software generally.

			    NO WARRANTY

  15. BECAUSE THE LIBRARY IS LICENSED FREE OF CHARGE, THERE IS NO
WARRANTY FOR THE LIBRARY, TO THE EXTENT PERMITTED BY APPLICABLE LAW.
EXCEPT WHEN OTHERWISE STATED IN WRITING THE COPYRIGHT HOLDERS AND/OR
OTHER PARTIES PROVIDE THE LIBRARY "AS IS" WITHOUT WARRANTY OF ANY
KIND, EITHER EXPRESSED OR IMPLIED, INCLUDING, BUT NOT LIMITED TO, THE
IMPLIED WARRANTIES OF MERCHANTABILITY AND FITNESS FOR A PARTICULAR
PURPOSE.  THE ENTIRE RISK AS TO THE QUALITY AND PERFORMANCE OF THE
LIBRARY IS WITH YOU.  SHOULD THE LIBRARY PROVE DEFECTIVE, YOU ASSUME
THE COST OF ALL NECESSARY SERVICING, REPAIR OR CORRECTION.

  16. IN NO EVENT UNLESS REQUIRED BY APPLICABLE LAW OR AGREED TO IN
WRITING WILL ANY COPYRIGHT HOLDER, OR ANY OTHER PARTY WHO MAY MODIFY
AND/OR REDISTRIBUTE THE LIBRARY AS PERMITTED ABOVE, BE LIABLE TO YOU
FOR DAMAGES, INCLUDING ANY GENERAL, SPECIAL, INCIDENTAL OR
CONSEQUENTIAL DAMAGES ARISING OUT OF THE USE OR INABILITY TO USE THE
LIBRARY (INCLUDING BUT NOT LIMITED TO LOSS OF DATA OR DATA BEING
RENDERED INACCURATE OR LOSSES SUSTAINED BY YOU OR THIRD PARTIES OR A
FAILURE OF THE LIBRARY TO OPERATE WITH ANY OTHER SOFTWARE), EVEN IF
SUCH HOLDER OR OTHER PARTY HAS BEEN ADVISED OF THE POSSIBILITY OF SUCH
DAMAGES.

		     END OF TERMS AND CONDITIONS

     Appendix: How to Apply These Terms to Your New Libraries

  If you develop a new library, and you want it to be of the greatest
possible use to the public, we recommend making it free software that
everyone can redistribute and change.  You can do so by permitting
redistribution under these terms (or, alternatively, under the terms of the
ordinary General Public License).

  To apply these terms, attach the following notices to the library.  It is
safest to attach them to the start of each source file to most effectively
convey the exclusion of warranty; and each file should have at least the
"copyright" line and a pointer to where the full notice is found.

    <one line to give the library's name and a brief idea of what it does.>
    Copyright (C) <year>  <name of author>

    This library is free software; you can redistribute it and/or
    modify it under the terms of the GNU Library General Public
    License as published by the Free Software Foundation; either
    version 2 of the License, or (at your option) any later version.

    This library is distributed in the hope that it will be useful,
    but WITHOUT ANY WARRANTY; without even the implied warranty of
    MERCHANTABILITY or FITNESS FOR A PARTICULAR PURPOSE.  See the GNU
    Library General Public License for more details.

    You should have received a copy of the GNU Library General Public
    License along with this library; if not, write to the Free
    Software Foundation, Inc., 675 Mass Ave, Cambridge, MA 02139, USA.

Also add information on how to contact you by electronic and paper mail.

You should also get your employer (if you work as a programmer) or your
school, if any, to sign a "copyright disclaimer" for the library, if
necessary.  Here is a sample; alter the names:

  Yoyodyne, Inc., hereby disclaims all copyright interest in the
  library `Frob' (a library for tweaking knobs) written by James Random Hacker.

  <signature of Ty Coon>, 1 April 1990
  Ty Coon, President of Vice

That's all there is to it!

\end{verbatim}


\chapter{Introduction}\label{introduction}
\pagenumbering{arabic}%
\setheader{{\it CHAPTER \thechapter}}{}{}{}{}{{\it CHAPTER \thechapter}}%
\setfooter{\thepage}{}{}{}{}{\thepage}%

\section{What is wxWindows?}

wxWindows is a C++ framework providing GUI (Graphical User
Interface) and other facilities on more than one platform.  Version 2.0 currently
supports MS Windows (16-bit, Windows 95 and Windows NT), Unix with GTK+, and Unix with Motif.
A Mac port is in an advanced state, an OS/2 port and a port to the MGL graphics library
have been started.

wxWindows was originally developed at the Artificial Intelligence
Applications Institute, University of Edinburgh, for internal use.
wxWindows has been released into the public domain in the hope
that others will also find it useful. Version 2.0 is written and
maintained by Julian Smart, Robert Roebling, Vadim Zeitlin and others.

This manual discusses wxWindows in the context of multi-platform
development.\helpignore{For more detail on the wxWindows version 2.0 API
(Application Programming Interface) please refer to the separate
wxWindows reference manual.}

Please note that in the following, ``MS Windows" often refers to all
platforms related to Microsoft Windows, including 16-bit and 32-bit
variants, unless otherwise stated. All trademarks are acknowledged.

\section{Why another cross-platform development tool?}

wxWindows was developed to provide a cheap and flexible way to maximize
investment in GUI application development.  While a number of commercial
class libraries already existed for cross-platform development,
none met all of the following criteria:

\begin{enumerate}\itemsep=0pt
\item low price;
\item source availability;
\item simplicity of programming;
\item support for a wide range of compilers.
\end{enumerate}

Since wxWindows was started, several other free or almost-free GUI frameworks have
emerged. However, none has the range of features, flexibility, documentation and the
well-established development team that wxWindows has.

As public domain software and a project open to everyone, wxWindows has
benefited from comments, ideas, bug fixes, enhancements and the sheer
enthusiasm of users, especially via the Internet. This gives wxWindows a
certain advantage over its commercial competitors (and over free libraries
without an independent development team), plus a robustness against
the transience of one individual or company. This openness and
availability of source code is especially important when the future of
thousands of lines of application code may depend upon the longevity of
the underlying class library.

Version 2.0 goes much further than previous versions in terms of generality and features,
allowing applications to be produced
that are often indistinguishable from those produced using single-platform
toolkits such as Motif and MFC.

The importance of using a platform-independent class library cannot be
overstated, since GUI application development is very time-consuming,
and sustained popularity of particular GUIs cannot be guaranteed.
Code can very quickly become obsolete if it addresses the wrong
platform or audience.  wxWindows helps to insulate the programmer from
these winds of change. Although wxWindows may not be suitable for
every application (such as an OLE-intensive program), it provides access to most of the functionality a
GUI program normally requires, plus some extras such as network programming
and PostScript output, and can of course be extended as needs dictate.  As a bonus, it provides
a cleaner programming interface than the native
APIs. Programmers may find it worthwhile to use wxWindows even if they
are developing on only one platform.

It is impossible to sum up the functionality of wxWindows in a few paragraphs, but
here are some of the benefits:

\begin{itemize}\itemsep=0pt
\item Low cost (free, in fact!)
\item You get the source.
\item Available on a variety of popular platforms.
\item Works with almost all popular C++ compilers.
\item Several example programs.
\item Over 900 pages of printable and on-line documentation.
\item Includes Tex2RTF, to allow you to produce your own documentation
in Windows Help, HTML and Word RTF formats.
\item Simple-to-use, object-oriented API.
\item Flexible event system.
\item Graphics calls include lines, rounded rectangles, splines, polylines, etc.
\item Constraint-based layout option.
\item Print/preview and document/view architectures.
\item Toolbar, notebook, tree control, advanced list control classes.
\item PostScript generation under Unix, normal MS Windows printing on the
PC.
\item MDI (Multiple Document Interface) support.
\item Can be used to create DLLs under Windows, dynamic libraries on Unix.
\item Common dialogs for file browsing, printing, colour selection, etc.
\item Under MS Windows, support for creating metafiles and copying
them to the clipboard.
\item An API for invoking help from applications.
\item Dialog Editor for building dialogs.
\item Network support via a family of socket and protocol classes.
\end{itemize}

\section{Changes from version 1.xx}\label{versionchanges}

These are a few of the major differences between versions 1.xx and 2.0.

Removals:

\begin{itemize}\itemsep=0pt
\item XView is no longer supported;
\item all controls (panel items) no longer have labels attached to them;
\item wxForm has been removed;
\item wxCanvasDC, wxPanelDC removed (replaced by wxClientDC, wxWindowDC, wxPaintDC which
can be used for any window);
\item wxMultiText, wxTextWindow, wxText removed and replaced by wxTextCtrl;
\item classes no longer divided into generic and platform-specific parts, for efficiency.
\end{itemize}

Additions and changes:

\begin{itemize}\itemsep=0pt
\item class hierarchy changed, and restrictions about subwindow nesting lifted;
\item header files reorganised to conform to normal C++ standards;
\item classes less dependent on each another, to reduce executable size;
\item wxString used instead of char* wherever possible;
\item the number of separate but mandatory utilities reduced;
\item the event system has been overhauled, with
virtual functions and callbacks being replaced with MFC-like event tables;
\item new controls, such as wxTreeCtrl, wxListCtrl, wxSpinButton;
\item less inconsistency about what events can be handled, so for example
mouse clicks or key presses on controls can now be intercepted;
\item the status bar is now a separate class, wxStatusBar, and is
implemented in generic wxWindows code;
\item some renaming of controls for greater consistency;
\item wxBitmap has the notion of bitmap handlers to allow for extension to new formats
without ifdefing;
\item new dialogs: wxPageSetupDialog, wxFileDialog, wxDirDialog,
wxMessageDialog, wxSingleChoiceDialog, wxTextEntryDialog;
\item GDI objects are reference-counted and are now passed to most functions
by reference, making memory management far easier;
\item wxSystemSettings class allows querying for various system-wide properties
such as dialog font, colours, user interface element sizes, and so on;
\item better platform look and feel conformance;
\item toolbar functionality now separated out into a family of classes with the
same API;
\item device contexts are no longer accessed using wxWindow::GetDC - they are created
temporarily with the window as an argument;
\item events from sliders and scrollbars can be handled more flexibly;
\item the handling of window close events has been changed in line with the new
event system;
\item the concept of {\it validator} has been added to allow much easier coding of
the relationship between controls and application data;
\item the documentation has been revised, with more cross-referencing.
\end{itemize}

Platform-specific changes:

\begin{itemize}\itemsep=0pt
\item The Windows header file (windows.h) is no longer included by wxWindows headers;
\item wx.dll supported under Visual C++;
\item the full range of Windows 95 window decorations are supported, such as modal frame
borders;
\item MDI classes brought out of wxFrame into separate classes, and made more flexible.
\end{itemize}

\section{wxWindows requirements}\label{requirements}

To make use of wxWindows, you currently need one or both of the
following setups.

(a) PC:

\begin{enumerate}\itemsep=0pt
\item A 486 or higher PC running MS Windows.
\item A Windows compiler: most are supported, but please see {\tt install.txt} for
details. Supported compilers include Microsoft Visual C++ 4.0 or higher, Borland C++, Cygwin,
Metrowerks CodeWarrior.
\item At least 60 MB of disk space.
\end{enumerate}

(b) Unix:

\begin{enumerate}\itemsep=0pt
\item Almost any C++ compiler, including GNU C++ (EGCS 1.1.1 or above).
\item Almost any Unix workstation, and one of: GTK+ 1.0, GTK+ 1.2, Motif 1.2 or higher, Lesstif.
\item At least 60 MB of disk space.
\end{enumerate}

\section{Availability and location of wxWindows}

wxWindows is currently available from the Artificial Intelligence
Applications Institute by anonymous FTP and World Wide Web:

\begin{verbatim}
  ftp://www.remstar.com/pub/wxwin
  http://www.wxwindows.org
\end{verbatim}

\section{Acknowledgments}

Thanks are due to AIAI for being willing to release the original version of
wxWindows into the public domain, and to our patient partners.

We would particularly like to thank the following for their contributions to wxWindows, and the many others who have been involved in
the project over the years. Apologies for any unintentional omissions from this list. 
 
Yiorgos Adamopoulos, Jamshid Afshar, Alejandro Aguilar-Sierra, AIAI, Patrick Albert, Karsten Ballueder, Michael Bedward, Kai Bendorf, Yura Bidus, Keith 
Gary Boyce, Chris Breeze, Pete Britton, Ian Brown, C. Buckley, Dmitri Chubraev, Robin Corbet, Cecil Coupe, Andrew Davison, Neil Dudman, Robin 
Dunn, Hermann Dunkel, Jos van Eijndhoven, Tom Felici, Thomas Fettig, Matthew Flatt, Pasquale Foggia, Josep Fortiana, Todd Fries, Dominic Gallagher, 
Wolfram Gloger, Norbert Grotz, Stefan Gunter, Bill Hale, Patrick Halke, Stefan Hammes, Guillaume Helle, Harco de Hilster, Cord Hockemeyer, Markus 
Holzem, Olaf Klein, Leif Jensen, Bart Jourquin, Guilhem Lavaux, Jan Lessner, Nicholas Liebmann, Torsten Liermann, Per Lindqvist, Thomas Runge, Tatu
M\"{a}nnist\"{o}, Scott Maxwell, Thomas Myers, Oliver Niedung, Hernan Otero, Ian Perrigo, Timothy Peters, Giordano Pezzoli, Harri Pasanen, Thomaso Paoletti, 
Garrett Potts, Marcel Rasche, Robert Roebling, Dino Scaringella, Jobst Schmalenbach, Arthur Seaton, Paul Shirley, Stein Somers, Petr Smilauer, Neil Smith, 
Kari Syst\"{a}, Arthur Tetzlaff-Deas, Jonathan Tonberg, Jyrki Tuomi, Janos Vegh, Andrea Venturoli, Vadim Zeitlin, Xiaokun Zhu, Edward Zimmermann.

`Graphplace', the basis for the wxGraphLayout library, is copyright Dr. Jos
T.J. van Eijndhoven of Eindhoven University of Technology. The code has
been used in wxGraphLayout with his permission.

We also acknowledge the author of XFIG, the excellent Unix drawing tool,
from the source of which we have borrowed some spline drawing code.
His copyright is included below.

{\it XFig2.1 is copyright (c) 1985 by Supoj Sutanthavibul. Permission to
use, copy, modify, distribute, and sell this software and its
documentation for any purpose is hereby granted without fee, provided
that the above copyright notice appear in all copies and that both that
copyright notice and this permission notice appear in supporting
documentation, and that the name of M.I.T. not be used in advertising or
publicity pertaining to distribution of the software without specific,
written prior permission.  M.I.T. makes no representations about the
suitability of this software for any purpose.  It is provided ``as is''
without express or implied warranty.}

\chapter{Multi-platform development with wxWindows}\label{multiplat}
\setheader{{\it CHAPTER \thechapter}}{}{}{}{}{{\it CHAPTER \thechapter}}%
\setfooter{\thepage}{}{}{}{}{\thepage}%

This chapter describes the practical details of using wxWindows. Please
see the file install.txt for up-to-date installation instructions, and
changes.txt for differences between versions.

\section{Include files}

The main include file is {\tt "wx/wx.h"}; this includes the most commonly
used modules of wxWindows.

To save on compilation time, include only those header files relevant to the
source file. If you are using precompiled headers, you should include
the following section before any other includes:

\begin{verbatim}
// For compilers that support precompilation, includes "wx.h".
#include <wx/wxprec.h>

#ifdef __BORLANDC__
#pragma hdrstop
#endif

#ifndef WX_PRECOMP
// Include your minimal set of headers here, or wx.h
#include <wx/wx.h>
#endif

... now your other include files ...
\end{verbatim}

The file {\tt "wx/wxprec.h"} includes {\tt "wx/wx.h"}. Although this incantation
may seem quirky, it is in fact the end result of a lot of experimentation,
and several Windows compilers to use precompilation (those tested are Microsoft Visual C++, Borland C++
and Watcom C++).

Borland precompilation is largely automatic. Visual C++ requires specification of {\tt "wx/wxprec.h"} as
the file to use for precompilation. Watcom C++ is automatic apart from the specification of
the .pch file. Watcom C++ is strange in requiring the precompiled header to be used only for
object files compiled in the same directory as that in which the precompiled header was created.
Therefore, the wxWindows Watcom C++ makefiles go through hoops deleting and recreating
a single precompiled header file for each module, thus preventing an accumulation of many
multi-megabyte .pch files.

\section{Libraries}

Please the wxGTK or wxMotif documentation for use of the Unix version of wxWindows.
Under Windows, use the library wx.lib for stand-alone Windows
applications, or wxdll.lib for creating DLLs.

\section{Configuration}

Options are configurable in the file
\rtfsp{\tt "wx/XXX/setup.h"} where XXX is the required platform (such as msw, motif, gtk, mac). Some settings are a matter
of taste, some help with platform-specific problems, and
others can be set to minimize the size of the library. Please see the setup.h file
and {\tt install.txt} files for details on configuration.

\section{Makefiles}

At the moment there is no attempt to make Unix makefiles and
PC makefiles compatible, i.e. one makefile is required for
each environment. wxGTK has its own configure system which can also
be used with wxMotif, although wxMotif has a simple makefile system of its own.

Sample makefiles for Unix (suffix .UNX), MS C++ (suffix .DOS and .NT), Borland
C++ (.BCC and .B32) and Symantec C++ (.SC) are included for the library, demos
and utilities.

The controlling makefile for wxWindows is in the platform-specific
directory, such as {\tt src/msw} or {\tt src/motif}.

Please see the platform-specific {\tt install.txt} file for further details.

\section{Windows-specific files}

wxWindows application compilation under MS Windows requires at least two
extra files, resource and module definition files.

\subsection{Resource file}\label{resources}

The least that must be defined in the Windows resource file (extension RC)
is the following statement:

\begin{verbatim}
rcinclude "wx/msw/wx.rc"
\end{verbatim}

which includes essential internal wxWindows definitions.  The resource script
may also contain references to icons, cursors, etc., for example:

\begin{verbatim}
wxicon icon wx.ico
\end{verbatim}

The icon can then be referenced by name when creating a frame icon. See
the MS Windows SDK documentation.

\normalbox{Note: include wx.rc {\it after} any ICON statements
so programs that search your executable for icons (such
as the Program Manager) find your application icon first.}

\subsection{Module definition file}

A module definition file (extension DEF) is required for 16-bit applications, and
looks like the following:

\begin{verbatim}
NAME         Hello
DESCRIPTION  'Hello'
EXETYPE      WINDOWS
STUB         'WINSTUB.EXE'
CODE         PRELOAD MOVEABLE DISCARDABLE
DATA         PRELOAD MOVEABLE MULTIPLE
HEAPSIZE     1024
STACKSIZE    8192
\end{verbatim}

The only lines which will usually have to be changed per application are
NAME and DESCRIPTION.

\section{Allocating and deleting wxWindows objects}

In general, classes derived from wxWindow must dynamically allocated
with {\it new} and deleted with {\it delete}. If you delete a window,
all of its children and descendants will be automatically deleted,
so you don't need to delete these descendants explicitly.

When deleting a frame or dialog, use {\bf Destroy} rather than {\bf delete} so
that the wxWindows delayed deletion can take effect. This waits until idle time
(when all messages have been processed) to actually delete the window, to avoid
problems associated with the GUI sending events to deleted windows.

Don't create a window on the stack, because this will interfere
with delayed deletion.

If you decide to allocate a C++ array of objects (such as wxBitmap) that may
be cleaned up by wxWindows, make sure you delete the array explicitly
before wxWindows has a chance to do so on exit, since calling {\it delete} on
array members will cause memory problems.

wxColour can be created statically: it is not automatically cleaned
up and is unlikely to be shared between other objects; it is lightweight
enough for copies to be made.

Beware of deleting objects such as a wxPen or wxBitmap if they are still in use.
Windows is particularly sensitive to this: so make sure you
make calls like wxDC::SetPen(wxNullPen) or wxDC::SelectObject(wxNullBitmap) before deleting
a drawing object that may be in use. Code that doesn't do this will probably work
fine on some platforms, and then fail under Windows.

\section{Architecture dependency}

A problem which sometimes arises from writing multi-platform programs is that
the basic C types are not defiend the same on all platforms. This holds true
for both the length in bits of the standard types (such as int and long) as 
well as their byte order, which might be little endian (typically
on Intel computers) or big endian (typically on some Unix workstations). wxWindows
defines types and macros that make it easy to write architecture independent
code. The types are:

wxInt32, wxInt16, wxInt8, wxUint32, wxUint16 = wxWord, wxUint8 = wxByte

where wxInt32 stands for a 32-bit signed integer type etc. You can also check
which architecture the program is compiled on using the wxBYTE\_ORDER define
which is either wxBIG\_ENDIAN or wxLITTLE\_ENDIAN (in the future maybe wxPDP\_ENDIAN
as well).

The macros handling bit-swapping with respect to the applications endianness
are described in the \helpref{Macros}{macros} section.

\section{Conditional compilation}

One of the purposes of wxWindows is to reduce the need for conditional
compilation in source code, which can be messy and confusing to follow.
However, sometimes it is necessary to incorporate platform-specific
features (such as metafile use under MS Windows). The symbols
listed in the file {\tt symbols.txt} may be used for this purpose,
along with any user-supplied ones.

\section{C++ issues}

The following documents some miscellaneous C++ issues.

\subsection{Templates}

wxWindows does not use templates since it is a notoriously unportable feature.

\subsection{RTTI}

wxWindows does not use run-time type information since wxWindows provides
its own run-time type information system, implemented using macros.

\subsection{Type of NULL}

Some compilers (e.g. the native IRIX cc) define NULL to be 0L so that
no conversion to pointers is allowed. Because of that, all these
occurences of NULL in the GTK port use an explicit conversion such 
as

{\small
\begin{verbatim}
  wxWindow *my_window = (wxWindow*) NULL;
\end{verbatim}
}

It is recommended to adhere to this in all code using wxWindows as
this make the code (a bit) more portable.

\subsection{Precompiled headers}

Some compilers, such as Borland C++ and Microsoft C++, support
precompiled headers. This can save a great deal of compiling time. The
recommended approach is to precompile {\tt "wx.h"}, using this
precompiled header for compiling both wxWindows itself and any
wxWindows applications. For Windows compilers, two dummy source files
are provided (one for normal applications and one for creating DLLs)
to allow initial creation of the precompiled header.

However, there are several downsides to using precompiled headers. One
is that to take advantage of the facility, you often need to include
more header files than would normally be the case. This means that
changing a header file will cause more recompilations (in the case of
wxWindows, everything needs to be recompiled since everything includes {\tt "wx.h"}!)

A related problem is that for compilers that don't have precompiled
headers, including a lot of header files slows down compilation
considerably. For this reason, you will find (in the common
X and Windows parts of the library) conditional
compilation that under Unix, includes a minimal set of headers;
and when using Visual C++, includes {\tt wx.h}. This should help provide
the optimal compilation for each compiler, although it is
biassed towards the precompiled headers facility available
in Microsoft C++.

\section{File handling}

When building an application which may be used under different
environments, one difficulty is coping with documents which may be
moved to different directories on other machines. Saving a file which
has pointers to full pathnames is going to be inherently unportable. One
approach is to store filenames on their own, with no directory
information.  The application searches through a number of locally
defined directories to find the file. To support this, the class {\bf
wxPathList} makes adding directories and searching for files easy, and
the global function {\bf wxFileNameFromPath} allows the application to
strip off the filename from the path if the filename must be stored.
This has undesirable ramifications for people who have documents of the
same name in different directories.

As regards the limitations of DOS 8+3 single-case filenames versus
unrestricted Unix filenames, the best solution is to use DOS filenames
for your application, and also for document filenames {\it if} the user
is likely to be switching platforms regularly. Obviously this latter
choice is up to the application user to decide.  Some programs (such as
YACC and LEX) generate filenames incompatible with DOS; the best
solution here is to have your Unix makefile rename the generated files
to something more compatible before transferring the source to DOS.
Transferring DOS files to Unix is no problem, of course, apart from EOL
conversion for which there should be a utility available (such as
dos2unix).

See also the File Functions section of the reference manual for
descriptions of miscellaneous file handling functions.

\begin{comment}
\chapter{Utilities supplied with wxWindows}\label{utilities}
\setheader{{\it CHAPTER \thechapter}}{}{}{}{}{{\it CHAPTER \thechapter}}%
\setfooter{\thepage}{}{}{}{}{\thepage}%

A number of `extras' are supplied with wxWindows, to complement
the GUI functionality in the main class library. These are found
below the utils directory and usually have their own source, library
and documentation directories. For other user-contributed packages,
see the directory ftp://www.remstar.com/pub/wxwin/contrib, which is
more easily accessed via the Contributions page on the Web site.

\section{wxHelp}\label{wxhelp}

wxHelp is a stand-alone program, written using wxWindows,
for displaying hypertext help. It is necessary since not all target
systems (notably X) supply an adequate
standard for on-line help. wxHelp is modelled on the MS Windows help
system, with contents, search and browse buttons, but does not reformat
text to suit the size of window, as WinHelp does, and its input files
are uncompressed ASCII with some embedded font commands and an .xlp
extension. Most wxWindows documentation (user manuals and class
references) is supplied in wxHelp format, and also in Windows Help
format. The wxWindows 2.0 project will presently use an HTML widget
in a new and improved wxHelp implementation, under X.

Note that an application can be programmed to use Windows Help under
MS Windows, and wxHelp under X. An alternative help viewer under X is
Mosaic, a World Wide Web viewer that uses HTML as its native hypertext
format. However, this is not currently integrated with wxWindows
applications.

wxHelp works in two modes---edit and end-user. In edit mode, an ASCII
file may be marked up with different fonts and colours, and divided into
sections. In end-user mode, no editing is possible, and the user browses
principally by clicking on highlighted blocks.

When an application invokes wxHelp, subsequent sections, blocks or
files may be viewed using the same instance of wxHelp since the two
programs are linked using wxWindows interprocess communication
facilities. When the application exits, that application's instance of
wxHelp may be made to exit also.  See the {\bf wxHelpControllerBase} entry in the
reference section for how an application controls wxHelp.

\section{Tex2RTF}\label{textortf}

Supplied with wxWindows is a utility called Tex2RTF for converting\rtfsp
\LaTeX\ manuals to the following formats:

\begin{description}
\item[wxHelp]
wxWindows help system format (XLP).
\item[Linear RTF]
Rich Text Format suitable for importing into a word processor.
\item[Windows Help RTF]
Rich Text Format suitable for compiling into a WinHelp HLP file with the
help compiler.
\item[HTML]
HTML is the native format for Mosaic, the main hypertext viewer for
the World Wide Web. Since it is freely available it is a good candidate
for being the wxWindows help system under X, as an alternative to wxHelp.
\end{description}

Tex2RTF is used for the wxWindows manuals and can be used independently
by authors wishing to create on-line and printed manuals from the same\rtfsp
\LaTeX\ source.  Please see the separate documentation for Tex2RTF.

\section{wxTreeLayout}

This is a simple class library for drawing trees in a reasonably pretty
fashion. It provides only minimal default drawing capabilities, since
the algorithm is meant to be used for implementing custom tree-based
tools.

Directed graphs may also be drawn using this library, if cycles are
removed before the nodes and arcs are passed to the algorithm.

Tree displays are used in many applications: directory browsers,
hypertext systems, class browsers, and decision trees are a few
possibilities.

See the separate manual and the directory utils/wxtree.

\section{wxGraphLayout}

The wxGraphLayout class is based on a tool called `graphplace' by Dr.
Jos T.J. van Eijndhoven of Eindhoven University of Technology. Given a
(possibly cyclic) directed graph, it does its best to lay out the nodes
in a sensible manner. There are many applications (such as diagramming)
where it is required to display a graph with no human intervention. Even
if manual repositioning is later required, this algorithm can make a good
first attempt.

See the separate manual and the directory utils/wxgraph. 

\section{Colours}\label{coloursampler}

A colour sampler for viewing colours and their names on each
platform.

%
\chapter{Tutorial}\label{tutorial}
\setheader{{\it CHAPTER \thechapter}}{}{}{}{}{{\it CHAPTER \thechapter}}%
\setfooter{\thepage}{}{}{}{}{\thepage}%

To be written.
\end{comment}

\chapter{Programming strategies}\label{strategies}
\setheader{{\it CHAPTER \thechapter}}{}{}{}{}{{\it CHAPTER \thechapter}}%
\setfooter{\thepage}{}{}{}{}{\thepage}%

This chapter is intended to list strategies that may be useful when
writing and debugging wxWindows programs. If you have any good tips,
please submit them for inclusion here.

\section{Strategies for reducing programming errors}

\subsection{Use ASSERT}

Although I haven't done this myself within wxWindows, it is good
practice to use ASSERT statements liberally, that check for conditions that
should or should not hold, and print out appropriate error messages.
These can be compiled out of a non-debugging version of wxWindows
and your application. Using ASSERT is an example of `defensive programming':
it can alert you to problems later on.

\subsection{Use wxString in preference to character arrays}

Using wxString can be much safer and more convenient than using char *.
Again, I haven't practised what I'm preaching, but I'm now trying to use
wxString wherever possible. You can reduce the possibility of memory
leaks substantially, and it's much more convenient to use the overloaded
operators than functions such as strcmp. wxString won't add a significant
overhead to your program; the overhead is compensated for by easier
manipulation (which means less code).

The same goes for other data types: use classes wherever possible.

\section{Strategies for portability}

\subsection{Use relative positioning or constraints}

Don't use absolute panel item positioning if you can avoid it. Different GUIs have
very differently sized panel items. Consider using the constraint system, although this
can be complex to program.

Alternatively, you could use alternative .wrc (wxWindows resource files) on different
platforms, with slightly different dimensions in each. Or space your panel items out
to avoid problems.

\subsection{Use wxWindows resource files}

Use .wrc (wxWindows resource files) where possible, because they can be easily changed
independently of source code. Bitmap resources can be set up to load different
kinds of bitmap depending on platform (see the section on resource files).

\section{Strategies for debugging}\label{debugstrategies}

\subsection{Positive thinking}

It's common to blow up the problem in one's imagination, so that it seems to threaten
weeks, months or even years of work. The problem you face may seem insurmountable:
but almost never is. Once you have been programming for some time, you will be able
to remember similar incidents that threw you into the depths of despair. But
remember, you always solved the problem, somehow!

Perseverance is often the key, even though a seemingly trivial problem
can take an apparently inordinate amount of time to solve. In the end,
you will probably wonder why you worried so much. That's not to say it
isn't painful at the time. Try not to worry -- there are many more important
things in life.

\subsection{Simplify the problem}

Reduce the code exhibiting the problem to the smallest program possible
that exhibits the problem. If it is not possible to reduce a large and
complex program to a very small program, then try to ensure your code
doesn't hide the problem (you may have attempted to minimize the problem
in some way: but now you want to expose it).

With luck, you can add a small amount of code that causes the program
to go from functioning to non-functioning state. This should give a clue
to the problem. In some cases though, such as memory leaks or wrong
deallocation, this can still give totally spurious results!

\subsection{Use a debugger}

This sounds like facetious advice, but it's surprising how often people
don't use a debugger. Often it's an overhead to install or learn how to
use a debugger, but it really is essential for anything but the most
trivial programs.

\subsection{Use logging functions}

There is a variety of logging functions that you can use in your program:
see \helpref{Logging functions}{logfunctions}.

Using tracing statements may be more convenient than using the debugger
in some circumstances (such as when your debugger doesn't support a lot
of debugging code, or you wish to print a bunch of variables).

\subsection{Use the wxWindows debugging facilities}

You can use wxDebugContext to check for
memory leaks and corrupt memory: in fact in debugging mode, wxWindows will
automatically check for memory leaks at the end of the program if wxWindows is suitably
configured. Depending on the operating system and compiler, more or less
specific information about the problem will be logged.

You should also use \helpref{debug macros}{debugmacros} as part of a `defensive programming' strategy,
scattering wxASSERTs liberally to test for problems in your code as early as possible. Forward thinking
will save a surprising amount of time in the long run.

See the \helpref{debugging overview}{debuggingoverview} for further information.

\subsection{Check Windows debug messages}

Under Windows, it's worth running your program with DBWIN running or
some other program that shows Windows-generated debug messages. It's
possible it'll show invalid handles being used. You may have fun seeing
what commercial programs cause these normally hidden errors! Microsoft
recommend using the debugging version of Windows, which shows up even
more problems. However, I doubt it's worth the hassle for most
applications. wxWindows is designed to minimize the possibility of such
errors, but they can still happen occasionally, slipping through unnoticed
because they are not severe enough to cause a crash.

\subsection{Genetic mutation}

If we had sophisticated genetic algorithm tools that could be applied
to programming, we could use them. Until then, a common -- if rather irrational --
technique is to just make arbitrary changes to the code until something
different happens. You may have an intuition why a change will make a difference;
otherwise, just try altering the order of code, comment lines out, anything
to get over an impasse. Obviously, this is usually a last resort.


\helpinput{classes.tex}
\helpinput{category.tex}
\helpinput{topics.tex}
\helpinput{wxhtml.tex}
\helpinput{wxPython.tex}

\begin{comment}
\newpage

% Puts books in the bibliography without needing to cite them in the
% text
\nocite{helpbook}%
\nocite{wong93}%
\nocite{pree94}%
\nocite{gamma95}%
\nocite{smart95a}%
\nocite{smart95b}%

\bibliography{refs}
\addcontentsline{toc}{chapter}{Bibliography}
\setheader{{\it REFERENCES}}{}{}{}{}{{\it REFERENCES}}%
\setfooter{\thepage}{}{}{}{}{\thepage}%
\end{comment}

\newpage
% Note: In RTF, the \printindex must come before the
% change of header/footer, since the \printindex inserts
% the RTF \sect command which divides one chapter from
% the next.
\rtfonly{\printindex
\addcontentsline{toc}{chapter}{Index}
\setheader{{\it INDEX}}{}{}{}{}{{\it INDEX}}%
\setfooter{\thepage}{}{}{}{}{\thepage}
}
% In Latex, it must be this way around (I think)
\latexonly{\addcontentsline{toc}{chapter}{Index}
\setheader{{\it INDEX}}{}{}{}{}{{\it INDEX}}%
\setfooter{\thepage}{}{}{}{}{\thepage}
\printindex
}

\end{document}
