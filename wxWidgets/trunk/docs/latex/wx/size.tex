\section{\class{wxSize}}\label{wxsize}

A {\bf wxSize} is a useful data structure for graphics operations.
It simply contains integer {\it x} and {\it y} members.

\pythonnote{wxPython defines aliases for the \tt{x} and \tt{y} members
named \tt{width} and \tt{height} since it makes much more sense for
sizes.  There is also coresponding aliases for the  \tt{GetWidth} and
\tt{GetHeight} methods.
}

\wxheading{Derived from}

None

\wxheading{Include files}

<wx/gdicmn.h>

\wxheading{See also}

\helpref{wxPoint}{wxpoint}, \helpref{wxRealPoint}{wxrealpoint}

\latexignore{\rtfignore{\wxheading{Members}}}

\membersection{wxSize::wxSize}

\func{}{wxSize}{\void}

\func{}{wxSize}{\param{int}{ x}, \param{int}{ y}}

Creates a size object.

\membersection{wxSize::x}

\member{int}{x}

x member.

\membersection{wxSize::y}

\member{int}{ y}

y member.

\membersection{wxSize::GetX}\label{wxsizegetx}

\constfunc{int}{GetX}{\void}

Gets the x member.

\membersection{wxSize::GetY}\label{wxsizegety}

\constfunc{int}{GetY}{\void}

Gets the y member.

\membersection{wxSize::Set}\label{wxsizeset}

\func{void}{Set}{\param{int}{ x}, \param{int}{ y}}

Sets the x and y members.

\membersection{wxSize::operator $=$}

\func{void}{operator $=$}{\param{const wxSize\& }{sz}}

Assignment operator.


