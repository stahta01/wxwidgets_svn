\chapter{Classes by category}\label{classesbycat}
\setheader{{\it CHAPTER \thechapter}}{}{}{}{}{{\it CHAPTER \thechapter}}%
\setfooter{\thepage}{}{}{}{}{\thepage}%

A classification of wxWindows classes by category.

{\large {\bf Managed windows}}

There are several types of window that are directly controlled by the
window manager (such as MS Windows, or the Motif Window Manager).
Frames may contain windows, and dialog boxes may directly contain controls.

\twocolwidtha{6cm}
\begin{twocollist}\itemsep=0pt
\twocolitem{\helpref{wxDialog}{wxdialog}}{Dialog box}
\twocolitem{\helpref{wxFrame}{wxframe}}{Normal frame}
\twocolitem{\helpref{wxMDIChildFrame}{wxmdichildframe}}{MDI child frame}
\twocolitem{\helpref{wxMDIParentFrame}{wxmdiparentframe}}{MDI parent frame}
\twocolitem{\helpref{wxMiniFrame}{wxminiframe}}{A frame with a small title bar}
\twocolitem{\helpref{wxSplashScreen}{wxsplashscreen}}{Splash screen class}
%\twocolitem{\helpref{wxTabbedDialog}{wxtabbeddialog}}{Tabbed dialog
%(deprecated, use wxNotebook instead)}
\twocolitem{\helpref{wxTipWindow}{wxtipwindow}}{Shows text in a small window}
\twocolitem{\helpref{wxWizard}{wxwizard}}{A wizard dialog}
\end{twocollist}

See also {\bf Common dialogs}.

{\large {\bf Miscellaneous windows}}

The following are a variety of classes that are derived from wxWindow.

\twocolwidtha{6cm}
\begin{twocollist}\itemsep=0pt
\twocolitem{\helpref{wxPanel}{wxpanel}}{A window whose colour changes according to current user settings}
\twocolitem{\helpref{wxScrolledWindow}{wxscrolledwindow}}{Window with automatically managed scrollbars}
\twocolitem{\helpref{wxGrid}{wxgrid}}{A grid (table) window}
\twocolitem{\helpref{wxSplitterWindow}{wxsplitterwindow}}{Window which can be split vertically or horizontally}
\twocolitem{\helpref{wxStatusBar}{wxstatusbar}}{Implements the status bar on a frame}
\twocolitem{\helpref{wxToolBar}{wxtoolbar}}{Toolbar class}
%\twocolitem{\helpref{wxTabbedPanel}{wxtabbedpanel}}{Tabbed panel (to be replaced with wxNotebook)}
\twocolitem{\helpref{wxNotebook}{wxnotebook}}{Notebook class}
\twocolitem{\helpref{wxPlotWindow}{wxplotwindow}}{A class to display data.}
\twocolitem{\helpref{wxSashWindow}{wxsashwindow}}{Window with four optional sashes that can be dragged}
\twocolitem{\helpref{wxSashLayoutWindow}{wxsashlayoutwindow}}{Window that can be involved in an IDE-like layout arrangement}
\twocolitem{\helpref{wxWizardPage}{wxwizardpage}}{A base class for the page in wizard dialog.}
\twocolitem{\helpref{wxWizardPageSimple}{wxwizardpagesimple}}{A page in wizard dialog.}
\end{twocollist}

{\large {\bf Common dialogs}}

\overview{Overview}{commondialogsoverview}

Common dialogs are ready-made dialog classes which are frequently used
in an application.

\twocolwidtha{6cm}
\begin{twocollist}\itemsep=0pt
\twocolitem{\helpref{wxDialog}{wxdialog}}{Base class for common dialogs}
\twocolitem{\helpref{wxColourDialog}{wxcolourdialog}}{Colour chooser dialog}
\twocolitem{\helpref{wxDirDialog}{wxdirdialog}}{Directory selector dialog}
\twocolitem{\helpref{wxFileDialog}{wxfiledialog}}{File selector dialog}
\twocolitem{\helpref{wxFindReplaceDialog}{wxfindreplacedialog}}{Text search/replace dialog}
\twocolitem{\helpref{wxMultipleChoiceDialog}{wxmultiplechoicedialog}}{Dialog to get one or more selections from a list}
\twocolitem{\helpref{wxSingleChoiceDialog}{wxsinglechoicedialog}}{Dialog to get a single selection from a list and return the string}
\twocolitem{\helpref{wxTextEntryDialog}{wxtextentrydialog}}{Dialog to get a single line of text from the user}
\twocolitem{\helpref{wxFontDialog}{wxfontdialog}}{Font chooser dialog}
\twocolitem{\helpref{wxPageSetupDialog}{wxpagesetupdialog}}{Standard page setup dialog}
\twocolitem{\helpref{wxPrintDialog}{wxprintdialog}}{Standard print dialog}
\twocolitem{\helpref{wxPageSetupDialog}{wxpagesetupdialog}}{Standard page setup dialog}
\twocolitem{\helpref{wxMessageDialog}{wxmessagedialog}}{Simple message box dialog}
\twocolitem{\helpref{wxWizard}{wxwizard}}{A wizard dialog.}
\end{twocollist}

{\large {\bf Controls}}

Typically, these are small windows which provide interaction with the user. Controls
that are not static can have \helpref{validators}{wxvalidator} associated with them.

\twocolwidtha{6cm}
\begin{twocollist}\itemsep=0pt
\twocolitem{\helpref{wxControl}{wxcontrol}}{The base class for controls}
\twocolitem{\helpref{wxButton}{wxbutton}}{Push button control, displaying text}
\twocolitem{\helpref{wxBitmapButton}{wxbitmapbutton}}{Push button control, displaying a bitmap}
\twocolitem{\helpref{wxToggleButton}{wxtogglebutton}}{A button which stays pressed when clicked by user.}
\twocolitem{\helpref{wxCalendarCtrl}{wxcalendarctrl}}{Date picker control}
\twocolitem{\helpref{wxCheckBox}{wxcheckbox}}{Checkbox control}
\twocolitem{\helpref{wxCheckListBox}{wxchecklistbox}}{A listbox with a checkbox to the left of each item}
\twocolitem{\helpref{wxChoice}{wxchoice}}{Choice control (a combobox without the editable area)}
\twocolitem{\helpref{wxComboBox}{wxcombobox}}{A choice with an editable area}
\twocolitem{\helpref{wxGauge}{wxgauge}}{A control to represent a varying quantity, such as time remaining}
\twocolitem{\helpref{wxGenericDirCtrl}{wxgenericdirctrl}}{A control for displaying a directory tree}
\twocolitem{\helpref{wxStaticBox}{wxstaticbox}}{A static, or group box for visually grouping related controls}
\twocolitem{\helpref{wxListBox}{wxlistbox}}{A list of strings for single or multiple selection}
\twocolitem{\helpref{wxListCtrl}{wxlistctrl}}{A control for displaying lists of strings and/or icons, plus a multicolumn report view}
\twocolitem{\helpref{wxTabCtrl}{wxtabctrl}}{Manages several tabs}
\twocolitem{\helpref{wxTextCtrl}{wxtextctrl}}{Single or multiline text editing control}
\twocolitem{\helpref{wxTreeCtrl}{wxtreectrl}}{Tree (hierarchy) control}
\twocolitem{\helpref{wxScrollBar}{wxscrollbar}}{Scrollbar control}
\twocolitem{\helpref{wxSpinButton}{wxspinbutton}}{A spin or `up-down' control}
\twocolitem{\helpref{wxSpinCtrl}{wxspinctrl}}{A spin control - i.e. spin button and text control}
\twocolitem{\helpref{wxStaticText}{wxstatictext}}{One or more lines of non-editable text}
\twocolitem{\helpref{wxStaticBitmap}{wxstaticbitmap}}{A control to display a bitmap}
\twocolitem{\helpref{wxRadioBox}{wxradiobox}}{A group of radio buttons}
\twocolitem{\helpref{wxRadioButton}{wxradiobutton}}{A round button to be used with others in a mutually exclusive way}
\twocolitem{\helpref{wxSlider}{wxslider}}{A slider that can be dragged by the user}
\end{twocollist}

{\large {\bf Menus}}

\twocolwidtha{6cm}
\begin{twocollist}\itemsep=0pt
\twocolitem{\helpref{wxMenu}{wxmenu}}{Displays a series of menu items for selection}
\twocolitem{\helpref{wxMenuBar}{wxmenubar}}{Contains a series of menus for use with a frame}
\twocolitem{\helpref{wxMenuItem}{wxmenuitem}}{Represents a single menu item}
\end{twocollist}

{\large {\bf Window layout}}

There are two different systems for laying out windows (and dialogs in particular).
One is based upon so-called sizers and it requires less typing, thinking and calculating
and will in almost all cases produce dialogs looking equally well on all platforms, the
other is based on so-called constraints and is deprecated, though still available.

\overview{Sizer overview}{sizeroverview} describes sizer-based layout.

These are the classes relevant to sizer-based layout.

\twocolwidtha{6cm}
\begin{twocollist}\itemsep=0pt
\twocolitem{\helpref{wxSizer}{wxsizer}}{Abstract base class}
\twocolitem{\helpref{wxGridSizer}{wxgridsizer}}{A sizer for laying out windows in a grid with all fields having the same size}
\twocolitem{\helpref{wxFlexGridSizer}{wxflexgridsizer}}{A sizer for laying out windows in a flexible grid}
\twocolitem{\helpref{wxBoxSizer}{wxboxsizer}}{A sizer for laying out windows in a row or column}
\twocolitem{\helpref{wxStaticBoxSizer}{wxstaticboxsizer}}{Same as wxBoxSizer, but with a surrounding static box}
\twocolitem{\helpref{wxNotebookSizer}{wxnotebooksizer}}{Sizer to use with the wxNotebook control}
\end{twocollist}

\overview{Constraints overview}{constraintsoverview} describes constraints-based layout.

These are the classes relevant to constraints-based window layout.

\twocolwidtha{6cm}
\begin{twocollist}\itemsep=0pt
\twocolitem{\helpref{wxIndividualLayoutConstraint}{wxindividuallayoutconstraint}}{Represents a single constraint dimension}
\twocolitem{\helpref{wxLayoutConstraints}{wxlayoutconstraints}}{Represents the constraints for a window class}
\end{twocollist}

{\large {\bf Device contexts}}

\overview{Overview}{dcoverview}

Device contexts are surfaces that may be drawn on, and provide an
abstraction that allows parameterisation of your drawing code
by passing different device contexts.

\twocolwidtha{6cm}
\begin{twocollist}\itemsep=0pt
\twocolitem{\helpref{wxClientDC}{wxclientdc}}{A device context to access the client area outside {\bf OnPaint} events}
\twocolitem{\helpref{wxPaintDC}{wxpaintdc}}{A device context to access the client area inside {\bf OnPaint} events}
\twocolitem{\helpref{wxWindowDC}{wxwindowdc}}{A device context to access the non-client area}
\twocolitem{\helpref{wxScreenDC}{wxscreendc}}{A device context to access the entire screen}
\twocolitem{\helpref{wxDC}{wxdc}}{The device context base class}
\twocolitem{\helpref{wxMemoryDC}{wxmemorydc}}{A device context for drawing into bitmaps}
\twocolitem{\helpref{wxMetafileDC}{wxmetafiledc}}{A device context for drawing into metafiles}
\twocolitem{\helpref{wxPostScriptDC}{wxpostscriptdc}}{A device context for drawing into PostScript files}
\twocolitem{\helpref{wxPrinterDC}{wxprinterdc}}{A device context for drawing to printers}
\end{twocollist}

{\large {\bf Graphics device interface}}

\overview{Bitmaps overview}{wxbitmapoverview}

These classes are related to drawing on device contexts and windows.

\twocolwidtha{6cm}
\begin{twocollist}\itemsep=0pt
\twocolitem{\helpref{wxColour}{wxcolour}}{Represents the red, blue and green elements of a colour}
\twocolitem{\helpref{wxDCClipper}{wxdcclipper}}{Wraps the operations of setting and destroying the clipping region}
\twocolitem{\helpref{wxBitmap}{wxbitmap}}{Represents a bitmap}
\twocolitem{\helpref{wxBrush}{wxbrush}}{Used for filling areas on a device context}
\twocolitem{\helpref{wxBrushList}{wxbrushlist}}{The list of previously-created brushes}
\twocolitem{\helpref{wxCursor}{wxcursor}}{A small, transparent bitmap representing the cursor}
\twocolitem{\helpref{wxFont}{wxfont}}{Represents fonts}
\twocolitem{\helpref{wxFontList}{wxfontlist}}{The list of previously-created fonts}
\twocolitem{\helpref{wxIcon}{wxicon}}{A small, transparent bitmap for assigning to frames and drawing on device contexts}
\twocolitem{\helpref{wxImage}{wximage}}{A platform-independent image class}
\twocolitem{\helpref{wxImageList}{wximagelist}}{A list of images, used with some controls}
\twocolitem{\helpref{wxMask}{wxmask}}{Represents a mask to be used with a bitmap for transparent drawing}
\twocolitem{\helpref{wxPen}{wxpen}}{Used for drawing lines on a device context}
\twocolitem{\helpref{wxPenList}{wxpenlist}}{The list of previously-created pens}
\twocolitem{\helpref{wxPalette}{wxpalette}}{Represents a table of indices into RGB values}
\twocolitem{\helpref{wxRegion}{wxregion}}{Represents a simple or complex region on a window or device context}
\end{twocollist}

{\large {\bf Events}}

\overview{Overview}{eventhandlingoverview}

An event object contains information about a specific event. Event handlers
(usually member functions) have a single, event argument.

\twocolwidtha{6cm}
\begin{twocollist}\itemsep=0pt
\twocolitem{\helpref{wxActivateEvent}{wxactivateevent}}{A window or application activation event}
\twocolitem{\helpref{wxCalendarEvent}{wxcalendarevent}}{Used with \helpref{wxCalendarCtrl}{wxcalendarctrl}}
\twocolitem{\helpref{wxCalculateLayoutEvent}{wxcalculatelayoutevent}}{Used to calculate window layout}
\twocolitem{\helpref{wxCloseEvent}{wxcloseevent}}{A close window or end session event}
\twocolitem{\helpref{wxCommandEvent}{wxcommandevent}}{An event from a variety of standard controls}
\twocolitem{\helpref{wxDialUpEvent}{wxdialupevent}}{Event send by \helpref{wxDialUpManager}{wxdialupmanager}}
\twocolitem{\helpref{wxDropFilesEvent}{wxdropfilesevent}}{A drop files event}
\twocolitem{\helpref{wxEraseEvent}{wxeraseevent}}{An erase background event}
\twocolitem{\helpref{wxEvent}{wxevent}}{The event base class}
\twocolitem{\helpref{wxFindDialogEvent}{wxfinddialogevent}}{Event sent by 
\helpref{wxFindReplaceDialog}{wxfindreplacedialog}}
\twocolitem{\helpref{wxFocusEvent}{wxfocusevent}}{A window focus event}
\twocolitem{\helpref{wxKeyEvent}{wxkeyevent}}{A keypress event}
\twocolitem{\helpref{wxIconizeEvent}{wxiconizeevent}}{An iconize/restore event}
\twocolitem{\helpref{wxIdleEvent}{wxidleevent}}{An idle event}
\twocolitem{\helpref{wxInitDialogEvent}{wxinitdialogevent}}{A dialog initialisation event}
\twocolitem{\helpref{wxJoystickEvent}{wxjoystickevent}}{A joystick event}
\twocolitem{\helpref{wxListEvent}{wxlistevent}}{A list control event}
\twocolitem{\helpref{wxMaximizeEvent}{wxmaximizeevent}}{A maximize event}
\twocolitem{\helpref{wxMenuEvent}{wxmenuevent}}{A menu event}
\twocolitem{\helpref{wxMouseEvent}{wxmouseevent}}{A mouse event}
\twocolitem{\helpref{wxMoveEvent}{wxmoveevent}}{A move event}
\twocolitem{\helpref{wxNotebookEvent}{wxnotebookevent}}{A notebook control event}
\twocolitem{\helpref{wxNotifyEvent}{wxnotifyevent}}{A notification event, which can be vetoed}
\twocolitem{\helpref{wxPaintEvent}{wxpaintevent}}{A paint event}
\twocolitem{\helpref{wxProcessEvent}{wxprocessevent}}{A process ending event}
\twocolitem{\helpref{wxQueryLayoutInfoEvent}{wxquerylayoutinfoevent}}{Used to query layout information}
\twocolitem{\helpref{wxScrollEvent}{wxscrollevent}}{A scroll event from sliders, stand-alone scrollbars and spin buttons}
\twocolitem{\helpref{wxScrollWinEvent}{wxscrollwinevent}}{A scroll event from scrolled windows}
\twocolitem{\helpref{wxSizeEvent}{wxsizeevent}}{A size event}
\twocolitem{\helpref{wxSocketEvent}{wxsocketevent}}{A socket event}
\twocolitem{\helpref{wxSpinEvent}{wxspinevent}}{An event from \helpref{wxSpinButton}{wxspinbutton}}
\twocolitem{\helpref{wxSysColourChangedEvent}{wxsyscolourchangedevent}}{A system colour change event}
\twocolitem{\helpref{wxTabEvent}{wxtabevent}}{A tab control event}
\twocolitem{\helpref{wxTreeEvent}{wxtreeevent}}{A tree control event}
\twocolitem{\helpref{wxUpdateUIEvent}{wxupdateuievent}}{A user interface update event}
\twocolitem{\helpref{wxWizardEvent}{wxwizardevent}}{A wizard event}
\end{twocollist}

{\large {\bf Validators}}

\overview{Overview}{validatoroverview}

These are the window validators, used for filtering and validating
user input.

\twocolwidtha{6cm}
\begin{twocollist}\itemsep=0pt
\twocolitem{\helpref{wxValidator}{wxvalidator}}{Base validator class}
\twocolitem{\helpref{wxTextValidator}{wxtextvalidator}}{Text control validator class}
\twocolitem{\helpref{wxGenericValidator}{wxgenericvalidator}}{Generic control validator class}
\end{twocollist}

{\large {\bf Data structures}}

These are the data structure classes supported by wxWindows.

\twocolwidtha{6cm}
\begin{twocollist}\itemsep=0pt
\twocolitem{\helpref{wxCmdLineParser}{wxcmdlineparser}}{Command line parser class}
\twocolitem{\helpref{wxDate}{wxdate}}{A class for date manipulation (deprecated in favour of wxDateTime)}
\twocolitem{\helpref{wxDateSpan}{wxdatespan}}{A logical time interval.}
\twocolitem{\helpref{wxDateTime}{wxdatetime}}{A class for date/time manipulations}
\twocolitem{\helpref{wxExpr}{wxexpr}}{A class for flexible I/O}
\twocolitem{\helpref{wxExprDatabase}{wxexprdatabase}}{A class for flexible I/O}
\twocolitem{\helpref{wxHashMap}{wxhashmap}}{A simple hash map implementation}
\twocolitem{\helpref{wxHashTable}{wxhashtable}}{A simple hash table implementation (deprecated, use wxHashMap)}
% \twocolitem{\helpref{wxHashTableLong}{wxhashtablelong}}{A wxHashTable version for storing long data}
\twocolitem{\helpref{wxList}{wxlist}}{A simple linked list implementation}
\twocolitem{\helpref{wxLongLong}{wxlonglong}}{A portable 64 bit integer type}
\twocolitem{\helpref{wxNode}{wxnode}}{Represents a node in the wxList implementation}
\twocolitem{\helpref{wxObject}{wxobject}}{The root class for most wxWindows classes}
\twocolitem{\helpref{wxPathList}{wxpathlist}}{A class to help search multiple paths}
\twocolitem{\helpref{wxPoint}{wxpoint}}{Representation of a point}
\twocolitem{\helpref{wxRect}{wxrect}}{A class representing a rectangle}
\twocolitem{\helpref{wxRegEx}{wxregex}}{Regular expression support}
\twocolitem{\helpref{wxRegion}{wxregion}}{A class representing a region}
\twocolitem{\helpref{wxString}{wxstring}}{A string class}
\twocolitem{\helpref{wxStringList}{wxstringlist}}{A class representing a list of strings}
\twocolitem{\helpref{wxStringTokenizer}{wxstringtokenizer}}{A class for interpreting a string as a list of tokens or words}
\twocolitem{\helpref{wxRealPoint}{wxrealpoint}}{Representation of a point using floating point numbers}
\twocolitem{\helpref{wxSize}{wxsize}}{Representation of a size}
\twocolitem{\helpref{wxTime}{wxtime}}{A class for time manipulation (deprecated in favour of wxDateTime)}
\twocolitem{\helpref{wxTimeSpan}{wxtimespan}}{A time interval.}
\twocolitem{\helpref{wxVariant}{wxvariant}}{A class for storing arbitrary types that may change at run-time}
\end{twocollist}

{\large {\bf Run-time class information system}}

\overview{Overview}{runtimeclassoverview}

wxWindows supports run-time manipulation of class information, and dynamic
creation of objects given class names.

\twocolwidtha{6cm}
\begin{twocollist}\itemsep=0pt
\twocolitem{\helpref{wxClassInfo}{wxclassinfo}}{Holds run-time class information}
\twocolitem{\helpref{wxObject}{wxobject}}{Root class for classes with run-time information}
\twocolitem{\helpref{RTTI macros}{rttimacros}}{Macros for manipulating run-time information}
\end{twocollist}

{\large {\bf Debugging features}}

\overview{Overview}{wxlogoverview}

wxWindows provides several classes and functions for the message logging.
Please see the \helpref{wxLog overview}{wxlogoverview} for more details.

\twocolwidtha{6cm}
\begin{twocollist}\itemsep=0pt
\twocolitem{\helpref{wxLog}{wxlog}}{The base log class}
\twocolitem{\helpref{wxLogStderr}{wxlogstderr}}{Log messages to a C STDIO stream}
\twocolitem{\helpref{wxLogStream}{wxlogstream}}{Log messages to a C++ iostream}
\twocolitem{\helpref{wxLogTextCtrl}{wxlogtextctrl}}{Log messages to a \helpref{wxTextCtrl}{wxtextctrl}}
\twocolitem{\helpref{wxLogWindow}{wxlogwindow}}{Log messages to a log frame}
\twocolitem{\helpref{wxLogGui}{wxloggui}}{Default log target for GUI programs}
\twocolitem{\helpref{wxLogNull}{wxlognull}}{Temporarily suppress message logging}
\twocolitem{\helpref{wxLogChain}{wxlogchain}}{Allows to chain two log targets}
\twocolitem{\helpref{wxLogPassThrough}{wxlogpassthrough}}{Allows to filter the log messages}
\twocolitem{\helpref{wxStreamToTextRedirector}{wxstreamtotextredirector}}{Allows
to redirect output sent to {\tt cout} to a \helpref{wxTextCtrl}{wxtextctrl}}
\twocolitem{\helpref{Log functions}{logfunctions}}{Error and warning logging functions}
\end{twocollist}

{\large {\bf Debugging features}}

\overview{Overview}{debuggingoverview}

wxWindows supports some aspects of debugging an application through
classes, functions and macros.

\twocolwidtha{6cm}
\begin{twocollist}\itemsep=0pt
\twocolitem{\helpref{wxDebugContext}{wxdebugcontext}}{Provides memory-checking facilities}
%\twocolitem{\helpref{wxDebugStreamBuf}{wxdebugstreambuf}}{A stream buffer writing to the debug stream}
\twocolitem{\helpref{Debugging macros}{debugmacros}}{Debug macros for assertion and checking}
\twocolitem{\helpref{WXDEBUG\_NEW}{debugnew}}{Use this macro to give further debugging information}
%\twocolitem{\helpref{WXTRACE}{trace}}{Trace macro}
%\twocolitem{\helpref{WXTRACELEVEL}{tracelevel}}{Trace macro with levels}
\end{twocollist}

{\large {\bf Networking classes}}

wxWindows provides its own classes for socket based networking.

\twocolwidtha{6cm}
\begin{twocollist}\itemsep=0pt
\twocolitem{\helpref{wxDialUpManager}{wxdialupmanager}}{Provides functions to check the status of network connection and to establish one}
\twocolitem{\helpref{wxIPV4address}{wxipv4address}}{Represents an Internet address}
\twocolitem{\helpref{wxSocketBase}{wxsocketbase}}{Represents a socket base object}
\twocolitem{\helpref{wxSocketClient}{wxsocketclient}}{Represents a socket client}
\twocolitem{\helpref{wxSocketServer}{wxsocketserver}}{Represents a socket server}
\twocolitem{\helpref{wxSocketEvent}{wxsocketevent}}{A socket event}
\twocolitem{\helpref{wxFTP}{wxftp}}{FTP protocol class}
\twocolitem{\helpref{wxHTTP}{wxhttp}}{HTTP protocol class}
\twocolitem{\helpref{wxURL}{wxurl}}{Represents a Universal Resource Locator}
\end{twocollist}


{\large {\bf Interprocess communication}}

\overview{Overview}{ipcoverview}

wxWindows provides a simple interprocess communications facilities
based on DDE.

\twocolwidtha{6cm}
\begin{twocollist}\itemsep=0pt
\twocolitem{\helpref{wxDDEClient}{wxddeclient}}{Represents a client}
\twocolitem{\helpref{wxDDEConnection}{wxddeconnection}}{Represents the connection between a client and a server}
\twocolitem{\helpref{wxDDEServer}{wxddeserver}}{Represents a server}
\twocolitem{\helpref{wxTCPClient}{wxtcpclient}}{Represents a client}
\twocolitem{\helpref{wxTCPConnection}{wxtcpconnection}}{Represents the connection between a client and a server}
\twocolitem{\helpref{wxTCPServer}{wxtcpserver}}{Represents a server}
%\twocolitem{\helpref{wxSocketHandler}{wxsockethandler}}{Represents a socket handler}
\end{twocollist}

{\large {\bf Document/view framework}}

\overview{Overview}{docviewoverview}

wxWindows supports a document/view framework which provides
housekeeping for a document-centric application.

\twocolwidtha{6cm}
\begin{twocollist}\itemsep=0pt
\twocolitem{\helpref{wxDocument}{wxdocument}}{Represents a document}
\twocolitem{\helpref{wxView}{wxview}}{Represents a view}
\twocolitem{\helpref{wxDocTemplate}{wxdoctemplate}}{Manages the relationship between a document class and a view class}
\twocolitem{\helpref{wxDocManager}{wxdocmanager}}{Manages the documents and views in an application}
\twocolitem{\helpref{wxDocChildFrame}{wxdocchildframe}}{A child frame for showing a document view}
\twocolitem{\helpref{wxDocParentFrame}{wxdocparentframe}}{A parent frame to contain views}
%\twocolitem{\helpref{wxMDIDocChildFrame}{wxmdidocchildframe}}{An MDI child frame for showing a document view}
%\twocolitem{\helpref{wxMDIDocParentFrame}{wxmdidocparentframe}}{An MDI parent frame to contain views}
\end{twocollist}

{\large {\bf Printing framework}}

\overview{Overview}{printingoverview}

A printing and previewing framework is implemented to
make it relatively straightforward to provide document printing
facilities.

\twocolwidtha{6cm}
\begin{twocollist}\itemsep=0pt
\twocolitem{\helpref{wxPreviewFrame}{wxpreviewframe}}{Frame for displaying a print preview}
\twocolitem{\helpref{wxPreviewCanvas}{wxpreviewcanvas}}{Canvas for displaying a print preview}
\twocolitem{\helpref{wxPreviewControlBar}{wxpreviewcontrolbar}}{Standard control bar for a print preview}
\twocolitem{\helpref{wxPrintDialog}{wxprintdialog}}{Standard print dialog}
\twocolitem{\helpref{wxPageSetupDialog}{wxpagesetupdialog}}{Standard page setup dialog}
\twocolitem{\helpref{wxPrinter}{wxprinter}}{Class representing the printer}
\twocolitem{\helpref{wxPrinterDC}{wxprinterdc}}{Printer device context}
\twocolitem{\helpref{wxPrintout}{wxprintout}}{Class representing a particular printout}
\twocolitem{\helpref{wxPrintPreview}{wxprintpreview}}{Class representing a print preview}
\twocolitem{\helpref{wxPrintData}{wxprintdata}}{Represents information about the document being printed}
\twocolitem{\helpref{wxPrintDialogData}{wxprintdialogdata}}{Represents information about the print dialog}
\twocolitem{\helpref{wxPageSetupDialogData}{wxpagesetupdialogdata}}{Represents information about the page setup dialog}
\end{twocollist}

{\large {\bf Drag and drop and clipboard classes}}

\overview{Drag and drop and clipboard overview}{wxdndoverview}

\twocolwidtha{6cm}
\begin{twocollist}\itemsep=0pt
\twocolitem{\helpref{wxDataObject}{wxdataobject}}{Data object class}
\twocolitem{\helpref{wxDataFormat}{wxdataformat}}{Represents a data format}
\twocolitem{\helpref{wxTextDataObject}{wxtextdataobject}}{Text data object class}
\twocolitem{\helpref{wxFileDataObject}{wxtextdataobject}}{File data object class}
\twocolitem{\helpref{wxBitmapDataObject}{wxbitmapdataobject}}{Bitmap data object class}
\twocolitem{\helpref{wxCustomDataObject}{wxcustomdataobject}}{Custom data object class}
\twocolitem{\helpref{wxClipboard}{wxclipboard}}{Clipboard class}
\twocolitem{\helpref{wxDropTarget}{wxdroptarget}}{Drop target class}
\twocolitem{\helpref{wxFileDropTarget}{wxfiledroptarget}}{File drop target class}
\twocolitem{\helpref{wxTextDropTarget}{wxtextdroptarget}}{Text drop target class}
\twocolitem{\helpref{wxDropSource}{wxdropsource}}{Drop source class}
\end{twocollist}

{\large {\bf File related classes}}

wxWindows has several small classes to work with disk files, see \helpref{file classes
overview}{wxfileoverview} for more details.

\twocolwidtha{6cm}
\begin{twocollist}\itemsep=0pt
\twocolitem{\helpref{wxFileName}{wxfilename}}{Operations with the file name and attributes}
\twocolitem{\helpref{wxDir}{wxdir}}{Class for enumerating files/subdirectories.}
\twocolitem{\helpref{wxDirTraverser}{wxdirtraverser}}{Class used together with wxDir for recursively enumerating the files/subdirectories}
\twocolitem{\helpref{wxFile}{wxfile}}{Low-level file input/output class.}
\twocolitem{\helpref{wxFFile}{wxffile}}{Another low-level file input/output class.}
\twocolitem{\helpref{wxTempFile}{wxtempfile}}{Class to safely replace an existing file}
\twocolitem{\helpref{wxTextFile}{wxtextfile}}{Class for working with text files as with arrays of lines}
\end{twocollist}

{\large {\bf Stream classes}}

wxWindows has its own set of stream classes, as an alternative to often buggy standard stream
libraries, and to provide enhanced functionality.

\twocolwidtha{6cm}
\begin{twocollist}\itemsep=0pt
\twocolitem{\helpref{wxStreamBase}{wxstreambase}}{Stream base class}
\twocolitem{\helpref{wxStreamBuffer}{wxstreambuffer}}{Stream buffer class}
\twocolitem{\helpref{wxInputStream}{wxinputstream}}{Input stream class}
\twocolitem{\helpref{wxOutputStream}{wxoutputstream}}{Output stream class}
\twocolitem{\helpref{wxCountingOutputStream}{wxcountingoutputstream}}{Stream class for querying what size a stream would have.}
\twocolitem{\helpref{wxFilterInputStream}{wxfilterinputstream}}{Filtered input stream class}
\twocolitem{\helpref{wxFilterOutputStream}{wxfilteroutputstream}}{Filtered output stream class}
\twocolitem{\helpref{wxBufferedInputStream}{wxbufferedinputstream}}{Buffered input stream class}
\twocolitem{\helpref{wxBufferedOutputStream}{wxbufferedoutputstream}}{Buffered output stream class}
\twocolitem{\helpref{wxMemoryInputStream}{wxmeminputstream}}{Memory input stream class}
\twocolitem{\helpref{wxMemoryOutputStream}{wxmemoutputstream}}{Memory output stream class}
\twocolitem{\helpref{wxDataInputStream}{wxdatainputstream}}{Platform-independent binary data input stream class}
\twocolitem{\helpref{wxDataOutputStream}{wxdataoutputstream}}{Platform-independent binary data output stream class}
\twocolitem{\helpref{wxTextInputStream}{wxtextinputstream}}{Platform-independent text data input stream class}
\twocolitem{\helpref{wxTextOutputStream}{wxtextoutputstream}}{Platform-independent text data output stream class}
\twocolitem{\helpref{wxFileInputStream}{wxfileinputstream}}{File input stream class}
\twocolitem{\helpref{wxFileOutputStream}{wxfileoutputstream}}{File output stream class}
\twocolitem{\helpref{wxFFileInputStream}{wxffileinputstream}}{Another file input stream class}
\twocolitem{\helpref{wxFFileOutputStream}{wxffileoutputstream}}{Another file output stream class}
\twocolitem{\helpref{wxZlibInputStream}{wxzlibinputstream}}{Zlib (compression) input stream class}
\twocolitem{\helpref{wxZlibOutputStream}{wxzliboutputstream}}{Zlib (compression) output stream class}
\twocolitem{\helpref{wxZipInputStream}{wxzipinputstream}}{Input stream for reading from ZIP archives}
\twocolitem{\helpref{wxSocketInputStream}{wxsocketinputstream}}{Socket input stream class}
\twocolitem{\helpref{wxSocketOutputStream}{wxsocketoutputstream}}{Socket output stream class}
\end{twocollist}

{\large {\bf Threading classes}}

\overview{Multithreading overview}{wxthreadoverview}

wxWindows provides a set of classes to make use of the native thread
capabilities of the various platforms.

\twocolwidtha{6cm}
\begin{twocollist}\itemsep=0pt
\twocolitem{\helpref{wxThread}{wxthread}}{Thread class}
\twocolitem{\helpref{wxMutex}{wxmutex}}{Mutex class}
\twocolitem{\helpref{wxMutexLocker}{wxmutexlocker}}{Mutex locker utility class}
\twocolitem{\helpref{wxCriticalSection}{wxcriticalsection}}{Critical section class}
\twocolitem{\helpref{wxCriticalSectionLocker}{wxcriticalsectionlocker}}{Critical section locker utility class}
\twocolitem{\helpref{wxCondition}{wxcondition}}{Condition class}
\twocolitem{\helpref{wxSemaphore}{wxsemaphore}}{Semaphore class}
\end{twocollist}

{\large {\bf HTML classes}}

wxWindows provides a set of classes to display text in HTML format. These
class include a help system based on the HTML widget.

\twocolwidtha{6cm}
\begin{twocollist}\itemsep=0pt
\twocolitem{\helpref{wxHtmlHelpController}{wxhtmlhelpcontroller}}{HTML help controller class}
\twocolitem{\helpref{wxHtmlWindow}{wxhtmlwindow}}{HTML window class}
\twocolitem{\helpref{wxHtmlEasyPrinting}{wxhtmleasyprinting}}{Simple class for printing HTML}
\twocolitem{\helpref{wxHtmlPrintout}{wxhtmlprintout}}{Generic HTML wxPrintout class}
\twocolitem{\helpref{wxHtmlParser}{wxhtmlparser}}{Generic HTML parser class}
\twocolitem{\helpref{wxHtmlTagHandler}{wxhtmltaghandler}}{HTML tag handler, pluginable into wxHtmlParser}
\twocolitem{\helpref{wxHtmlWinParser}{wxhtmlwinparser}}{HTML parser class for wxHtmlWindow}
\twocolitem{\helpref{wxHtmlWinTagHandler}{wxhtmlwintaghandler}}{HTML tag handler, pluginable into wxHtmlWinParser}
\end{twocollist}

{\large {\bf Virtual file system classes}}

wxWindows provides a set of classes that implement an extensible virtual file system,
used internally by the HTML classes.

\twocolwidtha{6cm}
\begin{twocollist}\itemsep=0pt
\twocolitem{\helpref{wxFSFile}{wxfsfile}}{Represents a file in the virtual file system}
\twocolitem{\helpref{wxFileSystem}{wxfilesystem}}{Main interface for the virtual file system}
\twocolitem{\helpref{wxFileSystemHandler}{wxfilesystemhandler}}{Class used to announce file system type}
\end{twocollist}

{\large {\bf XML-based resource system classes}}

\overview{XML-based resource system overview}{xrcoverview}

Resources allow your application to create controls and other user interface elements
from specifications stored in an XML format.

\twocolwidtha{6cm}
\begin{twocollist}\itemsep=0pt
\twocolitem{\helpref{wxXmlResource}{wxxmlresource}}{The main class for working with resources.}
\twocolitem{\helpref{wxXmlResourceHandler}{wxxmlresourcehandler}}{The base class for XML resource handlers.}
\end{twocollist}

{\large {\bf Online help}}

\twocolwidtha{6cm}
\begin{twocollist}\itemsep=0pt
\twocolitem{\helpref{wxHelpController}{wxhelpcontroller}}{Family of classes for controlling help windows}
\twocolitem{\helpref{wxHtmlHelpController}{wxhtmlhelpcontroller}}{HTML help controller class}
\twocolitem{\helpref{wxContextHelp}{wxcontexthelp}}{Class to put application into context-sensitive help mode}
\twocolitem{\helpref{wxContextHelpButton}{wxcontexthelpbutton}}{Button class for putting application into context-sensitive help mode}
\twocolitem{\helpref{wxHelpProvider}{wxhelpprovider}}{Abstract class for context-sensitive help provision}
\twocolitem{\helpref{wxSimpleHelpProvider}{wxsimplehelpprovider}}{Class for simple context-sensitive help provision}
\twocolitem{\helpref{wxHelpControllerHelpProvider}{wxhelpcontrollerhelpprovider}}{Class for context-sensitive help provision via a help controller}
\twocolitem{\helpref{wxToolTip}{wxtooltip}}{Class implementing tooltips}
\end{twocollist}

{\large {\bf Database classes}}

\overview{Database classes overview}{odbcoverview}

wxWindows provides two alternative sets of classes for accessing Microsoft's ODBC (Open Database Connectivity)
product. The new version by Remstar, known as wxODBC, is more powerful,
portable, flexible and better supported, so please use the classes below for
working with databases:

\twocolwidtha{6cm}
\begin{twocollist}\itemsep=0pt
\twocolitem{\helpref{wxDb}{wxdb}}{ODBC database connection}
\twocolitem{\helpref{wxDbTable}{wxdbtable}}{Provides access to a database table}
\twocolitem{\helpref{wxDbInf}{wxdbinf}}{}
\twocolitem{\helpref{wxDbTableInf}{wxdbtableinf}}{}
\twocolitem{\helpref{wxDbColDef}{wxdbcoldef}}{}
\twocolitem{\helpref{wxDbColInf}{wxdbcolinf}}{}
\twocolitem{\helpref{wxDbColDataPtr}{wxdbcoldataptr}}{}
\twocolitem{\helpref{wxDbColFor}{wxdbcolfor}}{}
\twocolitem{\helpref{wxDbConnectInf}{wxdbconnectinf}}{}
\twocolitem{\helpref{wxDbIdxDef}{wxdbidxdef}}{}
\end{twocollist}

The documentation for the older classes is still included, but you should avoid
using any of them in the new programs:

\twocolwidtha{6cm}
\begin{twocollist}\itemsep=0pt
\twocolitem{\helpref{wxDatabase}{wxdatabase}}{Database class}
\twocolitem{\helpref{wxQueryCol}{wxquerycol}}{Class representing a column}
\twocolitem{\helpref{wxQueryField}{wxqueryfield}}{Class representing a field}
\twocolitem{\helpref{wxRecordSet}{wxrecordset}}{Class representing one or more record}
\end{twocollist}

{\large {\bf Miscellaneous}}

\twocolwidtha{6cm}
\begin{twocollist}\itemsep=0pt
\twocolitem{\helpref{wxApp}{wxapp}}{Application class}
\twocolitem{\helpref{wxCaret}{wxcaret}}{A caret (cursor) object}
\twocolitem{\helpref{wxCmdLineParser}{wxcmdlineparser}}{Command line parser class}
\twocolitem{\helpref{wxConfig}{wxconfigbase}}{Classes for configuration reading/writing (using either INI files or registry)}
\twocolitem{\helpref{wxDllLoader}{wxdllloader}}{Class to work with shared libraries.}
\twocolitem{\helpref{wxLayoutAlgorithm}{wxlayoutalgorithm}}{An alternative window layout facility}
\twocolitem{\helpref{wxProcess}{wxprocess}}{Process class}
\twocolitem{\helpref{wxTimer}{wxtimer}}{Timer class}
\twocolitem{\helpref{wxStopWatch}{wxstopwatch}}{Stop watch class}
\twocolitem{\helpref{wxMimeTypesManager}{wxmimetypesmanager}}{MIME-types manager class}
\twocolitem{\helpref{wxSystemSettings}{wxsystemsettings}}{System settings class for obtaining various global parameters}
\twocolitem{\helpref{wxSystemOptions}{wxsystemoptions}}{System options class for run-time configuration}
\twocolitem{\helpref{wxAcceleratorTable}{wxacceleratortable}}{Accelerator table}
\twocolitem{\helpref{wxAutomationObject}{wxautomationobject}}{OLE automation class}
\twocolitem{\helpref{wxFontMapper}{wxfontmapper}}{Font mapping, finding suitable font for given encoding}
\twocolitem{\helpref{wxEncodingConverter}{wxencodingconverter}}{Encoding conversions}
\twocolitem{\helpref{wxCalendarDateAttr}{wxcalendardateattr}}{Used with \helpref{wxCalendarCtrl}{wxcalendarctrl}}
\twocolitem{\helpref{wxQuantize}{wxquantize}}{Class to perform quantization, or colour reduction}
\twocolitem{\helpref{wxSingleInstanceChecker}{wxsingleinstancechecker}}{Check that only single program instance is running}
\end{twocollist}

