%%%%%%%%%%%%%%%%%%%%%%%%%%%%%%%%%%%%%%%%%%%%%%%%%%%%%%%%%%%%%%%%%%%%%%%%%%%%%
%% Name:        hvscrolledwindow.tex
%% Purpose:     wxHVScrolledWindow Documentation
%% Author:      Bryan Petty
%% Modified by: 
%% Created:     2007-04-04
%% RCS-ID:      $Id$
%% Copyright:   (c) 2007 wxWidgets Team
%% License:     wxWindows Licence
%%%%%%%%%%%%%%%%%%%%%%%%%%%%%%%%%%%%%%%%%%%%%%%%%%%%%%%%%%%%%%%%%%%%%%%%%%%%%

\section{\class{wxHVScrolledWindow}}\label{wxhvscrolledwindow}

This window inherits all functionality of both vertical and horizontal,
variable scrolled windows. It automatically handles everything needed to
scroll both axis simultaneously with both variable row heights and variable
column widths.

This is a generalization of the \helpref{wxScrolledWindow}{wxscrolledwindow}
class which can be only used when all rows and columns are the same size. It
lacks some other wxScrolledWindow features however, notably it can't scroll
only a rectangle of the window and not its entire client area.

To use this class, you must derive from it and implement both the
\helpref{OnGetRowHeight()}{wxvarvscrollhelperongetrowheight} and
\helpref{OnGetColumnWidth()}{wxvarhscrollhelperongetcolumnwidth} pure virtual
methods to let the base class know how many rows and columns it should
display. You also need to set the total rows and columns the window contains,
but from that moment on the scrolling is handled entirely by
wxHVScrolledWindow. You only need to draw the visible part of contents in
your {\tt OnPaint()} method as usual. You should use
\helpref{GetVisibleBegin()}{wxvarhvscrollhelpergetvisiblebegin}
and \helpref{GetVisibleEnd()}{wxvarhvscrollhelpergetvisibleend} to select the
lines to display. Note that the device context origin is not shifted so the
first visible row and column always appear at the point $(0, 0)$ in physical
as well as logical coordinates.

\wxheading{Derived from}

\helpref{wxPanel}{wxpanel}\\
\helpref{wxWindow}{wxwindow}\\
\helpref{wxEvtHandler}{wxevthandler}\\
\helpref{wxObject}{wxobject}

\helpref{wxVarHVScrollHelper}{wxvarhvscrollhelper}

\helpref{wxVarVScrollHelper}{wxvarvscrollhelper}\\
\helpref{wxVarScrollHelperBase}{wxvarscrollhelperbase}

\helpref{wxVarHScrollHelper}{wxvarhscrollhelper}\\
\helpref{wxVarScrollHelperBase}{wxvarscrollhelperbase}

\wxheading{Include files}

<wx/vscroll.h>

\wxheading{See also}

\helpref{wxHScrolledWindow}{wxhscrolledwindow},
\rtfsp\helpref{wxVScrolledWindow}{wxvscrolledwindow}

\latexignore{\rtfignore{\wxheading{Members}}}


\membersection{wxHVScrolledWindow::wxHVScrolledWindow}\label{wxhvscrolledwindowwxhvscrolledwindow}

\func{}{wxHVScrolledWindow}{\void}

Default constructor, you must call \helpref{Create()}{wxhvscrolledwindowcreate}
later.

\func{}{wxHVScrolledWindow}{\param{wxWindow* }{parent}, \param{wxWindowID }{id = wxID\_ANY}, \param{const wxPoint\& }{pos = wxDefaultPosition}, \param{const wxSize\& }{size = wxDefaultSize}, \param{long }{style = 0}, \param{const wxString\& }{name = wxPanelNameStr}}

This is the normal constructor, no need to call {\tt Create()} after using this one.

Note that {\tt wxHSCROLL} and {\tt wxVSCROLL} are always automatically added
to our styles, there is no need to specify it explicitly.

\wxheading{Parameters}

\docparam{parent}{The parent window, must not be {\tt NULL}}

\docparam{id}{The identifier of this window, {\tt wxID\_ANY} by default}

\docparam{pos}{The initial window position}

\docparam{size}{The initial window size}

\docparam{style}{The window style. There are no special style bits defined for
this class.}

\docparam{name}{The name for this window; usually not used}


\membersection{wxHVScrolledWindow::Create}\label{wxhvscrolledwindowcreate}

\func{bool}{Create}{\param{wxWindow* }{parent}, \param{wxWindowID }{id = wxID\_ANY}, \param{const wxPoint\& }{pos = wxDefaultPosition}, \param{const wxSize\& }{size = wxDefaultSize}, \param{long }{style = 0}, \param{const wxString\& }{name = wxPanelNameStr}}

Same as the \helpref{non-default constuctor}{wxhvscrolledwindowwxhvscrolledwindow}
but returns status code: {\tt true} if ok, {\tt false} if the window couldn't
be created.

Just as with the constructor above, the {\tt wxHSCROLL} and {\tt wxVSCROLL}
styles are always used, there is no need to specify it explicitly.

