\chapter{wxPython Notes}\label{wxPython}
\pagenumbering{arabic}%
\setheader{{\it CHAPTER \thechapter}}{}{}{}{}{{\it CHAPTER \thechapter}}%
\setfooter{\thepage}{}{}{}{}{\thepage}%

This addendum is written by Robin Dunn, author of the wxPython wrapper

%----------------------------------------------------------------------
\section{What is wxPython?}\label{wxpwhat}

wxPython is a blending of the wxWindows GUI classes and the
\urlref{Python}{http://www.python.org/} programming language.

\wxheading{Python}

So what is Python?  Go to
\urlref{http://www.python.org}{http://www.python.org}
to learn more, but in a nutshell Python is an interpreted,
interactive, object-oriented programming language. It is often
compared to Tcl, Perl, Scheme or Java.

Python combines remarkable power with very clear syntax. It has
modules, classes, exceptions, very high level dynamic data types, and
dynamic typing. There are interfaces to many system calls and
libraries, and new built-in modules are easily written in C or
C++. Python is also usable as an extension language for applications
that need a programmable interface.

Python is copyrighted but freely usable and distributable, even for
commercial use.

\wxheading{wxPython}

wxPython is a Python package that can be imported at runtime that
includes a collection of Python modules and an extension module
(native code).  It provides a series of Python classes that mirror (or
shadow) many of the wxWindows GUI classes.  This extension module
attempts to mirror the class heiarchy of wxWindows as closely as
possble. This means that there is a wxFrame class in wxPython that
looks, smells, tastes and acts almost the same as the wxFrame class in
the C++ version.

wxPython is very versitile.  It can be used to create standalone GUI
applications, or in situations where Python is embedded in a C++
application as an internal scripting or macro language.

Currently wxPython is available for Win32 platforms and the GTK
toolkit (wxGTK) on most Unix/X-windows platforms.  The effort to
enable wxPython for wxMotif will begin shortly.  See \helpref{Building Python}{wxpbuild} for
details about getting wxPython working for you.


%----------------------------------------------------------------------
\section{Why use wxPython?}\label{wxpwhy}

So why would you want to use wxPython over just C++ and wxWindows?
Personally I prefer using Python for everything. I only use C++ when
I absolutely have to eek more performance out of an algorithm, and even
then I ususally code it as an extension module and leave the majority
of the program in Python.

Another good thing to use wxPython for is quick prototyping of your
wxWindows apps.  With C++ you have to continuously go though the
edit-compile-link-run cycle, which can be quite time comsuming.  With
Python it is only an edit-run cycle.  You can easily build an
application in a few hours with Python that would normally take a few
days or longer with C++.  Converting a wxPython app to a C++/wxWindows app
should be a straight forward task.

%----------------------------------------------------------------------
\section{Other Python GUIs}\label{wxpother}

There are other GUI solutions out there for Python.

\wxheading{Tkinter}

Tkinter is the defacto standard GUI for Python.  It is available
on nearly every platform that Python and Tcl/TK are.  Why Tcl/Tk?
Well because Tkinter is just a wrapper around Tcl's GUI toolkit, Tk.
This has its upsides and its downsides...

The upside is that Tk is a pretty veristile toolkit.  It can be made
to do a lot of things in a lot of different environments.  It is fairly
easy to create new widgets and use them interchangably in your
programs.

The downside is Tcl.  When using Tkinter you actually have two
separate language interpreters running, the Python interpreter and the
Tcl interpreter for the GUI.  Since the guts of Tcl is mostly about
string processing, it is fairly slow as well.  (Not too bad on a fast
Pentium II, but you really notice the difference on slower machines.)

It wasn't until the lastest version of Tcl/Tk that native Look and
Feel's were possible on non-Motif platforms.  This is because Tk
usually implements it's own widgets (controls) even when there are
native controls available.

Tkinter is a pretty low-level toolkit.  You have to do a lot of work
(verbose program code) to do things that would be much simpler with a higher
level of abstraction.

\wxheading{PythonWin}

PythonWin is an add-on package for Python for the Win32 platform.  It
includes wrappers for MFC as well as much of the win32 API.  Because
of its foundation, it is very familiar for programmers who have
experience with MFC and the Win32 API.  It is obviously not compatible
with other platforms and toolkits.  PythonWin is organized as separate
packages and modules so you can use the pieces you need without having
to use the GUI portions.

\wxheading{Others}

There are quite a few other GUI modules available for Python, some in
active use, some that havn't been updated for ages.  Most are simple
wrappers around some C or C++ toolkit or another, and most are not
cross-platform compatible.  See \urlref{this
link}{http://www.python.org/download/Contributed.html\#Graphics}
for a listing of a few of them.

%----------------------------------------------------------------------
\section{Building wxPython}\label{wxpbuild}

I used SWIG (\urlref{http://www.swig.org}{http://www.swig.org}) to
create the source code for the extension module.  This enabled me to
only have to deal with a small amount of code and only have to bother
with the exceptional issues.  SWIG takes care of the rest and
generates all the repetative code for me.  You don't need SWIG to
build the extension module as all the generated C++ code is included
under the src directory.  If you try to build wxPython and get errors
because SWIG is missing, then simply touch the .cpp and .py files so
make won't attempt to build them from the .i files.

I added a few minor features to SWIG to control some of the code
generation.  If you want to play around with this the patches are in
wxPython/SWIG.patches and they should be applied to the 1.1p5 version
of SWIG.  These new patches are documented at
\urlref{this site}{http://starship.skyport.net/crew/robind/python/\#swig},
and they should also end up in the 1.2 version of SWIG.

wxPython is organized as a Python package.  This means that the
directory containing the results of the build process should be a
subdirectory of a directory on the \tt{PYTHONPATH}, (and preferably
should be named wxPython.)  You can control where the build process
will dump wxPython by setting the \tt{TARGETDIR} makefile variable.
The default is \tt{\$(WXWIN)/utils/wxPython}.  If you leave it here
then you should add \tt{\$(WXWIN)/utils} to your \tt{PYTHONPATH}.
However, you may prefer to use something that is already on your
\tt{PYTHONPATH}, such as the \tt{site-packages} directory on Unix
systems.

\wxheading{Win32}

These instructions assume that you have Microsoft Visual C++ 5.0 or
6.0, that you have installed the command-line tools, and that the
appropriate environment variables are set for these tools.  You should
also have Python 1.5.1 installed, and wxWindows installed and built as
specified below.

\begin{enumerate}\itemsep=0pt
\item Build wxWindows with \tt{wxUSE_RESOURCE_LOADING_IN_MSW} set to 1 in
\tt{include/wx/msw/setup.h} so icons can be loaded dynamically.  While
there, make sure \tt{wxUSE_OWNER_DRAWN} is also set to 1.
\item Change into the \tt{\$(WXWIN)/utils/wxPython/src} directory.
\item Edit makefile.vc and specify where your python installation is at.
You may also want to fiddle with the \tt{TARGETDIR} variable as described
above.
\item Run \tt{nmake -f makefile.vc}
\item If it builds successfully, congratulations!  Move on to the next
step.  If not then you can try mailing the wxwin-developers list for
help.  Also, I will always have a pre-built win32 version of this extension module at
\urlref{http://alldunn.com/wxPython}{http://alldunn.com/wxPython}.
\item Change to the \tt{\$(WXWIN)/utils/wxPython/demo} directory.
\item Try executing the demo program.  Note that some of the demos print
diagnositc or test info to standard output, so they will require the
console version of python.  For example:

\tt{python demo.py}

To run them without requiring a console, you can use the \tt{pythonw.exe}
version of Python either from the command line or from a shortcut.
\end{enumerate}

\wxheading{Unix}

These directions assume that you have already successfully built
wxWindows for GTK, and installed Python 1.5.1 or later.  If you build Python
yourself, you will get everything installed that you need simply by
doing \bftt{make install}.  If you get Python from an RPM or other
pre-packaged source then there will probably be a separate package
with the development libraries, etc. that you will need to install.


\begin{enumerate}\itemsep=0pt
\item Change into the \tt{\$(WXWIN)/utils/wxPython/src} directory.
\item Edit \tt{Setup.in} and ensure that the flags, directories, and toolkit
options are correct, (hopefully this will be done by \tt{configure}
soon.)  See the above commentary about \tt{TARGETDIR}.  There are a
few sample Setup.in.[platform] files provided.
\item Run this command to generate a makefile:

\tt{make -f Makefile.pre.in boot}

\item Once you have the \tt{Makefile}, run \bftt{make} to build and then
\bftt{make install} to install the wxPython extension module.
\item Change to the \tt{\$(WXWIN)/utils/wxPython/demo} directory.
\item Try executing the demo program.  For example:

\tt{python demo.py}
\end{enumerate}


%----------------------------------------------------------------------
\section{Using wxPython}\label{wxpusing}

\wxheading{First things first...}

I'm not going to try and teach the Python language here.  You can do
that at the \urlref{Python Tutorial}{http://www.python.org/doc/tut/tut.html}.
I'm also going to assume that you know a bit about wxWindows already,
enough to notice the similarities in the classes used.

Take a look at the following wxPython program.  You can find a similar
program in the \tt{wxPython/demo} directory, named \tt{DialogUnits.py}.  If your
Python and wxPython are properly installed, you should be able to run
it by issuing this command:

\begin{indented}{1cm}
    \bftt{python DialogUnits.py}
\end{indented}

\hrule

\begin{verbatim}
001: ## import all of the wxPython GUI package
002: from wxPython.wx import *
003:
004: ## Create a new frame class, derived from the wxPython Frame.
005: class MyFrame(wxFrame):
006:
007:     def __init__(self, parent, id, title):
008:         # First, call the base class' __init__ method to create the frame
009:         wxFrame.__init__(self, parent, id, title,
010:                          wxPoint(100, 100), wxSize(160, 100))
011:
012:         # Associate some events with methods of this class
013:         EVT_SIZE(self, self.OnSize)
014:         EVT_MOVE(self, self.OnMove)
015:
016:         # Add a panel and some controls to display the size and position
017:         panel = wxPanel(self, -1)
018:         wxStaticText(panel, -1, "Size:",
019:                      wxDLG_PNT(panel, wxPoint(4, 4)),  wxDefaultSize)
020:         wxStaticText(panel, -1, "Pos:",
021:                      wxDLG_PNT(panel, wxPoint(4, 14)), wxDefaultSize)
022:         self.sizeCtrl = wxTextCtrl(panel, -1, "",
023:                                    wxDLG_PNT(panel, wxPoint(24, 4)),
024:                                    wxDLG_SZE(panel, wxSize(36, -1)),
025:                                    wxTE_READONLY)
026:         self.posCtrl = wxTextCtrl(panel, -1, "",
027:                                   wxDLG_PNT(panel, wxPoint(24, 14)),
028:                                   wxDLG_SZE(panel, wxSize(36, -1)),
029:                                   wxTE_READONLY)
030:
031:
032:     # This method is called automatically when the CLOSE event is
033:     # sent to this window
034:     def OnCloseWindow(self, event):
035:         # tell the window to kill itself
036:         self.Destroy()
037:
038:     # This method is called by the system when the window is resized,
039:     # because of the association above.
040:     def OnSize(self, event):
041:         size = event.GetSize()
042:         self.sizeCtrl.SetValue("%s, %s" % (size.width, size.height))
043:
044:         # tell the event system to continue looking for an event handler,
045:         # so the default handler will get called.
046:         event.Skip()
047:
048:     # This method is called by the system when the window is moved,
049:     # because of the association above.
050:     def OnMove(self, event):
051:         pos = event.GetPosition()
052:         self.posCtrl.SetValue("%s, %s" % (pos.x, pos.y))
053:
054:
055: # Every wxWindows application must have a class derived from wxApp
056: class MyApp(wxApp):
057:
058:     # wxWindows calls this method to initialize the application
059:     def OnInit(self):
060:
061:         # Create an instance of our customized Frame class
062:         frame = MyFrame(NULL, -1, "This is a test")
063:         frame.Show(true)
064:
065:         # Tell wxWindows that this is our main window
066:         self.SetTopWindow(frame)
067:
068:         # Return a success flag
069:         return true
070:
071:
072: app = MyApp(0)     # Create an instance of the application class
073: app.MainLoop()     # Tell it to start processing events
074:
\end{verbatim}
\hrule

\wxheading{Things to notice}

\begin{enumerate}\itemsep=0pt
\item At line 2 the wxPython classes, constants, and etc. are imported
into the current module's namespace.  If you prefer to reduce
namespace polution you can use "\tt{from wxPython import wx}" and
then access all the wxPython identifiers through the wx module, for
example, "\tt{wx.wxFrame}".
\item At line 13 the frame's sizing and moving events are connected to
methods of the class.  These helper functions are intended to be like
the event table macros that wxWindows employs.  But since static event
tables are impossible with wxPython, we use helpers that are named the
same to dynamically build the table.  The only real difference is
that the first arguemnt to the event helpers is always the window that
the event table entry should be added to.
\item Notice the use of \tt{wxDLG\_PNT} and \tt{wxDLG\_SZE} in lines 19
- 29 to convert from dialog units to pixels.  These helpers are unique
to wxPython since Python can't do method overloading like C++.
\item There is an \tt{OnCloseWindow} method at line 34 but no call to
EVT\_CLOSE to attach the event to the method.  Does it really get
called?  The answer is, yes it does.  This is because many of the
\em{standard} events are attached to windows that have the associated
\em{standard} method names.  I have tried to follow the lead of the
C++ classes in this area to determine what is \em{standard} but since
that changes from time to time I can make no guarentees, nor will it
be fully documented.  When in doubt, use an EVT\_*** function.
\item At lines 17 to 21 notice that there are no saved references to
the panel or the static text items that are created.  Those of you
who know Python might be wondering what happens when Python deletes
these objects when they go out of scope.  Do they disappear from the GUI?  They
don't.  Remember that in wxPython the Python objects are just shadows of the
coresponding C++ objects.  Once the C++ windows and controls are
attached to their parents, the parents manage them and delete them
when necessary.  For this reason, most wxPython objects do not need to
have a \_\_del\_\_ method that explicitly causes the C++ object to be
deleted.  If you ever have the need to forcibly delete a window, use
the Destroy() method as shown on line 36.
\item Just like wxWindows in C++, wxPython apps need to create a class
derived from \tt{wxApp} (line 56) that implements a method named
\tt{OnInit}, (line 59.) This method should create the application's
main window (line 62) and use \tt{wxApp.SetTopWindow()} (line 66) to
inform wxWindows about it.
\item And finally, at line 72 an instance of the application class is
created.  At this point wxPython finishes initializing itself, and calls
the \tt{OnInit} method to get things started.  (The zero parameter here is
a flag for functionality that isn't quite implemented yet.  Just
ignore it for now.)  The call to \tt{MainLoop} at line 73 starts the event
loop which continues until the application terminates or all the top
level windows are closed.
\end{enumerate}

%----------------------------------------------------------------------
\section{wxWindows classes implemented in wxPython}\label{wxpclasses}

The following classes are supported in wxPython.  Most provide nearly
full implementations of the public interfaces specified in the C++
documentation, others are less so.  They will all be brought as close
as possible to the C++ spec over time.

\begin{itemize}\itemsep=0pt
\item \helpref{wxAcceleratorEntry}{wxacceleratorentry}
\item \helpref{wxAcceleratorTable}{wxacceleratortable}
\item \helpref{wxActivateEvent}{wxactivateevent}
\item \helpref{wxBitmapButton}{wxbitmapbutton}
\item \helpref{wxBitmap}{wxbitmap}
\item wxBMPHandler
\item \helpref{wxBrush}{wxbrush}
\item \helpref{wxButton}{wxbutton}
\item \helpref{wxCalculateLayoutEvent}{wxcalculatelayoutevent}
\item \helpref{wxCheckBox}{wxcheckbox}
\item \helpref{wxCheckListBox}{wxchecklistbox}
\item \helpref{wxChoice}{wxchoice}
\item \helpref{wxClientDC}{wxclientdc}
\item \helpref{wxCloseEvent}{wxcloseevent}
\item \helpref{wxColourData}{wxcolourdata}
\item \helpref{wxColourDialog}{wxcolourdialog}
\item \helpref{wxColour}{wxcolour}
\item \helpref{wxComboBox}{wxcombobox}
\item \helpref{wxCommandEvent}{wxcommandevent}
\item \helpref{wxConfig}{wxconfigbase}
\item \helpref{wxControl}{wxcontrol}
\item \helpref{wxCursor}{wxcursor}
\item \helpref{wxDC}{wxdc}
\item \helpref{wxDialog}{wxdialog}
\item \helpref{wxDirDialog}{wxdirdialog}
\item \helpref{wxDropFilesEvent}{wxdropfilesevent}
\item \helpref{wxEraseEvent}{wxeraseevent}
\item \helpref{wxEvent}{wxevent}
\item \helpref{wxEvtHandler}{wxevthandler}
\item \helpref{wxFileDialog}{wxfiledialog}
\item \helpref{wxFocusEvent}{wxfocusevent}
\item \helpref{wxFontData}{wxfontdata}
\item \helpref{wxFontDialog}{wxfontdialog}
\item \helpref{wxFont}{wxfont}
\item \helpref{wxFrame}{wxframe}
\item \helpref{wxGauge}{wxgauge}
\item wxGIFHandler
\item wxGLCanvas
\item wxGridCell
\item wxGridEvent
\item \helpref{wxGrid}{wxgrid}
\item wxIconizeEvent
\item \helpref{wxIcon}{wxicon}
\item \helpref{wxIdleEvent}{wxidleevent}
\item \helpref{wxImage}{wximage}
\item \helpref{wxImageHandler}{wximagehandler}
\item \helpref{wxImageList}{wximagelist}
\item \helpref{wxIndividualLayoutConstraint}{wxindividuallayoutconstraint}
\item \helpref{wxInitDialogEvent}{wxinitdialogevent}
\item \helpref{wxJoystickEvent}{wxjoystickevent}
\item wxJPEGHandler
\item \helpref{wxKeyEvent}{wxkeyevent}
\item \helpref{wxLayoutAlgorithm}{wxlayoutalgorithm}
\item \helpref{wxLayoutConstraints}{wxlayoutconstraints}
\item \helpref{wxListBox}{wxlistbox}
\item \helpref{wxListCtrl}{wxlistctrl}
\item \helpref{wxListEvent}{wxlistevent}
\item \helpref{wxListItem}{wxlistctrlsetitem}
\item \helpref{wxMDIChildFrame}{wxmdichildframe}
\item \helpref{wxMDIClientWindow}{wxmdiclientwindow}
\item \helpref{wxMDIParentFrame}{wxmdiparentframe}
\item \helpref{wxMask}{wxmask}
\item wxMaximizeEvent
\item \helpref{wxMemoryDC}{wxmemorydc}
\item \helpref{wxMenuBar}{wxmenubar}
\item \helpref{wxMenuEvent}{wxmenuevent}
\item \helpref{wxMenuItem}{wxmenuitem}
\item \helpref{wxMenu}{wxmenu}
\item \helpref{wxMessageDialog}{wxmessagedialog}
\item \helpref{wxMetaFileDC}{wxmetafiledc}
\item \helpref{wxMiniFrame}{wxminiframe}
\item \helpref{wxMouseEvent}{wxmouseevent}
\item \helpref{wxMoveEvent}{wxmoveevent}
\item \helpref{wxNotebookEvent}{wxnotebookevent}
\item \helpref{wxNotebook}{wxnotebook}
\item \helpref{wxPageSetupDialogData}{wxpagesetupdialogdata}
\item \helpref{wxPageSetupDialog}{wxpagesetupdialog}
\item \helpref{wxPaintDC}{wxpaintdc}
\item \helpref{wxPaintEvent}{wxpaintevent}
\item \helpref{wxPalette}{wxpalette}
\item \helpref{wxPanel}{wxpanel}
\item \helpref{wxPen}{wxpen}
\item wxPNGHandler
\item \helpref{wxPoint}{wxpoint}
\item \helpref{wxPostScriptDC}{wxpostscriptdc}
\item \helpref{wxPrintData}{wxprintdata}
\item \helpref{wxPrintDialog}{wxprintdialog}
\item \helpref{wxPrinterDC}{wxprinterdc}
\item \helpref{wxQueryLayoutInfoEvent}{wxquerylayoutinfoevent}
\item \helpref{wxRadioBox}{wxradiobox}
\item \helpref{wxRadioButton}{wxradiobutton}
\item \helpref{wxRealPoint}{wxrealpoint}
\item \helpref{wxRect}{wxrect}
\item \helpref{wxRegionIterator}{wxregioniterator}
\item \helpref{wxRegion}{wxregion}
\item \helpref{wxSashEvent}{wxsashevent}
\item \helpref{wxSashLayoutWindow}{wxsashlayoutwindow}
\item \helpref{wxSashWindow}{wxsashwindow}
\item \helpref{wxScreenDC}{wxscreendc}
\item \helpref{wxScrollBar}{wxscrollbar}
\item \helpref{wxScrollEvent}{wxscrollevent}
\item \helpref{wxScrolledWindow}{wxscrolledwindow}
\item wxShowEvent
\item \helpref{wxSingleChoiceDialog}{wxsinglechoicedialog}
\item \helpref{wxSizeEvent}{wxsizeevent}
\item \helpref{wxSize}{wxsize}
\item \helpref{wxSlider}{wxslider}
\item \helpref{wxSpinButton}{wxspinbutton}
\item wxSpinEvent
\item \helpref{wxSplitterWindow}{wxsplitterwindow}
\item \helpref{wxStaticBitmap}{wxstaticbitmap}
\item \helpref{wxStaticBox}{wxstaticbox}
\item \helpref{wxStaticText}{wxstatictext}
\item \helpref{wxStatusBar}{wxstatusbar}
\item \helpref{wxSysColourChangedEvent}{wxsyscolourchangedevent}
\item \helpref{wxTaskBarIcon}{wxtaskbaricon}
\item \helpref{wxTextCtrl}{wxtextctrl}
\item \helpref{wxTextEntryDialog}{wxtextentrydialog}
\item \helpref{wxTimer}{wxtimer}
\item wxToolBarTool
\item \helpref{wxToolBar}{wxtoolbar}
\item wxToolTip
\item \helpref{wxTreeCtrl}{wxtreectrl}
\item \helpref{wxTreeEvent}{wxtreeevent}
\item \helpref{wxTreeItemData}{wxtreeitemdata}
\item wxTreeItemId
\item \helpref{wxUpdateUIEvent}{wxupdateuievent}
\item \helpref{wxWindowDC}{wxwindowdc}
\item \helpref{wxWindow}{wxwindow}
\end{itemize}

%----------------------------------------------------------------------
\section{Where to go for help}\label{wxphelp}

Since wxPython is a blending of multiple technologies, help comes from
multiple sources. See
\urlref{http://alldunn.com/wxPython}{http://alldunn.com/wxPython} for details on
various sources of help, but probably the best source is the
wxPython-users mail list.  You can view the archive or subscribe by
going to

\urlref{http://starship.python.net/mailman/listinfo/wxpython-users}{http://starship.python.net/mailman/listinfo/wxpython-users}

Or you can send mail directly to the list using this address:

wxpython-users@starship.python.net

