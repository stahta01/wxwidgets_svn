%
% automatically generated by HelpGen from
% spinctrl.h at 11/Mar/00 00:22:05
%

\section{\class{wxSpinCtrl}}\label{wxspinctrl}

wxSpinCtrl combines \helpref{wxTextCtrl}{wxtextctrl} and 
\helpref{wxSpinButton}{wxspinbutton} in one control.

\wxheading{Derived from}

\helpref{wxControl}{wxcontrol}\\
\helpref{wxWindow}{wxwindow}\\
\helpref{wxEvtHandler}{wxevthandler}\\
\helpref{wxObject}{wxobject}

\wxheading{Include files}

<wx/spinctrl.h>

\wxheading{Window styles}

\twocolwidtha{5cm}
\begin{twocollist}\itemsep=0pt
\twocolitem{\windowstyle{wxSP\_ARROW\_KEYS}}{The user can use arrow keys.}
\twocolitem{\windowstyle{wxSP\_WRAP}}{The value wraps at the minimum and maximum.}
\end{twocollist}

\wxheading{Event handling}

To process input from a spin button, use one of these event handler macros to direct input to member
functions that take a \helpref{wxSpinEvent}{wxspinevent} argument:

\twocolwidtha{7cm}
\begin{twocollist}\itemsep=0pt
\twocolitem{{\bf EVT\_SPINCTRL(id, func)}}{Generated whenever the numeric value
of the spinctrl is updated}
\end{twocollist}%

You may also use the \helpref{wxSpinButton}{wxspinbutton} event macros, however
the corresponding events will not be generated under all platforms. Finally, if
the user modifies the text in the edit part of the spin control directly, the
{\tt EVT\_TEXT} is generated, like for the \helpref{wxTextCtrl}{wxtextctrl}.

\wxheading{See also}

\helpref{Event handling overview}{eventhandlingoverview},
\helpref{wxSpinButton}{wxspinbutton},
\helpref{wxControl}{wxcontrol}

\latexignore{\rtfignore{\wxheading{Members}}}

\membersection{wxSpinCtrl::wxSpinCtrl}\label{wxspinctrlwxspinctrl}

\func{}{wxSpinCtrl}{\void}

Default constructor.

\func{}{wxSpinCtrl}{\param{wxWindow* }{parent}, \param{wxWindowID }{id = -1}, \param{const wxString\& }{value = wxEmptyString}, \param{const wxPoint\& }{pos = wxDefaultPosition}, \param{const wxSize\& }{size = wxDefaultSize}, \param{long }{style = wxSP\_ARROW\_KEYS}, \param{int }{min = 0}, \param{int }{max = 100}, \param{int }{initial = 0}, \param{const wxString\& }{name = \_T("wxSpinCtrl")}}

Constructor, creating and showing a spin control.

\wxheading{Parameters}

\docparam{parent}{Parent window. Must not be NULL.}

\docparam{value}{Default value.}

\docparam{id}{Window identifier. A value of -1 indicates a default value.}

\docparam{pos}{Window position. If the position (-1, -1) is specified then a default position is chosen.}

\docparam{size}{Window size. If the default size (-1, -1) is specified then a default size is chosen.}

\docparam{style}{Window style. See \helpref{wxSpinButton}{wxspinbutton}.}

\docparam{min}{Minimal value.}

\docparam{max}{Maximal value.}

\docparam{initial}{Initial value.}

\docparam{name}{Window name.}

\wxheading{See also}

\helpref{wxSpinCtrl::Create}{wxspinctrlcreate}

\membersection{wxSpinCtrl::Create}\label{wxspinctrlcreate}

\func{bool}{Create}{\param{wxWindow* }{parent}, \param{wxWindowID }{id = -1}, \param{const wxString\& }{value = wxEmptyString}, \param{const wxPoint\& }{pos = wxDefaultPosition}, \param{const wxSize\& }{size = wxDefaultSize}, \param{long }{style = wxSP\_ARROW\_KEYS}, \param{int }{min = 0}, \param{int }{max = 100}, \param{int }{initial = 0}, \param{const wxString\& }{name = \_T("wxSpinCtrl")}}

Creation function called by the spin control constructor.

See \helpref{wxSpinCtrl::wxSpinCtrl}{wxspinctrlwxspinctrl} for details.

\membersection{wxSpinCtrl::SetValue}\label{wxspinctrlsetvalue}

\func{void}{SetValue}{\param{const wxString\& }{text}}

\func{void}{SetValue}{\param{int }{value}}

Sets the value of the spin control.

\membersection{wxSpinCtrl::GetValue}\label{wxspinctrlgetvalue}

\constfunc{int}{GetValue}{\void}

Gets the value of the spin control.

\membersection{wxSpinCtrl::SetRange}\label{wxspinctrlsetrange}

\func{void}{SetRange}{\param{int }{minVal}, \param{int }{maxVal}}

Sets range of allowable values.

\membersection{wxSpinCtrl::SetSelection}\label{wxspinctrlsetselection}

\func{void}{SetSelection}{\param{long }{from}, \param{long }{to}}

Select the text in the text part of the control between  positions 
{\it from} (inclusive) and {\it to} (exclusive). This is similar to 
\helpref{wxTextCtrl::SetSelection}{wxtextctrlsetselection}.

{\bf NB:} this is currently only implemented for Windows and generic versions
of the control.

\membersection{wxSpinCtrl::GetMin}\label{wxspinctrlgetmin}

\constfunc{int}{GetMin}{\void}

Gets minimal allowable value.

\membersection{wxSpinCtrl::GetMax}\label{wxspinctrlgetmax}

\constfunc{int}{GetMax}{\void}

Gets maximal allowable value.

