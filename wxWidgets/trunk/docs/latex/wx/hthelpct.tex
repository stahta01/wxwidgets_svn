%
% automatically generated by HelpGen from
% htmlhelp.h at 02/May/99 19:58:53
%

\section{\class{wxHtmlHelpController}}\label{wxhtmlhelpcontroller}

{\bf WARNING! This help controller has an API incompatible with wxWindows
wxHelpController!}

This help controller provides an easy way of displaying HTML help in your
application (see {\it test} sample). The help system is based on {\bf books}
(see \helpref{AddBook}{wxhtmlhelpcontrolleraddbook}). A book is a logical
section of documentation (for example "User's Guide" or "Programmer's Guide" or
"C++ Reference" or "wxWindows Reference"). The help controller can handle as
many books as you want.

wxHTML uses Microsoft's HTML Help Workshop project files (.hhp, .hhk, .hhc) as its
native format. The file format is described \helpref{here}{helpformat}.
Have a look at docs/html/ directory where sample project files are stored.

You can use Tex2RTF to produce these files when generating HTML, if you set {\bf htmlWorkshopFiles} to {\bf true} in
your tex2rtf.ini file.

In order to use the controller in your application under Windows you must
have the following line in your .rc file:

\begin{verbatim}
#include "wx/html/msw/wxhtml.rc"
\end{verbatim}

\wxheading{Note}

It is strongly recommended to use preprocessed {\bf .hhp.cached} version of
projects. It can be either created on-the-fly (see 
\helpref{SetTempDir}{wxhtmlhelpcontrollersettempdir}) or you can use
{\bf hhp2cached} utility from {\it utils/hhp2cached} to create it and
distribute the cached version together with helpfiles. See {\it samples/html/help}
sample for demonstration of its use.

\wxheading{Derived from}

\helpref{wxEvtHandler}{wxevthandler}

\latexignore{\rtfignore{\wxheading{Members}}}

\membersection{wxHtmlHelpController::wxHtmlHelpController}\label{wxhtmlhelpcontrollerwxhtmlhelpcontroller}

\func{}{wxHtmlHelpController}{\param{int }{style = wxHF\_DEFAULTSTYLE}}

Constructor.

\wxheading{Parameters}

{\it style} is combination of these flags:

\begin{twocollist}
\twocolitem{\windowstyle{wxHF\_TOOLBAR}}{Help frame has toolbar.}
\twocolitem{\windowstyle{wxHF\_CONTENTS}}{Help frame has contents panel.}
\twocolitem{\windowstyle{wxHF\_INDEX}}{Help frame has index panel.}
\twocolitem{\windowstyle{wxHF\_SEARCH}}{Help frame has search panel.}
\twocolitem{\windowstyle{wxHF\_BOOKMARKS}}{Help frame has bookmarks controls.}
\twocolitem{\windowstyle{wxHF\_OPENFILES}}{Allow user to open arbitrary HTML document.}
\twocolitem{\windowstyle{wxHF\_PRINT}}{Toolbar contains "print" button.}
\end{twocollist}

Default value: everything but wxHF\_OPENFILES enabled.

\membersection{wxHtmlHelpController::AddBook}\label{wxhtmlhelpcontrolleraddbook}

\func{bool}{AddBook}{\param{const wxString\& }{book}, \param{bool }{show\_wait\_msg}}

Adds book (\helpref{.hhp file}{helpformat} - HTML Help Workshop project file) into the list of loaded books.
This must be called at least once before displaying  any help.

{\it book} may be either .hhp file or ZIP archive that contains arbitrary number of .hhp files in 
top-level directory. This ZIP archive must have .zip or .htb extension
(the latter stands for "HTML book"). In other words, {\tt AddBook("help.zip")} is possible and, in fact,
recommended way.

If {\it show\_wait\_msg} is TRUE then a decorationless window with progress message is displayed.

\membersection{wxHtmlHelpController::Display}\label{wxhtmlhelpcontrollerdisplay}

\func{void}{Display}{\param{const wxString\& }{x}}

Displays page {\it x}. This is THE important function - it is used to display
the help in application.

You can specify the page in many ways:

\begin{itemize}\itemsep=0pt
\item as direct filename of HTML document
\item as chapter name (from contents) or as a book name
\item as some word from index
\item even as any word (will be searched)
\end{itemize}

Looking for the page runs in these steps:

\begin{enumerate}\itemsep=0pt
\item try to locate file named x (if x is for example "doc/howto.htm")
\item try to open starting page of book named x
\item try to find x in contents (if x is for example "How To ...")
\item try to find x in index (if x is for example "How To ...")
\item switch to Search panel and start searching
\end{enumerate}

\func{void}{Display}{\param{const int }{id}}

This alternative form is used to search help contents by numeric IDs.

\pythonnote{The second form of this method is named DisplayId in
wxPython.}

\membersection{wxHtmlHelpController::DisplayContents}\label{wxhtmlhelpcontrollerdisplaycontents}

\func{void}{DisplayContents}{\void}

Displays help window and focuses contents panel.

\membersection{wxHtmlHelpController::DisplayIndex}\label{wxhtmlhelpcontrollerdisplayindex}

\func{void}{DisplayIndex}{\void}

Displays help window and focuses index panel.

\membersection{wxHtmlHelpController::KeywordSearch}\label{wxhtmlhelpcontrollerkeywordsearch}

\func{bool}{KeywordSearch}{\param{const wxString\& }{keyword}}

Displays help window, focuses search panel and starts searching.
Returns TRUE if the keyword was found.

{\bf Important:} KeywordSearch searches only pages listed in .htc file(s).
You should list all pages in the contents file.

\membersection{wxHtmlHelpController::ReadCustomization}\label{wxhtmlhelpcontrollerreadcustomization}

\func{void}{ReadCustomization}{\param{wxConfigBase* }{cfg}, \param{wxString }{path = wxEmptyString}}

Reads the controller's setting (position of window, etc.)

\membersection{wxHtmlHelpController::SetTempDir}\label{wxhtmlhelpcontrollersettempdir}

\func{void}{SetTempDir}{\param{const wxString\& }{path}}

Sets the path for storing temporary files - cached binary versions of index and contents files. These binary
forms are much faster to read. Default value is empty string (empty string means
that no cached data are stored). Note that these files are {\it not}
deleted when program exits.

Once created these cached files will be used in all subsequent executions 
of your application. If cached files become older than corresponding .hhp
file (e.g. if you regenerate documentation) it will be refreshed.

\membersection{wxHtmlHelpController::SetTitleFormat}\label{wxhtmlhelpcontrollersettitleformat}

\func{void}{SetTitleFormat}{\param{const wxString\& }{format}}

Sets format of title of the frame. Must contain exactly one "\%s"
(for title of displayed HTML page).

\membersection{wxHtmlHelpController::UseConfig}\label{wxhtmlhelpcontrolleruseconfig}

\func{void}{UseConfig}{\param{wxConfigBase* }{config}, \param{const wxString\& }{rootpath = wxEmptyString}}

Associates {\it config} object with the controller.

If there is associated config object, wxHtmlHelpController automatically
reads and writes settings (including wxHtmlWindow's settings) when needed.

The only thing you must do is create wxConfig object and call UseConfig.

If you do not use {\it UseConfig}, wxHtmlHelpController will use 
default wxConfig object if available (for details see 
\helpref{wxConfigBase::Get}{wxconfigbaseget} and 
\helpref{wxConfigBase::Set}{wxconfigbaseset}).

\membersection{wxHtmlHelpController::WriteCustomization}\label{wxhtmlhelpcontrollerwritecustomization}

\func{void}{WriteCustomization}{\param{wxConfigBase* }{cfg}, \param{wxString }{path = wxEmptyString}}

Stores controllers setting (position of window etc.)

\membersection{wxHtmlHelpController::CreateHelpFrame}\label{wxhtmlhelpcontrollercreatehelpframe}

\func{virtual wxHtmlHelpFrame*}{CreateHelpFrame}{\param{wxHtmlHelpData * }{data}}

This protected virtual method may be overriden so that the controller
uses slightly different frame. See {\it samples/html/helpview} sample for
an example.


