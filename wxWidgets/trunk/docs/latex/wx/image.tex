\section{\class{wxImage}}\label{wximage}

This class encapsulates a platform-independent image. An image can be created
from data, or using \helpref{wxBitmap::ConvertToImage}{wxbitmapconverttoimage}. An image
can be loaded from a file in a variety of formats, and is extensible to new formats
via image format handlers. Functions are available to set and get image bits, so
it can be used for basic image manipulation.

A wxImage cannot (currently) be drawn directly to a \helpref{wxDC}{wxdc}. Instead, 
a platform-specific \helpref{wxBitmap}{wxbitmap} object must be created from it using
the \helpref{wxBitmap::wxBitmap(wxImage,int depth)}{wxbitmapconstr} constructor. 
This bitmap can then
be drawn in a device context, using \helpref{wxDC::DrawBitmap}{wxdcdrawbitmap}.

One colour value of the image may be used as a mask colour which will lead to the automatic
creation of a \helpref{wxMask}{wxmask} object associated to the bitmap object.

\wxheading{Alpha channel support}

Starting from wxWidgets 2.5.0 wxImage supports alpha channel data, that is in
addition to a byte for the red, green and blue colour components for each pixel
it also stores a byte representing the pixel opacity. The alpha value of $0$
corresponds to a transparent pixel (null opacity) while the value of $255$
means that the pixel is 100\% opaque.

Unlike the RGB data, not all images have the alpha channel and before using 
\helpref{GetAlpha}{wximagegetalpha} you should check if this image contains
alpha value with \helpref{HasAlpha}{wximagehasalpha}. In fact, currently only
images loaded from PNG files with transparency information will have alpha
channel but support for it will be added to the other formats as well (as well
as support for saving images with alpha channel which is not still implemented
either).

\wxheading{Available image handlers}

The following image handlers are available. {\bf wxBMPHandler} is always
installed by default. To use other image formats, install the appropriate
handler with \helpref{wxImage::AddHandler}{wximageaddhandler} or 
\helpref{wxInitAllImageHandlers}{wxinitallimagehandlers}.

\twocolwidtha{5cm}%
\begin{twocollist}
\twocolitem{\indexit{wxBMPHandler}}{For loading and saving, always installed.}
\twocolitem{\indexit{wxPNGHandler}}{For loading (including alpha support) and saving.}
\twocolitem{\indexit{wxJPEGHandler}}{For loading and saving.}
\twocolitem{\indexit{wxGIFHandler}}{Only for loading, due to legal issues.}
\twocolitem{\indexit{wxPCXHandler}}{For loading and saving (see below).}
\twocolitem{\indexit{wxPNMHandler}}{For loading and saving (see below).}
\twocolitem{\indexit{wxTIFFHandler}}{For loading and saving.}
\twocolitem{\indexit{wxIFFHandler}}{For loading only.}
\twocolitem{\indexit{wxXPMHandler}}{For loading and saving.}
\twocolitem{\indexit{wxICOHandler}}{For loading and saving.}
\twocolitem{\indexit{wxCURHandler}}{For loading and saving.}
\twocolitem{\indexit{wxANIHandler}}{For loading only.}
\end{twocollist}

When saving in PCX format, {\bf wxPCXHandler} will count the number of
different colours in the image; if there are 256 or less colours, it will
save as 8 bit, else it will save as 24 bit.

Loading PNMs only works for ASCII or raw RGB images. When saving in
PNM format, {\bf wxPNMHandler} will always save as raw RGB.

\wxheading{Derived from}

\helpref{wxObject}{wxobject}

\wxheading{Include files}

<wx/image.h>

\wxheading{See also}

\helpref{wxBitmap}{wxbitmap}, 
\helpref{wxInitAllImageHandlers}{wxinitallimagehandlers}

\latexignore{\rtfignore{\wxheading{Members}}}

\membersection{wxImage::wxImage}\label{wximageconstr}

\func{}{wxImage}{\void}

Default constructor.

\func{}{wxImage}{\param{const wxImage\& }{image}}

Copy constructor.

\func{}{wxImage}{\param{const wxBitmap\&}{ bitmap}}

(Deprecated form, use \helpref{wxBitmap::ConvertToImage}{wxbitmapconverttoimage}
instead.) Constructs an image from a platform-dependent bitmap. This preserves
mask information so that bitmaps and images can be converted back
and forth without loss in that respect.

\func{}{wxImage}{\param{int}{ width}, \param{int}{ height}, \param{bool}{ clear=true}}

Creates an image with the given width and height.  If {\it clear} is true, the new image will be initialized to black.
Otherwise, the image data will be uninitialized.

\func{}{wxImage}{\param{int}{ width}, \param{int}{ height}, \param{unsigned char*}{ data}, \param{bool}{ static\_data=false}}

Creates an image from given data with the given width and height. If 
{\it static\_data} is true, then wxImage will not delete the actual
image data in its destructor, otherwise it will free it by calling
{\it free()}.

\func{}{wxImage}{\param{const wxString\& }{name}, \param{long}{ type = wxBITMAP\_TYPE\_ANY}, \param{int}{ index = -1}}

\func{}{wxImage}{\param{const wxString\& }{name}, \param{const wxString\&}{ mimetype}, \param{int}{ index = -1}}

Loads an image from a file.

\func{}{wxImage}{\param{wxInputStream\& }{stream}, \param{long}{ type = wxBITMAP\_TYPE\_ANY}, \param{int}{ index = -1}}

\func{}{wxImage}{\param{wxInputStream\& }{stream}, \param{const wxString\&}{ mimetype}, \param{int}{ index = -1}}

Loads an image from an input stream.

\wxheading{Parameters}

\docparam{width}{Specifies the width of the image.}

\docparam{height}{Specifies the height of the image.}

\docparam{name}{Name of the file from which to load the image.}

\docparam{stream}{Opened input stream from which to load the image. Currently, the stream must support seeking.}

\docparam{type}{May be one of the following:

\twocolwidtha{5cm}%
\begin{twocollist}
\twocolitem{\indexit{wxBITMAP\_TYPE\_BMP}}{Load a Windows bitmap file.}
\twocolitem{\indexit{wxBITMAP\_TYPE\_GIF}}{Load a GIF bitmap file.}
\twocolitem{\indexit{wxBITMAP\_TYPE\_JPEG}}{Load a JPEG bitmap file.}
\twocolitem{\indexit{wxBITMAP\_TYPE\_PNG}}{Load a PNG bitmap file.}
\twocolitem{\indexit{wxBITMAP\_TYPE\_PCX}}{Load a PCX bitmap file.}
\twocolitem{\indexit{wxBITMAP\_TYPE\_PNM}}{Load a PNM bitmap file.}
\twocolitem{\indexit{wxBITMAP\_TYPE\_TIF}}{Load a TIFF bitmap file.}
\twocolitem{\indexit{wxBITMAP\_TYPE\_XPM}}{Load a XPM bitmap file.}
\twocolitem{\indexit{wxBITMAP\_TYPE\_ICO}}{Load a Windows icon file (ICO).}
\twocolitem{\indexit{wxBITMAP\_TYPE\_CUR}}{Load a Windows cursor file (CUR).}
\twocolitem{\indexit{wxBITMAP\_TYPE\_ANI}}{Load a Windows animated cursor file (ANI).}
\twocolitem{\indexit{wxBITMAP\_TYPE\_ANY}}{Will try to autodetect the format.}
\end{twocollist}}

\docparam{mimetype}{MIME type string (for example 'image/jpeg')}

\docparam{index}{Index of the image to load in the case that the image file contains multiple images.
This is only used by GIF, ICO and TIFF handlers. The default value (-1) means 
"choose the default image" and is interpreted as the first image (index=0) by 
the GIF and TIFF handler and as the largest and most colourful one by the ICO handler.}

\wxheading{Remarks}

Depending on how wxWidgets has been configured, not all formats may be available.

Note: any handler other than BMP must be previously
initialized with \helpref{wxImage::AddHandler}{wximageaddhandler} or 
\helpref{wxInitAllImageHandlers}{wxinitallimagehandlers}.

Note: you can use \helpref{GetOptionInt}{wximagegetoptionint} to get the 
hotspot for loaded cursor file:
\begin{verbatim}
    int hotspot_x = image.GetOptionInt(wxIMAGE_OPTION_CUR_HOTSPOT_X);
    int hotspot_y = image.GetOptionInt(wxIMAGE_OPTION_CUR_HOTSPOT_Y);

\end{verbatim}

\wxheading{See also}

\helpref{wxImage::LoadFile}{wximageloadfile}

\pythonnote{Constructors supported by wxPython are:\par
\indented{2cm}{\begin{twocollist}
\twocolitem{{\bf wxImage(name, flag)}}{Loads an image from a file}
\twocolitem{{\bf wxNullImage()}}{Create a null image (has no size or
image data)}
\twocolitem{{\bf wxEmptyImage(width, height)}}{Creates an empty image
of the given size}
\twocolitem{{\bf wxImageFromMime(name, mimetype}}{Creates an image from
the given file of the given mimetype}
\twocolitem{{\bf wxImageFromBitmap(bitmap)}}{Creates an image from a
platform-dependent bitmap}
\end{twocollist}}
}

\perlnote{Constructors supported by wxPerl are:\par
\begin{itemize}
\item{Wx::Image->new( bitmap )}
\item{Wx::Image->new( icon )}
\item{Wx::Image->new( width, height )}
\item{Wx::Image->new( width, height, data )}
\item{Wx::Image->new( file, type, index )}
\item{Wx::Image->new( file, mimetype, index )}
\item{Wx::Image->new( stream, type, index )}
\item{Wx::Image->new( stream, mimetype, index )}
\end{itemize}
}

\membersection{wxImage::\destruct{wxImage}}

\func{}{\destruct{wxImage}}{\void}

Destructor.

\membersection{wxImage::AddHandler}\label{wximageaddhandler}

\func{static void}{AddHandler}{\param{wxImageHandler*}{ handler}}

Adds a handler to the end of the static list of format handlers.

\docparam{handler}{A new image format handler object. There is usually only one instance
of a given handler class in an application session.}

\wxheading{See also}

\helpref{wxImageHandler}{wximagehandler}

\func{bool}{CanRead}{\param{const wxString\&}{ filename}}

returns true if the current image handlers can read this file

\pythonnote{In wxPython this static method is named {\tt wxImage\_AddHandler}.}
\membersection{wxImage::CleanUpHandlers}

\func{static void}{CleanUpHandlers}{\void}

Deletes all image handlers.

This function is called by wxWidgets on exit.

\membersection{wxImage::ComputeHistogram}\label{wximagecomputehistogram}

\constfunc{unsigned long}{ComputeHistogram}{\param{wxImageHistogram\& }{histogram}}

Computes the histogram of the image. {\it histogram} is a reference to 
wxImageHistogram object. wxImageHistogram is a specialization of 
\helpref{wxHashMap}{wxhashmap} "template" and is defined as follows:

\begin{verbatim}
class WXDLLEXPORT wxImageHistogramEntry
{
public:
    wxImageHistogramEntry() : index(0), value(0) {}
    unsigned long index;
    unsigned long value;
};

WX_DECLARE_EXPORTED_HASH_MAP(unsigned long, wxImageHistogramEntry,
                             wxIntegerHash, wxIntegerEqual,
                             wxImageHistogram);
\end{verbatim}

\wxheading{Return value}

Returns number of colours in the histogram.

\membersection{wxImage::ConvertAlphaToMask}\label{wximageconvertalphatomask}

\func{bool}{ConvertAlphaToMask}{\param{unsigned char}{ threshold = 128}}

If the image has alpha channel, this method converts it to mask. All pixels
with alpha value less than \arg{threshold} are replaced with mask colour
and the alpha channel is removed. Mask colour is chosen automatically using
\helpref{FindFirstUnusedColour}{wximagefindfirstunusedcolour}.

If the image image doesn't have alpha channel,
ConvertAlphaToMask does nothing.

\wxheading{Return value}

\false if FindFirstUnusedColour returns \false, \true otherwise. 

\membersection{wxImage::ConvertToBitmap}\label{wximageconverttobitmap}

\constfunc{wxBitmap}{ConvertToBitmap}{\void}

Deprecated, use equivalent \helpref{wxBitmap constructor}{wxbitmapconstr}
(which takes wxImage and depth as its arguments) instead.

\membersection{wxImage::ConvertToMono}\label{wxbitmapconverttomono}

\constfunc{wxImage}{ConvertToMono}{\param{unsigned char}{ r}, \param{unsigned char}{ g}, \param{unsigned char}{ b}}

Returns monochromatic version of the image. The returned image has white
colour where the original has {\it (r,g,b)} colour and black colour 
everywhere else.

\membersection{wxImage::Copy}\label{wximagecopy}

\constfunc{wxImage}{Copy}{\void}

Returns an identical copy of the image.

\membersection{wxImage::Create}\label{wximagecreate}

\func{bool}{Create}{\param{int}{ width}, \param{int}{ height}, \param{bool}{ clear=true}}

Creates a fresh image.  If {\it clear} is true, the new image will be initialized to black.
Otherwise, the image data will be uninitialized.

\wxheading{Parameters}

\docparam{width}{The width of the image in pixels.}

\docparam{height}{The height of the image in pixels.}

\wxheading{Return value}

true if the call succeeded, false otherwise.

\membersection{wxImage::Destroy}\label{wximagedestroy}

\func{bool}{Destroy}{\void}

Destroys the image data.

\membersection{wxImage::FindFirstUnusedColour}\label{wximagefindfirstunusedcolour}

\func{bool}{FindFirstUnusedColour}{\param{unsigned char *}{ r}, \param{unsigned char *}{ g}, \param{unsigned char *}{ b}, \param{unsigned char}{ startR = 1}, \param{unsigned char}{ startG = 0}, \param{unsigned char}{ startB = 0}}

\wxheading{Parameters}

\docparam{r,g,b}{Pointers to variables to save the colour.}

\docparam{startR,startG,startB}{Initial values of the colour. Returned colour
will have RGB values equal to or greater than these.}

Finds the first colour that is never used in the image. The search begins at
given initial colour and continues by increasing R, G and B components (in this
order) by 1 until an unused colour is found or the colour space exhausted.

\wxheading{Return value}

Returns false if there is no unused colour left, true on success.

\wxheading{Notes}

Note that this method involves computing the histogram, which is
computationally intensive operation.

\membersection{wxImage::FindHandler}

\func{static wxImageHandler*}{FindHandler}{\param{const wxString\& }{name}}

Finds the handler with the given name.

\func{static wxImageHandler*}{FindHandler}{\param{const wxString\& }{extension}, \param{long}{ imageType}}

Finds the handler associated with the given extension and type.

\func{static wxImageHandler*}{FindHandler}{\param{long }{imageType}}

Finds the handler associated with the given image type.

\func{static wxImageHandler*}{FindHandlerMime}{\param{const wxString\& }{mimetype}}

Finds the handler associated with the given MIME type.

\docparam{name}{The handler name.}

\docparam{extension}{The file extension, such as ``bmp".}

\docparam{imageType}{The image type, such as wxBITMAP\_TYPE\_BMP.}

\docparam{mimetype}{MIME type.}

\wxheading{Return value}

A pointer to the handler if found, NULL otherwise.

\wxheading{See also}

\helpref{wxImageHandler}{wximagehandler}

\membersection{wxImage::GetImageExtWildcard}

\func{static wxString}{GetImageExtWildcard}{\void}

Iterates all registered wxImageHandler objects, and returns a string containing file extension masks
suitable for passing to file open/save dialog boxes.

\wxheading{Return value}

The format of the returned string is "(*.ext1;*.ext2)|*.ext1;*.ext2".

It is usually a good idea to prepend a description before passing the result to the dialog.

Example:

\begin{verbatim}
    wxFileDialog FileDlg( this, "Choose Image", ::wxGetWorkingDirectory(), "", _("Image Files ") + wxImage::GetImageExtWildcard(), wxOPEN );
\end{verbatim}

\wxheading{See also}

\helpref{wxImageHandler}{wximagehandler}

\membersection{wxImage::GetAlpha}\label{wximagegetalpha}

\constfunc{unsigned char}{GetAlpha}{\param{int}{ x}, \param{int}{ y}}

Returns the alpha value for the given pixel. This function may only be called
for the images with alpha channel, use \helpref{HasAlpha}{wximagehasalpha} to
check for this.

The returned value is the {\it opacity} of the image, i.e. the value of $0$
corresponds to the transparent pixels while the value of $255$ -- to the opaque
ones.

\constfunc{unsigned char *}{GetAlpha}{\void}

Returns pointer to the array storing the alpha values for this image. This
pointer is {\tt NULL} for the images without the alpha channel. If the image
does have it, this pointer may be used to directly manipulate the alpha values
which are stored as the \helpref{RGB}{wximagegetdata} ones.

\membersection{wxImage::GetBlue}\label{wximagegetblue}

\constfunc{unsigned char}{GetBlue}{\param{int}{ x}, \param{int}{ y}}

Returns the blue intensity at the given coordinate.

\membersection{wxImage::GetData}\label{wximagegetdata}

\constfunc{unsigned char*}{GetData}{\void}

Returns the image data as an array. This is most often used when doing
direct image manipulation. The return value points to an array of
characters in RGBRGBRGB$\ldots$ format in the top-to-bottom, left-to-right
order, that is the first RGB triplet corresponds to the pixel first pixel of
the first row, the second one --- to the second pixel of the first row and so
on until the end of the first row, with second row following after it and so
on.

You should not delete the returned pointer nor pass it to
\helpref{wxImage::SetData}{wximagesetdata}.

\membersection{wxImage::GetGreen}\label{wximagegetgreen}

\constfunc{unsigned char}{GetGreen}{\param{int}{ x}, \param{int}{ y}}

Returns the green intensity at the given coordinate.

\membersection{wxImage::GetImageCount}\label{wximagegetimagecount}

\func{static int}{GetImageCount}{\param{const wxString\&}{ filename}, \param{long}{ type = wxBITMAP\_TYPE\_ANY}}

\func{static int}{GetImageCount}{\param{wxInputStream\&}{ stream}, \param{long}{ type = wxBITMAP\_TYPE\_ANY}}

If the image file contains more than one image and the image handler is capable 
of retrieving these individually, this function will return the number of
available images.

\docparam{name}{Name of the file to query.}

\docparam{stream}{Opened input stream with image data. Currently, the stream must support seeking.}

\docparam{type}{May be one of the following:

\twocolwidtha{5cm}%
\begin{twocollist}
\twocolitem{\indexit{wxBITMAP\_TYPE\_BMP}}{Load a Windows bitmap file.}
\twocolitem{\indexit{wxBITMAP\_TYPE\_GIF}}{Load a GIF bitmap file.}
\twocolitem{\indexit{wxBITMAP\_TYPE\_JPEG}}{Load a JPEG bitmap file.}
\twocolitem{\indexit{wxBITMAP\_TYPE\_PNG}}{Load a PNG bitmap file.}
\twocolitem{\indexit{wxBITMAP\_TYPE\_PCX}}{Load a PCX bitmap file.}
\twocolitem{\indexit{wxBITMAP\_TYPE\_PNM}}{Load a PNM bitmap file.}
\twocolitem{\indexit{wxBITMAP\_TYPE\_TIF}}{Load a TIFF bitmap file.}
\twocolitem{\indexit{wxBITMAP\_TYPE\_XPM}}{Load a XPM bitmap file.}
\twocolitem{\indexit{wxBITMAP\_TYPE\_ICO}}{Load a Windows icon file (ICO).}
\twocolitem{\indexit{wxBITMAP\_TYPE\_CUR}}{Load a Windows cursor file (CUR).}
\twocolitem{\indexit{wxBITMAP\_TYPE\_ANI}}{Load a Windows animated cursor file (ANI).}
\twocolitem{\indexit{wxBITMAP\_TYPE\_ANY}}{Will try to autodetect the format.}
\end{twocollist}}

\wxheading{Return value}

Number of available images. For most image handlers, this is 1 (exceptions
are TIFF and ICO formats).

\membersection{wxImage::GetHandlers}

\func{static wxList\&}{GetHandlers}{\void}

Returns the static list of image format handlers.

\wxheading{See also}

\helpref{wxImageHandler}{wximagehandler}

\membersection{wxImage::GetHeight}\label{wximagegetheight}

\constfunc{int}{GetHeight}{\void}

Gets the height of the image in pixels.

\membersection{wxImage::GetMaskBlue}\label{wximagegetmaskblue}

\constfunc{unsigned char}{GetMaskBlue}{\void}

Gets the blue value of the mask colour.

\membersection{wxImage::GetMaskGreen}\label{wximagegetmaskgreen}

\constfunc{unsigned char}{GetMaskGreen}{\void}

Gets the green value of the mask colour.

\membersection{wxImage::GetMaskRed}\label{wximagegetmaskred}

\constfunc{unsigned char}{GetMaskRed}{\void}

Gets the red value of the mask colour.

\membersection{wxImage::GetPalette}\label{wximagegetpalette}

\constfunc{const wxPalette\&}{GetPalette}{\void}

Returns the palette associated with the image. Currently the palette is only
used when converting to wxBitmap under Windows.

Eventually wxImage handlers will set the palette if one exists in the image file.

\membersection{wxImage::GetRed}\label{wximagegetred}

\constfunc{unsigned char}{GetRed}{\param{int}{ x}, \param{int}{ y}}

Returns the red intensity at the given coordinate.

\membersection{wxImage::GetSubImage}\label{wximagegetsubimage}

\constfunc{wxImage}{GetSubImage}{\param{const wxRect\&}{ rect}}

Returns a sub image of the current one as long as the rect belongs entirely to 
the image.

\membersection{wxImage::GetWidth}\label{wximagegetwidth}

\constfunc{int}{GetWidth}{\void}

Gets the width of the image in pixels.

\wxheading{See also}

\helpref{wxImage::GetHeight}{wximagegetheight}

\membersection{wxImage::HasAlpha}\label{wximagehasalpha}

\constfunc{bool}{HasAlpha}{\void}

Returns true if this image has alpha channel, false otherwise.

\wxheading{See also}

\helpref{GetAlpha}{wximagegetalpha}, \helpref{SetAlpha}{wximagesetalpha}

\membersection{wxImage::HasMask}\label{wximagehasmask}

\constfunc{bool}{HasMask}{\void}

Returns true if there is a mask active, false otherwise.

\membersection{wxImage::GetOption}\label{wximagegetoption}

\constfunc{wxString}{GetOption}{\param{const wxString\&}{ name}}

Gets a user-defined option. The function is case-insensitive to {\it name}.

For example, when saving as a JPEG file, the option {\bf quality} is
used, which is a number between 0 and 100 (0 is terrible, 100 is very good).

\wxheading{See also}

\helpref{wxImage::SetOption}{wximagesetoption},\rtfsp
\helpref{wxImage::GetOptionInt}{wximagegetoptionint},\rtfsp
\helpref{wxImage::HasOption}{wximagehasoption}

\membersection{wxImage::GetOptionInt}\label{wximagegetoptionint}

\constfunc{int}{GetOptionInt}{\param{const wxString\&}{ name}}

Gets a user-defined option as an integer. The function is case-insensitive to {\it name}.

\wxheading{See also}

\helpref{wxImage::SetOption}{wximagesetoption},\rtfsp
\helpref{wxImage::GetOption}{wximagegetoption},\rtfsp
\helpref{wxImage::HasOption}{wximagehasoption}

\membersection{wxImage::HasOption}\label{wximagehasoption}

\constfunc{bool}{HasOption}{\param{const wxString\&}{ name}}

Returns true if the given option is present. The function is case-insensitive to {\it name}.

\wxheading{See also}

\helpref{wxImage::SetOption}{wximagesetoption},\rtfsp
\helpref{wxImage::GetOption}{wximagegetoption},\rtfsp
\helpref{wxImage::GetOptionInt}{wximagegetoptionint}

\membersection{wxImage::InitStandardHandlers}

\func{static void}{InitStandardHandlers}{\void}

Internal use only. Adds standard image format handlers. It only install BMP
for the time being, which is used by wxBitmap.

This function is called by wxWidgets on startup, and shouldn't be called by
the user.

\wxheading{See also}

\helpref{wxImageHandler}{wximagehandler}, 
\helpref{wxInitAllImageHandlers}{wxinitallimagehandlers}

\membersection{wxImage::InsertHandler}

\func{static void}{InsertHandler}{\param{wxImageHandler*}{ handler}}

Adds a handler at the start of the static list of format handlers.

\docparam{handler}{A new image format handler object. There is usually only one instance
of a given handler class in an application session.}

\wxheading{See also}

\helpref{wxImageHandler}{wximagehandler}

\membersection{wxImage::LoadFile}\label{wximageloadfile}

\func{bool}{LoadFile}{\param{const wxString\&}{ name}, \param{long}{ type = wxBITMAP\_TYPE\_ANY}, \param{int}{ index = -1}}

\func{bool}{LoadFile}{\param{const wxString\&}{ name}, \param{const wxString\&}{ mimetype}, \param{int}{ index = -1}}

Loads an image from a file. If no handler type is provided, the library will
try to autodetect the format.

\func{bool}{LoadFile}{\param{wxInputStream\&}{ stream}, \param{long}{ type}, \param{int}{ index = -1}}

\func{bool}{LoadFile}{\param{wxInputStream\&}{ stream}, \param{const wxString\&}{ mimetype}, \param{int}{ index = -1}}

Loads an image from an input stream.

\wxheading{Parameters}

\docparam{name}{Name of the file from which to load the image.}

\docparam{stream}{Opened input stream from which to load the image. Currently, the stream must support seeking.}

\docparam{type}{One of the following values:

\twocolwidtha{5cm}%
\begin{twocollist}
\twocolitem{{\bf wxBITMAP\_TYPE\_BMP}}{Load a Windows image file.}
\twocolitem{{\bf wxBITMAP\_TYPE\_GIF}}{Load a GIF image file.}
\twocolitem{{\bf wxBITMAP\_TYPE\_JPEG}}{Load a JPEG image file.}
\twocolitem{{\bf wxBITMAP\_TYPE\_PCX}}{Load a PCX image file.}
\twocolitem{{\bf wxBITMAP\_TYPE\_PNG}}{Load a PNG image file.}
\twocolitem{{\bf wxBITMAP\_TYPE\_PNM}}{Load a PNM image file.}
\twocolitem{{\bf wxBITMAP\_TYPE\_TIF}}{Load a TIFF image file.}
\twocolitem{{\bf wxBITMAP\_TYPE\_XPM}}{Load a XPM image file.}
\twocolitem{{\bf wxBITMAP\_TYPE\_ICO}}{Load a Windows icon file (ICO).}
\twocolitem{{\bf wxBITMAP\_TYPE\_CUR}}{Load a Windows cursor file (CUR).}
\twocolitem{\indexit{wxBITMAP\_TYPE\_ANI}}{Load a Windows animated cursor file (ANI).}
\twocolitem{{\bf wxBITMAP\_TYPE\_ANY}}{Will try to autodetect the format.}
\end{twocollist}}

\docparam{mimetype}{MIME type string (for example 'image/jpeg')}

\docparam{index}{Index of the image to load in the case that the image file contains multiple images.
This is only used by GIF, ICO and TIFF handlers. The default value (-1) means 
"choose the default image" and is interpreted as the first image (index=0) by 
the GIF and TIFF handler and as the largest and most colourful one by the ICO handler.}

\wxheading{Remarks}

Depending on how wxWidgets has been configured, not all formats may be available.

Note: you can use \helpref{GetOptionInt}{wximagegetoptionint} to get the 
hotspot for loaded cursor file:
\begin{verbatim}
    int hotspot_x = image.GetOptionInt(wxIMAGE_OPTION_CUR_HOTSPOT_X);
    int hotspot_y = image.GetOptionInt(wxIMAGE_OPTION_CUR_HOTSPOT_Y);

\end{verbatim}

\wxheading{Return value}

true if the operation succeeded, false otherwise. If the optional index parameter is out of range,
false is returned and a call to wxLogError() takes place.

\wxheading{See also}

\helpref{wxImage::SaveFile}{wximagesavefile}

\pythonnote{In place of a single overloaded method name, wxPython
implements the following methods:\par
\indented{2cm}{\begin{twocollist}
\twocolitem{{\bf LoadFile(filename, type)}}{Loads an image of the given
type from a file}
\twocolitem{{\bf LoadMimeFile(filename, mimetype)}}{Loads an image of the given
mimetype from a file}
\end{twocollist}}
}

\perlnote{Methods supported by wxPerl are:\par
\begin{itemize}
\item{bitmap->LoadFile( name, type )}
\item{bitmap->LoadFile( name, mimetype )}
\end{itemize}
}


\membersection{wxImage::Ok}\label{wximageok}

\constfunc{bool}{Ok}{\void}

Returns true if image data is present.

\membersection{wxImage::RemoveHandler}

\func{static bool}{RemoveHandler}{\param{const wxString\& }{name}}

Finds the handler with the given name, and removes it. The handler
is not deleted.

\docparam{name}{The handler name.}

\wxheading{Return value}

true if the handler was found and removed, false otherwise.

\wxheading{See also}

\helpref{wxImageHandler}{wximagehandler}

\membersection{wxImage::Mirror}\label{wximagemirror}

\constfunc{wxImage}{Mirror}{\param{bool}{ horizontally = true}}

Returns a mirrored copy of the image. The parameter {\it horizontally}
indicates the orientation.

\membersection{wxImage::Replace}\label{wximagereplace}

\func{void}{Replace}{\param{unsigned char}{ r1}, \param{unsigned char}{ g1}, \param{unsigned char}{ b1},
\param{unsigned char}{ r2}, \param{unsigned char}{ g2}, \param{unsigned char}{ b2}}

Replaces the colour specified by {\it r1,g1,b1} by the colour {\it r2,g2,b2}.

\membersection{wxImage::Rescale}\label{wximagerescale}

\func{wxImage \&}{Rescale}{\param{int}{ width}, \param{int}{ height}}

Changes the size of the image in-place: after a call to this function, the
image will have the given width and height.

Returns the (modified) image itself.

\wxheading{See also}

\helpref{Scale}{wximagescale}

\membersection{wxImage::Rotate}\label{wximagerotate}

\func{wxImage}{Rotate}{\param{double}{ angle}, \param{const wxPoint\& }{rotationCentre},
 \param{bool}{ interpolating = true}, \param{wxPoint*}{ offsetAfterRotation = NULL}}

Rotates the image about the given point, by {\it angle} radians. Passing true
to {\it interpolating} results in better image quality, but is slower. If the
image has a mask, then the mask colour is used for the uncovered pixels in the
rotated image background. Else, black (rgb 0, 0, 0) will be used.

Returns the rotated image, leaving this image intact.

\membersection{wxImage::Rotate90}\label{wximagerotate90}

\constfunc{wxImage}{Rotate90}{\param{bool}{ clockwise = true}}

Returns a copy of the image rotated 90 degrees in the direction
indicated by {\it clockwise}.

\membersection{wxImage::SaveFile}\label{wximagesavefile}

\constfunc{bool}{SaveFile}{\param{const wxString\& }{name}, \param{int}{ type}}

\constfunc{bool}{SaveFile}{\param{const wxString\& }{name}, \param{const wxString\&}{ mimetype}}

Saves an image in the named file.

\constfunc{bool}{SaveFile}{\param{const wxString\& }{name}}

Saves an image in the named file. File type is determined from the extension of the
file name. Note that this function may fail if the extension is not recognized! You
can use one of the forms above to save images to files with non-standard extensions.

\constfunc{bool}{SaveFile}{\param{wxOutputStream\& }{stream}, \param{int}{ type}}

\constfunc{bool}{SaveFile}{\param{wxOutputStream\& }{stream}, \param{const wxString\&}{ mimetype}}

Saves an image in the given stream.

\wxheading{Parameters}

\docparam{name}{Name of the file to save the image to.}

\docparam{stream}{Opened output stream to save the image to.}

\docparam{type}{Currently these types can be used:

\twocolwidtha{5cm}%
\begin{twocollist}
\twocolitem{{\bf wxBITMAP\_TYPE\_BMP}}{Save a BMP image file.}
\twocolitem{{\bf wxBITMAP\_TYPE\_JPEG}}{Save a JPEG image file.}
\twocolitem{{\bf wxBITMAP\_TYPE\_PNG}}{Save a PNG image file.}
\twocolitem{{\bf wxBITMAP\_TYPE\_PCX}}{Save a PCX image file (tries to save as 8-bit if possible, falls back to 24-bit otherwise).}
\twocolitem{{\bf wxBITMAP\_TYPE\_PNM}}{Save a PNM image file (as raw RGB always).}
\twocolitem{{\bf wxBITMAP\_TYPE\_TIFF}}{Save a TIFF image file.}
\twocolitem{{\bf wxBITMAP\_TYPE\_XPM}}{Save a XPM image file.}
\twocolitem{{\bf wxBITMAP\_TYPE\_ICO}}{Save a Windows icon file (ICO) (the size may be up to 255 wide by 127 high. A single image is saved in 8 colors at the size supplied).}
\twocolitem{{\bf wxBITMAP\_TYPE\_CUR}}{Save a Windows cursor file (CUR).}
\end{twocollist}}

\docparam{mimetype}{MIME type.}

\wxheading{Return value}

true if the operation succeeded, false otherwise.

\wxheading{Remarks}

Depending on how wxWidgets has been configured, not all formats may be available.

Note: you can use \helpref{GetOptionInt}{wximagegetoptionint} to set the 
hotspot before saving an image into a cursor file (default hotspot is in 
the centre of the image):
\begin{verbatim}
    image.SetOption(wxIMAGE_OPTION_CUR_HOTSPOT_X, hotspotX);
    image.SetOption(wxIMAGE_OPTION_CUR_HOTSPOT_Y, hotspotY);

\end{verbatim}

\wxheading{See also}

\helpref{wxImage::LoadFile}{wximageloadfile}

\pythonnote{In place of a single overloaded method name, wxPython
implements the following methods:\par
\indented{2cm}{\begin{twocollist}
\twocolitem{{\bf SaveFile(filename, type)}}{Saves the image using the given
type to the named file}
\twocolitem{{\bf SaveMimeFile(filename, mimetype)}}{Saves the image using the given
mimetype to the named file}
\end{twocollist}}
}

\perlnote{Methods supported by wxPerl are:\par
\begin{itemize}
\item{bitmap->SaveFile( name, type )}
\item{bitmap->SaveFile( name, mimetype )}
\end{itemize}
}

\membersection{wxImage::Scale}\label{wximagescale}

\constfunc{wxImage}{Scale}{\param{int}{ width}, \param{int}{ height}}

Returns a scaled version of the image. This is also useful for
scaling bitmaps in general as the only other way to scale bitmaps
is to blit a wxMemoryDC into another wxMemoryDC.

It may be mentioned that the GTK port uses this function internally
to scale bitmaps when using mapping modes in wxDC. 

Example:

\begin{verbatim}
    // get the bitmap from somewhere
    wxBitmap bmp = ...;

    // rescale it to have size of 32*32
    if ( bmp.GetWidth() != 32 || bmp.GetHeight() != 32 )
    {
        wxImage image = bmp.ConvertToImage();
        bmp = wxBitmap(image.Scale(32, 32));

        // another possibility:
        image.Rescale(32, 32);
        bmp = image;
    }

\end{verbatim}

\wxheading{See also}

\helpref{Rescale}{wximagerescale}

\membersection{wxImage::SetAlpha}\label{wximagesetalpha}

\func{void}{SetAlpha}{\param{unsigned char *}{alpha = {\tt NULL}}}

This function is similar to \helpref{SetData}{wximagesetdata} and has similar
restrictions. The pointer passed to it may however be {\tt NULL} in which case
the function will allocate the alpha array internally -- this is useful to add
alpha channel data to an image which doesn't have any. If the pointer is not 
{\tt NULL}, it must have one byte for each image pixel and be allocated with 
{\tt malloc()}. wxImage takes ownership of the pointer and will free it.

\func{void}{SetAlpha}{\param{int }{x}, \param{int }{y}, \param{unsigned char }{alpha}}

Sets the alpha value for the given pixel. This function should only be called
if the image has alpha channel data, use \helpref{HasAlpha}{wximagehasalpha} to
check for this.

\membersection{wxImage::SetData}\label{wximagesetdata}

\func{void}{SetData}{\param{unsigned char*}{data}}

Sets the image data without performing checks. The data given must have
the size (width*height*3) or results will be unexpected. Don't use this
method if you aren't sure you know what you are doing.

The data must have been allocated with {\tt malloc()}, {\large {\bf NOT}} with
{\tt operator new}.

After this call the pointer to the data is owned by the wxImage object,
that will be responsible for deleting it.
Do not pass to this function a pointer obtained through
\helpref{wxImage::GetData}{wximagegetdata}.

\membersection{wxImage::SetMask}\label{wximagesetmask}

\func{void}{SetMask}{\param{bool}{ hasMask = true}}

Specifies whether there is a mask or not. The area of the mask is determined by the current mask colour.

\membersection{wxImage::SetMaskColour}\label{wximagesetmaskcolour}

\func{void}{SetMaskColour}{\param{unsigned char }{red}, \param{unsigned char }{green}, \param{unsigned char }{blue}}

Sets the mask colour for this image (and tells the image to use the mask).

\membersection{wxImage::SetMaskFromImage}\label{wximagesetmaskfromimage}

\func{bool}{SetMaskFromImage}{\param{const wxImage\&}{ mask}, \param{unsigned char}{ mr}, \param{unsigned char}{ mg}, \param{unsigned char}{ mb}}

\wxheading{Parameters}

\docparam{mask}{The mask image to extract mask shape from. Must have same dimensions as the image.}

\docparam{mr,mg,mb}{RGB value of pixels in {\it mask} that will be used to create the mask.}

Sets image's mask so that the pixels that have RGB value of {\it mr,mg,mb}
in {\it mask} will be masked in the image. This is done by first finding an
unused colour in the image, setting this colour as the mask colour and then
using this colour to draw all pixels in the image who corresponding pixel 
in {\it mask} has given RGB value.

\wxheading{Return value}

Returns false if {\it mask} does not have same dimensions as the image or if
there is no unused colour left. Returns true if the mask was successfully 
applied.

\wxheading{Notes}

Note that this method involves computing the histogram, which is
computationally intensive operation.

\membersection{wxImage::SetOption}\label{wximagesetoption}

\func{void}{SetOption}{\param{const wxString\&}{ name}, \param{const wxString\&}{ value}}

\func{void}{SetOption}{\param{const wxString\&}{ name}, \param{int}{ value}}

Sets a user-defined option. The function is case-insensitive to {\it name}.

For example, when saving as a JPEG file, the option {\bf quality} is
used, which is a number between 0 and 100 (0 is terrible, 100 is very good).

\wxheading{See also}

\helpref{wxImage::GetOption}{wximagegetoption},\rtfsp
\helpref{wxImage::GetOptionInt}{wximagegetoptionint},\rtfsp
\helpref{wxImage::HasOption}{wximagehasoption}

\membersection{wxImage::SetPalette}\label{wximagesetpalette}

\func{void}{SetPalette}{\param{const wxPalette\&}{ palette}}

Associates a palette with the image. The palette may be used when converting
wxImage to wxBitmap (MSW only at present) or in file save operations (none as yet).

\membersection{wxImage::SetRGB}\label{wximagesetrgb}

\func{void}{SetRGB}{\param{int }{x}, \param{int }{y}, \param{unsigned char }{red}, \param{unsigned char }{green}, \param{unsigned char }{blue}}

Sets the pixel at the given coordinate. This routine performs bounds-checks
for the coordinate so it can be considered a safe way to manipulate the
data, but in some cases this might be too slow so that the data will have to
be set directly. In that case you will have to get access to the image data 
using the \helpref{GetData}{wximagegetdata} method.

\membersection{wxImage::operator $=$}

\func{wxImage\& }{operator $=$}{\param{const wxImage\& }{image}}

Assignment operator. This operator does not copy any data, but instead
passes a pointer to the data in {\it image} and increments a reference
counter. It is a fast operation.

\wxheading{Parameters}

\docparam{image}{Image to assign.}

\wxheading{Return value}

Returns 'this' object.

\membersection{wxImage::operator $==$}

\constfunc{bool}{operator $==$}{\param{const wxImage\& }{image}}

Equality operator. This operator tests whether the internal data pointers are
equal (a fast test).

\wxheading{Parameters}

\docparam{image}{Image to compare with 'this'}

\wxheading{Return value}

Returns true if the images were effectively equal, false otherwise.

\membersection{wxImage::operator $!=$}

\constfunc{bool}{operator $!=$}{\param{const wxImage\& }{image}}

Inequality operator. This operator tests whether the internal data pointers are
unequal (a fast test).

\wxheading{Parameters}

\docparam{image}{Image to compare with 'this'}

\wxheading{Return value}

Returns true if the images were unequal, false otherwise.

\section{\class{wxImageHandler}}\label{wximagehandler}

This is the base class for implementing image file loading/saving, and image creation from data.
It is used within wxImage and is not normally seen by the application.

If you wish to extend the capabilities of wxImage, derive a class from wxImageHandler
and add the handler using \helpref{wxImage::AddHandler}{wximageaddhandler} in your
application initialisation.

\wxheading{Note (Legal Issue)}

This software is based in part on the work of the Independent JPEG Group.

(Applies when wxWidgets is linked with JPEG support. wxJPEGHandler uses libjpeg
created by IJG.)

\wxheading{Derived from}

\helpref{wxObject}{wxobject}

\wxheading{Include files}

<wx/image.h>

\wxheading{See also}

\helpref{wxImage}{wximage}, 
\helpref{wxInitAllImageHandlers}{wxinitallimagehandlers}

\latexignore{\rtfignore{\wxheading{Members}}}

\membersection{wxImageHandler::wxImageHandler}\label{wximagehandlerconstr}

\func{}{wxImageHandler}{\void}

Default constructor. In your own default constructor, initialise the members
m\_name, m\_extension and m\_type.

\membersection{wxImageHandler::\destruct{wxImageHandler}}

\func{}{\destruct{wxImageHandler}}{\void}

Destroys the wxImageHandler object.

\membersection{wxImageHandler::GetName}

\constfunc{wxString}{GetName}{\void}

Gets the name of this handler.

\membersection{wxImageHandler::GetExtension}

\constfunc{wxString}{GetExtension}{\void}

Gets the file extension associated with this handler.

\membersection{wxImageHandler::GetImageCount}\label{wximagehandlergetimagecount}

\func{int}{GetImageCount}{\param{wxInputStream\&}{ stream}}

If the image file contains more than one image and the image handler is capable 
of retrieving these individually, this function will return the number of
available images.

\docparam{stream}{Opened input stream for reading image data. Currently, the stream must support seeking.}

\wxheading{Return value}

Number of available images. For most image handlers, this is 1 (exceptions
are TIFF and ICO formats).

\membersection{wxImageHandler::GetType}

\constfunc{long}{GetType}{\void}

Gets the image type associated with this handler.

\membersection{wxImageHandler::GetMimeType}

\constfunc{wxString}{GetMimeType}{\void}

Gets the MIME type associated with this handler.

\membersection{wxImageHandler::LoadFile}\label{wximagehandlerloadfile}

\func{bool}{LoadFile}{\param{wxImage* }{image}, \param{wxInputStream\&}{ stream}, \param{bool}{ verbose=true}, \param{int}{ index=0}}

Loads a image from a stream, putting the resulting data into {\it image}. If the image file contains
more than one image and the image handler is capable of retrieving these individually, {\it index}
indicates which image to read from the stream.

\wxheading{Parameters}

\docparam{image}{The image object which is to be affected by this operation.}

\docparam{stream}{Opened input stream for reading image data.}

\docparam{verbose}{If set to true, errors reported by the image handler will produce wxLogMessages.}

\docparam{index}{The index of the image in the file (starting from zero).}

\wxheading{Return value}

true if the operation succeeded, false otherwise.

\wxheading{See also}

\helpref{wxImage::LoadFile}{wximageloadfile}, 
\helpref{wxImage::SaveFile}{wximagesavefile}, 
\helpref{wxImageHandler::SaveFile}{wximagehandlersavefile}

\membersection{wxImageHandler::SaveFile}\label{wximagehandlersavefile}

\func{bool}{SaveFile}{\param{wxImage* }{image}, \param{wxOutputStream\& }{stream}}

Saves a image in the output stream.

\wxheading{Parameters}

\docparam{image}{The image object which is to be affected by this operation.}

\docparam{stream}{Opened output stream for writing the data.}

\wxheading{Return value}

true if the operation succeeded, false otherwise.

\wxheading{See also}

\helpref{wxImage::LoadFile}{wximageloadfile}, 
\helpref{wxImage::SaveFile}{wximagesavefile}, 
\helpref{wxImageHandler::LoadFile}{wximagehandlerloadfile}

\membersection{wxImageHandler::SetName}

\func{void}{SetName}{\param{const wxString\& }{name}}

Sets the handler name.

\wxheading{Parameters}

\docparam{name}{Handler name.}

\membersection{wxImageHandler::SetExtension}

\func{void}{SetExtension}{\param{const wxString\& }{extension}}

Sets the handler extension.

\wxheading{Parameters}

\docparam{extension}{Handler extension.}

\membersection{wxImageHandler::SetMimeType}\label{wximagehandlersetmimetype}

\func{void}{SetMimeType}{\param{const wxString\& }{mimetype}}

Sets the handler MIME type.

\wxheading{Parameters}

\docparam{mimename}{Handler MIME type.}

\membersection{wxImageHandler::SetType}

\func{void}{SetType}{\param{long }{type}}

Sets the handler type.

\wxheading{Parameters}

\docparam{name}{Handler type.}

