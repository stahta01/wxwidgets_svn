\section{\class{wxVector<T>}}\label{wxvector}

wxVector<T> is a template class which implements most of the std::vector
class and can be used like it. If wxWidgets is compiled in STL mode,
wxVector will just be a typedef to std::vector. Just like for std::vector,
objects stored in wxVector<T> need to be {\it assignable} but don't have to
be {\it default constructible}.

You can refer to the STL documentation for further information.

\wxheading{Derived from}

No base class

\wxheading{Include files}

<vector.h>

\wxheading{See also}

\helpref{Container classes overview}{wxcontaineroverview}, 
\helpref{wxList<T>}{wxlist},
\helpref{wxArray<T>}{wxarray}

\membersection{wxVector<T>::wxVector<T>}\label{wxvectorwxvector}

\func{}{wxVector<T>}{\void}

\func{}{wxVector<T>}{\param{const wxVector<T>\& }{c}}

Constructor.

\membersection{wxVector<T>::\destruct{wxVector<T>}}\label{wxvectordtor}

\func{}{\destruct{wxVector<T>}}{\void}

Destructor.

\membersection{wxVector<T>::operator=}\label{wxvectoroperatorassign}

\func{wxVector<T>\& operator}{operator=}{\param{const wxVector<T>\& }{vb}}

Assignment operator.

\membersection{wxVector<T>::at}\label{wxvectorat}

\constfunc{const value\_type\&}{at}{\param{size\_type }{idx}}

\func{value\_type\&}{at}{\param{size\_type }{idx}}

Returns item at position {\it idx}.

\membersection{wxVector<T>::back}\label{wxvectorback}

\constfunc{const value\_type\&}{back}{\void}

\func{value\_type\&}{back}{\void}

Return last item.

\membersection{wxVector<T>::begin}\label{wxvectorbegin}

\constfunc{const\_iterator}{begin}{\void}

\func{iterator}{begin}{\void}

Return iterator to beginning of the vector.

\membersection{wxVector<T>::capacity}\label{wxvectorcapacity}

\constfunc{size\_type}{capacity}{\void}


\membersection{wxVector<T>::clear}\label{wxvectorclear}

\func{void}{clear}{\void}

Clears the vector.

\membersection{wxVector<T>::empty}\label{wxvectorempty}

\constfunc{bool}{empty}{\void}

Returns true if the vector is empty.

\membersection{wxVector<T>::end}\label{wxvectorend}

\constfunc{const\_iterator}{end}{\void}

\func{iterator}{end}{\void}

Returns iterator to the end of the vector.

\membersection{wxVector<T>::erase}\label{wxvectorerase}

\func{iterator}{erase}{\param{iterator }{it}}

\func{iterator}{erase}{\param{iterator }{first}, \param{iterator }{last}}

Erase items. When using values other than built-in integrals 
or classes with reference counting this can be an inefficient
operation.

\membersection{wxVector<T>::front}\label{wxvectorfront}

\constfunc{const value\_type\&}{front}{\void}

\func{value\_type\&}{front}{\void}

Returns first item.

\membersection{wxVector<T>::insert}\label{wxvectorinsert}

\func{iterator}{insert}{\param{iterator }{it}, \param{const value\_type\& }{v = value\_type()}}

Insert an item. When using values other than built-in integrals 
or classes with reference counting this can be an inefficient
operation.

\membersection{wxVector<T>::operator[]}\label{wxvectoroperatorunknown}

\constfunc{const value\_type\&}{operator[]}{\param{size\_type }{idx}}

\func{value\_type\&}{operator[]}{\param{size\_type }{idx}}

Returns item at position {\it idx}.

\membersection{wxVector<T>::pop\_back}\label{wxvectorpopback}

\func{void}{pop\_back}{\void}

Removes the last item.

\membersection{wxVector<T>::push\_back}\label{wxvectorpushback}

\func{void}{push\_back}{\param{const value\_type\& }{v}}

Adds an item to the end of the vector.

\membersection{wxVector<T>::reserve}\label{wxvectorreserve}

\func{void}{reserve}{\param{size\_type }{n}}

Reserves more memory of {\it n} is greater then 
\helpref{size}{wxvectorsize}. Other this call has
no effect.

\membersection{wxVector<T>::size}\label{wxvectorsize}

\constfunc{size\_type}{size}{\void}

Returns the size of the vector.
