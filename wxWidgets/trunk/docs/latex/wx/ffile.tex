%%%%%%%%%%%%%%%%%%%%%%%%%%%%%%%%%%%%%%%%%%%%%%%%%%%%%%%%%%%%%%%%%%%%%%%%%%%%%%%
%% Name:        ffile.tex
%% Purpose:     wxFFile documentation
%% Author:      Vadim Zeitlin
%% Modified by:
%% Created:     14.01.02 (extracted from file.tex)
%% RCS-ID:      $Id$
%% Copyright:   (c) Vadim Zeitlin
%% License:     wxWidgets license
%%%%%%%%%%%%%%%%%%%%%%%%%%%%%%%%%%%%%%%%%%%%%%%%%%%%%%%%%%%%%%%%%%%%%%%%%%%%%%%

\section{\class{wxFFile}}\label{wxffile}

wxFFile implements buffered file I/O. This is a very small class designed to
minimize the overhead of using it - in fact, there is hardly any overhead at
all, but using it brings you automatic error checking and hides differences
between platforms and compilers. It wraps inside it a {\tt FILE *} handle used
by standard C IO library (also known as {\tt stdio}).

\wxheading{Derived from}

None.

\wxheading{Include files}

<wx/ffile.h>

\twocolwidtha{7cm}
\begin{twocollist}\itemsep=0pt%
\twocolitem{{\bf wxFromStart}}{Count offset from the start of the file}
\twocolitem{{\bf wxFromCurrent}}{Count offset from the current position of the file pointer}
\twocolitem{{\bf wxFromEnd}}{Count offset from the end of the file (backwards)}
\end{twocollist}

\latexignore{\rtfignore{\wxheading{Members}}}


\membersection{wxFFile::wxFFile}\label{wxffilector}

\func{}{wxFFile}{\void}

Default constructor.

\func{}{wxFFile}{\param{const char*}{ filename}, \param{const char*}{ mode = "r"}}

Opens a file with the given mode. As there is no way to return whether the
operation was successful or not from the constructor you should test the
return value of \helpref{IsOpened}{wxffileisopened} to check that it didn't
fail.

\func{}{wxFFile}{\param{FILE*}{ fp}}

Opens a file with the given file pointer, which has already been opened.

\wxheading{Parameters}

\docparam{filename}{The filename.}

\docparam{mode}{The mode in which to open the file using standard C strings.
Note that you should use {\tt "b"} flag if you use binary files under Windows
or the results might be unexpected due to automatic newline conversion done
for the text files.}

\docparam{fp}{An existing file descriptor, such as stderr.}


\membersection{wxFFile::\destruct{wxFFile}}\label{wxffiledtor}

\func{}{\destruct{wxFFile}}{\void}

Destructor will close the file.

NB: it is not virtual so you should {\it not} derive from wxFFile!


\membersection{wxFFile::Attach}\label{wxffileattach}

\func{void}{Attach}{\param{FILE*}{ fp}}

Attaches an existing file pointer to the wxFFile object.

The descriptor should be already opened and it will be closed by wxFFile
object.


\membersection{wxFFile::Close}\label{wxffileclose}

\func{bool}{Close}{\void}

Closes the file and returns \true on success.


\membersection{wxFFile::Detach}\label{wxffiledetach}

\func{void}{Detach}{\void}

Get back a file pointer from wxFFile object -- the caller is responsible for closing the file if this
descriptor is opened. \helpref{IsOpened()}{wxffileisopened} will return \false after call to Detach().


\membersection{wxFFile::fp}\label{wxffilefp}

\constfunc{FILE *}{fp}{\void}

Returns the file pointer associated with the file.


\membersection{wxFFile::Eof}\label{wxffileeof}

\constfunc{bool}{Eof}{\void}

Returns \true if the an attempt has been made to read {\it past}
the end of the file. 

Note that the behaviour of the file descriptor based class
\helpref{wxFile}{wxfile} is different as \helpref{wxFile::Eof}{wxfileeof}
will return \true here as soon as the last byte of the file has been
read.

Also note that this method may only be called for opened files and may crash if
the file is not opened.

\wxheading{See also}

\helpref{IsOpened}{wxffileisopened}


\membersection{wxFFile::Error}\label{wxffileerror}

Returns \true if an error has occured on this file, similar to the standard
\texttt{ferror()} function.

Please note that this method may only be called for opened files and may crash
if the file is not opened.

\wxheading{See also}

\helpref{IsOpened}{wxffileisopened}


\membersection{wxFFile::Flush}\label{wxffileflush}

\func{bool}{Flush}{\void}

Flushes the file and returns \true on success.


\membersection{wxFFile::GetKind}\label{wxffilegetfilekind}

\constfunc{wxFileKind}{GetKind}{\void}

Returns the type of the file. Possible return values are:

\begin{verbatim}
enum wxFileKind
{
  wxFILE_KIND_UNKNOWN,
  wxFILE_KIND_DISK,     // a file supporting seeking to arbitrary offsets
  wxFILE_KIND_TERMINAL, // a tty
  wxFILE_KIND_PIPE      // a pipe
};

\end{verbatim}


\membersection{wxFFile::IsOpened}\label{wxffileisopened}

\constfunc{bool}{IsOpened}{\void}

Returns \true if the file is opened. Most of the methods of this class may only
be used for an opened file.


\membersection{wxFFile::Length}\label{wxffilelength}

\constfunc{wxFileOffset}{Length}{\void}

Returns the length of the file.


\membersection{wxFFile::Open}\label{wxffileopen}

\func{bool}{Open}{\param{const char*}{ filename}, \param{const char*}{ mode = "r"}}

Opens the file, returning \true if successful.

\wxheading{Parameters}

\docparam{filename}{The filename.}

\docparam{mode}{The mode in which to open the file.}


\membersection{wxFFile::Read}\label{wxffileread}

\func{size\_t}{Read}{\param{void*}{ buffer}, \param{size\_t}{ count}}

Reads the specified number of bytes into a buffer, returning the actual number read.

\wxheading{Parameters}

\docparam{buffer}{A buffer to receive the data.}

\docparam{count}{The number of bytes to read.}

\wxheading{Return value}

The number of bytes read.


\membersection{wxFFile::ReadAll}\label{wxffilereadall}

\func{bool}{ReadAll}{\param{wxString *}{ str}, \param{wxMBConv\&}{ conv = wxConvUTF8}}

Reads the entire contents of the file into a string.

\wxheading{Parameters}

\docparam{str}{String to read data into.}

\docparam{conv}{Conversion object to use in Unicode build; by default supposes
that file contents is encoded in UTF-8.}

\wxheading{Return value}

\true if file was read successfully, \false otherwise.


\membersection{wxFFile::Seek}\label{wxffileseek}

\func{bool}{Seek}{\param{wxFileOffset }{ofs}, \param{wxSeekMode }{mode = wxFromStart}}

Seeks to the specified position and returns \true on success.

\wxheading{Parameters}

\docparam{ofs}{Offset to seek to.}

\docparam{mode}{One of {\bf wxFromStart}, {\bf wxFromEnd}, {\bf wxFromCurrent}.}


\membersection{wxFFile::SeekEnd}\label{wxffileseekend}

\func{bool}{SeekEnd}{\param{wxFileOffset }{ofs = 0}}

Moves the file pointer to the specified number of bytes before the end of the file
and returns \true on success.

\wxheading{Parameters}

\docparam{ofs}{Number of bytes before the end of the file.}


\membersection{wxFFile::Tell}\label{wxffiletell}

\constfunc{wxFileOffset}{Tell}{\void}

Returns the current position.


\membersection{wxFFile::Write}\label{wxffilewrite}

\func{size\_t}{Write}{\param{const void*}{ buffer}, \param{size\_t}{ count}}

Writes the specified number of bytes from a buffer.

\wxheading{Parameters}

\docparam{buffer}{A buffer containing the data.}

\docparam{count}{The number of bytes to write.}

\wxheading{Return value}

Number of bytes written.


\membersection{wxFFile::Write}\label{wxffilewrites}

\func{bool}{Write}{\param{const wxString\& }{s}, \param{wxMBConv\&}{ conv = wxConvUTF8}}

Writes the contents of the string to the file, returns \true on success.

The second argument is only meaningful in Unicode build of wxWidgets when
{\it conv} is used to convert {\it s} to multibyte representation.


