\section{\class{wxProcess}}\label{wxprocess}

The objects of this class are used in conjunction with 
\helpref{wxExecute}{wxexecute} function. When a wxProcess object is passed to
wxExecute(), its \helpref{OnTerminate()}{wxprocessonterminate} virtual method
is called when the process terminates. This allows the program to be
(asynchronously) notified about the process termination and also retrieve its
exit status which is unavailable from wxExecute() in the case of
asynchronous execution.

Please note that if the process termination notification is processed by the
parent, it is responsible for deleting the wxProcess object which sent it.
However, if it is not processed, the object will delete itself and so the
library users should only delete those objects whose notifications have been
processed (and call \helpref{Detach()}{wxprocessdetach} for others).

\wxheading{Derived from}

\helpref{wxEvtHandler}{wxevthandler}

\wxheading{Include files}

<wx/process.h>

\latexignore{\rtfignore{\wxheading{Members}}}

\membersection{wxProcess::wxProcess}\label{wxprocessconstr}

\func{}{wxProcess}{\param{wxEvtHandler *}{ parent = NULL}, \param{int}{ id = -1}}

Constructs a process object. {\it id} is only used in the case you want to
use wxWindows events. It identifies this object, or another window that will
receive the event.

If the {\it parent} parameter is different from NULL, it will receive
a wxEVT\_END\_PROCESS notification event (you should insert EVT\_END\_PROCESS
macro in the event table of the parent to handle it) with the given {\it id}.

\wxheading{Parameters}

\docparam{parent}{The event handler parent.}

\docparam{id}{id of an event.}

\membersection{wxProcess::\destruct{wxProcess}}

\func{}{\destruct{wxProcess}}{\void}

Destroys the wxProcess object.

\membersection{wxProcess::Detach}\label{wxprocessdetach}

\func{void}{Detach}{\void}

Normally, a wxProcess object is deleted by its parent when it receives the
notification about the process termination. However, it might happen that the
parent object is destroyed before the external process is terminated (e.g. a
window from which this external process was launched is closed by the user)
and in this case it {\bf should not delete} the wxProcess object, but 
{\bf should call Detach()} instead. After the wxProcess object is detached
from its parent, no notification events will be sent to the parent and the
object will delete itself upon reception of the process termination
notification.

\membersection{wxProcess::GetInputStream}\label{wxprocessgetinputstream}

\constfunc{wxInputStream* }{GetInputStream}{\void}

It returns a input stream correspoding to the output stream of the subprocess.
If it is NULL, you have not turned on the redirection.
See \helpref{wxProcess::Redirect}{wxprocessredirect}.

\membersection{wxProcess::GetInputStream}\label{wxprocessgetinputstream}

\constfunc{wxInputStream* }{GetInputStream}{\void}

It returns a output stream corresponding to the input stream of the subprocess. 
If it is NULL, you have not turned on the redirection.
See \helpref{wxProcess::Redirect}{wxprocessredirect}.

\membersection{wxProcess::OnTerminate}\label{wxprocessonterminate}

\constfunc{void}{OnTerminate}{\param{int}{ pid}, \param{int}{ status}}

It is called when the process with the pid {\it pid} finishes.
It raises a wxWindows event when it isn't overriden.

\docparam{pid}{The pid of the process which has just terminated.}

\docparam{status}{The exit code of the process.}

\membersection{wxProcess::Redirect}\label{wxprocessredirect}

\func{void}{Redirect}{\void}

It turns on the redirection, wxExecute will try to open a couple of pipes
to catch the subprocess stdio. The caught input stream is returned by
GetOutputStream() as a non-seekable stream. The caught output stream is returned
by GetInputStream() as a non-seekable stream.
