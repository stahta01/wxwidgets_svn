\section{\class{wxProcess}}\label{wxprocess}

This class contains a method which is invoked when a process finishes.
It can raise a \helpref{wxProcessEvent}{wxprocessevent} if wxProcess::OnTerminate
isn't overriden.

\wxheading{Derived from}

\helpref{wxEvtHandler}{wxevthandler}

\wxheading{Include files}

<wx/process.h>

\latexignore{\rtfignore{\wxheading{Members}}}

\membersection{wxProcess::wxProcess}\label{wxprocessconstr}

\func{}{wxProcess}{\param{wxEvtHandler *}{ parent = NULL}, \param{int}{ id = -1}}

Constructs a process object. {\it id} is only used in the case you want to
use wxWindows events. It identifies this object, or another window that will
receive the event.

\wxheading{Parameters}

\docparam{parent}{The event handler parent.}

\docparam{id}{id of an event.}

\membersection{wxProcess::\destruct{wxProcess}}

\func{}{\destruct{wxProcess}}{\void}

Destroys the wxProcess object.

\membersection{wxProcess::OnTerminate}\label{wxprocessonterminate}

\constfunc{void}{OnTerminate}{\param{int}{ pid}}

It is called when the process with the pid {\it pid} finishes.
It raises a wxWindows event when it isn't overriden.

\docparam{pid}{The pid of the process which ends.}

