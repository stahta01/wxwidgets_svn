%%%%%%%%%%%%%%%%%%%%%%%%%%%%%%%%%%%%%%%%%%%%%%%%%%%%%%%%%%%%%%%%%%%%%%%%%%%%%%%
%% Name:        bufferdc.tex
%% Purpose:     wxBufferedDC documentation
%% Author:      Vadim Zeitlin
%% Modified by:
%% Created:     07.02.04
%% RCS-ID:      $Id$
%% Copyright:   (c) 2004 Vadim Zeitlin
%% License:     wxWindows license
%%%%%%%%%%%%%%%%%%%%%%%%%%%%%%%%%%%%%%%%%%%%%%%%%%%%%%%%%%%%%%%%%%%%%%%%%%%%%%%

\section{\class{wxBufferedDC}}\label{wxbuffereddc}

This simple class provides a simple way to avoid flicker: when drawing on it,
everything is in fact first drawn on an in-memory buffer (a 
\helpref{wxBitmap}{wxbitmap}) and then copied to the screen only once, when this
object is destroyed.

It can be used in the same way as any other device context. wxBufferedDC itself
typically replaces \helpref{wxClientDC}{wxclientdc}, if you want to use it in
your \texttt{OnPaint()} handler, you should look at
\helpref{wxBufferedPaintDC}{wxbufferedpaintdc} or \helpref{wxAutoBufferedPaintDC}{wxautobufferedpaintdc}.

Please note that GTK+ 2.0 as well as OS X provide double buffering themselves
natively. wxBufferedDC is aware of this however, and will bypass the buffering
unless explicit buffer bitmap is given.

\wxheading{Derived from}

\helpref{wxDC}{wxdc}\\
\helpref{wxObject}{wxobject}

\wxheading{Include files}

<wx/dcbuffer.h>

\wxheading{See also}

\helpref{wxDC}{wxdc},\rtfsp
\helpref{wxMemoryDC}{wxmemorydc},\rtfsp
\helpref{wxBufferedPaintDC}{wxbufferedpaintdc},\rtfsp
\helpref{wxAutoBufferedPaintDC}{wxautobufferedpaintdc}


\latexignore{\rtfignore{\wxheading{Members}}}

\membersection{wxBufferedDC::wxBufferedDC}\label{wxbuffereddcctor}

\func{}{wxBufferedDC}{\void}

\func{}{wxBufferedDC}{\param{wxDC *}{dc}, \param{const wxBitmap\& }{buffer}, \param{int }{style = wxBUFFER\_CLIENT\_AREA}}

\func{}{wxBufferedDC}{\param{wxWindow*}{window}, \param{wxDC *}{dc}, \param{const wxSize\& }{area}, \param{int }{style = wxBUFFER\_CLIENT\_AREA}}

If you use the first, default, constructor, you must call one of the 
\helpref{Init}{wxbuffereddcinit} methods later in order to use the object.

The other constructors initialize the object immediately and \texttt{Init()} 
must not be called after using them.

\wxheading{Parameters}

\docparam{dc}{The underlying DC: everything drawn to this object will be
flushed to this DC when this object is destroyed.  You may pass NULL
in order to just initialize the buffer, and not flush it.}

\docparam{window}{The window on which the dc paints. May be NULL, but
you should normally specify this so that the DC can be aware whether the
surface is natively double-buffered or not.}

\docparam{area}{The size of the bitmap to be used for buffering (this bitmap is
created internally when it is not given explicitly).}

\docparam{buffer}{Explicitly provided bitmap to be used for buffering: this is
the most efficient solution as the bitmap doesn't have to be recreated each
time but it also requires more memory as the bitmap is never freed. The bitmap
should have appropriate size, anything drawn outside of its bounds is clipped.}

\docparam{style}{wxBUFFER\_CLIENT\_AREA to indicate that just the client area of
the window is buffered, or wxBUFFER\_VIRTUAL\_AREA to indicate that the buffer bitmap
covers the virtual area (in which case PrepareDC is automatically called for the actual window
device context).}

\membersection{wxBufferedDC::Init}\label{wxbuffereddcinit}

\func{void}{Init}{\param{wxDC *}{dc}, \param{const wxBitmap\& }{buffer}, \param{int }{style = wxBUFFER\_CLIENT\_AREA}}

\func{void}{Init}{\param{wxWindow*}{window}, \param{wxDC *}{dc}, \param{const wxSize\& }{area}, \param{int }{style = wxBUFFER\_CLIENT\_AREA}}

These functions initialize the object created using the default constructor.
Please see \helpref{constructors documentation}{wxbuffereddcctor} for details.


% VZ: UnMask() intentionally not documented, we might want to make it private


\membersection{wxBufferedDC::\destruct{wxBufferedDC}}\label{wxbuffereddcdtor}

Copies everything drawn on the DC so far to the underlying DC associated with
this object, if any.


%%%%%%%%%%%%%%%%%%%%%%%%%%%%%%%%%%%%%%%%%%%%%%%%%%%%%%%%%%%%%%%%%%%%%%%%%%%%%%%

\section{\class{wxBufferedPaintDC}}\label{wxbufferedpaintdc}

This is a subclass of \helpref{wxBufferedDC}{wxbuffereddc} which can be used
inside of an \texttt{OnPaint()} event handler. Just create an object of this class instead
of \helpref{wxPaintDC}{wxpaintdc} and make sure \helpref{wxWindow::SetBackgroundStyle}{wxwindowgetbackgroundstyle}
is called with wxBG\_STYLE\_CUSTOM somewhere in the class initialization code, and that's all
you have to do to (mostly) avoid flicker. The only thing to watch out for is that if you are
using this class together with \helpref{wxScrolledWindow}{wxscrolledwindow}, you probably
do \textbf{not} want to call \helpref{PrepareDC}{wxscrolledwindowpreparedc} on it as it
already does this internally for the real underlying wxPaintDC.

\wxheading{Derived from}

\helpref{wxBufferedDC}{wxbuffereddc}\\
\helpref{wxDC}{wxdc}\\
\helpref{wxObject}{wxobject}

\wxheading{Include files}

<wx/dcbuffer.h>

\wxheading{See also}

\helpref{wxDC}{wxdc},\rtfsp
\helpref{wxBufferedDC}{wxbuffereddc},\rtfsp
\helpref{wxAutoBufferedPaintDC}{wxautobufferedpaintdc}


\latexignore{\rtfignore{\wxheading{Members}}}

\membersection{wxBufferedPaintDC::wxBufferedPaintDC}\label{wxbufferedpaintdcctor}

\func{}{wxBufferedPaintDC}{\param{wxWindow *}{window}, \param{const wxBitmap\& }{buffer}, \param{int }{style = wxBUFFER\_CLIENT\_AREA}}

\func{}{wxBufferedPaintDC}{\param{wxWindow *}{window}, \param{int }{style = wxBUFFER\_CLIENT\_AREA}}

As with \helpref{wxBufferedDC}{wxbuffereddcctor}, you may either provide the
bitmap to be used for buffering or let this object create one internally (in
the latter case, the size of the client part of the window is used).

Pass wxBUFFER\_CLIENT\_AREA for the {\it style} parameter to indicate that just the client area of
the window is buffered, or wxBUFFER\_VIRTUAL\_AREA to indicate that the buffer bitmap
covers the virtual area (in which case PrepareDC is automatically called for the actual window
device context).

\membersection{wxBufferedPaintDC::\destruct{wxBufferedPaintDC}}\label{wxbufferedpaintdcdtor}

Copies everything drawn on the DC so far to the window associated with this
object, using a \helpref{wxPaintDC}{wxpaintdc}.


%%%%%%%%%%%%%%%%%%%%%%%%%%%%%%%%%%%%%%%%%%%%%%%%%%%%%%%%%%%%%%%%%%%%%%%%%%%%%%%

\section{\class{wxAutoBufferedPaintDC}}\label{wxautobufferedpaintdc}

This wxDC derivative can be used inside of an \texttt{OnPaint()} event handler to achieve
double-buffered drawing. Just create an object of this class instead of \helpref{wxPaintDC}{wxpaintdc}
and make sure \helpref{wxWindow::SetBackgroundStyle}{wxwindowgetbackgroundstyle} is called
with wxBG\_STYLE\_CUSTOM somewhere in the class initialization code, and that's all you have
to do to (mostly) avoid flicker.

The difference between \helpref{wxBufferedPaintDC}{wxbufferedpaintdc} and this class,
is the lightweigthness - on platforms which have native double-buffering, wxAutoBufferedPaintDC is simply
a typedef of wxPaintDC. Otherwise, it is a typedef of wxBufferedPaintDC.


\wxheading{Derived from}

\helpref{wxBufferedPaintDC}{wxbufferedpaintdc}\\
\helpref{wxPaintDC}{wxpaintdc}\\
\helpref{wxDC}{wxdc}\\
\helpref{wxObject}{wxobject}

\wxheading{Include files}

<wx/dcbuffer.h>

\wxheading{See also}

\helpref{wxDC}{wxdc},\rtfsp
\helpref{wxBufferedPaintDC}{wxbufferedpaintdc}


\latexignore{\rtfignore{\wxheading{Members}}}

\membersection{wxAutoBufferedPaintDC::wxAutoBufferedPaintDC}\label{wxautobufferedpaintdcctor}

\func{}{wxAutoBufferedPaintDC}{\param{wxWindow *}{window}}

Constructor. Pass a pointer to the window on which you wish to paint.

