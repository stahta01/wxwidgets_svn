% ----------------------------------------------------------------------------
% wxDataInputStream
% ----------------------------------------------------------------------------
\section{\class{wxDataInputStream}}\label{wxdatainputstream}

This class provides functions that read data types in a
portable way. So, a file written by an Intel processor can be read by a
Sparc or anything else.

\wxheading{Include files}

<wx/datstrm.h>

\latexignore{\rtfignore{\wxheading{Members}}}

\membersection{wxDataInputStream::wxDataInputStream}\label{wxdatainputstreamconstr}

\func{}{wxDataInputStream}{\param{wxInputStream\&}{ stream}}

Constructs a datastream object from an input stream. Only read methods will
be available.

\wxheading{Parameters}

\docparam{stream}{The input stream.}

\membersection{wxDataInputStream::\destruct{wxDataInputStream}}

\func{}{\destruct{wxDataInputStream}}{\void}

Destroys the wxDataInputStream object.

\membersection{wxDataInputStream::Read8}

\func{unsigned char}{Read8}{\void}

Reads a single byte from the stream.

\membersection{wxDataInputStream::Read16}

\func{unsigned short}{Read16}{\void}

Reads a 16 bit integer from the stream.

\membersection{wxDataInputStream::Read32}

\func{unsigned long}{Read32}{\void}

Reads a 32 bit integer from the stream.

\membersection{wxDataInputStream::ReadDouble}

\func{double}{ReadDouble}{\void}

Reads a double (IEEE encoded) from the stream.

\membersection{wxDataInputStream::ReadString}

\func{wxString}{wxDataInputStream::ReadString}{\void}

Reads a string from a stream. Actually, this function first reads a long integer
specifying the length of the string (without the last null character) and then
reads the string.

% ----------------------------------------------------------------------------
% wxDataOutputStream
% ----------------------------------------------------------------------------

\section{\class{wxDataOutputStream}}\label{wxdataoutputstream}

This class provides functions that write data types in a
portable way. So, a file written by an Intel processor can be read by a
Sparc or anything else.

\latexignore{\rtfignore{\wxheading{Members}}}

\membersection{wxDataOutputStream::wxDataOutputStream}\label{wxdataoutputstreamconstr}

\func{}{wxDataInputStream}{\param{wxOutputStream\&}{ stream}}

Constructs a datastream object from an output stream. Only read methods will
be available.

\wxheading{Parameters}

\docparam{stream}{The output stream.}

\membersection{wxDataOutputStream::\destruct{wxDataOutputStream}}

\func{}{\destruct{wxDataOutputStream}}{\void}

Destroys the wxDataOutputStream object.

\membersection{wxDataOutputStream::Write8}

\func{void}{wxDataOutputStream::Write8}{{\param unsigned char }{i8}}

Writes the single byte {\it i8} to the stream.

\membersection{wxDataOutputStream::Write16}

\func{void}{wxDataOutputStream::Write16}{{\param unsigned short }{i16}}

Writes the 16 bit integer {\it i16} to the stream.

\membersection{wxDataOutputStream::Write32}

\func{void}{wxDataOutputStream::Write32}{{\param unsigned long }{i32}}

Writes the 32 bit integer {\it i32} to the stream.

\membersection{wxDataOutputStream::WriteDouble}

\func{void}{wxDataOutputStream::WriteDouble}{{\param double }{f}}

Writes the double {\it f} to the stream using the IEEE format.

\membersection{wxDataOutputStream::WriteString}

\func{void}{wxDataOutputStream::WriteString}{{\param const wxString\& }{string}}

Writes {\it string} to the stream. Actually, this method writes the size of
the string before writing {\it string} itself.
