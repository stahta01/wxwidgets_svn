\section{wxMSW port}\label{wxmswport}

wxMSW is a port of wxWidgets for the Windows platforms
including Windows 95, 98, ME, 2000, NT, XP in ANSI and
Unicode mode (for Windows 95 through the MSLU extension
library). wxMSW ensures native look and feel for XP
as well when using wxWidgets version 2.3.3 or higher.
wxMSW can be compile with a great variety of compilers
including MS VC++, Borland 5.5, MinGW32, Cygwin and
Watcom as well as cross-compilation with a Linux hosted
MinGW32 tool chain.

For further information, please see the files in docs/msw
in the distribution.

\subsection{wxWinCE}\label{wxwince}

wxWinCE is the name given to wxMSW when compiled on Windows CE devices;
most of wxMSW is common to Win32 and Windows CE but there are
some simplifications, enhancements, and differences in
behaviour.

For building instructions, see docs/msw/wince in the
distribution. The rest of this section documents issues you
need to be aware of when programming for Windows CE devices.

\subsubsection{General issues for wxWinCE programming}

Mobile applications generally have fewer features and
simpler user interfaces. Simply omit whole sizers, static
lines and controls in your dialogs, and use comboboxes instead
of listboxes where appropriate. You also need to reduce
the amount of spacing used by sizers, for which you can
use a macro such as this:

\begin{verbatim}
#if defined(__WXWINCE__)
    #define wxLARGESMALL(large,small) small
#else
    #define wxLARGESMALL(large,small) large
#endif

// Usage
topsizer->Add( CreateTextSizer( message ), 0, wxALL, wxLARGESMALL(10,0) );
\end{verbatim}

There is only ever one instance of a Windows CE application running,
and wxWidgets will take care of showing the current instance and
shutting down the second instance if necessary.

You can test the return value of wxSystemSettings::GetScreenType()
for a qualitative assessment of what kind of display is available,
or use wxGetDisplaySize() if you need more information.

You can also use wxGetOsVersion to test for a version of Windows CE at
run-time (see the next section). However, because different builds
are currently required to target different kinds of device, these
values are hard-wired according to the build, and you cannot
dynamically adapt the same executable for different major Windows CE
platforms. This would require a different approach to the way
wxWidgets adapts its behaviour (such as for menubars) to suit the
style of device.

See the "Life!" example (demos/life) for an example of
an application that has been tailored for PocketPC and Smartphone use.

{\bf Note:} don't forget to have this line in your .rc file, as for
desktop Windows applications:

\begin{verbatim}
#include "wx/msw/wx.rc"
\end{verbatim}

\subsubsection{Testing for WinCE SDKs}

Use these preprocessor symbols to test for the different types of device or SDK:

\begin{twocollist}\itemsep=0pt
\twocolitem{\_\_SMARTPHONE\_\_}{Generic mobile devices with phone buttons and a small display}
\twocolitem{\_\_PDA\_\_}{Generic mobile devices with no phone}
\twocolitem{\_\_HANDHELDPC\_\_}{Generic mobile device with a keyboard}
\twocolitem{\_\_WXWINCE\_\_}{Microsoft-powered Windows CE devices, whether PocketPC, Smartphone or Standard SDK}
\twocolitem{WIN32\_PLATFORM\_WFSP}{Microsoft-powered smartphone}
\twocolitem{\_\_POCKETPC\_\_}{Microsoft-powered PocketPC devices with touch-screen}
\twocolitem{\_\_WINCE\_STANDARDSDK\_\_}{Microsoft-powered Windows CE devices, for generic Windows CE applications}
\twocolitem{\_\_WINCE\_NET\_\_}{Microsoft-powered Windows CE .NET devices (\_WIN32\_WCE is 400 or greater)}
\end{twocollist}

wxGetOsVersion will return these values:

\begin{twocollist}\itemsep=0pt
\twocolitem{wxWINDOWS\_POCKETPC}{The application is running under PocketPC.}
\twocolitem{wxWINDOWS\_SMARTPHONE}{The application is running under Smartphone.}
\twocolitem{wxWINDOWS\_CE}{The application is running under Windows CE (built with the Standard SDK).}
\end{twocollist}

\subsubsection{Window sizing in wxWinCE}

When creating frames and dialogs, create them with wxDefaultPosition and
wxDefaultSize, which will tell WinCE to create them full-screen.

Don't call Fit() and Centre(), so the content sizes to
the window rather than fitting the window to the content. (We really need a single API call
that will do the right thing on each platform.)

If the screen orientation changes, the windows will automatically be resized
so no further action needs to be taken (unless you want to change the layout
according to the orientation, which you could detect in idle time, for example).
However, if the input panel (SIP) is shown, windows do not yet resize accordingly. This will
be implemented soon.

\subsubsection{Closing top-level windows in wxWinCE}

You won't get a wxCloseEvent when the user clicks on the X in the titlebar
on Smartphone and PocketPC; the window is simply hidden instead. However the system may send the
event to force the application to close down.

\subsubsection{Hibernation in wxWinCE}

Smartphone and PocketPC will send a wxEVT\_HIBERNATE to the application object in low
memory conditions. Your application should release memory and close dialogs,
and wake up again when the next wxEVT\_ACTIVATE or wxEVT\_ACTIVATE\_APP message is received.
(wxEVT\_ACTIVATE\_APP is generated whenever a wxEVT\_ACTIVATE event is received
in Smartphone and PocketPC, since these platforms do not support WM\_ACTIVATEAPP.)

\subsubsection{Hardware buttons in wxWinCE}

Special hardware buttons are sent to a window via the wxEVT\_HOTKEY event
under Smartphone and PocketPC. You should first register each required button with \helpref{wxWindow::RegisterHotKey}{wxwindowregisterhotkey},
and unregister the button when you're done with it. For example:

\begin{verbatim}
  win->RegisterHotKey(0, wxMOD_WIN, WXK_SPECIAL1);
  win->UnregisterHotKey(0);
\end{verbatim}

You may have to register the buttons in a wxEVT_ACTIVATE event handler
since other applications will grab the buttons.

There is currently no method of finding out the names of the special
buttons or how many there are.

\subsubsection{Dialogs in wxWinCE}

PocketPC dialogs have an OK button on the caption, and so you should generally
not repeat an OK button on the dialog. You can add a Cancel button if necessary, but some dialogs
simply don't offer you the choice (the guidelines recommend you offer an Undo facility
to make up for it). When the user clicks on the OK button, your dialog will receive
a wxID\_OK event by default. If you wish to change this, call wxDialog::SetAffirmativeId
with the required identifier to be used. Or, override wxDialog::DoOK (return false to
have wxWidgets simply call Close to dismiss the dialog).

Smartphone dialogs do {\it not} have an OK button on the caption, and are closed
using one of the two menu buttons. You need to assign these using wxTopLevelWindow::SetLeftMenu
and wxTopLevelWindow::SetRightMenu, for example:

\begin{verbatim}
#ifdef __SMARTPHONE__
    SetLeftMenu(wxID_OK);
    SetRightMenu(wxID_CANCEL, _("Cancel"));
#elif defined(__POCKETPC__)
    // No OK/Cancel buttons on PocketPC, OK on caption will close
#else
    topsizer->Add( CreateButtonSizer( wxOK|wxCANCEL ), 0, wxEXPAND | wxALL, 10 );
#endif
\end{verbatim}

For implementing property sheets (flat tabs), use a wxNotebook with wxNB_FLAT|wxNB_BOTTOM
and have the notebook left, top and right sides overlap the dialog by about 3 pixels
to eliminate spurious borders. You can do this by using a negative spacing in your
sizer Add() call. The cross-platform property sheet dialog \helpref{wxPropertySheetDialog}{wxpropertysheetdialog} is
provided, to show settings in the correct style on PocketPC and on other platforms.

Notifications (bubble HTML text with optional buttons and links) will also be
implemented in the future for PocketPC.

Modeless dialogs probably don't make sense for PocketPC and Smartphone, since
frames and dialogs are normally full-screen, and a modeless dialog is normally
intended to co-exist with the main application frame.

\subsubsection{Menubars and toolbars in wxWinCE}

\wxheading{Menubars and toolbars in PocketPC}

On PocketPC, a frame must always have a menubar, even if it's empty.
An empty menubar/toolbar is automatically provided for dialogs, to hide
any existing menubar for the duration of the dialog.

Menubars and toolbars are implemented using a combined control,
but you can use essentially the usual wxWidgets API; wxWidgets will combine the menubar
and toolbar. However, there are some restrictions:

\itemsep=0pt
\begin{itemize}
\item You must create the frame's primary toolbar with wxFrame::CreateToolBar,
because this uses the special wxToolMenuBar class (derived from wxToolBar)
to implement the combined toolbar and menubar. Otherwise, you can create and manage toolbars
using the wxToolBar class as usual, for example to implement an optional
formatting toolbar above the menubar as Pocket Word does. But don't assign
a wxToolBar to a frame using SetToolBar - you should always use CreateToolBar
for the main frame toolbar.
\item Deleting and adding tools to wxToolMenuBar after Realize is called is not supported.
\item For speed, colours are not remapped to the system colours as they are
in wxMSW. Provide the tool bitmaps either with the correct system button background,
or with transparency (for example, using XPMs).
\item Adding controls to wxToolMenuBar is not supported. However, wxToolBar supports
controls.
\end{itemize}

Unlike in all other ports, a wxDialog has a wxToolBar, automatically created
for you. You may either leave it blank, or access it with wxDialog::GetToolBar
and add buttons, then calling wxToolBar::Realize. You cannot set or recreate
the toolbar.

\wxheading{Menubars and toolbars in Smartphone}

On Smartphone, there are only two menu buttons, so a menubar is simulated
using a nested menu on the right menu button. Any toolbars are simply ignored on
Smartphone.

\subsubsection{Closing windows in wxWinCE}

The guidelines state that applications should not have a Quit menu item,
since the user should not have to know whether an application is in memory
or not. The close button on a window does not call the window's
close handler; it simply hides the window. However, the guidelines say that
the Ctrl+Q accelerator can be used to quit the application, so wxWidgets
defines this accelerator by default and if your application handles
wxID\_EXIT, it will do the right thing.

\subsubsection{Control differences on wxWinCE}

These controls and styles are specific to wxWinCE:

\itemsep=0pt
\begin{itemize}
\item {\bf wxTextCtrl} The wxTE\_CAPITALIZE style causes a CAPEDIT control to
be created, which capitalizes the first letter.
\end{itemize}

These controls are missing from wxWinCE:

\itemsep=0pt
\begin{itemize}
\item {\bf wxCheckListBox} This can be implemented using a wxListCtrl in report mode
with checked/unchecked images.
\item {\bf MDI classes} MDI is not supported under Windows CE.
\item {\bf wxMiniFrame} Not supported under Windows CE.
\end{itemize}

Tooltips are not currently supported for controls, since on PocketPC controls with
tooltips are distinct controls, and it will be hard to add dynamic
tooltip support.

\subsubsection{Online help in wxWinCE}

You can use the help controller wxWinceHelpController which controls
simple {\tt .htm} files, usually installed in the Windows directory.
See the Windows CE reference for how to format the HTML files.

\subsubsection{Installing your PocketPC and Smartphone applications}

To install your application, you need to build a CAB file using
the parameters defined in a special .inf file. The CabWiz program
in your SDK will compile the CAB file from the .inf file and
files that it specifies.

For delivery, you can simply ask the user to copy the CAB file to the
device and execute the CAB file using File Explorer. Or, you can
write a program for the desktop PC that will find the ActiveSync
Application Manager and install the CAB file on the device,
which is obviously much easier for the user.

Here are some links that may help.

\itemsep=0pt
\begin{itemize}
\item A setup builder that takes CABs and builds a setup program is at \urlref{http://www.eskimo.com/~scottlu/win/index.html}{http://www.eskimo.com/~scottlu/win/index.html}.
\item Sample installation files can be found in {\tt Windows CE Tools/wce420/POCKET PC 2003/Samples/Win32/AppInst}.
\item An installer generator using wxPython can be found at \urlref{http://ppcquicksoft.iespana.es/ppcquicksoft/myinstall.html}{http://ppcquicksoft.iespana.es/ppcquicksoft/myinstall.html}.
\item Miscellaneous Windows CE resources can be found at \urlref{http://www.orbworks.com/pcce/resources.html}{http://www.orbworks.com/pcce/resources.html}.
\item Installer creation instructions with a setup.exe for installing to PPC can be found at \urlref{http://www.pocketpcdn.com/articles/creatingsetup.html}{http://www.pocketpcdn.com/articles/creatingsetup.html}.
\item Microsoft instructions are at \urlref{http://msdn.microsoft.com/library/default.asp?url=/library/en-us/dnce30/html/appinstall30.asp?frame=true&hidetoc=true}{http://msdn.microsoft.com/library/default.asp?url=/library/en-us/dnce30/html/appinstall30.asp?frame=true&hidetoc=true}.
\item Troubleshooting WinCE application installations: \urlref{http://support.microsoft.com/default.aspx?scid=KB;en-us;q181007}{http://support.microsoft.com/default.aspx?scid=KB;en-us;q181007}
\end{itemize}

You may also check out {\tt demos/life/setup/wince} which contains
scripts to create a PocketPC installation for ARM-based
devices. In particular, {\tt build.bat} builds the distribution and
copies it to a directory called {\tt Deliver}.

\subsubsection{wxFileDialog in PocketPC}

Allowing the user to access files on memory cards, or on arbitrary
parts of the filesystem, is a pain; the standard file dialog only
shows folders under My Documents or folders on memory cards
(not the system or card root directory, for example). This is
a known problem for PocketPC developers, and a wxFileDialog
replacement will need to be written.

\subsubsection{Remaining issues}

These are some of the remaining problems to be sorted out, and features
to be supported.

\itemsep=0pt
\begin{itemize}
\item {\bf Font dialog.} The generic font dialog is currently used, which
needs to be simplified (and speeded up).
\item {\bf Sizer speed.} Particularly for dialogs containing notebooks,
layout seems slow. Some analysis is required.
\item {\bf Notification boxes.} The balloon-like notification messages, and their
icons, should be implemented. This will be quite straightforward.
\item {\bf SIP size.} We need to be able to get the area taken up by the SIP (input panel),
and the remaining area, by calling SHSipInfo. We also may need to be able to show and hide
the SIP programmatically, with SHSipPreference. See also the {\it Input Dialogs} topic in
the {\it Programming Windows CE} guide for more on this, and how to have dialogs
show the SIP automatically using the WC\_SIPREF control.
\item {\bf Drawing.} The "Life!" demo shows some droppings being left on the window,
indicating that drawing works a bit differently between desktop and mobile versions of
Win32.
\item {\bf wxStaticBitmap.} The About box in the "Life!" demo shows a bitmap that is
the correct size on the emulator, but too small on a VGA Pocket Loox device.
\item {\bf wxStaticLine.} Lines don't show up, and the documentation suggests that
missing styles are implemented with WM\_PAINT.
\item {\bf wxCheckListBox.} This class needs to be implemented in terms of a wxListCtrl
in report mode, using icons for checkbox states. This is necessary because owner-draw listboxes
are not supported on Windows CE.
\item {\bf wxFileDialog.} A more flexible dialog needs to be written (probably using wxGenericFileDialog)
that can access arbitrary locations.
\item {\bf HTML control.} PocketPC has its own HTML control which can be used for showing
local pages or navigating the web. We should create a version of wxHtmlWindow that uses this
control, or have a separately-named control (wxHtmlCtrl), with a syntax as close as possible to wxHtmlWindow.
\item {\bf Tooltip control.} PocketPC uses special TTBUTTON and TTSTATIC controls for adding
tooltips, with the tooltip separated from the label with a double tilde. We need to support this using SetToolTip.
(Unfortunately it does not seem possible to dynamically remove the tooltip, so an extra style may
be required.)
\item {\bf OK button.} We should allow the OK button on a dialog to be optional, perhaps
by using wxCLOSE\_BOX to indicate when the OK button should be displayed.
\item {\bf Dynamic adaptation.} We should probably be using run-time tests more
than preprocessor tests, so that the same WinCE application can run on different
versions of the operating system.
\item {\bf Modeless dialogs.} When a modeless dialog is hidden with the OK button, it doesn't restore the
frame's menubar. See for example the find dialog in the dialogs sample. However, the menubar is restored
if pressing Cancel (the window is closed). This reflects the fact that modeless dialogs are
not very useful on Windows CE; however, we could perhaps destroy/restore a modeless dialog's menubar
on deactivation and activation.
\item {\bf Home screen plugins.} Figure out how to make home screen plugins for use with wxWidgets
applications (see {\tt http://www.codeproject.com/ce/CTodayWindow.asp} for inspiration).
Although we can't use wxWidgets to create the plugin (too large), we could perhaps write
a generic plugin that takes registry information from a given application, with
options to display information in a particular way using icons and text from
a specified location.
\item {\bf Further abstraction.} We should be able to abstract away more of the differences
between desktop and mobile applications, in particular for sizer layout.
\end{itemize}

