%%%%%%%%%%%%%%%%%%%%%%%%%%%%%%%%%%%%%%%%%%%%%%%%%%%%%%%%%%%%%%%%%%%%%%%%%%%%%%%
%% Name:        window.tex
%% Purpose:     wxWindow documentation
%% Author:      wxWidgets Team
%% Modified by:
%% Created:     
%% RCS-ID:      $Id$
%% Copyright:   (c) wxWidgets Team
%% License:     wxWindows license
%%%%%%%%%%%%%%%%%%%%%%%%%%%%%%%%%%%%%%%%%%%%%%%%%%%%%%%%%%%%%%%%%%%%%%%%%%%%%%%

\section{\class{wxWindow}}\label{wxwindow}

wxWindow is the base class for all windows and represents any visible object on
screen. All controls, top level windows and so on are windows. Sizers and
device contexts are not, however, as they don't appear on screen themselves.

Please note that all children of the window will be deleted automatically by
the destructor before the window itself is deleted which means that you don't
have to worry about deleting them manually. Please see the \helpref{window
deletion overview}{windowdeletionoverview} for more information.

Also note that in this, and many others, wxWidgets classes some
\texttt{GetXXX()} methods may be overloaded (as, for example,
\helpref{GetSize}{wxwindowgetsize} or
\helpref{GetClientSize}{wxwindowgetclientsize}). In this case, the overloads
are non-virtual because having multiple virtual functions with the same name
results in a virtual function name hiding at the derived class level (in
English, this means that the derived class has to override all overloaded
variants if it overrides any of them). To allow overriding them in the derived
class, wxWidgets uses a unique protected virtual \texttt{DoGetXXX()} method
and all \texttt{GetXXX()} ones are forwarded to it, so overriding the former
changes the behaviour of the latter.

\wxheading{Derived from}

\helpref{wxEvtHandler}{wxevthandler}\\
\helpref{wxObject}{wxobject}

\wxheading{Include files}

<wx/window.h>

\wxheading{Window styles}

The following styles can apply to all windows, although they will not always make sense for a particular
window class or on all platforms.

\twocolwidtha{5cm}%
\begin{twocollist}\itemsep=0pt
\twocolitem{\windowstyle{wxSIMPLE\_BORDER}}{Displays a thin border around the window. wxBORDER is the old name
for this style. }
\twocolitem{\windowstyle{wxDOUBLE\_BORDER}}{Displays a double border. Windows and Mac only.}
\twocolitem{\windowstyle{wxSUNKEN\_BORDER}}{Displays a sunken border.}
\twocolitem{\windowstyle{wxRAISED\_BORDER}}{Displays a raised border.}
\twocolitem{\windowstyle{wxSTATIC\_BORDER}}{Displays a border suitable for a static control. Windows only. }
\twocolitem{\windowstyle{wxNO\_BORDER}}{Displays no border, overriding the default border style for the window.}
\twocolitem{\windowstyle{wxTRANSPARENT\_WINDOW}}{The window is transparent, that is, it will not receive paint
events. Windows only.}
\twocolitem{\windowstyle{wxTAB\_TRAVERSAL}}{Use this to enable tab traversal for non-dialog windows.}
\twocolitem{\windowstyle{wxWANTS\_CHARS}}{Use this to indicate that
the window wants to get all char/key events for all keys - even for
keys like TAB or ENTER which are usually used for dialog navigation
and which wouldn't be generated without this style.  If you need to
use this style in order to get the arrows or etc., but would still like to have
normal keyboard navigation take place, you should create and send a
wxNavigationKeyEvent in response to the key events for Tab and
Shift-Tab.}
\twocolitem{\windowstyle{wxNO\_FULL\_REPAINT\_ON\_RESIZE}}{Disables repainting
the window completely when its size is changed - you will have to repaint the
new window area manually if you use this style. Currently only has an effect for
Windows.}
\twocolitem{\windowstyle{wxVSCROLL}}{Use this style to enable a vertical scrollbar.}
\twocolitem{\windowstyle{wxHSCROLL}}{Use this style to enable a horizontal scrollbar.}
\twocolitem{\windowstyle{wxALWAYS\_SHOW\_SB}}{If a window has scrollbars,
disable them instead of hiding them when they are not needed (i.e. when the
size of the window is big enough to not require the scrollbars to navigate it).
This style is currently only implemented for wxMSW and wxUniversal and does
nothing on the other platforms.}
\twocolitem{\windowstyle{wxCLIP\_CHILDREN}}{Use this style to eliminate flicker caused by the background being
repainted, then children being painted over them. Windows only.}
\twocolitem{\windowstyle{wxFULL\_REPAINT\_ON\_RESIZE}}{Use this style to force
a complete redraw of the window whenever it is resized instead of redrawing
just the part of the window affected by resizing. Note that this was the
behaviour by default before 2.5.1 release and that if you experience redraw
problems with the code which previously used to work you may want to try this.}
\end{twocollist}

See also \helpref{window styles overview}{windowstyles}.

\wxheading{Extra window styles}

The following are extra styles, set using \helpref{wxWindow::SetExtraStyle}{wxwindowsetextrastyle}.

\twocolwidtha{5cm}%
\begin{twocollist}\itemsep=0pt
\twocolitem{\windowstyle{wxWS\_EX\_VALIDATE\_RECURSIVELY}}{By default, Validate/TransferDataTo/FromWindow()
only work on direct children of the window (compatible behaviour). Set this flag to make them recursively
descend into all subwindows.}
\twocolitem{\windowstyle{wxWS\_EX\_BLOCK\_EVENTS}}{wxCommandEvents and the objects of the derived classes are forwarded to the
parent window and so on recursively by default. Using this flag for the
given window allows to block this propagation at this window, i.e. prevent
the events from being propagated further upwards. Dialogs have this
flag on by default.}
\twocolitem{\windowstyle{wxWS\_EX\_TRANSIENT}}{Don't use this window as an implicit parent for the other windows: this must
be used with transient windows as otherwise there is the risk of creating a
dialog/frame with this window as a parent which would lead to a crash if the
parent is destroyed before the child.}
\twocolitem{\windowstyle{wxWS\_EX\_PROCESS\_IDLE}}{This window should always process idle events, even
if the mode set by \helpref{wxIdleEvent::SetMode}{wxidleeventsetmode} is wxIDLE\_PROCESS\_SPECIFIED.}
\twocolitem{\windowstyle{wxWS\_EX\_PROCESS\_UI\_UPDATES}}{This window should always process UI update events,
even if the mode set by \helpref{wxUpdateUIEvent::SetMode}{wxupdateuieventsetmode} is wxUPDATE\_UI\_PROCESS\_SPECIFIED.}
\end{twocollist}

\wxheading{See also}

\helpref{Event handling overview}{eventhandlingoverview}

\latexignore{\rtfignore{\wxheading{Members}}}


\membersection{wxWindow::wxWindow}\label{wxwindowctor}

\func{}{wxWindow}{\void}

Default constructor.

\func{}{wxWindow}{\param{wxWindow*}{ parent}, \param{wxWindowID }{id},
 \param{const wxPoint\& }{pos = wxDefaultPosition},
 \param{const wxSize\& }{size = wxDefaultSize},
 \param{long }{style = 0},
 \param{const wxString\& }{name = wxPanelNameStr}}

Constructs a window, which can be a child of a frame, dialog or any other non-control window.

\wxheading{Parameters}

\docparam{parent}{Pointer to a parent window.}

\docparam{id}{Window identifier. If -1, will automatically create an identifier.}

\docparam{pos}{Window position. wxDefaultPosition is (-1, -1) which indicates that wxWidgets
should generate a default position for the window. If using the wxWindow class directly, supply
an actual position.}

\docparam{size}{Window size. wxDefaultSize is (-1, -1) which indicates that wxWidgets
should generate a default size for the window. If no suitable size can  be found, the
window will be sized to 20x20 pixels so that the window is visible but obviously not
correctly sized. }

\docparam{style}{Window style. For generic window styles, please see \helpref{wxWindow}{wxwindow}.}

\docparam{name}{Window name.}


\membersection{wxWindow::\destruct{wxWindow}}\label{wxwindowdtor}

\func{}{\destruct{wxWindow}}{\void}

Destructor. Deletes all subwindows, then deletes itself. Instead of using
the {\bf delete} operator explicitly, you should normally
use \helpref{wxWindow::Destroy}{wxwindowdestroy} so that wxWidgets
can delete a window only when it is safe to do so, in idle time.

\wxheading{See also}

\helpref{Window deletion overview}{windowdeletionoverview},\rtfsp
\helpref{wxWindow::Destroy}{wxwindowdestroy},\rtfsp
\helpref{wxCloseEvent}{wxcloseevent}


\membersection{wxWindow::AddChild}\label{wxwindowaddchild}

\func{virtual void}{AddChild}{\param{wxWindow* }{child}}

Adds a child window.  This is called automatically by window creation
functions so should not be required by the application programmer.

Notice that this function is mostly internal to wxWidgets and shouldn't be
called by the user code.

\wxheading{Parameters}

\docparam{child}{Child window to add.}


\membersection{wxWindow::CacheBestSize}\label{wxwindowcachebestsize}

\constfunc{void}{CacheBestSize}{\param{const wxSize\& }{size}}

Sets the cached best size value.


\membersection{wxWindow::CaptureMouse}\label{wxwindowcapturemouse}

\func{virtual void}{CaptureMouse}{\void}

Directs all mouse input to this window. Call \helpref{wxWindow::ReleaseMouse}{wxwindowreleasemouse} to
release the capture.

Note that wxWidgets maintains the stack of windows having captured the mouse
and when the mouse is released the capture returns to the window which had had
captured it previously and it is only really released if there were no previous
window. In particular, this means that you must release the mouse as many times
as you capture it.

\wxheading{See also}

\helpref{wxWindow::ReleaseMouse}{wxwindowreleasemouse}


\membersection{wxWindow::Center}\label{wxwindowcenter}

\func{void}{Center}{\param{int}{ direction}}

A synonym for \helpref{Centre}{wxwindowcentre}.


\membersection{wxWindow::CenterOnParent}\label{wxwindowcenteronparent}

\func{void}{CenterOnParent}{\param{int}{ direction}}

A synonym for \helpref{CentreOnParent}{wxwindowcentreonparent}.


\membersection{wxWindow::CenterOnScreen}\label{wxwindowcenteronscreen}

\func{void}{CenterOnScreen}{\param{int}{ direction}}

A synonym for \helpref{CentreOnScreen}{wxwindowcentreonscreen}.


\membersection{wxWindow::Centre}\label{wxwindowcentre}

\func{void}{Centre}{\param{int}{ direction = wxBOTH}}

Centres the window.

\wxheading{Parameters}

\docparam{direction}{Specifies the direction for the centering. May be {\tt wxHORIZONTAL}, {\tt wxVERTICAL}\rtfsp
or {\tt wxBOTH}. It may also include {\tt wxCENTRE\_ON\_SCREEN} flag
if you want to center the window on the entire screen and not on its
parent window.}

The flag {\tt wxCENTRE\_FRAME} is obsolete and should not be used any longer
(it has no effect).

\wxheading{Remarks}

If the window is a top level one (i.e. doesn't have a parent), it will be
centered relative to the screen anyhow.

\wxheading{See also}

\helpref{wxWindow::Center}{wxwindowcenter}


\membersection{wxWindow::CentreOnParent}\label{wxwindowcentreonparent}

\func{void}{CentreOnParent}{\param{int}{ direction = wxBOTH}}

Centres the window on its parent. This is a more readable synonym for
\helpref{Centre}{wxwindowcentre}.

\wxheading{Parameters}

\docparam{direction}{Specifies the direction for the centering. May be {\tt wxHORIZONTAL}, {\tt wxVERTICAL}\rtfsp
or {\tt wxBOTH}.}

\wxheading{Remarks}

This methods provides for a way to center top level windows over their
parents instead of the entire screen.  If there is no parent or if the
window is not a top level window, then behaviour is the same as
\helpref{wxWindow::Centre}{wxwindowcentre}.

\wxheading{See also}

\helpref{wxWindow::CentreOnScreen}{wxwindowcenteronscreen}


\membersection{wxWindow::CentreOnScreen}\label{wxwindowcentreonscreen}

\func{void}{CentreOnScreen}{\param{int}{ direction = wxBOTH}}

Centres the window on screen. This only works for top level windows -
otherwise, the window will still be centered on its parent.

\wxheading{Parameters}

\docparam{direction}{Specifies the direction for the centering. May be {\tt wxHORIZONTAL}, {\tt wxVERTICAL}\rtfsp
or {\tt wxBOTH}.}

\wxheading{See also}

\helpref{wxWindow::CentreOnParent}{wxwindowcenteronparent}


\membersection{wxWindow::ClearBackground}\label{wxwindowclearbackground}

\func{void}{ClearBackground}{\void}

Clears the window by filling it with the current background colour. Does not
cause an erase background event to be generated.


\membersection{wxWindow::ClientToScreen}\label{wxwindowclienttoscreen}

\constfunc{virtual void}{ClientToScreen}{\param{int* }{x}, \param{int* }{y}}

\perlnote{In wxPerl this method returns a 2-element list instead of
modifying its parameters.}

\constfunc{virtual wxPoint}{ClientToScreen}{\param{const wxPoint\&}{ pt}}

Converts to screen coordinates from coordinates relative to this window.

\docparam{x}{A pointer to a integer value for the x coordinate. Pass the client coordinate in, and
a screen coordinate will be passed out.}

\docparam{y}{A pointer to a integer value for the y coordinate. Pass the client coordinate in, and
a screen coordinate will be passed out.}

\docparam{pt}{The client position for the second form of the function.}

\pythonnote{In place of a single overloaded method name, wxPython
implements the following methods:\par
\indented{2cm}{\begin{twocollist}
\twocolitem{{\bf ClientToScreen(point)}}{Accepts and returns a wxPoint}
\twocolitem{{\bf ClientToScreenXY(x, y)}}{Returns a 2-tuple, (x, y)}
\end{twocollist}}
}


\membersection{wxWindow::Close}\label{wxwindowclose}

\func{bool}{Close}{\param{bool}{ force = {\tt false}}}

This function simply generates a \helpref{wxCloseEvent}{wxcloseevent} whose
handler usually tries to close the window. It doesn't close the window itself,
however.

\wxheading{Parameters}

\docparam{force}{{\tt false} if the window's close handler should be able to veto the destruction
of this window, {\tt true} if it cannot.}

\wxheading{Remarks}

Close calls the \helpref{close handler}{wxcloseevent} for the window, providing
an opportunity for the window to choose whether to destroy the window.
Usually it is only used with the top level windows (wxFrame and wxDialog
classes) as the others are not supposed to have any special OnClose() logic.

The close handler should check whether the window is being deleted forcibly,
using \helpref{wxCloseEvent::GetForce}{wxcloseeventgetforce}, in which case it
should destroy the window using \helpref{wxWindow::Destroy}{wxwindowdestroy}.

{\it Note} that calling Close does not guarantee that the window will be
destroyed; but it provides a way to simulate a manual close of a window, which
may or may not be implemented by destroying the window. The default
implementation of wxDialog::OnCloseWindow does not necessarily delete the
dialog, since it will simply simulate an wxID\_CANCEL event which is handled by
the appropriate button event handler and may do anything at all.

To guarantee that the window will be destroyed, call
\helpref{wxWindow::Destroy}{wxwindowdestroy} instead

\wxheading{See also}

\helpref{Window deletion overview}{windowdeletionoverview},\rtfsp
\helpref{wxWindow::Destroy}{wxwindowdestroy},\rtfsp
\helpref{wxCloseEvent}{wxcloseevent}


\membersection{wxWindow::ConvertDialogToPixels}\label{wxwindowconvertdialogtopixels}

\func{wxPoint}{ConvertDialogToPixels}{\param{const wxPoint\&}{ pt}}

\func{wxSize}{ConvertDialogToPixels}{\param{const wxSize\&}{ sz}}

Converts a point or size from dialog units to pixels.

For the x dimension, the dialog units are multiplied by the average character width
and then divided by 4.

For the y dimension, the dialog units are multiplied by the average character height
and then divided by 8.

\wxheading{Remarks}

Dialog units are used for maintaining a dialog's proportions even if the font changes.

You can also use these functions programmatically. A convenience macro is defined:

{\small
\begin{verbatim}
#define wxDLG_UNIT(parent, pt) parent->ConvertDialogToPixels(pt)
\end{verbatim}
}

\wxheading{See also}

\helpref{wxWindow::ConvertPixelsToDialog}{wxwindowconvertpixelstodialog}

\pythonnote{In place of a single overloaded method name, wxPython
implements the following methods:\par
\indented{2cm}{\begin{twocollist}
\twocolitem{{\bf ConvertDialogPointToPixels(point)}}{Accepts and returns a wxPoint}
\twocolitem{{\bf ConvertDialogSizeToPixels(size)}}{Accepts and returns a wxSize}
\end{twocollist}}

Additionally, the following helper functions are defined:\par
\indented{2cm}{\begin{twocollist}
\twocolitem{{\bf wxDLG\_PNT(win, point)}}{Converts a wxPoint from dialog
units to pixels}
\twocolitem{{\bf wxDLG\_SZE(win, size)}}{Converts a wxSize from dialog
units to pixels}
\end{twocollist}}
}



\membersection{wxWindow::ConvertPixelsToDialog}\label{wxwindowconvertpixelstodialog}

\func{wxPoint}{ConvertPixelsToDialog}{\param{const wxPoint\&}{ pt}}

\func{wxSize}{ConvertPixelsToDialog}{\param{const wxSize\&}{ sz}}

Converts a point or size from pixels to dialog units.

For the x dimension, the pixels are multiplied by 4 and then divided by the average
character width.

For the y dimension, the pixels are multiplied by 8 and then divided by the average
character height.

\wxheading{Remarks}

Dialog units are used for maintaining a dialog's proportions even if the font changes.

\wxheading{See also}

\helpref{wxWindow::ConvertDialogToPixels}{wxwindowconvertdialogtopixels}

\pythonnote{In place of a single overloaded method name, wxPython implements the following methods:\par
\indented{2cm}{\begin{twocollist}
\twocolitem{{\bf ConvertDialogPointToPixels(point)}}{Accepts and returns a wxPoint}
\twocolitem{{\bf ConvertDialogSizeToPixels(size)}}{Accepts and returns a wxSize}
\end{twocollist}}
}


\membersection{wxWindow::Destroy}\label{wxwindowdestroy}

\func{virtual bool}{Destroy}{\void}

Destroys the window safely. Use this function instead of the delete operator, since
different window classes can be destroyed differently. Frames and dialogs
are not destroyed immediately when this function is called -- they are added
to a list of windows to be deleted on idle time, when all the window's events
have been processed. This prevents problems with events being sent to non-existent
windows.

\wxheading{Return value}

{\tt true} if the window has either been successfully deleted, or it has been added
to the list of windows pending real deletion.


\membersection{wxWindow::DestroyChildren}\label{wxwindowdestroychildren}

\func{virtual void}{DestroyChildren}{\void}

Destroys all children of a window.  Called automatically by the destructor.


\membersection{wxWindow::Disable}\label{wxwindowdisable}

\func{bool}{Disable}{\void}

Disables the window, same as \helpref{Enable({\tt false})}{wxwindowenable}.

\wxheading{Return value}

Returns {\tt true} if the window has been disabled, {\tt false} if it had been
already disabled before the call to this function.


\membersection{wxWindow::DoGetBestSize}\label{wxwindowdogetbestsize}

\constfunc{virtual wxSize}{DoGetBestSize}{\void}

Gets the size which best suits the window: for a control, it would be
the minimal size which doesn't truncate the control, for a panel - the
same size as it would have after a call to \helpref{Fit()}{wxwindowfit}.


\membersection{wxWindow::DoUpdateWindowUI}\label{wxwindowdoupdatewindowui}

\func{virtual void}{DoUpdateWindowUI}{\param{wxUpdateUIEvent\&}{ event}}

Does the window-specific updating after processing the update event.
This function is called by \helpref{wxWindow::UpdateWindowUI}{wxwindowupdatewindowui}
in order to check return values in the \helpref{wxUpdateUIEvent}{wxupdateuievent} and
act appropriately. For example, to allow frame and dialog title updating, wxWidgets
implements this function as follows:

\begin{verbatim}
// do the window-specific processing after processing the update event
void wxTopLevelWindowBase::DoUpdateWindowUI(wxUpdateUIEvent& event)
{
    if ( event.GetSetEnabled() )
        Enable(event.GetEnabled());

    if ( event.GetSetText() )
    {
        if ( event.GetText() != GetTitle() )
            SetTitle(event.GetText());
    }
}
\end{verbatim}



\membersection{wxWindow::DragAcceptFiles}\label{wxwindowdragacceptfiles}

\func{virtual void}{DragAcceptFiles}{\param{bool}{ accept}}

Enables or disables eligibility for drop file events (OnDropFiles).

\wxheading{Parameters}

\docparam{accept}{If {\tt true}, the window is eligible for drop file events. If {\tt false}, the window
will not accept drop file events.}

\wxheading{Remarks}

Windows only.


\membersection{wxWindow::Enable}\label{wxwindowenable}

\func{virtual bool}{Enable}{\param{bool}{ enable = {\tt true}}}

Enable or disable the window for user input. Note that when a parent window is
disabled, all of its children are disabled as well and they are reenabled again
when the parent is.

\wxheading{Parameters}

\docparam{enable}{If {\tt true}, enables the window for input. If {\tt false}, disables the window.}

\wxheading{Return value}

Returns {\tt true} if the window has been enabled or disabled, {\tt false} if
nothing was done, i.e. if the window had already been in the specified state.

\wxheading{See also}

\helpref{wxWindow::IsEnabled}{wxwindowisenabled},\rtfsp
\helpref{wxWindow::Disable}{wxwindowdisable},\rtfsp
\helpref{wxRadioBox::Enable}{wxradioboxenable}


\membersection{wxWindow::FindFocus}\label{wxwindowfindfocus}

\func{static wxWindow*}{FindFocus}{\void}

Finds the window or control which currently has the keyboard focus.

\wxheading{Remarks}

Note that this is a static function, so it can be called without needing a wxWindow pointer.

\wxheading{See also}

\helpref{wxWindow::SetFocus}{wxwindowsetfocus}



\membersection{wxWindow::FindWindow}\label{wxwindowfindwindow}

\constfunc{wxWindow*}{FindWindow}{\param{long}{ id}}

Find a child of this window, by identifier.

\constfunc{wxWindow*}{FindWindow}{\param{const wxString\&}{ name}}

Find a child of this window, by name.

\pythonnote{In place of a single overloaded method name, wxPython
implements the following methods:\par
\indented{2cm}{\begin{twocollist}
\twocolitem{{\bf FindWindowById(id)}}{Accepts an integer}
\twocolitem{{\bf FindWindowByName(name)}}{Accepts a string}
\end{twocollist}}
}


\membersection{wxWindow::FindWindowById}\label{wxwindowfindwindowbyid}

\func{static wxWindow*}{FindWindowById}{\param{long}{ id}, \param{wxWindow*}{ parent = NULL}}

Find the first window with the given {\it id}.

If {\it parent} is NULL, the search will start from all top-level
frames and dialog boxes; if non-NULL, the search will be limited to the given window hierarchy.
The search is recursive in both cases.

\wxheading{See also}

\helpref{FindWindow}{wxwindowfindwindow}


\membersection{wxWindow::FindWindowByName}\label{wxwindowfindwindowbyname}

\func{static wxWindow*}{FindWindowByName}{\param{const wxString\&}{ name}, \param{wxWindow*}{ parent = NULL}}

Find a window by its name (as given in a window constructor or {\bf Create} function call).
If {\it parent} is NULL, the search will start from all top-level
frames and dialog boxes; if non-NULL, the search will be limited to the given window hierarchy.
The search is recursive in both cases.

If no window with such name is found,
\helpref{FindWindowByLabel}{wxwindowfindwindowbylabel} is called.

\wxheading{See also}

\helpref{FindWindow}{wxwindowfindwindow}


\membersection{wxWindow::FindWindowByLabel}\label{wxwindowfindwindowbylabel}

\func{static wxWindow*}{FindWindowByLabel}{\param{const wxString\&}{ label}, \param{wxWindow*}{ parent = NULL}}

Find a window by its label. Depending on the type of window, the label may be a window title
or panel item label. If {\it parent} is NULL, the search will start from all top-level
frames and dialog boxes; if non-NULL, the search will be limited to the given window hierarchy.
The search is recursive in both cases.

\wxheading{See also}

\helpref{FindWindow}{wxwindowfindwindow}


\membersection{wxWindow::Fit}\label{wxwindowfit}

\func{virtual void}{Fit}{\void}

Sizes the window so that it fits around its subwindows. This function won't do
anything if there are no subwindows and will only really work correctly if the
sizers are used for the subwindows layout. Also, if the window has exactly one
subwindow it is better (faster and the result is more precise as Fit adds some
margin to account for fuzziness of its calculations) to call

\begin{verbatim}
    window->SetClientSize(child->GetSize());
\end{verbatim}

instead of calling Fit.


\membersection{wxWindow::FitInside}\label{wxwindowfitinside}

\func{virtual void}{FitInside}{\void}

Similar to \helpref{Fit}{wxwindowfit}, but sizes the interior (virtual) size
of a window.  Mainly useful with scrolled windows to reset scrollbars after
sizing changes that do not trigger a size event, and/or scrolled windows without
an interior sizer.  This function similarly won't do anything if there are no
subwindows.


\membersection{wxWindow::Freeze}\label{wxwindowfreeze}

\func{virtual void}{Freeze}{\void}

Freezes the window or, in other words, prevents any updates from taking place
on screen, the window is not redrawn at all. \helpref{Thaw}{wxwindowthaw} must
be called to reenable window redrawing. Calls to these two functions may be
nested.

This method is useful for visual appearance optimization (for example, it
is a good idea to use it before doing many large text insertions in a row into
a wxTextCtrl under wxGTK) but is not implemented on all platforms nor for all
controls so it is mostly just a hint to wxWidgets and not a mandatory
directive.


\membersection{wxWindow::GetAcceleratorTable}\label{wxwindowgetacceleratortable}

\constfunc{wxAcceleratorTable*}{GetAcceleratorTable}{\void}

Gets the accelerator table for this window. See \helpref{wxAcceleratorTable}{wxacceleratortable}.


\membersection{wxWindow::GetAccessible}\label{wxwindowgetaccessible}

\func{wxAccessible*}{GetAccessible}{\void}

Returns the accessible object for this window, if any.

See also \helpref{wxAccessible}{wxaccessible}.


\membersection{wxWindow::GetAdjustedBestSize}\label{wxwindowgetadjustedbestsize}

\constfunc{wxSize}{GetAdjustedBestSize}{\void}

This method is similar to \helpref{GetBestSize}{wxwindowgetbestsize}, except
in one thing.  GetBestSize should return the minimum untruncated size of the
window, while this method will return the largest of BestSize and any user
specified minimum size.  ie. it is the minimum size the window should currently
be drawn at, not the minimal size it can possibly tolerate.


\membersection{wxWindow::GetBackgroundColour}\label{wxwindowgetbackgroundcolour}

\constfunc{virtual wxColour}{GetBackgroundColour}{\void}

Returns the background colour of the window.

\wxheading{See also}

\helpref{wxWindow::SetBackgroundColour}{wxwindowsetbackgroundcolour},\rtfsp
\helpref{wxWindow::SetForegroundColour}{wxwindowsetforegroundcolour},\rtfsp
\helpref{wxWindow::GetForegroundColour}{wxwindowgetforegroundcolour}

\membersection{wxWindow::GetBackgroundStyle}\label{wxwindowgetbackgroundstyle}

\constfunc{virtual wxBackgroundStyle}{GetBackgroundStyle}{\void}

Returns the background style of the window. The background style indicates
whether background colour should be determined by the system (wxBG\_STYLE\_SYSTEM),
be set to a specific colour (wxBG\_STYLE\_COLOUR), or should be left to the
application to implement (wxBG\_STYLE\_CUSTOM).

On GTK+, use of wxBG\_STYLE\_CUSTOM allows the flicker-free drawing of a custom
background, such as a tiled bitmap. Currently the style has no effect on other platforms.

\wxheading{See also}

\helpref{wxWindow::SetBackgroundColour}{wxwindowsetbackgroundcolour},\rtfsp
\helpref{wxWindow::GetForegroundColour}{wxwindowgetforegroundcolour},\rtfsp
\helpref{wxWindow::SetBackgroundStyle}{wxwindowsetbackgroundstyle}

\membersection{wxWindow::GetBestFittingSize}\label{wxwindowgetbestfittingsize}

\constfunc{wxSize}{GetBestFittingSize}{\void}

Merges the window's best size into the min size and returns the result.

\wxheading{See also}

\helpref{wxWindow::GetBestSize}{wxwindowgetbestsize},\rtfsp
\helpref{wxWindow::SetBestFittingSize}{wxwindowsetbestfittingsize},\rtfsp
\helpref{wxWindow::SetSizeHints}{wxwindowsetsizehints}


\membersection{wxWindow::GetBestSize}\label{wxwindowgetbestsize}

\constfunc{wxSize}{GetBestSize}{\void}

This functions returns the best acceptable minimal size for the window. For
example, for a static control, it will be the minimal size such that the
control label is not truncated. For windows containing subwindows (typically
\helpref{wxPanel}{wxpanel}), the size returned by this function will be the
same as the size the window would have had after calling
\helpref{Fit}{wxwindowfit}.


\membersection{wxWindow::GetCaret}\label{wxwindowgetcaret}

\constfunc{wxCaret *}{GetCaret}{\void}

Returns the \helpref{caret}{wxcaret} associated with the window.


\membersection{wxWindow::GetCapture}\label{wxwindowgetcapture}

\func{static wxWindow *}{GetCapture}{\void}

Returns the currently captured window.

\wxheading{See also}

\helpref{wxWindow::HasCapture}{wxwindowhascapture},
\helpref{wxWindow::CaptureMouse}{wxwindowcapturemouse},
\helpref{wxWindow::ReleaseMouse}{wxwindowreleasemouse},
\helpref{wxMouseCaptureChangedEvent}{wxmousecapturechangedevent}


\membersection{wxWindow::GetCharHeight}\label{wxwindowgetcharheight}

\constfunc{virtual int}{GetCharHeight}{\void}

Returns the character height for this window.


\membersection{wxWindow::GetCharWidth}\label{wxwindowgetcharwidth}

\constfunc{virtual int}{GetCharWidth}{\void}

Returns the average character width for this window.


\membersection{wxWindow::GetChildren}\label{wxwindowgetchildren}

\func{wxList\&}{GetChildren}{\void}

Returns a reference to the list of the window's children.


\membersection{wxWindow::GetClassDefaultAttributes}\label{wxwindowgetclassdefaultattributes}

\func{static wxVisualAttributes}{GetClassDefaultAttributes}{\param{wxWindowVariant}{ variant = \texttt{wxWINDOW\_VARIANT\_NORMAL}}}

Returns the default font and colours which are used by the control. This is
useful if you want to use the same font or colour in your own control as in a
standard control -- which is a much better idea than hard coding specific
colours or fonts which might look completely out of place on the users
system, especially if it uses themes.

The \arg{variant} parameter is only relevant under Mac currently and is
ignore under other platforms. Under Mac, it will change the size of the
returned font. See \helpref{wxWindow::SetWindowVariant}{wxwindowsetwindowvariant}
for more about this.

This static method is ``overridden'' in many derived classes and so calling,
for example, \helpref{wxButton}{wxbutton}::GetClassDefaultAttributes() will typically
return the values appropriate for a button which will be normally different
from those returned by, say, \helpref{wxListCtrl}{wxlistctrl}::GetClassDefaultAttributes().

The \texttt{wxVisualAttributes} structure has at least the fields
\texttt{font}, \texttt{colFg} and \texttt{colBg}. All of them may be invalid
if it was not possible to determine the default control appearance or,
especially for the background colour, if the field doesn't make sense as is
the case for \texttt{colBg} for the controls with themed background.

\wxheading{See also}

\helpref{InheritAttributes}{wxwindowinheritattributes}


\membersection{wxWindow::GetClientSize}\label{wxwindowgetclientsize}

\constfunc{void}{GetClientSize}{\param{int* }{width}, \param{int* }{height}}

\perlnote{In wxPerl this method takes no parameter and returns
a 2-element list {\tt (width, height)}.}

\constfunc{wxSize}{GetClientSize}{\void}

This gets the size of the window `client area' in pixels.
The client area is the area which may be drawn on by the programmer,
excluding title bar, border, scrollbars, etc.

\wxheading{Parameters}

\docparam{width}{Receives the client width in pixels.}

\docparam{height}{Receives the client height in pixels.}

\pythonnote{In place of a single overloaded method name, wxPython
implements the following methods:\par
\indented{2cm}{\begin{twocollist}
\twocolitem{{\bf GetClientSizeTuple()}}{Returns a 2-tuple of (width, height)}
\twocolitem{{\bf GetClientSize()}}{Returns a wxSize object}
\end{twocollist}}
}

\wxheading{See also}

\helpref{GetSize}{wxwindowgetsize},\rtfsp
\helpref{GetVirtualSize}{wxwindowgetvirtualsize}



\membersection{wxWindow::GetConstraints}\label{wxwindowgetconstraints}

\constfunc{wxLayoutConstraints*}{GetConstraints}{\void}

Returns a pointer to the window's layout constraints, or NULL if there are none.


\membersection{wxWindow::GetContainingSizer}\label{wxwindowgetcontainingsizer}

\constfunc{const wxSizer *}{GetContainingSizer}{\void}

Return the sizer that this window is a member of, if any, otherwise
{\tt NULL}.


\membersection{wxWindow::GetCursor}\label{wxwindowgetcursor}

\constfunc{const wxCursor\&}{GetCursor}{\void}

Return the cursor associated with this window.

\wxheading{See also}

\helpref{wxWindow::SetCursor}{wxwindowsetcursor}


\membersection{wxWindow::GetDefaultAttributes}\label{wxwindowgetdefaultattributes}

\constfunc{virtual wxVisualAttributes}{GetDefaultAttributes}{\void}

Currently this is the same as calling
\helpref{GetClassDefaultAttributes}{wxwindowgetclassdefaultattributes}(\helpref{GetWindowVariant}{wxwindowgetwindowvariant}()).

One advantage of using this function compared to the static version is that
the call is automatically dispatched to the correct class (as usual with
virtual functions) and you don't have to specify the class name explicitly.

The other one is that in the future this function could return different
results, for example it might return a different font for an ``Ok'' button
than for a generic button if the users GUI is configured to show such buttons
in bold font. Of course, the down side is that it is impossible to call this
function without actually having an object to apply it to whereas the static
version can be used without having to create an object first.


\membersection{wxWindow::GetDropTarget}\label{wxwindowgetdroptarget}

\constfunc{wxDropTarget*}{GetDropTarget}{\void}

Returns the associated drop target, which may be NULL.

\wxheading{See also}

\helpref{wxWindow::SetDropTarget}{wxwindowsetdroptarget},
\helpref{Drag and drop overview}{wxdndoverview}


\membersection{wxWindow::GetEventHandler}\label{wxwindowgeteventhandler}

\constfunc{wxEvtHandler*}{GetEventHandler}{\void}

Returns the event handler for this window. By default, the window is its
own event handler.

\wxheading{See also}

\helpref{wxWindow::SetEventHandler}{wxwindowseteventhandler},\rtfsp
\helpref{wxWindow::PushEventHandler}{wxwindowpusheventhandler},\rtfsp
\helpref{wxWindow::PopEventHandler}{wxwindowpusheventhandler},\rtfsp
\helpref{wxEvtHandler::ProcessEvent}{wxevthandlerprocessevent},\rtfsp
\helpref{wxEvtHandler}{wxevthandler}\rtfsp


\membersection{wxWindow::GetExtraStyle}\label{wxwindowgetextrastyle}

\constfunc{long}{GetExtraStyle}{\void}

Returns the extra style bits for the window.


\membersection{wxWindow::GetFont}\label{wxwindowgetfont}

\constfunc{wxFont}{GetFont}{\void}

Returns the font for this window.

\wxheading{See also}

\helpref{wxWindow::SetFont}{wxwindowsetfont}


\membersection{wxWindow::GetForegroundColour}\label{wxwindowgetforegroundcolour}

\func{virtual wxColour}{GetForegroundColour}{\void}

Returns the foreground colour of the window.

\wxheading{Remarks}

The interpretation of foreground colour is open to interpretation according
to the window class; it may be the text colour or other colour, or it may not
be used at all.

\wxheading{See also}

\helpref{wxWindow::SetForegroundColour}{wxwindowsetforegroundcolour},\rtfsp
\helpref{wxWindow::SetBackgroundColour}{wxwindowsetbackgroundcolour},\rtfsp
\helpref{wxWindow::GetBackgroundColour}{wxwindowgetbackgroundcolour}


\membersection{wxWindow::GetGrandParent}\label{wxwindowgetgrandparent}

\constfunc{wxWindow*}{GetGrandParent}{\void}

Returns the grandparent of a window, or NULL if there isn't one.


\membersection{wxWindow::GetHandle}\label{wxwindowgethandle}

\constfunc{void*}{GetHandle}{\void}

Returns the platform-specific handle of the physical window. Cast it to an appropriate
handle, such as {\bf HWND} for Windows, {\bf Widget} for Motif, {\bf GtkWidget} for GTK or {\bf WinHandle} for PalmOS.

\pythonnote{This method will return an integer in wxPython.}

\perlnote{This method will return an integer in wxPerl.}


\membersection{wxWindow::GetHelpText}\label{wxwindowgethelptext}

\constfunc{virtual wxString}{GetHelpText}{\void}

Gets the help text to be used as context-sensitive help for this window.

Note that the text is actually stored by the current \helpref{wxHelpProvider}{wxhelpprovider} implementation,
and not in the window object itself.

\wxheading{See also}

\helpref{SetHelpText}{wxwindowsethelptext}, \helpref{wxHelpProvider}{wxhelpprovider}


\membersection{wxWindow::GetId}\label{wxwindowgetid}

\constfunc{int}{GetId}{\void}

Returns the identifier of the window.

\wxheading{Remarks}

Each window has an integer identifier. If the application has not provided one
(or the default Id -1) an unique identifier with a negative value will be generated.

\wxheading{See also}

\helpref{wxWindow::SetId}{wxwindowsetid},\rtfsp
\helpref{Window identifiers}{windowids}


\membersection{wxWindow::GetLabel}\label{wxwindowgetlabel}

\constfunc{virtual wxString }{GetLabel}{\void}

Generic way of getting a label from any window, for
identification purposes.

\wxheading{Remarks}

The interpretation of this function differs from class to class.
For frames and dialogs, the value returned is the title. For buttons or static text controls, it is
the button text. This function can be useful for meta-programs (such as testing
tools or special-needs access programs) which need to identify windows
by name.

\membersection{wxWindow::GetMaxSize}\label{wxwindowgetmaxsize}

\constfunc{wxSize}{GetMaxSize}{\void}

Returns the maximum size of the window, an indication to the sizer layout mechanism
that this is the maximum possible size.

\membersection{wxWindow::GetMinSize}\label{wxwindowgetminsize}

\constfunc{wxSize}{GetMinSize}{\void}

Returns the minimum size of the window, an indication to the sizer layout mechanism
that this is the minimum required size.

\membersection{wxWindow::GetName}\label{wxwindowgetname}

\constfunc{virtual wxString }{GetName}{\void}

Returns the window's name.

\wxheading{Remarks}

This name is not guaranteed to be unique; it is up to the programmer to supply an appropriate
name in the window constructor or via \helpref{wxWindow::SetName}{wxwindowsetname}.

\wxheading{See also}

\helpref{wxWindow::SetName}{wxwindowsetname}


\membersection{wxWindow::GetParent}\label{wxwindowgetparent}

\constfunc{virtual wxWindow*}{GetParent}{\void}

Returns the parent of the window, or NULL if there is no parent.


\membersection{wxWindow::GetPosition}\label{wxwindowgetposition}

\constfunc{virtual void}{GetPosition}{\param{int* }{x}, \param{int* }{y}}

\constfunc{wxPoint}{GetPosition}{\void}

This gets the position of the window in pixels, relative to the parent window
for the child windows or relative to the display origin for the top level
windows.

\wxheading{Parameters}

\docparam{x}{Receives the x position of the window.}

\docparam{y}{Receives the y position of the window.}

\pythonnote{In place of a single overloaded method name, wxPython
implements the following methods:\par
\indented{2cm}{\begin{twocollist}
\twocolitem{{\bf GetPosition()}}{Returns a wxPoint}
\twocolitem{{\bf GetPositionTuple()}}{Returns a tuple (x, y)}
\end{twocollist}}
}

\perlnote{In wxPerl there are two methods instead of a single overloaded
method:\par
\indented{2cm}{\begin{twocollist}
\twocolitem{{\bf GetPosition()}}{Returns a Wx::Point}
\twocolitem{{\bf GetPositionXY()}}{Returns a 2-element list
 {\tt ( x, y )}}
\end{twocollist}
}}


\membersection{wxWindow::GetRect}\label{wxwindowgetrect}

\constfunc{virtual wxRect}{GetRect}{\void}

Returns the size and position of the window as a \helpref{wxRect}{wxrect} object.


\membersection{wxWindow::GetScrollPos}\label{wxwindowgetscrollpos}

\func{virtual int}{GetScrollPos}{\param{int }{orientation}}

Returns the built-in scrollbar position.

\wxheading{See also}

See \helpref{wxWindow::SetScrollbar}{wxwindowsetscrollbar}


\membersection{wxWindow::GetScrollRange}\label{wxwindowgetscrollrange}

\func{virtual int}{GetScrollRange}{\param{int }{orientation}}

Returns the built-in scrollbar range.

\wxheading{See also}

\helpref{wxWindow::SetScrollbar}{wxwindowsetscrollbar}


\membersection{wxWindow::GetScrollThumb}\label{wxwindowgetscrollthumb}

\func{virtual int}{GetScrollThumb}{\param{int }{orientation}}

Returns the built-in scrollbar thumb size.

\wxheading{See also}

\helpref{wxWindow::SetScrollbar}{wxwindowsetscrollbar}


\membersection{wxWindow::GetSize}\label{wxwindowgetsize}

\constfunc{void}{GetSize}{\param{int* }{width}, \param{int* }{height}}

\constfunc{wxSize}{GetSize}{\void}

This gets the size of the entire window in pixels,
including title bar, border, scrollbars, etc.

\wxheading{Parameters}

\docparam{width}{Receives the window width.}

\docparam{height}{Receives the window height.}

\pythonnote{In place of a single overloaded method name, wxPython
implements the following methods:\par
\indented{2cm}{\begin{twocollist}
\twocolitem{{\bf GetSize()}}{Returns a wxSize}
\twocolitem{{\bf GetSizeTuple()}}{Returns a 2-tuple (width, height)}
\end{twocollist}}
}

\perlnote{In wxPerl there are two methods instead of a single overloaded
method:\par
\indented{2cm}{\begin{twocollist}
\twocolitem{{\bf GetSize()}}{Returns a Wx::Size}
\twocolitem{{\bf GetSizeWH()}}{Returns a 2-element list
 {\tt ( width, height )}}
\end{twocollist}
}}

\wxheading{See also}

\helpref{GetClientSize}{wxwindowgetclientsize},\rtfsp
\helpref{GetVirtualSize}{wxwindowgetvirtualsize}


\membersection{wxWindow::GetSizer}\label{wxwindowgetsizer}

\constfunc{wxSizer *}{GetSizer}{\void}

Return the sizer associated with the window by a previous call to
\helpref{SetSizer()}{wxwindowsetsizer} or {\tt NULL}.


\membersection{wxWindow::GetTextExtent}\label{wxwindowgettextextent}

\constfunc{virtual void}{GetTextExtent}{\param{const wxString\& }{string}, \param{int* }{x}, \param{int* }{y},
 \param{int* }{descent = NULL}, \param{int* }{externalLeading = NULL},
 \param{const wxFont* }{font = NULL}, \param{bool}{ use16 = {\tt false}}}

Gets the dimensions of the string as it would be drawn on the
window with the currently selected font.

\wxheading{Parameters}

\docparam{string}{String whose extent is to be measured.}

\docparam{x}{Return value for width.}

\docparam{y}{Return value for height.}

\docparam{descent}{Return value for descent (optional).}

\docparam{externalLeading}{Return value for external leading (optional).}

\docparam{font}{Font to use instead of the current window font (optional).}

\docparam{use16}{If {\tt true}, {\it string} contains 16-bit characters. The default is {\tt false}.}

\pythonnote{In place of a single overloaded method name, wxPython
implements the following methods:\par
\indented{2cm}{\begin{twocollist}
\twocolitem{{\bf GetTextExtent(string)}}{Returns a 2-tuple,  (width, height)}
\twocolitem{{\bf GetFullTextExtent(string, font=NULL)}}{Returns a
4-tuple, (width, height, descent, externalLeading) }
\end{twocollist}}
}

\perlnote{In wxPerl this method takes only the {\tt string} and optionally
 {\tt font} parameters, and returns a 4-element list
 {\tt ( x, y, descent, externalLeading )}.}


\membersection{wxWindow::GetTitle}\label{wxwindowgettitle}

\func{virtual wxString}{GetTitle}{\void}

Gets the window's title. Applicable only to frames and dialogs.

\wxheading{See also}

\helpref{wxWindow::SetTitle}{wxwindowsettitle}


\membersection{wxWindow::GetToolTip}\label{wxwindowgettooltip}

\constfunc{wxToolTip*}{GetToolTip}{\void}

Get the associated tooltip or NULL if none.


\membersection{wxWindow::GetUpdateRegion}\label{wxwindowgetupdateregion}

\constfunc{virtual wxRegion}{GetUpdateRegion}{\void}

Returns the region specifying which parts of the window have been damaged. Should
only be called within an \helpref{wxPaintEvent}{wxpaintevent} handler.

\wxheading{See also}

\helpref{wxRegion}{wxregion},\rtfsp
\helpref{wxRegionIterator}{wxregioniterator}


\membersection{wxWindow::GetValidator}\label{wxwindowgetvalidator}

\constfunc{wxValidator*}{GetValidator}{\void}

Returns a pointer to the current validator for the window, or NULL if there is none.


\membersection{wxWindow::GetVirtualSize}\label{wxwindowgetvirtualsize}

\constfunc{void}{GetVirtualSize}{\param{int* }{width}, \param{int* }{height}}

\constfunc{wxSize}{GetVirtualSize}{\void}

This gets the virtual size of the window in pixels. By default it
returns the client size of the window, but after a call to
\helpref{SetVirtualSize}{wxwindowsetvirtualsize} it will return
that size.

\wxheading{Parameters}

\docparam{width}{Receives the window virtual width.}

\docparam{height}{Receives the window virtual height.}

\helpref{GetSize}{wxwindowgetsize},\rtfsp
\helpref{GetClientSize}{wxwindowgetclientsize}


\membersection{wxWindow::GetWindowStyleFlag}\label{wxwindowgetwindowstyleflag}

\constfunc{long}{GetWindowStyleFlag}{\void}

Gets the window style that was passed to the constructor or {\bf Create}
method. {\bf GetWindowStyle()} is another name for the same function.


\membersection{wxWindow::GetWindowVariant}\label{wxwindowgetwindowvariant}

\constfunc{wxWindowVariant}{GetWindowVariant}{\void}

Returns the value previous passed to
\helpref{wxWindow::SetWindowVariant}{wxwindowsetwindowvariant}.


\membersection{wxWindow::HasCapture}\label{wxwindowhascapture}

\constfunc{virtual bool}{HasCapture}{\void}

Returns {\tt true} if this window has the current mouse capture.

\wxheading{See also}

\helpref{wxWindow::CaptureMouse}{wxwindowcapturemouse},
\helpref{wxWindow::ReleaseMouse}{wxwindowreleasemouse},
\helpref{wxMouseCaptureChangedEvent}{wxmousecapturechangedevent}


\membersection{wxWindow::HasScrollbar}\label{wxwindowhasscrollbar}

\constfunc{virtual bool}{HasScrollbar}{\param{int }{orient}}

Returns {\tt true} if this window has a scroll bar for this orientation.

\wxheading{Parameters}

\docparam{orient}{Orientation to check, either {\tt wxHORIZONTAL} or {\tt wxVERTICAL}.}


\membersection{wxWindow::HasTransparentBackground}\label{wxwindowhastransparentbackground}

\constfunc{virtual bool}{HasTransparentBackground}{\void}

Returns \true if this window background is transparent (as, for example, for
wxStaticText) and should show the parent window background.

This method is mostly used internally by the library itself and you normally
shouldn't have to call it. You may, however, have to override it in your
wxWindow-derived class to ensure that background is painted correctly.


\membersection{wxWindow::Hide}\label{wxwindowhide}

\func{bool}{Hide}{\void}

Equivalent to calling \helpref{Show}{wxwindowshow}({\tt false}).


\membersection{wxWindow::InheritAttributes}\label{wxwindowinheritattributes}

\func{void}{InheritAttributes}{\void}

This function is (or should be, in case of custom controls) called during
window creation to intelligently set up the window visual attributes, that is
the font and the foreground and background colours.

By ``intelligently'' the following is meant: by default, all windows use their
own \helpref{default}{wxwindowgetclassdefaultattributes} attributes. However
if some of the parents attributes are explicitly (that is, using
\helpref{SetFont}{wxwindowsetfont} and not
\helpref{SetOwnFont}{wxwindowsetownfont}) changed \emph{and} if the
corresponding attribute hadn't been explicitly set for this window itself,
then this window takes the same value as used by the parent. In addition, if
the window overrides \helpref{ShouldInheritColours}{wxwindowshouldinheritcolours}
to return \false, the colours will not be changed no matter what and only the
font might.

This rather complicated logic is necessary in order to accommodate the
different usage scenarios. The most common one is when all default attributes
are used and in this case, nothing should be inherited as in modern GUIs
different controls use different fonts (and colours) than their siblings so
they can't inherit the same value from the parent. However it was also deemed
desirable to allow to simply change the attributes of all children at once by
just changing the font or colour of their common parent, hence in this case we
do inherit the parents attributes.


\membersection{wxWindow::InitDialog}\label{wxwindowinitdialog}

\func{void}{InitDialog}{\void}

Sends an {\tt wxEVT\_INIT\_DIALOG} event, whose handler usually transfers data
to the dialog via validators.


\membersection{wxWindow::InvalidateBestSize}\label{wxwindowinvalidatebestsize}

\func{void}{InvalidateBestSize}{\void}

Resets the cached best size value so it will be recalculated the next time it is needed.


\membersection{wxWindow::IsEnabled}\label{wxwindowisenabled}

\constfunc{virtual bool}{IsEnabled}{\void}

Returns {\tt true} if the window is enabled for input, {\tt false} otherwise.

\wxheading{See also}

\helpref{wxWindow::Enable}{wxwindowenable}


\membersection{wxWindow::IsExposed}\label{wxwindowisexposed}

\constfunc{bool}{IsExposed}{\param{int }{x}, \param{int }{y}}

\constfunc{bool}{IsExposed}{\param{wxPoint }{\&pt}}

\constfunc{bool}{IsExposed}{\param{int }{x}, \param{int }{y}, \param{int }{w}, \param{int }{h}}

\constfunc{bool}{IsExposed}{\param{wxRect }{\&rect}}

Returns {\tt true} if the given point or rectangle area has been exposed since the
last repaint. Call this in an paint event handler to optimize redrawing by
only redrawing those areas, which have been exposed.

\pythonnote{In place of a single overloaded method name, wxPython
implements the following methods:\par
\indented{2cm}{\begin{twocollist}
\twocolitem{{\bf IsExposed(x,y, w=0,h=0)}}{}
\twocolitem{{\bf IsExposedPoint(pt)}}{}
\twocolitem{{\bf IsExposedRect(rect)}}{}
\end{twocollist}}}


\membersection{wxWindow::IsRetained}\label{wxwindowisretained}

\constfunc{virtual bool}{IsRetained}{\void}

Returns {\tt true} if the window is retained, {\tt false} otherwise.

\wxheading{Remarks}

Retained windows are only available on X platforms.


\membersection{wxWindow::IsShown}\label{wxwindowisshown}

\constfunc{virtual bool}{IsShown}{\void}

Returns {\tt true} if the window is shown, {\tt false} if it has been hidden.


\membersection{wxWindow::IsTopLevel}\label{wxwindowistoplevel}

\constfunc{bool}{IsTopLevel}{\void}

Returns {\tt true} if the given window is a top-level one. Currently all frames and
dialogs are considered to be top-level windows (even if they have a parent
window).


\membersection{wxWindow::Layout}\label{wxwindowlayout}

\func{void}{Layout}{\void}

Invokes the constraint-based layout algorithm or the sizer-based algorithm
for this window.

See \helpref{wxWindow::SetAutoLayout}{wxwindowsetautolayout}: when auto
layout is on, this function gets called automatically when the window is resized.


\membersection{wxWindow::LineDown}\label{wxwindowlinedown}

This is just a wrapper for \helpref{ScrollLines()}{wxwindowscrolllines}$(1)$.


\membersection{wxWindow::LineUp}\label{wxwindowlineup}

This is just a wrapper for \helpref{ScrollLines()}{wxwindowscrolllines}$(-1)$.


\membersection{wxWindow::Lower}\label{wxwindowlower}

\func{void}{Lower}{\void}

Lowers the window to the bottom of the window hierarchy if it is a managed window (dialog
or frame).


\membersection{wxWindow::MakeModal}\label{wxwindowmakemodal}

\func{virtual void}{MakeModal}{\param{bool }{flag}}

Disables all other windows in the application so that
the user can only interact with this window.

\wxheading{Parameters}

\docparam{flag}{If {\tt true}, this call disables all other windows in the application so that
the user can only interact with this window. If {\tt false}, the effect is reversed.}


\membersection{wxWindow::Move}\label{wxwindowmove}

\func{void}{Move}{\param{int}{ x}, \param{int}{ y}}

\func{void}{Move}{\param{const wxPoint\&}{ pt}}

Moves the window to the given position.

\wxheading{Parameters}

\docparam{x}{Required x position.}

\docparam{y}{Required y position.}

\docparam{pt}{\helpref{wxPoint}{wxpoint} object representing the position.}

\wxheading{Remarks}

Implementations of SetSize can also implicitly implement the
wxWindow::Move function, which is defined in the base wxWindow class
as the call:

\begin{verbatim}
  SetSize(x, y, -1, -1, wxSIZE_USE_EXISTING);
\end{verbatim}

\wxheading{See also}

\helpref{wxWindow::SetSize}{wxwindowsetsize}

\pythonnote{In place of a single overloaded method name, wxPython
implements the following methods:\par
\indented{2cm}{\begin{twocollist}
\twocolitem{{\bf Move(point)}}{Accepts a wxPoint}
\twocolitem{{\bf MoveXY(x, y)}}{Accepts a pair of integers}
\end{twocollist}}
}


\membersection{wxWindow::MoveAfterInTabOrder}\label{wxwindowmoveafterintaborder}

\func{void}{MoveAfterInTabOrder}{\param{wxWindow *}{win}}

Moves this window in the tab navigation order after the specified \arg{win}.
This means that when the user presses \texttt{TAB} key on that other window,
the focus switches to this window.

Default tab order is the same as creation order, this function and
\helpref{MoveBeforeInTabOrder()}{wxwindowmovebeforeintaborder} allow to change
it after creating all the windows.

\wxheading{Parameters}

\docparam{win}{A sibling of this window which should precede it in tab order,
must not be NULL}


\membersection{wxWindow::MoveBeforeInTabOrder}\label{wxwindowmovebeforeintaborder}

\func{void}{MoveBeforeInTabOrder}{\param{wxWindow *}{win}}

Same as \helpref{MoveAfterInTabOrder}{wxwindowmoveafterintaborder} except that
it inserts this window just before \arg{win} instead of putting it right after
it.


\membersection{wxWindow::Navigate}\label{wxwindownavigate}

\func{bool}{Navigate}{\param{int}{ flags = wxNavigationKeyEvent::IsForward}}

Does keyboard navigation from this window to another, by sending
a wxNavigationKeyEvent.

\wxheading{Parameters}

\docparam{flags}{A combination of wxNavigationKeyEvent::IsForward and wxNavigationKeyEvent::WinChange.}

\wxheading{Remarks}

You may wish to call this from a text control custom keypress handler to do the default
navigation behaviour for the tab key, since the standard default behaviour for
a multiline text control with the wxTE\_PROCESS\_TAB style is to insert a tab
and not navigate to the next control.

%% VZ: wxWindow::OnXXX() functions should not be documented but I'm leaving
%%     the old docs here in case we want to move any still needed bits to
%%     the right location (i.e. probably the corresponding events docs)
%%
%% \membersection{wxWindow::OnActivate}\label{wxwindowonactivate}
%%
%% \func{void}{OnActivate}{\param{wxActivateEvent\&}{ event}}
%%
%% Called when a window is activated or deactivated.
%%
%% \wxheading{Parameters}
%%
%% \docparam{event}{Object containing activation information.}
%%
%% \wxheading{Remarks}
%%
%% If the window is being activated, \helpref{wxActivateEvent::GetActive}{wxactivateeventgetactive} returns {\tt true},
%% otherwise it returns {\tt false} (it is being deactivated).
%%
%% \wxheading{See also}
%%
%% \helpref{wxActivateEvent}{wxactivateevent},\rtfsp
%% \helpref{Event handling overview}{eventhandlingoverview}
%%
%% \membersection{wxWindow::OnChar}\label{wxwindowonchar}
%%
%% \func{void}{OnChar}{\param{wxKeyEvent\&}{ event}}
%%
%% Called when the user has pressed a key that is not a modifier (SHIFT, CONTROL or ALT).
%%
%% \wxheading{Parameters}
%%
%% \docparam{event}{Object containing keypress information. See \helpref{wxKeyEvent}{wxkeyevent} for
%% details about this class.}
%%
%% \wxheading{Remarks}
%%
%% This member function is called in response to a keypress. To intercept this event,
%% use the EVT\_CHAR macro in an event table definition. Your {\bf OnChar} handler may call this
%% default function to achieve default keypress functionality.
%%
%% Note that the ASCII values do not have explicit key codes: they are passed as ASCII
%% values.
%%
%% Note that not all keypresses can be intercepted this way. If you wish to intercept modifier
%% keypresses, then you will need to use \helpref{wxWindow::OnKeyDown}{wxwindowonkeydown} or
%% \helpref{wxWindow::OnKeyUp}{wxwindowonkeyup}.
%%
%% Most, but not all, windows allow keypresses to be intercepted.
%%
%% {\bf Tip:} be sure to call {\tt event.Skip()} for events that you don't process in this function,
%% otherwise menu shortcuts may cease to work under Windows.
%%
%% \wxheading{See also}
%%
%% \helpref{wxWindow::OnKeyDown}{wxwindowonkeydown}, \helpref{wxWindow::OnKeyUp}{wxwindowonkeyup},\rtfsp
%% \helpref{wxKeyEvent}{wxkeyevent}, \helpref{wxWindow::OnCharHook}{wxwindowoncharhook},\rtfsp
%% \helpref{Event handling overview}{eventhandlingoverview}
%%
%% \membersection{wxWindow::OnCharHook}\label{wxwindowoncharhook}
%%
%% \func{void}{OnCharHook}{\param{wxKeyEvent\&}{ event}}
%%
%% This member is called to allow the window to intercept keyboard events
%% before they are processed by child windows.
%%
%% \wxheading{Parameters}
%%
%% \docparam{event}{Object containing keypress information. See \helpref{wxKeyEvent}{wxkeyevent} for
%% details about this class.}
%%
%% \wxheading{Remarks}
%%
%% This member function is called in response to a keypress, if the window is active. To intercept this event,
%% use the EVT\_CHAR\_HOOK macro in an event table definition. If you do not process a particular
%% keypress, call \helpref{wxEvent::Skip}{wxeventskip} to allow default processing.
%%
%% An example of using this function is in the implementation of escape-character processing for wxDialog,
%% where pressing ESC dismisses the dialog by {\bf OnCharHook} 'forging' a cancel button press event.
%%
%% Note that the ASCII values do not have explicit key codes: they are passed as ASCII
%% values.
%%
%% This function is only relevant to top-level windows (frames and dialogs), and under
%% Windows only. Under GTK the normal EVT\_CHAR\_ event has the functionality, i.e.
%% you can intercept it, and if you don't call \helpref{wxEvent::Skip}{wxeventskip}
%% the window won't get the event.
%%
%% \wxheading{See also}
%%
%% \helpref{wxKeyEvent}{wxkeyevent},\rtfsp
%% \helpref{wxWindow::OnCharHook}{wxwindowoncharhook},\rtfsp
%% %% GD: OnXXX functions are not documented
%% %%\helpref{wxApp::OnCharHook}{wxapponcharhook},\rtfsp
%% \helpref{Event handling overview}{eventhandlingoverview}
%%
%% \membersection{wxWindow::OnCommand}\label{wxwindowoncommand}
%%
%% \func{virtual void}{OnCommand}{\param{wxEvtHandler\& }{object}, \param{wxCommandEvent\& }{event}}
%%
%% This virtual member function is called if the control does not handle the command event.
%%
%% \wxheading{Parameters}
%%
%% \docparam{object}{Object receiving the command event.}
%%
%% \docparam{event}{Command event}
%%
%% \wxheading{Remarks}
%%
%% This virtual function is provided mainly for backward compatibility. You can also intercept commands
%% from child controls by using an event table, with identifiers or identifier ranges to identify
%% the control(s) in question.
%%
%% \wxheading{See also}
%%
%% \helpref{wxCommandEvent}{wxcommandevent},\rtfsp
%% \helpref{Event handling overview}{eventhandlingoverview}
%%
%% \membersection{wxWindow::OnClose}\label{wxwindowonclose}
%%
%% \func{virtual bool}{OnClose}{\void}
%%
%% Called when the user has tried to close a a frame
%% or dialog box using the window manager (X) or system menu (Windows).
%%
%% {\bf Note:} This is an obsolete function.
%% It is superseded by the \helpref{wxWindow::OnCloseWindow}{wxwindowonclosewindow} event
%% handler.
%%
%% \wxheading{Return value}
%%
%% If {\tt true} is returned by OnClose, the window will be deleted by the system, otherwise the
%% attempt will be ignored. Do not delete the window from within this handler, although
%% you may delete other windows.
%%
%% \wxheading{See also}
%%
%% \helpref{Window deletion overview}{windowdeletionoverview},\rtfsp
%% \helpref{wxWindow::Close}{wxwindowclose},\rtfsp
%% \helpref{wxWindow::OnCloseWindow}{wxwindowonclosewindow},\rtfsp
%% \helpref{wxCloseEvent}{wxcloseevent}
%%
%% \membersection{wxWindow::OnKeyDown}\label{wxwindowonkeydown}
%%
%% \func{void}{OnKeyDown}{\param{wxKeyEvent\&}{ event}}
%%
%% Called when the user has pressed a key, before it is translated into an ASCII value using other
%% modifier keys that might be pressed at the same time.
%%
%% \wxheading{Parameters}
%%
%% \docparam{event}{Object containing keypress information. See \helpref{wxKeyEvent}{wxkeyevent} for
%% details about this class.}
%%
%% \wxheading{Remarks}
%%
%% This member function is called in response to a key down event. To intercept this event,
%% use the EVT\_KEY\_DOWN macro in an event table definition. Your {\bf OnKeyDown} handler may call this
%% default function to achieve default keypress functionality.
%%
%% Note that not all keypresses can be intercepted this way. If you wish to intercept special
%% keys, such as shift, control, and function keys, then you will need to use \helpref{wxWindow::OnKeyDown}{wxwindowonkeydown} or
%% \helpref{wxWindow::OnKeyUp}{wxwindowonkeyup}.
%%
%% Most, but not all, windows allow keypresses to be intercepted.
%%
%% {\bf Tip:} be sure to call {\tt event.Skip()} for events that you don't process in this function,
%% otherwise menu shortcuts may cease to work under Windows.
%%
%% \wxheading{See also}
%%
%% \helpref{wxWindow::OnChar}{wxwindowonchar}, \helpref{wxWindow::OnKeyUp}{wxwindowonkeyup},\rtfsp
%% \helpref{wxKeyEvent}{wxkeyevent}, \helpref{wxWindow::OnCharHook}{wxwindowoncharhook},\rtfsp
%% \helpref{Event handling overview}{eventhandlingoverview}
%%
%% \membersection{wxWindow::OnKeyUp}\label{wxwindowonkeyup}
%%
%% \func{void}{OnKeyUp}{\param{wxKeyEvent\&}{ event}}
%%
%% Called when the user has released a key.
%%
%% \wxheading{Parameters}
%%
%% \docparam{event}{Object containing keypress information. See \helpref{wxKeyEvent}{wxkeyevent} for
%% details about this class.}
%%
%% \wxheading{Remarks}
%%
%% This member function is called in response to a key up event. To intercept this event,
%% use the EVT\_KEY\_UP macro in an event table definition. Your {\bf OnKeyUp} handler may call this
%% default function to achieve default keypress functionality.
%%
%% Note that not all keypresses can be intercepted this way. If you wish to intercept special
%% keys, such as shift, control, and function keys, then you will need to use \helpref{wxWindow::OnKeyDown}{wxwindowonkeydown} or
%% \helpref{wxWindow::OnKeyUp}{wxwindowonkeyup}.
%%
%% Most, but not all, windows allow key up events to be intercepted.
%%
%% \wxheading{See also}
%%
%% \helpref{wxWindow::OnChar}{wxwindowonchar}, \helpref{wxWindow::OnKeyDown}{wxwindowonkeydown},\rtfsp
%% \helpref{wxKeyEvent}{wxkeyevent}, \helpref{wxWindow::OnCharHook}{wxwindowoncharhook},\rtfsp
%% \helpref{Event handling overview}{eventhandlingoverview}
%%
%% \membersection{wxWindow::OnInitDialog}\label{wxwindowoninitdialog}
%%
%% \func{void}{OnInitDialog}{\param{wxInitDialogEvent\&}{ event}}
%%
%% Default handler for the wxEVT\_INIT\_DIALOG event. Calls \helpref{wxWindow::TransferDataToWindow}{wxwindowtransferdatatowindow}.
%%
%% \wxheading{Parameters}
%%
%% \docparam{event}{Dialog initialisation event.}
%%
%% \wxheading{Remarks}
%%
%% Gives the window the default behaviour of transferring data to child controls via
%% the validator that each control has.
%%
%% \wxheading{See also}
%%
%% \helpref{wxValidator}{wxvalidator}, \helpref{wxWindow::TransferDataToWindow}{wxwindowtransferdatatowindow}
%%
%% \membersection{wxWindow::OnMenuCommand}\label{wxwindowonmenucommand}
%%
%% \func{void}{OnMenuCommand}{\param{wxCommandEvent\& }{event}}
%%
%% Called when a menu command is received from a menu bar.
%%
%% \wxheading{Parameters}
%%
%% \docparam{event}{The menu command event. For more information, see \helpref{wxCommandEvent}{wxcommandevent}.}
%%
%% \wxheading{Remarks}
%%
%% A function with this name doesn't actually exist; you can choose any member function to receive
%% menu command events, using the EVT\_COMMAND macro for individual commands or EVT\_COMMAND\_RANGE for
%% a range of commands.
%%
%% \wxheading{See also}
%%
%% \helpref{wxCommandEvent}{wxcommandevent},\rtfsp
%% \helpref{wxWindow::OnMenuHighlight}{wxwindowonmenuhighlight},\rtfsp
%% \helpref{Event handling overview}{eventhandlingoverview}
%%
%% \membersection{wxWindow::OnMenuHighlight}\label{wxwindowonmenuhighlight}
%%
%% \func{void}{OnMenuHighlight}{\param{wxMenuEvent\& }{event}}
%%
%% Called when a menu select is received from a menu bar: that is, the
%% mouse cursor is over a menu item, but the left mouse button has not been
%% pressed.
%%
%% \wxheading{Parameters}
%%
%% \docparam{event}{The menu highlight event. For more information, see \helpref{wxMenuEvent}{wxmenuevent}.}
%%
%% \wxheading{Remarks}
%%
%% You can choose any member function to receive
%% menu select events, using the EVT\_MENU\_HIGHLIGHT macro for individual menu items or EVT\_MENU\_HIGHLIGHT\_ALL macro
%% for all menu items.
%%
%% The default implementation for \helpref{wxFrame::OnMenuHighlight}{wxframeonmenuhighlight} displays help
%% text in the first field of the status bar.
%%
%% This function was known as {\bf OnMenuSelect} in earlier versions of wxWidgets, but this was confusing
%% since a selection is normally a left-click action.
%%
%% \wxheading{See also}
%%
%% \helpref{wxMenuEvent}{wxmenuevent},\rtfsp
%% \helpref{wxWindow::OnMenuCommand}{wxwindowonmenucommand},\rtfsp
%% \helpref{Event handling overview}{eventhandlingoverview}
%%
%%
%% \membersection{wxWindow::OnMouseEvent}\label{wxwindowonmouseevent}
%%
%% \func{void}{OnMouseEvent}{\param{wxMouseEvent\&}{ event}}
%%
%% Called when the user has initiated an event with the
%% mouse.
%%
%% \wxheading{Parameters}
%%
%% \docparam{event}{The mouse event. See \helpref{wxMouseEvent}{wxmouseevent} for
%% more details.}
%%
%% \wxheading{Remarks}
%%
%% Most, but not all, windows respond to this event.
%%
%% To intercept this event, use the EVT\_MOUSE\_EVENTS macro in an event table definition, or individual
%% mouse event macros such as EVT\_LEFT\_DOWN.
%%
%% \wxheading{See also}
%%
%% \helpref{wxMouseEvent}{wxmouseevent},\rtfsp
%% \helpref{Event handling overview}{eventhandlingoverview}
%%
%% \membersection{wxWindow::OnMove}\label{wxwindowonmove}
%%
%% \func{void}{OnMove}{\param{wxMoveEvent\& }{event}}
%%
%% Called when a window is moved.
%%
%% \wxheading{Parameters}
%%
%% \docparam{event}{The move event. For more information, see \helpref{wxMoveEvent}{wxmoveevent}.}
%%
%% \wxheading{Remarks}
%%
%% Use the EVT\_MOVE macro to intercept move events.
%%
%% \wxheading{Remarks}
%%
%% Not currently implemented.
%%
%% \wxheading{See also}
%%
%% \helpref{wxMoveEvent}{wxmoveevent},\rtfsp
%% \helpref{wxFrame::OnSize}{wxframeonsize},\rtfsp
%% \helpref{Event handling overview}{eventhandlingoverview}
%%
%% \membersection{wxWindow::OnPaint}\label{wxwindowonpaint}
%%
%% \func{void}{OnPaint}{\param{wxPaintEvent\& }{event}}
%%
%% Sent to the event handler when the window must be refreshed.
%%
%% \wxheading{Parameters}
%%
%% \docparam{event}{Paint event. For more information, see \helpref{wxPaintEvent}{wxpaintevent}.}
%%
%% \wxheading{Remarks}
%%
%% Use the EVT\_PAINT macro in an event table definition to intercept paint events.
%%
%% Note that In a paint event handler, the application must {\it always} create a \helpref{wxPaintDC}{wxpaintdc} object,
%% even if you do not use it. Otherwise, under MS Windows, refreshing for this and other windows will go wrong.
%%
%% For example:
%%
%% \small{%
%% \begin{verbatim}
%%   void MyWindow::OnPaint(wxPaintEvent\& event)
%%   {
%%       wxPaintDC dc(this);
%%
%%       DrawMyDocument(dc);
%%   }
%% \end{verbatim}
%% }%
%%
%% You can optimize painting by retrieving the rectangles
%% that have been damaged and only repainting these. The rectangles are in
%% terms of the client area, and are unscrolled, so you will need to do
%% some calculations using the current view position to obtain logical,
%% scrolled units.
%%
%% Here is an example of using the \helpref{wxRegionIterator}{wxregioniterator} class:
%%
%% {\small%
%% \begin{verbatim}
%% // Called when window needs to be repainted.
%% void MyWindow::OnPaint(wxPaintEvent\& event)
%% {
%%   wxPaintDC dc(this);
%%
%%   // Find Out where the window is scrolled to
%%   int vbX,vbY;                     // Top left corner of client
%%   GetViewStart(&vbX,&vbY);
%%
%%   int vX,vY,vW,vH;                 // Dimensions of client area in pixels
%%   wxRegionIterator upd(GetUpdateRegion()); // get the update rect list
%%
%%   while (upd)
%%   {
%%     vX = upd.GetX();
%%     vY = upd.GetY();
%%     vW = upd.GetW();
%%     vH = upd.GetH();
%%
%%     // Alternatively we can do this:
%%     // wxRect rect;
%%     // upd.GetRect(&rect);
%%
%%     // Repaint this rectangle
%%     ...some code...
%%
%%     upd ++ ;
%%   }
%% }
%% \end{verbatim}
%% }%
%%
%% \wxheading{See also}
%%
%% \helpref{wxPaintEvent}{wxpaintevent},\rtfsp
%% \helpref{wxPaintDC}{wxpaintdc},\rtfsp
%% \helpref{Event handling overview}{eventhandlingoverview}
%%
%% \membersection{wxWindow::OnScroll}\label{wxwindowonscroll}
%%
%% \func{void}{OnScroll}{\param{wxScrollWinEvent\& }{event}}
%%
%% Called when a scroll window event is received from one of the window's built-in scrollbars.
%%
%% \wxheading{Parameters}
%%
%% \docparam{event}{Command event. Retrieve the new scroll position by
%% calling \helpref{wxScrollEvent::GetPosition}{wxscrolleventgetposition}, and the
%% scrollbar orientation by calling \helpref{wxScrollEvent::GetOrientation}{wxscrolleventgetorientation}.}
%%
%% \wxheading{Remarks}
%%
%% Note that it is not possible to distinguish between horizontal and vertical scrollbars
%% until the function is executing (you can't have one function for vertical, another
%% for horizontal events).
%%
%% \wxheading{See also}
%%
%% \helpref{wxScrollWinEvent}{wxscrollwinevent},\rtfsp
%% \helpref{Event handling overview}{eventhandlingoverview}
%%
%% \membersection{wxWindow::OnSetFocus}\label{wxwindowonsetfocus}
%%
%% \func{void}{OnSetFocus}{\param{wxFocusEvent\& }{event}}
%%
%% Called when a window's focus is being set.
%%
%% \wxheading{Parameters}
%%
%% \docparam{event}{The focus event. For more information, see \helpref{wxFocusEvent}{wxfocusevent}.}
%%
%% \wxheading{Remarks}
%%
%% To intercept this event, use the macro EVT\_SET\_FOCUS in an event table definition.
%%
%% Most, but not all, windows respond to this event.
%%
%% \wxheading{See also}
%%
%% \helpref{wxFocusEvent}{wxfocusevent}, \helpref{wxWindow::OnKillFocus}{wxwindowonkillfocus},\rtfsp
%% \helpref{Event handling overview}{eventhandlingoverview}
%%
%% \membersection{wxWindow::OnSize}\label{wxwindowonsize}
%%
%% \func{void}{OnSize}{\param{wxSizeEvent\& }{event}}
%%
%% Called when the window has been resized. This is not a virtual function; you should
%% provide your own non-virtual OnSize function and direct size events to it using EVT\_SIZE
%% in an event table definition.
%%
%% \wxheading{Parameters}
%%
%% \docparam{event}{Size event. For more information, see \helpref{wxSizeEvent}{wxsizeevent}.}
%%
%% \wxheading{Remarks}
%%
%% You may wish to use this for frames to resize their child windows as appropriate.
%%
%% Note that the size passed is of
%% the whole window: call \helpref{wxWindow::GetClientSize}{wxwindowgetclientsize} for the area which may be
%% used by the application.
%%
%% When a window is resized, usually only a small part of the window is damaged and you
%% may only need to repaint that area. However, if your drawing depends on the size of the window,
%% you may need to clear the DC explicitly and repaint the whole window. In which case, you
%% may need to call \helpref{wxWindow::Refresh}{wxwindowrefresh} to invalidate the entire window.
%%
%% \wxheading{See also}
%%
%% \helpref{wxSizeEvent}{wxsizeevent},\rtfsp
%% \helpref{Event handling overview}{eventhandlingoverview}
%%
%% \membersection{wxWindow::OnSysColourChanged}\label{wxwindowonsyscolourchanged}
%%
%% \func{void}{OnSysColourChanged}{\param{wxOnSysColourChangedEvent\& }{event}}
%%
%% Called when the user has changed the system colours. Windows only.
%%
%% \wxheading{Parameters}
%%
%% \docparam{event}{System colour change event. For more information, see \helpref{wxSysColourChangedEvent}{wxsyscolourchangedevent}.}
%%
%% \wxheading{See also}
%%
%% \helpref{wxSysColourChangedEvent}{wxsyscolourchangedevent},\rtfsp
%% \helpref{Event handling overview}{eventhandlingoverview}


\membersection{wxWindow::OnInternalIdle}\label{wxwindowoninternalidle}

\func{virtual void}{OnInternalIdle}{\void}

This virtual function is normally only used internally, but
sometimes an application may need it to implement functionality
that should not be disabled by an application defining an OnIdle
handler in a derived class.

This function may be used to do delayed painting, for example,
and most implementations call \helpref{wxWindow::UpdateWindowUI}{wxwindowupdatewindowui}
in order to send update events to the window in idle time.


\membersection{wxWindow::PageDown}\label{wxwindowpagedown}

This is just a wrapper for \helpref{ScrollPages()}{wxwindowscrollpages}$(1)$.


\membersection{wxWindow::PageUp}\label{wxwindowpageup}

This is just a wrapper for \helpref{ScrollPages()}{wxwindowscrollpages}$(-1)$.


\membersection{wxWindow::PopEventHandler}\label{wxwindowpopeventhandler}

\constfunc{wxEvtHandler*}{PopEventHandler}{\param{bool }{deleteHandler = {\tt false}}}

Removes and returns the top-most event handler on the event handler stack.

\wxheading{Parameters}

\docparam{deleteHandler}{If this is {\tt true}, the handler will be deleted after it is removed. The
default value is {\tt false}.}

\wxheading{See also}

\helpref{wxWindow::SetEventHandler}{wxwindowseteventhandler},\rtfsp
\helpref{wxWindow::GetEventHandler}{wxwindowgeteventhandler},\rtfsp
\helpref{wxWindow::PushEventHandler}{wxwindowpusheventhandler},\rtfsp
\helpref{wxEvtHandler::ProcessEvent}{wxevthandlerprocessevent},\rtfsp
\helpref{wxEvtHandler}{wxevthandler}\rtfsp


\membersection{wxWindow::PopupMenu}\label{wxwindowpopupmenu}

\func{bool}{PopupMenu}{\param{wxMenu* }{menu}, \param{const wxPoint\& }{pos = wxDefaultPosition}}

\func{bool}{PopupMenu}{\param{wxMenu* }{menu}, \param{int }{x}, \param{int }{y}}

Pops up the given menu at the specified coordinates, relative to this
window, and returns control when the user has dismissed the menu. If a
menu item is selected, the corresponding menu event is generated and will be
processed as usually. If the coordinates are not specified, current mouse
cursor position is used.

\wxheading{Parameters}

\docparam{menu}{Menu to pop up.}

\docparam{pos}{The position where the menu will appear.}

\docparam{x}{Required x position for the menu to appear.}

\docparam{y}{Required y position for the menu to appear.}

\wxheading{See also}

\helpref{wxMenu}{wxmenu}

\wxheading{Remarks}

Just before the menu is popped up, \helpref{wxMenu::UpdateUI}{wxmenuupdateui}
is called to ensure that the menu items are in the correct state. The menu does
not get deleted by the window.

It is recommended to not explicitly specify coordinates when calling PopupMenu
in response to mouse click, because some of the ports (namely, wxGTK) can do
a better job of positioning the menu in that case.

\pythonnote{In place of a single overloaded method name, wxPython
implements the following methods:\par
\indented{2cm}{\begin{twocollist}
\twocolitem{{\bf PopupMenu(menu, point)}}{Specifies position with a wxPoint}
\twocolitem{{\bf PopupMenuXY(menu, x, y)}}{Specifies position with two integers (x, y)}
\end{twocollist}}
}


\membersection{wxWindow::PushEventHandler}\label{wxwindowpusheventhandler}

\func{void}{PushEventHandler}{\param{wxEvtHandler* }{handler}}

Pushes this event handler onto the event stack for the window.

\wxheading{Parameters}

\docparam{handler}{Specifies the handler to be pushed.}

\wxheading{Remarks}

An event handler is an object that is capable of processing the events
sent to a window. By default, the window is its own event handler, but
an application may wish to substitute another, for example to allow
central implementation of event-handling for a variety of different
window classes.

\helpref{wxWindow::PushEventHandler}{wxwindowpusheventhandler} allows
an application to set up a chain of event handlers, where an event not handled by one event handler is
handed to the next one in the chain. Use \helpref{wxWindow::PopEventHandler}{wxwindowpopeventhandler} to
remove the event handler.

\wxheading{See also}

\helpref{wxWindow::SetEventHandler}{wxwindowseteventhandler},\rtfsp
\helpref{wxWindow::GetEventHandler}{wxwindowgeteventhandler},\rtfsp
\helpref{wxWindow::PopEventHandler}{wxwindowpusheventhandler},\rtfsp
\helpref{wxEvtHandler::ProcessEvent}{wxevthandlerprocessevent},\rtfsp
\helpref{wxEvtHandler}{wxevthandler}


\membersection{wxWindow::Raise}\label{wxwindowraise}

\func{void}{Raise}{\void}

Raises the window to the top of the window hierarchy if it is a managed window (dialog
or frame).


\membersection{wxWindow::Refresh}\label{wxwindowrefresh}

\func{virtual void}{Refresh}{\param{bool}{ eraseBackground = {\tt true}}, \param{const wxRect* }{rect
= NULL}}

Causes an event to be generated to repaint the
window.

\wxheading{Parameters}

\docparam{eraseBackground}{If {\tt true}, the background will be
erased.}

\docparam{rect}{If non-NULL, only the given rectangle will
be treated as damaged.}

\wxheading{See also}

\helpref{wxWindow::RefreshRect}{wxwindowrefreshrect}


\membersection{wxWindow::RefreshRect}\label{wxwindowrefreshrect}

\func{void}{RefreshRect}{\param{const wxRect\& }{rect}, \param{bool }{eraseBackground = \true}}

Redraws the contents of the given rectangle: only the area inside it will be
repainted.

This is the same as \helpref{Refresh}{wxwindowrefresh} but has a nicer syntax
as it can be called with a temporary wxRect object as argument like this
\texttt{RefreshRect(wxRect(x, y, w, h))}.


\membersection{wxWindow::RegisterHotKey}\label{wxwindowregisterhotkey}

\func{bool}{RegisterHotKey}{\param{int}{ hotkeyId}, \param{int}{ modifiers}, \param{int}{ virtualKeyCode}}

Registers a system wide hotkey. Every time the user presses the hotkey registered here, this window
will receive a hotkey event. It will receive the event even if the application is in the background
and does not have the input focus because the user is working with some other application.

\wxheading{Parameters}

\docparam{hotkeyId}{Numeric identifier of the hotkey. For applications this must be between 0 and 0xBFFF. If
this function is called from a shared DLL, it must be a system wide unique identifier between 0xC000 and 0xFFFF.
This is a MSW specific detail.}

\docparam{modifiers}{A bitwise combination of {\tt wxMOD\_SHIFT}, {\tt wxMOD\_CONTROL}, {\tt wxMOD\_ALT}
or {\tt wxMOD\_WIN} specifying the modifier keys that have to be pressed along with the key.}

\docparam{virtualKeyCode}{The virtual key code of the hotkey.}

\wxheading{Return value}

{\tt true} if the hotkey was registered successfully. {\tt false} if some other application already registered a
hotkey with this modifier/virtualKeyCode combination.

\wxheading{Remarks}

Use EVT\_HOTKEY(hotkeyId, fnc) in the event table to capture the event.
This function is currently only implemented under MSW.

\wxheading{See also}

\helpref{wxWindow::UnregisterHotKey}{wxwindowunregisterhotkey}


\membersection{wxWindow::ReleaseMouse}\label{wxwindowreleasemouse}

\func{virtual void}{ReleaseMouse}{\void}

Releases mouse input captured with \helpref{wxWindow::CaptureMouse}{wxwindowcapturemouse}.

\wxheading{See also}

\helpref{wxWindow::CaptureMouse}{wxwindowcapturemouse},
\helpref{wxWindow::HasCapture}{wxwindowhascapture},
\helpref{wxWindow::ReleaseMouse}{wxwindowreleasemouse},
\helpref{wxMouseCaptureChangedEvent}{wxmousecapturechangedevent}


\membersection{wxWindow::RemoveChild}\label{wxwindowremovechild}

\func{virtual void}{RemoveChild}{\param{wxWindow* }{child}}

Removes a child window.  This is called automatically by window deletion
functions so should not be required by the application programmer.

Notice that this function is mostly internal to wxWidgets and shouldn't be
called by the user code.

\wxheading{Parameters}

\docparam{child}{Child window to remove.}


\membersection{wxWindow::RemoveEventHandler}\label{wxwindowremoveeventhandler}

\func{bool}{RemoveEventHandler}{\param{wxEvtHandler *}{handler}}

Find the given {\it handler} in the windows event handler chain and remove (but
not delete) it from it.

\wxheading{Parameters}

\docparam{handler}{The event handler to remove, must be non {\tt NULL} and
must be present in this windows event handlers chain}

\wxheading{Return value}

Returns {\tt true} if it was found and {\tt false} otherwise (this also results
in an assert failure so this function should only be called when the
handler is supposed to be there).

\wxheading{See also}

\helpref{PushEventHandler}{wxwindowpusheventhandler},\rtfsp
\helpref{PopEventHandler}{wxwindowpopeventhandler}


\membersection{wxWindow::Reparent}\label{wxwindowreparent}

\func{virtual bool}{Reparent}{\param{wxWindow* }{newParent}}

Reparents the window, i.e the window will be removed from its
current parent window (e.g. a non-standard toolbar in a wxFrame)
and then re-inserted into another. Available on Windows and GTK.

\wxheading{Parameters}

\docparam{newParent}{New parent.}


\membersection{wxWindow::ScreenToClient}\label{wxwindowscreentoclient}

\constfunc{virtual void}{ScreenToClient}{\param{int* }{x}, \param{int* }{y}}

\constfunc{virtual wxPoint}{ScreenToClient}{\param{const wxPoint\& }{pt}}

Converts from screen to client window coordinates.

\wxheading{Parameters}

\docparam{x}{Stores the screen x coordinate and receives the client x coordinate.}

\docparam{y}{Stores the screen x coordinate and receives the client x coordinate.}

\docparam{pt}{The screen position for the second form of the function.}

\pythonnote{In place of a single overloaded method name, wxPython
implements the following methods:\par
\indented{2cm}{\begin{twocollist}
\twocolitem{{\bf ScreenToClient(point)}}{Accepts and returns a wxPoint}
\twocolitem{{\bf ScreenToClientXY(x, y)}}{Returns a 2-tuple, (x, y)}
\end{twocollist}}
}


\membersection{wxWindow::ScrollLines}\label{wxwindowscrolllines}

\func{virtual bool}{ScrollLines}{\param{int }{lines}}

Scrolls the window by the given number of lines down (if {\it lines} is
positive) or up.

\wxheading{Return value}

Returns {\tt true} if the window was scrolled, {\tt false} if it was already
on top/bottom and nothing was done.

\wxheading{Remarks}

This function is currently only implemented under MSW and wxTextCtrl under
wxGTK (it also works for wxScrolledWindow derived classes under all
platforms).

\wxheading{See also}

\helpref{ScrollPages}{wxwindowscrollpages}


\membersection{wxWindow::ScrollPages}\label{wxwindowscrollpages}

\func{virtual bool}{ScrollPages}{\param{int }{pages}}

Scrolls the window by the given number of pages down (if {\it pages} is
positive) or up.

\wxheading{Return value}

Returns {\tt true} if the window was scrolled, {\tt false} if it was already
on top/bottom and nothing was done.

\wxheading{Remarks}

This function is currently only implemented under MSW and wxTextCtrl under
wxGTK (it also works for wxScrolledWindow derived classes under all
platforms).

\wxheading{See also}

\helpref{ScrollLines}{wxwindowscrolllines}


\membersection{wxWindow::ScrollWindow}\label{wxwindowscrollwindow}

\func{virtual void}{ScrollWindow}{\param{int }{dx}, \param{int }{dy}, \param{const wxRect*}{ rect = NULL}}

Physically scrolls the pixels in the window and move child windows accordingly.

\wxheading{Parameters}

\docparam{dx}{Amount to scroll horizontally.}

\docparam{dy}{Amount to scroll vertically.}

\docparam{rect}{Rectangle to invalidate. If this is NULL, the whole window is invalidated. If you
pass a rectangle corresponding to the area of the window exposed by the scroll, your painting handler
can optimize painting by checking for the invalidated region. This parameter is ignored under GTK.}

\wxheading{Remarks}

Use this function to optimise your scrolling implementations, to minimise the area that must be
redrawn. Note that it is rarely required to call this function from a user program.


\membersection{wxWindow::SetAcceleratorTable}\label{wxwindowsetacceleratortable}

\func{virtual void}{SetAcceleratorTable}{\param{const wxAcceleratorTable\&}{ accel}}

Sets the accelerator table for this window. See \helpref{wxAcceleratorTable}{wxacceleratortable}.


\membersection{wxWindow::SetAccessible}\label{wxwindowsetaccessible}

\func{void}{SetAccessible}{\param{wxAccessible*}{ accessible}}

Sets the accessible for this window. Any existing accessible for this window
will be deleted first, if not identical to {\it accessible}.

See also \helpref{wxAccessible}{wxaccessible}.


\membersection{wxWindow::SetAutoLayout}\label{wxwindowsetautolayout}

\func{void}{SetAutoLayout}{\param{bool}{ autoLayout}}

Determines whether the \helpref{wxWindow::Layout}{wxwindowlayout} function will
be called automatically when the window is resized. It is called implicitly by
\helpref{wxWindow::SetSizer}{wxwindowsetsizer} but if you use
\helpref{wxWindow::SetConstraints}{wxwindowsetconstraints} you should call it
manually or otherwise the window layout won't be correctly updated when its
size changes.

\wxheading{Parameters}

\docparam{autoLayout}{Set this to {\tt true} if you wish the Layout function to be called
from within wxWindow::OnSize functions.}

\wxheading{See also}

\helpref{wxWindow::SetConstraints}{wxwindowsetconstraints}


\membersection{wxWindow::SetBackgroundColour}\label{wxwindowsetbackgroundcolour}

\func{virtual bool}{SetBackgroundColour}{\param{const wxColour\& }{colour}}

Sets the background colour of the window.

Please see \helpref{InheritAttributes}{wxwindowinheritattributes} for
explanation of the difference between this method and
\helpref{SetOwnBackgroundColour}{wxwindowsetownbackgroundcolour}.

\wxheading{Parameters}

\docparam{colour}{The colour to be used as the background colour, pass
  {\tt wxNullColour} to reset to the default colour.}

\wxheading{Remarks}

The background colour is usually painted by the default\rtfsp
\helpref{wxEraseEvent}{wxeraseevent} event handler function
under Windows and automatically under GTK.

Note that setting the background colour does not cause an immediate refresh, so you
may wish to call \helpref{wxWindow::ClearBackground}{wxwindowclearbackground} or \helpref{wxWindow::Refresh}{wxwindowrefresh} after
calling this function.

Using this function will disable attempts to use themes for this
window, if the system supports them.  Use with care since usually the
themes represent the appearance chosen by the user to be used for all
applications on the system.


\wxheading{See also}

\helpref{wxWindow::GetBackgroundColour}{wxwindowgetbackgroundcolour},\rtfsp
\helpref{wxWindow::SetForegroundColour}{wxwindowsetforegroundcolour},\rtfsp
\helpref{wxWindow::GetForegroundColour}{wxwindowgetforegroundcolour},\rtfsp
\helpref{wxWindow::ClearBackground}{wxwindowclearbackground},\rtfsp
\helpref{wxWindow::Refresh}{wxwindowrefresh},\rtfsp
\helpref{wxEraseEvent}{wxeraseevent}

\membersection{wxWindow::SetBackgroundStyle}\label{wxwindowsetbackgroundstyle}

\func{virtual void}{SetBackgroundStyle}{\param{wxBackgroundStyle}{ style}}

Sets the background style of the window. The background style indicates
whether background colour should be determined by the system (wxBG\_STYLE\_SYSTEM),
be set to a specific colour (wxBG\_STYLE\_COLOUR), or should be left to the
application to implement (wxBG\_STYLE\_CUSTOM).

On GTK+, use of wxBG\_STYLE\_CUSTOM allows the flicker-free drawing of a custom
background, such as a tiled bitmap. Currently the style has no effect on other platforms.

\wxheading{See also}

\helpref{wxWindow::SetBackgroundColour}{wxwindowsetbackgroundcolour},\rtfsp
\helpref{wxWindow::GetForegroundColour}{wxwindowgetforegroundcolour},\rtfsp
\helpref{wxWindow::GetBackgroundStyle}{wxwindowgetbackgroundstyle}


\membersection{wxWindow::SetBestFittingSize}\label{wxwindowsetbestfittingsize}

\func{void}{SetBestFittingSize}{\param{const wxSize\& }{size = wxDefaultSize}}

A {\it smart} SetSize that will fill in default size components with the
window's {\it best} size values.  Also sets the window's minsize to
the value passed in for use with sizers.  This means that if a full or
partial size is passed to this function then the sizers will use that
size instead of the results of GetBestSize to determine the minimum
needs of the window for layout.

\wxheading{See also}

\helpref{wxWindow::SetSize}{wxwindowsetsize},\rtfsp
\helpref{wxWindow::GetBestSize}{wxwindowgetbestsize},\rtfsp
\helpref{wxWindow::GetBestFittingSize}{wxwindowgetbestfittingsize},\rtfsp
\helpref{wxWindow::SetSizeHints}{wxwindowsetsizehints}


\membersection{wxWindow::SetCaret}\label{wxwindowsetcaret}

\constfunc{void}{SetCaret}{\param{wxCaret *}{caret}}

Sets the \helpref{caret}{wxcaret} associated with the window.


\membersection{wxWindow::SetClientSize}\label{wxwindowsetclientsize}

\func{virtual void}{SetClientSize}{\param{int}{ width}, \param{int}{ height}}

\func{virtual void}{SetClientSize}{\param{const wxSize\&}{ size}}

This sets the size of the window client area in pixels. Using this function to size a window
tends to be more device-independent than \helpref{wxWindow::SetSize}{wxwindowsetsize}, since the application need not
worry about what dimensions the border or title bar have when trying to fit the window
around panel items, for example.

\wxheading{Parameters}

\docparam{width}{The required client area width.}

\docparam{height}{The required client area height.}

\docparam{size}{The required client size.}

\pythonnote{In place of a single overloaded method name, wxPython
implements the following methods:\par
\indented{2cm}{\begin{twocollist}
\twocolitem{{\bf SetClientSize(size)}}{Accepts a wxSize}
\twocolitem{{\bf SetClientSizeWH(width, height)}}{}
\end{twocollist}}
}


\membersection{wxWindow::SetContainingSizer}\label{wxwindowsetcontainingsizer}

\func{void}{SetContainingSizer}{\param{wxSizer* }{sizer}}

This normally does not need to be called by user code.  It is called
when a window is added to a sizer, and is used so the window can
remove itself from the sizer when it is destroyed.


\membersection{wxWindow::SetCursor}\label{wxwindowsetcursor}

\func{virtual void}{SetCursor}{\param{const wxCursor\&}{cursor}}

% VZ: the docs are correct, if the code doesn't behave like this, it must be
%     changed
Sets the window's cursor. Notice that the window cursor also sets it for the
children of the window implicitly.

The {\it cursor} may be {\tt wxNullCursor} in which case the window cursor will
be reset back to default.

\wxheading{Parameters}

\docparam{cursor}{Specifies the cursor that the window should normally display.}

\wxheading{See also}

\helpref{::wxSetCursor}{wxsetcursor}, \helpref{wxCursor}{wxcursor}


\membersection{wxWindow::SetConstraints}\label{wxwindowsetconstraints}

\func{void}{SetConstraints}{\param{wxLayoutConstraints* }{constraints}}

Sets the window to have the given layout constraints. The window
will then own the object, and will take care of its deletion.
If an existing layout constraints object is already owned by the
window, it will be deleted.

\wxheading{Parameters}

\docparam{constraints}{The constraints to set. Pass NULL to disassociate and delete the window's
constraints.}

\wxheading{Remarks}

You must call \helpref{wxWindow::SetAutoLayout}{wxwindowsetautolayout} to tell a window to use
the constraints automatically in OnSize; otherwise, you must override OnSize and call Layout()
explicitly. When setting both a wxLayoutConstraints and a \helpref{wxSizer}{wxsizer}, only the
sizer will have effect.

\membersection{wxWindow::SetInitialBestSize}\label{wxwindowsetinitialbestsize}

\func{virtual void}{SetInitialBestSize}{\param{const wxSize\& }{size}}

Sets the initial window size if none is given (i.e. at least one of the
components of the size passed to ctor/Create() is wxDefaultCoord).

\membersection{wxWindow::SetMaxSize}\label{wxwindowsetmaxsize}

\func{void}{SetMaxSize}{\param{const wxSize\& }{size}}

Sets the maximum size of the window, to indicate to the sizer layout mechanism
that this is the maximum possible size.

\membersection{wxWindow::SetMinSize}\label{wxwindowsetminsize}

\func{void}{SetMinSize}{\param{const wxSize\& }{size}}

Sets the minimum size of the window, to indicate to the sizer layout mechanism
that this is the minimum required size. You may need to call this
if you change the window size after construction and before adding
to its parent sizer.

\membersection{wxWindow::SetOwnBackgroundColour}\label{wxwindowsetownbackgroundcolour}

\func{void}{SetOwnBackgroundColour}{\param{const wxColour\& }{colour}}

Sets the background colour of the window but prevents it from being inherited
by the children of this window.

\wxheading{See also}

\helpref{SetBackgroundColour}{wxwindowsetbackgroundcolour},\rtfsp
\helpref{InheritAttributes}{wxwindowinheritattributes}


\membersection{wxWindow::SetOwnFont}\label{wxwindowsetownfont}

\func{void}{SetOwnFont}{\param{const wxFont\& }{font}}

Sets the font of the window but prevents it from being inherited by the
children of this window.

\wxheading{See also}

\helpref{SetFont}{wxwindowsetfont},\rtfsp
\helpref{InheritAttributes}{wxwindowinheritattributes}


\membersection{wxWindow::SetOwnForegroundColour}\label{wxwindowsetownforegroundcolour}

\func{void}{SetOwnForegroundColour}{\param{const wxColour\& }{colour}}

Sets the foreground colour of the window but prevents it from being inherited
by the children of this window.

\wxheading{See also}

\helpref{SetForegroundColour}{wxwindowsetforegroundcolour},\rtfsp
\helpref{InheritAttributes}{wxwindowinheritattributes}


\membersection{wxWindow::SetDropTarget}\label{wxwindowsetdroptarget}

\func{void}{SetDropTarget}{\param{wxDropTarget*}{ target}}

Associates a drop target with this window.

If the window already has a drop target, it is deleted.

\wxheading{See also}

\helpref{wxWindow::GetDropTarget}{wxwindowgetdroptarget},
\helpref{Drag and drop overview}{wxdndoverview}



\membersection{wxWindow::SetEventHandler}\label{wxwindowseteventhandler}

\func{void}{SetEventHandler}{\param{wxEvtHandler* }{handler}}

Sets the event handler for this window.

\wxheading{Parameters}

\docparam{handler}{Specifies the handler to be set.}

\wxheading{Remarks}

An event handler is an object that is capable of processing the events
sent to a window. By default, the window is its own event handler, but
an application may wish to substitute another, for example to allow
central implementation of event-handling for a variety of different
window classes.

It is usually better to use \helpref{wxWindow::PushEventHandler}{wxwindowpusheventhandler} since
this sets up a chain of event handlers, where an event not handled by one event handler is
handed to the next one in the chain.

\wxheading{See also}

\helpref{wxWindow::GetEventHandler}{wxwindowgeteventhandler},\rtfsp
\helpref{wxWindow::PushEventHandler}{wxwindowpusheventhandler},\rtfsp
\helpref{wxWindow::PopEventHandler}{wxwindowpusheventhandler},\rtfsp
\helpref{wxEvtHandler::ProcessEvent}{wxevthandlerprocessevent},\rtfsp
\helpref{wxEvtHandler}{wxevthandler}


\membersection{wxWindow::SetExtraStyle}\label{wxwindowsetextrastyle}

\func{void}{SetExtraStyle}{\param{long }{exStyle}}

Sets the extra style bits for the window. The currently defined extra style
bits are:

\twocolwidtha{5cm}%
\begin{twocollist}\itemsep=0pt
\twocolitem{\windowstyle{wxWS\_EX\_VALIDATE\_RECURSIVELY}}{TransferDataTo/FromWindow()
and Validate() methods will recursively descend into all children of the
window if it has this style flag set.}
\twocolitem{\windowstyle{wxWS\_EX\_BLOCK\_EVENTS}}{Normally, the command
events are propagated upwards to the window parent recursively until a handler
for them is found. Using this style allows to prevent them from being
propagated beyond this window. Notice that wxDialog has this style on by
default for the reasons explained in the
\helpref{event processing overview}{eventprocessing}.}
\twocolitem{\windowstyle{wxWS\_EX\_TRANSIENT}}{This can be used to prevent a
window from being used as an implicit parent for the dialogs which were
created without a parent. It is useful for the windows which can disappear at
any moment as creating children of such windows results in fatal problems.}
\twocolitem{\windowstyle{wxFRAME\_EX\_CONTEXTHELP}}{Under Windows, puts a query button on the
caption. When pressed, Windows will go into a context-sensitive help mode and wxWidgets will send
a wxEVT\_HELP event if the user clicked on an application window.
This style cannot be used together with wxMAXIMIZE\_BOX or wxMINIMIZE\_BOX, so
you should use the style of
{\tt wxDEFAULT\_FRAME\_STYLE \& \textasciitilde(wxMINIMIZE\_BOX | wxMAXIMIZE\_BOX)} for the
frames having this style (the dialogs don't have minimize nor maximize box by
default)}
\twocolitem{\windowstyle{wxWS\_EX\_PROCESS\_IDLE}}{This window should always process idle events, even
if the mode set by \helpref{wxIdleEvent::SetMode}{wxidleeventsetmode} is wxIDLE\_PROCESS\_SPECIFIED.}
\twocolitem{\windowstyle{wxWS\_EX\_PROCESS\_UI\_UPDATES}}{This window should always process UI update events,
even if the mode set by \helpref{wxUpdateUIEvent::SetMode}{wxupdateuieventsetmode} is wxUPDATE\_UI\_PROCESS\_SPECIFIED.}
\end{twocollist}


\membersection{wxWindow::SetFocus}\label{wxwindowsetfocus}

\func{virtual void}{SetFocus}{\void}

This sets the window to receive keyboard input.

\wxheading{See also}

\helpref{wxFocusEvent}{wxfocusevent}
\helpref{wxPanel::SetFocus}{wxpanelsetfocus}
\helpref{wxPanel::SetFocusIgnoringChildren}{wxpanelsetfocusignoringchildren}


\membersection{wxWindow::SetFocusFromKbd}\label{wxwindowsetfocusfromkbd}

\func{virtual void}{SetFocusFromKbd}{\void}

This function is called by wxWidgets keyboard navigation code when the user
gives the focus to this window from keyboard (e.g. using {\tt TAB} key).
By default this method simply calls \helpref{SetFocus}{wxwindowsetfocus} but
can be overridden to do something in addition to this in the derived classes.


\membersection{wxWindow::SetFont}\label{wxwindowsetfont}

\func{void}{SetFont}{\param{const wxFont\& }{font}}

Sets the font for this window. This function should not be called for the
parent window if you don't want its font to be inherited by its children,
use \helpref{SetOwnFont}{wxwindowsetownfont} instead in this case and
see \helpref{InheritAttributes}{wxwindowinheritattributes} for more
explanations.

\wxheading{Parameters}

\docparam{font}{Font to associate with this window, pass
{\tt wxNullFont} to reset to the default font.}

\wxheading{See also}

\helpref{wxWindow::GetFont}{wxwindowgetfont},\\
\helpref{InheritAttributes}{wxwindowinheritattributes}


\membersection{wxWindow::SetForegroundColour}\label{wxwindowsetforegroundcolour}

\func{virtual void}{SetForegroundColour}{\param{const wxColour\& }{colour}}

Sets the foreground colour of the window.

Please see \helpref{InheritAttributes}{wxwindowinheritattributes} for
explanation of the difference between this method and
\helpref{SetOwnForegroundColour}{wxwindowsetownforegroundcolour}.

\wxheading{Parameters}

\docparam{colour}{The colour to be used as the foreground colour, pass
  {\tt wxNullColour} to reset to the default colour.}

\wxheading{Remarks}

The interpretation of foreground colour is open to interpretation according
to the window class; it may be the text colour or other colour, or it may not
be used at all.

Using this function will disable attempts to use themes for this
window, if the system supports them.  Use with care since usually the
themes represent the appearance chosen by the user to be used for all
applications on the system.

\wxheading{See also}

\helpref{wxWindow::GetForegroundColour}{wxwindowgetforegroundcolour},\rtfsp
\helpref{wxWindow::SetBackgroundColour}{wxwindowsetbackgroundcolour},\rtfsp
\helpref{wxWindow::GetBackgroundColour}{wxwindowgetbackgroundcolour},\rtfsp
\helpref{wxWindow::ShouldInheritColours}{wxwindowshouldinheritcolours}


\membersection{wxWindow::SetHelpText}\label{wxwindowsethelptext}

\func{virtual void}{SetHelpText}{\param{const wxString\& }{helpText}}

Sets the help text to be used as context-sensitive help for this window.

Note that the text is actually stored by the current \helpref{wxHelpProvider}{wxhelpprovider} implementation,
and not in the window object itself.

\wxheading{See also}

\helpref{GetHelpText}{wxwindowgethelptext}, \helpref{wxHelpProvider}{wxhelpprovider}


\membersection{wxWindow::SetId}\label{wxwindowsetid}

\func{void}{SetId}{\param{int}{ id}}

Sets the identifier of the window.

\wxheading{Remarks}

Each window has an integer identifier. If the application has not provided one,
an identifier will be generated. Normally, the identifier should be provided
on creation and should not be modified subsequently.

\wxheading{See also}

\helpref{wxWindow::GetId}{wxwindowgetid},\rtfsp
\helpref{Window identifiers}{windowids}



\membersection{wxWindow::SetLabel}\label{wxwindowsetlabel}

\func{virtual void}{SetLabel}{\param{const wxString\& }{label}}

Sets the window's label.

\wxheading{Parameters}

\docparam{label}{The window label.}

\wxheading{See also}

\helpref{wxWindow::GetLabel}{wxwindowgetlabel}


\membersection{wxWindow::SetName}\label{wxwindowsetname}

\func{virtual void}{SetName}{\param{const wxString\& }{name}}

Sets the window's name.

\wxheading{Parameters}

\docparam{name}{A name to set for the window.}

\wxheading{See also}

\helpref{wxWindow::GetName}{wxwindowgetname}


\membersection{wxWindow::SetPalette}\label{wxwindowsetpalette}

\func{virtual void}{SetPalette}{\param{wxPalette* }{palette}}

Obsolete - use \helpref{wxDC::SetPalette}{wxdcsetpalette} instead.


\membersection{wxWindow::SetScrollbar}\label{wxwindowsetscrollbar}

\func{virtual void}{SetScrollbar}{\param{int }{orientation}, \param{int }{position},\rtfsp
\param{int }{thumbSize}, \param{int }{range},\rtfsp
\param{bool }{refresh = {\tt true}}}

Sets the scrollbar properties of a built-in scrollbar.

\wxheading{Parameters}

\docparam{orientation}{Determines the scrollbar whose page size is to be set. May be wxHORIZONTAL or wxVERTICAL.}

\docparam{position}{The position of the scrollbar in scroll units.}

\docparam{thumbSize}{The size of the thumb, or visible portion of the scrollbar, in scroll units.}

\docparam{range}{The maximum position of the scrollbar.}

\docparam{refresh}{{\tt true} to redraw the scrollbar, {\tt false} otherwise.}

\wxheading{Remarks}

Let's say you wish to display 50 lines of text, using the same font.
The window is sized so that you can only see 16 lines at a time.

You would use:

{\small%
\begin{verbatim}
  SetScrollbar(wxVERTICAL, 0, 16, 50);
\end{verbatim}
}

Note that with the window at this size, the thumb position can never go
above 50 minus 16, or 34.

You can determine how many lines are currently visible by dividing the current view
size by the character height in pixels.

When defining your own scrollbar behaviour, you will always need to recalculate
the scrollbar settings when the window size changes. You could therefore put your
scrollbar calculations and SetScrollbar
call into a function named AdjustScrollbars, which can be called initially and also
from your \helpref{wxSizeEvent}{wxsizeevent} handler function.

\wxheading{See also}

\helpref{Scrolling overview}{scrollingoverview},\rtfsp
\helpref{wxScrollBar}{wxscrollbar}, \helpref{wxScrolledWindow}{wxscrolledwindow}

\begin{comment}


\membersection{wxWindow::SetScrollPage}\label{wxwindowsetscrollpage}

\func{virtual void}{SetScrollPage}{\param{int }{orientation}, \param{int }{pageSize}, \param{bool }{refresh = {\tt true}}}

Sets the page size of one of the built-in scrollbars.

\wxheading{Parameters}

\docparam{orientation}{Determines the scrollbar whose page size is to be set. May be wxHORIZONTAL or wxVERTICAL.}

\docparam{pageSize}{Page size in scroll units.}

\docparam{refresh}{{\tt true} to redraw the scrollbar, {\tt false} otherwise.}

\wxheading{Remarks}

The page size of a scrollbar is the number of scroll units that the scroll thumb travels when you
click on the area above/left of or below/right of the thumb. Normally you will want a whole visible
page to be scrolled, i.e. the size of the current view (perhaps the window client size). This
value has to be adjusted when the window is resized, since the page size will have changed.

In addition to specifying how far the scroll thumb travels when paging, in Motif and some versions of Windows
the thumb changes size to reflect the page size relative to the length of the document. When the
document size is only slightly bigger than the current view (window) size, almost all of the scrollbar
will be taken up by the thumb. When the two values become the same, the scrollbar will (on some systems)
disappear.

Currently, this function should be called before SetPageRange, because of a quirk in the Windows
handling of pages and ranges.

\wxheading{See also}

\helpref{wxWindow::SetScrollPos}{wxwindowsetscrollpos},\rtfsp
\helpref{wxWindow::GetScrollPos}{wxwindowgetscrollpos},\rtfsp
\helpref{wxWindow::GetScrollPage}{wxwindowgetscrollpage},\rtfsp
\helpref{wxScrollBar}{wxscrollbar}, \helpref{wxScrolledWindow}{wxscrolledwindow}
\end{comment}


\membersection{wxWindow::SetScrollPos}\label{wxwindowsetscrollpos}

\func{virtual void}{SetScrollPos}{\param{int }{orientation}, \param{int }{pos}, \param{bool }{refresh = {\tt true}}}

Sets the position of one of the built-in scrollbars.

\wxheading{Parameters}

\docparam{orientation}{Determines the scrollbar whose position is to be set. May be wxHORIZONTAL or wxVERTICAL.}

\docparam{pos}{Position in scroll units.}

\docparam{refresh}{{\tt true} to redraw the scrollbar, {\tt false} otherwise.}

\wxheading{Remarks}

This function does not directly affect the contents of the window: it is up to the
application to take note of scrollbar attributes and redraw contents accordingly.

\wxheading{See also}

\helpref{wxWindow::SetScrollbar}{wxwindowsetscrollbar},\rtfsp
\helpref{wxWindow::GetScrollPos}{wxwindowgetscrollpos},\rtfsp
\helpref{wxWindow::GetScrollThumb}{wxwindowgetscrollthumb},\rtfsp
\helpref{wxScrollBar}{wxscrollbar}, \helpref{wxScrolledWindow}{wxscrolledwindow}

\begin{comment}


\membersection{wxWindow::SetScrollRange}\label{wxwindowsetscrollrange}

\func{virtual void}{SetScrollRange}{\param{int }{orientation}, \param{int }{range}, \param{bool }{refresh = {\tt true}}}

Sets the range of one of the built-in scrollbars.

\wxheading{Parameters}

\docparam{orientation}{Determines the scrollbar whose range is to be set. May be wxHORIZONTAL or wxVERTICAL.}

\docparam{range}{Scroll range.}

\docparam{refresh}{{\tt true} to redraw the scrollbar, {\tt false} otherwise.}

\wxheading{Remarks}

The range of a scrollbar is the number of steps that the thumb may travel, rather than the total
object length of the scrollbar. If you are implementing a scrolling window, for example, you
would adjust the scroll range when the window is resized, by subtracting the window view size from the
total virtual window size. When the two sizes are the same (all the window is visible), the range goes to zero
and usually the scrollbar will be automatically hidden.

\wxheading{See also}

\helpref{wxWindow::SetScrollPos}{wxwindowsetscrollpos},\rtfsp
\helpref{wxWindow::SetScrollPage}{wxwindowsetscrollpage},\rtfsp
\helpref{wxWindow::GetScrollPos}{wxwindowgetscrollpos},\rtfsp
\helpref{wxWindow::GetScrollPage}{wxwindowgetscrollpage},\rtfsp
\helpref{wxScrollBar}{wxscrollbar}, \helpref{wxScrolledWindow}{wxscrolledwindow}
\end{comment}


\membersection{wxWindow::SetSize}\label{wxwindowsetsize}

\func{virtual void}{SetSize}{\param{int}{ x}, \param{int}{ y}, \param{int}{ width}, \param{int}{ height},
 \param{int}{ sizeFlags = wxSIZE\_AUTO}}

\func{virtual void}{SetSize}{\param{const wxRect\&}{ rect}}

Sets the size and position of the window in pixels.

\func{virtual void}{SetSize}{\param{int}{ width}, \param{int}{ height}}

\func{virtual void}{SetSize}{\param{const wxSize\&}{ size}}

Sets the size of the window in pixels.

\wxheading{Parameters}

\docparam{x}{Required x position in pixels, or -1 to indicate that the existing
value should be used.}

\docparam{y}{Required y position in pixels, or -1 to indicate that the existing
value should be used.}

\docparam{width}{Required width in pixels, or -1 to indicate that the existing
value should be used.}

\docparam{height}{Required height position in pixels, or -1 to indicate that the existing
value should be used.}

\docparam{size}{\helpref{wxSize}{wxsize} object for setting the size.}

\docparam{rect}{\helpref{wxRect}{wxrect} object for setting the position and size.}

\docparam{sizeFlags}{Indicates the interpretation of other parameters. It is a bit list of the following:

{\bf wxSIZE\_AUTO\_WIDTH}: a -1 width value is taken to indicate
a wxWidgets-supplied default width.\\
{\bf wxSIZE\_AUTO\_HEIGHT}: a -1 height value is taken to indicate
a wxWidgets-supplied default width.\\
{\bf wxSIZE\_AUTO}: -1 size values are taken to indicate
a wxWidgets-supplied default size.\\
{\bf wxSIZE\_USE\_EXISTING}: existing dimensions should be used
if -1 values are supplied.\\
{\bf wxSIZE\_ALLOW\_MINUS\_ONE}: allow dimensions of -1 and less to be interpreted
as real dimensions, not default values.
}

\wxheading{Remarks}

The second form is a convenience for calling the first form with default
x and y parameters, and must be used with non-default width and height values.

The first form sets the position and optionally size, of the window.
Parameters may be -1 to indicate either that a default should be supplied
by wxWidgets, or that the current value of the dimension should be used.

\wxheading{See also}

\helpref{wxWindow::Move}{wxwindowmove}

\pythonnote{In place of a single overloaded method name, wxPython
implements the following methods:\par
\indented{2cm}{\begin{twocollist}
\twocolitem{{\bf SetDimensions(x, y, width, height, sizeFlags=wxSIZE\_AUTO)}}{}
\twocolitem{{\bf SetSize(size)}}{}
\twocolitem{{\bf SetPosition(point)}}{}
\end{twocollist}}
}


\membersection{wxWindow::SetSizeHints}\label{wxwindowsetsizehints}

\func{virtual void}{SetSizeHints}{\param{int}{ minW=-1}, \param{int}{ minH=-1}, \param{int}{ maxW=-1}, \param{int}{ maxH=-1},
 \param{int}{ incW=-1}, \param{int}{ incH=-1}}

\func{void}{SetSizeHints}{\param{const wxSize\&}{ minSize},
\param{const wxSize\&}{ maxSize=wxDefaultSize}, \param{const wxSize\&}{ incSize=wxDefaultSize}}


Allows specification of minimum and maximum window sizes, and window size increments.
If a pair of values is not set (or set to -1), the default values will be used.

\wxheading{Parameters}

\docparam{minW}{Specifies the minimum width allowable.}

\docparam{minH}{Specifies the minimum height allowable.}

\docparam{maxW}{Specifies the maximum width allowable.}

\docparam{maxH}{Specifies the maximum height allowable.}

\docparam{incW}{Specifies the increment for sizing the width (Motif/Xt only).}

\docparam{incH}{Specifies the increment for sizing the height (Motif/Xt only).}

\docparam{minSize}{Minimum size.}

\docparam{maxSize}{Maximum size.}

\docparam{incSize}{Increment size (Motif/Xt only).}

\wxheading{Remarks}

If this function is called, the user will not be able to size the window outside the
given bounds.

The resizing increments are only significant under Motif or Xt.


\membersection{wxWindow::SetSizer}\label{wxwindowsetsizer}

\func{void}{SetSizer}{\param{wxSizer* }{sizer}, \param{bool }{deleteOld=true}}

Sets the window to have the given layout sizer. The window
will then own the object, and will take care of its deletion.
If an existing layout constraints object is already owned by the
window, it will be deleted if the deleteOld parameter is true.

Note that this function will also call
\helpref{SetAutoLayout}{wxwindowsetautolayout} implicitly with {\tt true}
parameter if the {\it sizer}\/ is non-NULL and {\tt false} otherwise.

\wxheading{Parameters}

\docparam{sizer}{The sizer to set. Pass NULL to disassociate and conditionally delete
the window's sizer.  See below.}

\docparam{deleteOld}{If true (the default), this will delete any prexisting sizer.
Pass false if you wish to handle deleting the old sizer yourself.}

\wxheading{Remarks}

SetSizer now enables and disables Layout automatically, but prior to wxWidgets 2.3.3
the following applied:

You must call \helpref{wxWindow::SetAutoLayout}{wxwindowsetautolayout} to tell a window to use
the sizer automatically in OnSize; otherwise, you must override OnSize and call Layout()
explicitly. When setting both a wxSizer and a \helpref{wxLayoutConstraints}{wxlayoutconstraints},
only the sizer will have effect.


\membersection{wxWindow::SetSizerAndFit}\label{wxwindowsetsizerandfit}

\func{void}{SetSizerAndFit}{\param{wxSizer* }{sizer}, \param{bool }{deleteOld=true}}

The same as \helpref{SetSizer}{wxwindowsetsizer}, except it also sets the size hints
for the window based on the sizer's minimum size.


\membersection{wxWindow::SetTitle}\label{wxwindowsettitle}

\func{virtual void}{SetTitle}{\param{const wxString\& }{title}}

Sets the window's title. Applicable only to frames and dialogs.

\wxheading{Parameters}

\docparam{title}{The window's title.}

\wxheading{See also}

\helpref{wxWindow::GetTitle}{wxwindowgettitle}


\membersection{wxWindow::SetThemeEnabled}\label{wxwindowsetthemeenabled}

\func{virtual void}{SetThemeEnabled}{\param{bool }{enable}}

This function tells a window if it should use the system's "theme" code
to draw the windows' background instead if its own background drawing
code. This does not always have any effect since the underlying platform
obviously needs to support the notion of themes in user defined windows.
One such platform is GTK+ where windows can have (very colourful) backgrounds
defined by a user's selected theme.

Dialogs, notebook pages and the status bar have this flag set to true
by default so that the default look and feel is simulated best.


\membersection{wxWindow::SetToolTip}\label{wxwindowsettooltip}

\func{void}{SetToolTip}{\param{const wxString\& }{tip}}

\func{void}{SetToolTip}{\param{wxToolTip* }{tip}}

Attach a tooltip to the window.

See also: \helpref{GetToolTip}{wxwindowgettooltip},
 \helpref{wxToolTip}{wxtooltip}


\membersection{wxWindow::SetValidator}\label{wxwindowsetvalidator}

\func{virtual void}{SetValidator}{\param{const wxValidator\&}{ validator}}

Deletes the current validator (if any) and sets the window validator, having called wxValidator::Clone to
create a new validator of this type.


\membersection{wxWindow::SetVirtualSize}\label{wxwindowsetvirtualsize}

\func{void}{SetVirtualSize}{\param{int}{ width}, \param{int}{ height}}

\func{void}{SetVirtualSize}{\param{const wxSize\&}{ size}}

Sets the virtual size of the window in pixels.


\membersection{wxWindow::SetVirtualSizeHints}\label{wxwindowsetvirtualsizehints}

\func{virtual void}{SetVirtualSizeHints}{\param{int}{ minW},\param{int}{ minH}, \param{int}{ maxW=-1}, \param{int}{ maxH=-1}}

\func{void}{SetVirtualSizeHints}{\param{const wxSize\&}{ minSize=wxDefaultSize},
\param{const wxSize\&}{ maxSize=wxDefaultSize}}


Allows specification of minimum and maximum virtual window sizes.
If a pair of values is not set (or set to -1), the default values
will be used.

\wxheading{Parameters}

\docparam{minW}{Specifies the minimum width allowable.}

\docparam{minH}{Specifies the minimum height allowable.}

\docparam{maxW}{Specifies the maximum width allowable.}

\docparam{maxH}{Specifies the maximum height allowable.}

\docparam{minSize}{Minimum size.}

\docparam{maxSize}{Maximum size.}

\wxheading{Remarks}

If this function is called, the user will not be able to size the virtual area
of the window outside the given bounds.


\membersection{wxWindow::SetWindowStyle}\label{wxwindowsetwindowstyle}

\func{void}{SetWindowStyle}{\param{long}{ style}}

Identical to \helpref{SetWindowStyleFlag}{wxwindowsetwindowstyleflag}.


\membersection{wxWindow::SetWindowStyleFlag}\label{wxwindowsetwindowstyleflag}

\func{virtual void}{SetWindowStyleFlag}{\param{long}{ style}}

Sets the style of the window. Please note that some styles cannot be changed
after the window creation and that \helpref{Refresh()}{wxwindowrefresh} might
be called after changing the others for the change to take place immediately.

See \helpref{Window styles}{windowstyles} for more information about flags.

\wxheading{See also}

\helpref{GetWindowStyleFlag}{wxwindowgetwindowstyleflag}


\membersection{wxWindow::SetWindowVariant}\label{wxwindowsetwindowvariant}

\func{void}{SetWindowVariant}{\param{wxWindowVariant}{variant}}

This function can be called under all platforms but only does anything under
Mac OS X 10.3+ currently. Under this system, each of the standard control can
exist in several sizes which correspond to the elements of wxWindowVariant
enum:
\begin{verbatim}
enum wxWindowVariant
{
    wxWINDOW_VARIANT_NORMAL,        // Normal size
    wxWINDOW_VARIANT_SMALL,         // Smaller size (about 25 % smaller than normal )
    wxWINDOW_VARIANT_MINI,          // Mini size (about 33 % smaller than normal )
    wxWINDOW_VARIANT_LARGE,         // Large size (about 25 % larger than normal )
};
\end{verbatim}

By default the controls use the normal size, of course, but this function can
be used to change this.


\membersection{wxWindow::ShouldInheritColours}\label{wxwindowshouldinheritcolours}

\func{virtual bool}{ShouldInheritColours}{\void}

Return \true from here to allow the colours of this window to be changed by
\helpref{InheritAttributes}{wxwindowinheritattributes}, returning \false
forbids inheriting them from the parent window.

The base class version returns \false, but this method is overridden in
\helpref{wxControl}{wxcontrol} where it returns \true.


\membersection{wxWindow::Show}\label{wxwindowshow}

\func{virtual bool}{Show}{\param{bool}{ show = {\tt true}}}

Shows or hides the window. You may need to call \helpref{Raise}{wxwindowraise}
for a top level window if you want to bring it to top, although this is not
needed if Show() is called immediately after the frame creation.

\wxheading{Parameters}

\docparam{show}{If {\tt true} displays the window. Otherwise, hides it.}

\wxheading{Return value}

{\tt true} if the window has been shown or hidden or {\tt false} if nothing was
done because it already was in the requested state.

\wxheading{See also}

\helpref{wxWindow::IsShown}{wxwindowisshown},\rtfsp
\helpref{wxWindow::Hide}{wxwindowhide},\rtfsp
\helpref{wxRadioBox::Show}{wxradioboxshow}


\membersection{wxWindow::Thaw}\label{wxwindowthaw}

\func{virtual void}{Thaw}{\void}

Reenables window updating after a previous call to
\helpref{Freeze}{wxwindowfreeze}. To really thaw the control, it must be called
exactly the same number of times as \helpref{Freeze}{wxwindowfreeze}.


\membersection{wxWindow::TransferDataFromWindow}\label{wxwindowtransferdatafromwindow}

\func{virtual bool}{TransferDataFromWindow}{\void}

Transfers values from child controls to data areas specified by their validators. Returns
{\tt false} if a transfer failed.

If the window has {\tt wxWS\_EX\_VALIDATE\_RECURSIVELY} extra style flag set,
the method will also call TransferDataFromWindow() of all child windows.

\wxheading{See also}

\helpref{wxWindow::TransferDataToWindow}{wxwindowtransferdatatowindow},\rtfsp
\helpref{wxValidator}{wxvalidator}, \helpref{wxWindow::Validate}{wxwindowvalidate}


\membersection{wxWindow::TransferDataToWindow}\label{wxwindowtransferdatatowindow}

\func{virtual bool}{TransferDataToWindow}{\void}

Transfers values to child controls from data areas specified by their validators.

If the window has {\tt wxWS\_EX\_VALIDATE\_RECURSIVELY} extra style flag set,
the method will also call TransferDataToWindow() of all child windows.

\wxheading{Return value}

Returns {\tt false} if a transfer failed.

\wxheading{See also}

\helpref{wxWindow::TransferDataFromWindow}{wxwindowtransferdatafromwindow},\rtfsp
\helpref{wxValidator}{wxvalidator}, \helpref{wxWindow::Validate}{wxwindowvalidate}


\membersection{wxWindow::UnregisterHotKey}\label{wxwindowunregisterhotkey}

\func{bool}{UnregisterHotKey}{\param{int}{ hotkeyId}}

Unregisters a system wide hotkey.

\wxheading{Parameters}

\docparam{hotkeyId}{Numeric identifier of the hotkey. Must be the same id that was passed to RegisterHotKey.}

\wxheading{Return value}

{\tt true} if the hotkey was unregistered successfully, {\tt false} if the id was invalid.

\wxheading{Remarks}

This function is currently only implemented under MSW.

\wxheading{See also}

\helpref{wxWindow::RegisterHotKey}{wxwindowregisterhotkey}


\membersection{wxWindow::Update}\label{wxwindowupdate}

\func{virtual void}{Update}{\void}

Calling this method immediately repaints the invalidated area of the window
while this would usually only happen when the flow of control returns to the
event loop. Notice that this function doesn't refresh the window and does
nothing if the window hadn't been already repainted. Use
\helpref{Refresh}{wxwindowrefresh} first if you want to immediately redraw the
window unconditionally.


\membersection{wxWindow::UpdateWindowUI}\label{wxwindowupdatewindowui}

\func{virtual void}{UpdateWindowUI}{\param{long}{ flags = wxUPDATE\_UI\_NONE}}

This function sends \helpref{wxUpdateUIEvents}{wxupdateuievent} to
the window. The particular implementation depends on the window; for
example a wxToolBar will send an update UI event for each toolbar button,
and a wxFrame will send an update UI event for each menubar menu item.
You can call this function from your application to ensure that your
UI is up-to-date at this point (as far as your wxUpdateUIEvent handlers
are concerned). This may be necessary if you have called
\helpref{wxUpdateUIEvent::SetMode}{wxupdateuieventsetmode} or
\helpref{wxUpdateUIEvent::SetUpdateInterval}{wxupdateuieventsetupdateinterval} to
limit the overhead that wxWidgets incurs by sending update UI events in idle time.

{\it flags} should be a bitlist of one or more of the following values.

\begin{verbatim}
enum wxUpdateUI
{
    wxUPDATE_UI_NONE          = 0x0000, // No particular value
    wxUPDATE_UI_RECURSE       = 0x0001, // Call the function for descendants
    wxUPDATE_UI_FROMIDLE      = 0x0002  // Invoked from On(Internal)Idle
};
\end{verbatim}

If you are calling this function from an OnInternalIdle or OnIdle
function, make sure you pass the wxUPDATE\_UI\_FROMIDLE flag, since
this tells the window to only update the UI elements that need
to be updated in idle time. Some windows update their elements
only when necessary, for example when a menu is about to be shown.
The following is an example of how to call UpdateWindowUI from
an idle function.

\begin{verbatim}
void MyWindow::OnInternalIdle()
{
    if (wxUpdateUIEvent::CanUpdate(this))
        UpdateWindowUI(wxUPDATE_UI_FROMIDLE);
}
\end{verbatim}

\wxheading{See also}

\helpref{wxUpdateUIEvent}{wxupdateuievent},
\helpref{wxWindow::DoUpdateWindowUI}{wxwindowdoupdatewindowui},
\helpref{wxWindow::OnInternalIdle}{wxwindowoninternalidle}


\membersection{wxWindow::Validate}\label{wxwindowvalidate}

\func{virtual bool}{Validate}{\void}

Validates the current values of the child controls using their validators.

If the window has {\tt wxWS\_EX\_VALIDATE\_RECURSIVELY} extra style flag set,
the method will also call Validate() of all child windows.

\wxheading{Return value}

Returns {\tt false} if any of the validations failed.

\wxheading{See also}

\helpref{wxWindow::TransferDataFromWindow}{wxwindowtransferdatafromwindow},\rtfsp
\helpref{wxWindow::TransferDataToWindow}{wxwindowtransferdatatowindow},\rtfsp
\helpref{wxValidator}{wxvalidator}


\membersection{wxWindow::WarpPointer}\label{wxwindowwarppointer}

\func{void}{WarpPointer}{\param{int}{ x}, \param{int}{ y}}

Moves the pointer to the given position on the window.

{\bf NB: } This function is not supported under Mac because Apple Human
Interface Guidelines forbid moving the mouse cursor programmatically.

\wxheading{Parameters}

\docparam{x}{The new x position for the cursor.}

\docparam{y}{The new y position for the cursor.}

