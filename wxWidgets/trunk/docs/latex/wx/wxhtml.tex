\chapter{wxHTML Notes}\label{wxHTML}
\pagenumbering{arabic}%
\setheader{{\it CHAPTER \thechapter}}{}{}{}{}{{\it CHAPTER \thechapter}}%
\setfooter{\thepage}{}{}{}{}{\thepage}%

\section{wxHTML Sub-library Overview}\label{wxhtmloverview}

This library provides classes for parsing and displaying HTML.

It never intented to be hi-end HTML browser. If you're looking for
something like that try \urlref{http://www.mozilla.org}{http://www.mozilla.org} - there's a 
chance you'll be able to make their widget wxWindows-compatible. I'm sure
everyone will enjoy your work in that case...

But back to wxHTML. 

It can be used as generic rich text viewer - for example to display 
nice About Box (like these of GNOME apps) or to display result of
database searching. There is \helpref{wxFileSystem}{wxfilesystem} 
class which allows you to use your own virtual file systems...

wxHtmlWindow supports tag handlers. This means that you can easily
extend wxHtml library  with new, unsupported tags. Not only that,
you can even use your own application specific tags!
See lib/mod_*.cpp files for details.

There is generic (non-wxHtmlWindow) wxHtmlParser class.



\input htmlstrt.tex 
\input htmlprn.tex
\input htmlhlpf.tex
\input htmlfilt.tex
\input htmlcell.tex
\input htmlhand.tex
