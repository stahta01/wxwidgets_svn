%%%%%%%%%%%%%%%%%%%%%%%%%%%%%%%%%%%%%%%%%%%%%%%%%%%%%%%%%%%%%%%%%%%%%%%%%%%%%%%
%% Name:        mediactrl.tex
%% Purpose:     wxMediaCtrl docs
%% Author:      Ryan Norton <wxprojects@comcast.net>
%% Modified by:
%% Created:     11/7/2004
%% RCS-ID:      $Id$
%% Copyright:   (c) Ryan Norton
%% License:     wxWindows license
%%%%%%%%%%%%%%%%%%%%%%%%%%%%%%%%%%%%%%%%%%%%%%%%%%%%%%%%%%%%%%%%%%%%%%%%%%%%%%%

\section{\class{wxMediaCtrl}}\label{wxmediactrl}

wxMediaCtrl is a class that allows a way to convieniently display types of media,
such as videos, audio files, natively through native codecs.

On Windows wxMediaCtrl uses DirectShow, and on Macintosh wxMediaCtrl uses QuickTime.

wxMediaCtrl is not currently available on unix systems. 

\wxheading{Derived from}

\helpref{wxControl}{wxcontrol}

\wxheading{Include files}

<wx/mediactrl.h>

\latexignore{\rtfignore{\wxheading{Members}}}


\membersection{wxMediaCtrl::wxMediaCtrl}\label{wxmediactrlwxmediactrl}

\func{}{wxMediaCtrl}{\void}

Default constructor - you \tt{must} call Create before calling any other methods
of wxMediaCtrl.

\func{}{wxMediaCtrl}{\param{wxWindow* }{parent}, \param{wxWindowID }{id}, \param{const wxString\& }{fileName}, \param{const wxString\& }{label = wxT("")}, \param{const wxPoint\& }{pos = wxDefaultPosition}, \param{const wxSize\& }{size = wxDefaultSize}, \param{long }{style = 0}, \param{long }{driver = 0}, \param{const wxString\& }{name = wxPanelNameStr}}

Constructor that calls Create.  You may prefer to call Create directly to check to see if wxMediaCtrl is available
on the system.

\docparam{parent}{parent of this control.  Must not be NULL.}
\docparam{id}{id to use for events}
\docparam{fileName}{If not empty, loads this file and starts playing it immediately.}
\docparam{pos}{Position to put control at.}
\docparam{size}{Size to put the control at and to stretch movie to.}
\docparam{style}{Optional styles.}
\docparam{driver}{Reserved for future use.}
\docparam{name}{Window name.}


\membersection{wxMediaCtrl::Create}\label{wxmediactrlcreate}

\func{bool}{Create}{\param{wxWindow* }{parent}, \param{wxWindowID }{id}, \param{const wxString\& }{fileName}, \param{const wxString\& }{label = wxT("")}, \param{const wxPoint\& }{pos = wxDefaultPosition}, \param{const wxSize\& }{size = wxDefaultSize}, \param{long }{style = 0}, \param{long }{driver = 0}, \param{const wxString\& }{name = wxPanelNameStr}}

Creates this control.  Returns \tt{false} if it can't load the movie located at \tt{fileName} or it cannot load
one of its native backends.

\docparam{parent}{parent of this control.  Must not be NULL.}
\docparam{id}{id to use for events}
\docparam{fileName}{If not empty, loads this file and starts playing it immediately.}
\docparam{pos}{Position to put control at.}
\docparam{size}{Size to put the control at and to stretch movie to.}
\docparam{style}{Optional styles.}
\docparam{driver}{Reserved for future use.}
\docparam{name}{Window name.}


\membersection{wxMediaCtrl::GetDuration}\label{wxmediactrlgetduration}

\func{long}{GetDuration}{\void}

Obtains the length - the total amount of time the movie has in milliseconds.


\membersection{wxMediaCtrl::GetPosition}\label{wxmediactrlgetposition}

\func{long}{GetPosition}{\void}

Obtains the current position in time within the movie in milliseconds.


\membersection{wxMediaCtrl::GetState}\label{wxmediactrlgetstate}

\func{wxMediaCtrlState}{GetState}{\void}

Obtains the state the playback of the movie is in - 

\twocolwidtha{7cm}
\begin{twocollist}\itemsep=0pt
\twocolitem{{\bf wxMEDIASTATE\_STOPPED}}{The movie has stopped.}
\twocolitem{{\bf wxMEDIASTATE\_PAUSED}}{The movie is paused.}
\twocolitem{{\bf wxMEDIASTATE\_PLAYING}}{The movie is currently playing.}
\end{twocollist}


\membersection{wxMediaCtrl::Load}\label{wxmediactrlload}

\func{bool}{Load}{\param{const wxString\& }{fileName}}

Loads the file that \tt{fileName} refers to.  Returns false if loading fails.

\membersection{wxMediaCtrl::Pause}\label{wxmediactrlpause}

\func{bool}{Pause}{\void}

Pauses playback of the movie.


\membersection{wxMediaCtrl::Play}\label{wxmediactrlplay}

\func{bool}{Play}{\void}

Resumes playback of the movie.


\membersection{wxMediaCtrl::SetPosition}\label{wxmediactrlsetposition}

\func{bool}{SetPosition}{\param{long }{where}}

Seeks to a position within the movie.


\membersection{wxMediaCtrl::Stop}\label{wxmediactrlstop}

\func{bool}{Stop}{\void}

Stops the media.
