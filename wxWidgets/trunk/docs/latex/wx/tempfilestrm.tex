%
% automatically generated by HelpGen $Revision$ from
% wx/wfstream.h at 07/Mar/05 20:45:33
%

\section{\class{wxTempFileOutputStream}}\label{wxtempfileoutputstream}

wxTempFileOutputStream is an output stream based on \helpref{wxTempFile}{wxtempfile}. It
provides a relatively safe way to replace the contents of the
existing file.

\wxheading{Derived from}

\helpref{wxOutputStream}{wxoutputstream}\\
\helpref{wxStreamBase}{wxstreambase}

\wxheading{Include files}

<wx/wfstream.h>

\wxheading{Library}

\helpref{wxBase}{librarieslist}

\wxheading{See also}

\helpref{wxTempFile}{wxtempfile}

\latexignore{\rtfignore{\wxheading{Members}}}


\membersection{wxTempFileOutputStream::wxTempFileOutputStream}\label{wxtempfileoutputstreamwxtempfileoutputstream}

\func{}{wxTempFileOutputStream}{\param{const wxString\& }{fileName}}

Associates wxTempFileOutputStream with the file to be replaced and opens it. You should use 
\helpref{IsOk}{wxstreambaseisok} to verify if the constructor succeeded.

Call \helpref{Commit()}{wxtempfileoutputstreamcommit} or \helpref{Close()}{wxoutputstreamclose} to
replace the old file and close this one. Calling \helpref{Discard()}{wxtempfileoutputstreamdiscard} 
(or allowing the destructor to do it) will discard the changes.


\membersection{wxTempFileOutputStream::Commit}\label{wxtempfileoutputstreamcommit}

\func{bool}{Commit}{\void}

Validate changes: deletes the old file of the given name and renames the new
file to the old name. Returns {\tt true} if both actions succeeded. If {\tt false} is
returned it may unfortunately mean two quite different things: either that
either the old file couldn't be deleted or that the new file couldn't be renamed
to the old name.


\membersection{wxTempFileOutputStream::Discard}\label{wxtempfileoutputstreamdiscard}

\func{void}{Discard}{\void}

Discard changes: the old file contents are not changed, the temporary file is
deleted.

