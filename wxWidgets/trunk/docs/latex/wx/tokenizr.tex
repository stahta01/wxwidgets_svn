\section{\class{wxStringTokenizer}}\label{wxstringtokenizer}

wxStringTokenizer helps you to break a string up into a number of tokens.

\wxheading{Derived from}

\helpref{wxObject}{wxobject}

\latexignore{\rtfignore{\wxheading{Members}}}

\membersection{wxStringTokenizer::wxStringTokenizer}\label{wxstringtokenizerwxstringtokenizer}

\func{}{wxStringTokenizer}{\param{const wxString\& }{to\_tokenize}, \param{const wxString\& }{delims = " \t\r\n"}, \param{bool }{ret\_delim = FALSE}}

Constructor. Pass the string to tokenze, a string containing delimiters,
a flag specifying whether delimiters are retained.

\membersection{wxStringTokenizer::\destruct{wxStringTokenizer}}\label{wxstringtokenizerdtor}

\func{}{\destruct{wxStringTokenizer}}{\void}

Destructor.

\membersection{wxStringTokenizer::CountTokens}\label{wxstringtokenizercounttokens}

\constfunc{virtual int}{CountTokens}{\void}

Returns the number of tokens in the input string.

\membersection{wxStringTokenizer::HasMoreToken}\label{wxstringtokenizerhasmoretoken}

\constfunc{virtual bool}{HasMoreToken}{\void}

Returns TRUE if the tokenizer has further tokens.

\membersection{wxStringTokenizer::NextToken}\label{wxstringtokenizernexttoken}

\constfunc{virtual wxString}{NextToken}{\void}

Returns the next token.

\membersection{wxStringTokenizer::GetString}\label{wxstringtokenizergetstring}

\constfunc{virtual wxString}{GetString}{\void}

Returns the input string.



