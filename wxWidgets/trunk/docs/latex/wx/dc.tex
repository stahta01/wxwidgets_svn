\section{\class{wxDC}}\label{wxdc}

A wxDC is a {\it device context} onto which graphics and text can be drawn.
It is intended to represent a number of output devices in a generic way,
so a window can have a device context associated with it, and a printer also has a device context.
In this way, the same piece of code may write to a number of different devices,
if the device context is used as a parameter.

Derived types of wxDC have documentation for specific features
only, so refer to this section for most device context information.

% VZ: we should really document them instead of this lame excuse, but I don't
%     have time for it now, when it is done please remove this
Please note that in addition to the versions of the methods documented here,
there are also versions which accept single {\tt wxPoint} parameter instead of
two {\tt wxCoord} ones or {\tt wxPoint} and {\tt wxSize} instead of four of
them.

\wxheading{Derived from}

\helpref{wxObject}{wxobject}

\wxheading{Include files}

<wx/dc.h>

\wxheading{See also}

\helpref{Overview}{dcoverview}

\latexignore{\rtfignore{\wxheading{Members}}}

\membersection{wxDC::wxDC}

\func{}{wxDC}{\void}

Constructor.

\membersection{wxDC::\destruct{wxDC}}

\func{}{\destruct{wxDC}}{\void}

Destructor.

\membersection{wxDC::BeginDrawing}\label{wxdcbegindrawing}

\func{void}{BeginDrawing}{\void}

Allows optimization of drawing code under MS Windows. Enclose
drawing primitives between {\bf BeginDrawing} and {\bf EndDrawing}\rtfsp
calls.

Drawing to a wxDialog panel device context outside of a
system-generated OnPaint event {\it requires} this pair of calls to
enclose drawing code. This is because a Windows dialog box does not have
a retained device context associated with it, and selections such as pen
and brush settings would be lost if the device context were obtained and
released for each drawing operation.

\membersection{wxDC::Blit}\label{wxdcblit}

\func{bool}{Blit}{\param{wxCoord}{ xdest}, \param{wxCoord}{ ydest}, \param{wxCoord}{ width}, \param{wxCoord}{ height},
  \param{wxDC* }{source}, \param{wxCoord}{ xsrc}, \param{wxCoord}{ ysrc}, \param{int}{ logicalFunc = wxCOPY},
  \param{bool }{useMask = FALSE}}

Copy from a source DC to this DC, specifying the destination
coordinates, size of area to copy, source DC, source coordinates, and
logical function.

\wxheading{Parameters}

\docparam{xdest}{Destination device context x position.}

\docparam{ydest}{Destination device context y position.}

\docparam{width}{Width of source area to be copied.}

\docparam{height}{Height of source area to be copied.}

\docparam{source}{Source device context.}

\docparam{xsrc}{Source device context x position.}

\docparam{ysrc}{Source device context y position.}

\docparam{logicalFunc}{Logical function to use: see \helpref{wxDC::SetLogicalFunction}{wxdcsetlogicalfunction}.}

\docparam{useMask}{If TRUE, Blit does a transparent blit using the mask that is associated with the bitmap
selected into the source device context. The Windows implementation does the following:

\begin{enumerate}
\item Creates a temporary bitmap and copies the destination area into it.
\item Copies the source area into the temporary bitmap using the specified logical function.
\item Sets the masked area in the temporary bitmap to BLACK by ANDing the
mask bitmap with the temp bitmap with the foreground colour set to WHITE
and the bg colour set to BLACK.
\item Sets the unmasked area in the destination area to BLACK by ANDing the
mask bitmap with the destination area with the foreground colour set to BLACK
and the background colour set to WHITE.
\item ORs the temporary bitmap with the destination area.
\item Deletes the temporary bitmap.
\end{enumerate}

This sequence of operations ensures that the source's transparent area need not be black,
and logical functions are supported.
}

\wxheading{Remarks}

There is partial support for Blit in wxPostScriptDC, under X.

See \helpref{wxMemoryDC}{wxmemorydc} for typical usage.

\wxheading{See also}

\helpref{wxMemoryDC}{wxmemorydc}, \helpref{wxBitmap}{wxbitmap}, \helpref{wxMask}{wxmask}

\membersection{wxDC::CalcBoundingBox}\label{wxdccalcboundingbox}

\func{void}{CalcBoundingBox}{\param{wxCoord }{x}, \param{wxCoord }{y}}

Adds the specified point to the bounding box which can be retrieved with 
\helpref{MinX}{wxdcminx}, \helpref{MaxX}{wxdcmaxx} and 
\helpref{MinY}{wxdcminy}, \helpref{MaxY}{wxdcmaxy} functions.

\wxheading{See also}

\helpref{ResetBoundingBox}{wxdcresetboundingbox}

\membersection{wxDC::Clear}\label{wxdcclear}

\func{void}{Clear}{\void}

Clears the device context using the current background brush.

\membersection{wxDC::CrossHair}\label{wxdccrosshair}

\func{void}{CrossHair}{\param{wxCoord}{ x}, \param{wxCoord}{ y}}

Displays a cross hair using the current pen. This is a vertical
and horizontal line the height and width of the window, centred
on the given point.

\membersection{wxDC::DestroyClippingRegion}\label{wxdcdestroyclippingregion}

\func{void}{DestroyClippingRegion}{\void}

Destroys the current clipping region so that none of the DC is clipped.
See also \helpref{wxDC::SetClippingRegion}{wxdcsetclippingregion}.

\membersection{wxDC::DeviceToLogicalX}\label{wxdcdevicetologicalx}

\func{wxCoord}{DeviceToLogicalX}{\param{wxCoord}{ x}}

Convert device X coordinate to logical coordinate, using the current
mapping mode.

\membersection{wxDC::DeviceToLogicalXRel}\label{wxdcdevicetologicalxrel}

\func{wxCoord}{DeviceToLogicalXRel}{\param{wxCoord}{ x}}

Convert device X coordinate to relative logical coordinate, using the current
mapping mode. Use this function for converting a width, for example.

\membersection{wxDC::DeviceToLogicalY}\label{wxdcdevicetologicaly}

\func{wxCoord}{DeviceToLogicalY}{\param{wxCoord}{ y}}

Converts device Y coordinate to logical coordinate, using the current
mapping mode.

\membersection{wxDC::DeviceToLogicalYRel}\label{wxdcdevicetologicalyrel}

\func{wxCoord}{DeviceToLogicalYRel}{\param{wxCoord}{ y}}

Convert device Y coordinate to relative logical coordinate, using the current
mapping mode. Use this function for converting a height, for example.

\membersection{wxDC::DrawArc}\label{wxdcdrawarc}

\func{void}{DrawArc}{\param{wxCoord}{ x1}, \param{wxCoord}{ y1}, \param{wxCoord}{ x2}, \param{wxCoord}{ y2}, \param{double}{ xc}, \param{double}{ yc}}

Draws an arc of a circle, centred on ({\it xc, yc}), with starting point ({\it x1, y1})
and ending at ({\it x2, y2}).   The current pen is used for the outline
and the current brush for filling the shape.

The arc is drawn in an anticlockwise direction from the start point to the end point.

\membersection{wxDC::DrawBitmap}\label{wxdcdrawbitmap}

\func{void}{DrawBitmap}{\param{const wxBitmap\&}{ bitmap}, \param{wxCoord}{ x}, \param{wxCoord}{ y}, \param{bool}{ transparent}}

Draw a bitmap on the device context at the specified point. If {\it transparent} is TRUE and the bitmap has
a transparency mask, the bitmap will be drawn transparently.

When drawing a mono-bitmap, the current text foreground colour will be used to draw the foreground
of the bitmap (all bits set to 1), and the current text background colour to draw the background
(all bits set to 0). See also \helpref{SetTextForeground}{wxdcsettextforeground}, 
\helpref{SetTextBackground}{wxdcsettextbackground} and \helpref{wxMemoryDC}{wxmemorydc}.

\membersection{wxDC::DrawCheckMark}\label{wxdcdrawcheckmark}

\func{void}{DrawCheckMark}{\param{wxCoord}{ x}, \param{wxCoord}{ y}, \param{wxCoord}{ width}, \param{wxCoord}{ height}}

\func{void}{DrawCheckMark}{\param{const wxRect \&}{rect}}

Draws a check mark inside the given rectangle.

\membersection{wxDC::DrawEllipse}\label{wxdcdrawellipse}

\func{void}{DrawEllipse}{\param{wxCoord}{ x}, \param{wxCoord}{ y}, \param{wxCoord}{ width}, \param{wxCoord}{ height}}

Draws an ellipse contained in the rectangle with the given top left corner, and with the
given size.  The current pen is used for the outline and the current brush for
filling the shape.

\membersection{wxDC::DrawEllipticArc}\label{wxdcdrawellipticarc}

\func{void}{DrawEllipticArc}{\param{wxCoord}{ x}, \param{wxCoord}{ y}, \param{wxCoord}{ width}, \param{wxCoord}{ height},
 \param{double}{ start}, \param{double}{ end}}

Draws an arc of an ellipse. The current pen is used for drawing the arc and
the current brush is used for drawing the pie.

{\it x} and {\it y} specify the x and y coordinates of the upper-left corner of the rectangle that contains
the ellipse.

{\it width} and {\it height} specify the width and height of the rectangle that contains
the ellipse.

{\it start} and {\it end} specify the start and end of the arc relative to the three-o'clock
position from the center of the rectangle. Angles are specified
in degrees (360 is a complete circle). Positive values mean
counter-clockwise motion. If {\it start} is equal to {\it end}, a
complete ellipse will be drawn.

\membersection{wxDC::DrawIcon}\label{wxdcdrawicon}

\func{void}{DrawIcon}{\param{const wxIcon\&}{ icon}, \param{wxCoord}{ x}, \param{wxCoord}{ y}}

Draw an icon on the display (does nothing if the device context is PostScript).
This can be the simplest way of drawing bitmaps on a window.

\membersection{wxDC::DrawLine}\label{wxdcdrawline}

\func{void}{DrawLine}{\param{wxCoord}{ x1}, \param{wxCoord}{ y1}, \param{wxCoord}{ x2}, \param{wxCoord}{ y2}}

Draws a line from the first point to the second. The current pen is used
for drawing the line.

\membersection{wxDC::DrawLines}\label{wxdcdrawlines}

\func{void}{DrawLines}{\param{int}{ n}, \param{wxPoint}{ points[]}, \param{wxCoord}{ xoffset = 0}, \param{wxCoord}{ yoffset = 0}}

\func{void}{DrawLines}{\param{wxList *}{points}, \param{wxCoord}{ xoffset = 0}, \param{wxCoord}{ yoffset = 0}}

Draws lines using an array of {\it points} of size {\it n}, or list of
pointers to points, adding the optional offset coordinate. The current
pen is used for drawing the lines.  The programmer is responsible for
deleting the list of points.

\pythonnote{The wxPython version of this method accepts a Python list
of wxPoint objects.}

\perlnote{The wxPerl version of this method accepts 
  as its first parameter a reference to an array
  of wxPoint objects.}

\membersection{wxDC::DrawPolygon}\label{wxdcdrawpolygon}

\func{void}{DrawPolygon}{\param{int}{ n}, \param{wxPoint}{ points[]}, \param{wxCoord}{ xoffset = 0}, \param{wxCoord}{ yoffset = 0},\\
  \param{int }{fill\_style = wxODDEVEN\_RULE}}

\func{void}{DrawPolygon}{\param{wxList *}{points}, \param{wxCoord}{ xoffset = 0}, \param{wxCoord}{ yoffset = 0},\\
  \param{int }{fill\_style = wxODDEVEN\_RULE}}

Draws a filled polygon using an array of {\it points} of size {\it n},
or list of pointers to points, adding the optional offset coordinate.

The last argument specifies the fill rule: {\bf wxODDEVEN\_RULE} (the
default) or {\bf wxWINDING\_RULE}.

The current pen is used for drawing the outline, and the current brush
for filling the shape.  Using a transparent brush suppresses filling.
The programmer is responsible for deleting the list of points.

Note that wxWindows automatically closes the first and last points.

\pythonnote{The wxPython version of this method accepts a Python list
of wxPoint objects.}

\perlnote{The wxPerl version of this method accepts 
  as its first parameter a reference to an array
  of wxPoint objects.}

\membersection{wxDC::DrawPoint}\label{wxdcdrawpoint}

\func{void}{DrawPoint}{\param{wxCoord}{ x}, \param{wxCoord}{ y}}

Draws a point using the current pen.

\membersection{wxDC::DrawRectangle}\label{wxdcdrawrectangle}

\func{void}{DrawRectangle}{\param{wxCoord}{ x}, \param{wxCoord}{ y}, \param{wxCoord}{ width}, \param{wxCoord}{ height}}

Draws a rectangle with the given top left corner, and with the given
size.  The current pen is used for the outline and the current brush
for filling the shape.

\membersection{wxDC::DrawRotatedText}\label{wxdcdrawrotatedtext}

\func{void}{DrawRotatedText}{\param{const wxString\& }{text}, \param{wxCoord}{ x}, \param{wxCoord}{ y}, \param{double}{ angle}}

Draws the text rotated by {\it angle} degrees.

\wxheading{See also}

\helpref{DrawText}{wxdcdrawtext}

\membersection{wxDC::DrawRoundedRectangle}\label{wxdcdrawroundedrectangle}

\func{void}{DrawRoundedRectangle}{\param{wxCoord}{ x}, \param{wxCoord}{ y}, \param{wxCoord}{ width}, \param{wxCoord}{ height}, \param{double}{ radius = 20}}

Draws a rectangle with the given top left corner, and with the given
size.  The corners are quarter-circles using the given radius. The
current pen is used for the outline and the current brush for filling
the shape.

If {\it radius} is positive, the value is assumed to be the
radius of the rounded corner. If {\it radius} is negative,
the absolute value is assumed to be the {\it proportion} of the smallest
dimension of the rectangle. This means that the corner can be
a sensible size relative to the size of the rectangle, and also avoids
the strange effects X produces when the corners are too big for
the rectangle.

\membersection{wxDC::DrawSpline}\label{wxdcdrawspline}

\func{void}{DrawSpline}{\param{wxList *}{points}}

Draws a spline between all given control points, using the current
pen.  Doesn't delete the wxList and contents. The spline is drawn
using a series of lines, using an algorithm taken from the X drawing
program `XFIG'.

\func{void}{DrawSpline}{\param{wxCoord}{ x1}, \param{wxCoord}{ y1}, \param{wxCoord}{ x2}, \param{wxCoord}{ y2}, \param{wxCoord}{ x3}, \param{wxCoord}{ y3}}

Draws a three-point spline using the current pen.

\pythonnote{The wxPython version of this method accepts a Python list
of wxPoint objects.}

\perlnote{The wxPerl version of this method accepts a reference to an array
  of wxPoint objects.}

\membersection{wxDC::DrawText}\label{wxdcdrawtext}

\func{void}{DrawText}{\param{const wxString\& }{text}, \param{wxCoord}{ x}, \param{wxCoord}{ y}}

Draws a text string at the specified point, using the current text font,
and the current text foreground and background colours.

The coordinates refer to the top-left corner of the rectangle bounding
the string. See \helpref{wxDC::GetTextExtent}{wxdcgettextextent} for how
to get the dimensions of a text string, which can be used to position the
text more precisely.

{\bf NB:} under wxGTK the current 
\helpref{logical function}{wxdcgetlogicalfunction} is used by this function
but it is ignored by wxMSW. Thus, you should avoid using logical functions
with this function in portable programs.

\membersection{wxDC::EndDoc}\label{wxdcenddoc}

\func{void}{EndDoc}{\void}

Ends a document (only relevant when outputting to a printer).

\membersection{wxDC::EndDrawing}\label{wxdcenddrawing}

\func{void}{EndDrawing}{\void}

Allows optimization of drawing code under MS Windows. Enclose
drawing primitives between {\bf BeginDrawing} and {\bf EndDrawing}\rtfsp
calls.

\membersection{wxDC::EndPage}\label{wxdcendpage}

\func{void}{EndPage}{\void}

Ends a document page (only relevant when outputting to a printer).

\membersection{wxDC::FloodFill}\label{wxdcfloodfill}

\func{void}{FloodFill}{\param{wxCoord}{ x}, \param{wxCoord}{ y}, \param{const wxColour\&}{ colour}, \param{int}{ style=wxFLOOD\_SURFACE}}

Flood fills the device context starting from the given point, using
the {\it current brush colour}, and using a style:

\begin{itemize}\itemsep=0pt
\item wxFLOOD\_SURFACE: the flooding occurs until a colour other than the given colour is encountered.
\item wxFLOOD\_BORDER: the area to be flooded is bounded by the given colour.
\end{itemize}

{\it Note:} this function is available in MS Windows only.

\membersection{wxDC::GetBackground}\label{wxdcgetbackground}

\func{wxBrush\&}{GetBackground}{\void}

\constfunc{const wxBrush\&}{GetBackground}{\void}

Gets the brush used for painting the background (see \helpref{wxDC::SetBackground}{wxdcsetbackground}).

\membersection{wxDC::GetBackgroundMode}\label{wxdcgetbackgroundmode}

\constfunc{int}{GetBackgroundMode}{\void}

Returns the current background mode: {\tt wxSOLID} or {\tt wxTRANSPARENT}.

\wxheading{See also}

\helpref{SetBackgroundMode}{wxdcsetbackgroundmode}

\membersection{wxDC::GetBrush}\label{wxdcgetbrush}

\func{wxBrush\&}{GetBrush}{\void}

\constfunc{const wxBrush\&}{GetBrush}{\void}

Gets the current brush (see \helpref{wxDC::SetBrush}{wxdcsetbrush}).

\membersection{wxDC::GetCharHeight}\label{wxdcgetcharheight}

\func{wxCoord}{GetCharHeight}{\void}

Gets the character height of the currently set font.

\membersection{wxDC::GetCharWidth}\label{wxdcgetcharwidth}

\func{wxCoord}{GetCharWidth}{\void}

Gets the average character width of the currently set font.

\membersection{wxDC::GetClippingBox}\label{wxdcgetclippingbox}

\func{void}{GetClippingBox}{\param{wxCoord}{ *x}, \param{wxCoord}{ *y}, \param{wxCoord}{ *width}, \param{wxCoord}{ *height}}

Gets the rectangle surrounding the current clipping region.

\pythonnote{No arguments are required and the four values defining the
rectangle are returned as a tuple.}

\perlnote{This method takes no arguments and returns a four element list
{\tt ( \$x, \$y, \$width, \$height )}}

\membersection{wxDC::GetFont}\label{wxdcgetfont}

\func{wxFont\&}{GetFont}{\void}

\constfunc{const wxFont\&}{GetFont}{\void}

Gets the current font (see \helpref{wxDC::SetFont}{wxdcsetfont}).

\membersection{wxDC::GetLogicalFunction}\label{wxdcgetlogicalfunction}

\func{int}{GetLogicalFunction}{\void}

Gets the current logical function (see \helpref{wxDC::SetLogicalFunction}{wxdcsetlogicalfunction}).

\membersection{wxDC::GetMapMode}\label{wxdcgetmapmode}

\func{int}{GetMapMode}{\void}

Gets the {\it mapping mode} for the device context (see \helpref{wxDC::SetMapMode}{wxdcsetmapmode}).

\membersection{wxDC::GetOptimization}\label{wxdcgetoptimization}

\func{bool}{GetOptimization}{\void}

Returns TRUE if device context optimization is on.
See \helpref{wxDC::SetOptimization}{wxsetoptimization} for details.

\membersection{wxDC::GetPen}\label{wxdcgetpen}

\func{wxPen\&}{GetPen}{\void}

\constfunc{const wxPen\&}{GetPen}{\void}

Gets the current pen (see \helpref{wxDC::SetPen}{wxdcsetpen}).

\membersection{wxDC::GetPixel}\label{wxdcgetpixel}

\func{bool}{GetPixel}{\param{wxCoord}{ x}, \param{wxCoord}{ y}, \param{wxColour *}{colour}}

Sets {\it colour} to the colour at the specified location. Windows only; an X implementation
is being worked on. Not available for wxPostScriptDC or wxMetafileDC.

\pythonnote{For wxPython the wxColour value is returned and is not
required as a parameter.}

\perlnote{This method only takes the parameters {\tt x} and {\tt y} and returns
a Wx::Colour value}

\membersection{wxDC::GetSize}\label{wxdcgetsize}

\func{void}{GetSize}{\param{wxCoord *}{width}, \param{wxCoord *}{height}}

For a PostScript device context, this gets the maximum size of graphics
drawn so far on the device context.

For a Windows printer device context, this gets the horizontal and vertical
resolution. It can be used to scale graphics to fit the page when using
a Windows printer device context. For example, if {\it maxX} and {\it maxY}\rtfsp
represent the maximum horizontal and vertical `pixel' values used in your
application, the following code will scale the graphic to fit on the
printer page:

\begin{verbatim}
  wxCoord w, h;
  dc.GetSize(&w, &h);
  double scaleX=(double)(maxX/w);
  double scaleY=(double)(maxY/h);
  dc.SetUserScale(min(scaleX,scaleY),min(scaleX,scaleY));
\end{verbatim}

\pythonnote{In place of a single overloaded method name, wxPython
implements the following methods:\par
\indented{2cm}{\begin{twocollist}
\twocolitem{{\bf GetSize()}}{Returns a wxSize}
\twocolitem{{\bf GetSizeTuple()}}{Returns a 2-tuple (width, height)}
\end{twocollist}}
}

\perlnote{In place of a single overloaded method, wxPerl uses:\par
\indented{2cm}{\begin{twocollist}
\twocolitem{{\bf GetSize()}}{Returns a Wx::Size}
\twocolitem{{\bf GetSizeWH()}}{Returns a 2-element list
  {\tt ( \$width, \$height )}}
\end{twocollist}
}}

\membersection{wxDC::GetTextBackground}\label{wxdcgettextbackground}

\func{wxColour\&}{GetTextBackground}{\void}

\constfunc{const wxColour\&}{GetTextBackground}{\void}

Gets the current text background colour (see \helpref{wxDC::SetTextBackground}{wxdcsettextbackground}).

\membersection{wxDC::GetTextExtent}\label{wxdcgettextextent}

\func{void}{GetTextExtent}{\param{const wxString\& }{string}, \param{wxCoord *}{w}, \param{wxCoord *}{h},\\
  \param{wxCoord *}{descent = NULL}, \param{wxCoord *}{externalLeading = NULL}, \param{wxFont *}{font = NULL}}

Gets the dimensions of the string using the currently selected font.
\rtfsp{\it string} is the text string to measure, {\it w} and {\it h} are
the total width and height respectively, {\it descent} is the
dimension from the baseline of the font to the bottom of the
descender, and {\it externalLeading} is any extra vertical space added
to the font by the font designer (usually is zero).

The optional parameter {\it font} specifies an alternative
to the currently selected font: but note that this does not
yet work under Windows, so you need to set a font for
the device context first.

See also \helpref{wxFont}{wxfont}, \helpref{wxDC::SetFont}{wxdcsetfont}.

\pythonnote{The following methods are implemented in wxPython:\par
\indented{2cm}{\begin{twocollist}
\twocolitem{{\bf GetTextExtent(string)}}{Returns a 2-tuple, (width, height)}
\twocolitem{{\bf GetFullTextExtent(string, font=NULL)}}{Returns a
4-tuple, (width, height, descent, externalLeading) }
\end{twocollist}}
}

\perlnote{In wxPerl this method is implemented as 
  {\bf GetTextExtent( string, font = undef )} returning a four element
  array {\tt ( \$width, \$height, \$descent, \$externalLeading )}
}

\membersection{wxDC::GetTextForeground}\label{wxdcgettextforeground}

\func{wxColour\&}{GetTextForeground}{\void}

\constfunc{const wxColour\&}{GetTextForeground}{\void}

Gets the current text foreground colour (see \helpref{wxDC::SetTextForeground}{wxdcsettextforeground}).


\membersection{wxDC::GetUserScale}\label{wxdcgetuserscale}

\func{void}{GetUserScale}{\param{double}{ *x}, \param{double}{ *y}}

Gets the current user scale factor (set by \helpref{SetUserScale}{wxdcsetuserscale}).

\perlnote{In wxPerl this method takes no arguments and returna a two element
 array {\tt ( \$x, \$y )}}

\membersection{wxDC::LogicalToDeviceX}\label{wxdclogicaltodevicex}

\func{wxCoord}{LogicalToDeviceX}{\param{wxCoord}{ x}}

Converts logical X coordinate to device coordinate, using the current
mapping mode.

\membersection{wxDC::LogicalToDeviceXRel}\label{wxdclogicaltodevicexrel}

\func{wxCoord}{LogicalToDeviceXRel}{\param{wxCoord}{ x}}

Converts logical X coordinate to relative device coordinate, using the current
mapping mode. Use this for converting a width, for example.

\membersection{wxDC::LogicalToDeviceY}\label{wxdclogicaltodevicey}

\func{wxCoord}{LogicalToDeviceY}{\param{wxCoord}{ y}}

Converts logical Y coordinate to device coordinate, using the current
mapping mode.

\membersection{wxDC::LogicalToDeviceYRel}\label{wxdclogicaltodeviceyrel}

\func{wxCoord}{LogicalToDeviceYRel}{\param{wxCoord}{ y}}

Converts logical Y coordinate to relative device coordinate, using the current
mapping mode. Use this for converting a height, for example.

\membersection{wxDC::MaxX}\label{wxdcmaxx}

\func{wxCoord}{MaxX}{\void}

Gets the maximum horizontal extent used in drawing commands so far.

\membersection{wxDC::MaxY}\label{wxdcmaxy}

\func{wxCoord}{MaxY}{\void}

Gets the maximum vertical extent used in drawing commands so far.

\membersection{wxDC::MinX}\label{wxdcminx}

\func{wxCoord}{MinX}{\void}

Gets the minimum horizontal extent used in drawing commands so far.

\membersection{wxDC::MinY}\label{wxdcminy}

\func{wxCoord}{MinY}{\void}

Gets the minimum vertical extent used in drawing commands so far.

\membersection{wxDC::Ok}\label{wxdcok}

\func{bool}{Ok}{\void}

Returns TRUE if the DC is ok to use.

\membersection{wxDC::ResetBoundingBox}\label{wxdcresetboundingbox}

\func{void}{ResetBoundingBox}{\void}

Resets the bounding box: after a call to this function, the bounding box
doesn't contain anything.

\wxheading{See also}

\helpref{CalcBoundingBox}{wxdccalcboundingbox}

\membersection{wxDC::SetDeviceOrigin}\label{wxdcsetdeviceorigin}

\func{void}{SetDeviceOrigin}{\param{wxCoord}{ x}, \param{wxCoord}{ y}}

Sets the device origin (i.e., the origin in pixels after scaling has been
applied).

This function may be useful in Windows printing
operations for placing a graphic on a page.

\membersection{wxDC::SetBackground}\label{wxdcsetbackground}

\func{void}{SetBackground}{\param{const wxBrush\& }{brush}}

Sets the current background brush for the DC.

\membersection{wxDC::SetBackgroundMode}\label{wxdcsetbackgroundmode}

\func{void}{SetBackgroundMode}{\param{int}{ mode}}

{\it mode} may be one of wxSOLID and wxTRANSPARENT. This setting determines
whether text will be drawn with a background colour or not.

\membersection{wxDC::SetClippingRegion}\label{wxdcsetclippingregion}

\func{void}{SetClippingRegion}{\param{wxCoord}{ x}, \param{wxCoord}{ y}, \param{wxCoord}{ width}, \param{wxCoord}{ height}}

\func{void}{SetClippingRegion}{\param{const wxRegion\&}{ region}}

Sets the clipping region for the DC. The clipping region is an area
to which drawing is restricted. Possible uses for the clipping region are for clipping text
or for speeding up window redraws when only a known area of the screen is damaged.

\wxheading{See also}

\helpref{wxDC::DestroyClippingRegion}{wxdcdestroyclippingregion}, \helpref{wxRegion}{wxregion}

\membersection{wxDC::SetPalette}\label{wxdcsetpalette}

\func{void}{SetPalette}{\param{const wxPalette\& }{palette}}

If this is a window DC or memory DC, assigns the given palette to the window
or bitmap associated with the DC. If the argument is wxNullPalette, the current
palette is selected out of the device context, and the original palette
restored.

See \helpref{wxPalette}{wxpalette} for further details.

\membersection{wxDC::SetBrush}\label{wxdcsetbrush}

\func{void}{SetBrush}{\param{const wxBrush\& }{brush}}

Sets the current brush for the DC.

If the argument is wxNullBrush, the current brush is selected out of the device
context, and the original brush restored, allowing the current brush to
be destroyed safely.

See also \helpref{wxBrush}{wxbrush}.

See also \helpref{wxMemoryDC}{wxmemorydc} for the interpretation of colours
when drawing into a monochrome bitmap.

\membersection{wxDC::SetFont}\label{wxdcsetfont}

\func{void}{SetFont}{\param{const wxFont\& }{font}}

Sets the current font for the DC.

If the argument is wxNullFont, the current font is selected out of the device
context, and the original font restored, allowing the current font to
be destroyed safely.

See also \helpref{wxFont}{wxfont}.

\membersection{wxDC::SetLogicalFunction}\label{wxdcsetlogicalfunction}

\func{void}{SetLogicalFunction}{\param{int}{ function}}

Sets the current logical function for the device context.  This determines how
a source pixel (from a pen or brush colour, or source device context if
using \helpref{wxDC::Blit}{wxdcblit}) combines with a destination pixel in the
current device context.

The possible values
and their meaning in terms of source and destination pixel values are
as follows:

\begin{verbatim}
wxAND                 src AND dst
wxAND_INVERT          (NOT src) AND dst
wxAND_REVERSE         src AND (NOT dst)
wxCLEAR               0
wxCOPY                src
wxEQUIV               (NOT src) XOR dst
wxINVERT              NOT dst
wxNAND                (NOT src) OR (NOT dst)
wxNOR                 (NOT src) AND (NOT dst)
wxNO_OP               dst
wxOR                  src OR dst
wxOR_INVERT           (NOT src) OR dst
wxOR_REVERSE          src OR (NOT dst)
wxSET                 1
wxSRC_INVERT          NOT src
wxXOR                 src XOR dst
\end{verbatim}

The default is wxCOPY, which simply draws with the current colour.
The others combine the current colour and the background using a
logical operation.  wxINVERT is commonly used for drawing rubber bands or
moving outlines, since drawing twice reverts to the original colour.

\membersection{wxDC::SetMapMode}\label{wxdcsetmapmode}

\func{void}{SetMapMode}{\param{int}{ int}}

The {\it mapping mode} of the device context defines the unit of
measurement used to convert logical units to device units. Note that
in X, text drawing isn't handled consistently with the mapping mode; a
font is always specified in point size. However, setting the {\it
user scale} (see \helpref{wxDC::SetUserScale}{wxdcsetuserscale}) scales the text appropriately. In
Windows, scaleable TrueType fonts are always used; in X, results depend
on availability of fonts, but usually a reasonable match is found.

Note that the coordinate origin should ideally be selectable, but for
now is always at the top left of the screen/printer.

Drawing to a Windows printer device context under UNIX
uses the current mapping mode, but mapping mode is currently ignored for
PostScript output.

The mapping mode can be one of the following:

\begin{twocollist}\itemsep=0pt
\twocolitem{wxMM\_TWIPS}{Each logical unit is 1/20 of a point, or 1/1440 of
  an inch.}
\twocolitem{wxMM\_POINTS}{Each logical unit is a point, or 1/72 of an inch.}
\twocolitem{wxMM\_METRIC}{Each logical unit is 1 mm.}
\twocolitem{wxMM\_LOMETRIC}{Each logical unit is 1/10 of a mm.}
\twocolitem{wxMM\_TEXT}{Each logical unit is 1 pixel.}
\end{twocollist}

\membersection{wxDC::SetOptimization}\label{wxsetoptimization}

\func{void}{SetOptimization}{\param{bool }{optimize}}

If {\it optimize} is TRUE (the default), this function sets optimization mode on.
This currently means that under X, the device context will not try to set a pen or brush
property if it is known to be set already. This approach can fall down
if non-wxWindows code is using the same device context or window, for example
when the window is a panel on which the windowing system draws panel items.
The wxWindows device context 'memory' will now be out of step with reality.

Setting optimization off, drawing, then setting it back on again, is a trick
that must occasionally be employed.

\membersection{wxDC::SetPen}\label{wxdcsetpen}

\func{void}{SetPen}{\param{const wxPen\& }{pen}}

Sets the current pen for the DC.

If the argument is wxNullPen, the current pen is selected out of the device
context, and the original pen restored.

See also \helpref{wxMemoryDC}{wxmemorydc} for the interpretation of colours
when drawing into a monochrome bitmap.

\membersection{wxDC::SetTextBackground}\label{wxdcsettextbackground}

\func{void}{SetTextBackground}{\param{const wxColour\& }{colour}}

Sets the current text background colour for the DC.

\membersection{wxDC::SetTextForeground}\label{wxdcsettextforeground}

\func{void}{SetTextForeground}{\param{const wxColour\& }{colour}}

Sets the current text foreground colour for the DC.

See also \helpref{wxMemoryDC}{wxmemorydc} for the interpretation of colours
when drawing into a monochrome bitmap.

\membersection{wxDC::SetUserScale}\label{wxdcsetuserscale}

\func{void}{SetUserScale}{\param{double}{ xScale}, \param{double}{ yScale}}

Sets the user scaling factor, useful for applications which require
`zooming'.

\membersection{wxDC::StartDoc}\label{wxdcstartdoc}

\func{bool}{StartDoc}{\param{const wxString\& }{message}}

Starts a document (only relevant when outputting to a printer).
Message is a message to show whilst printing.

\membersection{wxDC::StartPage}\label{wxdcstartpage}

\func{bool}{StartPage}{\void}

Starts a document page (only relevant when outputting to a printer).

