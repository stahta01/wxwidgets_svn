\section{Preprocesser symbols defined by wxWidgets}\label{cppconst}

Here is the list of preprocessor symbols used in the wxWidgets source grouped
by category (and sorted by alphabetical order inside each category). All of
these macros except for \texttt{wxUSE\_XXX} variety is defined if the
corresponding condition is true and undefined if it isn't, so they should be
always tested usin \texttt{#ifdef} and not \texttt{#if}.


\subsection{GUI system}\label{guisystemconst}

\begin{twocollist}\itemsep=0pt
\twocolitem{\_\_WINDOWS\_\_}{any Windows, yom may also use \_\_WXMSW\_\_}
\twocolitem{\_\_WIN16\_\_}{Win16 API (not supported since wxWidgets 2.6)}
\twocolitem{\_\_WIN32\_\_}{Win32 API}
\twocolitem{\_\_WIN95\_\_}{Windows 95 or NT 4.0 and above system (not NT 3.5x)}
\twocolitem{\_\_WXBASE\_\_}{Only wxBase, no GUI features (same as \texttt{wxUSE\_GUI} $== 0$)}
\twocolitem{\_\_WXCOCOA\_\_}{OS X using Cocoa API}
\twocolitem{\_\_WXWINCE\_\_}{Windows CE}
\twocolitem{\_\_WXGTK\_\_}{GTK+}
\twocolitem{\_\_WXGTK12\_\_}{GTK+ 1.2 or higher}
\twocolitem{\_\_WXGTK20\_\_}{GTK+ 2.0 or higher}
\twocolitem{\_\_WXMOTIF\_\_}{Motif}
\twocolitem{\_\_WXMOTIF20\_\_}{Motif 2.0 or higher}
\twocolitem{\_\_WXMAC\_\_}{Mac OS all targets}
\twocolitem{\_\_WXMAC\_CLASSIC\_\_}{MacOS for Classic}
\twocolitem{\_\_WXMAC\_CARBON\_\_}{MacOS for Carbon CFM (running under Classic or OSX) or true OS X Mach-O Builds}
\twocolitem{\_\_WXMAC\_OSX\_\_}{MacOS X Carbon Mach-O Builds}
\twocolitem{\_\_WXMGL\_\_}{SciTech Soft MGL (\_\_WXUNIVERSAL\_\_ will be also
defined)}
\twocolitem{\_\_WXMSW\_\_}{Any Windows}
\twocolitem{\_\_WXOS2\_\_}{Identical to \_\_WXPM\_\_}
\twocolitem{\_\_WXOSX\_\_}{Any Mac OS X port (either Carbon or Cocoa)}
\twocolitem{\_\_WXPM\_\_}{OS/2 native Presentation Manager}
\twocolitem{\_\_WXSTUBS\_\_}{Stubbed version ('template' wxWin implementation)}
\twocolitem{\_\_WXXT\_\_}{Xt; mutually exclusive with WX\_MOTIF, not
implemented in wxWidgets 2.x}
\twocolitem{\_\_WXX11\_\_}{wxX11 (\_\_WXUNIVERSAL\_\_ will be also defined)}
\twocolitem{\_\_WXWINE\_\_}{WINE (i.e. WIN32 on Unix)}
\twocolitem{\_\_WXUNIVERSAL\_\_}{wxUniversal port, always defined in addition
to one of the symbols above so this should be tested first.}
\twocolitem{\_\_X\_\_}{any X11-based GUI toolkit except GTK+}
\end{twocollist}

Mac situation is a bit confusing so a few extra words to explain it: there are
2 wx ports to Mac OS. One of them, wxMac, exists in 2 versions: Classic and
Carbon. The Classic version is the only one to work on Mac OS version 8. The
Carbon version may be built either as CFM or Mach-O (binary format, like ELF)
and the former may run under OS 9 while the latter only runs under OS X.
Finally, there is a new Cocoa port which can only be used under OS X. To
summarize:
\begin{itemize}
    \item If you want to test for all Mac platforms, classic and OS X, you
        should test both \texttt{\_\_WXMAC\_\_} and \texttt{\_\_WXCOCOA\_\_}
    \item If you want to test for any GUI Mac port under OS X, use
        \texttt{\_\_WXOSX\_\_}
    \item If you want to test for any port under Mac OS X, including, for
        example, wxGTK and also wxBase, use \texttt{\_\_DARWIN\_\_} (see below)
\end{itemize}


Note to implementors: although some of the symbols above don't start with 
\texttt{\_\_WX} prefix, they really should always use it, so please do start
any new symbols with it.


\subsection{Operating systems}\label{osconst}

\begin{twocollist}\itemsep=0pt
\twocolitem{\_\_APPLE\_\_}{any Mac OS version}
\twocolitem{\_\_AIX\_\_}{AIX}
\twocolitem{\_\_BSD\_\_}{Any *BSD system}
\twocolitem{\_\_CYGWIN\_\_}{Cygwin: Unix on Win32}
\twocolitem{\_\_DARWIN\_\_}{Mac OS X using the BSD Unix C library (as opposed to using the Metrowerks MSL C/C++ library)}
\twocolitem{\_\_DATA\_GENERAL\_\_}{DG-UX}
\twocolitem{\_\_DOS\_GENERAL\_\_}{DOS (used with wxMGL only)}
\twocolitem{\_\_FREEBSD\_\_}{FreeBSD}
\twocolitem{\_\_HPUX\_\_}{HP-UX (Unix)}
\twocolitem{\_\_GNU\_\_}{GNU Hurd}
\twocolitem{\_\_LINUX\_\_}{Linux}
\twocolitem{\_\_MACH\_\_}{Mach-O Architecture (Mac OS X only builds)}
\twocolitem{\_\_OSF\_\_}{OSF/1}
\twocolitem{\_\_SGI\_\_}{IRIX}
\twocolitem{\_\_SOLARIS\_\_}{Solaris}
\twocolitem{\_\_SUN\_\_}{Any Sun}
\twocolitem{\_\_SUNOS\_\_}{Sun OS}
\twocolitem{\_\_SVR4\_\_}{SystemV R4}
\twocolitem{\_\_SYSV\_\_}{SystemV generic}
\twocolitem{\_\_ULTRIX\_\_}{Ultrix}
\twocolitem{\_\_UNIX\_\_}{any Unix}
\twocolitem{\_\_UNIX\_LIKE\_\_}{Unix, BeOS or VMS}
\twocolitem{\_\_VMS\_\_}{VMS}
\twocolitem{\_\_WINDOWS\_\_}{any Windows}
\end{twocollist}


\subsection{Hardware architectures (CPU)}\label{cpuconst}

Note that not all of these symbols are always defined, it depends on the
compiler used.

\begin{twocollist}\itemsep=0pt
\twocolitem{\_\_ALPHA\_\_}{DEC Alpha architecture}
\twocolitem{\_\_INTEL\_\_}{Intel i386 or compatible}
\twocolitem{\_\_POWERPC\_\_}{Motorola Power PC}
\end{twocollist}


\subsection{Hardware type}\label{hardwareconst}

Combination of these symbols with GUI symbols describes real hardware
(like \_\_PDA\_\_ $&&$ \_\_WXWINCE\_\_ $==$ PocketPC devices).

\begin{twocollist}\itemsep=0pt
\twocolitem{\_\_SMARTPHONE\_\_}{Mobile devices with dialog capability through 
phone buttons and small display}
\twocolitem{\_\_PDA\_\_}{Personal digital assistant usually with touch screen and
middle sized screen}
\twocolitem{\_\_HANDHELD\_\_}{Small enough but powerful computer}
\end{twocollist}


\subsection{Compilers}\label{compilerconst}

\begin{twocollist}\itemsep=0pt
\twocolitem{\_\_BORLANDC\_\_}{Borland C++. The value of the macro corresponds
to the compiler version: $500$ is $5.0$.}
\twocolitem{\_\_DJGPP\_\_}{DJGPP}
\twocolitem{\_\_DIGITALMARS\_\_}{Digital Mars}
\twocolitem{\_\_GNUG\_\_}{Gnu C++ on any platform, see also 
\helpref{wxCHECK\_GCC\_VERSION}{wxcheckgccversion}}
\twocolitem{\_\_GNUWIN32\_\_}{Gnu-Win32 compiler, see also 
\helpref{wxCHECK\_W32API\_VERSION}{wxcheckw32apiversion}}
\twocolitem{\_\_MINGW32\_\_}{MinGW}
\twocolitem{\_\_MWERKS\_\_}{CodeWarrior MetroWerks compiler}
\twocolitem{\_\_SUNCC\_\_}{Sun CC}
\twocolitem{\_\_SYMANTECC\_\_}{Symantec C++}
\twocolitem{\_\_VISAGECPP\_\_}{IBM Visual Age (OS/2)}
\twocolitem{\_\_VISUALC\_\_}{Microsoft Visual C++. The value of this macro
corresponds to the compiler version: $1020$ for $4.2$ (the first supported
version), $1100$ for $5.0$, $1200$ for $6.0$ and so on}
\twocolitem{\_\_XLC\_\_}{AIX compiler}
\twocolitem{\_\_WATCOMC\_\_}{Watcom C++. The value of this macro corresponds to
the compiler version, $1100$ is $11.0$ and $1200$ is OpenWatcom.}
\twocolitem{\_WIN32\_WCE}{Windows CE version}
\end{twocollist}


\subsection{Miscellaneous}\label{miscellaneousconst}

\begin{twocollist}\itemsep=0pt
\twocolitem{\_\_WXWINDOWS\_\_}{always defined in wxWidgets applications, see
also \helpref{wxCHECK\_VERSION}{wxcheckversion}}
\twocolitem{\_\_WXDEBUG\_\_}{defined in debug mode, undefined in release mode}
\twocolitem{wxUSE\_XXX}{if defined as $1$, feature XXX is active
(the symbols of this form are always defined, use \#if and not \#ifdef to test
 for them)}
\twocolitem{wxUSE\_GUI}{this particular feature test macro is defined to $1$
when compiling or using the library with the GUI features activated, if it is
defined as $0$, only wxBase is available.}
\twocolitem{wxUSE\_BASE}{only used by wxWidgets internally (defined as $1$ when
building wxBase code, either as a standalone library or as part of the
monolithic wxWidgets library, defined as $0$ when building GUI library only)}
\end{twocollist}

