\section{Drag-and-drop and clipboard overview}\label{wxdndoverview}

Classes: \helpref{wxDataObject}{wxdataobject}

% \helpref{wxTextDataObject}{wxtextdataobject}
% \helpref{wxDropSource}{wxdropsource}
% \helpref{wxDropTarget}{wxdroptarget}
% \helpref{wxTextDropTarget}{wxtextdroptarget}
% \helpref{wxFileDropTarget}{wxfiledroptarget}

Samples: see the dnd sample.

Headers: <wx/dataobj.h>, <wx/dropsrc.h and <wx/droptgt.h>>
(note that wxUSE\_DRAG\_AND\_DROP must be defined in setup.h)

This overview describes wxWindows support for drag and drop and clipboard
operations. Both of these topics are discussed here because, in fact, they're
quite related. Drag and drop and clipboard are just two ways of passing the
data around and so the code required to implement both types of the operations
is almost the same.

Both operations involve passing some data from one program to another,
although the data can be received in the same program as the source. In the case
of clipboard transfer, the data is first placed on the clipboard and then
pasted into the destination program, while for a drag-and-drop operation the
data object is not stored anywhere but is created when the user starts
dragging and is destroyed as soon as he ends it, whether the operation was
ended successfully or cancelled.

To be a {\it drag source}, i.e. to provide the data which may be dragged by
user elsewhere, you should implement the following steps:

\begin{itemize}\itemsep=0pt
\item {\bf Preparation:} First of all, the data object must be created and
initialized with the data you wish to drag. For example:

\begin{verbatim}
	wxTextDataObject data("This string will be dragged.");
\end{verbatim}

Of course, the data object may contain arbitrary data of any type, but for
this you should derive your own class from \helpref{wxDataObject}{wxdataobject} overriding all of its pure virtual
functions.

\item{\bf Drag start:} To start dragging process (typically in response to a
mouse click) you must call \helpref{DoDragDrop}{wxdropsourcedodragdrop} function
of wxDropSource object which should be constructed like this:

\begin{verbatim}
	wxDropSource dragSource(data, this);

	// or also:

	wxDropSource dragSource(this);
	dragSource.SetData(data);
\end{verbatim}

\item {\bf Dragging:} The call to DoDragDrop() blocks until the user release the
mouse button (unless you override \helpref{GiveFeedback}{wxdropsourcegivefeedback} function
to do something special). When the mouse moves in a window of a program which understands the
same drag-and-drop protocol (any program under Windows or any program supporting XDnD protocol
under X Windows), the corresponding \helpref{wxDropTarget}{wxdroptarget} methods
are called - see below.

\item {\bf Processing the result:} DoDragDrop() returns an {\it effect code} which
is one of the values of \helpref{wxDragResult}{wxdragresult} enum. Codes
of wxDragError, wxDragNone and wxDragCancel have the obvious meaning and mean
that there is nothing to do on the sending end (except of possibly logging the
error in the first case). wxDragCopy means that the data has been successfully
copied and doesn't require any specific actions neither. But wxDragMove is
special because it means that the data must be deleted from where it was
copied. If it doesn't make sense (dragging selected text from a read-only
file) you should pass FALSE as parameter to DoDragDrop() in the previous step.
\end{itemize}

To be a {\it drop target}, i.e. to receive the data dropped by user you should
follow the instructions below:

\begin{itemize}\itemsep=0pt
\item {\bf Initialization:} For a window to be drop target, it needs to have
an associated \helpref{wxDropTarget}{wxdroptarget} object. Normally, you will
call \helpref{wxWindow::SetDropTarget}{wxwindowsetdroptarget} during window
creation associating you drop target with it. You must derive a class from
wxDropTarget and override its pure virtual methods. Alternatively, you may
derive from \helpref{wxTextDropTarget}{wxtextdroptarget} or
\helpref{wxFileDropTarget}{wxfiledroptarget} and override their OnDropText()
or OnDropFiles() method.

\item {\bf Drop:} When the user releases the mouse over a window, wxWindows
queries the associated wxDropTarget object if it accepts the data. For
this, \helpref{GetFormatCount}{wxdroptargetgetformatcount} and \helpref{GetFormat}{wxdroptargetgetformat} are
used and if the format is
supported (i.e. is one of returned by GetFormat()), 
then \helpref{OnDrop}{wxdroptargetondrop} is called. 
Otherwise, wxDragNone is returned by DoDragDrop() and
nothing happens.

\item {\bf The end:} After processing the data, DoDragDrop() returns either
wxDragCopy or wxDragMove depending on the state of the keys (<Ctrl>, <Shift>
and <Alt>) at the moment of drop. There is currently no way for the drop
target to change this return code.
\end{itemize}

