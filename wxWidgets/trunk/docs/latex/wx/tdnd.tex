\section{Drag-and-drop and clipboard overview}\label{wxdndoverview}

Classes: \helpref{wxDataObject}{wxdataobject}

% \helpref{wxTextDataObject}{wxtextdataobject}
% \helpref{wxDropSource}{wxdropsource}
% \helpref{wxDropTarget}{wxdroptarget}
% \helpref{wxTextDropTarget}{wxtextdroptarget}
% \helpref{wxFileDropTarget}{wxfiledroptarget}

Samples: see the dnd sample.

This overview describes wxWindows support for drag and drop and clipboard
operations. Both of these topics are discussed here because, in fact, they're
quite related. Drag and drop and clipboard are just too ways of passing the
data around and so the code required to implement both types of the operations
is almost the same.

In any case, you work with some data which is represented by
the \helpref{wxDataObject}{wxdataobject} class. It is capable to contain any kind
data in one of any of predefined formats (see enum \helpref{StdFormatand}{stdformat}) and is smart enough to describe the format
of data it contains. There is also a specialization of this class which stores
only text - the only difference between \helpref{wxTextDataObject}{wxtextdataobject} and wxDataObject is that the
first one is easily constructed from wxString.

Also, for both kinds of operations, there is a sender which provides data and
a receiver who gets it. The sender is responsible for constructing the
wxDataObject and the receiver can query it and process the data it contains
in any way he likes.

In the case of a drag and drop operation, the sender is called a {\it drop
source} while the receiver is a {\it dtop target}. There are several steps in
the dragging process:

\begin{itemize}\itemsep=0pt
\item {\bf preparation} First of all, the data object must be created and
initilized with the data you wish to drag. For example:

\begin{verbatim}
	wxTextDataObject data("This string will be dragged.");
\end{verbatim}. Of course, the data object may contain arbitrary data of any
type.

\item{drag start} This happens when you call \helpref{DoDragDrop}{wxdropsourcedodragdrop} function. For this you must first
construct a wxDropSource object and associate the data object from the
previous step with it like this:

\begin{verbatim}
	wxDropSource dragSource(data, this);

	// or also:
	wxDropSource dragSource(this);
	dragSource.SetData(data);
\end{verbatim},

\item {\bf dragging} The call to DoDragDrop() blocks until the user release the
mouse button (unless you override \helpref{GiveFeedback}{wxdropsourcegivefeedback} function to do something
special). When the mouse moves in a window of a wxWindows program, the
corresponding wxDropTarget methods are called (the data can be also dragged to
any other program under Windows or to any program supporting the same protocol
under X Windows).
\item {\bf drop} When the user releases the mouse over a window, wxWindows verifies
if the wxDropTarget object associated (with \helpref{SetDropTarget}{setdroptarget}) with this window accepts the data. For
this, \helpref{GetFormatCount}{wxdroptargetgetformatcount} and \helpref{GetFormat}{wxdroptargetgetformat} are used and if the format is
supported (i.e. is one of returned by GetFormat()), then \helpref{OnDrop}{wxdroptargetondrop} is called. Otherwise, wxDragNone is
returned by DoDragDrop() and nothing happens.
\item {\bf the end} Finally, the receiver processes the data (e.g. pastes the text
in its window). DoDragDrop() returns either wxDragCopy or wxDragMove
depending on the state of the keys (<Ctrl>, <Shift> and <Alt>) at the moment
of drop.
\end{itemize}

