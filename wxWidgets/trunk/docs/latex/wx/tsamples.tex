%%%%%%%%%%%%%%%%%%%%%%%%%%%%%%%%%%%%%%%%%%%%%%%%%%%%%%%%%%%%%%%%%%%%%%%%%%%%%%%
%% Name:        tsamples.tex
%% Purpose:     Samples description
%% Author:      Vadim Zeitlin
%% Modified by:
%% Created:     02.11.99
%% RCS-ID:      $Id$
%% Copyright:   (c) wxWindows team
%% Licence:     wxWindows licence
%%%%%%%%%%%%%%%%%%%%%%%%%%%%%%%%%%%%%%%%%%%%%%%%%%%%%%%%%%%%%%%%%%%%%%%%%%%%%%%

\section{wxWindows samples}\label{samples}

Probably the best way to learn wxWindows is by reading the source of some 50+
samples provided with it. Many aspects of wxWindows programming can be learnt
from them, but sometimes it is not simple to just choose the right sample to
look at. This overview aims at describing what each sample does/demonstrates to
make it easier to find the relevant one if a simple grep through all sources
didn't help. They also provide some notes about using the samples and what
features of wxWindows are they supposed to test.

\subsection{Font sample}\label{samplefont}

The font sample demonstrates \helpref{wxFont}{wxfont}, 
\helpref{wxFontEnumerator}{wxfontenumerator} and 
\helpref{wxFontMapper}{wxfontmapper} classes. It allows you to see the fonts
available (to wxWindows) on the computer and shows all characters of the
chosen font as well.

\subsection{DnD sample}\label{samplednd}

This sample shows both clipboard and drag and drop in action. It is quite non
trivial and may be safely used as a basis for implementing the clipboard and
drag and drop operations in a real-life program.

When you run the sample, its screen is split in several parts. On the top,
there are two listboxes which show the standard derivations of 
\helpref{wxDropTarget}{wxdroptarget}: 
\helpref{wxTextDropTarget}{wxtextdroptarget} and 
\helpref{wxFileDropTarget}{wxfiledroptarget}.

The middle of the sample window is taken by the log window which shows what is
going on (of course, this only works in debug builds) and may be helpful to see
the sequence of steps of data transfer.

Finally, the last part is used for two things: you can drag text from it to
either one of the listboxes (only one will accept it) or another application
and, also, bitmap pasted from clipboard will be shown there.

So far, everything we mentioned was implemented with minimal amount of code
using standard wxWindows classes. The more advanced features are demonstrated
if you create a shape frame from the main frame menu. A shape is a geometric
object which has a position, size and color. It models some
application-specific data in this sample. A shape object supports its own
private \helpref{wxDataFormat}{wxdataformat} which means that you may cut and
paste it or drag and drop (between one and the same or different shapes) from
one sample instance to another (or the same). However, chances are that no
other program supports this format and so shapes can also be rendered as
bitmaps which allows them to be pasted/dropped in many other applications.

Take a look at DnDShapeDataObject class to see how you may use 
\helpref{wxDataObject}{wxdataobject} to achieve this.


\subsection{HTML samples}\label{samplehtml}

Eight HTML samples (you can find them in directory {\tt samples/html})
cover all features of HTML sub-library.

{\bf Test} demonstrates how to create \helpref{wxHtmlWindow}{wxhtmlwindow}
and also shows most of supported HTML tags.

{\bf Widget} shows how you can embed ordinary controls or windows within
HTML page. It also nicely explains how to write new tag handlers and extend
the library to work with unsupported tags.

{\bf About} may give you an idea how to write good-looking about boxes.

{\bf Zip} demonstrates use of virtual file systems in wxHTML. The zip archives
handler (ships with wxWindows) allows you to access HTML pages stored 
in compressed archive as if they were ordinary files.

{\bf Virtual} is yet another VFS demo. This one generates pages at run-time.
You may find it useful if you need to display some reports in your application.

{\bf Printing} explains use of \helpref{wxHtmlEasyPrinting}{wxhtmleasyprinting}
class which serves as as-simple-as-possible interface for printing HTML 
documents without much work. In fact, only few function calls are sufficient.

{\bf Help} and {\bf Helpview} are variations on displaying HTML help 
(compatible with MS HTML Help Workshop). {\it Help} shows how to embed
\helpref{wxHtmlHelpController}{wxhtmlhelpcontroller} in your application
while {\it Helpview} is simple tool that only pops up help window and
displays help books given at command line.

