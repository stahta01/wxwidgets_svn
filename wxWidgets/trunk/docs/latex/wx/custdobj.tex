\section{\class{wxCustomDataObject}}\label{wxcustomdataobject}

wxCustomDataObject is a specialization of 
\helpref{wxDataObjectSimple}{wxdataobjectsimple} for some
application-specific data in arbitrary (either custom or one of the standard
ones). The only restriction is that it is supposed that this data can be
copied bitwise (i.e. with {\tt memcpy()}), so it would be a bad idea to make
it contain a C++ object (though C struct is fine).

\wxheading{Derived from}

\helpref{wxDataObjectSimple}{wxdataobjectsimple}
\helpref{wxDataObject}{wxdataobject}

\wxheading{Include files}

<wx/dataobj.h>

\wxheading{See also}

\helpref{wxDataObject}{wxdataobject}

\latexignore{\rtfignore{\wxheading{Members}}}

\membersection{wxCustomDataObject::wxCustomDataObject}\label{wxcustomdataobjectwxcustomdataobject}

\func{}{wxCustomDataObject}{\param{const wxDataFormat\& }{format = wxFormatInvalid}}

The constructor accepts a {\it format} argument which specifies the (single)
format supported by this object. If it isn't set here, 
\helpref{SetFormat}{wxdataobjectsimplesetformat} should be used.

\membersection{wxCustomDataObject::\destruct{wxCustomDataObject}}\label{wxcustomdataobjectdtor}

\func{}{\destruct{wxCustomDataObject}}{\void}

The destructor will free the data hold by the object. Notice that although it
calls a virtual \helpref{Free()}{wxcustomdataobjectfree} function, the base
class version will always be called (C++ doesn't allow calling virtual
functions from constructors or destructors), so if you override {\tt Free()}, you
should override the destructor in your class as well (which would probably
just call the derived class' version of {\tt Free()}).


\membersection{wxCustomDataObject::SetData}\label{wxcustomdataobjectsetdata}

\func{virtual void}{SetData}{\param{const char }{*data}, \param{size\_t }{size}}

Set the data. The data object will make an internal copy.

\membersection{wxCustomDataObject::GetSize}\label{wxcustomdataobjectgetsize}

\constfunc{virtual size\_t}{GetDataSize}{\void}

Returns the data size.

\membersection{wxCustomDataObject::GetData}\label{wxcustomdataobjectgetdata}

\func{virtual char*}{GetData}{\void}

Returns a pointer to the data.

