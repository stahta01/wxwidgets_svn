\section{\class{wxHashMap}}\label{wxhashmap}

This is a simple, type-safe, and reasonably efficient hash map class,
whose interface is a subset of the interface of STL containers. In
particular, the interface is modelled after std::map, and the various,
non standard, std::hash\_map.

\wxheading{Example}

\begin{verbatim}
    class MyClass { /* ... */ };

    // declare a hash map with string keys and int values
    WX_DECLARE_STRING_HASH_MAP( int, MyHash5 );
    // same, with int keys and MyClass* values
    WX_DECLARE_HASH_MAP( int, MyClass*, wxIntegerHash, wxIntegerEqual, MyHash1 );
    // same, with wxString keys and int values
    WX_DECLARE_STRING_HASH_MAP( int, MyHash3 );
    // same, with wxString keys and values
    WX_DECLARE_STRING_HASH_MAP( wxString, MyHash2 );

    MyHash1 h1;
    MyHash2 h2;

    // store and retrieve values
    h1[1] = new MyClass( 1 );
    h1[10000000] = NULL;
    h1[50000] = new MyClass( 2 );
    h2["Bill"] = "ABC";
    wxString tmp = h2["Bill"];
    // since element with key "Joe" is not present, this will return
    // the default value, which is an empty string in the case of wxString
    MyClass tmp2 = h2["Joe"];

    // iterate over all the elements in the class
    MyHash2::iterator it;
    for( it = h2.begin(); it != h2.end(); ++it )
    {
        wxString key = it->first, value = it->second;
        // do something useful with key and value
    }
\end{verbatim}

\wxheading{Declaring new hash table types}

\begin{verbatim}
    WX_DECLARE_STRING_HASH_MAP( VALUE_T,     // type of the values
                                CLASSNAME ); // name of the class
\end{verbatim}

Declares an hash map class named CLASSNAME, with {\tt wxString} keys
and VALUE\_T values.

\begin{verbatim}
    WX_DECLARE_VOIDPTR_HASH_MAP( VALUE_T,     // type of the values
                                 CLASSNAME ); // name of the class
\end{verbatim}

Declares an hash map class named CLASSNAME, with {\tt void*} keys
and VALUE\_T values.

\begin{verbatim}
    WX_DECLARE_HASH_MAP( KEY_T,      // type of the keys
                         VALUE_T,    // type of the values
                         HASH_T,     // hasher
                         KEY_EQ_T,   // key equality predicate
                         CLASSNAME); // name of the class
\end{verbatim}

The HASH\_T and KEY\_EQ\_T are the types
used for the hashing function and key comparison. wxWidgets provides
three predefined hashing functions: {\tt wxIntegerHash}
for integer types ( {\tt int}, {\tt long}, {\tt short},
and their unsigned counterparts ), {\tt wxStringHash} for strings
( {\tt wxString}, {\tt wxChar*}, {\tt char*} ), and
{\tt wxPointerHash} for any kind of pointer.
Similarly three equality predicates:
{\tt wxIntegerEqual}, {\tt wxStringEqual}, {\tt wxPointerEqual} are provided.

Using this you could declare an hash map mapping {\tt int} values
to {\tt wxString} like this:

\begin{verbatim}
    WX_DECLARE_HASH_MAP( int,
                         wxString,
                         wxIntegerHash,
                         wxIntegerEqual,
                         MyHash );

    // using an user-defined class for keys
    class MyKey { /* ... */ };

    // hashing function
    class MyKeyHash
    {
    public:
        MyKeyHash() { }

        unsigned long operator()( const MyKey& k ) const
            { /* compute the hash */ }

        MyKeyHash& operator=(const MyKeyHash&) { return *this; }
    };

    // comparison operator
    class MyKeyEqual
    {
    public:
        MyKeyEqual() { }
        bool operator()( const MyKey& a, const MyKey& b ) const
            { /* compare for equality */ }

        MyKeyEqual& operator=(const MyKeyEqual&) { return *this; }
    };

    WX_DECLARE_HASH_MAP( MyKey,      // type of the keys
                         SOME_TYPE,  // any type you like
                         MyKeyHash,  // hasher
                         MyKeyEqual, // key equality predicate
                         CLASSNAME); // name of the class
\end{verbatim}

\latexignore{\rtfignore{\wxheading{Types}}}

In the documentation below you should replace wxHashMap with the name
you used in the class declaration.

\begin{twocollist}
\twocolitem{wxHashMap::key\_type}{Type of the hash keys}
\twocolitem{wxHashMap::mapped\_type}{Type of the values stored in the hash map}
\twocolitem{wxHashMap::value\_type}{Equivalent to
{\tt struct \{ key\_type first; mapped\_type second \};} }
\twocolitem{wxHashMap::iterator}{Used to enumerate all the elements in an hash
map; it is similar to a {\tt value\_type*}}
\twocolitem{wxHashMap::const\_iterator}{Used to enumerate all the elements
in a constant hash map; it is similar to a {\tt const value\_type*}}
\twocolitem{wxHashMap::size\_type}{Used for sizes}
\end{twocollist}

\wxheading{Iterators}

An iterator is similar to a pointer, and so you can use the usual pointer
operations: {\tt ++it} ( and {\tt it++} ) to move to the next element,
{\tt *it} to access the element pointed to, {\tt it->first}
( {\tt it->second} ) to access the key ( value )
of the element pointed to. Hash maps provide forward only iterators, this
means that you can't use {\tt --it}, {\tt it + 3}, {\tt it1 - it2}.

\wxheading{Include files}

<wx/hashmap.h>

\latexignore{\rtfignore{\wxheading{Members}}}

\membersection{wxHashMap::wxHashMap}

\func{}{wxHashMap}{\param{size\_type}{ size = 10}}

The size parameter is just a hint, the table will resize automatically
to preserve performance.

\func{}{wxHashMap}{\param{const wxHashMap\&}{ map}}

Copy constructor.

\membersection{wxHashMap::begin}

\constfunc{const\_iterator}{begin}{}

\func{iterator}{begin}{}

Returns an iterator pointing at the first element of the hash map.
Please remember that hash maps do not guarantee ordering.

\membersection{wxHashMap::clear}

\func{void}{clear}{}

Removes all elements from the hash map.

\membersection{wxHashMap::count}

\constfunc{size\_type}{count}{\param{const key\_type\&}{ key}}

Counts the number of elements with the given key present in the map.
This function can actually return 0 or 1.

\membersection{wxHashMap::empty}

\constfunc{bool}{empty}{}

Returns true if the hash map does not contain any element, false otherwise.

\membersection{wxHashMap::end}

\constfunc{const\_iterator}{end}{}

\func{iterator}{end}{}

Returns an iterator pointing at the one-after-the-last element of the hash map.
Please remember that hash maps do not guarantee ordering.

\membersection{wxHashMap::erase}

\func{size\_type}{erase}{\param{const key\_type\&}{ key}}

Erases the element with the given key, and returns the number of elements
erased (either 0 or 1).

\func{void}{erase}{\param{iterator}{ it}}

\func{void}{erase}{\param{const\_iterator}{ it}}

Erases the element pointed to by the iterator. After the deletion
the iterator is no longer valid and must not be used.

\membersection{wxHashMap::find}

\func{iterator}{find}{\param{const key\_type\&}{ key}}

\constfunc{const\_iterator}{find}{\param{const key\_type\&}{ key}}

If an element with the given key is present, the functions returns
an iterator pointing at that element, otherwise an invalid iterator
is returned (i.e. hashmap.find( non\_existent\_key ) == hashmap.end()).

\membersection{wxHashMap::insert}

\func{void}{insert}{\param{const value\_type\&}{ v}}

Inserts the given value in the hash map.

\membersection{wxHashMap::operator[]}

\func{mapped\_type\&}{operator[]}{\param{const key\_type\&}{ key}}

Use it as an array subscript. The only difference is that if the
given key is not present in the hash map, an element with the
default {\tt value\_type()} is inserted in the table.

\membersection{wxHashMap::size}

\constfunc{size\_type}{size}{}

Returns the number of elements in the map.

