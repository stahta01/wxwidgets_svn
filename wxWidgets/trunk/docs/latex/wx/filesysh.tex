%
% automatically generated by HelpGen from
% filesystemhandler.tex at 21/Mar/99 23:00:52
%

\section{\class{wxFileSystemHandler}}\label{wxfilesystemhandler}

wxFileSystemHandler (or derived classes to be exact) is used
to access virtual file systems. It's public interface consists
from two methods: \helpref{CanOpen}{wxfilesystemhandlercanopen}
and \helpref{OpenFile}{wxfilesystemhandleropenfile}. 
It provides additional protected methods to simplify process
of opening the file : GetProtocol, GetLeftLocation, GetRightLocation,
GetAnchor, GetMimeTypeFromExt.

Please have a look at \helpref{overview}{fs} if you don't know how locations
are constructed.

\wxheading{Notes}

\begin{itemize}
\item The handlers are shared by all instances of wxFileSystem.

\item wxHTML library provides handlers for local files and HTTP or FTP protocol

\item The {\it location} parameter passed to OpenFile or CanOpen methods
is always {\bf absolute} path. You don't need to check the FS's current path!
\end{itemize}

\wxheading{Derived from}

wxObject

\wxheading{See also}

\helpref{wxFileSystem}{wxfilesystem},
\helpref{wxFSFile}{wxfsfile},
\helpref{Overview}{fs}

\membersection{wxFileSystemHandler::wxFileSystemHandler}\label{wxfilesystemhandlerwxfilesystemhandler}

\func{}{wxFileSystemHandler}{\void}

Constructor.

\membersection{wxFileSystemHandler::CanOpen}\label{wxfilesystemhandlercanopen}

\func{virtual bool}{CanOpen}{\param{const wxString\& }{location}}

Returns TRUE if the handler is able to open this file (this function doesn't
check whether the file exists or not, it only checks if it knows the protocol).
Example:

\begin{verbatim}
bool MyHand::CanOpen(const wxString& location) 
{
    return (GetProtocol(location) == "http");
}
\end{verbatim}

Must be overwriten in derived handlers.

\membersection{wxFileSystemHandler::OpenFile}\label{wxfilesystemhandleropenfile}

\func{virtual wxFSFile*}{OpenFile}{\param{wxFileSystem\& }{fs}, \param{const wxString\& }{location}}

Opens the file and returns wxFSFile pointer or NULL if failed.

Must be overwriten in derived handlers.

\wxheading{Parameters}

\docparam{fs}{Parent FS (the FS from that OpenFile was called). See ZIP handler
for details how to use it.}

\docparam{location}{The {\bf absolute} location of file.}

\membersection{wxFileSystemHandler::GetProtocol}\label{wxfilesystemhandlergetprotocol}

\constfunc{wxString}{GetProtocol}{\param{const wxString\& }{location}}

Returns protocol string extracted from {\it location}. 

Example: GetProtocol("file:myzipfile.zip\#zip:index.htm") == "zip"

\membersection{wxFileSystemHandler::GetLeftLocation}\label{wxfilesystemhandlergetleftlocation}

\constfunc{wxString}{GetLeftLocation}{\param{const wxString\& }{location}}

Returns left location string extracted from {\it location}. 

Example: GetLeftLocation("file:myzipfile.zip\#zip:index.htm") == "file:myzipfile.zip"

\membersection{wxFileSystemHandler::GetAnchor}\label{wxfilesystemhandlergetanchor}

\constfunc{wxString}{GetAnchor}{\param{const wxString\& }{location}}

Returns anchor if present in the location.
See \helpref{wxFSFile}{wxfsfilegetanchor} for details.

Example : GetAnchor("index.htm\#chapter2") == "chapter2"

{\bf Note:} anchor is NOT part of left location.

\membersection{wxFileSystemHandler::GetRightLocation}\label{wxfilesystemhandlergetrightlocation}

\constfunc{wxString}{GetRightLocation}{\param{const wxString\& }{location}}

Returns right location string extracted from {\it location}. 

Example : GetRightLocation("file:myzipfile.zip\#zip:index.htm") == "index.htm"

\membersection{wxFileSystemHandler::GetMimeTypeFromExt}\label{wxfilesystemhandlergetmimetypefromext}

\func{wxString}{GetMimeTypeFromExt}{\param{const wxString\& }{location}}

Returns MIME type based on {\bf extension} of {\it location}. (While wxFSFile::GetMimeType
returns real MIME type - either extension-based or queried from HTTP)

Example : GetMimeTypeFromExt("index.htm") == "text/html"

