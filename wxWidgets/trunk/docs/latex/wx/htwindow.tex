%
% automatically generated by HelpGen from
% htmlwindow.tex at 14/Mar/99 20:13:37
%

\section{\class{wxHtmlWindow}}\label{wxhtmlwindow}

wxHtmlWindow is probably the only class you will directly use
unless you want to do something special (like adding new tag
handlers or MIME filters).

The purpose of this class is to display HTML pages (either local
file or downloaded via HTTP protocol) in a window. The width
of the window is constant - given in the constructor - and virtual height
is changed dynamically depending on page size.
Once the window is created you can set its content by calling 
\helpref{SetPage(text)}{wxhtmlwindowsetpage} or 
\helpref{LoadPage(filename)}{wxhtmlwindowloadpage}. 

\wxheading{Derived from}

\helpref{wxScrolledWindow}{wxscrolledwindow}

\wxheading{Include files}

<wx/html/htmlwin.h>

\membersection{wxHtmlWindow::wxHtmlWindow}\label{wxhtmlwindowwxhtmlwindow}

\func{}{wxHtmlWindow}{\void}

Default constructor.

\func{}{wxHtmlWindow}{\param{wxWindow }{*parent}, \param{wxWindowID }{id = -1}, \param{const wxPoint\& }{pos = wxDefaultPosition}, \param{const wxSize\& }{size = wxDefaultSize}, \param{long }{style = wxHW\_SCROLLBAR\_AUTO}, \param{const wxString\& }{name = "htmlWindow"}}

Constructor. The parameters are the same as for the \helpref{wxScrolledWindow}{wxscrolledwindow} constructor.

\wxheading{Parameters}

\docparam{style}{wxHW\_SCROLLBAR\_NEVER,  or wxHW\_SCROLLBAR\_AUTO. 
Affects the appearance of vertical scrollbar in the window.}

\membersection{wxHtmlWindow::AddFilter}\label{wxhtmlwindowaddfilter}

\func{static void}{AddFilter}{\param{wxHtmlFilter }{*filter}}

Adds \helpref{input filter}{filters} to the static list of available
filters. These filters are present by default:

\begin{itemize}\itemsep=0pt
\item {\tt text/html} MIME type
\item {\tt image/*} MIME types
\item Plain Text filter (this filter is used if no other filter matches)
\end{itemize}

\membersection{wxHtmlWindow::AppendToPage}\label{wxhtmlwindowappendtopage}

\func{bool}{AppendToPage}{\param{const wxString\& }{source}}

Appends HTML fragment to currently displayed text and refreshes the window. 

\wxheading{Parameters}

\docparam{source}{HTML code fragment}

\wxheading{Return value}

FALSE if an error occurred, TRUE otherwise.

\membersection{wxHtmlWindow::GetInternalRepresentation}\label{wxhtmlwindowgetinternalrepresentation}

\constfunc{wxHtmlContainerCell*}{GetInternalRepresentation}{\void}

Returns pointer to the top-level container.

See also: \helpref{Cells Overview}{cells}, 
\helpref{Printing Overview}{printing}

\membersection{wxHtmlWindow::GetOpenedAnchor}\label{wxhtmlwindowgetopenedanchor}

\func{wxString}{GetOpenedAnchor}{\void}

Returns anchor within currently opened page
(see \helpref{GetOpenedPage}{wxhtmlwindowgetopenedpage}). 
If no page is opened or if the displayed page wasn't
produced by call to LoadPage, empty string is returned.


\membersection{wxHtmlWindow::GetOpenedPage}\label{wxhtmlwindowgetopenedpage}

\func{wxString}{GetOpenedPage}{\void}

Returns full location of the opened page. If no page is opened or if the displayed page wasn't
produced by call to LoadPage, empty string is returned.

\membersection{wxHtmlWindow::GetOpenedPageTitle}\label{wxhtmlwindowgetopenedpagetitle}

\func{wxString}{GetOpenedPageTitle}{\void}

Returns title of the opened page or wxEmptyString if current page does not contain {\tt <TITLE>} tag.

\membersection{wxHtmlWindow::GetRelatedFrame}\label{wxhtmlwindowgetrelatedframe}

\constfunc{wxFrame*}{GetRelatedFrame}{\void}

Returns the related frame.

\membersection{wxHtmlWindow::HistoryBack}\label{wxhtmlwindowhistoryback}

\func{bool}{HistoryBack}{\void}

Moves back to the previous page. (each page displayed using 
\helpref{LoadPage}{wxhtmlwindowloadpage} is stored in history list.)

\membersection{wxHtmlWindow::HistoryCanBack}\label{wxhtmlwindowhistorycanback}

\func{bool}{HistoryCanBack}{\void}

Returns true if it is possible to go back in the history (i.e. HistoryBack()
won't fail).

\membersection{wxHtmlWindow::HistoryCanForward}\label{wxhtmlwindowhistorycanforward}

\func{bool}{HistoryCanForward}{\void}

Returns true if it is possible to go forward in the history (i.e. HistoryBack()
won't fail).


\membersection{wxHtmlWindow::HistoryClear}\label{wxhtmlwindowhistoryclear}

\func{void}{HistoryClear}{\void}

Clears history.

\membersection{wxHtmlWindow::HistoryForward}\label{wxhtmlwindowhistoryforward}

\func{bool}{HistoryForward}{\void}

Moves to next page in history.

\membersection{wxHtmlWindow::LoadPage}\label{wxhtmlwindowloadpage}

\func{virtual bool}{LoadPage}{\param{const wxString\& }{location}}

Unlike SetPage this function first loads HTML page from {\it location} 
and then displays it. See example:

\begin{verbatim}
htmlwin -> SetPage("help/myproject/index.htm");
\end{verbatim}

\wxheading{Parameters}

\docparam{location}{The address of document. See \helpref{wxFileSystem}{wxfilesystem} for details on address format and behaviour of "opener".}

\wxheading{Return value}

FALSE if an error occurred, TRUE otherwise

\membersection{wxHtmlWindow::OnCellClicked}\label{wxhtmlwindowoncellclicked}

\func{virtual void}{OnCellClicked}{\param{wxHtmlCell }{*cell}, \param{wxCoord }{x}, \param{wxCoord }{y}, \param{const wxMouseEvent\& }{event}}

This method is called when a mouse button is clicked inside wxHtmlWindow.
The default behaviour is to call 
\helpref{OnLinkClicked}{wxhtmlwindowonlinkclicked} if the cell contains a
hypertext link.

\wxheading{Parameters}

\docparam{cell}{The cell inside which the mouse was clicked, always a simple
(i.e. non container) cell}

\docparam{x, y}{The logical coordinates of the click point}

\docparam{event}{The mouse event containing other information about the click}

\membersection{wxHtmlWindow::OnCellMouseHover}\label{wxhtmlwindowoncellmousehover}

\func{virtual void}{OnCellMouseHover}{\param{wxHtmlCell }{*cell}, \param{wxCoord }{x}, \param{wxCoord }{y}}

This method is called when a mouse moves over an HTML cell.

\wxheading{Parameters}

\docparam{cell}{The cell inside which the mouse is currently, always a simple
(i.e. non container) cell}

\docparam{x, y}{The logical coordinates of the click point}

\membersection{wxHtmlWindow::OnLinkClicked}\label{wxhtmlwindowonlinkclicked}

\func{virtual void}{OnLinkClicked}{\param{const wxHtmlLinkInfo\& }{link}}

Called when user clicks on hypertext link. Default behaviour is to call 
\helpref{LoadPage}{wxhtmlwindowloadpage} and do nothing else.

Also see \helpref{wxHtmlLinkInfo}{wxhtmllinkinfo}.


\membersection{wxHtmlWindow::OnSetTitle}\label{wxhtmlwindowonsettitle}

\func{virtual void}{OnSetTitle}{\param{const wxString\& }{title}}

Called on parsing {\tt <TITLE>} tag.


\membersection{wxHtmlWindow::ReadCustomization}\label{wxhtmlwindowreadcustomization}

\func{virtual void}{ReadCustomization}{\param{wxConfigBase }{*cfg}, \param{wxString }{path = wxEmptyString}}

This reads custom settings from wxConfig. It uses the path 'path'
if given, otherwise it saves info into currently selected path.
The values are stored in sub-path {\tt wxHtmlWindow}

Read values: all things set by SetFonts, SetBorders.

\wxheading{Parameters}

\docparam{cfg}{wxConfig from which you want to read the configuration.}

\docparam{path}{Optional path in config tree. If not given current path is used.}

\membersection{wxHtmlWindow::SetBorders}\label{wxhtmlwindowsetborders}

\func{void}{SetBorders}{\param{int }{b}}

This function sets the space between border of window and HTML contents. See image:

\helponly{\image{}{border.bmp}}

\wxheading{Parameters}

\docparam{b}{indentation from borders in pixels}

\membersection{wxHtmlWindow::SetFonts}\label{wxhtmlwindowsetfonts}

\func{void}{SetFonts}{\param{wxString }{normal\_face}, \param{wxString }{fixed\_face}, \param{const int }{*sizes}}

This function sets font sizes and faces.

\wxheading{Parameters}

\docparam{normal\_face}{This is face name for normal (i.e. non-fixed) font. 
It can be either empty string (then the default face is choosen) or
platform-specific face name. Examples are "helvetica" under Unix or
"Times New Roman" under Windows.}

\docparam{fixed\_face}{The same thing for fixed face ( <TT>..</TT> )}

\docparam{sizes}{This is an array of 7 items of {\it int} type.
The values represent size of font with HTML size from -2 to +4
( <FONT SIZE=-2> to <FONT SIZE=+4> )}

\wxheading{Defaults}

Under wxGTK:

\begin{verbatim}
    SetFonts("", "", {10, 12, 14, 16, 19, 24, 32});
\end{verbatim}

Under Windows:

\begin{verbatim}
    SetFonts("", "", {7, 8, 10, 12, 16, 22, 30});
\end{verbatim}

Athough it seems different the fact is that the fonts are of approximately
same size under both platforms (due to wxMSW / wxGTK inconsistency)

\membersection{wxHtmlWindow::SetPage}\label{wxhtmlwindowsetpage}

\func{bool}{SetPage}{\param{const wxString\& }{source}}

Sets HTML page and display it. This won't {\bf load} the page!!
It will display the {\it source}. See example:

\begin{verbatim}
htmlwin -> SetPage("<html><body>Hello, world!</body></html>");
\end{verbatim}

If you want to load a document from some location use 
\helpref{LoadPage}{wxhtmlwindowloadpage} instead.

\wxheading{Parameters}

\docparam{source}{The HTML document source to be displayed.}

\wxheading{Return value}

FALSE if an error occurred, TRUE otherwise.

\membersection{wxHtmlWindow::SetRelatedFrame}\label{wxhtmlwindowsetrelatedframe}

\func{void}{SetRelatedFrame}{\param{wxFrame* }{frame}, \param{const wxString\& }{format}}

Sets the frame in which page title will be displayed. {\it format} is format of
frame title, e.g. "HtmlHelp : \%s". It must contain exactly one \%s. This
\%s is substituted with HTML page title.

\membersection{wxHtmlWindow::SetRelatedStatusBar}\label{wxhtmlwindowsetrelatedstatusbar}

\func{void}{SetRelatedStatusBar}{\param{int }{bar}}

{\bf After} calling \helpref{SetRelatedFrame}{wxhtmlwindowsetrelatedframe},
this sets statusbar slot where messages will be displayed.
(Default is -1 = no messages.)

\wxheading{Parameters}

\docparam{bar}{statusbar slot number (0..n)}


\membersection{wxHtmlWindow::WriteCustomization}\label{wxhtmlwindowwritecustomization}

\func{virtual void}{WriteCustomization}{\param{wxConfigBase }{*cfg}, \param{wxString }{path = wxEmptyString}}

Saves custom settings into wxConfig. It uses the path 'path'
if given, otherwise it saves info into currently selected path.
Regardless of whether the path is given or not, the function creates sub-path 
{\tt wxHtmlWindow}.

Saved values: all things set by SetFonts, SetBorders.

\wxheading{Parameters}

\docparam{cfg}{wxConfig to which you want to save the configuration.}

\docparam{path}{Optional path in config tree. If not given, the current path is used.}

